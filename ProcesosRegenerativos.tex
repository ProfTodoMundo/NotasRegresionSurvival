%___________________________________________________________________________________________
%
% Cap 1
%
%___________________________________________________________________________________________
\vspace{5.5cm}
\section{Procesos Regenerativos}
\vspace{-1.0cm}
%___________________________________________________________________________________________


\begin{Def}
Sea $X$ un conjunto y $\mathcal{F}$ una $\sigma$-\'algebra de
subconjuntos de $X$, la pareja $\left(X,\mathcal{F}\right)$ es
llamado espacio medible. Un subconjunto $A$ de $X$ es llamado
medible, o medible con respecto a $\mathcal{F}$, si
$A\in\mathcal{F}$.
\end{Def}

\begin{Def}
Sea $\left(X,\mathcal{F},\mu\right)$ espacio de medida. Se dice
que la medida $\mu$ es $\sigma$-finita si se puede escribir
$X=\bigcup_{n\geq1}X_{n}$ con $X_{n}\in\mathcal{F}$ y
$\mu\left(X_{n}\right)<\infty$.
\end{Def}

\begin{Def}\label{Cto.Borel}
Sea $X$ el conjunto de los \'umeros reales $\rea$. El \'algebra de
Borel es la $\sigma$-\'algebra $B$ generada por los intervalos
abiertos $\left(a,b\right)\in\rea$. Cualquier conjunto en $B$ es
llamado {\em Conjunto de Borel}.
\end{Def}

\begin{Def}\label{Funcion.Medible}
Una funci\'on $f:X\rightarrow\rea$, es medible si para cualquier
n\'umero real $\alpha$ el conjunto
\[\left\{x\in X:f\left(x\right)>\alpha\right\}\]
pertenece a $X$. Equivalentemente, se dice que $f$ es medible si
\[f^{-1}\left(\left(\alpha,\infty\right)\right)=\left\{x\in X:f\left(x\right)>\alpha\right\}\in\mathcal{F}.\]
\end{Def}


\begin{Def}\label{Def.Cilindros}
Sean $\left(\Omega_{i},\mathcal{F}_{i}\right)$, $i=1,2,\ldots,$
espacios medibles y $\Omega=\prod_{i=1}^{\infty}\Omega_{i}$ el
conjunto de todas las sucesiones
$\left(\omega_{1},\omega_{2},\ldots,\right)$ tales que
$\omega_{i}\in\Omega_{i}$, $i=1,2,\ldots,$. Si
$B^{n}\subset\prod_{i=1}^{\infty}\Omega_{i}$, definimos
$B_{n}=\left\{\omega\in\Omega:\left(\omega_{1},\omega_{2},\ldots,\omega_{n}\right)\in
B^{n}\right\}$. Al conjunto $B_{n}$ se le llama {\em cilindro} con
base $B^{n}$, el cilindro es llamado medible si
$B^{n}\in\prod_{i=1}^{\infty}\mathcal{F}_{i}$.
\end{Def}


\begin{Def}\label{Def.Proc.Adaptado}[TSP, Ash \cite{RBA}]
Sea $X\left(t\right),t\geq0$ proceso estoc\'astico, el proceso es
adaptado a la familia de $\sigma$-\'algebras $\mathcal{F}_{t}$,
para $t\geq0$, si para $s<t$ implica que
$\mathcal{F}_{s}\subset\mathcal{F}_{t}$, y $X\left(t\right)$ es
$\mathcal{F}_{t}$-medible para cada $t$. Si no se especifica
$\mathcal{F}_{t}$ entonces se toma $\mathcal{F}_{t}$ como
$\mathcal{F}\left(X\left(s\right),s\leq t\right)$, la m\'as
peque\~na $\sigma$-\'algebra de subconjuntos de $\Omega$ que hace
que cada $X\left(s\right)$, con $s\leq t$ sea Borel medible.
\end{Def}


\begin{Def}\label{Def.Tiempo.Paro}[TSP, Ash \cite{RBA}]
Sea $\left\{\mathcal{F}\left(t\right),t\geq0\right\}$ familia
creciente de sub $\sigma$-\'algebras. es decir,
$\mathcal{F}\left(s\right)\subset\mathcal{F}\left(t\right)$ para
$s\leq t$. Un tiempo de paro para $\mathcal{F}\left(t\right)$ es
una funci\'on $T:\Omega\rightarrow\left[0,\infty\right]$ tal que
$\left\{T\leq t\right\}\in\mathcal{F}\left(t\right)$ para cada
$t\geq0$. Un tiempo de paro para el proceso estoc\'astico
$X\left(t\right),t\geq0$ es un tiempo de paro para las
$\sigma$-\'algebras
$\mathcal{F}\left(t\right)=\mathcal{F}\left(X\left(s\right)\right)$.
\end{Def}

\begin{Def}
Sea $X\left(t\right),t\geq0$ proceso estoc\'astico, con
$\left(S,\chi\right)$ espacio de estados. Se dice que el proceso
es adaptado a $\left\{\mathcal{F}\left(t\right)\right\}$, es
decir, si para cualquier $s,t\in I$, $I$ conjunto de \'indices,
$s<t$, se tiene que
$\mathcal{F}\left(s\right)\subset\mathcal{F}\left(t\right)$ y
$X\left(t\right)$ es $\mathcal{F}\left(t\right)$-medible,
\end{Def}

\begin{Def}
Sea $X\left(t\right),t\geq0$ proceso estoc\'astico, se dice que es
un Proceso de Markov relativo a $\mathcal{F}\left(t\right)$ o que
$\left\{X\left(t\right),\mathcal{F}\left(t\right)\right\}$ es de
Markov si y s\'olo si para cualquier conjunto $B\in\chi$,  y
$s,t\in I$, $s<t$ se cumple que
\begin{equation}\label{Prop.Markov}
P\left\{X\left(t\right)\in
B|\mathcal{F}\left(s\right)\right\}=P\left\{X\left(t\right)\in
B|X\left(s\right)\right\}.
\end{equation}
\end{Def}
\begin{Note}
Si se dice que $\left\{X\left(t\right)\right\}$ es un Proceso de
Markov sin mencionar $\mathcal{F}\left(t\right)$, se asumir\'a que
\begin{eqnarray*}
\mathcal{F}\left(t\right)=\mathcal{F}_{0}\left(t\right)=\mathcal{F}\left(X\left(r\right),r\leq
t\right),
\end{eqnarray*}
entonces la ecuaci\'on (\ref{Prop.Markov}) se puede escribir como
\begin{equation}
P\left\{X\left(t\right)\in B|X\left(r\right),r\leq s\right\} =
P\left\{X\left(t\right)\in B|X\left(s\right)\right\}
\end{equation}
\end{Note}

\begin{Teo}
Sea $\left(X_{n},\mathcal{F}_{n},n=0,1,\ldots,\right\}$ Proceso de
Markov con espacio de estados $\left(S_{0},\chi_{0}\right)$
generado por una distribuici\'on inicial $P_{o}$ y probabilidad de
transici\'on $p_{mn}$, para $m,n=0,1,\ldots,$ $m<n$, que por
notaci\'on se escribir\'a como $p\left(m,n,x,B\right)\rightarrow
p_{mn}\left(x,B\right)$. Sea $S$ tiempo de paro relativo a la
$\sigma$-\'algebra $\mathcal{F}_{n}$. Sea $T$ funci\'on medible,
$T:\Omega\rightarrow\left\{0,1,\ldots,\right\}$. Sup\'ongase que
$T\geq S$, entonces $T$ es tiempo de paro. Si $B\in\chi_{0}$,
entonces
\begin{equation}\label{Prop.Fuerte.Markov}
P\left\{X\left(T\right)\in
B,T<\infty|\mathcal{F}\left(S\right)\right\} =
p\left(S,T,X\left(s\right),B\right)
\end{equation}
en $\left\{T<\infty\right\}$.
\end{Teo}


\begin{Def}
Sea $X$ un conjunto y $\mathcal{F}$ una $\sigma$-\'algebra de
subconjuntos de $X$, la pareja $\left(X,\mathcal{F}\right)$ es
llamado espacio medible. Un subconjunto $A$ de $X$ es llamado
medible, o medible con respecto a $\mathcal{F}$, si
$A\in\mathcal{F}$.
\end{Def}

\begin{Def}
Sea $\left(X,\mathcal{F},\mu\right)$ espacio de medida. Se dice
que la medida $\mu$ es $\sigma$-finita si se puede escribir
$X=\bigcup_{n\geq1}X_{n}$ con $X_{n}\in\mathcal{F}$ y
$\mu\left(X_{n}\right)<\infty$.
\end{Def}

\begin{Def}\label{Cto.Borel}
Sea $X$ el conjunto de los n\'umeros reales $\rea$. El \'algebra
de Borel es la $\sigma$-\'algebra $B$ generada por los intervalos
abiertos $\left(a,b\right)\in\rea$. Cualquier conjunto en $B$ es
llamado {\em Conjunto de Borel}.
\end{Def}

\begin{Def}\label{Funcion.Medible}
Una funci\'on $f:X\rightarrow\rea$, es medible si para cualquier
n\'umero real $\alpha$ el conjunto
\[\left\{x\in X:f\left(x\right)>\alpha\right\}\]
pertenece a $\mathcal{F}$. Equivalentemente, se dice que $f$ es
medible si
\[f^{-1}\left(\left(\alpha,\infty\right)\right)=\left\{x\in X:f\left(x\right)>\alpha\right\}\in\mathcal{F}.\]
\end{Def}


\begin{Def}\label{Def.Cilindros}
Sean $\left(\Omega_{i},\mathcal{F}_{i}\right)$, $i=1,2,\ldots,$
espacios medibles y $\Omega=\prod_{i=1}^{\infty}\Omega_{i}$ el
conjunto de todas las sucesiones
$\left(\omega_{1},\omega_{2},\ldots,\right)$ tales que
$\omega_{i}\in\Omega_{i}$, $i=1,2,\ldots,$. Si
$B^{n}\subset\prod_{i=1}^{\infty}\Omega_{i}$, definimos
$B_{n}=\left\{\omega\in\Omega:\left(\omega_{1},\omega_{2},\ldots,\omega_{n}\right)\in
B^{n}\right\}$. Al conjunto $B_{n}$ se le llama {\em cilindro} con
base $B^{n}$, el cilindro es llamado medible si
$B^{n}\in\prod_{i=1}^{\infty}\mathcal{F}_{i}$.
\end{Def}


\begin{Def}\label{Def.Proc.Adaptado}[TSP, Ash \cite{RBA}]
Sea $X\left(t\right),t\geq0$ proceso estoc\'astico, el proceso es
adaptado a la familia de $\sigma$-\'algebras $\mathcal{F}_{t}$,
para $t\geq0$, si para $s<t$ implica que
$\mathcal{F}_{s}\subset\mathcal{F}_{t}$, y $X\left(t\right)$ es
$\mathcal{F}_{t}$-medible para cada $t$. Si no se especifica
$\mathcal{F}_{t}$ entonces se toma $\mathcal{F}_{t}$ como
$\mathcal{F}\left(X\left(s\right),s\leq t\right)$, la m\'as
peque\~na $\sigma$-\'algebra de subconjuntos de $\Omega$ que hace
que cada $X\left(s\right)$, con $s\leq t$ sea Borel medible.
\end{Def}


\begin{Def}\label{Def.Tiempo.Paro}[TSP, Ash \cite{RBA}]
Sea $\left\{\mathcal{F}\left(t\right),t\geq0\right\}$ familia
creciente de sub $\sigma$-\'algebras. es decir,
$\mathcal{F}\left(s\right)\subset\mathcal{F}\left(t\right)$ para
$s\leq t$. Un tiempo de paro para $\mathcal{F}\left(t\right)$ es
una funci\'on $T:\Omega\rightarrow\left[0,\infty\right]$ tal que
$\left\{T\leq t\right\}\in\mathcal{F}\left(t\right)$ para cada
$t\geq0$. Un tiempo de paro para el proceso estoc\'astico
$X\left(t\right),t\geq0$ es un tiempo de paro para las
$\sigma$-\'algebras
$\mathcal{F}\left(t\right)=\mathcal{F}\left(X\left(s\right)\right)$.
\end{Def}

\begin{Def}
Sea $X\left(t\right),t\geq0$ proceso estoc\'astico, con
$\left(S,\chi\right)$ espacio de estados. Se dice que el proceso
es adaptado a $\left\{\mathcal{F}\left(t\right)\right\}$, es
decir, si para cualquier $s,t\in I$, $I$ conjunto de \'indices,
$s<t$, se tiene que
$\mathcal{F}\left(s\right)\subset\mathcal{F}\left(t\right)$ y
$X\left(t\right)$ es $\mathcal{F}\left(t\right)$-medible,
\end{Def}

\begin{Def}
Sea $X\left(t\right),t\geq0$ proceso estoc\'astico, se dice que es
un Proceso de Markov relativo a $\mathcal{F}\left(t\right)$ o que
$\left\{X\left(t\right),\mathcal{F}\left(t\right)\right\}$ es de
Markov si y s\'olo si para cualquier conjunto $B\in\chi$,  y
$s,t\in I$, $s<t$ se cumple que
\begin{equation}\label{Prop.Markov}
P\left\{X\left(t\right)\in
B|\mathcal{F}\left(s\right)\right\}=P\left\{X\left(t\right)\in
B|X\left(s\right)\right\}.
\end{equation}
\end{Def}
\begin{Note}
Si se dice que $\left\{X\left(t\right)\right\}$ es un Proceso de
Markov sin mencionar $\mathcal{F}\left(t\right)$, se asumir\'a que
\begin{eqnarray*}
\mathcal{F}\left(t\right)=\mathcal{F}_{0}\left(t\right)=\mathcal{F}\left(X\left(r\right),r\leq
t\right),
\end{eqnarray*}
entonces la ecuaci\'on (\ref{Prop.Markov}) se puede escribir como
\begin{equation}
P\left\{X\left(t\right)\in B|X\left(r\right),r\leq s\right\} =
P\left\{X\left(t\right)\in B|X\left(s\right)\right\}
\end{equation}
\end{Note}


%\newpage
%_______________________________________________________________
%\subsection{Procesos de Estados de Markov}
%_______________________________________________________________

\begin{Teo}
Sea $\left(X_{n},\mathcal{F}_{n},n=0,1,\ldots,\right\}$ Proceso de
Markov con espacio de estados $\left(S_{0},\chi_{0}\right)$
generado por una distribuici\'on inicial $P_{o}$ y probabilidad de
transici\'on $p_{mn}$, para $m,n=0,1,\ldots,$ $m<n$, que por
notaci\'on se escribir\'a como $p\left(m,n,x,B\right)\rightarrow
p_{mn}\left(x,B\right)$. Sea $S$ tiempo de paro relativo a la
$\sigma$-\'algebra $\mathcal{F}_{n}$. Sea $T$ funci\'on medible,
$T:\Omega\rightarrow\left\{0,1,\ldots,\right\}$. Sup\'ongase que
$T\geq S$, entonces $T$ es tiempo de paro. Si $B\in\chi_{0}$,
entonces
\begin{equation}\label{Prop.Fuerte.Markov}
P\left\{X\left(T\right)\in
B,T<\infty|\mathcal{F}\left(S\right)\right\} =
p\left(S,T,X\left(s\right),B\right)
\end{equation}
en $\left\{T<\infty\right\}$.
\end{Teo}


Sea $K$ conjunto numerable y sea $d:K\rightarrow\nat$ funci\'on.
Para $v\in K$, $M_{v}$ es un conjunto abierto de
$\rea^{d\left(v\right)}$. Entonces \[E=\bigcup_{v\in
K}M_{v}=\left\{\left(v,\zeta\right):v\in K,\zeta\in
M_{v}\right\}.\]

Sea $\mathcal{E}$ la clase de conjuntos medibles en $E$:
\[\mathcal{E}=\left\{\bigcup_{v\in K}A_{v}:A_{v}\in \mathcal{M}_{v}\right\}.\]

donde $l{M}$ son los conjuntos de Borel de $M_{v}$.
Entonces $\left(E,\mathcal{E}\right)$ es un espacio de Borel. El
estado del proceso se denotar\'a por
$\mathbf{x}_{t}=\left(v_{t},\zeta_{t}\right)$. La distribuci\'on
de $\left(\mathbf{x}_{t}\right)$ est\'a determinada por por los
siguientes objetos:

\begin{itemize}
\item[i)] Los campos vectoriales $\left(\mathcal{H}_{v},v\in
K\right)$. \item[ii)] Una funci\'on medible $\lambda:E\rightarrow
\rea_{+}$. \item[iii)] Una medida de transici\'on
$Q:\mathcal{E}\times\left(E\cup\Gamma^{*}\right)\rightarrow\left[0,1\right]$
donde
\begin{equation}
\Gamma^{*}=\bigcup_{v\in K}\partial^{*}M_{v}.
\end{equation}
y
\begin{equation}
\partial^{*}M_{v}=\left\{z\in\partial M_{v}:\mathbf{\mathbf{\phi}_{v}\left(t,\zeta\right)=\mathbf{z}}\textrm{ para alguna }\left(t,\zeta\right)\in\rea_{+}\times M_{v}\right\}.
\end{equation}
$\partial M_{v}$ denota  la frontera de $M_{v}$.
\end{itemize}

El campo vectorial $\left(\mathcal{H}_{v},v\in K\right)$ se supone
tal que para cada $\mathbf{z}\in M_{v}$ existe una \'unica curva
integral $\mathbf{\phi}_{v}\left(t,\zeta\right)$ que satisface la
ecuaci\'on

\begin{equation}
\frac{d}{dt}f\left(\zeta_{t}\right)=\mathcal{H}f\left(\zeta_{t}\right),
\end{equation}
con $\zeta_{0}=\mathbf{z}$, para cualquier funci\'on suave
$f:\rea^{d}\rightarrow\rea$ y $\mathcal{H}$ denota el operador
diferencial de primer orden, con $\mathcal{H}=\mathcal{H}_{v}$ y
$\zeta_{t}=\mathbf{\phi}\left(t,\mathbf{z}\right)$. Adem\'as se
supone que $\mathcal{H}_{v}$ es conservativo, es decir, las curvas
integrales est\'an definidas para todo $t>0$.

Para $\mathbf{x}=\left(v,\zeta\right)\in E$ se denota
\[t^{*}\mathbf{x}=inf\left\{t>0:\mathbf{\phi}_{v}\left(t,\zeta\right)\in\partial^{*}M_{v}\right\}\]

En lo que respecta a la funci\'on $\lambda$, se supondr\'a que
para cada $\left(v,\zeta\right)\in E$ existe un $\epsilon>0$ tal
que la funci\'on
$s\rightarrow\lambda\left(v,\phi_{v}\left(s,\zeta\right)\right)\in
E$ es integrable para $s\in\left[0,\epsilon\right)$. La medida de
transici\'on $Q\left(A;\mathbf{x}\right)$ es una funci\'on medible
de $\mathbf{x}$ para cada $A\in\mathcal{E}$, definida para
$\mathbf{x}\in E\cup\Gamma^{*}$ y es una medida de probabilidad en
$\left(E,\mathcal{E}\right)$ para cada $\mathbf{x}\in E$.

El movimiento del proceso $\left(\mathbf{x}_{t}\right)$ comenzando
en $\mathbf{x}=\left(n,\mathbf{z}\right)\in E$ se puede construir
de la siguiente manera, def\'inase la funci\'on $F$ por

\begin{equation}
F\left(t\right)=\left\{\begin{array}{ll}\\
exp\left(-\int_{0}^{t}\lambda\left(n,\phi_{n}\left(s,\mathbf{z}\right)\right)ds\right), & t<t^{*}\left(\mathbf{x}\right),\\
0, & t\geq t^{*}\left(\mathbf{x}\right)
\end{array}\right.
\end{equation}

Sea $T_{1}$ una variable aleatoria tal que
$\prob\left[T_{1}>t\right]=F\left(t\right)$, ahora sea la variable
aleatoria $\left(N,Z\right)$ con distribuici\'on
$Q\left(\cdot;\phi_{n}\left(T_{1},\mathbf{z}\right)\right)$. La
trayectoria de $\left(\mathbf{x}_{t}\right)$ para $t\leq T_{1}$ es
\begin{eqnarray*}
\mathbf{x}_{t}=\left(v_{t},\zeta_{t}\right)=\left\{\begin{array}{ll}
\left(n,\phi_{n}\left(t,\mathbf{z}\right)\right), & t<T_{1},\\
\left(N,\mathbf{Z}\right), & t=t_{1}.
\end{array}\right.
\end{eqnarray*}

Comenzando en $\mathbf{x}_{T_{1}}$ se selecciona el siguiente
tiempo de intersalto $T_{2}-T_{1}$ lugar del post-salto
$\mathbf{x}_{T_{2}}$ de manera similar y as\'i sucesivamente. Este
procedimiento nos da una trayectoria determinista por partes
$\mathbf{x}_{t}$ con tiempos de salto $T_{1},T_{2},\ldots$. Bajo
las condiciones enunciadas para $\lambda,T_{1}>0$  y
$T_{1}-T_{2}>0$ para cada $i$, con probabilidad 1. Se supone que
se cumple la siguiente condici\'on.

\begin{Sup}[Supuesto 3.1, Davis \cite{Davis}]\label{Sup3.1.Davis}
Sea $N_{t}:=\sum_{t}\indora_{\left(t\geq t\right)}$ el n\'umero de
saltos en $\left[0,t\right]$. Entonces
\begin{equation}
\esp\left[N_{t}\right]<\infty\textrm{ para toda }t.
\end{equation}
\end{Sup}

es un proceso de Markov, m\'as a\'un, es un Proceso Fuerte de
Markov, es decir, la Propiedad Fuerte de Markov\footnote{Revisar
p\'agina 362, y 364 de Davis \cite{Davis}.} se cumple para
cualquier tiempo de paro.
%_________________________________________________________________________
%\subsection{Teor\'ia General de Procesos Estoc\'asticos}
%_________________________________________________________________________
En esta secci\'on se har\'an las siguientes consideraciones: $E$
es un espacio m\'etrico separable y la m\'etrica $d$ es compatible
con la topolog\'ia.

\begin{Def}
Una medida finita, $\lambda$ en la $\sigma$-\'algebra de Borel de
un espacio metrizable $E$ se dice cerrada si
\begin{equation}\label{Eq.A2.3}
\lambda\left(E\right)=sup\left\{\lambda\left(K\right):K\textrm{ es
compacto en }E\right\}.
\end{equation}
\end{Def}

\begin{Def}
$E$ es un espacio de Rad\'on si cada medida finita en
$\left(E,\mathcal{B}\left(E\right)\right)$ es regular interior o cerrada,
{\em tight}.
\end{Def}


El siguiente teorema nos permite tener una mejor caracterizaci\'on de los espacios de Rad\'on:
\begin{Teo}\label{Tma.A2.2}
Sea $E$ espacio separable metrizable. Entonces $E$ es de Rad\'on
si y s\'olo s\'i cada medida finita en
$\left(E,\mathcal{B}\left(E\right)\right)$ es cerrada.
\end{Teo}

%_________________________________________________________________________________________
%\subsection{Propiedades de Markov}
%_________________________________________________________________________________________

Sea $E$ espacio de estados, tal que $E$ es un espacio de Rad\'on, $\mathcal{B}\left(E\right)$ $\sigma$-\'algebra de Borel en $E$, que se denotar\'a por $\mathcal{E}$.

Sea $\left(X,\mathcal{G},\prob\right)$ espacio de probabilidad,
$I\subset\rea$ conjunto de índices. Sea $\mathcal{F}_{\leq t}$ la
$\sigma$-\'algebra natural definida como
$\sigma\left\{f\left(X_{r}\right):r\in I, r\leq
t,f\in\mathcal{E}\right\}$. Se considerar\'a una
$\sigma$-\'algebra m\'as general\footnote{qu\'e se quiere decir
con el t\'ermino: m\'as general?}, $ \left(\mathcal{G}_{t}\right)$
tal que $\left(X_{t}\right)$ sea $\mathcal{E}$-adaptado.

\begin{Def}
Una familia $\left(P_{s,t}\right)$ de kernels de Markov en $\left(E,\mathcal{E}\right)$ indexada por pares $s,t\in I$, con $s\leq t$ es una funci\'on de transici\'on en $\ER$, si  para todo $r\leq s< t$ en $I$ y todo $x\in E$, $B\in\mathcal{E}$
\begin{equation}\label{Eq.Kernels}
P_{r,t}\left(x,B\right)=\int_{E}P_{r,s}\left(x,dy\right)P_{s,t}\left(y,B\right)\footnote{Ecuaci\'on de Chapman-Kolmogorov}.
\end{equation}
\end{Def}

Se dice que la funci\'on de transici\'on $\KM$ en $\ER$ es la funci\'on de transici\'on para un proceso $\PE$  con valores en $E$ y que satisface la propiedad de Markov\footnote{\begin{equation}\label{Eq.1.4.S}
\prob\left\{H|\mathcal{G}_{t}\right\}=\prob\left\{H|X_{t}\right\}\textrm{ }H\in p\mathcal{F}_{\geq t}.
\end{equation}} (\ref{Eq.1.4.S}) relativa a $\left(\mathcal{G}_{t}\right)$ si

\begin{equation}\label{Eq.1.6.S}
\prob\left\{f\left(X_{t}\right)|\mathcal{G}_{s}\right\}=P_{s,t}f\left(X_{t}\right)\textrm{ }s\leq t\in I,\textrm{ }f\in b\mathcal{E}.
\end{equation}

\begin{Def}
Una familia $\left(P_{t}\right)_{t\geq0}$ de kernels de Markov en $\ER$ es llamada {\em Semigrupo de Transici\'on de Markov} o {\em Semigrupo de Transici\'on} si
\[P_{t+s}f\left(x\right)=P_{t}\left(P_{s}f\right)\left(x\right),\textrm{ }t,s\geq0,\textrm{ }x\in E\textrm{ }f\in b\mathcal{E}\footnote{Definir los t\'ermino $b\mathcal{E}$ y $p\mathcal{E}$}.\]
\end{Def}
\begin{Note}
Si la funci\'on de transici\'on $\KM$ es llamada homog\'enea si $P_{s,t}=P_{t-s}$.
\end{Note}

Un proceso de Markov que satisface la ecuaci\'on (\ref{Eq.1.6.S}) con funci\'on de transici\'on homog\'enea $\left(P_{t}\right)$ tiene la propiedad caracter\'istica
\begin{equation}\label{Eq.1.8.S}
\prob\left\{f\left(X_{t+s}\right)|\mathcal{G}_{t}\right\}=P_{s}f\left(X_{t}\right)\textrm{ }t,s\geq0,\textrm{ }f\in b\mathcal{E}.
\end{equation}
La ecuaci\'on anterior es la {\em Propiedad Simple de Markov} de $X$ relativa a $\left(P_{t}\right)$.

En este sentido el proceso $\PE$ cumple con la propiedad de Markov (\ref{Eq.1.8.S}) relativa a $\left(\Omega,\mathcal{G},\mathcal{G}_{t},\prob\right)$ con semigrupo de transici\'on $\left(P_{t}\right)$.
%_________________________________________________________________________________________
%\subsection{Primer Condici\'on de Regularidad}
%_________________________________________________________________________________________
%\newcommand{\EM}{\left(\Omega,\mathcal{G},\prob\right)}
%\newcommand{\E4}{\left(\Omega,\mathcal{G},\mathcal{G}_{t},\prob\right)}
\begin{Def}
Un proceso estoc\'astico $\PE$ definido en
$\left(\Omega,\mathcal{G},\prob\right)$ con valores en el espacio
topol\'ogico $E$ es continuo por la derecha si cada trayectoria
muestral $t\rightarrow X_{t}\left(w\right)$ es un mapeo continuo
por la derecha de $I$ en $E$.
\end{Def}

\begin{Def}[HD1]\label{Eq.2.1.S}
Un semigrupo de Markov $\left(P_{t}\right)$ en un espacio de
Rad\'on $E$ se dice que satisface la condici\'on {\em HD1} si,
dada una medida de probabilidad $\mu$ en $E$, existe una
$\sigma$-\'algebra $\mathcal{E^{*}}$ con
$\mathcal{E}\subset\mathcal{E}^{*}$ y
$P_{t}\left(b\mathcal{E}^{*}\right)\subset b\mathcal{E}^{*}$, y un
$\mathcal{E}^{*}$-proceso $E$-valuado continuo por la derecha
$\PE$ en alg\'un espacio de probabilidad filtrado
$\left(\Omega,\mathcal{G},\mathcal{G}_{t},\prob\right)$ tal que
$X=\left(\Omega,\mathcal{G},\mathcal{G}_{t},\prob\right)$ es de
Markov (Homog\'eneo) con semigrupo de transici\'on $(P_{t})$ y
distribuci\'on inicial $\mu$.
\end{Def}

Consid\'erese la colecci\'on de variables aleatorias $X_{t}$
definidas en alg\'un espacio de probabilidad, y una colecci\'on de
medidas $\mathbf{P}^{x}$ tales que
$\mathbf{P}^{x}\left\{X_{0}=x\right\}$, y bajo cualquier
$\mathbf{P}^{x}$, $X_{t}$ es de Markov con semigrupo
$\left(P_{t}\right)$. $\mathbf{P}^{x}$ puede considerarse como la
distribuci\'on condicional de $\mathbf{P}$ dado $X_{0}=x$.

\begin{Def}\label{Def.2.2.S}
Sea $E$ espacio de Rad\'on, $\SG$ semigrupo de Markov en $\ER$. La colecci\'on $\mathbf{X}=\left(\Omega,\mathcal{G},\mathcal{G}_{t},X_{t},\theta_{t},\CM\right)$ es un proceso $\mathcal{E}$-Markov continuo por la derecha simple, con espacio de estados $E$ y semigrupo de transici\'on $\SG$ en caso de que $\mathbf{X}$ satisfaga las siguientes condiciones:
\begin{itemize}
\item[i)] $\left(\Omega,\mathcal{G},\mathcal{G}_{t}\right)$ es un espacio de medida filtrado, y $X_{t}$ es un proceso $E$-valuado continuo por la derecha $\mathcal{E}^{*}$-adaptado a $\left(\mathcal{G}_{t}\right)$;

\item[ii)] $\left(\theta_{t}\right)_{t\geq0}$ es una colecci\'on de operadores {\em shift} para $X$, es decir, mapea $\Omega$ en s\'i mismo satisfaciendo para $t,s\geq0$,

\begin{equation}\label{Eq.Shift}
\theta_{t}\circ\theta_{s}=\theta_{t+s}\textrm{ y }X_{t}\circ\theta_{t}=X_{t+s};
\end{equation}

\item[iii)] Para cualquier $x\in E$,$\CM\left\{X_{0}=x\right\}=1$, y el proceso $\PE$ tiene la propiedad de Markov (\ref{Eq.1.8.S}) con semigrupo de transici\'on $\SG$ relativo a $\left(\Omega,\mathcal{G},\mathcal{G}_{t},\CM\right)$.
\end{itemize}
\end{Def}

\begin{Def}[HD2]\label{Eq.2.2.S}
Para cualquier $\alpha>0$ y cualquier $f\in S^{\alpha}$, el proceso $t\rightarrow f\left(X_{t}\right)$ es continuo por la derecha casi seguramente.
\end{Def}

\begin{Def}\label{Def.PD}
Un sistema $\mathbf{X}=\left(\Omega,\mathcal{G},\mathcal{G}_{t},X_{t},\theta_{t},\CM\right)$ es un proceso derecho en el espacio de Rad\'on $E$ con semigrupo de transici\'on $\SG$ provisto de:
\begin{itemize}
\item[i)] $\mathbf{X}$ es una realizaci\'on  continua por la derecha, \ref{Def.2.2.S}, de $\SG$.

\item[ii)] $\mathbf{X}$ satisface la condicion HD2, \ref{Eq.2.2.S}, relativa a $\mathcal{G}_{t}$.

\item[iii)] $\mathcal{G}_{t}$ es aumentado y continuo por la derecha.
\end{itemize}
\end{Def}

%__________________________________________________________________________________________
\subsection{Procesos Regenerativos Estacionarios - Stidham \cite{Stidham}}
%__________________________________________________________________________________________


Un proceso estoc\'astico a tiempo continuo $\left\{V\left(t\right),t\geq0\right\}$ es un proceso regenerativo si existe una sucesi\'on de variables aleatorias independientes e id\'enticamente distribuidas $\left\{X_{1},X_{2},\ldots\right\}$, sucesi\'on de renovaci\'on, tal que para cualquier conjunto de Borel $A$, 

\begin{eqnarray*}
\prob\left\{V\left(t\right)\in A|X_{1}+X_{2}+\cdots+X_{R\left(t\right)}=s,\left\{V\left(\tau\right),\tau<s\right\}\right\}=\prob\left\{V\left(t-s\right)\in A|X_{1}>t-s\right\},
\end{eqnarray*}
para todo $0\leq s\leq t$, donde $R\left(t\right)=\max\left\{X_{1}+X_{2}+\cdots+X_{j}\leq t\right\}=$n\'umero de renovaciones ({\emph{puntos de regeneraci\'on}}) que ocurren en $\left[0,t\right]$. El intervalo $\left[0,X_{1}\right)$ es llamado {\emph{primer ciclo de regeneraci\'on}} de $\left\{V\left(t \right),t\geq0\right\}$, $\left[X_{1},X_{1}+X_{2}\right)$ el {\emph{segundo ciclo de regeneraci\'on}}, y as\'i sucesivamente.

Sea $X=X_{1}$ y sea $F$ la funci\'on de distrbuci\'on de $X$


\begin{Def}
Se define el proceso estacionario, $\left\{V^{*}\left(t\right),t\geq0\right\}$, para $\left\{V\left(t\right),t\geq0\right\}$ por

\begin{eqnarray*}
\prob\left\{V\left(t\right)\in A\right\}=\frac{1}{\esp\left[X\right]}\int_{0}^{\infty}\prob\left\{V\left(t+x\right)\in A|X>x\right\}\left(1-F\left(x\right)\right)dx,
\end{eqnarray*} 
para todo $t\geq0$ y todo conjunto de Borel $A$.
\end{Def}

\begin{Def}
Una distribuci\'on se dice que es {\emph{aritm\'etica}} si todos sus puntos de incremento son m\'ultiplos de la forma $0,\lambda, 2\lambda,\ldots$ para alguna $\lambda>0$ entera.
\end{Def}


\begin{Def}
Una modificaci\'on medible de un proceso $\left\{V\left(t\right),t\geq0\right\}$, es una versi\'on de este, $\left\{V\left(t,w\right)\right\}$ conjuntamente medible para $t\geq0$ y para $w\in S$, $S$ espacio de estados para $\left\{V\left(t\right),t\geq0\right\}$.
\end{Def}

\begin{Teo}
Sea $\left\{V\left(t\right),t\geq\right\}$ un proceso regenerativo no negativo con modificaci\'on medible. Sea $\esp\left[X\right]<\infty$. Entonces el proceso estacionario dado por la ecuaci\'on anterior est\'a bien definido y tiene funci\'on de distribuci\'on independiente de $t$, adem\'as
\begin{itemize}
\item[i)] \begin{eqnarray*}
\esp\left[V^{*}\left(0\right)\right]&=&\frac{\esp\left[\int_{0}^{X}V\left(s\right)ds\right]}{\esp\left[X\right]}\end{eqnarray*}
\item[ii)] Si $\esp\left[V^{*}\left(0\right)\right]<\infty$, equivalentemente, si $\esp\left[\int_{0}^{X}V\left(s\right)ds\right]<\infty$,entonces
\begin{eqnarray*}
\frac{\int_{0}^{t}V\left(s\right)ds}{t}\rightarrow\frac{\esp\left[\int_{0}^{X}V\left(s\right)ds\right]}{\esp\left[X\right]}
\end{eqnarray*}
con probabilidad 1 y en media, cuando $t\rightarrow\infty$.
\end{itemize}
\end{Teo}

\begin{Coro}
Sea $\left\{V\left(t\right),t\geq0\right\}$ un proceso regenerativo no negativo, con modificaci\'on medible. Si $\esp <\infty$, $F$ es no-aritm\'etica, y para todo $x\geq0$, $P\left\{V\left(t\right)\leq x,C>x\right\}$ es de variaci\'on acotada como funci\'on de $t$ en cada intervalo finito $\left[0,\tau\right]$, entonces $V\left(t\right)$ converge en distribuci\'on  cuando $t\rightarrow\infty$ y $$\esp V=\frac{\esp \int_{0}^{X}V\left(s\right)ds}{\esp X}$$
Donde $V$ tiene la distribuci\'on l\'imite de $V\left(t\right)$ cuando $t\rightarrow\infty$.

\end{Coro}

Para el caso discreto se tienen resultados similares.



%______________________________________________________________________
\subsection{Procesos de Renovaci\'on}
%______________________________________________________________________

\begin{Def}%\label{Def.Tn}
Sean $0\leq T_{1}\leq T_{2}\leq \ldots$ son tiempos aleatorios infinitos en los cuales ocurren ciertos eventos. El n\'umero de tiempos $T_{n}$ en el intervalo $\left[0,t\right)$ es

\begin{eqnarray}
N\left(t\right)=\sum_{n=1}^{\infty}\indora\left(T_{n}\leq t\right),
\end{eqnarray}
para $t\geq0$.
\end{Def}

Si se consideran los puntos $T_{n}$ como elementos de $\rea_{+}$, y $N\left(t\right)$ es el n\'umero de puntos en $\rea$. El proceso denotado por $\left\{N\left(t\right):t\geq0\right\}$, denotado por $N\left(t\right)$, es un proceso puntual en $\rea_{+}$. Los $T_{n}$ son los tiempos de ocurrencia, el proceso puntual $N\left(t\right)$ es simple si su n\'umero de ocurrencias son distintas: $0<T_{1}<T_{2}<\ldots$ casi seguramente.

\begin{Def}
Un proceso puntual $N\left(t\right)$ es un proceso de renovaci\'on si los tiempos de interocurrencia $\xi_{n}=T_{n}-T_{n-1}$, para $n\geq1$, son independientes e identicamente distribuidos con distribuci\'on $F$, donde $F\left(0\right)=0$ y $T_{0}=0$. Los $T_{n}$ son llamados tiempos de renovaci\'on, referente a la independencia o renovaci\'on de la informaci\'on estoc\'astica en estos tiempos. Los $\xi_{n}$ son los tiempos de inter-renovaci\'on, y $N\left(t\right)$ es el n\'umero de renovaciones en el intervalo $\left[0,t\right)$
\end{Def}


\begin{Note}
Para definir un proceso de renovaci\'on para cualquier contexto, solamente hay que especificar una distribuci\'on $F$, con $F\left(0\right)=0$, para los tiempos de inter-renovaci\'on. La funci\'on $F$ en turno degune las otra variables aleatorias. De manera formal, existe un espacio de probabilidad y una sucesi\'on de variables aleatorias $\xi_{1},\xi_{2},\ldots$ definidas en este con distribuci\'on $F$. Entonces las otras cantidades son $T_{n}=\sum_{k=1}^{n}\xi_{k}$ y $N\left(t\right)=\sum_{n=1}^{\infty}\indora\left(T_{n}\leq t\right)$, donde $T_{n}\rightarrow\infty$ casi seguramente por la Ley Fuerte de los Grandes Números.
\end{Note}

%___________________________________________________________________________________________
%
\subsection{Teorema Principal de Renovaci\'on}
%___________________________________________________________________________________________
%

\begin{Note} Una funci\'on $h:\rea_{+}\rightarrow\rea$ es Directamente Riemann Integrable en los siguientes casos:
\begin{itemize}
\item[a)] $h\left(t\right)\geq0$ es decreciente y Riemann Integrable.
\item[b)] $h$ es continua excepto posiblemente en un conjunto de Lebesgue de medida 0, y $|h\left(t\right)|\leq b\left(t\right)$, donde $b$ es DRI.
\end{itemize}
\end{Note}

\begin{Teo}[Teorema Principal de Renovaci\'on]
Si $F$ es no aritm\'etica y $h\left(t\right)$ es Directamente Riemann Integrable (DRI), entonces

\begin{eqnarray*}
lim_{t\rightarrow\infty}U\star h=\frac{1}{\mu}\int_{\rea_{+}}h\left(s\right)ds.
\end{eqnarray*}
\end{Teo}

\begin{Prop}
Cualquier funci\'on $H\left(t\right)$ acotada en intervalos finitos y que es 0 para $t<0$ puede expresarse como
\begin{eqnarray*}
H\left(t\right)=U\star h\left(t\right)\textrm{,  donde }h\left(t\right)=H\left(t\right)-F\star H\left(t\right)
\end{eqnarray*}
\end{Prop}

\begin{Def}
Un proceso estoc\'astico $X\left(t\right)$ es crudamente regenerativo en un tiempo aleatorio positivo $T$ si
\begin{eqnarray*}
\esp\left[X\left(T+t\right)|T\right]=\esp\left[X\left(t\right)\right]\textrm{, para }t\geq0,\end{eqnarray*}
y con las esperanzas anteriores finitas.
\end{Def}

\begin{Prop}
Sup\'ongase que $X\left(t\right)$ es un proceso crudamente regenerativo en $T$, que tiene distribuci\'on $F$. Si $\esp\left[X\left(t\right)\right]$ es acotado en intervalos finitos, entonces
\begin{eqnarray*}
\esp\left[X\left(t\right)\right]=U\star h\left(t\right)\textrm{,  donde }h\left(t\right)=\esp\left[X\left(t\right)\indora\left(T>t\right)\right].
\end{eqnarray*}
\end{Prop}

\begin{Teo}[Regeneraci\'on Cruda]
Sup\'ongase que $X\left(t\right)$ es un proceso con valores positivo crudamente regenerativo en $T$, y def\'inase $M=\sup\left\{|X\left(t\right)|:t\leq T\right\}$. Si $T$ es no aritm\'etico y $M$ y $MT$ tienen media finita, entonces
\begin{eqnarray*}
lim_{t\rightarrow\infty}\esp\left[X\left(t\right)\right]=\frac{1}{\mu}\int_{\rea_{+}}h\left(s\right)ds,
\end{eqnarray*}
donde $h\left(t\right)=\esp\left[X\left(t\right)\indora\left(T>t\right)\right]$.
\end{Teo}

%___________________________________________________________________________________________
%
\subsection{Propiedades de los Procesos de Renovaci\'on}
%___________________________________________________________________________________________
%

Los tiempos $T_{n}$ est\'an relacionados con los conteos de $N\left(t\right)$ por

\begin{eqnarray*}
\left\{N\left(t\right)\geq n\right\}&=&\left\{T_{n}\leq t\right\}\\
T_{N\left(t\right)}\leq &t&<T_{N\left(t\right)+1},
\end{eqnarray*}

adem\'as $N\left(T_{n}\right)=n$, y 

\begin{eqnarray*}
N\left(t\right)=\max\left\{n:T_{n}\leq t\right\}=\min\left\{n:T_{n+1}>t\right\}
\end{eqnarray*}

Por propiedades de la convoluci\'on se sabe que

\begin{eqnarray*}
P\left\{T_{n}\leq t\right\}=F^{n\star}\left(t\right)
\end{eqnarray*}
que es la $n$-\'esima convoluci\'on de $F$. Entonces 

\begin{eqnarray*}
\left\{N\left(t\right)\geq n\right\}&=&\left\{T_{n}\leq t\right\}\\
P\left\{N\left(t\right)\leq n\right\}&=&1-F^{\left(n+1\right)\star}\left(t\right)
\end{eqnarray*}

Adem\'as usando el hecho de que $\esp\left[N\left(t\right)\right]=\sum_{n=1}^{\infty}P\left\{N\left(t\right)\geq n\right\}$
se tiene que

\begin{eqnarray*}
\esp\left[N\left(t\right)\right]=\sum_{n=1}^{\infty}F^{n\star}\left(t\right)
\end{eqnarray*}

\begin{Prop}
Para cada $t\geq0$, la funci\'on generadora de momentos $\esp\left[e^{\alpha N\left(t\right)}\right]$ existe para alguna $\alpha$ en una vecindad del 0, y de aqu\'i que $\esp\left[N\left(t\right)^{m}\right]<\infty$, para $m\geq1$.
\end{Prop}


\begin{Note}
Si el primer tiempo de renovaci\'on $\xi_{1}$ no tiene la misma distribuci\'on que el resto de las $\xi_{n}$, para $n\geq2$, a $N\left(t\right)$ se le llama Proceso de Renovaci\'on retardado, donde si $\xi$ tiene distribuci\'on $G$, entonces el tiempo $T_{n}$ de la $n$-\'esima renovaci\'on tiene distribuci\'on $G\star F^{\left(n-1\right)\star}\left(t\right)$
\end{Note}


\begin{Teo}
Para una constante $\mu\leq\infty$ ( o variable aleatoria), las siguientes expresiones son equivalentes:

\begin{eqnarray}
lim_{n\rightarrow\infty}n^{-1}T_{n}&=&\mu,\textrm{ c.s.}\\
lim_{t\rightarrow\infty}t^{-1}N\left(t\right)&=&1/\mu,\textrm{ c.s.}
\end{eqnarray}
\end{Teo}


Es decir, $T_{n}$ satisface la Ley Fuerte de los Grandes N\'umeros s\'i y s\'olo s\'i $N\left/t\right)$ la cumple.


\begin{Coro}[Ley Fuerte de los Grandes N\'umeros para Procesos de Renovaci\'on]
Si $N\left(t\right)$ es un proceso de renovaci\'on cuyos tiempos de inter-renovaci\'on tienen media $\mu\leq\infty$, entonces
\begin{eqnarray}
t^{-1}N\left(t\right)\rightarrow 1/\mu,\textrm{ c.s. cuando }t\rightarrow\infty.
\end{eqnarray}

\end{Coro}


Considerar el proceso estoc\'astico de valores reales $\left\{Z\left(t\right):t\geq0\right\}$ en el mismo espacio de probabilidad que $N\left(t\right)$

\begin{Def}
Para el proceso $\left\{Z\left(t\right):t\geq0\right\}$ se define la fluctuaci\'on m\'axima de $Z\left(t\right)$ en el intervalo $\left(T_{n-1},T_{n}\right]$:
\begin{eqnarray*}
M_{n}=\sup_{T_{n-1}<t\leq T_{n}}|Z\left(t\right)-Z\left(T_{n-1}\right)|
\end{eqnarray*}
\end{Def}

\begin{Teo}
Sup\'ongase que $n^{-1}T_{n}\rightarrow\mu$ c.s. cuando $n\rightarrow\infty$, donde $\mu\leq\infty$ es una constante o variable aleatoria. Sea $a$ una constante o variable aleatoria que puede ser infinita cuando $\mu$ es finita, y considere las expresiones l\'imite:
\begin{eqnarray}
lim_{n\rightarrow\infty}n^{-1}Z\left(T_{n}\right)&=&a,\textrm{ c.s.}\\
lim_{t\rightarrow\infty}t^{-1}Z\left(t\right)&=&a/\mu,\textrm{ c.s.}
\end{eqnarray}
La segunda expresi\'on implica la primera. Conversamente, la primera implica la segunda si el proceso $Z\left(t\right)$ es creciente, o si $lim_{n\rightarrow\infty}n^{-1}M_{n}=0$ c.s.
\end{Teo}

\begin{Coro}
Si $N\left(t\right)$ es un proceso de renovaci\'on, y $\left(Z\left(T_{n}\right)-Z\left(T_{n-1}\right),M_{n}\right)$, para $n\geq1$, son variables aleatorias independientes e id\'enticamente distribuidas con media finita, entonces,
\begin{eqnarray}
lim_{t\rightarrow\infty}t^{-1}Z\left(t\right)\rightarrow\frac{\esp\left[Z\left(T_{1}\right)-Z\left(T_{0}\right)\right]}{\esp\left[T_{1}\right]},\textrm{ c.s. cuando  }t\rightarrow\infty.
\end{eqnarray}
\end{Coro}



%___________________________________________________________________________________________
%
\subsection{Propiedades de los Procesos de Renovaci\'on}
%___________________________________________________________________________________________
%

Los tiempos $T_{n}$ est\'an relacionados con los conteos de $N\left(t\right)$ por

\begin{eqnarray*}
\left\{N\left(t\right)\geq n\right\}&=&\left\{T_{n}\leq t\right\}\\
T_{N\left(t\right)}\leq &t&<T_{N\left(t\right)+1},
\end{eqnarray*}

adem\'as $N\left(T_{n}\right)=n$, y 

\begin{eqnarray*}
N\left(t\right)=\max\left\{n:T_{n}\leq t\right\}=\min\left\{n:T_{n+1}>t\right\}
\end{eqnarray*}

Por propiedades de la convoluci\'on se sabe que

\begin{eqnarray*}
P\left\{T_{n}\leq t\right\}=F^{n\star}\left(t\right)
\end{eqnarray*}
que es la $n$-\'esima convoluci\'on de $F$. Entonces 

\begin{eqnarray*}
\left\{N\left(t\right)\geq n\right\}&=&\left\{T_{n}\leq t\right\}\\
P\left\{N\left(t\right)\leq n\right\}&=&1-F^{\left(n+1\right)\star}\left(t\right)
\end{eqnarray*}

Adem\'as usando el hecho de que $\esp\left[N\left(t\right)\right]=\sum_{n=1}^{\infty}P\left\{N\left(t\right)\geq n\right\}$
se tiene que

\begin{eqnarray*}
\esp\left[N\left(t\right)\right]=\sum_{n=1}^{\infty}F^{n\star}\left(t\right)
\end{eqnarray*}

\begin{Prop}
Para cada $t\geq0$, la funci\'on generadora de momentos $\esp\left[e^{\alpha N\left(t\right)}\right]$ existe para alguna $\alpha$ en una vecindad del 0, y de aqu\'i que $\esp\left[N\left(t\right)^{m}\right]<\infty$, para $m\geq1$.
\end{Prop}


\begin{Note}
Si el primer tiempo de renovaci\'on $\xi_{1}$ no tiene la misma distribuci\'on que el resto de las $\xi_{n}$, para $n\geq2$, a $N\left(t\right)$ se le llama Proceso de Renovaci\'on retardado, donde si $\xi$ tiene distribuci\'on $G$, entonces el tiempo $T_{n}$ de la $n$-\'esima renovaci\'on tiene distribuci\'on $G\star F^{\left(n-1\right)\star}\left(t\right)$
\end{Note}


\begin{Teo}
Para una constante $\mu\leq\infty$ ( o variable aleatoria), las siguientes expresiones son equivalentes:

\begin{eqnarray}
lim_{n\rightarrow\infty}n^{-1}T_{n}&=&\mu,\textrm{ c.s.}\\
lim_{t\rightarrow\infty}t^{-1}N\left(t\right)&=&1/\mu,\textrm{ c.s.}
\end{eqnarray}
\end{Teo}


Es decir, $T_{n}$ satisface la Ley Fuerte de los Grandes N\'umeros s\'i y s\'olo s\'i $N\left/t\right)$ la cumple.


\begin{Coro}[Ley Fuerte de los Grandes N\'umeros para Procesos de Renovaci\'on]
Si $N\left(t\right)$ es un proceso de renovaci\'on cuyos tiempos de inter-renovaci\'on tienen media $\mu\leq\infty$, entonces
\begin{eqnarray}
t^{-1}N\left(t\right)\rightarrow 1/\mu,\textrm{ c.s. cuando }t\rightarrow\infty.
\end{eqnarray}

\end{Coro}


Considerar el proceso estoc\'astico de valores reales $\left\{Z\left(t\right):t\geq0\right\}$ en el mismo espacio de probabilidad que $N\left(t\right)$

\begin{Def}
Para el proceso $\left\{Z\left(t\right):t\geq0\right\}$ se define la fluctuaci\'on m\'axima de $Z\left(t\right)$ en el intervalo $\left(T_{n-1},T_{n}\right]$:
\begin{eqnarray*}
M_{n}=\sup_{T_{n-1}<t\leq T_{n}}|Z\left(t\right)-Z\left(T_{n-1}\right)|
\end{eqnarray*}
\end{Def}

\begin{Teo}
Sup\'ongase que $n^{-1}T_{n}\rightarrow\mu$ c.s. cuando $n\rightarrow\infty$, donde $\mu\leq\infty$ es una constante o variable aleatoria. Sea $a$ una constante o variable aleatoria que puede ser infinita cuando $\mu$ es finita, y considere las expresiones l\'imite:
\begin{eqnarray}
lim_{n\rightarrow\infty}n^{-1}Z\left(T_{n}\right)&=&a,\textrm{ c.s.}\\
lim_{t\rightarrow\infty}t^{-1}Z\left(t\right)&=&a/\mu,\textrm{ c.s.}
\end{eqnarray}
La segunda expresi\'on implica la primera. Conversamente, la primera implica la segunda si el proceso $Z\left(t\right)$ es creciente, o si $lim_{n\rightarrow\infty}n^{-1}M_{n}=0$ c.s.
\end{Teo}

\begin{Coro}
Si $N\left(t\right)$ es un proceso de renovaci\'on, y $\left(Z\left(T_{n}\right)-Z\left(T_{n-1}\right),M_{n}\right)$, para $n\geq1$, son variables aleatorias independientes e id\'enticamente distribuidas con media finita, entonces,
\begin{eqnarray}
lim_{t\rightarrow\infty}t^{-1}Z\left(t\right)\rightarrow\frac{\esp\left[Z\left(T_{1}\right)-Z\left(T_{0}\right)\right]}{\esp\left[T_{1}\right]},\textrm{ c.s. cuando  }t\rightarrow\infty.
\end{eqnarray}
\end{Coro}


%___________________________________________________________________________________________
%
\subsection{Propiedades de los Procesos de Renovaci\'on}
%___________________________________________________________________________________________
%

Los tiempos $T_{n}$ est\'an relacionados con los conteos de $N\left(t\right)$ por

\begin{eqnarray*}
\left\{N\left(t\right)\geq n\right\}&=&\left\{T_{n}\leq t\right\}\\
T_{N\left(t\right)}\leq &t&<T_{N\left(t\right)+1},
\end{eqnarray*}

adem\'as $N\left(T_{n}\right)=n$, y 

\begin{eqnarray*}
N\left(t\right)=\max\left\{n:T_{n}\leq t\right\}=\min\left\{n:T_{n+1}>t\right\}
\end{eqnarray*}

Por propiedades de la convoluci\'on se sabe que

\begin{eqnarray*}
P\left\{T_{n}\leq t\right\}=F^{n\star}\left(t\right)
\end{eqnarray*}
que es la $n$-\'esima convoluci\'on de $F$. Entonces 

\begin{eqnarray*}
\left\{N\left(t\right)\geq n\right\}&=&\left\{T_{n}\leq t\right\}\\
P\left\{N\left(t\right)\leq n\right\}&=&1-F^{\left(n+1\right)\star}\left(t\right)
\end{eqnarray*}

Adem\'as usando el hecho de que $\esp\left[N\left(t\right)\right]=\sum_{n=1}^{\infty}P\left\{N\left(t\right)\geq n\right\}$
se tiene que

\begin{eqnarray*}
\esp\left[N\left(t\right)\right]=\sum_{n=1}^{\infty}F^{n\star}\left(t\right)
\end{eqnarray*}

\begin{Prop}
Para cada $t\geq0$, la funci\'on generadora de momentos $\esp\left[e^{\alpha N\left(t\right)}\right]$ existe para alguna $\alpha$ en una vecindad del 0, y de aqu\'i que $\esp\left[N\left(t\right)^{m}\right]<\infty$, para $m\geq1$.
\end{Prop}


\begin{Note}
Si el primer tiempo de renovaci\'on $\xi_{1}$ no tiene la misma distribuci\'on que el resto de las $\xi_{n}$, para $n\geq2$, a $N\left(t\right)$ se le llama Proceso de Renovaci\'on retardado, donde si $\xi$ tiene distribuci\'on $G$, entonces el tiempo $T_{n}$ de la $n$-\'esima renovaci\'on tiene distribuci\'on $G\star F^{\left(n-1\right)\star}\left(t\right)$
\end{Note}


\begin{Teo}
Para una constante $\mu\leq\infty$ ( o variable aleatoria), las siguientes expresiones son equivalentes:

\begin{eqnarray}
lim_{n\rightarrow\infty}n^{-1}T_{n}&=&\mu,\textrm{ c.s.}\\
lim_{t\rightarrow\infty}t^{-1}N\left(t\right)&=&1/\mu,\textrm{ c.s.}
\end{eqnarray}
\end{Teo}


Es decir, $T_{n}$ satisface la Ley Fuerte de los Grandes N\'umeros s\'i y s\'olo s\'i $N\left/t\right)$ la cumple.


\begin{Coro}[Ley Fuerte de los Grandes N\'umeros para Procesos de Renovaci\'on]
Si $N\left(t\right)$ es un proceso de renovaci\'on cuyos tiempos de inter-renovaci\'on tienen media $\mu\leq\infty$, entonces
\begin{eqnarray}
t^{-1}N\left(t\right)\rightarrow 1/\mu,\textrm{ c.s. cuando }t\rightarrow\infty.
\end{eqnarray}

\end{Coro}


Considerar el proceso estoc\'astico de valores reales $\left\{Z\left(t\right):t\geq0\right\}$ en el mismo espacio de probabilidad que $N\left(t\right)$

\begin{Def}
Para el proceso $\left\{Z\left(t\right):t\geq0\right\}$ se define la fluctuaci\'on m\'axima de $Z\left(t\right)$ en el intervalo $\left(T_{n-1},T_{n}\right]$:
\begin{eqnarray*}
M_{n}=\sup_{T_{n-1}<t\leq T_{n}}|Z\left(t\right)-Z\left(T_{n-1}\right)|
\end{eqnarray*}
\end{Def}

\begin{Teo}
Sup\'ongase que $n^{-1}T_{n}\rightarrow\mu$ c.s. cuando $n\rightarrow\infty$, donde $\mu\leq\infty$ es una constante o variable aleatoria. Sea $a$ una constante o variable aleatoria que puede ser infinita cuando $\mu$ es finita, y considere las expresiones l\'imite:
\begin{eqnarray}
lim_{n\rightarrow\infty}n^{-1}Z\left(T_{n}\right)&=&a,\textrm{ c.s.}\\
lim_{t\rightarrow\infty}t^{-1}Z\left(t\right)&=&a/\mu,\textrm{ c.s.}
\end{eqnarray}
La segunda expresi\'on implica la primera. Conversamente, la primera implica la segunda si el proceso $Z\left(t\right)$ es creciente, o si $lim_{n\rightarrow\infty}n^{-1}M_{n}=0$ c.s.
\end{Teo}

\begin{Coro}
Si $N\left(t\right)$ es un proceso de renovaci\'on, y $\left(Z\left(T_{n}\right)-Z\left(T_{n-1}\right),M_{n}\right)$, para $n\geq1$, son variables aleatorias independientes e id\'enticamente distribuidas con media finita, entonces,
\begin{eqnarray}
lim_{t\rightarrow\infty}t^{-1}Z\left(t\right)\rightarrow\frac{\esp\left[Z\left(T_{1}\right)-Z\left(T_{0}\right)\right]}{\esp\left[T_{1}\right]},\textrm{ c.s. cuando  }t\rightarrow\infty.
\end{eqnarray}
\end{Coro}

%___________________________________________________________________________________________
%
\subsection{Propiedades de los Procesos de Renovaci\'on}
%___________________________________________________________________________________________
%

Los tiempos $T_{n}$ est\'an relacionados con los conteos de $N\left(t\right)$ por

\begin{eqnarray*}
\left\{N\left(t\right)\geq n\right\}&=&\left\{T_{n}\leq t\right\}\\
T_{N\left(t\right)}\leq &t&<T_{N\left(t\right)+1},
\end{eqnarray*}

adem\'as $N\left(T_{n}\right)=n$, y 

\begin{eqnarray*}
N\left(t\right)=\max\left\{n:T_{n}\leq t\right\}=\min\left\{n:T_{n+1}>t\right\}
\end{eqnarray*}

Por propiedades de la convoluci\'on se sabe que

\begin{eqnarray*}
P\left\{T_{n}\leq t\right\}=F^{n\star}\left(t\right)
\end{eqnarray*}
que es la $n$-\'esima convoluci\'on de $F$. Entonces 

\begin{eqnarray*}
\left\{N\left(t\right)\geq n\right\}&=&\left\{T_{n}\leq t\right\}\\
P\left\{N\left(t\right)\leq n\right\}&=&1-F^{\left(n+1\right)\star}\left(t\right)
\end{eqnarray*}

Adem\'as usando el hecho de que $\esp\left[N\left(t\right)\right]=\sum_{n=1}^{\infty}P\left\{N\left(t\right)\geq n\right\}$
se tiene que

\begin{eqnarray*}
\esp\left[N\left(t\right)\right]=\sum_{n=1}^{\infty}F^{n\star}\left(t\right)
\end{eqnarray*}

\begin{Prop}
Para cada $t\geq0$, la funci\'on generadora de momentos $\esp\left[e^{\alpha N\left(t\right)}\right]$ existe para alguna $\alpha$ en una vecindad del 0, y de aqu\'i que $\esp\left[N\left(t\right)^{m}\right]<\infty$, para $m\geq1$.
\end{Prop}


\begin{Note}
Si el primer tiempo de renovaci\'on $\xi_{1}$ no tiene la misma distribuci\'on que el resto de las $\xi_{n}$, para $n\geq2$, a $N\left(t\right)$ se le llama Proceso de Renovaci\'on retardado, donde si $\xi$ tiene distribuci\'on $G$, entonces el tiempo $T_{n}$ de la $n$-\'esima renovaci\'on tiene distribuci\'on $G\star F^{\left(n-1\right)\star}\left(t\right)$
\end{Note}


\begin{Teo}
Para una constante $\mu\leq\infty$ ( o variable aleatoria), las siguientes expresiones son equivalentes:

\begin{eqnarray}
lim_{n\rightarrow\infty}n^{-1}T_{n}&=&\mu,\textrm{ c.s.}\\
lim_{t\rightarrow\infty}t^{-1}N\left(t\right)&=&1/\mu,\textrm{ c.s.}
\end{eqnarray}
\end{Teo}


Es decir, $T_{n}$ satisface la Ley Fuerte de los Grandes N\'umeros s\'i y s\'olo s\'i $N\left/t\right)$ la cumple.


\begin{Coro}[Ley Fuerte de los Grandes N\'umeros para Procesos de Renovaci\'on]
Si $N\left(t\right)$ es un proceso de renovaci\'on cuyos tiempos de inter-renovaci\'on tienen media $\mu\leq\infty$, entonces
\begin{eqnarray}
t^{-1}N\left(t\right)\rightarrow 1/\mu,\textrm{ c.s. cuando }t\rightarrow\infty.
\end{eqnarray}

\end{Coro}


Considerar el proceso estoc\'astico de valores reales $\left\{Z\left(t\right):t\geq0\right\}$ en el mismo espacio de probabilidad que $N\left(t\right)$

\begin{Def}
Para el proceso $\left\{Z\left(t\right):t\geq0\right\}$ se define la fluctuaci\'on m\'axima de $Z\left(t\right)$ en el intervalo $\left(T_{n-1},T_{n}\right]$:
\begin{eqnarray*}
M_{n}=\sup_{T_{n-1}<t\leq T_{n}}|Z\left(t\right)-Z\left(T_{n-1}\right)|
\end{eqnarray*}
\end{Def}

\begin{Teo}
Sup\'ongase que $n^{-1}T_{n}\rightarrow\mu$ c.s. cuando $n\rightarrow\infty$, donde $\mu\leq\infty$ es una constante o variable aleatoria. Sea $a$ una constante o variable aleatoria que puede ser infinita cuando $\mu$ es finita, y considere las expresiones l\'imite:
\begin{eqnarray}
lim_{n\rightarrow\infty}n^{-1}Z\left(T_{n}\right)&=&a,\textrm{ c.s.}\\
lim_{t\rightarrow\infty}t^{-1}Z\left(t\right)&=&a/\mu,\textrm{ c.s.}
\end{eqnarray}
La segunda expresi\'on implica la primera. Conversamente, la primera implica la segunda si el proceso $Z\left(t\right)$ es creciente, o si $lim_{n\rightarrow\infty}n^{-1}M_{n}=0$ c.s.
\end{Teo}

\begin{Coro}
Si $N\left(t\right)$ es un proceso de renovaci\'on, y $\left(Z\left(T_{n}\right)-Z\left(T_{n-1}\right),M_{n}\right)$, para $n\geq1$, son variables aleatorias independientes e id\'enticamente distribuidas con media finita, entonces,
\begin{eqnarray}
lim_{t\rightarrow\infty}t^{-1}Z\left(t\right)\rightarrow\frac{\esp\left[Z\left(T_{1}\right)-Z\left(T_{0}\right)\right]}{\esp\left[T_{1}\right]},\textrm{ c.s. cuando  }t\rightarrow\infty.
\end{eqnarray}
\end{Coro}
%___________________________________________________________________________________________
%
\subsection{Propiedades de los Procesos de Renovaci\'on}
%___________________________________________________________________________________________
%

Los tiempos $T_{n}$ est\'an relacionados con los conteos de $N\left(t\right)$ por

\begin{eqnarray*}
\left\{N\left(t\right)\geq n\right\}&=&\left\{T_{n}\leq t\right\}\\
T_{N\left(t\right)}\leq &t&<T_{N\left(t\right)+1},
\end{eqnarray*}

adem\'as $N\left(T_{n}\right)=n$, y 

\begin{eqnarray*}
N\left(t\right)=\max\left\{n:T_{n}\leq t\right\}=\min\left\{n:T_{n+1}>t\right\}
\end{eqnarray*}

Por propiedades de la convoluci\'on se sabe que

\begin{eqnarray*}
P\left\{T_{n}\leq t\right\}=F^{n\star}\left(t\right)
\end{eqnarray*}
que es la $n$-\'esima convoluci\'on de $F$. Entonces 

\begin{eqnarray*}
\left\{N\left(t\right)\geq n\right\}&=&\left\{T_{n}\leq t\right\}\\
P\left\{N\left(t\right)\leq n\right\}&=&1-F^{\left(n+1\right)\star}\left(t\right)
\end{eqnarray*}

Adem\'as usando el hecho de que $\esp\left[N\left(t\right)\right]=\sum_{n=1}^{\infty}P\left\{N\left(t\right)\geq n\right\}$
se tiene que

\begin{eqnarray*}
\esp\left[N\left(t\right)\right]=\sum_{n=1}^{\infty}F^{n\star}\left(t\right)
\end{eqnarray*}

\begin{Prop}
Para cada $t\geq0$, la funci\'on generadora de momentos $\esp\left[e^{\alpha N\left(t\right)}\right]$ existe para alguna $\alpha$ en una vecindad del 0, y de aqu\'i que $\esp\left[N\left(t\right)^{m}\right]<\infty$, para $m\geq1$.
\end{Prop}


\begin{Note}
Si el primer tiempo de renovaci\'on $\xi_{1}$ no tiene la misma distribuci\'on que el resto de las $\xi_{n}$, para $n\geq2$, a $N\left(t\right)$ se le llama Proceso de Renovaci\'on retardado, donde si $\xi$ tiene distribuci\'on $G$, entonces el tiempo $T_{n}$ de la $n$-\'esima renovaci\'on tiene distribuci\'on $G\star F^{\left(n-1\right)\star}\left(t\right)$
\end{Note}


\begin{Teo}
Para una constante $\mu\leq\infty$ ( o variable aleatoria), las siguientes expresiones son equivalentes:

\begin{eqnarray}
lim_{n\rightarrow\infty}n^{-1}T_{n}&=&\mu,\textrm{ c.s.}\\
lim_{t\rightarrow\infty}t^{-1}N\left(t\right)&=&1/\mu,\textrm{ c.s.}
\end{eqnarray}
\end{Teo}


Es decir, $T_{n}$ satisface la Ley Fuerte de los Grandes N\'umeros s\'i y s\'olo s\'i $N\left/t\right)$ la cumple.


\begin{Coro}[Ley Fuerte de los Grandes N\'umeros para Procesos de Renovaci\'on]
Si $N\left(t\right)$ es un proceso de renovaci\'on cuyos tiempos de inter-renovaci\'on tienen media $\mu\leq\infty$, entonces
\begin{eqnarray}
t^{-1}N\left(t\right)\rightarrow 1/\mu,\textrm{ c.s. cuando }t\rightarrow\infty.
\end{eqnarray}

\end{Coro}


Considerar el proceso estoc\'astico de valores reales $\left\{Z\left(t\right):t\geq0\right\}$ en el mismo espacio de probabilidad que $N\left(t\right)$

\begin{Def}
Para el proceso $\left\{Z\left(t\right):t\geq0\right\}$ se define la fluctuaci\'on m\'axima de $Z\left(t\right)$ en el intervalo $\left(T_{n-1},T_{n}\right]$:
\begin{eqnarray*}
M_{n}=\sup_{T_{n-1}<t\leq T_{n}}|Z\left(t\right)-Z\left(T_{n-1}\right)|
\end{eqnarray*}
\end{Def}

\begin{Teo}
Sup\'ongase que $n^{-1}T_{n}\rightarrow\mu$ c.s. cuando $n\rightarrow\infty$, donde $\mu\leq\infty$ es una constante o variable aleatoria. Sea $a$ una constante o variable aleatoria que puede ser infinita cuando $\mu$ es finita, y considere las expresiones l\'imite:
\begin{eqnarray}
lim_{n\rightarrow\infty}n^{-1}Z\left(T_{n}\right)&=&a,\textrm{ c.s.}\\
lim_{t\rightarrow\infty}t^{-1}Z\left(t\right)&=&a/\mu,\textrm{ c.s.}
\end{eqnarray}
La segunda expresi\'on implica la primera. Conversamente, la primera implica la segunda si el proceso $Z\left(t\right)$ es creciente, o si $lim_{n\rightarrow\infty}n^{-1}M_{n}=0$ c.s.
\end{Teo}

\begin{Coro}
Si $N\left(t\right)$ es un proceso de renovaci\'on, y $\left(Z\left(T_{n}\right)-Z\left(T_{n-1}\right),M_{n}\right)$, para $n\geq1$, son variables aleatorias independientes e id\'enticamente distribuidas con media finita, entonces,
\begin{eqnarray}
lim_{t\rightarrow\infty}t^{-1}Z\left(t\right)\rightarrow\frac{\esp\left[Z\left(T_{1}\right)-Z\left(T_{0}\right)\right]}{\esp\left[T_{1}\right]},\textrm{ c.s. cuando  }t\rightarrow\infty.
\end{eqnarray}
\end{Coro}


%___________________________________________________________________________________________
%
\subsection{Funci\'on de Renovaci\'on}
%___________________________________________________________________________________________
%


\begin{Def}
Sea $h\left(t\right)$ funci\'on de valores reales en $\rea$ acotada en intervalos finitos e igual a cero para $t<0$ La ecuaci\'on de renovaci\'on para $h\left(t\right)$ y la distribuci\'on $F$ es

\begin{eqnarray}%\label{Ec.Renovacion}
H\left(t\right)=h\left(t\right)+\int_{\left[0,t\right]}H\left(t-s\right)dF\left(s\right)\textrm{,    }t\geq0,
\end{eqnarray}
donde $H\left(t\right)$ es una funci\'on de valores reales. Esto es $H=h+F\star H$. Decimos que $H\left(t\right)$ es soluci\'on de esta ecuaci\'on si satisface la ecuaci\'on, y es acotada en intervalos finitos e iguales a cero para $t<0$.
\end{Def}

\begin{Prop}
La funci\'on $U\star h\left(t\right)$ es la \'unica soluci\'on de la ecuaci\'on de renovaci\'on (\ref{Ec.Renovacion}).
\end{Prop}

\begin{Teo}[Teorema Renovaci\'on Elemental]
\begin{eqnarray*}
t^{-1}U\left(t\right)\rightarrow 1/\mu\textrm{,    cuando }t\rightarrow\infty.
\end{eqnarray*}
\end{Teo}

%___________________________________________________________________________________________
%
\subsection{Funci\'on de Renovaci\'on}
%___________________________________________________________________________________________
%


Sup\'ongase que $N\left(t\right)$ es un proceso de renovaci\'on con distribuci\'on $F$ con media finita $\mu$.

\begin{Def}
La funci\'on de renovaci\'on asociada con la distribuci\'on $F$, del proceso $N\left(t\right)$, es
\begin{eqnarray*}
U\left(t\right)=\sum_{n=1}^{\infty}F^{n\star}\left(t\right),\textrm{   }t\geq0,
\end{eqnarray*}
donde $F^{0\star}\left(t\right)=\indora\left(t\geq0\right)$.
\end{Def}


\begin{Prop}
Sup\'ongase que la distribuci\'on de inter-renovaci\'on $F$ tiene densidad $f$. Entonces $U\left(t\right)$ tambi\'en tiene densidad, para $t>0$, y es $U^{'}\left(t\right)=\sum_{n=0}^{\infty}f^{n\star}\left(t\right)$. Adem\'as
\begin{eqnarray*}
\prob\left\{N\left(t\right)>N\left(t-\right)\right\}=0\textrm{,   }t\geq0.
\end{eqnarray*}
\end{Prop}

\begin{Def}
La Transformada de Laplace-Stieljes de $F$ est\'a dada por

\begin{eqnarray*}
\hat{F}\left(\alpha\right)=\int_{\rea_{+}}e^{-\alpha t}dF\left(t\right)\textrm{,  }\alpha\geq0.
\end{eqnarray*}
\end{Def}

Entonces

\begin{eqnarray*}
\hat{U}\left(\alpha\right)=\sum_{n=0}^{\infty}\hat{F^{n\star}}\left(\alpha\right)=\sum_{n=0}^{\infty}\hat{F}\left(\alpha\right)^{n}=\frac{1}{1-\hat{F}\left(\alpha\right)}.
\end{eqnarray*}


\begin{Prop}
La Transformada de Laplace $\hat{U}\left(\alpha\right)$ y $\hat{F}\left(\alpha\right)$ determina una a la otra de manera \'unica por la relaci\'on $\hat{U}\left(\alpha\right)=\frac{1}{1-\hat{F}\left(\alpha\right)}$.
\end{Prop}


\begin{Note}
Un proceso de renovaci\'on $N\left(t\right)$ cuyos tiempos de inter-renovaci\'on tienen media finita, es un proceso Poisson con tasa $\lambda$ si y s\'olo s\'i $\esp\left[U\left(t\right)\right]=\lambda t$, para $t\geq0$.
\end{Note}


\begin{Teo}
Sea $N\left(t\right)$ un proceso puntual simple con puntos de localizaci\'on $T_{n}$ tal que $\eta\left(t\right)=\esp\left[N\left(\right)\right]$ es finita para cada $t$. Entonces para cualquier funci\'on $f:\rea_{+}\rightarrow\rea$,
\begin{eqnarray*}
\esp\left[\sum_{n=1}^{N\left(\right)}f\left(T_{n}\right)\right]=\int_{\left(0,t\right]}f\left(s\right)d\eta\left(s\right)\textrm{,  }t\geq0,
\end{eqnarray*}
suponiendo que la integral exista. Adem\'as si $X_{1},X_{2},\ldots$ son variables aleatorias definidas en el mismo espacio de probabilidad que el proceso $N\left(t\right)$ tal que $\esp\left[X_{n}|T_{n}=s\right]=f\left(s\right)$, independiente de $n$. Entonces
\begin{eqnarray*}
\esp\left[\sum_{n=1}^{N\left(t\right)}X_{n}\right]=\int_{\left(0,t\right]}f\left(s\right)d\eta\left(s\right)\textrm{,  }t\geq0,
\end{eqnarray*} 
suponiendo que la integral exista. 
\end{Teo}

\begin{Coro}[Identidad de Wald para Renovaciones]
Para el proceso de renovaci\'on $N\left(t\right)$,
\begin{eqnarray*}
\esp\left[T_{N\left(t\right)+1}\right]=\mu\esp\left[N\left(t\right)+1\right]\textrm{,  }t\geq0,
\end{eqnarray*}  
\end{Coro}

%______________________________________________________________________
\subsection{Procesos de Renovaci\'on}
%______________________________________________________________________

\begin{Def}%\label{Def.Tn}
Sean $0\leq T_{1}\leq T_{2}\leq \ldots$ son tiempos aleatorios infinitos en los cuales ocurren ciertos eventos. El n\'umero de tiempos $T_{n}$ en el intervalo $\left[0,t\right)$ es

\begin{eqnarray}
N\left(t\right)=\sum_{n=1}^{\infty}\indora\left(T_{n}\leq t\right),
\end{eqnarray}
para $t\geq0$.
\end{Def}

Si se consideran los puntos $T_{n}$ como elementos de $\rea_{+}$, y $N\left(t\right)$ es el n\'umero de puntos en $\rea$. El proceso denotado por $\left\{N\left(t\right):t\geq0\right\}$, denotado por $N\left(t\right)$, es un proceso puntual en $\rea_{+}$. Los $T_{n}$ son los tiempos de ocurrencia, el proceso puntual $N\left(t\right)$ es simple si su n\'umero de ocurrencias son distintas: $0<T_{1}<T_{2}<\ldots$ casi seguramente.

\begin{Def}
Un proceso puntual $N\left(t\right)$ es un proceso de renovaci\'on si los tiempos de interocurrencia $\xi_{n}=T_{n}-T_{n-1}$, para $n\geq1$, son independientes e identicamente distribuidos con distribuci\'on $F$, donde $F\left(0\right)=0$ y $T_{0}=0$. Los $T_{n}$ son llamados tiempos de renovaci\'on, referente a la independencia o renovaci\'on de la informaci\'on estoc\'astica en estos tiempos. Los $\xi_{n}$ son los tiempos de inter-renovaci\'on, y $N\left(t\right)$ es el n\'umero de renovaciones en el intervalo $\left[0,t\right)$
\end{Def}


\begin{Note}
Para definir un proceso de renovaci\'on para cualquier contexto, solamente hay que especificar una distribuci\'on $F$, con $F\left(0\right)=0$, para los tiempos de inter-renovaci\'on. La funci\'on $F$ en turno degune las otra variables aleatorias. De manera formal, existe un espacio de probabilidad y una sucesi\'on de variables aleatorias $\xi_{1},\xi_{2},\ldots$ definidas en este con distribuci\'on $F$. Entonces las otras cantidades son $T_{n}=\sum_{k=1}^{n}\xi_{k}$ y $N\left(t\right)=\sum_{n=1}^{\infty}\indora\left(T_{n}\leq t\right)$, donde $T_{n}\rightarrow\infty$ casi seguramente por la Ley Fuerte de los Grandes Números.
\end{Note}

%___________________________________________________________________________________________
%
\subsection{Renewal and Regenerative Processes: Serfozo\cite{Serfozo}}
%___________________________________________________________________________________________
%
\begin{Def}%\label{Def.Tn}
Sean $0\leq T_{1}\leq T_{2}\leq \ldots$ son tiempos aleatorios infinitos en los cuales ocurren ciertos eventos. El n\'umero de tiempos $T_{n}$ en el intervalo $\left[0,t\right)$ es

\begin{eqnarray}
N\left(t\right)=\sum_{n=1}^{\infty}\indora\left(T_{n}\leq t\right),
\end{eqnarray}
para $t\geq0$.
\end{Def}

Si se consideran los puntos $T_{n}$ como elementos de $\rea_{+}$, y $N\left(t\right)$ es el n\'umero de puntos en $\rea$. El proceso denotado por $\left\{N\left(t\right):t\geq0\right\}$, denotado por $N\left(t\right)$, es un proceso puntual en $\rea_{+}$. Los $T_{n}$ son los tiempos de ocurrencia, el proceso puntual $N\left(t\right)$ es simple si su n\'umero de ocurrencias son distintas: $0<T_{1}<T_{2}<\ldots$ casi seguramente.

\begin{Def}
Un proceso puntual $N\left(t\right)$ es un proceso de renovaci\'on si los tiempos de interocurrencia $\xi_{n}=T_{n}-T_{n-1}$, para $n\geq1$, son independientes e identicamente distribuidos con distribuci\'on $F$, donde $F\left(0\right)=0$ y $T_{0}=0$. Los $T_{n}$ son llamados tiempos de renovaci\'on, referente a la independencia o renovaci\'on de la informaci\'on estoc\'astica en estos tiempos. Los $\xi_{n}$ son los tiempos de inter-renovaci\'on, y $N\left(t\right)$ es el n\'umero de renovaciones en el intervalo $\left[0,t\right)$
\end{Def}


\begin{Note}
Para definir un proceso de renovaci\'on para cualquier contexto, solamente hay que especificar una distribuci\'on $F$, con $F\left(0\right)=0$, para los tiempos de inter-renovaci\'on. La funci\'on $F$ en turno degune las otra variables aleatorias. De manera formal, existe un espacio de probabilidad y una sucesi\'on de variables aleatorias $\xi_{1},\xi_{2},\ldots$ definidas en este con distribuci\'on $F$. Entonces las otras cantidades son $T_{n}=\sum_{k=1}^{n}\xi_{k}$ y $N\left(t\right)=\sum_{n=1}^{\infty}\indora\left(T_{n}\leq t\right)$, donde $T_{n}\rightarrow\infty$ casi seguramente por la Ley Fuerte de los Grandes N\'umeros.
\end{Note}

Los tiempos $T_{n}$ est\'an relacionados con los conteos de $N\left(t\right)$ por

\begin{eqnarray*}
\left\{N\left(t\right)\geq n\right\}&=&\left\{T_{n}\leq t\right\}\\
T_{N\left(t\right)}\leq &t&<T_{N\left(t\right)+1},
\end{eqnarray*}

adem\'as $N\left(T_{n}\right)=n$, y 

\begin{eqnarray*}
N\left(t\right)=\max\left\{n:T_{n}\leq t\right\}=\min\left\{n:T_{n+1}>t\right\}
\end{eqnarray*}

Por propiedades de la convoluci\'on se sabe que

\begin{eqnarray*}
P\left\{T_{n}\leq t\right\}=F^{n\star}\left(t\right)
\end{eqnarray*}
que es la $n$-\'esima convoluci\'on de $F$. Entonces 

\begin{eqnarray*}
\left\{N\left(t\right)\geq n\right\}&=&\left\{T_{n}\leq t\right\}\\
P\left\{N\left(t\right)\leq n\right\}&=&1-F^{\left(n+1\right)\star}\left(t\right)
\end{eqnarray*}

Adem\'as usando el hecho de que $\esp\left[N\left(t\right)\right]=\sum_{n=1}^{\infty}P\left\{N\left(t\right)\geq n\right\}$
se tiene que

\begin{eqnarray*}
\esp\left[N\left(t\right)\right]=\sum_{n=1}^{\infty}F^{n\star}\left(t\right)
\end{eqnarray*}

\begin{Prop}
Para cada $t\geq0$, la funci\'on generadora de momentos $\esp\left[e^{\alpha N\left(t\right)}\right]$ existe para alguna $\alpha$ en una vecindad del 0, y de aqu\'i que $\esp\left[N\left(t\right)^{m}\right]<\infty$, para $m\geq1$.
\end{Prop}

\begin{Ejem}[\textbf{Proceso Poisson}]

Suponga que se tienen tiempos de inter-renovaci\'on \textit{i.i.d.} del proceso de renovaci\'on $N\left(t\right)$ tienen distribuci\'on exponencial $F\left(t\right)=q-e^{-\lambda t}$ con tasa $\lambda$. Entonces $N\left(t\right)$ es un proceso Poisson con tasa $\lambda$.

\end{Ejem}


\begin{Note}
Si el primer tiempo de renovaci\'on $\xi_{1}$ no tiene la misma distribuci\'on que el resto de las $\xi_{n}$, para $n\geq2$, a $N\left(t\right)$ se le llama Proceso de Renovaci\'on retardado, donde si $\xi$ tiene distribuci\'on $G$, entonces el tiempo $T_{n}$ de la $n$-\'esima renovaci\'on tiene distribuci\'on $G\star F^{\left(n-1\right)\star}\left(t\right)$
\end{Note}


\begin{Teo}
Para una constante $\mu\leq\infty$ ( o variable aleatoria), las siguientes expresiones son equivalentes:

\begin{eqnarray}
lim_{n\rightarrow\infty}n^{-1}T_{n}&=&\mu,\textrm{ c.s.}\\
lim_{t\rightarrow\infty}t^{-1}N\left(t\right)&=&1/\mu,\textrm{ c.s.}
\end{eqnarray}
\end{Teo}


Es decir, $T_{n}$ satisface la Ley Fuerte de los Grandes N\'umeros s\'i y s\'olo s\'i $N\left/t\right)$ la cumple.


\begin{Coro}[Ley Fuerte de los Grandes N\'umeros para Procesos de Renovaci\'on]
Si $N\left(t\right)$ es un proceso de renovaci\'on cuyos tiempos de inter-renovaci\'on tienen media $\mu\leq\infty$, entonces
\begin{eqnarray}
t^{-1}N\left(t\right)\rightarrow 1/\mu,\textrm{ c.s. cuando }t\rightarrow\infty.
\end{eqnarray}

\end{Coro}


Considerar el proceso estoc\'astico de valores reales $\left\{Z\left(t\right):t\geq0\right\}$ en el mismo espacio de probabilidad que $N\left(t\right)$

\begin{Def}
Para el proceso $\left\{Z\left(t\right):t\geq0\right\}$ se define la fluctuaci\'on m\'axima de $Z\left(t\right)$ en el intervalo $\left(T_{n-1},T_{n}\right]$:
\begin{eqnarray*}
M_{n}=\sup_{T_{n-1}<t\leq T_{n}}|Z\left(t\right)-Z\left(T_{n-1}\right)|
\end{eqnarray*}
\end{Def}

\begin{Teo}
Sup\'ongase que $n^{-1}T_{n}\rightarrow\mu$ c.s. cuando $n\rightarrow\infty$, donde $\mu\leq\infty$ es una constante o variable aleatoria. Sea $a$ una constante o variable aleatoria que puede ser infinita cuando $\mu$ es finita, y considere las expresiones l\'imite:
\begin{eqnarray}
lim_{n\rightarrow\infty}n^{-1}Z\left(T_{n}\right)&=&a,\textrm{ c.s.}\\
lim_{t\rightarrow\infty}t^{-1}Z\left(t\right)&=&a/\mu,\textrm{ c.s.}
\end{eqnarray}
La segunda expresi\'on implica la primera. Conversamente, la primera implica la segunda si el proceso $Z\left(t\right)$ es creciente, o si $lim_{n\rightarrow\infty}n^{-1}M_{n}=0$ c.s.
\end{Teo}

\begin{Coro}
Si $N\left(t\right)$ es un proceso de renovaci\'on, y $\left(Z\left(T_{n}\right)-Z\left(T_{n-1}\right),M_{n}\right)$, para $n\geq1$, son variables aleatorias independientes e id\'enticamente distribuidas con media finita, entonces,
\begin{eqnarray}
lim_{t\rightarrow\infty}t^{-1}Z\left(t\right)\rightarrow\frac{\esp\left[Z\left(T_{1}\right)-Z\left(T_{0}\right)\right]}{\esp\left[T_{1}\right]},\textrm{ c.s. cuando  }t\rightarrow\infty.
\end{eqnarray}
\end{Coro}


Sup\'ongase que $N\left(t\right)$ es un proceso de renovaci\'on con distribuci\'on $F$ con media finita $\mu$.

\begin{Def}
La funci\'on de renovaci\'on asociada con la distribuci\'on $F$, del proceso $N\left(t\right)$, es
\begin{eqnarray*}
U\left(t\right)=\sum_{n=1}^{\infty}F^{n\star}\left(t\right),\textrm{   }t\geq0,
\end{eqnarray*}
donde $F^{0\star}\left(t\right)=\indora\left(t\geq0\right)$.
\end{Def}


\begin{Prop}
Sup\'ongase que la distribuci\'on de inter-renovaci\'on $F$ tiene densidad $f$. Entonces $U\left(t\right)$ tambi\'en tiene densidad, para $t>0$, y es $U^{'}\left(t\right)=\sum_{n=0}^{\infty}f^{n\star}\left(t\right)$. Adem\'as
\begin{eqnarray*}
\prob\left\{N\left(t\right)>N\left(t-\right)\right\}=0\textrm{,   }t\geq0.
\end{eqnarray*}
\end{Prop}

\begin{Def}
La Transformada de Laplace-Stieljes de $F$ est\'a dada por

\begin{eqnarray*}
\hat{F}\left(\alpha\right)=\int_{\rea_{+}}e^{-\alpha t}dF\left(t\right)\textrm{,  }\alpha\geq0.
\end{eqnarray*}
\end{Def}

Entonces

\begin{eqnarray*}
\hat{U}\left(\alpha\right)=\sum_{n=0}^{\infty}\hat{F^{n\star}}\left(\alpha\right)=\sum_{n=0}^{\infty}\hat{F}\left(\alpha\right)^{n}=\frac{1}{1-\hat{F}\left(\alpha\right)}.
\end{eqnarray*}


\begin{Prop}
La Transformada de Laplace $\hat{U}\left(\alpha\right)$ y $\hat{F}\left(\alpha\right)$ determina una a la otra de manera \'unica por la relaci\'on $\hat{U}\left(\alpha\right)=\frac{1}{1-\hat{F}\left(\alpha\right)}$.
\end{Prop}


\begin{Note}
Un proceso de renovaci\'on $N\left(t\right)$ cuyos tiempos de inter-renovaci\'on tienen media finita, es un proceso Poisson con tasa $\lambda$ si y s\'olo s\'i $\esp\left[U\left(t\right)\right]=\lambda t$, para $t\geq0$.
\end{Note}


\begin{Teo}
Sea $N\left(t\right)$ un proceso puntual simple con puntos de localizaci\'on $T_{n}$ tal que $\eta\left(t\right)=\esp\left[N\left(\right)\right]$ es finita para cada $t$. Entonces para cualquier funci\'on $f:\rea_{+}\rightarrow\rea$,
\begin{eqnarray*}
\esp\left[\sum_{n=1}^{N\left(\right)}f\left(T_{n}\right)\right]=\int_{\left(0,t\right]}f\left(s\right)d\eta\left(s\right)\textrm{,  }t\geq0,
\end{eqnarray*}
suponiendo que la integral exista. Adem\'as si $X_{1},X_{2},\ldots$ son variables aleatorias definidas en el mismo espacio de probabilidad que el proceso $N\left(t\right)$ tal que $\esp\left[X_{n}|T_{n}=s\right]=f\left(s\right)$, independiente de $n$. Entonces
\begin{eqnarray*}
\esp\left[\sum_{n=1}^{N\left(t\right)}X_{n}\right]=\int_{\left(0,t\right]}f\left(s\right)d\eta\left(s\right)\textrm{,  }t\geq0,
\end{eqnarray*} 
suponiendo que la integral exista. 
\end{Teo}

\begin{Coro}[Identidad de Wald para Renovaciones]
Para el proceso de renovaci\'on $N\left(t\right)$,
\begin{eqnarray*}
\esp\left[T_{N\left(t\right)+1}\right]=\mu\esp\left[N\left(t\right)+1\right]\textrm{,  }t\geq0,
\end{eqnarray*}  
\end{Coro}


\begin{Def}
Sea $h\left(t\right)$ funci\'on de valores reales en $\rea$ acotada en intervalos finitos e igual a cero para $t<0$ La ecuaci\'on de renovaci\'on para $h\left(t\right)$ y la distribuci\'on $F$ es

\begin{eqnarray}%\label{Ec.Renovacion}
H\left(t\right)=h\left(t\right)+\int_{\left[0,t\right]}H\left(t-s\right)dF\left(s\right)\textrm{,    }t\geq0,
\end{eqnarray}
donde $H\left(t\right)$ es una funci\'on de valores reales. Esto es $H=h+F\star H$. Decimos que $H\left(t\right)$ es soluci\'on de esta ecuaci\'on si satisface la ecuaci\'on, y es acotada en intervalos finitos e iguales a cero para $t<0$.
\end{Def}

\begin{Prop}
La funci\'on $U\star h\left(t\right)$ es la \'unica soluci\'on de la ecuaci\'on de renovaci\'on (\ref{Ec.Renovacion}).
\end{Prop}

\begin{Teo}[Teorema Renovaci\'on Elemental]
\begin{eqnarray*}
t^{-1}U\left(t\right)\rightarrow 1/\mu\textrm{,    cuando }t\rightarrow\infty.
\end{eqnarray*}
\end{Teo}



Sup\'ongase que $N\left(t\right)$ es un proceso de renovaci\'on con distribuci\'on $F$ con media finita $\mu$.

\begin{Def}
La funci\'on de renovaci\'on asociada con la distribuci\'on $F$, del proceso $N\left(t\right)$, es
\begin{eqnarray*}
U\left(t\right)=\sum_{n=1}^{\infty}F^{n\star}\left(t\right),\textrm{   }t\geq0,
\end{eqnarray*}
donde $F^{0\star}\left(t\right)=\indora\left(t\geq0\right)$.
\end{Def}


\begin{Prop}
Sup\'ongase que la distribuci\'on de inter-renovaci\'on $F$ tiene densidad $f$. Entonces $U\left(t\right)$ tambi\'en tiene densidad, para $t>0$, y es $U^{'}\left(t\right)=\sum_{n=0}^{\infty}f^{n\star}\left(t\right)$. Adem\'as
\begin{eqnarray*}
\prob\left\{N\left(t\right)>N\left(t-\right)\right\}=0\textrm{,   }t\geq0.
\end{eqnarray*}
\end{Prop}

\begin{Def}
La Transformada de Laplace-Stieljes de $F$ est\'a dada por

\begin{eqnarray*}
\hat{F}\left(\alpha\right)=\int_{\rea_{+}}e^{-\alpha t}dF\left(t\right)\textrm{,  }\alpha\geq0.
\end{eqnarray*}
\end{Def}

Entonces

\begin{eqnarray*}
\hat{U}\left(\alpha\right)=\sum_{n=0}^{\infty}\hat{F^{n\star}}\left(\alpha\right)=\sum_{n=0}^{\infty}\hat{F}\left(\alpha\right)^{n}=\frac{1}{1-\hat{F}\left(\alpha\right)}.
\end{eqnarray*}


\begin{Prop}
La Transformada de Laplace $\hat{U}\left(\alpha\right)$ y $\hat{F}\left(\alpha\right)$ determina una a la otra de manera \'unica por la relaci\'on $\hat{U}\left(\alpha\right)=\frac{1}{1-\hat{F}\left(\alpha\right)}$.
\end{Prop}


\begin{Note}
Un proceso de renovaci\'on $N\left(t\right)$ cuyos tiempos de inter-renovaci\'on tienen media finita, es un proceso Poisson con tasa $\lambda$ si y s\'olo s\'i $\esp\left[U\left(t\right)\right]=\lambda t$, para $t\geq0$.
\end{Note}


\begin{Teo}
Sea $N\left(t\right)$ un proceso puntual simple con puntos de localizaci\'on $T_{n}$ tal que $\eta\left(t\right)=\esp\left[N\left(\right)\right]$ es finita para cada $t$. Entonces para cualquier funci\'on $f:\rea_{+}\rightarrow\rea$,
\begin{eqnarray*}
\esp\left[\sum_{n=1}^{N\left(\right)}f\left(T_{n}\right)\right]=\int_{\left(0,t\right]}f\left(s\right)d\eta\left(s\right)\textrm{,  }t\geq0,
\end{eqnarray*}
suponiendo que la integral exista. Adem\'as si $X_{1},X_{2},\ldots$ son variables aleatorias definidas en el mismo espacio de probabilidad que el proceso $N\left(t\right)$ tal que $\esp\left[X_{n}|T_{n}=s\right]=f\left(s\right)$, independiente de $n$. Entonces
\begin{eqnarray*}
\esp\left[\sum_{n=1}^{N\left(t\right)}X_{n}\right]=\int_{\left(0,t\right]}f\left(s\right)d\eta\left(s\right)\textrm{,  }t\geq0,
\end{eqnarray*} 
suponiendo que la integral exista. 
\end{Teo}

\begin{Coro}[Identidad de Wald para Renovaciones]
Para el proceso de renovaci\'on $N\left(t\right)$,
\begin{eqnarray*}
\esp\left[T_{N\left(t\right)+1}\right]=\mu\esp\left[N\left(t\right)+1\right]\textrm{,  }t\geq0,
\end{eqnarray*}  
\end{Coro}


\begin{Def}
Sea $h\left(t\right)$ funci\'on de valores reales en $\rea$ acotada en intervalos finitos e igual a cero para $t<0$ La ecuaci\'on de renovaci\'on para $h\left(t\right)$ y la distribuci\'on $F$ es

\begin{eqnarray}%\label{Ec.Renovacion}
H\left(t\right)=h\left(t\right)+\int_{\left[0,t\right]}H\left(t-s\right)dF\left(s\right)\textrm{,    }t\geq0,
\end{eqnarray}
donde $H\left(t\right)$ es una funci\'on de valores reales. Esto es $H=h+F\star H$. Decimos que $H\left(t\right)$ es soluci\'on de esta ecuaci\'on si satisface la ecuaci\'on, y es acotada en intervalos finitos e iguales a cero para $t<0$.
\end{Def}

\begin{Prop}
La funci\'on $U\star h\left(t\right)$ es la \'unica soluci\'on de la ecuaci\'on de renovaci\'on (\ref{Ec.Renovacion}).
\end{Prop}

\begin{Teo}[Teorema Renovaci\'on Elemental]
\begin{eqnarray*}
t^{-1}U\left(t\right)\rightarrow 1/\mu\textrm{,    cuando }t\rightarrow\infty.
\end{eqnarray*}
\end{Teo}


\begin{Note} Una funci\'on $h:\rea_{+}\rightarrow\rea$ es Directamente Riemann Integrable en los siguientes casos:
\begin{itemize}
\item[a)] $h\left(t\right)\geq0$ es decreciente y Riemann Integrable.
\item[b)] $h$ es continua excepto posiblemente en un conjunto de Lebesgue de medida 0, y $|h\left(t\right)|\leq b\left(t\right)$, donde $b$ es DRI.
\end{itemize}
\end{Note}

\begin{Teo}[Teorema Principal de Renovaci\'on]
Si $F$ es no aritm\'etica y $h\left(t\right)$ es Directamente Riemann Integrable (DRI), entonces

\begin{eqnarray*}
lim_{t\rightarrow\infty}U\star h=\frac{1}{\mu}\int_{\rea_{+}}h\left(s\right)ds.
\end{eqnarray*}
\end{Teo}

\begin{Prop}
Cualquier funci\'on $H\left(t\right)$ acotada en intervalos finitos y que es 0 para $t<0$ puede expresarse como
\begin{eqnarray*}
H\left(t\right)=U\star h\left(t\right)\textrm{,  donde }h\left(t\right)=H\left(t\right)-F\star H\left(t\right)
\end{eqnarray*}
\end{Prop}

\begin{Def}
Un proceso estoc\'astico $X\left(t\right)$ es crudamente regenerativo en un tiempo aleatorio positivo $T$ si
\begin{eqnarray*}
\esp\left[X\left(T+t\right)|T\right]=\esp\left[X\left(t\right)\right]\textrm{, para }t\geq0,\end{eqnarray*}
y con las esperanzas anteriores finitas.
\end{Def}

\begin{Prop}
Sup\'ongase que $X\left(t\right)$ es un proceso crudamente regenerativo en $T$, que tiene distribuci\'on $F$. Si $\esp\left[X\left(t\right)\right]$ es acotado en intervalos finitos, entonces
\begin{eqnarray*}
\esp\left[X\left(t\right)\right]=U\star h\left(t\right)\textrm{,  donde }h\left(t\right)=\esp\left[X\left(t\right)\indora\left(T>t\right)\right].
\end{eqnarray*}
\end{Prop}

\begin{Teo}[Regeneraci\'on Cruda]
Sup\'ongase que $X\left(t\right)$ es un proceso con valores positivo crudamente regenerativo en $T$, y def\'inase $M=\sup\left\{|X\left(t\right)|:t\leq T\right\}$. Si $T$ es no aritm\'etico y $M$ y $MT$ tienen media finita, entonces
\begin{eqnarray*}
lim_{t\rightarrow\infty}\esp\left[X\left(t\right)\right]=\frac{1}{\mu}\int_{\rea_{+}}h\left(s\right)ds,
\end{eqnarray*}
donde $h\left(t\right)=\esp\left[X\left(t\right)\indora\left(T>t\right)\right]$.
\end{Teo}


\begin{Note} Una funci\'on $h:\rea_{+}\rightarrow\rea$ es Directamente Riemann Integrable en los siguientes casos:
\begin{itemize}
\item[a)] $h\left(t\right)\geq0$ es decreciente y Riemann Integrable.
\item[b)] $h$ es continua excepto posiblemente en un conjunto de Lebesgue de medida 0, y $|h\left(t\right)|\leq b\left(t\right)$, donde $b$ es DRI.
\end{itemize}
\end{Note}

\begin{Teo}[Teorema Principal de Renovaci\'on]
Si $F$ es no aritm\'etica y $h\left(t\right)$ es Directamente Riemann Integrable (DRI), entonces

\begin{eqnarray*}
lim_{t\rightarrow\infty}U\star h=\frac{1}{\mu}\int_{\rea_{+}}h\left(s\right)ds.
\end{eqnarray*}
\end{Teo}

\begin{Prop}
Cualquier funci\'on $H\left(t\right)$ acotada en intervalos finitos y que es 0 para $t<0$ puede expresarse como
\begin{eqnarray*}
H\left(t\right)=U\star h\left(t\right)\textrm{,  donde }h\left(t\right)=H\left(t\right)-F\star H\left(t\right)
\end{eqnarray*}
\end{Prop}

\begin{Def}
Un proceso estoc\'astico $X\left(t\right)$ es crudamente regenerativo en un tiempo aleatorio positivo $T$ si
\begin{eqnarray*}
\esp\left[X\left(T+t\right)|T\right]=\esp\left[X\left(t\right)\right]\textrm{, para }t\geq0,\end{eqnarray*}
y con las esperanzas anteriores finitas.
\end{Def}

\begin{Prop}
Sup\'ongase que $X\left(t\right)$ es un proceso crudamente regenerativo en $T$, que tiene distribuci\'on $F$. Si $\esp\left[X\left(t\right)\right]$ es acotado en intervalos finitos, entonces
\begin{eqnarray*}
\esp\left[X\left(t\right)\right]=U\star h\left(t\right)\textrm{,  donde }h\left(t\right)=\esp\left[X\left(t\right)\indora\left(T>t\right)\right].
\end{eqnarray*}
\end{Prop}

\begin{Teo}[Regeneraci\'on Cruda]
Sup\'ongase que $X\left(t\right)$ es un proceso con valores positivo crudamente regenerativo en $T$, y def\'inase $M=\sup\left\{|X\left(t\right)|:t\leq T\right\}$. Si $T$ es no aritm\'etico y $M$ y $MT$ tienen media finita, entonces
\begin{eqnarray*}
lim_{t\rightarrow\infty}\esp\left[X\left(t\right)\right]=\frac{1}{\mu}\int_{\rea_{+}}h\left(s\right)ds,
\end{eqnarray*}
donde $h\left(t\right)=\esp\left[X\left(t\right)\indora\left(T>t\right)\right]$.
\end{Teo}

\begin{Def}
Para el proceso $\left\{\left(N\left(t\right),X\left(t\right)\right):t\geq0\right\}$, sus trayectoria muestrales en el intervalo de tiempo $\left[T_{n-1},T_{n}\right)$ est\'an descritas por
\begin{eqnarray*}
\zeta_{n}=\left(\xi_{n},\left\{X\left(T_{n-1}+t\right):0\leq t<\xi_{n}\right\}\right)
\end{eqnarray*}
Este $\zeta_{n}$ es el $n$-\'esimo segmento del proceso. El proceso es regenerativo sobre los tiempos $T_{n}$ si sus segmentos $\zeta_{n}$ son independientes e id\'enticamennte distribuidos.
\end{Def}


\begin{Note}
Si $\tilde{X}\left(t\right)$ con espacio de estados $\tilde{S}$ es regenerativo sobre $T_{n}$, entonces $X\left(t\right)=f\left(\tilde{X}\left(t\right)\right)$ tambi\'en es regenerativo sobre $T_{n}$, para cualquier funci\'on $f:\tilde{S}\rightarrow S$.
\end{Note}

\begin{Note}
Los procesos regenerativos son crudamente regenerativos, pero no al rev\'es.
\end{Note}


\begin{Note}
Un proceso estoc\'astico a tiempo continuo o discreto es regenerativo si existe un proceso de renovaci\'on  tal que los segmentos del proceso entre tiempos de renovaci\'on sucesivos son i.i.d., es decir, para $\left\{X\left(t\right):t\geq0\right\}$ proceso estoc\'astico a tiempo continuo con espacio de estados $S$, espacio m\'etrico.
\end{Note}

Para $\left\{X\left(t\right):t\geq0\right\}$ Proceso Estoc\'astico a tiempo continuo con estado de espacios $S$, que es un espacio m\'etrico, con trayectorias continuas por la derecha y con l\'imites por la izquierda c.s. Sea $N\left(t\right)$ un proceso de renovaci\'on en $\rea_{+}$ definido en el mismo espacio de probabilidad que $X\left(t\right)$, con tiempos de renovaci\'on $T$ y tiempos de inter-renovaci\'on $\xi_{n}=T_{n}-T_{n-1}$, con misma distribuci\'on $F$ de media finita $\mu$.



\begin{Def}
Para el proceso $\left\{\left(N\left(t\right),X\left(t\right)\right):t\geq0\right\}$, sus trayectoria muestrales en el intervalo de tiempo $\left[T_{n-1},T_{n}\right)$ est\'an descritas por
\begin{eqnarray*}
\zeta_{n}=\left(\xi_{n},\left\{X\left(T_{n-1}+t\right):0\leq t<\xi_{n}\right\}\right)
\end{eqnarray*}
Este $\zeta_{n}$ es el $n$-\'esimo segmento del proceso. El proceso es regenerativo sobre los tiempos $T_{n}$ si sus segmentos $\zeta_{n}$ son independientes e id\'enticamennte distribuidos.
\end{Def}

\begin{Note}
Un proceso regenerativo con media de la longitud de ciclo finita es llamado positivo recurrente.
\end{Note}

\begin{Teo}[Procesos Regenerativos]
Suponga que el proceso
\end{Teo}


\begin{Def}[Renewal Process Trinity]
Para un proceso de renovaci\'on $N\left(t\right)$, los siguientes procesos proveen de informaci\'on sobre los tiempos de renovaci\'on.
\begin{itemize}
\item $A\left(t\right)=t-T_{N\left(t\right)}$, el tiempo de recurrencia hacia atr\'as al tiempo $t$, que es el tiempo desde la \'ultima renovaci\'on para $t$.

\item $B\left(t\right)=T_{N\left(t\right)+1}-t$, el tiempo de recurrencia hacia adelante al tiempo $t$, residual del tiempo de renovaci\'on, que es el tiempo para la pr\'oxima renovaci\'on despu\'es de $t$.

\item $L\left(t\right)=\xi_{N\left(t\right)+1}=A\left(t\right)+B\left(t\right)$, la longitud del intervalo de renovaci\'on que contiene a $t$.
\end{itemize}
\end{Def}

\begin{Note}
El proceso tridimensional $\left(A\left(t\right),B\left(t\right),L\left(t\right)\right)$ es regenerativo sobre $T_{n}$, y por ende cada proceso lo es. Cada proceso $A\left(t\right)$ y $B\left(t\right)$ son procesos de MArkov a tiempo continuo con trayectorias continuas por partes en el espacio de estados $\rea_{+}$. Una expresi\'on conveniente para su distribuci\'on conjunta es, para $0\leq x<t,y\geq0$
\begin{equation}\label{NoRenovacion}
P\left\{A\left(t\right)>x,B\left(t\right)>y\right\}=
P\left\{N\left(t+y\right)-N\left((t-x)\right)=0\right\}
\end{equation}
\end{Note}

\begin{Ejem}[Tiempos de recurrencia Poisson]
Si $N\left(t\right)$ es un proceso Poisson con tasa $\lambda$, entonces de la expresi\'on (\ref{NoRenovacion}) se tiene que

\begin{eqnarray*}
\begin{array}{lc}
P\left\{A\left(t\right)>x,B\left(t\right)>y\right\}=e^{-\lambda\left(x+y\right)},&0\leq x<t,y\geq0,
\end{array}
\end{eqnarray*}
que es la probabilidad Poisson de no renovaciones en un intervalo de longitud $x+y$.

\end{Ejem}

\begin{Note}
Una cadena de Markov erg\'odica tiene la propiedad de ser estacionaria si la distribuci\'on de su estado al tiempo $0$ es su distribuci\'on estacionaria.
\end{Note}


\begin{Def}
Un proceso estoc\'astico a tiempo continuo $\left\{X\left(t\right):t\geq0\right\}$ en un espacio general es estacionario si sus distribuciones finito dimensionales son invariantes bajo cualquier  traslado: para cada $0\leq s_{1}<s_{2}<\cdots<s_{k}$ y $t\geq0$,
\begin{eqnarray*}
\left(X\left(s_{1}+t\right),\ldots,X\left(s_{k}+t\right)\right)=_{d}\left(X\left(s_{1}\right),\ldots,X\left(s_{k}\right)\right).
\end{eqnarray*}
\end{Def}

\begin{Note}
Un proceso de Markov es estacionario si $X\left(t\right)=_{d}X\left(0\right)$, $t\geq0$.
\end{Note}

Considerese el proceso $N\left(t\right)=\sum_{n}\indora\left(\tau_{n}\leq t\right)$ en $\rea_{+}$, con puntos $0<\tau_{1}<\tau_{2}<\cdots$.

\begin{Prop}
Si $N$ es un proceso puntual estacionario y $\esp\left[N\left(1\right)\right]<\infty$, entonces $\esp\left[N\left(t\right)\right]=t\esp\left[N\left(1\right)\right]$, $t\geq0$

\end{Prop}

\begin{Teo}
Los siguientes enunciados son equivalentes
\begin{itemize}
\item[i)] El proceso retardado de renovaci\'on $N$ es estacionario.

\item[ii)] EL proceso de tiempos de recurrencia hacia adelante $B\left(t\right)$ es estacionario.


\item[iii)] $\esp\left[N\left(t\right)\right]=t/\mu$,


\item[iv)] $G\left(t\right)=F_{e}\left(t\right)=\frac{1}{\mu}\int_{0}^{t}\left[1-F\left(s\right)\right]ds$
\end{itemize}
Cuando estos enunciados son ciertos, $P\left\{B\left(t\right)\leq x\right\}=F_{e}\left(x\right)$, para $t,x\geq0$.

\end{Teo}

\begin{Note}
Una consecuencia del teorema anterior es que el Proceso Poisson es el \'unico proceso sin retardo que es estacionario.
\end{Note}

\begin{Coro}
El proceso de renovaci\'on $N\left(t\right)$ sin retardo, y cuyos tiempos de inter renonaci\'on tienen media finita, es estacionario si y s\'olo si es un proceso Poisson.

\end{Coro}


%________________________________________________________________________
\subsection{Procesos Regenerativos}
%________________________________________________________________________



\begin{Note}
Si $\tilde{X}\left(t\right)$ con espacio de estados $\tilde{S}$ es regenerativo sobre $T_{n}$, entonces $X\left(t\right)=f\left(\tilde{X}\left(t\right)\right)$ tambi\'en es regenerativo sobre $T_{n}$, para cualquier funci\'on $f:\tilde{S}\rightarrow S$.
\end{Note}

\begin{Note}
Los procesos regenerativos son crudamente regenerativos, pero no al rev\'es.
\end{Note}
%\subsection*{Procesos Regenerativos: Sigman\cite{Sigman1}}
\begin{Def}[Definici\'on Cl\'asica]
Un proceso estoc\'astico $X=\left\{X\left(t\right):t\geq0\right\}$ es llamado regenerativo is existe una variable aleatoria $R_{1}>0$ tal que
\begin{itemize}
\item[i)] $\left\{X\left(t+R_{1}\right):t\geq0\right\}$ es independiente de $\left\{\left\{X\left(t\right):t<R_{1}\right\},\right\}$
\item[ii)] $\left\{X\left(t+R_{1}\right):t\geq0\right\}$ es estoc\'asticamente equivalente a $\left\{X\left(t\right):t>0\right\}$
\end{itemize}

Llamamos a $R_{1}$ tiempo de regeneraci\'on, y decimos que $X$ se regenera en este punto.
\end{Def}

$\left\{X\left(t+R_{1}\right)\right\}$ es regenerativo con tiempo de regeneraci\'on $R_{2}$, independiente de $R_{1}$ pero con la misma distribuci\'on que $R_{1}$. Procediendo de esta manera se obtiene una secuencia de variables aleatorias independientes e id\'enticamente distribuidas $\left\{R_{n}\right\}$ llamados longitudes de ciclo. Si definimos a $Z_{k}\equiv R_{1}+R_{2}+\cdots+R_{k}$, se tiene un proceso de renovaci\'on llamado proceso de renovaci\'on encajado para $X$.




\begin{Def}
Para $x$ fijo y para cada $t\geq0$, sea $I_{x}\left(t\right)=1$ si $X\left(t\right)\leq x$,  $I_{x}\left(t\right)=0$ en caso contrario, y def\'inanse los tiempos promedio
\begin{eqnarray*}
\overline{X}&=&lim_{t\rightarrow\infty}\frac{1}{t}\int_{0}^{\infty}X\left(u\right)du\\
\prob\left(X_{\infty}\leq x\right)&=&lim_{t\rightarrow\infty}\frac{1}{t}\int_{0}^{\infty}I_{x}\left(u\right)du,
\end{eqnarray*}
cuando estos l\'imites existan.
\end{Def}

Como consecuencia del teorema de Renovaci\'on-Recompensa, se tiene que el primer l\'imite  existe y es igual a la constante
\begin{eqnarray*}
\overline{X}&=&\frac{\esp\left[\int_{0}^{R_{1}}X\left(t\right)dt\right]}{\esp\left[R_{1}\right]},
\end{eqnarray*}
suponiendo que ambas esperanzas son finitas.

\begin{Note}
\begin{itemize}
\item[a)] Si el proceso regenerativo $X$ es positivo recurrente y tiene trayectorias muestrales no negativas, entonces la ecuaci\'on anterior es v\'alida.
\item[b)] Si $X$ es positivo recurrente regenerativo, podemos construir una \'unica versi\'on estacionaria de este proceso, $X_{e}=\left\{X_{e}\left(t\right)\right\}$, donde $X_{e}$ es un proceso estoc\'astico regenerativo y estrictamente estacionario, con distribuci\'on marginal distribuida como $X_{\infty}$
\end{itemize}
\end{Note}

%________________________________________________________________________
\subsection{Procesos Regenerativos}
%________________________________________________________________________

Para $\left\{X\left(t\right):t\geq0\right\}$ Proceso Estoc\'astico a tiempo continuo con estado de espacios $S$, que es un espacio m\'etrico, con trayectorias continuas por la derecha y con l\'imites por la izquierda c.s. Sea $N\left(t\right)$ un proceso de renovaci\'on en $\rea_{+}$ definido en el mismo espacio de probabilidad que $X\left(t\right)$, con tiempos de renovaci\'on $T$ y tiempos de inter-renovaci\'on $\xi_{n}=T_{n}-T_{n-1}$, con misma distribuci\'on $F$ de media finita $\mu$.



\begin{Def}
Para el proceso $\left\{\left(N\left(t\right),X\left(t\right)\right):t\geq0\right\}$, sus trayectoria muestrales en el intervalo de tiempo $\left[T_{n-1},T_{n}\right)$ est\'an descritas por
\begin{eqnarray*}
\zeta_{n}=\left(\xi_{n},\left\{X\left(T_{n-1}+t\right):0\leq t<\xi_{n}\right\}\right)
\end{eqnarray*}
Este $\zeta_{n}$ es el $n$-\'esimo segmento del proceso. El proceso es regenerativo sobre los tiempos $T_{n}$ si sus segmentos $\zeta_{n}$ son independientes e id\'enticamennte distribuidos.
\end{Def}


\begin{Note}
Si $\tilde{X}\left(t\right)$ con espacio de estados $\tilde{S}$ es regenerativo sobre $T_{n}$, entonces $X\left(t\right)=f\left(\tilde{X}\left(t\right)\right)$ tambi\'en es regenerativo sobre $T_{n}$, para cualquier funci\'on $f:\tilde{S}\rightarrow S$.
\end{Note}

\begin{Note}
Los procesos regenerativos son crudamente regenerativos, pero no al rev\'es.
\end{Note}

\begin{Def}[Definici\'on Cl\'asica]
Un proceso estoc\'astico $X=\left\{X\left(t\right):t\geq0\right\}$ es llamado regenerativo is existe una variable aleatoria $R_{1}>0$ tal que
\begin{itemize}
\item[i)] $\left\{X\left(t+R_{1}\right):t\geq0\right\}$ es independiente de $\left\{\left\{X\left(t\right):t<R_{1}\right\},\right\}$
\item[ii)] $\left\{X\left(t+R_{1}\right):t\geq0\right\}$ es estoc\'asticamente equivalente a $\left\{X\left(t\right):t>0\right\}$
\end{itemize}

Llamamos a $R_{1}$ tiempo de regeneraci\'on, y decimos que $X$ se regenera en este punto.
\end{Def}

$\left\{X\left(t+R_{1}\right)\right\}$ es regenerativo con tiempo de regeneraci\'on $R_{2}$, independiente de $R_{1}$ pero con la misma distribuci\'on que $R_{1}$. Procediendo de esta manera se obtiene una secuencia de variables aleatorias independientes e id\'enticamente distribuidas $\left\{R_{n}\right\}$ llamados longitudes de ciclo. Si definimos a $Z_{k}\equiv R_{1}+R_{2}+\cdots+R_{k}$, se tiene un proceso de renovaci\'on llamado proceso de renovaci\'on encajado para $X$.

\begin{Note}
Un proceso regenerativo con media de la longitud de ciclo finita es llamado positivo recurrente.
\end{Note}


\begin{Def}
Para $x$ fijo y para cada $t\geq0$, sea $I_{x}\left(t\right)=1$ si $X\left(t\right)\leq x$,  $I_{x}\left(t\right)=0$ en caso contrario, y def\'inanse los tiempos promedio
\begin{eqnarray*}
\overline{X}&=&lim_{t\rightarrow\infty}\frac{1}{t}\int_{0}^{\infty}X\left(u\right)du\\
\prob\left(X_{\infty}\leq x\right)&=&lim_{t\rightarrow\infty}\frac{1}{t}\int_{0}^{\infty}I_{x}\left(u\right)du,
\end{eqnarray*}
cuando estos l\'imites existan.
\end{Def}

Como consecuencia del teorema de Renovaci\'on-Recompensa, se tiene que el primer l\'imite  existe y es igual a la constante
\begin{eqnarray*}
\overline{X}&=&\frac{\esp\left[\int_{0}^{R_{1}}X\left(t\right)dt\right]}{\esp\left[R_{1}\right]},
\end{eqnarray*}
suponiendo que ambas esperanzas son finitas.

\begin{Note}
\begin{itemize}
\item[a)] Si el proceso regenerativo $X$ es positivo recurrente y tiene trayectorias muestrales no negativas, entonces la ecuaci\'on anterior es v\'alida.
\item[b)] Si $X$ es positivo recurrente regenerativo, podemos construir una \'unica versi\'on estacionaria de este proceso, $X_{e}=\left\{X_{e}\left(t\right)\right\}$, donde $X_{e}$ es un proceso estoc\'astico regenerativo y estrictamente estacionario, con distribuci\'on marginal distribuida como $X_{\infty}$
\end{itemize}
\end{Note}

%__________________________________________________________________________________________
\subsection{Procesos Regenerativos Estacionarios - Stidham \cite{Stidham}}
%__________________________________________________________________________________________


Un proceso estoc\'astico a tiempo continuo $\left\{V\left(t\right),t\geq0\right\}$ es un proceso regenerativo si existe una sucesi\'on de variables aleatorias independientes e id\'enticamente distribuidas $\left\{X_{1},X_{2},\ldots\right\}$, sucesi\'on de renovaci\'on, tal que para cualquier conjunto de Borel $A$, 

\begin{eqnarray*}
\prob\left\{V\left(t\right)\in A|X_{1}+X_{2}+\cdots+X_{R\left(t\right)}=s,\left\{V\left(\tau\right),\tau<s\right\}\right\}=\prob\left\{V\left(t-s\right)\in A|X_{1}>t-s\right\},
\end{eqnarray*}
para todo $0\leq s\leq t$, donde $R\left(t\right)=\max\left\{X_{1}+X_{2}+\cdots+X_{j}\leq t\right\}=$n\'umero de renovaciones ({\emph{puntos de regeneraci\'on}}) que ocurren en $\left[0,t\right]$. El intervalo $\left[0,X_{1}\right)$ es llamado {\emph{primer ciclo de regeneraci\'on}} de $\left\{V\left(t \right),t\geq0\right\}$, $\left[X_{1},X_{1}+X_{2}\right)$ el {\emph{segundo ciclo de regeneraci\'on}}, y as\'i sucesivamente.

Sea $X=X_{1}$ y sea $F$ la funci\'on de distrbuci\'on de $X$


\begin{Def}
Se define el proceso estacionario, $\left\{V^{*}\left(t\right),t\geq0\right\}$, para $\left\{V\left(t\right),t\geq0\right\}$ por

\begin{eqnarray*}
\prob\left\{V\left(t\right)\in A\right\}=\frac{1}{\esp\left[X\right]}\int_{0}^{\infty}\prob\left\{V\left(t+x\right)\in A|X>x\right\}\left(1-F\left(x\right)\right)dx,
\end{eqnarray*} 
para todo $t\geq0$ y todo conjunto de Borel $A$.
\end{Def}

\begin{Def}
Una distribuci\'on se dice que es {\emph{aritm\'etica}} si todos sus puntos de incremento son m\'ultiplos de la forma $0,\lambda, 2\lambda,\ldots$ para alguna $\lambda>0$ entera.
\end{Def}


\begin{Def}
Una modificaci\'on medible de un proceso $\left\{V\left(t\right),t\geq0\right\}$, es una versi\'on de este, $\left\{V\left(t,w\right)\right\}$ conjuntamente medible para $t\geq0$ y para $w\in S$, $S$ espacio de estados para $\left\{V\left(t\right),t\geq0\right\}$.
\end{Def}

\begin{Teo}
Sea $\left\{V\left(t\right),t\geq\right\}$ un proceso regenerativo no negativo con modificaci\'on medible. Sea $\esp\left[X\right]<\infty$. Entonces el proceso estacionario dado por la ecuaci\'on anterior est\'a bien definido y tiene funci\'on de distribuci\'on independiente de $t$, adem\'as
\begin{itemize}
\item[i)] \begin{eqnarray*}
\esp\left[V^{*}\left(0\right)\right]&=&\frac{\esp\left[\int_{0}^{X}V\left(s\right)ds\right]}{\esp\left[X\right]}\end{eqnarray*}
\item[ii)] Si $\esp\left[V^{*}\left(0\right)\right]<\infty$, equivalentemente, si $\esp\left[\int_{0}^{X}V\left(s\right)ds\right]<\infty$,entonces
\begin{eqnarray*}
\frac{\int_{0}^{t}V\left(s\right)ds}{t}\rightarrow\frac{\esp\left[\int_{0}^{X}V\left(s\right)ds\right]}{\esp\left[X\right]}
\end{eqnarray*}
con probabilidad 1 y en media, cuando $t\rightarrow\infty$.
\end{itemize}
\end{Teo}

%__________________________________________________________________________________________
\subsection{Procesos Regenerativos Estacionarios - Stidham \cite{Stidham}}
%__________________________________________________________________________________________


Un proceso estoc\'astico a tiempo continuo $\left\{V\left(t\right),t\geq0\right\}$ es un proceso regenerativo si existe una sucesi\'on de variables aleatorias independientes e id\'enticamente distribuidas $\left\{X_{1},X_{2},\ldots\right\}$, sucesi\'on de renovaci\'on, tal que para cualquier conjunto de Borel $A$, 

\begin{eqnarray*}
\prob\left\{V\left(t\right)\in A|X_{1}+X_{2}+\cdots+X_{R\left(t\right)}=s,\left\{V\left(\tau\right),\tau<s\right\}\right\}=\prob\left\{V\left(t-s\right)\in A|X_{1}>t-s\right\},
\end{eqnarray*}
para todo $0\leq s\leq t$, donde $R\left(t\right)=\max\left\{X_{1}+X_{2}+\cdots+X_{j}\leq t\right\}=$n\'umero de renovaciones ({\emph{puntos de regeneraci\'on}}) que ocurren en $\left[0,t\right]$. El intervalo $\left[0,X_{1}\right)$ es llamado {\emph{primer ciclo de regeneraci\'on}} de $\left\{V\left(t \right),t\geq0\right\}$, $\left[X_{1},X_{1}+X_{2}\right)$ el {\emph{segundo ciclo de regeneraci\'on}}, y as\'i sucesivamente.

Sea $X=X_{1}$ y sea $F$ la funci\'on de distrbuci\'on de $X$


\begin{Def}
Se define el proceso estacionario, $\left\{V^{*}\left(t\right),t\geq0\right\}$, para $\left\{V\left(t\right),t\geq0\right\}$ por

\begin{eqnarray*}
\prob\left\{V\left(t\right)\in A\right\}=\frac{1}{\esp\left[X\right]}\int_{0}^{\infty}\prob\left\{V\left(t+x\right)\in A|X>x\right\}\left(1-F\left(x\right)\right)dx,
\end{eqnarray*} 
para todo $t\geq0$ y todo conjunto de Borel $A$.
\end{Def}

\begin{Def}
Una distribuci\'on se dice que es {\emph{aritm\'etica}} si todos sus puntos de incremento son m\'ultiplos de la forma $0,\lambda, 2\lambda,\ldots$ para alguna $\lambda>0$ entera.
\end{Def}


\begin{Def}
Una modificaci\'on medible de un proceso $\left\{V\left(t\right),t\geq0\right\}$, es una versi\'on de este, $\left\{V\left(t,w\right)\right\}$ conjuntamente medible para $t\geq0$ y para $w\in S$, $S$ espacio de estados para $\left\{V\left(t\right),t\geq0\right\}$.
\end{Def}

\begin{Teo}
Sea $\left\{V\left(t\right),t\geq\right\}$ un proceso regenerativo no negativo con modificaci\'on medible. Sea $\esp\left[X\right]<\infty$. Entonces el proceso estacionario dado por la ecuaci\'on anterior est\'a bien definido y tiene funci\'on de distribuci\'on independiente de $t$, adem\'as
\begin{itemize}
\item[i)] \begin{eqnarray*}
\esp\left[V^{*}\left(0\right)\right]&=&\frac{\esp\left[\int_{0}^{X}V\left(s\right)ds\right]}{\esp\left[X\right]}\end{eqnarray*}
\item[ii)] Si $\esp\left[V^{*}\left(0\right)\right]<\infty$, equivalentemente, si $\esp\left[\int_{0}^{X}V\left(s\right)ds\right]<\infty$,entonces
\begin{eqnarray*}
\frac{\int_{0}^{t}V\left(s\right)ds}{t}\rightarrow\frac{\esp\left[\int_{0}^{X}V\left(s\right)ds\right]}{\esp\left[X\right]}
\end{eqnarray*}
con probabilidad 1 y en media, cuando $t\rightarrow\infty$.
\end{itemize}
\end{Teo}

Para $\left\{X\left(t\right):t\geq0\right\}$ Proceso Estoc\'astico a tiempo continuo con estado de espacios $S$, que es un espacio m\'etrico, con trayectorias continuas por la derecha y con l\'imites por la izquierda c.s. Sea $N\left(t\right)$ un proceso de renovaci\'on en $\rea_{+}$ definido en el mismo espacio de probabilidad que $X\left(t\right)$, con tiempos de renovaci\'on $T$ y tiempos de inter-renovaci\'on $\xi_{n}=T_{n}-T_{n-1}$, con misma distribuci\'on $F$ de media finita $\mu$.



\subsection{Renewal and Regenerative Processes: Serfozo\cite{Serfozo}}
\begin{Def}\label{Def.Tn}
Sean $0\leq T_{1}\leq T_{2}\leq \ldots$ son tiempos aleatorios infinitos en los cuales ocurren ciertos eventos. El n\'umero de tiempos $T_{n}$ en el intervalo $\left[0,t\right)$ es

\begin{eqnarray}
N\left(t\right)=\sum_{n=1}^{\infty}\indora\left(T_{n}\leq t\right),
\end{eqnarray}
para $t\geq0$.
\end{Def}

Si se consideran los puntos $T_{n}$ como elementos de $\rea_{+}$, y $N\left(t\right)$ es el n\'umero de puntos en $\rea$. El proceso denotado por $\left\{N\left(t\right):t\geq0\right\}$, denotado por $N\left(t\right)$, es un proceso puntual en $\rea_{+}$. Los $T_{n}$ son los tiempos de ocurrencia, el proceso puntual $N\left(t\right)$ es simple si su n\'umero de ocurrencias son distintas: $0<T_{1}<T_{2}<\ldots$ casi seguramente.

\begin{Def}
Un proceso puntual $N\left(t\right)$ es un proceso de renovaci\'on si los tiempos de interocurrencia $\xi_{n}=T_{n}-T_{n-1}$, para $n\geq1$, son independientes e identicamente distribuidos con distribuci\'on $F$, donde $F\left(0\right)=0$ y $T_{0}=0$. Los $T_{n}$ son llamados tiempos de renovaci\'on, referente a la independencia o renovaci\'on de la informaci\'on estoc\'astica en estos tiempos. Los $\xi_{n}$ son los tiempos de inter-renovaci\'on, y $N\left(t\right)$ es el n\'umero de renovaciones en el intervalo $\left[0,t\right)$
\end{Def}


\begin{Note}
Para definir un proceso de renovaci\'on para cualquier contexto, solamente hay que especificar una distribuci\'on $F$, con $F\left(0\right)=0$, para los tiempos de inter-renovaci\'on. La funci\'on $F$ en turno degune las otra variables aleatorias. De manera formal, existe un espacio de probabilidad y una sucesi\'on de variables aleatorias $\xi_{1},\xi_{2},\ldots$ definidas en este con distribuci\'on $F$. Entonces las otras cantidades son $T_{n}=\sum_{k=1}^{n}\xi_{k}$ y $N\left(t\right)=\sum_{n=1}^{\infty}\indora\left(T_{n}\leq t\right)$, donde $T_{n}\rightarrow\infty$ casi seguramente por la Ley Fuerte de los Grandes N\'umeros.
\end{Note}


Los tiempos $T_{n}$ est\'an relacionados con los conteos de $N\left(t\right)$ por

\begin{eqnarray*}
\left\{N\left(t\right)\geq n\right\}&=&\left\{T_{n}\leq t\right\}\\
T_{N\left(t\right)}\leq &t&<T_{N\left(t\right)+1},
\end{eqnarray*}

adem\'as $N\left(T_{n}\right)=n$, y 

\begin{eqnarray*}
N\left(t\right)=\max\left\{n:T_{n}\leq t\right\}=\min\left\{n:T_{n+1}>t\right\}
\end{eqnarray*}

Por propiedades de la convoluci\'on se sabe que

\begin{eqnarray*}
P\left\{T_{n}\leq t\right\}=F^{n\star}\left(t\right)
\end{eqnarray*}
que es la $n$-\'esima convoluci\'on de $F$. Entonces 

\begin{eqnarray*}
\left\{N\left(t\right)\geq n\right\}&=&\left\{T_{n}\leq t\right\}\\
P\left\{N\left(t\right)\leq n\right\}&=&1-F^{\left(n+1\right)\star}\left(t\right)
\end{eqnarray*}

Adem\'as usando el hecho de que $\esp\left[N\left(t\right)\right]=\sum_{n=1}^{\infty}P\left\{N\left(t\right)\geq n\right\}$
se tiene que

\begin{eqnarray*}
\esp\left[N\left(t\right)\right]=\sum_{n=1}^{\infty}F^{n\star}\left(t\right)
\end{eqnarray*}

\begin{Prop}
Para cada $t\geq0$, la funci\'on generadora de momentos $\esp\left[e^{\alpha N\left(t\right)}\right]$ existe para alguna $\alpha$ en una vecindad del 0, y de aqu\'i que $\esp\left[N\left(t\right)^{m}\right]<\infty$, para $m\geq1$.
\end{Prop}


\begin{Note}
Si el primer tiempo de renovaci\'on $\xi_{1}$ no tiene la misma distribuci\'on que el resto de las $\xi_{n}$, para $n\geq2$, a $N\left(t\right)$ se le llama Proceso de Renovaci\'on retardado, donde si $\xi$ tiene distribuci\'on $G$, entonces el tiempo $T_{n}$ de la $n$-\'esima renovaci\'on tiene distribuci\'on $G\star F^{\left(n-1\right)\star}\left(t\right)$
\end{Note}


\begin{Teo}
Para una constante $\mu\leq\infty$ ( o variable aleatoria), las siguientes expresiones son equivalentes:

\begin{eqnarray}
lim_{n\rightarrow\infty}n^{-1}T_{n}&=&\mu,\textrm{ c.s.}\\
lim_{t\rightarrow\infty}t^{-1}N\left(t\right)&=&1/\mu,\textrm{ c.s.}
\end{eqnarray}
\end{Teo}


Es decir, $T_{n}$ satisface la Ley Fuerte de los Grandes N\'umeros s\'i y s\'olo s\'i $N\left/t\right)$ la cumple.


\begin{Coro}[Ley Fuerte de los Grandes N\'umeros para Procesos de Renovaci\'on]
Si $N\left(t\right)$ es un proceso de renovaci\'on cuyos tiempos de inter-renovaci\'on tienen media $\mu\leq\infty$, entonces
\begin{eqnarray}
t^{-1}N\left(t\right)\rightarrow 1/\mu,\textrm{ c.s. cuando }t\rightarrow\infty.
\end{eqnarray}

\end{Coro}


Considerar el proceso estoc\'astico de valores reales $\left\{Z\left(t\right):t\geq0\right\}$ en el mismo espacio de probabilidad que $N\left(t\right)$

\begin{Def}
Para el proceso $\left\{Z\left(t\right):t\geq0\right\}$ se define la fluctuaci\'on m\'axima de $Z\left(t\right)$ en el intervalo $\left(T_{n-1},T_{n}\right]$:
\begin{eqnarray*}
M_{n}=\sup_{T_{n-1}<t\leq T_{n}}|Z\left(t\right)-Z\left(T_{n-1}\right)|
\end{eqnarray*}
\end{Def}

\begin{Teo}
Sup\'ongase que $n^{-1}T_{n}\rightarrow\mu$ c.s. cuando $n\rightarrow\infty$, donde $\mu\leq\infty$ es una constante o variable aleatoria. Sea $a$ una constante o variable aleatoria que puede ser infinita cuando $\mu$ es finita, y considere las expresiones l\'imite:
\begin{eqnarray}
lim_{n\rightarrow\infty}n^{-1}Z\left(T_{n}\right)&=&a,\textrm{ c.s.}\\
lim_{t\rightarrow\infty}t^{-1}Z\left(t\right)&=&a/\mu,\textrm{ c.s.}
\end{eqnarray}
La segunda expresi\'on implica la primera. Conversamente, la primera implica la segunda si el proceso $Z\left(t\right)$ es creciente, o si $lim_{n\rightarrow\infty}n^{-1}M_{n}=0$ c.s.
\end{Teo}

\begin{Coro}
Si $N\left(t\right)$ es un proceso de renovaci\'on, y $\left(Z\left(T_{n}\right)-Z\left(T_{n-1}\right),M_{n}\right)$, para $n\geq1$, son variables aleatorias independientes e id\'enticamente distribuidas con media finita, entonces,
\begin{eqnarray}
lim_{t\rightarrow\infty}t^{-1}Z\left(t\right)\rightarrow\frac{\esp\left[Z\left(T_{1}\right)-Z\left(T_{0}\right)\right]}{\esp\left[T_{1}\right]},\textrm{ c.s. cuando  }t\rightarrow\infty.
\end{eqnarray}
\end{Coro}


Sup\'ongase que $N\left(t\right)$ es un proceso de renovaci\'on con distribuci\'on $F$ con media finita $\mu$.

\begin{Def}
La funci\'on de renovaci\'on asociada con la distribuci\'on $F$, del proceso $N\left(t\right)$, es
\begin{eqnarray*}
U\left(t\right)=\sum_{n=1}^{\infty}F^{n\star}\left(t\right),\textrm{   }t\geq0,
\end{eqnarray*}
donde $F^{0\star}\left(t\right)=\indora\left(t\geq0\right)$.
\end{Def}


\begin{Prop}
Sup\'ongase que la distribuci\'on de inter-renovaci\'on $F$ tiene densidad $f$. Entonces $U\left(t\right)$ tambi\'en tiene densidad, para $t>0$, y es $U^{'}\left(t\right)=\sum_{n=0}^{\infty}f^{n\star}\left(t\right)$. Adem\'as
\begin{eqnarray*}
\prob\left\{N\left(t\right)>N\left(t-\right)\right\}=0\textrm{,   }t\geq0.
\end{eqnarray*}
\end{Prop}

\begin{Def}
La Transformada de Laplace-Stieljes de $F$ est\'a dada por

\begin{eqnarray*}
\hat{F}\left(\alpha\right)=\int_{\rea_{+}}e^{-\alpha t}dF\left(t\right)\textrm{,  }\alpha\geq0.
\end{eqnarray*}
\end{Def}

Entonces

\begin{eqnarray*}
\hat{U}\left(\alpha\right)=\sum_{n=0}^{\infty}\hat{F^{n\star}}\left(\alpha\right)=\sum_{n=0}^{\infty}\hat{F}\left(\alpha\right)^{n}=\frac{1}{1-\hat{F}\left(\alpha\right)}.
\end{eqnarray*}


\begin{Prop}
La Transformada de Laplace $\hat{U}\left(\alpha\right)$ y $\hat{F}\left(\alpha\right)$ determina una a la otra de manera \'unica por la relaci\'on $\hat{U}\left(\alpha\right)=\frac{1}{1-\hat{F}\left(\alpha\right)}$.
\end{Prop}


\begin{Note}
Un proceso de renovaci\'on $N\left(t\right)$ cuyos tiempos de inter-renovaci\'on tienen media finita, es un proceso Poisson con tasa $\lambda$ si y s\'olo s\'i $\esp\left[U\left(t\right)\right]=\lambda t$, para $t\geq0$.
\end{Note}


\begin{Teo}
Sea $N\left(t\right)$ un proceso puntual simple con puntos de localizaci\'on $T_{n}$ tal que $\eta\left(t\right)=\esp\left[N\left(\right)\right]$ es finita para cada $t$. Entonces para cualquier funci\'on $f:\rea_{+}\rightarrow\rea$,
\begin{eqnarray*}
\esp\left[\sum_{n=1}^{N\left(\right)}f\left(T_{n}\right)\right]=\int_{\left(0,t\right]}f\left(s\right)d\eta\left(s\right)\textrm{,  }t\geq0,
\end{eqnarray*}
suponiendo que la integral exista. Adem\'as si $X_{1},X_{2},\ldots$ son variables aleatorias definidas en el mismo espacio de probabilidad que el proceso $N\left(t\right)$ tal que $\esp\left[X_{n}|T_{n}=s\right]=f\left(s\right)$, independiente de $n$. Entonces
\begin{eqnarray*}
\esp\left[\sum_{n=1}^{N\left(t\right)}X_{n}\right]=\int_{\left(0,t\right]}f\left(s\right)d\eta\left(s\right)\textrm{,  }t\geq0,
\end{eqnarray*} 
suponiendo que la integral exista. 
\end{Teo}

\begin{Coro}[Identidad de Wald para Renovaciones]
Para el proceso de renovaci\'on $N\left(t\right)$,
\begin{eqnarray*}
\esp\left[T_{N\left(t\right)+1}\right]=\mu\esp\left[N\left(t\right)+1\right]\textrm{,  }t\geq0,
\end{eqnarray*}  
\end{Coro}


\begin{Def}
Sea $h\left(t\right)$ funci\'on de valores reales en $\rea$ acotada en intervalos finitos e igual a cero para $t<0$ La ecuaci\'on de renovaci\'on para $h\left(t\right)$ y la distribuci\'on $F$ es

\begin{eqnarray}\label{Ec.Renovacion}
H\left(t\right)=h\left(t\right)+\int_{\left[0,t\right]}H\left(t-s\right)dF\left(s\right)\textrm{,    }t\geq0,
\end{eqnarray}
donde $H\left(t\right)$ es una funci\'on de valores reales. Esto es $H=h+F\star H$. Decimos que $H\left(t\right)$ es soluci\'on de esta ecuaci\'on si satisface la ecuaci\'on, y es acotada en intervalos finitos e iguales a cero para $t<0$.
\end{Def}

\begin{Prop}
La funci\'on $U\star h\left(t\right)$ es la \'unica soluci\'on de la ecuaci\'on de renovaci\'on (\ref{Ec.Renovacion}).
\end{Prop}

\begin{Teo}[Teorema Renovaci\'on Elemental]
\begin{eqnarray*}
t^{-1}U\left(t\right)\rightarrow 1/\mu\textrm{,    cuando }t\rightarrow\infty.
\end{eqnarray*}
\end{Teo}



Sup\'ongase que $N\left(t\right)$ es un proceso de renovaci\'on con distribuci\'on $F$ con media finita $\mu$.

\begin{Def}
La funci\'on de renovaci\'on asociada con la distribuci\'on $F$, del proceso $N\left(t\right)$, es
\begin{eqnarray*}
U\left(t\right)=\sum_{n=1}^{\infty}F^{n\star}\left(t\right),\textrm{   }t\geq0,
\end{eqnarray*}
donde $F^{0\star}\left(t\right)=\indora\left(t\geq0\right)$.
\end{Def}


\begin{Prop}
Sup\'ongase que la distribuci\'on de inter-renovaci\'on $F$ tiene densidad $f$. Entonces $U\left(t\right)$ tambi\'en tiene densidad, para $t>0$, y es $U^{'}\left(t\right)=\sum_{n=0}^{\infty}f^{n\star}\left(t\right)$. Adem\'as
\begin{eqnarray*}
\prob\left\{N\left(t\right)>N\left(t-\right)\right\}=0\textrm{,   }t\geq0.
\end{eqnarray*}
\end{Prop}

\begin{Def}
La Transformada de Laplace-Stieljes de $F$ est\'a dada por

\begin{eqnarray*}
\hat{F}\left(\alpha\right)=\int_{\rea_{+}}e^{-\alpha t}dF\left(t\right)\textrm{,  }\alpha\geq0.
\end{eqnarray*}
\end{Def}

Entonces

\begin{eqnarray*}
\hat{U}\left(\alpha\right)=\sum_{n=0}^{\infty}\hat{F^{n\star}}\left(\alpha\right)=\sum_{n=0}^{\infty}\hat{F}\left(\alpha\right)^{n}=\frac{1}{1-\hat{F}\left(\alpha\right)}.
\end{eqnarray*}


\begin{Prop}
La Transformada de Laplace $\hat{U}\left(\alpha\right)$ y $\hat{F}\left(\alpha\right)$ determina una a la otra de manera \'unica por la relaci\'on $\hat{U}\left(\alpha\right)=\frac{1}{1-\hat{F}\left(\alpha\right)}$.
\end{Prop}


\begin{Note}
Un proceso de renovaci\'on $N\left(t\right)$ cuyos tiempos de inter-renovaci\'on tienen media finita, es un proceso Poisson con tasa $\lambda$ si y s\'olo s\'i $\esp\left[U\left(t\right)\right]=\lambda t$, para $t\geq0$.
\end{Note}


\begin{Teo}
Sea $N\left(t\right)$ un proceso puntual simple con puntos de localizaci\'on $T_{n}$ tal que $\eta\left(t\right)=\esp\left[N\left(\right)\right]$ es finita para cada $t$. Entonces para cualquier funci\'on $f:\rea_{+}\rightarrow\rea$,
\begin{eqnarray*}
\esp\left[\sum_{n=1}^{N\left(\right)}f\left(T_{n}\right)\right]=\int_{\left(0,t\right]}f\left(s\right)d\eta\left(s\right)\textrm{,  }t\geq0,
\end{eqnarray*}
suponiendo que la integral exista. Adem\'as si $X_{1},X_{2},\ldots$ son variables aleatorias definidas en el mismo espacio de probabilidad que el proceso $N\left(t\right)$ tal que $\esp\left[X_{n}|T_{n}=s\right]=f\left(s\right)$, independiente de $n$. Entonces
\begin{eqnarray*}
\esp\left[\sum_{n=1}^{N\left(t\right)}X_{n}\right]=\int_{\left(0,t\right]}f\left(s\right)d\eta\left(s\right)\textrm{,  }t\geq0,
\end{eqnarray*} 
suponiendo que la integral exista. 
\end{Teo}

\begin{Coro}[Identidad de Wald para Renovaciones]
Para el proceso de renovaci\'on $N\left(t\right)$,
\begin{eqnarray*}
\esp\left[T_{N\left(t\right)+1}\right]=\mu\esp\left[N\left(t\right)+1\right]\textrm{,  }t\geq0,
\end{eqnarray*}  
\end{Coro}


\begin{Def}
Sea $h\left(t\right)$ funci\'on de valores reales en $\rea$ acotada en intervalos finitos e igual a cero para $t<0$ La ecuaci\'on de renovaci\'on para $h\left(t\right)$ y la distribuci\'on $F$ es

\begin{eqnarray}\label{Ec.Renovacion}
H\left(t\right)=h\left(t\right)+\int_{\left[0,t\right]}H\left(t-s\right)dF\left(s\right)\textrm{,    }t\geq0,
\end{eqnarray}
donde $H\left(t\right)$ es una funci\'on de valores reales. Esto es $H=h+F\star H$. Decimos que $H\left(t\right)$ es soluci\'on de esta ecuaci\'on si satisface la ecuaci\'on, y es acotada en intervalos finitos e iguales a cero para $t<0$.
\end{Def}

\begin{Prop}
La funci\'on $U\star h\left(t\right)$ es la \'unica soluci\'on de la ecuaci\'on de renovaci\'on (\ref{Ec.Renovacion}).
\end{Prop}

\begin{Teo}[Teorema Renovaci\'on Elemental]
\begin{eqnarray*}
t^{-1}U\left(t\right)\rightarrow 1/\mu\textrm{,    cuando }t\rightarrow\infty.
\end{eqnarray*}
\end{Teo}


\begin{Note} Una funci\'on $h:\rea_{+}\rightarrow\rea$ es Directamente Riemann Integrable en los siguientes casos:
\begin{itemize}
\item[a)] $h\left(t\right)\geq0$ es decreciente y Riemann Integrable.
\item[b)] $h$ es continua excepto posiblemente en un conjunto de Lebesgue de medida 0, y $|h\left(t\right)|\leq b\left(t\right)$, donde $b$ es DRI.
\end{itemize}
\end{Note}

\begin{Teo}[Teorema Principal de Renovaci\'on]
Si $F$ es no aritm\'etica y $h\left(t\right)$ es Directamente Riemann Integrable (DRI), entonces

\begin{eqnarray*}
lim_{t\rightarrow\infty}U\star h=\frac{1}{\mu}\int_{\rea_{+}}h\left(s\right)ds.
\end{eqnarray*}
\end{Teo}

\begin{Prop}
Cualquier funci\'on $H\left(t\right)$ acotada en intervalos finitos y que es 0 para $t<0$ puede expresarse como
\begin{eqnarray*}
H\left(t\right)=U\star h\left(t\right)\textrm{,  donde }h\left(t\right)=H\left(t\right)-F\star H\left(t\right)
\end{eqnarray*}
\end{Prop}

\begin{Def}
Un proceso estoc\'astico $X\left(t\right)$ es crudamente regenerativo en un tiempo aleatorio positivo $T$ si
\begin{eqnarray*}
\esp\left[X\left(T+t\right)|T\right]=\esp\left[X\left(t\right)\right]\textrm{, para }t\geq0,\end{eqnarray*}
y con las esperanzas anteriores finitas.
\end{Def}

\begin{Prop}
Sup\'ongase que $X\left(t\right)$ es un proceso crudamente regenerativo en $T$, que tiene distribuci\'on $F$. Si $\esp\left[X\left(t\right)\right]$ es acotado en intervalos finitos, entonces
\begin{eqnarray*}
\esp\left[X\left(t\right)\right]=U\star h\left(t\right)\textrm{,  donde }h\left(t\right)=\esp\left[X\left(t\right)\indora\left(T>t\right)\right].
\end{eqnarray*}
\end{Prop}

\begin{Teo}[Regeneraci\'on Cruda]
Sup\'ongase que $X\left(t\right)$ es un proceso con valores positivo crudamente regenerativo en $T$, y def\'inase $M=\sup\left\{|X\left(t\right)|:t\leq T\right\}$. Si $T$ es no aritm\'etico y $M$ y $MT$ tienen media finita, entonces
\begin{eqnarray*}
lim_{t\rightarrow\infty}\esp\left[X\left(t\right)\right]=\frac{1}{\mu}\int_{\rea_{+}}h\left(s\right)ds,
\end{eqnarray*}
donde $h\left(t\right)=\esp\left[X\left(t\right)\indora\left(T>t\right)\right]$.
\end{Teo}


\begin{Note} Una funci\'on $h:\rea_{+}\rightarrow\rea$ es Directamente Riemann Integrable en los siguientes casos:
\begin{itemize}
\item[a)] $h\left(t\right)\geq0$ es decreciente y Riemann Integrable.
\item[b)] $h$ es continua excepto posiblemente en un conjunto de Lebesgue de medida 0, y $|h\left(t\right)|\leq b\left(t\right)$, donde $b$ es DRI.
\end{itemize}
\end{Note}

\begin{Teo}[Teorema Principal de Renovaci\'on]
Si $F$ es no aritm\'etica y $h\left(t\right)$ es Directamente Riemann Integrable (DRI), entonces

\begin{eqnarray*}
lim_{t\rightarrow\infty}U\star h=\frac{1}{\mu}\int_{\rea_{+}}h\left(s\right)ds.
\end{eqnarray*}
\end{Teo}

\begin{Prop}
Cualquier funci\'on $H\left(t\right)$ acotada en intervalos finitos y que es 0 para $t<0$ puede expresarse como
\begin{eqnarray*}
H\left(t\right)=U\star h\left(t\right)\textrm{,  donde }h\left(t\right)=H\left(t\right)-F\star H\left(t\right)
\end{eqnarray*}
\end{Prop}

\begin{Def}
Un proceso estoc\'astico $X\left(t\right)$ es crudamente regenerativo en un tiempo aleatorio positivo $T$ si
\begin{eqnarray*}
\esp\left[X\left(T+t\right)|T\right]=\esp\left[X\left(t\right)\right]\textrm{, para }t\geq0,\end{eqnarray*}
y con las esperanzas anteriores finitas.
\end{Def}

\begin{Prop}
Sup\'ongase que $X\left(t\right)$ es un proceso crudamente regenerativo en $T$, que tiene distribuci\'on $F$. Si $\esp\left[X\left(t\right)\right]$ es acotado en intervalos finitos, entonces
\begin{eqnarray*}
\esp\left[X\left(t\right)\right]=U\star h\left(t\right)\textrm{,  donde }h\left(t\right)=\esp\left[X\left(t\right)\indora\left(T>t\right)\right].
\end{eqnarray*}
\end{Prop}

\begin{Teo}[Regeneraci\'on Cruda]
Sup\'ongase que $X\left(t\right)$ es un proceso con valores positivo crudamente regenerativo en $T$, y def\'inase $M=\sup\left\{|X\left(t\right)|:t\leq T\right\}$. Si $T$ es no aritm\'etico y $M$ y $MT$ tienen media finita, entonces
\begin{eqnarray*}
lim_{t\rightarrow\infty}\esp\left[X\left(t\right)\right]=\frac{1}{\mu}\int_{\rea_{+}}h\left(s\right)ds,
\end{eqnarray*}
donde $h\left(t\right)=\esp\left[X\left(t\right)\indora\left(T>t\right)\right]$.
\end{Teo}

%________________________________________________________________________
\subsection{Procesos Regenerativos}
%________________________________________________________________________

Para $\left\{X\left(t\right):t\geq0\right\}$ Proceso Estoc\'astico a tiempo continuo con estado de espacios $S$, que es un espacio m\'etrico, con trayectorias continuas por la derecha y con l\'imites por la izquierda c.s. Sea $N\left(t\right)$ un proceso de renovaci\'on en $\rea_{+}$ definido en el mismo espacio de probabilidad que $X\left(t\right)$, con tiempos de renovaci\'on $T$ y tiempos de inter-renovaci\'on $\xi_{n}=T_{n}-T_{n-1}$, con misma distribuci\'on $F$ de media finita $\mu$.



\begin{Def}
Para el proceso $\left\{\left(N\left(t\right),X\left(t\right)\right):t\geq0\right\}$, sus trayectoria muestrales en el intervalo de tiempo $\left[T_{n-1},T_{n}\right)$ est\'an descritas por
\begin{eqnarray*}
\zeta_{n}=\left(\xi_{n},\left\{X\left(T_{n-1}+t\right):0\leq t<\xi_{n}\right\}\right)
\end{eqnarray*}
Este $\zeta_{n}$ es el $n$-\'esimo segmento del proceso. El proceso es regenerativo sobre los tiempos $T_{n}$ si sus segmentos $\zeta_{n}$ son independientes e id\'enticamennte distribuidos.
\end{Def}


\begin{Obs}
Si $\tilde{X}\left(t\right)$ con espacio de estados $\tilde{S}$ es regenerativo sobre $T_{n}$, entonces $X\left(t\right)=f\left(\tilde{X}\left(t\right)\right)$ tambi\'en es regenerativo sobre $T_{n}$, para cualquier funci\'on $f:\tilde{S}\rightarrow S$.
\end{Obs}

\begin{Obs}
Los procesos regenerativos son crudamente regenerativos, pero no al rev\'es.
\end{Obs}

\begin{Def}[Definici\'on Cl\'asica]
Un proceso estoc\'astico $X=\left\{X\left(t\right):t\geq0\right\}$ es llamado regenerativo is existe una variable aleatoria $R_{1}>0$ tal que
\begin{itemize}
\item[i)] $\left\{X\left(t+R_{1}\right):t\geq0\right\}$ es independiente de $\left\{\left\{X\left(t\right):t<R_{1}\right\},\right\}$
\item[ii)] $\left\{X\left(t+R_{1}\right):t\geq0\right\}$ es estoc\'asticamente equivalente a $\left\{X\left(t\right):t>0\right\}$
\end{itemize}

Llamamos a $R_{1}$ tiempo de regeneraci\'on, y decimos que $X$ se regenera en este punto.
\end{Def}

$\left\{X\left(t+R_{1}\right)\right\}$ es regenerativo con tiempo de regeneraci\'on $R_{2}$, independiente de $R_{1}$ pero con la misma distribuci\'on que $R_{1}$. Procediendo de esta manera se obtiene una secuencia de variables aleatorias independientes e id\'enticamente distribuidas $\left\{R_{n}\right\}$ llamados longitudes de ciclo. Si definimos a $Z_{k}\equiv R_{1}+R_{2}+\cdots+R_{k}$, se tiene un proceso de renovaci\'on llamado proceso de renovaci\'on encajado para $X$.

\begin{Note}
Un proceso regenerativo con media de la longitud de ciclo finita es llamado positivo recurrente.
\end{Note}


\begin{Def}
Para $x$ fijo y para cada $t\geq0$, sea $I_{x}\left(t\right)=1$ si $X\left(t\right)\leq x$,  $I_{x}\left(t\right)=0$ en caso contrario, y def\'inanse los tiempos promedio
\begin{eqnarray*}
\overline{X}&=&lim_{t\rightarrow\infty}\frac{1}{t}\int_{0}^{\infty}X\left(u\right)du\\
\prob\left(X_{\infty}\leq x\right)&=&lim_{t\rightarrow\infty}\frac{1}{t}\int_{0}^{\infty}I_{x}\left(u\right)du,
\end{eqnarray*}
cuando estos l\'imites existan.
\end{Def}

Como consecuencia del teorema de Renovaci\'on-Recompensa, se tiene que el primer l\'imite  existe y es igual a la constante
\begin{eqnarray*}
\overline{X}&=&\frac{\esp\left[\int_{0}^{R_{1}}X\left(t\right)dt\right]}{\esp\left[R_{1}\right]},
\end{eqnarray*}
suponiendo que ambas esperanzas son finitas.

\begin{Note}
\begin{itemize}
\item[a)] Si el proceso regenerativo $X$ es positivo recurrente y tiene trayectorias muestrales no negativas, entonces la ecuaci\'on anterior es v\'alida.
\item[b)] Si $X$ es positivo recurrente regenerativo, podemos construir una \'unica versi\'on estacionaria de este proceso, $X_{e}=\left\{X_{e}\left(t\right)\right\}$, donde $X_{e}$ es un proceso estoc\'astico regenerativo y estrictamente estacionario, con distribuci\'on marginal distribuida como $X_{\infty}$
\end{itemize}
\end{Note}

%________________________________________________________________________
\subsection{Procesos Regenerativos}
%________________________________________________________________________

Para $\left\{X\left(t\right):t\geq0\right\}$ Proceso Estoc\'astico a tiempo continuo con estado de espacios $S$, que es un espacio m\'etrico, con trayectorias continuas por la derecha y con l\'imites por la izquierda c.s. Sea $N\left(t\right)$ un proceso de renovaci\'on en $\rea_{+}$ definido en el mismo espacio de probabilidad que $X\left(t\right)$, con tiempos de renovaci\'on $T$ y tiempos de inter-renovaci\'on $\xi_{n}=T_{n}-T_{n-1}$, con misma distribuci\'on $F$ de media finita $\mu$.



\begin{Def}
Para el proceso $\left\{\left(N\left(t\right),X\left(t\right)\right):t\geq0\right\}$, sus trayectoria muestrales en el intervalo de tiempo $\left[T_{n-1},T_{n}\right)$ est\'an descritas por
\begin{eqnarray*}
\zeta_{n}=\left(\xi_{n},\left\{X\left(T_{n-1}+t\right):0\leq t<\xi_{n}\right\}\right)
\end{eqnarray*}
Este $\zeta_{n}$ es el $n$-\'esimo segmento del proceso. El proceso es regenerativo sobre los tiempos $T_{n}$ si sus segmentos $\zeta_{n}$ son independientes e id\'enticamennte distribuidos.
\end{Def}


\begin{Obs}
Si $\tilde{X}\left(t\right)$ con espacio de estados $\tilde{S}$ es regenerativo sobre $T_{n}$, entonces $X\left(t\right)=f\left(\tilde{X}\left(t\right)\right)$ tambi\'en es regenerativo sobre $T_{n}$, para cualquier funci\'on $f:\tilde{S}\rightarrow S$.
\end{Obs}

\begin{Obs}
Los procesos regenerativos son crudamente regenerativos, pero no al rev\'es.
\end{Obs}

\begin{Def}[Definici\'on Cl\'asica]
Un proceso estoc\'astico $X=\left\{X\left(t\right):t\geq0\right\}$ es llamado regenerativo is existe una variable aleatoria $R_{1}>0$ tal que
\begin{itemize}
\item[i)] $\left\{X\left(t+R_{1}\right):t\geq0\right\}$ es independiente de $\left\{\left\{X\left(t\right):t<R_{1}\right\},\right\}$
\item[ii)] $\left\{X\left(t+R_{1}\right):t\geq0\right\}$ es estoc\'asticamente equivalente a $\left\{X\left(t\right):t>0\right\}$
\end{itemize}

Llamamos a $R_{1}$ tiempo de regeneraci\'on, y decimos que $X$ se regenera en este punto.
\end{Def}

$\left\{X\left(t+R_{1}\right)\right\}$ es regenerativo con tiempo de regeneraci\'on $R_{2}$, independiente de $R_{1}$ pero con la misma distribuci\'on que $R_{1}$. Procediendo de esta manera se obtiene una secuencia de variables aleatorias independientes e id\'enticamente distribuidas $\left\{R_{n}\right\}$ llamados longitudes de ciclo. Si definimos a $Z_{k}\equiv R_{1}+R_{2}+\cdots+R_{k}$, se tiene un proceso de renovaci\'on llamado proceso de renovaci\'on encajado para $X$.

\begin{Note}
Un proceso regenerativo con media de la longitud de ciclo finita es llamado positivo recurrente.
\end{Note}


\begin{Def}
Para $x$ fijo y para cada $t\geq0$, sea $I_{x}\left(t\right)=1$ si $X\left(t\right)\leq x$,  $I_{x}\left(t\right)=0$ en caso contrario, y def\'inanse los tiempos promedio
\begin{eqnarray*}
\overline{X}&=&lim_{t\rightarrow\infty}\frac{1}{t}\int_{0}^{\infty}X\left(u\right)du\\
\prob\left(X_{\infty}\leq x\right)&=&lim_{t\rightarrow\infty}\frac{1}{t}\int_{0}^{\infty}I_{x}\left(u\right)du,
\end{eqnarray*}
cuando estos l\'imites existan.
\end{Def}

Como consecuencia del teorema de Renovaci\'on-Recompensa, se tiene que el primer l\'imite  existe y es igual a la constante
\begin{eqnarray*}
\overline{X}&=&\frac{\esp\left[\int_{0}^{R_{1}}X\left(t\right)dt\right]}{\esp\left[R_{1}\right]},
\end{eqnarray*}
suponiendo que ambas esperanzas son finitas.

\begin{Note}
\begin{itemize}
\item[a)] Si el proceso regenerativo $X$ es positivo recurrente y tiene trayectorias muestrales no negativas, entonces la ecuaci\'on anterior es v\'alida.
\item[b)] Si $X$ es positivo recurrente regenerativo, podemos construir una \'unica versi\'on estacionaria de este proceso, $X_{e}=\left\{X_{e}\left(t\right)\right\}$, donde $X_{e}$ es un proceso estoc\'astico regenerativo y estrictamente estacionario, con distribuci\'on marginal distribuida como $X_{\infty}$
\end{itemize}
\end{Note}
%__________________________________________________________________________________________
\subsection{Procesos Regenerativos Estacionarios - Stidham \cite{Stidham}}
%__________________________________________________________________________________________


Un proceso estoc\'astico a tiempo continuo $\left\{V\left(t\right),t\geq0\right\}$ es un proceso regenerativo si existe una sucesi\'on de variables aleatorias independientes e id\'enticamente distribuidas $\left\{X_{1},X_{2},\ldots\right\}$, sucesi\'on de renovaci\'on, tal que para cualquier conjunto de Borel $A$, 

\begin{eqnarray*}
\prob\left\{V\left(t\right)\in A|X_{1}+X_{2}+\cdots+X_{R\left(t\right)}=s,\left\{V\left(\tau\right),\tau<s\right\}\right\}=\prob\left\{V\left(t-s\right)\in A|X_{1}>t-s\right\},
\end{eqnarray*}
para todo $0\leq s\leq t$, donde $R\left(t\right)=\max\left\{X_{1}+X_{2}+\cdots+X_{j}\leq t\right\}=$n\'umero de renovaciones ({\emph{puntos de regeneraci\'on}}) que ocurren en $\left[0,t\right]$. El intervalo $\left[0,X_{1}\right)$ es llamado {\emph{primer ciclo de regeneraci\'on}} de $\left\{V\left(t \right),t\geq0\right\}$, $\left[X_{1},X_{1}+X_{2}\right)$ el {\emph{segundo ciclo de regeneraci\'on}}, y as\'i sucesivamente.

Sea $X=X_{1}$ y sea $F$ la funci\'on de distrbuci\'on de $X$


\begin{Def}
Se define el proceso estacionario, $\left\{V^{*}\left(t\right),t\geq0\right\}$, para $\left\{V\left(t\right),t\geq0\right\}$ por

\begin{eqnarray*}
\prob\left\{V\left(t\right)\in A\right\}=\frac{1}{\esp\left[X\right]}\int_{0}^{\infty}\prob\left\{V\left(t+x\right)\in A|X>x\right\}\left(1-F\left(x\right)\right)dx,
\end{eqnarray*} 
para todo $t\geq0$ y todo conjunto de Borel $A$.
\end{Def}

\begin{Def}
Una distribuci\'on se dice que es {\emph{aritm\'etica}} si todos sus puntos de incremento son m\'ultiplos de la forma $0,\lambda, 2\lambda,\ldots$ para alguna $\lambda>0$ entera.
\end{Def}


\begin{Def}
Una modificaci\'on medible de un proceso $\left\{V\left(t\right),t\geq0\right\}$, es una versi\'on de este, $\left\{V\left(t,w\right)\right\}$ conjuntamente medible para $t\geq0$ y para $w\in S$, $S$ espacio de estados para $\left\{V\left(t\right),t\geq0\right\}$.
\end{Def}

\begin{Teo}
Sea $\left\{V\left(t\right),t\geq\right\}$ un proceso regenerativo no negativo con modificaci\'on medible. Sea $\esp\left[X\right]<\infty$. Entonces el proceso estacionario dado por la ecuaci\'on anterior est\'a bien definido y tiene funci\'on de distribuci\'on independiente de $t$, adem\'as
\begin{itemize}
\item[i)] \begin{eqnarray*}
\esp\left[V^{*}\left(0\right)\right]&=&\frac{\esp\left[\int_{0}^{X}V\left(s\right)ds\right]}{\esp\left[X\right]}\end{eqnarray*}
\item[ii)] Si $\esp\left[V^{*}\left(0\right)\right]<\infty$, equivalentemente, si $\esp\left[\int_{0}^{X}V\left(s\right)ds\right]<\infty$,entonces
\begin{eqnarray*}
\frac{\int_{0}^{t}V\left(s\right)ds}{t}\rightarrow\frac{\esp\left[\int_{0}^{X}V\left(s\right)ds\right]}{\esp\left[X\right]}
\end{eqnarray*}
con probabilidad 1 y en media, cuando $t\rightarrow\infty$.
\end{itemize}
\end{Teo}


%__________________________________________________________________________________________
\subsection{Procesos Regenerativos Estacionarios - Stidham \cite{Stidham}}
%__________________________________________________________________________________________


Un proceso estoc\'astico a tiempo continuo $\left\{V\left(t\right),t\geq0\right\}$ es un proceso regenerativo si existe una sucesi\'on de variables aleatorias independientes e id\'enticamente distribuidas $\left\{X_{1},X_{2},\ldots\right\}$, sucesi\'on de renovaci\'on, tal que para cualquier conjunto de Borel $A$, 

\begin{eqnarray*}
\prob\left\{V\left(t\right)\in A|X_{1}+X_{2}+\cdots+X_{R\left(t\right)}=s,\left\{V\left(\tau\right),\tau<s\right\}\right\}=\prob\left\{V\left(t-s\right)\in A|X_{1}>t-s\right\},
\end{eqnarray*}
para todo $0\leq s\leq t$, donde $R\left(t\right)=\max\left\{X_{1}+X_{2}+\cdots+X_{j}\leq t\right\}=$n\'umero de renovaciones ({\emph{puntos de regeneraci\'on}}) que ocurren en $\left[0,t\right]$. El intervalo $\left[0,X_{1}\right)$ es llamado {\emph{primer ciclo de regeneraci\'on}} de $\left\{V\left(t \right),t\geq0\right\}$, $\left[X_{1},X_{1}+X_{2}\right)$ el {\emph{segundo ciclo de regeneraci\'on}}, y as\'i sucesivamente.

Sea $X=X_{1}$ y sea $F$ la funci\'on de distrbuci\'on de $X$


\begin{Def}
Se define el proceso estacionario, $\left\{V^{*}\left(t\right),t\geq0\right\}$, para $\left\{V\left(t\right),t\geq0\right\}$ por

\begin{eqnarray*}
\prob\left\{V\left(t\right)\in A\right\}=\frac{1}{\esp\left[X\right]}\int_{0}^{\infty}\prob\left\{V\left(t+x\right)\in A|X>x\right\}\left(1-F\left(x\right)\right)dx,
\end{eqnarray*} 
para todo $t\geq0$ y todo conjunto de Borel $A$.
\end{Def}

\begin{Def}
Una distribuci\'on se dice que es {\emph{aritm\'etica}} si todos sus puntos de incremento son m\'ultiplos de la forma $0,\lambda, 2\lambda,\ldots$ para alguna $\lambda>0$ entera.
\end{Def}


\begin{Def}
Una modificaci\'on medible de un proceso $\left\{V\left(t\right),t\geq0\right\}$, es una versi\'on de este, $\left\{V\left(t,w\right)\right\}$ conjuntamente medible para $t\geq0$ y para $w\in S$, $S$ espacio de estados para $\left\{V\left(t\right),t\geq0\right\}$.
\end{Def}

\begin{Teo}
Sea $\left\{V\left(t\right),t\geq\right\}$ un proceso regenerativo no negativo con modificaci\'on medible. Sea $\esp\left[X\right]<\infty$. Entonces el proceso estacionario dado por la ecuaci\'on anterior est\'a bien definido y tiene funci\'on de distribuci\'on independiente de $t$, adem\'as
\begin{itemize}
\item[i)] \begin{eqnarray*}
\esp\left[V^{*}\left(0\right)\right]&=&\frac{\esp\left[\int_{0}^{X}V\left(s\right)ds\right]}{\esp\left[X\right]}\end{eqnarray*}
\item[ii)] Si $\esp\left[V^{*}\left(0\right)\right]<\infty$, equivalentemente, si $\esp\left[\int_{0}^{X}V\left(s\right)ds\right]<\infty$,entonces
\begin{eqnarray*}
\frac{\int_{0}^{t}V\left(s\right)ds}{t}\rightarrow\frac{\esp\left[\int_{0}^{X}V\left(s\right)ds\right]}{\esp\left[X\right]}
\end{eqnarray*}
con probabilidad 1 y en media, cuando $t\rightarrow\infty$.
\end{itemize}
\end{Teo}
%___________________________________________________________________________________________
%
\subsection{Propiedades de los Procesos de Renovaci\'on}
%___________________________________________________________________________________________
%

Los tiempos $T_{n}$ est\'an relacionados con los conteos de $N\left(t\right)$ por

\begin{eqnarray*}
\left\{N\left(t\right)\geq n\right\}&=&\left\{T_{n}\leq t\right\}\\
T_{N\left(t\right)}\leq &t&<T_{N\left(t\right)+1},
\end{eqnarray*}

adem\'as $N\left(T_{n}\right)=n$, y 

\begin{eqnarray*}
N\left(t\right)=\max\left\{n:T_{n}\leq t\right\}=\min\left\{n:T_{n+1}>t\right\}
\end{eqnarray*}

Por propiedades de la convoluci\'on se sabe que

\begin{eqnarray*}
P\left\{T_{n}\leq t\right\}=F^{n\star}\left(t\right)
\end{eqnarray*}
que es la $n$-\'esima convoluci\'on de $F$. Entonces 

\begin{eqnarray*}
\left\{N\left(t\right)\geq n\right\}&=&\left\{T_{n}\leq t\right\}\\
P\left\{N\left(t\right)\leq n\right\}&=&1-F^{\left(n+1\right)\star}\left(t\right)
\end{eqnarray*}

Adem\'as usando el hecho de que $\esp\left[N\left(t\right)\right]=\sum_{n=1}^{\infty}P\left\{N\left(t\right)\geq n\right\}$
se tiene que

\begin{eqnarray*}
\esp\left[N\left(t\right)\right]=\sum_{n=1}^{\infty}F^{n\star}\left(t\right)
\end{eqnarray*}

\begin{Prop}
Para cada $t\geq0$, la funci\'on generadora de momentos $\esp\left[e^{\alpha N\left(t\right)}\right]$ existe para alguna $\alpha$ en una vecindad del 0, y de aqu\'i que $\esp\left[N\left(t\right)^{m}\right]<\infty$, para $m\geq1$.
\end{Prop}


\begin{Note}
Si el primer tiempo de renovaci\'on $\xi_{1}$ no tiene la misma distribuci\'on que el resto de las $\xi_{n}$, para $n\geq2$, a $N\left(t\right)$ se le llama Proceso de Renovaci\'on retardado, donde si $\xi$ tiene distribuci\'on $G$, entonces el tiempo $T_{n}$ de la $n$-\'esima renovaci\'on tiene distribuci\'on $G\star F^{\left(n-1\right)\star}\left(t\right)$
\end{Note}


\begin{Teo}
Para una constante $\mu\leq\infty$ ( o variable aleatoria), las siguientes expresiones son equivalentes:

\begin{eqnarray}
lim_{n\rightarrow\infty}n^{-1}T_{n}&=&\mu,\textrm{ c.s.}\\
lim_{t\rightarrow\infty}t^{-1}N\left(t\right)&=&1/\mu,\textrm{ c.s.}
\end{eqnarray}
\end{Teo}


Es decir, $T_{n}$ satisface la Ley Fuerte de los Grandes N\'umeros s\'i y s\'olo s\'i $N\left/t\right)$ la cumple.


\begin{Coro}[Ley Fuerte de los Grandes N\'umeros para Procesos de Renovaci\'on]
Si $N\left(t\right)$ es un proceso de renovaci\'on cuyos tiempos de inter-renovaci\'on tienen media $\mu\leq\infty$, entonces
\begin{eqnarray}
t^{-1}N\left(t\right)\rightarrow 1/\mu,\textrm{ c.s. cuando }t\rightarrow\infty.
\end{eqnarray}

\end{Coro}


Considerar el proceso estoc\'astico de valores reales $\left\{Z\left(t\right):t\geq0\right\}$ en el mismo espacio de probabilidad que $N\left(t\right)$

\begin{Def}
Para el proceso $\left\{Z\left(t\right):t\geq0\right\}$ se define la fluctuaci\'on m\'axima de $Z\left(t\right)$ en el intervalo $\left(T_{n-1},T_{n}\right]$:
\begin{eqnarray*}
M_{n}=\sup_{T_{n-1}<t\leq T_{n}}|Z\left(t\right)-Z\left(T_{n-1}\right)|
\end{eqnarray*}
\end{Def}

\begin{Teo}
Sup\'ongase que $n^{-1}T_{n}\rightarrow\mu$ c.s. cuando $n\rightarrow\infty$, donde $\mu\leq\infty$ es una constante o variable aleatoria. Sea $a$ una constante o variable aleatoria que puede ser infinita cuando $\mu$ es finita, y considere las expresiones l\'imite:
\begin{eqnarray}
lim_{n\rightarrow\infty}n^{-1}Z\left(T_{n}\right)&=&a,\textrm{ c.s.}\\
lim_{t\rightarrow\infty}t^{-1}Z\left(t\right)&=&a/\mu,\textrm{ c.s.}
\end{eqnarray}
La segunda expresi\'on implica la primera. Conversamente, la primera implica la segunda si el proceso $Z\left(t\right)$ es creciente, o si $lim_{n\rightarrow\infty}n^{-1}M_{n}=0$ c.s.
\end{Teo}

\begin{Coro}
Si $N\left(t\right)$ es un proceso de renovaci\'on, y $\left(Z\left(T_{n}\right)-Z\left(T_{n-1}\right),M_{n}\right)$, para $n\geq1$, son variables aleatorias independientes e id\'enticamente distribuidas con media finita, entonces,
\begin{eqnarray}
lim_{t\rightarrow\infty}t^{-1}Z\left(t\right)\rightarrow\frac{\esp\left[Z\left(T_{1}\right)-Z\left(T_{0}\right)\right]}{\esp\left[T_{1}\right]},\textrm{ c.s. cuando  }t\rightarrow\infty.
\end{eqnarray}
\end{Coro}


%___________________________________________________________________________________________
%
\subsection{Propiedades de los Procesos de Renovaci\'on}
%___________________________________________________________________________________________
%

Los tiempos $T_{n}$ est\'an relacionados con los conteos de $N\left(t\right)$ por

\begin{eqnarray*}
\left\{N\left(t\right)\geq n\right\}&=&\left\{T_{n}\leq t\right\}\\
T_{N\left(t\right)}\leq &t&<T_{N\left(t\right)+1},
\end{eqnarray*}

adem\'as $N\left(T_{n}\right)=n$, y 

\begin{eqnarray*}
N\left(t\right)=\max\left\{n:T_{n}\leq t\right\}=\min\left\{n:T_{n+1}>t\right\}
\end{eqnarray*}

Por propiedades de la convoluci\'on se sabe que

\begin{eqnarray*}
P\left\{T_{n}\leq t\right\}=F^{n\star}\left(t\right)
\end{eqnarray*}
que es la $n$-\'esima convoluci\'on de $F$. Entonces 

\begin{eqnarray*}
\left\{N\left(t\right)\geq n\right\}&=&\left\{T_{n}\leq t\right\}\\
P\left\{N\left(t\right)\leq n\right\}&=&1-F^{\left(n+1\right)\star}\left(t\right)
\end{eqnarray*}

Adem\'as usando el hecho de que $\esp\left[N\left(t\right)\right]=\sum_{n=1}^{\infty}P\left\{N\left(t\right)\geq n\right\}$
se tiene que

\begin{eqnarray*}
\esp\left[N\left(t\right)\right]=\sum_{n=1}^{\infty}F^{n\star}\left(t\right)
\end{eqnarray*}

\begin{Prop}
Para cada $t\geq0$, la funci\'on generadora de momentos $\esp\left[e^{\alpha N\left(t\right)}\right]$ existe para alguna $\alpha$ en una vecindad del 0, y de aqu\'i que $\esp\left[N\left(t\right)^{m}\right]<\infty$, para $m\geq1$.
\end{Prop}


\begin{Note}
Si el primer tiempo de renovaci\'on $\xi_{1}$ no tiene la misma distribuci\'on que el resto de las $\xi_{n}$, para $n\geq2$, a $N\left(t\right)$ se le llama Proceso de Renovaci\'on retardado, donde si $\xi$ tiene distribuci\'on $G$, entonces el tiempo $T_{n}$ de la $n$-\'esima renovaci\'on tiene distribuci\'on $G\star F^{\left(n-1\right)\star}\left(t\right)$
\end{Note}


\begin{Teo}
Para una constante $\mu\leq\infty$ ( o variable aleatoria), las siguientes expresiones son equivalentes:

\begin{eqnarray}
lim_{n\rightarrow\infty}n^{-1}T_{n}&=&\mu,\textrm{ c.s.}\\
lim_{t\rightarrow\infty}t^{-1}N\left(t\right)&=&1/\mu,\textrm{ c.s.}
\end{eqnarray}
\end{Teo}


Es decir, $T_{n}$ satisface la Ley Fuerte de los Grandes N\'umeros s\'i y s\'olo s\'i $N\left/t\right)$ la cumple.


\begin{Coro}[Ley Fuerte de los Grandes N\'umeros para Procesos de Renovaci\'on]
Si $N\left(t\right)$ es un proceso de renovaci\'on cuyos tiempos de inter-renovaci\'on tienen media $\mu\leq\infty$, entonces
\begin{eqnarray}
t^{-1}N\left(t\right)\rightarrow 1/\mu,\textrm{ c.s. cuando }t\rightarrow\infty.
\end{eqnarray}

\end{Coro}


Considerar el proceso estoc\'astico de valores reales $\left\{Z\left(t\right):t\geq0\right\}$ en el mismo espacio de probabilidad que $N\left(t\right)$

\begin{Def}
Para el proceso $\left\{Z\left(t\right):t\geq0\right\}$ se define la fluctuaci\'on m\'axima de $Z\left(t\right)$ en el intervalo $\left(T_{n-1},T_{n}\right]$:
\begin{eqnarray*}
M_{n}=\sup_{T_{n-1}<t\leq T_{n}}|Z\left(t\right)-Z\left(T_{n-1}\right)|
\end{eqnarray*}
\end{Def}

\begin{Teo}
Sup\'ongase que $n^{-1}T_{n}\rightarrow\mu$ c.s. cuando $n\rightarrow\infty$, donde $\mu\leq\infty$ es una constante o variable aleatoria. Sea $a$ una constante o variable aleatoria que puede ser infinita cuando $\mu$ es finita, y considere las expresiones l\'imite:
\begin{eqnarray}
lim_{n\rightarrow\infty}n^{-1}Z\left(T_{n}\right)&=&a,\textrm{ c.s.}\\
lim_{t\rightarrow\infty}t^{-1}Z\left(t\right)&=&a/\mu,\textrm{ c.s.}
\end{eqnarray}
La segunda expresi\'on implica la primera. Conversamente, la primera implica la segunda si el proceso $Z\left(t\right)$ es creciente, o si $lim_{n\rightarrow\infty}n^{-1}M_{n}=0$ c.s.
\end{Teo}

\begin{Coro}
Si $N\left(t\right)$ es un proceso de renovaci\'on, y $\left(Z\left(T_{n}\right)-Z\left(T_{n-1}\right),M_{n}\right)$, para $n\geq1$, son variables aleatorias independientes e id\'enticamente distribuidas con media finita, entonces,
\begin{eqnarray}
lim_{t\rightarrow\infty}t^{-1}Z\left(t\right)\rightarrow\frac{\esp\left[Z\left(T_{1}\right)-Z\left(T_{0}\right)\right]}{\esp\left[T_{1}\right]},\textrm{ c.s. cuando  }t\rightarrow\infty.
\end{eqnarray}
\end{Coro}

%___________________________________________________________________________________________
%
\subsection{Propiedades de los Procesos de Renovaci\'on}
%___________________________________________________________________________________________
%

Los tiempos $T_{n}$ est\'an relacionados con los conteos de $N\left(t\right)$ por

\begin{eqnarray*}
\left\{N\left(t\right)\geq n\right\}&=&\left\{T_{n}\leq t\right\}\\
T_{N\left(t\right)}\leq &t&<T_{N\left(t\right)+1},
\end{eqnarray*}

adem\'as $N\left(T_{n}\right)=n$, y 

\begin{eqnarray*}
N\left(t\right)=\max\left\{n:T_{n}\leq t\right\}=\min\left\{n:T_{n+1}>t\right\}
\end{eqnarray*}

Por propiedades de la convoluci\'on se sabe que

\begin{eqnarray*}
P\left\{T_{n}\leq t\right\}=F^{n\star}\left(t\right)
\end{eqnarray*}
que es la $n$-\'esima convoluci\'on de $F$. Entonces 

\begin{eqnarray*}
\left\{N\left(t\right)\geq n\right\}&=&\left\{T_{n}\leq t\right\}\\
P\left\{N\left(t\right)\leq n\right\}&=&1-F^{\left(n+1\right)\star}\left(t\right)
\end{eqnarray*}

Adem\'as usando el hecho de que $\esp\left[N\left(t\right)\right]=\sum_{n=1}^{\infty}P\left\{N\left(t\right)\geq n\right\}$
se tiene que

\begin{eqnarray*}
\esp\left[N\left(t\right)\right]=\sum_{n=1}^{\infty}F^{n\star}\left(t\right)
\end{eqnarray*}

\begin{Prop}
Para cada $t\geq0$, la funci\'on generadora de momentos $\esp\left[e^{\alpha N\left(t\right)}\right]$ existe para alguna $\alpha$ en una vecindad del 0, y de aqu\'i que $\esp\left[N\left(t\right)^{m}\right]<\infty$, para $m\geq1$.
\end{Prop}


\begin{Note}
Si el primer tiempo de renovaci\'on $\xi_{1}$ no tiene la misma distribuci\'on que el resto de las $\xi_{n}$, para $n\geq2$, a $N\left(t\right)$ se le llama Proceso de Renovaci\'on retardado, donde si $\xi$ tiene distribuci\'on $G$, entonces el tiempo $T_{n}$ de la $n$-\'esima renovaci\'on tiene distribuci\'on $G\star F^{\left(n-1\right)\star}\left(t\right)$
\end{Note}


\begin{Teo}
Para una constante $\mu\leq\infty$ ( o variable aleatoria), las siguientes expresiones son equivalentes:

\begin{eqnarray}
lim_{n\rightarrow\infty}n^{-1}T_{n}&=&\mu,\textrm{ c.s.}\\
lim_{t\rightarrow\infty}t^{-1}N\left(t\right)&=&1/\mu,\textrm{ c.s.}
\end{eqnarray}
\end{Teo}


Es decir, $T_{n}$ satisface la Ley Fuerte de los Grandes N\'umeros s\'i y s\'olo s\'i $N\left/t\right)$ la cumple.


\begin{Coro}[Ley Fuerte de los Grandes N\'umeros para Procesos de Renovaci\'on]
Si $N\left(t\right)$ es un proceso de renovaci\'on cuyos tiempos de inter-renovaci\'on tienen media $\mu\leq\infty$, entonces
\begin{eqnarray}
t^{-1}N\left(t\right)\rightarrow 1/\mu,\textrm{ c.s. cuando }t\rightarrow\infty.
\end{eqnarray}

\end{Coro}


Considerar el proceso estoc\'astico de valores reales $\left\{Z\left(t\right):t\geq0\right\}$ en el mismo espacio de probabilidad que $N\left(t\right)$

\begin{Def}
Para el proceso $\left\{Z\left(t\right):t\geq0\right\}$ se define la fluctuaci\'on m\'axima de $Z\left(t\right)$ en el intervalo $\left(T_{n-1},T_{n}\right]$:
\begin{eqnarray*}
M_{n}=\sup_{T_{n-1}<t\leq T_{n}}|Z\left(t\right)-Z\left(T_{n-1}\right)|
\end{eqnarray*}
\end{Def}

\begin{Teo}
Sup\'ongase que $n^{-1}T_{n}\rightarrow\mu$ c.s. cuando $n\rightarrow\infty$, donde $\mu\leq\infty$ es una constante o variable aleatoria. Sea $a$ una constante o variable aleatoria que puede ser infinita cuando $\mu$ es finita, y considere las expresiones l\'imite:
\begin{eqnarray}
lim_{n\rightarrow\infty}n^{-1}Z\left(T_{n}\right)&=&a,\textrm{ c.s.}\\
lim_{t\rightarrow\infty}t^{-1}Z\left(t\right)&=&a/\mu,\textrm{ c.s.}
\end{eqnarray}
La segunda expresi\'on implica la primera. Conversamente, la primera implica la segunda si el proceso $Z\left(t\right)$ es creciente, o si $lim_{n\rightarrow\infty}n^{-1}M_{n}=0$ c.s.
\end{Teo}

\begin{Coro}
Si $N\left(t\right)$ es un proceso de renovaci\'on, y $\left(Z\left(T_{n}\right)-Z\left(T_{n-1}\right),M_{n}\right)$, para $n\geq1$, son variables aleatorias independientes e id\'enticamente distribuidas con media finita, entonces,
\begin{eqnarray}
lim_{t\rightarrow\infty}t^{-1}Z\left(t\right)\rightarrow\frac{\esp\left[Z\left(T_{1}\right)-Z\left(T_{0}\right)\right]}{\esp\left[T_{1}\right]},\textrm{ c.s. cuando  }t\rightarrow\infty.
\end{eqnarray}
\end{Coro}

%___________________________________________________________________________________________
%
\subsection{Propiedades de los Procesos de Renovaci\'on}
%___________________________________________________________________________________________
%

Los tiempos $T_{n}$ est\'an relacionados con los conteos de $N\left(t\right)$ por

\begin{eqnarray*}
\left\{N\left(t\right)\geq n\right\}&=&\left\{T_{n}\leq t\right\}\\
T_{N\left(t\right)}\leq &t&<T_{N\left(t\right)+1},
\end{eqnarray*}

adem\'as $N\left(T_{n}\right)=n$, y 

\begin{eqnarray*}
N\left(t\right)=\max\left\{n:T_{n}\leq t\right\}=\min\left\{n:T_{n+1}>t\right\}
\end{eqnarray*}

Por propiedades de la convoluci\'on se sabe que

\begin{eqnarray*}
P\left\{T_{n}\leq t\right\}=F^{n\star}\left(t\right)
\end{eqnarray*}
que es la $n$-\'esima convoluci\'on de $F$. Entonces 

\begin{eqnarray*}
\left\{N\left(t\right)\geq n\right\}&=&\left\{T_{n}\leq t\right\}\\
P\left\{N\left(t\right)\leq n\right\}&=&1-F^{\left(n+1\right)\star}\left(t\right)
\end{eqnarray*}

Adem\'as usando el hecho de que $\esp\left[N\left(t\right)\right]=\sum_{n=1}^{\infty}P\left\{N\left(t\right)\geq n\right\}$
se tiene que

\begin{eqnarray*}
\esp\left[N\left(t\right)\right]=\sum_{n=1}^{\infty}F^{n\star}\left(t\right)
\end{eqnarray*}

\begin{Prop}
Para cada $t\geq0$, la funci\'on generadora de momentos $\esp\left[e^{\alpha N\left(t\right)}\right]$ existe para alguna $\alpha$ en una vecindad del 0, y de aqu\'i que $\esp\left[N\left(t\right)^{m}\right]<\infty$, para $m\geq1$.
\end{Prop}


\begin{Note}
Si el primer tiempo de renovaci\'on $\xi_{1}$ no tiene la misma distribuci\'on que el resto de las $\xi_{n}$, para $n\geq2$, a $N\left(t\right)$ se le llama Proceso de Renovaci\'on retardado, donde si $\xi$ tiene distribuci\'on $G$, entonces el tiempo $T_{n}$ de la $n$-\'esima renovaci\'on tiene distribuci\'on $G\star F^{\left(n-1\right)\star}\left(t\right)$
\end{Note}


\begin{Teo}
Para una constante $\mu\leq\infty$ ( o variable aleatoria), las siguientes expresiones son equivalentes:

\begin{eqnarray}
lim_{n\rightarrow\infty}n^{-1}T_{n}&=&\mu,\textrm{ c.s.}\\
lim_{t\rightarrow\infty}t^{-1}N\left(t\right)&=&1/\mu,\textrm{ c.s.}
\end{eqnarray}
\end{Teo}


Es decir, $T_{n}$ satisface la Ley Fuerte de los Grandes N\'umeros s\'i y s\'olo s\'i $N\left/t\right)$ la cumple.


\begin{Coro}[Ley Fuerte de los Grandes N\'umeros para Procesos de Renovaci\'on]
Si $N\left(t\right)$ es un proceso de renovaci\'on cuyos tiempos de inter-renovaci\'on tienen media $\mu\leq\infty$, entonces
\begin{eqnarray}
t^{-1}N\left(t\right)\rightarrow 1/\mu,\textrm{ c.s. cuando }t\rightarrow\infty.
\end{eqnarray}

\end{Coro}


Considerar el proceso estoc\'astico de valores reales $\left\{Z\left(t\right):t\geq0\right\}$ en el mismo espacio de probabilidad que $N\left(t\right)$

\begin{Def}
Para el proceso $\left\{Z\left(t\right):t\geq0\right\}$ se define la fluctuaci\'on m\'axima de $Z\left(t\right)$ en el intervalo $\left(T_{n-1},T_{n}\right]$:
\begin{eqnarray*}
M_{n}=\sup_{T_{n-1}<t\leq T_{n}}|Z\left(t\right)-Z\left(T_{n-1}\right)|
\end{eqnarray*}
\end{Def}

\begin{Teo}
Sup\'ongase que $n^{-1}T_{n}\rightarrow\mu$ c.s. cuando $n\rightarrow\infty$, donde $\mu\leq\infty$ es una constante o variable aleatoria. Sea $a$ una constante o variable aleatoria que puede ser infinita cuando $\mu$ es finita, y considere las expresiones l\'imite:
\begin{eqnarray}
lim_{n\rightarrow\infty}n^{-1}Z\left(T_{n}\right)&=&a,\textrm{ c.s.}\\
lim_{t\rightarrow\infty}t^{-1}Z\left(t\right)&=&a/\mu,\textrm{ c.s.}
\end{eqnarray}
La segunda expresi\'on implica la primera. Conversamente, la primera implica la segunda si el proceso $Z\left(t\right)$ es creciente, o si $lim_{n\rightarrow\infty}n^{-1}M_{n}=0$ c.s.
\end{Teo}

\begin{Coro}
Si $N\left(t\right)$ es un proceso de renovaci\'on, y $\left(Z\left(T_{n}\right)-Z\left(T_{n-1}\right),M_{n}\right)$, para $n\geq1$, son variables aleatorias independientes e id\'enticamente distribuidas con media finita, entonces,
\begin{eqnarray}
lim_{t\rightarrow\infty}t^{-1}Z\left(t\right)\rightarrow\frac{\esp\left[Z\left(T_{1}\right)-Z\left(T_{0}\right)\right]}{\esp\left[T_{1}\right]},\textrm{ c.s. cuando  }t\rightarrow\infty.
\end{eqnarray}
\end{Coro}


%__________________________________________________________________________________________
\subsection{Procesos Regenerativos Estacionarios - Stidham \cite{Stidham}}
%__________________________________________________________________________________________


Un proceso estoc\'astico a tiempo continuo $\left\{V\left(t\right),t\geq0\right\}$ es un proceso regenerativo si existe una sucesi\'on de variables aleatorias independientes e id\'enticamente distribuidas $\left\{X_{1},X_{2},\ldots\right\}$, sucesi\'on de renovaci\'on, tal que para cualquier conjunto de Borel $A$, 

\begin{eqnarray*}
\prob\left\{V\left(t\right)\in A|X_{1}+X_{2}+\cdots+X_{R\left(t\right)}=s,\left\{V\left(\tau\right),\tau<s\right\}\right\}=\prob\left\{V\left(t-s\right)\in A|X_{1}>t-s\right\},
\end{eqnarray*}
para todo $0\leq s\leq t$, donde $R\left(t\right)=\max\left\{X_{1}+X_{2}+\cdots+X_{j}\leq t\right\}=$n\'umero de renovaciones ({\emph{puntos de regeneraci\'on}}) que ocurren en $\left[0,t\right]$. El intervalo $\left[0,X_{1}\right)$ es llamado {\emph{primer ciclo de regeneraci\'on}} de $\left\{V\left(t \right),t\geq0\right\}$, $\left[X_{1},X_{1}+X_{2}\right)$ el {\emph{segundo ciclo de regeneraci\'on}}, y as\'i sucesivamente.

Sea $X=X_{1}$ y sea $F$ la funci\'on de distrbuci\'on de $X$


\begin{Def}
Se define el proceso estacionario, $\left\{V^{*}\left(t\right),t\geq0\right\}$, para $\left\{V\left(t\right),t\geq0\right\}$ por

\begin{eqnarray*}
\prob\left\{V\left(t\right)\in A\right\}=\frac{1}{\esp\left[X\right]}\int_{0}^{\infty}\prob\left\{V\left(t+x\right)\in A|X>x\right\}\left(1-F\left(x\right)\right)dx,
\end{eqnarray*} 
para todo $t\geq0$ y todo conjunto de Borel $A$.
\end{Def}

\begin{Def}
Una distribuci\'on se dice que es {\emph{aritm\'etica}} si todos sus puntos de incremento son m\'ultiplos de la forma $0,\lambda, 2\lambda,\ldots$ para alguna $\lambda>0$ entera.
\end{Def}


\begin{Def}
Una modificaci\'on medible de un proceso $\left\{V\left(t\right),t\geq0\right\}$, es una versi\'on de este, $\left\{V\left(t,w\right)\right\}$ conjuntamente medible para $t\geq0$ y para $w\in S$, $S$ espacio de estados para $\left\{V\left(t\right),t\geq0\right\}$.
\end{Def}

\begin{Teo}
Sea $\left\{V\left(t\right),t\geq\right\}$ un proceso regenerativo no negativo con modificaci\'on medible. Sea $\esp\left[X\right]<\infty$. Entonces el proceso estacionario dado por la ecuaci\'on anterior est\'a bien definido y tiene funci\'on de distribuci\'on independiente de $t$, adem\'as
\begin{itemize}
\item[i)] \begin{eqnarray*}
\esp\left[V^{*}\left(0\right)\right]&=&\frac{\esp\left[\int_{0}^{X}V\left(s\right)ds\right]}{\esp\left[X\right]}\end{eqnarray*}
\item[ii)] Si $\esp\left[V^{*}\left(0\right)\right]<\infty$, equivalentemente, si $\esp\left[\int_{0}^{X}V\left(s\right)ds\right]<\infty$,entonces
\begin{eqnarray*}
\frac{\int_{0}^{t}V\left(s\right)ds}{t}\rightarrow\frac{\esp\left[\int_{0}^{X}V\left(s\right)ds\right]}{\esp\left[X\right]}
\end{eqnarray*}
con probabilidad 1 y en media, cuando $t\rightarrow\infty$.
\end{itemize}
\end{Teo}

%______________________________________________________________________
\subsection{Procesos de Renovaci\'on}
%______________________________________________________________________

\begin{Def}\label{Def.Tn}
Sean $0\leq T_{1}\leq T_{2}\leq \ldots$ son tiempos aleatorios infinitos en los cuales ocurren ciertos eventos. El n\'umero de tiempos $T_{n}$ en el intervalo $\left[0,t\right)$ es

\begin{eqnarray}
N\left(t\right)=\sum_{n=1}^{\infty}\indora\left(T_{n}\leq t\right),
\end{eqnarray}
para $t\geq0$.
\end{Def}

Si se consideran los puntos $T_{n}$ como elementos de $\rea_{+}$, y $N\left(t\right)$ es el n\'umero de puntos en $\rea$. El proceso denotado por $\left\{N\left(t\right):t\geq0\right\}$, denotado por $N\left(t\right)$, es un proceso puntual en $\rea_{+}$. Los $T_{n}$ son los tiempos de ocurrencia, el proceso puntual $N\left(t\right)$ es simple si su n\'umero de ocurrencias son distintas: $0<T_{1}<T_{2}<\ldots$ casi seguramente.

\begin{Def}
Un proceso puntual $N\left(t\right)$ es un proceso de renovaci\'on si los tiempos de interocurrencia $\xi_{n}=T_{n}-T_{n-1}$, para $n\geq1$, son independientes e identicamente distribuidos con distribuci\'on $F$, donde $F\left(0\right)=0$ y $T_{0}=0$. Los $T_{n}$ son llamados tiempos de renovaci\'on, referente a la independencia o renovaci\'on de la informaci\'on estoc\'astica en estos tiempos. Los $\xi_{n}$ son los tiempos de inter-renovaci\'on, y $N\left(t\right)$ es el n\'umero de renovaciones en el intervalo $\left[0,t\right)$
\end{Def}


\begin{Note}
Para definir un proceso de renovaci\'on para cualquier contexto, solamente hay que especificar una distribuci\'on $F$, con $F\left(0\right)=0$, para los tiempos de inter-renovaci\'on. La funci\'on $F$ en turno degune las otra variables aleatorias. De manera formal, existe un espacio de probabilidad y una sucesi\'on de variables aleatorias $\xi_{1},\xi_{2},\ldots$ definidas en este con distribuci\'on $F$. Entonces las otras cantidades son $T_{n}=\sum_{k=1}^{n}\xi_{k}$ y $N\left(t\right)=\sum_{n=1}^{\infty}\indora\left(T_{n}\leq t\right)$, donde $T_{n}\rightarrow\infty$ casi seguramente por la Ley Fuerte de los Grandes Números.
\end{Note}

%___________________________________________________________________________________________
%
\subsection{Teorema Principal de Renovaci\'on}
%___________________________________________________________________________________________
%

\begin{Note} Una funci\'on $h:\rea_{+}\rightarrow\rea$ es Directamente Riemann Integrable en los siguientes casos:
\begin{itemize}
\item[a)] $h\left(t\right)\geq0$ es decreciente y Riemann Integrable.
\item[b)] $h$ es continua excepto posiblemente en un conjunto de Lebesgue de medida 0, y $|h\left(t\right)|\leq b\left(t\right)$, donde $b$ es DRI.
\end{itemize}
\end{Note}

\begin{Teo}[Teorema Principal de Renovaci\'on]
Si $F$ es no aritm\'etica y $h\left(t\right)$ es Directamente Riemann Integrable (DRI), entonces

\begin{eqnarray*}
lim_{t\rightarrow\infty}U\star h=\frac{1}{\mu}\int_{\rea_{+}}h\left(s\right)ds.
\end{eqnarray*}
\end{Teo}

\begin{Prop}
Cualquier funci\'on $H\left(t\right)$ acotada en intervalos finitos y que es 0 para $t<0$ puede expresarse como
\begin{eqnarray*}
H\left(t\right)=U\star h\left(t\right)\textrm{,  donde }h\left(t\right)=H\left(t\right)-F\star H\left(t\right)
\end{eqnarray*}
\end{Prop}

\begin{Def}
Un proceso estoc\'astico $X\left(t\right)$ es crudamente regenerativo en un tiempo aleatorio positivo $T$ si
\begin{eqnarray*}
\esp\left[X\left(T+t\right)|T\right]=\esp\left[X\left(t\right)\right]\textrm{, para }t\geq0,\end{eqnarray*}
y con las esperanzas anteriores finitas.
\end{Def}

\begin{Prop}
Sup\'ongase que $X\left(t\right)$ es un proceso crudamente regenerativo en $T$, que tiene distribuci\'on $F$. Si $\esp\left[X\left(t\right)\right]$ es acotado en intervalos finitos, entonces
\begin{eqnarray*}
\esp\left[X\left(t\right)\right]=U\star h\left(t\right)\textrm{,  donde }h\left(t\right)=\esp\left[X\left(t\right)\indora\left(T>t\right)\right].
\end{eqnarray*}
\end{Prop}

\begin{Teo}[Regeneraci\'on Cruda]
Sup\'ongase que $X\left(t\right)$ es un proceso con valores positivo crudamente regenerativo en $T$, y def\'inase $M=\sup\left\{|X\left(t\right)|:t\leq T\right\}$. Si $T$ es no aritm\'etico y $M$ y $MT$ tienen media finita, entonces
\begin{eqnarray*}
lim_{t\rightarrow\infty}\esp\left[X\left(t\right)\right]=\frac{1}{\mu}\int_{\rea_{+}}h\left(s\right)ds,
\end{eqnarray*}
donde $h\left(t\right)=\esp\left[X\left(t\right)\indora\left(T>t\right)\right]$.
\end{Teo}



%___________________________________________________________________________________________
%
\subsection{Funci\'on de Renovaci\'on}
%___________________________________________________________________________________________
%


\begin{Def}
Sea $h\left(t\right)$ funci\'on de valores reales en $\rea$ acotada en intervalos finitos e igual a cero para $t<0$ La ecuaci\'on de renovaci\'on para $h\left(t\right)$ y la distribuci\'on $F$ es

\begin{eqnarray}\label{Ec.Renovacion}
H\left(t\right)=h\left(t\right)+\int_{\left[0,t\right]}H\left(t-s\right)dF\left(s\right)\textrm{,    }t\geq0,
\end{eqnarray}
donde $H\left(t\right)$ es una funci\'on de valores reales. Esto es $H=h+F\star H$. Decimos que $H\left(t\right)$ es soluci\'on de esta ecuaci\'on si satisface la ecuaci\'on, y es acotada en intervalos finitos e iguales a cero para $t<0$.
\end{Def}

\begin{Prop}
La funci\'on $U\star h\left(t\right)$ es la \'unica soluci\'on de la ecuaci\'on de renovaci\'on (\ref{Ec.Renovacion}).
\end{Prop}

\begin{Teo}[Teorema Renovaci\'on Elemental]
\begin{eqnarray*}
t^{-1}U\left(t\right)\rightarrow 1/\mu\textrm{,    cuando }t\rightarrow\infty.
\end{eqnarray*}
\end{Teo}

%___________________________________________________________________________________________
%
\subsection{Propiedades de los Procesos de Renovaci\'on}
%___________________________________________________________________________________________
%

Los tiempos $T_{n}$ est\'an relacionados con los conteos de $N\left(t\right)$ por

\begin{eqnarray*}
\left\{N\left(t\right)\geq n\right\}&=&\left\{T_{n}\leq t\right\}\\
T_{N\left(t\right)}\leq &t&<T_{N\left(t\right)+1},
\end{eqnarray*}

adem\'as $N\left(T_{n}\right)=n$, y 

\begin{eqnarray*}
N\left(t\right)=\max\left\{n:T_{n}\leq t\right\}=\min\left\{n:T_{n+1}>t\right\}
\end{eqnarray*}

Por propiedades de la convoluci\'on se sabe que

\begin{eqnarray*}
P\left\{T_{n}\leq t\right\}=F^{n\star}\left(t\right)
\end{eqnarray*}
que es la $n$-\'esima convoluci\'on de $F$. Entonces 

\begin{eqnarray*}
\left\{N\left(t\right)\geq n\right\}&=&\left\{T_{n}\leq t\right\}\\
P\left\{N\left(t\right)\leq n\right\}&=&1-F^{\left(n+1\right)\star}\left(t\right)
\end{eqnarray*}

Adem\'as usando el hecho de que $\esp\left[N\left(t\right)\right]=\sum_{n=1}^{\infty}P\left\{N\left(t\right)\geq n\right\}$
se tiene que

\begin{eqnarray*}
\esp\left[N\left(t\right)\right]=\sum_{n=1}^{\infty}F^{n\star}\left(t\right)
\end{eqnarray*}

\begin{Prop}
Para cada $t\geq0$, la funci\'on generadora de momentos $\esp\left[e^{\alpha N\left(t\right)}\right]$ existe para alguna $\alpha$ en una vecindad del 0, y de aqu\'i que $\esp\left[N\left(t\right)^{m}\right]<\infty$, para $m\geq1$.
\end{Prop}


\begin{Note}
Si el primer tiempo de renovaci\'on $\xi_{1}$ no tiene la misma distribuci\'on que el resto de las $\xi_{n}$, para $n\geq2$, a $N\left(t\right)$ se le llama Proceso de Renovaci\'on retardado, donde si $\xi$ tiene distribuci\'on $G$, entonces el tiempo $T_{n}$ de la $n$-\'esima renovaci\'on tiene distribuci\'on $G\star F^{\left(n-1\right)\star}\left(t\right)$
\end{Note}


\begin{Teo}
Para una constante $\mu\leq\infty$ ( o variable aleatoria), las siguientes expresiones son equivalentes:

\begin{eqnarray}
lim_{n\rightarrow\infty}n^{-1}T_{n}&=&\mu,\textrm{ c.s.}\\
lim_{t\rightarrow\infty}t^{-1}N\left(t\right)&=&1/\mu,\textrm{ c.s.}
\end{eqnarray}
\end{Teo}


Es decir, $T_{n}$ satisface la Ley Fuerte de los Grandes N\'umeros s\'i y s\'olo s\'i $N\left/t\right)$ la cumple.


\begin{Coro}[Ley Fuerte de los Grandes N\'umeros para Procesos de Renovaci\'on]
Si $N\left(t\right)$ es un proceso de renovaci\'on cuyos tiempos de inter-renovaci\'on tienen media $\mu\leq\infty$, entonces
\begin{eqnarray}
t^{-1}N\left(t\right)\rightarrow 1/\mu,\textrm{ c.s. cuando }t\rightarrow\infty.
\end{eqnarray}

\end{Coro}


Considerar el proceso estoc\'astico de valores reales $\left\{Z\left(t\right):t\geq0\right\}$ en el mismo espacio de probabilidad que $N\left(t\right)$

\begin{Def}
Para el proceso $\left\{Z\left(t\right):t\geq0\right\}$ se define la fluctuaci\'on m\'axima de $Z\left(t\right)$ en el intervalo $\left(T_{n-1},T_{n}\right]$:
\begin{eqnarray*}
M_{n}=\sup_{T_{n-1}<t\leq T_{n}}|Z\left(t\right)-Z\left(T_{n-1}\right)|
\end{eqnarray*}
\end{Def}

\begin{Teo}
Sup\'ongase que $n^{-1}T_{n}\rightarrow\mu$ c.s. cuando $n\rightarrow\infty$, donde $\mu\leq\infty$ es una constante o variable aleatoria. Sea $a$ una constante o variable aleatoria que puede ser infinita cuando $\mu$ es finita, y considere las expresiones l\'imite:
\begin{eqnarray}
lim_{n\rightarrow\infty}n^{-1}Z\left(T_{n}\right)&=&a,\textrm{ c.s.}\\
lim_{t\rightarrow\infty}t^{-1}Z\left(t\right)&=&a/\mu,\textrm{ c.s.}
\end{eqnarray}
La segunda expresi\'on implica la primera. Conversamente, la primera implica la segunda si el proceso $Z\left(t\right)$ es creciente, o si $lim_{n\rightarrow\infty}n^{-1}M_{n}=0$ c.s.
\end{Teo}

\begin{Coro}
Si $N\left(t\right)$ es un proceso de renovaci\'on, y $\left(Z\left(T_{n}\right)-Z\left(T_{n-1}\right),M_{n}\right)$, para $n\geq1$, son variables aleatorias independientes e id\'enticamente distribuidas con media finita, entonces,
\begin{eqnarray}
lim_{t\rightarrow\infty}t^{-1}Z\left(t\right)\rightarrow\frac{\esp\left[Z\left(T_{1}\right)-Z\left(T_{0}\right)\right]}{\esp\left[T_{1}\right]},\textrm{ c.s. cuando  }t\rightarrow\infty.
\end{eqnarray}
\end{Coro}

%___________________________________________________________________________________________
%
\subsection{Funci\'on de Renovaci\'on}
%___________________________________________________________________________________________
%


Sup\'ongase que $N\left(t\right)$ es un proceso de renovaci\'on con distribuci\'on $F$ con media finita $\mu$.

\begin{Def}
La funci\'on de renovaci\'on asociada con la distribuci\'on $F$, del proceso $N\left(t\right)$, es
\begin{eqnarray*}
U\left(t\right)=\sum_{n=1}^{\infty}F^{n\star}\left(t\right),\textrm{   }t\geq0,
\end{eqnarray*}
donde $F^{0\star}\left(t\right)=\indora\left(t\geq0\right)$.
\end{Def}


\begin{Prop}
Sup\'ongase que la distribuci\'on de inter-renovaci\'on $F$ tiene densidad $f$. Entonces $U\left(t\right)$ tambi\'en tiene densidad, para $t>0$, y es $U^{'}\left(t\right)=\sum_{n=0}^{\infty}f^{n\star}\left(t\right)$. Adem\'as
\begin{eqnarray*}
\prob\left\{N\left(t\right)>N\left(t-\right)\right\}=0\textrm{,   }t\geq0.
\end{eqnarray*}
\end{Prop}

\begin{Def}
La Transformada de Laplace-Stieljes de $F$ est\'a dada por

\begin{eqnarray*}
\hat{F}\left(\alpha\right)=\int_{\rea_{+}}e^{-\alpha t}dF\left(t\right)\textrm{,  }\alpha\geq0.
\end{eqnarray*}
\end{Def}

Entonces

\begin{eqnarray*}
\hat{U}\left(\alpha\right)=\sum_{n=0}^{\infty}\hat{F^{n\star}}\left(\alpha\right)=\sum_{n=0}^{\infty}\hat{F}\left(\alpha\right)^{n}=\frac{1}{1-\hat{F}\left(\alpha\right)}.
\end{eqnarray*}


\begin{Prop}
La Transformada de Laplace $\hat{U}\left(\alpha\right)$ y $\hat{F}\left(\alpha\right)$ determina una a la otra de manera \'unica por la relaci\'on $\hat{U}\left(\alpha\right)=\frac{1}{1-\hat{F}\left(\alpha\right)}$.
\end{Prop}


\begin{Note}
Un proceso de renovaci\'on $N\left(t\right)$ cuyos tiempos de inter-renovaci\'on tienen media finita, es un proceso Poisson con tasa $\lambda$ si y s\'olo s\'i $\esp\left[U\left(t\right)\right]=\lambda t$, para $t\geq0$.
\end{Note}


\begin{Teo}
Sea $N\left(t\right)$ un proceso puntual simple con puntos de localizaci\'on $T_{n}$ tal que $\eta\left(t\right)=\esp\left[N\left(\right)\right]$ es finita para cada $t$. Entonces para cualquier funci\'on $f:\rea_{+}\rightarrow\rea$,
\begin{eqnarray*}
\esp\left[\sum_{n=1}^{N\left(\right)}f\left(T_{n}\right)\right]=\int_{\left(0,t\right]}f\left(s\right)d\eta\left(s\right)\textrm{,  }t\geq0,
\end{eqnarray*}
suponiendo que la integral exista. Adem\'as si $X_{1},X_{2},\ldots$ son variables aleatorias definidas en el mismo espacio de probabilidad que el proceso $N\left(t\right)$ tal que $\esp\left[X_{n}|T_{n}=s\right]=f\left(s\right)$, independiente de $n$. Entonces
\begin{eqnarray*}
\esp\left[\sum_{n=1}^{N\left(t\right)}X_{n}\right]=\int_{\left(0,t\right]}f\left(s\right)d\eta\left(s\right)\textrm{,  }t\geq0,
\end{eqnarray*} 
suponiendo que la integral exista. 
\end{Teo}

\begin{Coro}[Identidad de Wald para Renovaciones]
Para el proceso de renovaci\'on $N\left(t\right)$,
\begin{eqnarray*}
\esp\left[T_{N\left(t\right)+1}\right]=\mu\esp\left[N\left(t\right)+1\right]\textrm{,  }t\geq0,
\end{eqnarray*}  
\end{Coro}

%______________________________________________________________________
\subsection{Procesos de Renovaci\'on}
%______________________________________________________________________

\begin{Def}\label{Def.Tn}
Sean $0\leq T_{1}\leq T_{2}\leq \ldots$ son tiempos aleatorios infinitos en los cuales ocurren ciertos eventos. El n\'umero de tiempos $T_{n}$ en el intervalo $\left[0,t\right)$ es

\begin{eqnarray}
N\left(t\right)=\sum_{n=1}^{\infty}\indora\left(T_{n}\leq t\right),
\end{eqnarray}
para $t\geq0$.
\end{Def}

Si se consideran los puntos $T_{n}$ como elementos de $\rea_{+}$, y $N\left(t\right)$ es el n\'umero de puntos en $\rea$. El proceso denotado por $\left\{N\left(t\right):t\geq0\right\}$, denotado por $N\left(t\right)$, es un proceso puntual en $\rea_{+}$. Los $T_{n}$ son los tiempos de ocurrencia, el proceso puntual $N\left(t\right)$ es simple si su n\'umero de ocurrencias son distintas: $0<T_{1}<T_{2}<\ldots$ casi seguramente.

\begin{Def}
Un proceso puntual $N\left(t\right)$ es un proceso de renovaci\'on si los tiempos de interocurrencia $\xi_{n}=T_{n}-T_{n-1}$, para $n\geq1$, son independientes e identicamente distribuidos con distribuci\'on $F$, donde $F\left(0\right)=0$ y $T_{0}=0$. Los $T_{n}$ son llamados tiempos de renovaci\'on, referente a la independencia o renovaci\'on de la informaci\'on estoc\'astica en estos tiempos. Los $\xi_{n}$ son los tiempos de inter-renovaci\'on, y $N\left(t\right)$ es el n\'umero de renovaciones en el intervalo $\left[0,t\right)$
\end{Def}


\begin{Note}
Para definir un proceso de renovaci\'on para cualquier contexto, solamente hay que especificar una distribuci\'on $F$, con $F\left(0\right)=0$, para los tiempos de inter-renovaci\'on. La funci\'on $F$ en turno degune las otra variables aleatorias. De manera formal, existe un espacio de probabilidad y una sucesi\'on de variables aleatorias $\xi_{1},\xi_{2},\ldots$ definidas en este con distribuci\'on $F$. Entonces las otras cantidades son $T_{n}=\sum_{k=1}^{n}\xi_{k}$ y $N\left(t\right)=\sum_{n=1}^{\infty}\indora\left(T_{n}\leq t\right)$, donde $T_{n}\rightarrow\infty$ casi seguramente por la Ley Fuerte de los Grandes Números.
\end{Note}
%_____________________________________________________
\subsection{Puntos de Renovaci\'on}
%_____________________________________________________

Para cada cola $Q_{i}$ se tienen los procesos de arribo a la cola, para estas, los tiempos de arribo est\'an dados por $$\left\{T_{1}^{i},T_{2}^{i},\ldots,T_{k}^{i},\ldots\right\},$$ entonces, consideremos solamente los primeros tiempos de arribo a cada una de las colas, es decir, $$\left\{T_{1}^{1},T_{1}^{2},T_{1}^{3},T_{1}^{4}\right\},$$ se sabe que cada uno de estos tiempos se distribuye de manera exponencial con par\'ametro $1/mu_{i}$. Adem\'as se sabe que para $$T^{*}=\min\left\{T_{1}^{1},T_{1}^{2},T_{1}^{3},T_{1}^{4}\right\},$$ $T^{*}$ se distribuye de manera exponencial con par\'ametro $$\mu^{*}=\sum_{i=1}^{4}\mu_{i}.$$ Ahora, dado que 
\begin{center}
\begin{tabular}{lcl}
$\tilde{r}=r_{1}+r_{2}$ & y &$\hat{r}=r_{3}+r_{4}.$
\end{tabular}
\end{center}


Supongamos que $$\tilde{r},\hat{r}<\mu^{*},$$ entonces si tomamos $$r^{*}=\min\left\{\tilde{r},\hat{r}\right\},$$ se tiene que para  $$t^{*}\in\left(0,r^{*}\right)$$ se cumple que 
\begin{center}
\begin{tabular}{lcl}
$\tau_{1}\left(1\right)=0$ & y por tanto & $\overline{\tau}_{1}=0,$
\end{tabular}
\end{center}
entonces para la segunda cola en este primer ciclo se cumple que $$\tau_{2}=\overline{\tau}_{1}+r_{1}=r_{1}<\mu^{*},$$ y por tanto se tiene que  $$\overline{\tau}_{2}=\tau_{2}.$$ Por lo tanto, nuevamente para la primer cola en el segundo ciclo $$\tau_{1}\left(2\right)=\tau_{2}\left(1\right)+r_{2}=\tilde{r}<\mu^{*}.$$ An\'alogamente para el segundo sistema se tiene que ambas colas est\'an vac\'ias, es decir, existe un valor $t^{*}$ tal que en el intervalo $\left(0,t^{*}\right)$ no ha llegado ning\'un usuario, es decir, $$L_{i}\left(t^{*}\right)=0$$ para $i=1,2,3,4$.



%________________________________________________________________________
\subsection{Procesos Regenerativos}
%________________________________________________________________________

Para $\left\{X\left(t\right):t\geq0\right\}$ Proceso Estoc\'astico a tiempo continuo con estado de espacios $S$, que es un espacio m\'etrico, con trayectorias continuas por la derecha y con l\'imites por la izquierda c.s. Sea $N\left(t\right)$ un proceso de renovaci\'on en $\rea_{+}$ definido en el mismo espacio de probabilidad que $X\left(t\right)$, con tiempos de renovaci\'on $T$ y tiempos de inter-renovaci\'on $\xi_{n}=T_{n}-T_{n-1}$, con misma distribuci\'on $F$ de media finita $\mu$.



\begin{Def}
Para el proceso $\left\{\left(N\left(t\right),X\left(t\right)\right):t\geq0\right\}$, sus trayectoria muestrales en el intervalo de tiempo $\left[T_{n-1},T_{n}\right)$ est\'an descritas por
\begin{eqnarray*}
\zeta_{n}=\left(\xi_{n},\left\{X\left(T_{n-1}+t\right):0\leq t<\xi_{n}\right\}\right)
\end{eqnarray*}
Este $\zeta_{n}$ es el $n$-\'esimo segmento del proceso. El proceso es regenerativo sobre los tiempos $T_{n}$ si sus segmentos $\zeta_{n}$ son independientes e id\'enticamennte distribuidos.
\end{Def}


\begin{Obs}
Si $\tilde{X}\left(t\right)$ con espacio de estados $\tilde{S}$ es regenerativo sobre $T_{n}$, entonces $X\left(t\right)=f\left(\tilde{X}\left(t\right)\right)$ tambi\'en es regenerativo sobre $T_{n}$, para cualquier funci\'on $f:\tilde{S}\rightarrow S$.
\end{Obs}

\begin{Obs}
Los procesos regenerativos son crudamente regenerativos, pero no al rev\'es.
\end{Obs}

\begin{Def}[Definici\'on Cl\'asica]
Un proceso estoc\'astico $X=\left\{X\left(t\right):t\geq0\right\}$ es llamado regenerativo is existe una variable aleatoria $R_{1}>0$ tal que
\begin{itemize}
\item[i)] $\left\{X\left(t+R_{1}\right):t\geq0\right\}$ es independiente de $\left\{\left\{X\left(t\right):t<R_{1}\right\},\right\}$
\item[ii)] $\left\{X\left(t+R_{1}\right):t\geq0\right\}$ es estoc\'asticamente equivalente a $\left\{X\left(t\right):t>0\right\}$
\end{itemize}

Llamamos a $R_{1}$ tiempo de regeneraci\'on, y decimos que $X$ se regenera en este punto.
\end{Def}

$\left\{X\left(t+R_{1}\right)\right\}$ es regenerativo con tiempo de regeneraci\'on $R_{2}$, independiente de $R_{1}$ pero con la misma distribuci\'on que $R_{1}$. Procediendo de esta manera se obtiene una secuencia de variables aleatorias independientes e id\'enticamente distribuidas $\left\{R_{n}\right\}$ llamados longitudes de ciclo. Si definimos a $Z_{k}\equiv R_{1}+R_{2}+\cdots+R_{k}$, se tiene un proceso de renovaci\'on llamado proceso de renovaci\'on encajado para $X$.

\begin{Note}
Un proceso regenerativo con media de la longitud de ciclo finita es llamado positivo recurrente.
\end{Note}


\begin{Def}
Para $x$ fijo y para cada $t\geq0$, sea $I_{x}\left(t\right)=1$ si $X\left(t\right)\leq x$,  $I_{x}\left(t\right)=0$ en caso contrario, y def\'inanse los tiempos promedio
\begin{eqnarray*}
\overline{X}&=&lim_{t\rightarrow\infty}\frac{1}{t}\int_{0}^{\infty}X\left(u\right)du\\
\prob\left(X_{\infty}\leq x\right)&=&lim_{t\rightarrow\infty}\frac{1}{t}\int_{0}^{\infty}I_{x}\left(u\right)du,
\end{eqnarray*}
cuando estos l\'imites existan.
\end{Def}

Como consecuencia del teorema de Renovaci\'on-Recompensa, se tiene que el primer l\'imite  existe y es igual a la constante
\begin{eqnarray*}
\overline{X}&=&\frac{\esp\left[\int_{0}^{R_{1}}X\left(t\right)dt\right]}{\esp\left[R_{1}\right]},
\end{eqnarray*}
suponiendo que ambas esperanzas son finitas.

\begin{Note}
\begin{itemize}
\item[a)] Si el proceso regenerativo $X$ es positivo recurrente y tiene trayectorias muestrales no negativas, entonces la ecuaci\'on anterior es v\'alida.
\item[b)] Si $X$ es positivo recurrente regenerativo, podemos construir una \'unica versi\'on estacionaria de este proceso, $X_{e}=\left\{X_{e}\left(t\right)\right\}$, donde $X_{e}$ es un proceso estoc\'astico regenerativo y estrictamente estacionario, con distribuci\'on marginal distribuida como $X_{\infty}$
\end{itemize}
\end{Note}

\subsection{Renewal and Regenerative Processes: Serfozo\cite{Serfozo}}
\begin{Def}\label{Def.Tn}
Sean $0\leq T_{1}\leq T_{2}\leq \ldots$ son tiempos aleatorios infinitos en los cuales ocurren ciertos eventos. El n\'umero de tiempos $T_{n}$ en el intervalo $\left[0,t\right)$ es

\begin{eqnarray}
N\left(t\right)=\sum_{n=1}^{\infty}\indora\left(T_{n}\leq t\right),
\end{eqnarray}
para $t\geq0$.
\end{Def}

Si se consideran los puntos $T_{n}$ como elementos de $\rea_{+}$, y $N\left(t\right)$ es el n\'umero de puntos en $\rea$. El proceso denotado por $\left\{N\left(t\right):t\geq0\right\}$, denotado por $N\left(t\right)$, es un proceso puntual en $\rea_{+}$. Los $T_{n}$ son los tiempos de ocurrencia, el proceso puntual $N\left(t\right)$ es simple si su n\'umero de ocurrencias son distintas: $0<T_{1}<T_{2}<\ldots$ casi seguramente.

\begin{Def}
Un proceso puntual $N\left(t\right)$ es un proceso de renovaci\'on si los tiempos de interocurrencia $\xi_{n}=T_{n}-T_{n-1}$, para $n\geq1$, son independientes e identicamente distribuidos con distribuci\'on $F$, donde $F\left(0\right)=0$ y $T_{0}=0$. Los $T_{n}$ son llamados tiempos de renovaci\'on, referente a la independencia o renovaci\'on de la informaci\'on estoc\'astica en estos tiempos. Los $\xi_{n}$ son los tiempos de inter-renovaci\'on, y $N\left(t\right)$ es el n\'umero de renovaciones en el intervalo $\left[0,t\right)$
\end{Def}


\begin{Note}
Para definir un proceso de renovaci\'on para cualquier contexto, solamente hay que especificar una distribuci\'on $F$, con $F\left(0\right)=0$, para los tiempos de inter-renovaci\'on. La funci\'on $F$ en turno degune las otra variables aleatorias. De manera formal, existe un espacio de probabilidad y una sucesi\'on de variables aleatorias $\xi_{1},\xi_{2},\ldots$ definidas en este con distribuci\'on $F$. Entonces las otras cantidades son $T_{n}=\sum_{k=1}^{n}\xi_{k}$ y $N\left(t\right)=\sum_{n=1}^{\infty}\indora\left(T_{n}\leq t\right)$, donde $T_{n}\rightarrow\infty$ casi seguramente por la Ley Fuerte de los Grandes N\'umeros.
\end{Note}







Los tiempos $T_{n}$ est\'an relacionados con los conteos de $N\left(t\right)$ por

\begin{eqnarray*}
\left\{N\left(t\right)\geq n\right\}&=&\left\{T_{n}\leq t\right\}\\
T_{N\left(t\right)}\leq &t&<T_{N\left(t\right)+1},
\end{eqnarray*}

adem\'as $N\left(T_{n}\right)=n$, y 

\begin{eqnarray*}
N\left(t\right)=\max\left\{n:T_{n}\leq t\right\}=\min\left\{n:T_{n+1}>t\right\}
\end{eqnarray*}

Por propiedades de la convoluci\'on se sabe que

\begin{eqnarray*}
P\left\{T_{n}\leq t\right\}=F^{n\star}\left(t\right)
\end{eqnarray*}
que es la $n$-\'esima convoluci\'on de $F$. Entonces 

\begin{eqnarray*}
\left\{N\left(t\right)\geq n\right\}&=&\left\{T_{n}\leq t\right\}\\
P\left\{N\left(t\right)\leq n\right\}&=&1-F^{\left(n+1\right)\star}\left(t\right)
\end{eqnarray*}

Adem\'as usando el hecho de que $\esp\left[N\left(t\right)\right]=\sum_{n=1}^{\infty}P\left\{N\left(t\right)\geq n\right\}$
se tiene que

\begin{eqnarray*}
\esp\left[N\left(t\right)\right]=\sum_{n=1}^{\infty}F^{n\star}\left(t\right)
\end{eqnarray*}

\begin{Prop}
Para cada $t\geq0$, la funci\'on generadora de momentos $\esp\left[e^{\alpha N\left(t\right)}\right]$ existe para alguna $\alpha$ en una vecindad del 0, y de aqu\'i que $\esp\left[N\left(t\right)^{m}\right]<\infty$, para $m\geq1$.
\end{Prop}


\begin{Note}
Si el primer tiempo de renovaci\'on $\xi_{1}$ no tiene la misma distribuci\'on que el resto de las $\xi_{n}$, para $n\geq2$, a $N\left(t\right)$ se le llama Proceso de Renovaci\'on retardado, donde si $\xi$ tiene distribuci\'on $G$, entonces el tiempo $T_{n}$ de la $n$-\'esima renovaci\'on tiene distribuci\'on $G\star F^{\left(n-1\right)\star}\left(t\right)$
\end{Note}


\begin{Teo}
Para una constante $\mu\leq\infty$ ( o variable aleatoria), las siguientes expresiones son equivalentes:

\begin{eqnarray}
lim_{n\rightarrow\infty}n^{-1}T_{n}&=&\mu,\textrm{ c.s.}\\
lim_{t\rightarrow\infty}t^{-1}N\left(t\right)&=&1/\mu,\textrm{ c.s.}
\end{eqnarray}
\end{Teo}


Es decir, $T_{n}$ satisface la Ley Fuerte de los Grandes N\'umeros s\'i y s\'olo s\'i $N\left/t\right)$ la cumple.


\begin{Coro}[Ley Fuerte de los Grandes N\'umeros para Procesos de Renovaci\'on]
Si $N\left(t\right)$ es un proceso de renovaci\'on cuyos tiempos de inter-renovaci\'on tienen media $\mu\leq\infty$, entonces
\begin{eqnarray}
t^{-1}N\left(t\right)\rightarrow 1/\mu,\textrm{ c.s. cuando }t\rightarrow\infty.
\end{eqnarray}

\end{Coro}


Considerar el proceso estoc\'astico de valores reales $\left\{Z\left(t\right):t\geq0\right\}$ en el mismo espacio de probabilidad que $N\left(t\right)$

\begin{Def}
Para el proceso $\left\{Z\left(t\right):t\geq0\right\}$ se define la fluctuaci\'on m\'axima de $Z\left(t\right)$ en el intervalo $\left(T_{n-1},T_{n}\right]$:
\begin{eqnarray*}
M_{n}=\sup_{T_{n-1}<t\leq T_{n}}|Z\left(t\right)-Z\left(T_{n-1}\right)|
\end{eqnarray*}
\end{Def}

\begin{Teo}
Sup\'ongase que $n^{-1}T_{n}\rightarrow\mu$ c.s. cuando $n\rightarrow\infty$, donde $\mu\leq\infty$ es una constante o variable aleatoria. Sea $a$ una constante o variable aleatoria que puede ser infinita cuando $\mu$ es finita, y considere las expresiones l\'imite:
\begin{eqnarray}
lim_{n\rightarrow\infty}n^{-1}Z\left(T_{n}\right)&=&a,\textrm{ c.s.}\\
lim_{t\rightarrow\infty}t^{-1}Z\left(t\right)&=&a/\mu,\textrm{ c.s.}
\end{eqnarray}
La segunda expresi\'on implica la primera. Conversamente, la primera implica la segunda si el proceso $Z\left(t\right)$ es creciente, o si $lim_{n\rightarrow\infty}n^{-1}M_{n}=0$ c.s.
\end{Teo}

\begin{Coro}
Si $N\left(t\right)$ es un proceso de renovaci\'on, y $\left(Z\left(T_{n}\right)-Z\left(T_{n-1}\right),M_{n}\right)$, para $n\geq1$, son variables aleatorias independientes e id\'enticamente distribuidas con media finita, entonces,
\begin{eqnarray}
lim_{t\rightarrow\infty}t^{-1}Z\left(t\right)\rightarrow\frac{\esp\left[Z\left(T_{1}\right)-Z\left(T_{0}\right)\right]}{\esp\left[T_{1}\right]},\textrm{ c.s. cuando  }t\rightarrow\infty.
\end{eqnarray}
\end{Coro}


Sup\'ongase que $N\left(t\right)$ es un proceso de renovaci\'on con distribuci\'on $F$ con media finita $\mu$.

\begin{Def}
La funci\'on de renovaci\'on asociada con la distribuci\'on $F$, del proceso $N\left(t\right)$, es
\begin{eqnarray*}
U\left(t\right)=\sum_{n=1}^{\infty}F^{n\star}\left(t\right),\textrm{   }t\geq0,
\end{eqnarray*}
donde $F^{0\star}\left(t\right)=\indora\left(t\geq0\right)$.
\end{Def}


\begin{Prop}
Sup\'ongase que la distribuci\'on de inter-renovaci\'on $F$ tiene densidad $f$. Entonces $U\left(t\right)$ tambi\'en tiene densidad, para $t>0$, y es $U^{'}\left(t\right)=\sum_{n=0}^{\infty}f^{n\star}\left(t\right)$. Adem\'as
\begin{eqnarray*}
\prob\left\{N\left(t\right)>N\left(t-\right)\right\}=0\textrm{,   }t\geq0.
\end{eqnarray*}
\end{Prop}

\begin{Def}
La Transformada de Laplace-Stieljes de $F$ est\'a dada por

\begin{eqnarray*}
\hat{F}\left(\alpha\right)=\int_{\rea_{+}}e^{-\alpha t}dF\left(t\right)\textrm{,  }\alpha\geq0.
\end{eqnarray*}
\end{Def}

Entonces

\begin{eqnarray*}
\hat{U}\left(\alpha\right)=\sum_{n=0}^{\infty}\hat{F^{n\star}}\left(\alpha\right)=\sum_{n=0}^{\infty}\hat{F}\left(\alpha\right)^{n}=\frac{1}{1-\hat{F}\left(\alpha\right)}.
\end{eqnarray*}


\begin{Prop}
La Transformada de Laplace $\hat{U}\left(\alpha\right)$ y $\hat{F}\left(\alpha\right)$ determina una a la otra de manera \'unica por la relaci\'on $\hat{U}\left(\alpha\right)=\frac{1}{1-\hat{F}\left(\alpha\right)}$.
\end{Prop}


\begin{Note}
Un proceso de renovaci\'on $N\left(t\right)$ cuyos tiempos de inter-renovaci\'on tienen media finita, es un proceso Poisson con tasa $\lambda$ si y s\'olo s\'i $\esp\left[U\left(t\right)\right]=\lambda t$, para $t\geq0$.
\end{Note}


\begin{Teo}
Sea $N\left(t\right)$ un proceso puntual simple con puntos de localizaci\'on $T_{n}$ tal que $\eta\left(t\right)=\esp\left[N\left(\right)\right]$ es finita para cada $t$. Entonces para cualquier funci\'on $f:\rea_{+}\rightarrow\rea$,
\begin{eqnarray*}
\esp\left[\sum_{n=1}^{N\left(\right)}f\left(T_{n}\right)\right]=\int_{\left(0,t\right]}f\left(s\right)d\eta\left(s\right)\textrm{,  }t\geq0,
\end{eqnarray*}
suponiendo que la integral exista. Adem\'as si $X_{1},X_{2},\ldots$ son variables aleatorias definidas en el mismo espacio de probabilidad que el proceso $N\left(t\right)$ tal que $\esp\left[X_{n}|T_{n}=s\right]=f\left(s\right)$, independiente de $n$. Entonces
\begin{eqnarray*}
\esp\left[\sum_{n=1}^{N\left(t\right)}X_{n}\right]=\int_{\left(0,t\right]}f\left(s\right)d\eta\left(s\right)\textrm{,  }t\geq0,
\end{eqnarray*} 
suponiendo que la integral exista. 
\end{Teo}

\begin{Coro}[Identidad de Wald para Renovaciones]
Para el proceso de renovaci\'on $N\left(t\right)$,
\begin{eqnarray*}
\esp\left[T_{N\left(t\right)+1}\right]=\mu\esp\left[N\left(t\right)+1\right]\textrm{,  }t\geq0,
\end{eqnarray*}  
\end{Coro}


\begin{Def}
Sea $h\left(t\right)$ funci\'on de valores reales en $\rea$ acotada en intervalos finitos e igual a cero para $t<0$ La ecuaci\'on de renovaci\'on para $h\left(t\right)$ y la distribuci\'on $F$ es

\begin{eqnarray}\label{Ec.Renovacion}
H\left(t\right)=h\left(t\right)+\int_{\left[0,t\right]}H\left(t-s\right)dF\left(s\right)\textrm{,    }t\geq0,
\end{eqnarray}
donde $H\left(t\right)$ es una funci\'on de valores reales. Esto es $H=h+F\star H$. Decimos que $H\left(t\right)$ es soluci\'on de esta ecuaci\'on si satisface la ecuaci\'on, y es acotada en intervalos finitos e iguales a cero para $t<0$.
\end{Def}

\begin{Prop}
La funci\'on $U\star h\left(t\right)$ es la \'unica soluci\'on de la ecuaci\'on de renovaci\'on (\ref{Ec.Renovacion}).
\end{Prop}

\begin{Teo}[Teorema Renovaci\'on Elemental]
\begin{eqnarray*}
t^{-1}U\left(t\right)\rightarrow 1/\mu\textrm{,    cuando }t\rightarrow\infty.
\end{eqnarray*}
\end{Teo}



Sup\'ongase que $N\left(t\right)$ es un proceso de renovaci\'on con distribuci\'on $F$ con media finita $\mu$.

\begin{Def}
La funci\'on de renovaci\'on asociada con la distribuci\'on $F$, del proceso $N\left(t\right)$, es
\begin{eqnarray*}
U\left(t\right)=\sum_{n=1}^{\infty}F^{n\star}\left(t\right),\textrm{   }t\geq0,
\end{eqnarray*}
donde $F^{0\star}\left(t\right)=\indora\left(t\geq0\right)$.
\end{Def}


\begin{Prop}
Sup\'ongase que la distribuci\'on de inter-renovaci\'on $F$ tiene densidad $f$. Entonces $U\left(t\right)$ tambi\'en tiene densidad, para $t>0$, y es $U^{'}\left(t\right)=\sum_{n=0}^{\infty}f^{n\star}\left(t\right)$. Adem\'as
\begin{eqnarray*}
\prob\left\{N\left(t\right)>N\left(t-\right)\right\}=0\textrm{,   }t\geq0.
\end{eqnarray*}
\end{Prop}

\begin{Def}
La Transformada de Laplace-Stieljes de $F$ est\'a dada por

\begin{eqnarray*}
\hat{F}\left(\alpha\right)=\int_{\rea_{+}}e^{-\alpha t}dF\left(t\right)\textrm{,  }\alpha\geq0.
\end{eqnarray*}
\end{Def}

Entonces

\begin{eqnarray*}
\hat{U}\left(\alpha\right)=\sum_{n=0}^{\infty}\hat{F^{n\star}}\left(\alpha\right)=\sum_{n=0}^{\infty}\hat{F}\left(\alpha\right)^{n}=\frac{1}{1-\hat{F}\left(\alpha\right)}.
\end{eqnarray*}


\begin{Prop}
La Transformada de Laplace $\hat{U}\left(\alpha\right)$ y $\hat{F}\left(\alpha\right)$ determina una a la otra de manera \'unica por la relaci\'on $\hat{U}\left(\alpha\right)=\frac{1}{1-\hat{F}\left(\alpha\right)}$.
\end{Prop}


\begin{Note}
Un proceso de renovaci\'on $N\left(t\right)$ cuyos tiempos de inter-renovaci\'on tienen media finita, es un proceso Poisson con tasa $\lambda$ si y s\'olo s\'i $\esp\left[U\left(t\right)\right]=\lambda t$, para $t\geq0$.
\end{Note}


\begin{Teo}
Sea $N\left(t\right)$ un proceso puntual simple con puntos de localizaci\'on $T_{n}$ tal que $\eta\left(t\right)=\esp\left[N\left(\right)\right]$ es finita para cada $t$. Entonces para cualquier funci\'on $f:\rea_{+}\rightarrow\rea$,
\begin{eqnarray*}
\esp\left[\sum_{n=1}^{N\left(\right)}f\left(T_{n}\right)\right]=\int_{\left(0,t\right]}f\left(s\right)d\eta\left(s\right)\textrm{,  }t\geq0,
\end{eqnarray*}
suponiendo que la integral exista. Adem\'as si $X_{1},X_{2},\ldots$ son variables aleatorias definidas en el mismo espacio de probabilidad que el proceso $N\left(t\right)$ tal que $\esp\left[X_{n}|T_{n}=s\right]=f\left(s\right)$, independiente de $n$. Entonces
\begin{eqnarray*}
\esp\left[\sum_{n=1}^{N\left(t\right)}X_{n}\right]=\int_{\left(0,t\right]}f\left(s\right)d\eta\left(s\right)\textrm{,  }t\geq0,
\end{eqnarray*} 
suponiendo que la integral exista. 
\end{Teo}

\begin{Coro}[Identidad de Wald para Renovaciones]
Para el proceso de renovaci\'on $N\left(t\right)$,
\begin{eqnarray*}
\esp\left[T_{N\left(t\right)+1}\right]=\mu\esp\left[N\left(t\right)+1\right]\textrm{,  }t\geq0,
\end{eqnarray*}  
\end{Coro}


\begin{Def}
Sea $h\left(t\right)$ funci\'on de valores reales en $\rea$ acotada en intervalos finitos e igual a cero para $t<0$ La ecuaci\'on de renovaci\'on para $h\left(t\right)$ y la distribuci\'on $F$ es

\begin{eqnarray}\label{Ec.Renovacion}
H\left(t\right)=h\left(t\right)+\int_{\left[0,t\right]}H\left(t-s\right)dF\left(s\right)\textrm{,    }t\geq0,
\end{eqnarray}
donde $H\left(t\right)$ es una funci\'on de valores reales. Esto es $H=h+F\star H$. Decimos que $H\left(t\right)$ es soluci\'on de esta ecuaci\'on si satisface la ecuaci\'on, y es acotada en intervalos finitos e iguales a cero para $t<0$.
\end{Def}

\begin{Prop}
La funci\'on $U\star h\left(t\right)$ es la \'unica soluci\'on de la ecuaci\'on de renovaci\'on (\ref{Ec.Renovacion}).
\end{Prop}

\begin{Teo}[Teorema Renovaci\'on Elemental]
\begin{eqnarray*}
t^{-1}U\left(t\right)\rightarrow 1/\mu\textrm{,    cuando }t\rightarrow\infty.
\end{eqnarray*}
\end{Teo}


\begin{Note} Una funci\'on $h:\rea_{+}\rightarrow\rea$ es Directamente Riemann Integrable en los siguientes casos:
\begin{itemize}
\item[a)] $h\left(t\right)\geq0$ es decreciente y Riemann Integrable.
\item[b)] $h$ es continua excepto posiblemente en un conjunto de Lebesgue de medida 0, y $|h\left(t\right)|\leq b\left(t\right)$, donde $b$ es DRI.
\end{itemize}
\end{Note}

\begin{Teo}[Teorema Principal de Renovaci\'on]
Si $F$ es no aritm\'etica y $h\left(t\right)$ es Directamente Riemann Integrable (DRI), entonces

\begin{eqnarray*}
lim_{t\rightarrow\infty}U\star h=\frac{1}{\mu}\int_{\rea_{+}}h\left(s\right)ds.
\end{eqnarray*}
\end{Teo}

\begin{Prop}
Cualquier funci\'on $H\left(t\right)$ acotada en intervalos finitos y que es 0 para $t<0$ puede expresarse como
\begin{eqnarray*}
H\left(t\right)=U\star h\left(t\right)\textrm{,  donde }h\left(t\right)=H\left(t\right)-F\star H\left(t\right)
\end{eqnarray*}
\end{Prop}

\begin{Def}
Un proceso estoc\'astico $X\left(t\right)$ es crudamente regenerativo en un tiempo aleatorio positivo $T$ si
\begin{eqnarray*}
\esp\left[X\left(T+t\right)|T\right]=\esp\left[X\left(t\right)\right]\textrm{, para }t\geq0,\end{eqnarray*}
y con las esperanzas anteriores finitas.
\end{Def}

\begin{Prop}
Sup\'ongase que $X\left(t\right)$ es un proceso crudamente regenerativo en $T$, que tiene distribuci\'on $F$. Si $\esp\left[X\left(t\right)\right]$ es acotado en intervalos finitos, entonces
\begin{eqnarray*}
\esp\left[X\left(t\right)\right]=U\star h\left(t\right)\textrm{,  donde }h\left(t\right)=\esp\left[X\left(t\right)\indora\left(T>t\right)\right].
\end{eqnarray*}
\end{Prop}

\begin{Teo}[Regeneraci\'on Cruda]
Sup\'ongase que $X\left(t\right)$ es un proceso con valores positivo crudamente regenerativo en $T$, y def\'inase $M=\sup\left\{|X\left(t\right)|:t\leq T\right\}$. Si $T$ es no aritm\'etico y $M$ y $MT$ tienen media finita, entonces
\begin{eqnarray*}
lim_{t\rightarrow\infty}\esp\left[X\left(t\right)\right]=\frac{1}{\mu}\int_{\rea_{+}}h\left(s\right)ds,
\end{eqnarray*}
donde $h\left(t\right)=\esp\left[X\left(t\right)\indora\left(T>t\right)\right]$.
\end{Teo}


\begin{Note} Una funci\'on $h:\rea_{+}\rightarrow\rea$ es Directamente Riemann Integrable en los siguientes casos:
\begin{itemize}
\item[a)] $h\left(t\right)\geq0$ es decreciente y Riemann Integrable.
\item[b)] $h$ es continua excepto posiblemente en un conjunto de Lebesgue de medida 0, y $|h\left(t\right)|\leq b\left(t\right)$, donde $b$ es DRI.
\end{itemize}
\end{Note}

\begin{Teo}[Teorema Principal de Renovaci\'on]
Si $F$ es no aritm\'etica y $h\left(t\right)$ es Directamente Riemann Integrable (DRI), entonces

\begin{eqnarray*}
lim_{t\rightarrow\infty}U\star h=\frac{1}{\mu}\int_{\rea_{+}}h\left(s\right)ds.
\end{eqnarray*}
\end{Teo}

\begin{Prop}
Cualquier funci\'on $H\left(t\right)$ acotada en intervalos finitos y que es 0 para $t<0$ puede expresarse como
\begin{eqnarray*}
H\left(t\right)=U\star h\left(t\right)\textrm{,  donde }h\left(t\right)=H\left(t\right)-F\star H\left(t\right)
\end{eqnarray*}
\end{Prop}

\begin{Def}
Un proceso estoc\'astico $X\left(t\right)$ es crudamente regenerativo en un tiempo aleatorio positivo $T$ si
\begin{eqnarray*}
\esp\left[X\left(T+t\right)|T\right]=\esp\left[X\left(t\right)\right]\textrm{, para }t\geq0,\end{eqnarray*}
y con las esperanzas anteriores finitas.
\end{Def}

\begin{Prop}
Sup\'ongase que $X\left(t\right)$ es un proceso crudamente regenerativo en $T$, que tiene distribuci\'on $F$. Si $\esp\left[X\left(t\right)\right]$ es acotado en intervalos finitos, entonces
\begin{eqnarray*}
\esp\left[X\left(t\right)\right]=U\star h\left(t\right)\textrm{,  donde }h\left(t\right)=\esp\left[X\left(t\right)\indora\left(T>t\right)\right].
\end{eqnarray*}
\end{Prop}

\begin{Teo}[Regeneraci\'on Cruda]
Sup\'ongase que $X\left(t\right)$ es un proceso con valores positivo crudamente regenerativo en $T$, y def\'inase $M=\sup\left\{|X\left(t\right)|:t\leq T\right\}$. Si $T$ es no aritm\'etico y $M$ y $MT$ tienen media finita, entonces
\begin{eqnarray*}
lim_{t\rightarrow\infty}\esp\left[X\left(t\right)\right]=\frac{1}{\mu}\int_{\rea_{+}}h\left(s\right)ds,
\end{eqnarray*}
donde $h\left(t\right)=\esp\left[X\left(t\right)\indora\left(T>t\right)\right]$.
\end{Teo}

%________________________________________________________________________
\subsection{Procesos Regenerativos}
%________________________________________________________________________

Para $\left\{X\left(t\right):t\geq0\right\}$ Proceso Estoc\'astico a tiempo continuo con estado de espacios $S$, que es un espacio m\'etrico, con trayectorias continuas por la derecha y con l\'imites por la izquierda c.s. Sea $N\left(t\right)$ un proceso de renovaci\'on en $\rea_{+}$ definido en el mismo espacio de probabilidad que $X\left(t\right)$, con tiempos de renovaci\'on $T$ y tiempos de inter-renovaci\'on $\xi_{n}=T_{n}-T_{n-1}$, con misma distribuci\'on $F$ de media finita $\mu$.



\begin{Def}
Para el proceso $\left\{\left(N\left(t\right),X\left(t\right)\right):t\geq0\right\}$, sus trayectoria muestrales en el intervalo de tiempo $\left[T_{n-1},T_{n}\right)$ est\'an descritas por
\begin{eqnarray*}
\zeta_{n}=\left(\xi_{n},\left\{X\left(T_{n-1}+t\right):0\leq t<\xi_{n}\right\}\right)
\end{eqnarray*}
Este $\zeta_{n}$ es el $n$-\'esimo segmento del proceso. El proceso es regenerativo sobre los tiempos $T_{n}$ si sus segmentos $\zeta_{n}$ son independientes e id\'enticamennte distribuidos.
\end{Def}


\begin{Obs}
Si $\tilde{X}\left(t\right)$ con espacio de estados $\tilde{S}$ es regenerativo sobre $T_{n}$, entonces $X\left(t\right)=f\left(\tilde{X}\left(t\right)\right)$ tambi\'en es regenerativo sobre $T_{n}$, para cualquier funci\'on $f:\tilde{S}\rightarrow S$.
\end{Obs}

\begin{Obs}
Los procesos regenerativos son crudamente regenerativos, pero no al rev\'es.
\end{Obs}

\begin{Def}[Definici\'on Cl\'asica]
Un proceso estoc\'astico $X=\left\{X\left(t\right):t\geq0\right\}$ es llamado regenerativo is existe una variable aleatoria $R_{1}>0$ tal que
\begin{itemize}
\item[i)] $\left\{X\left(t+R_{1}\right):t\geq0\right\}$ es independiente de $\left\{\left\{X\left(t\right):t<R_{1}\right\},\right\}$
\item[ii)] $\left\{X\left(t+R_{1}\right):t\geq0\right\}$ es estoc\'asticamente equivalente a $\left\{X\left(t\right):t>0\right\}$
\end{itemize}

Llamamos a $R_{1}$ tiempo de regeneraci\'on, y decimos que $X$ se regenera en este punto.
\end{Def}

$\left\{X\left(t+R_{1}\right)\right\}$ es regenerativo con tiempo de regeneraci\'on $R_{2}$, independiente de $R_{1}$ pero con la misma distribuci\'on que $R_{1}$. Procediendo de esta manera se obtiene una secuencia de variables aleatorias independientes e id\'enticamente distribuidas $\left\{R_{n}\right\}$ llamados longitudes de ciclo. Si definimos a $Z_{k}\equiv R_{1}+R_{2}+\cdots+R_{k}$, se tiene un proceso de renovaci\'on llamado proceso de renovaci\'on encajado para $X$.

\begin{Note}
Un proceso regenerativo con media de la longitud de ciclo finita es llamado positivo recurrente.
\end{Note}


\begin{Def}
Para $x$ fijo y para cada $t\geq0$, sea $I_{x}\left(t\right)=1$ si $X\left(t\right)\leq x$,  $I_{x}\left(t\right)=0$ en caso contrario, y def\'inanse los tiempos promedio
\begin{eqnarray*}
\overline{X}&=&lim_{t\rightarrow\infty}\frac{1}{t}\int_{0}^{\infty}X\left(u\right)du\\
\prob\left(X_{\infty}\leq x\right)&=&lim_{t\rightarrow\infty}\frac{1}{t}\int_{0}^{\infty}I_{x}\left(u\right)du,
\end{eqnarray*}
cuando estos l\'imites existan.
\end{Def}

Como consecuencia del teorema de Renovaci\'on-Recompensa, se tiene que el primer l\'imite  existe y es igual a la constante
\begin{eqnarray*}
\overline{X}&=&\frac{\esp\left[\int_{0}^{R_{1}}X\left(t\right)dt\right]}{\esp\left[R_{1}\right]},
\end{eqnarray*}
suponiendo que ambas esperanzas son finitas.

\begin{Note}
\begin{itemize}
\item[a)] Si el proceso regenerativo $X$ es positivo recurrente y tiene trayectorias muestrales no negativas, entonces la ecuaci\'on anterior es v\'alida.
\item[b)] Si $X$ es positivo recurrente regenerativo, podemos construir una \'unica versi\'on estacionaria de este proceso, $X_{e}=\left\{X_{e}\left(t\right)\right\}$, donde $X_{e}$ es un proceso estoc\'astico regenerativo y estrictamente estacionario, con distribuci\'on marginal distribuida como $X_{\infty}$
\end{itemize}
\end{Note}

%________________________________________________________________________
\subsection{Procesos Regenerativos}
%________________________________________________________________________

Para $\left\{X\left(t\right):t\geq0\right\}$ Proceso Estoc\'astico a tiempo continuo con estado de espacios $S$, que es un espacio m\'etrico, con trayectorias continuas por la derecha y con l\'imites por la izquierda c.s. Sea $N\left(t\right)$ un proceso de renovaci\'on en $\rea_{+}$ definido en el mismo espacio de probabilidad que $X\left(t\right)$, con tiempos de renovaci\'on $T$ y tiempos de inter-renovaci\'on $\xi_{n}=T_{n}-T_{n-1}$, con misma distribuci\'on $F$ de media finita $\mu$.



\begin{Def}
Para el proceso $\left\{\left(N\left(t\right),X\left(t\right)\right):t\geq0\right\}$, sus trayectoria muestrales en el intervalo de tiempo $\left[T_{n-1},T_{n}\right)$ est\'an descritas por
\begin{eqnarray*}
\zeta_{n}=\left(\xi_{n},\left\{X\left(T_{n-1}+t\right):0\leq t<\xi_{n}\right\}\right)
\end{eqnarray*}
Este $\zeta_{n}$ es el $n$-\'esimo segmento del proceso. El proceso es regenerativo sobre los tiempos $T_{n}$ si sus segmentos $\zeta_{n}$ son independientes e id\'enticamennte distribuidos.
\end{Def}


\begin{Obs}
Si $\tilde{X}\left(t\right)$ con espacio de estados $\tilde{S}$ es regenerativo sobre $T_{n}$, entonces $X\left(t\right)=f\left(\tilde{X}\left(t\right)\right)$ tambi\'en es regenerativo sobre $T_{n}$, para cualquier funci\'on $f:\tilde{S}\rightarrow S$.
\end{Obs}

\begin{Obs}
Los procesos regenerativos son crudamente regenerativos, pero no al rev\'es.
\end{Obs}

\begin{Def}[Definici\'on Cl\'asica]
Un proceso estoc\'astico $X=\left\{X\left(t\right):t\geq0\right\}$ es llamado regenerativo is existe una variable aleatoria $R_{1}>0$ tal que
\begin{itemize}
\item[i)] $\left\{X\left(t+R_{1}\right):t\geq0\right\}$ es independiente de $\left\{\left\{X\left(t\right):t<R_{1}\right\},\right\}$
\item[ii)] $\left\{X\left(t+R_{1}\right):t\geq0\right\}$ es estoc\'asticamente equivalente a $\left\{X\left(t\right):t>0\right\}$
\end{itemize}

Llamamos a $R_{1}$ tiempo de regeneraci\'on, y decimos que $X$ se regenera en este punto.
\end{Def}

$\left\{X\left(t+R_{1}\right)\right\}$ es regenerativo con tiempo de regeneraci\'on $R_{2}$, independiente de $R_{1}$ pero con la misma distribuci\'on que $R_{1}$. Procediendo de esta manera se obtiene una secuencia de variables aleatorias independientes e id\'enticamente distribuidas $\left\{R_{n}\right\}$ llamados longitudes de ciclo. Si definimos a $Z_{k}\equiv R_{1}+R_{2}+\cdots+R_{k}$, se tiene un proceso de renovaci\'on llamado proceso de renovaci\'on encajado para $X$.

\begin{Note}
Un proceso regenerativo con media de la longitud de ciclo finita es llamado positivo recurrente.
\end{Note}


\begin{Def}
Para $x$ fijo y para cada $t\geq0$, sea $I_{x}\left(t\right)=1$ si $X\left(t\right)\leq x$,  $I_{x}\left(t\right)=0$ en caso contrario, y def\'inanse los tiempos promedio
\begin{eqnarray*}
\overline{X}&=&lim_{t\rightarrow\infty}\frac{1}{t}\int_{0}^{\infty}X\left(u\right)du\\
\prob\left(X_{\infty}\leq x\right)&=&lim_{t\rightarrow\infty}\frac{1}{t}\int_{0}^{\infty}I_{x}\left(u\right)du,
\end{eqnarray*}
cuando estos l\'imites existan.
\end{Def}

Como consecuencia del teorema de Renovaci\'on-Recompensa, se tiene que el primer l\'imite  existe y es igual a la constante
\begin{eqnarray*}
\overline{X}&=&\frac{\esp\left[\int_{0}^{R_{1}}X\left(t\right)dt\right]}{\esp\left[R_{1}\right]},
\end{eqnarray*}
suponiendo que ambas esperanzas son finitas.

\begin{Note}
\begin{itemize}
\item[a)] Si el proceso regenerativo $X$ es positivo recurrente y tiene trayectorias muestrales no negativas, entonces la ecuaci\'on anterior es v\'alida.
\item[b)] Si $X$ es positivo recurrente regenerativo, podemos construir una \'unica versi\'on estacionaria de este proceso, $X_{e}=\left\{X_{e}\left(t\right)\right\}$, donde $X_{e}$ es un proceso estoc\'astico regenerativo y estrictamente estacionario, con distribuci\'on marginal distribuida como $X_{\infty}$
\end{itemize}
\end{Note}
%__________________________________________________________________________________________
\subsection{Procesos Regenerativos Estacionarios - Stidham \cite{Stidham}}
%__________________________________________________________________________________________


Un proceso estoc\'astico a tiempo continuo $\left\{V\left(t\right),t\geq0\right\}$ es un proceso regenerativo si existe una sucesi\'on de variables aleatorias independientes e id\'enticamente distribuidas $\left\{X_{1},X_{2},\ldots\right\}$, sucesi\'on de renovaci\'on, tal que para cualquier conjunto de Borel $A$, 

\begin{eqnarray*}
\prob\left\{V\left(t\right)\in A|X_{1}+X_{2}+\cdots+X_{R\left(t\right)}=s,\left\{V\left(\tau\right),\tau<s\right\}\right\}=\prob\left\{V\left(t-s\right)\in A|X_{1}>t-s\right\},
\end{eqnarray*}
para todo $0\leq s\leq t$, donde $R\left(t\right)=\max\left\{X_{1}+X_{2}+\cdots+X_{j}\leq t\right\}=$n\'umero de renovaciones ({\emph{puntos de regeneraci\'on}}) que ocurren en $\left[0,t\right]$. El intervalo $\left[0,X_{1}\right)$ es llamado {\emph{primer ciclo de regeneraci\'on}} de $\left\{V\left(t \right),t\geq0\right\}$, $\left[X_{1},X_{1}+X_{2}\right)$ el {\emph{segundo ciclo de regeneraci\'on}}, y as\'i sucesivamente.

Sea $X=X_{1}$ y sea $F$ la funci\'on de distrbuci\'on de $X$


\begin{Def}
Se define el proceso estacionario, $\left\{V^{*}\left(t\right),t\geq0\right\}$, para $\left\{V\left(t\right),t\geq0\right\}$ por

\begin{eqnarray*}
\prob\left\{V\left(t\right)\in A\right\}=\frac{1}{\esp\left[X\right]}\int_{0}^{\infty}\prob\left\{V\left(t+x\right)\in A|X>x\right\}\left(1-F\left(x\right)\right)dx,
\end{eqnarray*} 
para todo $t\geq0$ y todo conjunto de Borel $A$.
\end{Def}

\begin{Def}
Una distribuci\'on se dice que es {\emph{aritm\'etica}} si todos sus puntos de incremento son m\'ultiplos de la forma $0,\lambda, 2\lambda,\ldots$ para alguna $\lambda>0$ entera.
\end{Def}


\begin{Def}
Una modificaci\'on medible de un proceso $\left\{V\left(t\right),t\geq0\right\}$, es una versi\'on de este, $\left\{V\left(t,w\right)\right\}$ conjuntamente medible para $t\geq0$ y para $w\in S$, $S$ espacio de estados para $\left\{V\left(t\right),t\geq0\right\}$.
\end{Def}

\begin{Teo}
Sea $\left\{V\left(t\right),t\geq\right\}$ un proceso regenerativo no negativo con modificaci\'on medible. Sea $\esp\left[X\right]<\infty$. Entonces el proceso estacionario dado por la ecuaci\'on anterior est\'a bien definido y tiene funci\'on de distribuci\'on independiente de $t$, adem\'as
\begin{itemize}
\item[i)] \begin{eqnarray*}
\esp\left[V^{*}\left(0\right)\right]&=&\frac{\esp\left[\int_{0}^{X}V\left(s\right)ds\right]}{\esp\left[X\right]}\end{eqnarray*}
\item[ii)] Si $\esp\left[V^{*}\left(0\right)\right]<\infty$, equivalentemente, si $\esp\left[\int_{0}^{X}V\left(s\right)ds\right]<\infty$,entonces
\begin{eqnarray*}
\frac{\int_{0}^{t}V\left(s\right)ds}{t}\rightarrow\frac{\esp\left[\int_{0}^{X}V\left(s\right)ds\right]}{\esp\left[X\right]}
\end{eqnarray*}
con probabilidad 1 y en media, cuando $t\rightarrow\infty$.
\end{itemize}
\end{Teo}


%__________________________________________________________________________________________
\subsection{Procesos Regenerativos Estacionarios - Stidham \cite{Stidham}}
%__________________________________________________________________________________________


Un proceso estoc\'astico a tiempo continuo $\left\{V\left(t\right),t\geq0\right\}$ es un proceso regenerativo si existe una sucesi\'on de variables aleatorias independientes e id\'enticamente distribuidas $\left\{X_{1},X_{2},\ldots\right\}$, sucesi\'on de renovaci\'on, tal que para cualquier conjunto de Borel $A$, 

\begin{eqnarray*}
\prob\left\{V\left(t\right)\in A|X_{1}+X_{2}+\cdots+X_{R\left(t\right)}=s,\left\{V\left(\tau\right),\tau<s\right\}\right\}=\prob\left\{V\left(t-s\right)\in A|X_{1}>t-s\right\},
\end{eqnarray*}
para todo $0\leq s\leq t$, donde $R\left(t\right)=\max\left\{X_{1}+X_{2}+\cdots+X_{j}\leq t\right\}=$n\'umero de renovaciones ({\emph{puntos de regeneraci\'on}}) que ocurren en $\left[0,t\right]$. El intervalo $\left[0,X_{1}\right)$ es llamado {\emph{primer ciclo de regeneraci\'on}} de $\left\{V\left(t \right),t\geq0\right\}$, $\left[X_{1},X_{1}+X_{2}\right)$ el {\emph{segundo ciclo de regeneraci\'on}}, y as\'i sucesivamente.

Sea $X=X_{1}$ y sea $F$ la funci\'on de distrbuci\'on de $X$


\begin{Def}
Se define el proceso estacionario, $\left\{V^{*}\left(t\right),t\geq0\right\}$, para $\left\{V\left(t\right),t\geq0\right\}$ por

\begin{eqnarray*}
\prob\left\{V\left(t\right)\in A\right\}=\frac{1}{\esp\left[X\right]}\int_{0}^{\infty}\prob\left\{V\left(t+x\right)\in A|X>x\right\}\left(1-F\left(x\right)\right)dx,
\end{eqnarray*} 
para todo $t\geq0$ y todo conjunto de Borel $A$.
\end{Def}

\begin{Def}
Una distribuci\'on se dice que es {\emph{aritm\'etica}} si todos sus puntos de incremento son m\'ultiplos de la forma $0,\lambda, 2\lambda,\ldots$ para alguna $\lambda>0$ entera.
\end{Def}


\begin{Def}
Una modificaci\'on medible de un proceso $\left\{V\left(t\right),t\geq0\right\}$, es una versi\'on de este, $\left\{V\left(t,w\right)\right\}$ conjuntamente medible para $t\geq0$ y para $w\in S$, $S$ espacio de estados para $\left\{V\left(t\right),t\geq0\right\}$.
\end{Def}

\begin{Teo}
Sea $\left\{V\left(t\right),t\geq\right\}$ un proceso regenerativo no negativo con modificaci\'on medible. Sea $\esp\left[X\right]<\infty$. Entonces el proceso estacionario dado por la ecuaci\'on anterior est\'a bien definido y tiene funci\'on de distribuci\'on independiente de $t$, adem\'as
\begin{itemize}
\item[i)] \begin{eqnarray*}
\esp\left[V^{*}\left(0\right)\right]&=&\frac{\esp\left[\int_{0}^{X}V\left(s\right)ds\right]}{\esp\left[X\right]}\end{eqnarray*}
\item[ii)] Si $\esp\left[V^{*}\left(0\right)\right]<\infty$, equivalentemente, si $\esp\left[\int_{0}^{X}V\left(s\right)ds\right]<\infty$,entonces
\begin{eqnarray*}
\frac{\int_{0}^{t}V\left(s\right)ds}{t}\rightarrow\frac{\esp\left[\int_{0}^{X}V\left(s\right)ds\right]}{\esp\left[X\right]}
\end{eqnarray*}
con probabilidad 1 y en media, cuando $t\rightarrow\infty$.
\end{itemize}
\end{Teo}
%
%___________________________________________________________________________________________
%\vspace{5.5cm}
%\chapter{Cadenas de Markov estacionarias}
%\vspace{-1.0cm}
%___________________________________________________________________________________________
%
\subsection{Propiedades de los Procesos de Renovaci\'on}
%___________________________________________________________________________________________
%

Los tiempos $T_{n}$ est\'an relacionados con los conteos de $N\left(t\right)$ por

\begin{eqnarray*}
\left\{N\left(t\right)\geq n\right\}&=&\left\{T_{n}\leq t\right\}\\
T_{N\left(t\right)}\leq &t&<T_{N\left(t\right)+1},
\end{eqnarray*}

adem\'as $N\left(T_{n}\right)=n$, y 

\begin{eqnarray*}
N\left(t\right)=\max\left\{n:T_{n}\leq t\right\}=\min\left\{n:T_{n+1}>t\right\}
\end{eqnarray*}

Por propiedades de la convoluci\'on se sabe que

\begin{eqnarray*}
P\left\{T_{n}\leq t\right\}=F^{n\star}\left(t\right)
\end{eqnarray*}
que es la $n$-\'esima convoluci\'on de $F$. Entonces 

\begin{eqnarray*}
\left\{N\left(t\right)\geq n\right\}&=&\left\{T_{n}\leq t\right\}\\
P\left\{N\left(t\right)\leq n\right\}&=&1-F^{\left(n+1\right)\star}\left(t\right)
\end{eqnarray*}

Adem\'as usando el hecho de que $\esp\left[N\left(t\right)\right]=\sum_{n=1}^{\infty}P\left\{N\left(t\right)\geq n\right\}$
se tiene que

\begin{eqnarray*}
\esp\left[N\left(t\right)\right]=\sum_{n=1}^{\infty}F^{n\star}\left(t\right)
\end{eqnarray*}

\begin{Prop}
Para cada $t\geq0$, la funci\'on generadora de momentos $\esp\left[e^{\alpha N\left(t\right)}\right]$ existe para alguna $\alpha$ en una vecindad del 0, y de aqu\'i que $\esp\left[N\left(t\right)^{m}\right]<\infty$, para $m\geq1$.
\end{Prop}


\begin{Note}
Si el primer tiempo de renovaci\'on $\xi_{1}$ no tiene la misma distribuci\'on que el resto de las $\xi_{n}$, para $n\geq2$, a $N\left(t\right)$ se le llama Proceso de Renovaci\'on retardado, donde si $\xi$ tiene distribuci\'on $G$, entonces el tiempo $T_{n}$ de la $n$-\'esima renovaci\'on tiene distribuci\'on $G\star F^{\left(n-1\right)\star}\left(t\right)$
\end{Note}


\begin{Teo}
Para una constante $\mu\leq\infty$ ( o variable aleatoria), las siguientes expresiones son equivalentes:

\begin{eqnarray}
lim_{n\rightarrow\infty}n^{-1}T_{n}&=&\mu,\textrm{ c.s.}\\
lim_{t\rightarrow\infty}t^{-1}N\left(t\right)&=&1/\mu,\textrm{ c.s.}
\end{eqnarray}
\end{Teo}


Es decir, $T_{n}$ satisface la Ley Fuerte de los Grandes N\'umeros s\'i y s\'olo s\'i $N\left/t\right)$ la cumple.


\begin{Coro}[Ley Fuerte de los Grandes N\'umeros para Procesos de Renovaci\'on]
Si $N\left(t\right)$ es un proceso de renovaci\'on cuyos tiempos de inter-renovaci\'on tienen media $\mu\leq\infty$, entonces
\begin{eqnarray}
t^{-1}N\left(t\right)\rightarrow 1/\mu,\textrm{ c.s. cuando }t\rightarrow\infty.
\end{eqnarray}

\end{Coro}


Considerar el proceso estoc\'astico de valores reales $\left\{Z\left(t\right):t\geq0\right\}$ en el mismo espacio de probabilidad que $N\left(t\right)$

\begin{Def}
Para el proceso $\left\{Z\left(t\right):t\geq0\right\}$ se define la fluctuaci\'on m\'axima de $Z\left(t\right)$ en el intervalo $\left(T_{n-1},T_{n}\right]$:
\begin{eqnarray*}
M_{n}=\sup_{T_{n-1}<t\leq T_{n}}|Z\left(t\right)-Z\left(T_{n-1}\right)|
\end{eqnarray*}
\end{Def}

\begin{Teo}
Sup\'ongase que $n^{-1}T_{n}\rightarrow\mu$ c.s. cuando $n\rightarrow\infty$, donde $\mu\leq\infty$ es una constante o variable aleatoria. Sea $a$ una constante o variable aleatoria que puede ser infinita cuando $\mu$ es finita, y considere las expresiones l\'imite:
\begin{eqnarray}
lim_{n\rightarrow\infty}n^{-1}Z\left(T_{n}\right)&=&a,\textrm{ c.s.}\\
lim_{t\rightarrow\infty}t^{-1}Z\left(t\right)&=&a/\mu,\textrm{ c.s.}
\end{eqnarray}
La segunda expresi\'on implica la primera. Conversamente, la primera implica la segunda si el proceso $Z\left(t\right)$ es creciente, o si $lim_{n\rightarrow\infty}n^{-1}M_{n}=0$ c.s.
\end{Teo}

\begin{Coro}
Si $N\left(t\right)$ es un proceso de renovaci\'on, y $\left(Z\left(T_{n}\right)-Z\left(T_{n-1}\right),M_{n}\right)$, para $n\geq1$, son variables aleatorias independientes e id\'enticamente distribuidas con media finita, entonces,
\begin{eqnarray}
lim_{t\rightarrow\infty}t^{-1}Z\left(t\right)\rightarrow\frac{\esp\left[Z\left(T_{1}\right)-Z\left(T_{0}\right)\right]}{\esp\left[T_{1}\right]},\textrm{ c.s. cuando  }t\rightarrow\infty.
\end{eqnarray}
\end{Coro}


%___________________________________________________________________________________________
%
%\subsection{Propiedades de los Procesos de Renovaci\'on}
%___________________________________________________________________________________________
%

Los tiempos $T_{n}$ est\'an relacionados con los conteos de $N\left(t\right)$ por

\begin{eqnarray*}
\left\{N\left(t\right)\geq n\right\}&=&\left\{T_{n}\leq t\right\}\\
T_{N\left(t\right)}\leq &t&<T_{N\left(t\right)+1},
\end{eqnarray*}

adem\'as $N\left(T_{n}\right)=n$, y 

\begin{eqnarray*}
N\left(t\right)=\max\left\{n:T_{n}\leq t\right\}=\min\left\{n:T_{n+1}>t\right\}
\end{eqnarray*}

Por propiedades de la convoluci\'on se sabe que

\begin{eqnarray*}
P\left\{T_{n}\leq t\right\}=F^{n\star}\left(t\right)
\end{eqnarray*}
que es la $n$-\'esima convoluci\'on de $F$. Entonces 

\begin{eqnarray*}
\left\{N\left(t\right)\geq n\right\}&=&\left\{T_{n}\leq t\right\}\\
P\left\{N\left(t\right)\leq n\right\}&=&1-F^{\left(n+1\right)\star}\left(t\right)
\end{eqnarray*}

Adem\'as usando el hecho de que $\esp\left[N\left(t\right)\right]=\sum_{n=1}^{\infty}P\left\{N\left(t\right)\geq n\right\}$
se tiene que

\begin{eqnarray*}
\esp\left[N\left(t\right)\right]=\sum_{n=1}^{\infty}F^{n\star}\left(t\right)
\end{eqnarray*}

\begin{Prop}
Para cada $t\geq0$, la funci\'on generadora de momentos $\esp\left[e^{\alpha N\left(t\right)}\right]$ existe para alguna $\alpha$ en una vecindad del 0, y de aqu\'i que $\esp\left[N\left(t\right)^{m}\right]<\infty$, para $m\geq1$.
\end{Prop}


\begin{Note}
Si el primer tiempo de renovaci\'on $\xi_{1}$ no tiene la misma distribuci\'on que el resto de las $\xi_{n}$, para $n\geq2$, a $N\left(t\right)$ se le llama Proceso de Renovaci\'on retardado, donde si $\xi$ tiene distribuci\'on $G$, entonces el tiempo $T_{n}$ de la $n$-\'esima renovaci\'on tiene distribuci\'on $G\star F^{\left(n-1\right)\star}\left(t\right)$
\end{Note}


\begin{Teo}
Para una constante $\mu\leq\infty$ ( o variable aleatoria), las siguientes expresiones son equivalentes:

\begin{eqnarray}
lim_{n\rightarrow\infty}n^{-1}T_{n}&=&\mu,\textrm{ c.s.}\\
lim_{t\rightarrow\infty}t^{-1}N\left(t\right)&=&1/\mu,\textrm{ c.s.}
\end{eqnarray}
\end{Teo}


Es decir, $T_{n}$ satisface la Ley Fuerte de los Grandes N\'umeros s\'i y s\'olo s\'i $N\left/t\right)$ la cumple.


\begin{Coro}[Ley Fuerte de los Grandes N\'umeros para Procesos de Renovaci\'on]
Si $N\left(t\right)$ es un proceso de renovaci\'on cuyos tiempos de inter-renovaci\'on tienen media $\mu\leq\infty$, entonces
\begin{eqnarray}
t^{-1}N\left(t\right)\rightarrow 1/\mu,\textrm{ c.s. cuando }t\rightarrow\infty.
\end{eqnarray}

\end{Coro}


Considerar el proceso estoc\'astico de valores reales $\left\{Z\left(t\right):t\geq0\right\}$ en el mismo espacio de probabilidad que $N\left(t\right)$

\begin{Def}
Para el proceso $\left\{Z\left(t\right):t\geq0\right\}$ se define la fluctuaci\'on m\'axima de $Z\left(t\right)$ en el intervalo $\left(T_{n-1},T_{n}\right]$:
\begin{eqnarray*}
M_{n}=\sup_{T_{n-1}<t\leq T_{n}}|Z\left(t\right)-Z\left(T_{n-1}\right)|
\end{eqnarray*}
\end{Def}

\begin{Teo}
Sup\'ongase que $n^{-1}T_{n}\rightarrow\mu$ c.s. cuando $n\rightarrow\infty$, donde $\mu\leq\infty$ es una constante o variable aleatoria. Sea $a$ una constante o variable aleatoria que puede ser infinita cuando $\mu$ es finita, y considere las expresiones l\'imite:
\begin{eqnarray}
lim_{n\rightarrow\infty}n^{-1}Z\left(T_{n}\right)&=&a,\textrm{ c.s.}\\
lim_{t\rightarrow\infty}t^{-1}Z\left(t\right)&=&a/\mu,\textrm{ c.s.}
\end{eqnarray}
La segunda expresi\'on implica la primera. Conversamente, la primera implica la segunda si el proceso $Z\left(t\right)$ es creciente, o si $lim_{n\rightarrow\infty}n^{-1}M_{n}=0$ c.s.
\end{Teo}

\begin{Coro}
Si $N\left(t\right)$ es un proceso de renovaci\'on, y $\left(Z\left(T_{n}\right)-Z\left(T_{n-1}\right),M_{n}\right)$, para $n\geq1$, son variables aleatorias independientes e id\'enticamente distribuidas con media finita, entonces,
\begin{eqnarray}
lim_{t\rightarrow\infty}t^{-1}Z\left(t\right)\rightarrow\frac{\esp\left[Z\left(T_{1}\right)-Z\left(T_{0}\right)\right]}{\esp\left[T_{1}\right]},\textrm{ c.s. cuando  }t\rightarrow\infty.
\end{eqnarray}
\end{Coro}

%___________________________________________________________________________________________
%
%\subsection{Propiedades de los Procesos de Renovaci\'on}
%___________________________________________________________________________________________
%

Los tiempos $T_{n}$ est\'an relacionados con los conteos de $N\left(t\right)$ por

\begin{eqnarray*}
\left\{N\left(t\right)\geq n\right\}&=&\left\{T_{n}\leq t\right\}\\
T_{N\left(t\right)}\leq &t&<T_{N\left(t\right)+1},
\end{eqnarray*}

adem\'as $N\left(T_{n}\right)=n$, y 

\begin{eqnarray*}
N\left(t\right)=\max\left\{n:T_{n}\leq t\right\}=\min\left\{n:T_{n+1}>t\right\}
\end{eqnarray*}

Por propiedades de la convoluci\'on se sabe que

\begin{eqnarray*}
P\left\{T_{n}\leq t\right\}=F^{n\star}\left(t\right)
\end{eqnarray*}
que es la $n$-\'esima convoluci\'on de $F$. Entonces 

\begin{eqnarray*}
\left\{N\left(t\right)\geq n\right\}&=&\left\{T_{n}\leq t\right\}\\
P\left\{N\left(t\right)\leq n\right\}&=&1-F^{\left(n+1\right)\star}\left(t\right)
\end{eqnarray*}

Adem\'as usando el hecho de que $\esp\left[N\left(t\right)\right]=\sum_{n=1}^{\infty}P\left\{N\left(t\right)\geq n\right\}$
se tiene que

\begin{eqnarray*}
\esp\left[N\left(t\right)\right]=\sum_{n=1}^{\infty}F^{n\star}\left(t\right)
\end{eqnarray*}

\begin{Prop}
Para cada $t\geq0$, la funci\'on generadora de momentos $\esp\left[e^{\alpha N\left(t\right)}\right]$ existe para alguna $\alpha$ en una vecindad del 0, y de aqu\'i que $\esp\left[N\left(t\right)^{m}\right]<\infty$, para $m\geq1$.
\end{Prop}


\begin{Note}
Si el primer tiempo de renovaci\'on $\xi_{1}$ no tiene la misma distribuci\'on que el resto de las $\xi_{n}$, para $n\geq2$, a $N\left(t\right)$ se le llama Proceso de Renovaci\'on retardado, donde si $\xi$ tiene distribuci\'on $G$, entonces el tiempo $T_{n}$ de la $n$-\'esima renovaci\'on tiene distribuci\'on $G\star F^{\left(n-1\right)\star}\left(t\right)$
\end{Note}


\begin{Teo}
Para una constante $\mu\leq\infty$ ( o variable aleatoria), las siguientes expresiones son equivalentes:

\begin{eqnarray}
lim_{n\rightarrow\infty}n^{-1}T_{n}&=&\mu,\textrm{ c.s.}\\
lim_{t\rightarrow\infty}t^{-1}N\left(t\right)&=&1/\mu,\textrm{ c.s.}
\end{eqnarray}
\end{Teo}


Es decir, $T_{n}$ satisface la Ley Fuerte de los Grandes N\'umeros s\'i y s\'olo s\'i $N\left/t\right)$ la cumple.


\begin{Coro}[Ley Fuerte de los Grandes N\'umeros para Procesos de Renovaci\'on]
Si $N\left(t\right)$ es un proceso de renovaci\'on cuyos tiempos de inter-renovaci\'on tienen media $\mu\leq\infty$, entonces
\begin{eqnarray}
t^{-1}N\left(t\right)\rightarrow 1/\mu,\textrm{ c.s. cuando }t\rightarrow\infty.
\end{eqnarray}

\end{Coro}


Considerar el proceso estoc\'astico de valores reales $\left\{Z\left(t\right):t\geq0\right\}$ en el mismo espacio de probabilidad que $N\left(t\right)$

\begin{Def}
Para el proceso $\left\{Z\left(t\right):t\geq0\right\}$ se define la fluctuaci\'on m\'axima de $Z\left(t\right)$ en el intervalo $\left(T_{n-1},T_{n}\right]$:
\begin{eqnarray*}
M_{n}=\sup_{T_{n-1}<t\leq T_{n}}|Z\left(t\right)-Z\left(T_{n-1}\right)|
\end{eqnarray*}
\end{Def}

\begin{Teo}
Sup\'ongase que $n^{-1}T_{n}\rightarrow\mu$ c.s. cuando $n\rightarrow\infty$, donde $\mu\leq\infty$ es una constante o variable aleatoria. Sea $a$ una constante o variable aleatoria que puede ser infinita cuando $\mu$ es finita, y considere las expresiones l\'imite:
\begin{eqnarray}
lim_{n\rightarrow\infty}n^{-1}Z\left(T_{n}\right)&=&a,\textrm{ c.s.}\\
lim_{t\rightarrow\infty}t^{-1}Z\left(t\right)&=&a/\mu,\textrm{ c.s.}
\end{eqnarray}
La segunda expresi\'on implica la primera. Conversamente, la primera implica la segunda si el proceso $Z\left(t\right)$ es creciente, o si $lim_{n\rightarrow\infty}n^{-1}M_{n}=0$ c.s.
\end{Teo}

\begin{Coro}
Si $N\left(t\right)$ es un proceso de renovaci\'on, y $\left(Z\left(T_{n}\right)-Z\left(T_{n-1}\right),M_{n}\right)$, para $n\geq1$, son variables aleatorias independientes e id\'enticamente distribuidas con media finita, entonces,
\begin{eqnarray}
lim_{t\rightarrow\infty}t^{-1}Z\left(t\right)\rightarrow\frac{\esp\left[Z\left(T_{1}\right)-Z\left(T_{0}\right)\right]}{\esp\left[T_{1}\right]},\textrm{ c.s. cuando  }t\rightarrow\infty.
\end{eqnarray}
\end{Coro}



%___________________________________________________________________________________________
%
\subsection{Propiedades de los Procesos de Renovaci\'on}
%___________________________________________________________________________________________
%

Los tiempos $T_{n}$ est\'an relacionados con los conteos de $N\left(t\right)$ por

\begin{eqnarray*}
\left\{N\left(t\right)\geq n\right\}&=&\left\{T_{n}\leq t\right\}\\
T_{N\left(t\right)}\leq &t&<T_{N\left(t\right)+1},
\end{eqnarray*}

adem\'as $N\left(T_{n}\right)=n$, y 

\begin{eqnarray*}
N\left(t\right)=\max\left\{n:T_{n}\leq t\right\}=\min\left\{n:T_{n+1}>t\right\}
\end{eqnarray*}

Por propiedades de la convoluci\'on se sabe que

\begin{eqnarray*}
P\left\{T_{n}\leq t\right\}=F^{n\star}\left(t\right)
\end{eqnarray*}
que es la $n$-\'esima convoluci\'on de $F$. Entonces 

\begin{eqnarray*}
\left\{N\left(t\right)\geq n\right\}&=&\left\{T_{n}\leq t\right\}\\
P\left\{N\left(t\right)\leq n\right\}&=&1-F^{\left(n+1\right)\star}\left(t\right)
\end{eqnarray*}

Adem\'as usando el hecho de que $\esp\left[N\left(t\right)\right]=\sum_{n=1}^{\infty}P\left\{N\left(t\right)\geq n\right\}$
se tiene que

\begin{eqnarray*}
\esp\left[N\left(t\right)\right]=\sum_{n=1}^{\infty}F^{n\star}\left(t\right)
\end{eqnarray*}

\begin{Prop}
Para cada $t\geq0$, la funci\'on generadora de momentos $\esp\left[e^{\alpha N\left(t\right)}\right]$ existe para alguna $\alpha$ en una vecindad del 0, y de aqu\'i que $\esp\left[N\left(t\right)^{m}\right]<\infty$, para $m\geq1$.
\end{Prop}


\begin{Note}
Si el primer tiempo de renovaci\'on $\xi_{1}$ no tiene la misma distribuci\'on que el resto de las $\xi_{n}$, para $n\geq2$, a $N\left(t\right)$ se le llama Proceso de Renovaci\'on retardado, donde si $\xi$ tiene distribuci\'on $G$, entonces el tiempo $T_{n}$ de la $n$-\'esima renovaci\'on tiene distribuci\'on $G\star F^{\left(n-1\right)\star}\left(t\right)$
\end{Note}


\begin{Teo}
Para una constante $\mu\leq\infty$ ( o variable aleatoria), las siguientes expresiones son equivalentes:

\begin{eqnarray}
lim_{n\rightarrow\infty}n^{-1}T_{n}&=&\mu,\textrm{ c.s.}\\
lim_{t\rightarrow\infty}t^{-1}N\left(t\right)&=&1/\mu,\textrm{ c.s.}
\end{eqnarray}
\end{Teo}


Es decir, $T_{n}$ satisface la Ley Fuerte de los Grandes N\'umeros s\'i y s\'olo s\'i $N\left/t\right)$ la cumple.


\begin{Coro}[Ley Fuerte de los Grandes N\'umeros para Procesos de Renovaci\'on]
Si $N\left(t\right)$ es un proceso de renovaci\'on cuyos tiempos de inter-renovaci\'on tienen media $\mu\leq\infty$, entonces
\begin{eqnarray}
t^{-1}N\left(t\right)\rightarrow 1/\mu,\textrm{ c.s. cuando }t\rightarrow\infty.
\end{eqnarray}

\end{Coro}


Considerar el proceso estoc\'astico de valores reales $\left\{Z\left(t\right):t\geq0\right\}$ en el mismo espacio de probabilidad que $N\left(t\right)$

\begin{Def}
Para el proceso $\left\{Z\left(t\right):t\geq0\right\}$ se define la fluctuaci\'on m\'axima de $Z\left(t\right)$ en el intervalo $\left(T_{n-1},T_{n}\right]$:
\begin{eqnarray*}
M_{n}=\sup_{T_{n-1}<t\leq T_{n}}|Z\left(t\right)-Z\left(T_{n-1}\right)|
\end{eqnarray*}
\end{Def}

\begin{Teo}
Sup\'ongase que $n^{-1}T_{n}\rightarrow\mu$ c.s. cuando $n\rightarrow\infty$, donde $\mu\leq\infty$ es una constante o variable aleatoria. Sea $a$ una constante o variable aleatoria que puede ser infinita cuando $\mu$ es finita, y considere las expresiones l\'imite:
\begin{eqnarray}
lim_{n\rightarrow\infty}n^{-1}Z\left(T_{n}\right)&=&a,\textrm{ c.s.}\\
lim_{t\rightarrow\infty}t^{-1}Z\left(t\right)&=&a/\mu,\textrm{ c.s.}
\end{eqnarray}
La segunda expresi\'on implica la primera. Conversamente, la primera implica la segunda si el proceso $Z\left(t\right)$ es creciente, o si $lim_{n\rightarrow\infty}n^{-1}M_{n}=0$ c.s.
\end{Teo}

\begin{Coro}
Si $N\left(t\right)$ es un proceso de renovaci\'on, y $\left(Z\left(T_{n}\right)-Z\left(T_{n-1}\right),M_{n}\right)$, para $n\geq1$, son variables aleatorias independientes e id\'enticamente distribuidas con media finita, entonces,
\begin{eqnarray}
lim_{t\rightarrow\infty}t^{-1}Z\left(t\right)\rightarrow\frac{\esp\left[Z\left(T_{1}\right)-Z\left(T_{0}\right)\right]}{\esp\left[T_{1}\right]},\textrm{ c.s. cuando  }t\rightarrow\infty.
\end{eqnarray}
\end{Coro}




%__________________________________________________________________________________________
\subsection{Procesos Regenerativos Estacionarios - Stidham \cite{Stidham}}
%__________________________________________________________________________________________


Un proceso estoc\'astico a tiempo continuo $\left\{V\left(t\right),t\geq0\right\}$ es un proceso regenerativo si existe una sucesi\'on de variables aleatorias independientes e id\'enticamente distribuidas $\left\{X_{1},X_{2},\ldots\right\}$, sucesi\'on de renovaci\'on, tal que para cualquier conjunto de Borel $A$, 

\begin{eqnarray*}
\prob\left\{V\left(t\right)\in A|X_{1}+X_{2}+\cdots+X_{R\left(t\right)}=s,\left\{V\left(\tau\right),\tau<s\right\}\right\}=\prob\left\{V\left(t-s\right)\in A|X_{1}>t-s\right\},
\end{eqnarray*}
para todo $0\leq s\leq t$, donde $R\left(t\right)=\max\left\{X_{1}+X_{2}+\cdots+X_{j}\leq t\right\}=$n\'umero de renovaciones ({\emph{puntos de regeneraci\'on}}) que ocurren en $\left[0,t\right]$. El intervalo $\left[0,X_{1}\right)$ es llamado {\emph{primer ciclo de regeneraci\'on}} de $\left\{V\left(t \right),t\geq0\right\}$, $\left[X_{1},X_{1}+X_{2}\right)$ el {\emph{segundo ciclo de regeneraci\'on}}, y as\'i sucesivamente.

Sea $X=X_{1}$ y sea $F$ la funci\'on de distrbuci\'on de $X$


\begin{Def}
Se define el proceso estacionario, $\left\{V^{*}\left(t\right),t\geq0\right\}$, para $\left\{V\left(t\right),t\geq0\right\}$ por

\begin{eqnarray*}
\prob\left\{V\left(t\right)\in A\right\}=\frac{1}{\esp\left[X\right]}\int_{0}^{\infty}\prob\left\{V\left(t+x\right)\in A|X>x\right\}\left(1-F\left(x\right)\right)dx,
\end{eqnarray*} 
para todo $t\geq0$ y todo conjunto de Borel $A$.
\end{Def}

\begin{Def}
Una distribuci\'on se dice que es {\emph{aritm\'etica}} si todos sus puntos de incremento son m\'ultiplos de la forma $0,\lambda, 2\lambda,\ldots$ para alguna $\lambda>0$ entera.
\end{Def}


\begin{Def}
Una modificaci\'on medible de un proceso $\left\{V\left(t\right),t\geq0\right\}$, es una versi\'on de este, $\left\{V\left(t,w\right)\right\}$ conjuntamente medible para $t\geq0$ y para $w\in S$, $S$ espacio de estados para $\left\{V\left(t\right),t\geq0\right\}$.
\end{Def}

\begin{Teo}
Sea $\left\{V\left(t\right),t\geq\right\}$ un proceso regenerativo no negativo con modificaci\'on medible. Sea $\esp\left[X\right]<\infty$. Entonces el proceso estacionario dado por la ecuaci\'on anterior est\'a bien definido y tiene funci\'on de distribuci\'on independiente de $t$, adem\'as
\begin{itemize}
\item[i)] \begin{eqnarray*}
\esp\left[V^{*}\left(0\right)\right]&=&\frac{\esp\left[\int_{0}^{X}V\left(s\right)ds\right]}{\esp\left[X\right]}\end{eqnarray*}
\item[ii)] Si $\esp\left[V^{*}\left(0\right)\right]<\infty$, equivalentemente, si $\esp\left[\int_{0}^{X}V\left(s\right)ds\right]<\infty$,entonces
\begin{eqnarray*}
\frac{\int_{0}^{t}V\left(s\right)ds}{t}\rightarrow\frac{\esp\left[\int_{0}^{X}V\left(s\right)ds\right]}{\esp\left[X\right]}
\end{eqnarray*}
con probabilidad 1 y en media, cuando $t\rightarrow\infty$.
\end{itemize}
\end{Teo}

%______________________________________________________________________
\subsection{Procesos de Renovaci\'on}
%______________________________________________________________________

\begin{Def}\label{Def.Tn}
Sean $0\leq T_{1}\leq T_{2}\leq \ldots$ son tiempos aleatorios infinitos en los cuales ocurren ciertos eventos. El n\'umero de tiempos $T_{n}$ en el intervalo $\left[0,t\right)$ es

\begin{eqnarray}
N\left(t\right)=\sum_{n=1}^{\infty}\indora\left(T_{n}\leq t\right),
\end{eqnarray}
para $t\geq0$.
\end{Def}

Si se consideran los puntos $T_{n}$ como elementos de $\rea_{+}$, y $N\left(t\right)$ es el n\'umero de puntos en $\rea$. El proceso denotado por $\left\{N\left(t\right):t\geq0\right\}$, denotado por $N\left(t\right)$, es un proceso puntual en $\rea_{+}$. Los $T_{n}$ son los tiempos de ocurrencia, el proceso puntual $N\left(t\right)$ es simple si su n\'umero de ocurrencias son distintas: $0<T_{1}<T_{2}<\ldots$ casi seguramente.

\begin{Def}
Un proceso puntual $N\left(t\right)$ es un proceso de renovaci\'on si los tiempos de interocurrencia $\xi_{n}=T_{n}-T_{n-1}$, para $n\geq1$, son independientes e identicamente distribuidos con distribuci\'on $F$, donde $F\left(0\right)=0$ y $T_{0}=0$. Los $T_{n}$ son llamados tiempos de renovaci\'on, referente a la independencia o renovaci\'on de la informaci\'on estoc\'astica en estos tiempos. Los $\xi_{n}$ son los tiempos de inter-renovaci\'on, y $N\left(t\right)$ es el n\'umero de renovaciones en el intervalo $\left[0,t\right)$
\end{Def}


\begin{Note}
Para definir un proceso de renovaci\'on para cualquier contexto, solamente hay que especificar una distribuci\'on $F$, con $F\left(0\right)=0$, para los tiempos de inter-renovaci\'on. La funci\'on $F$ en turno degune las otra variables aleatorias. De manera formal, existe un espacio de probabilidad y una sucesi\'on de variables aleatorias $\xi_{1},\xi_{2},\ldots$ definidas en este con distribuci\'on $F$. Entonces las otras cantidades son $T_{n}=\sum_{k=1}^{n}\xi_{k}$ y $N\left(t\right)=\sum_{n=1}^{\infty}\indora\left(T_{n}\leq t\right)$, donde $T_{n}\rightarrow\infty$ casi seguramente por la Ley Fuerte de los Grandes Números.
\end{Note}

%___________________________________________________________________________________________
%
\subsection{Teorema Principal de Renovaci\'on}
%___________________________________________________________________________________________
%

\begin{Note} Una funci\'on $h:\rea_{+}\rightarrow\rea$ es Directamente Riemann Integrable en los siguientes casos:
\begin{itemize}
\item[a)] $h\left(t\right)\geq0$ es decreciente y Riemann Integrable.
\item[b)] $h$ es continua excepto posiblemente en un conjunto de Lebesgue de medida 0, y $|h\left(t\right)|\leq b\left(t\right)$, donde $b$ es DRI.
\end{itemize}
\end{Note}

\begin{Teo}[Teorema Principal de Renovaci\'on]
Si $F$ es no aritm\'etica y $h\left(t\right)$ es Directamente Riemann Integrable (DRI), entonces

\begin{eqnarray*}
lim_{t\rightarrow\infty}U\star h=\frac{1}{\mu}\int_{\rea_{+}}h\left(s\right)ds.
\end{eqnarray*}
\end{Teo}

\begin{Prop}
Cualquier funci\'on $H\left(t\right)$ acotada en intervalos finitos y que es 0 para $t<0$ puede expresarse como
\begin{eqnarray*}
H\left(t\right)=U\star h\left(t\right)\textrm{,  donde }h\left(t\right)=H\left(t\right)-F\star H\left(t\right)
\end{eqnarray*}
\end{Prop}

\begin{Def}
Un proceso estoc\'astico $X\left(t\right)$ es crudamente regenerativo en un tiempo aleatorio positivo $T$ si
\begin{eqnarray*}
\esp\left[X\left(T+t\right)|T\right]=\esp\left[X\left(t\right)\right]\textrm{, para }t\geq0,\end{eqnarray*}
y con las esperanzas anteriores finitas.
\end{Def}

\begin{Prop}
Sup\'ongase que $X\left(t\right)$ es un proceso crudamente regenerativo en $T$, que tiene distribuci\'on $F$. Si $\esp\left[X\left(t\right)\right]$ es acotado en intervalos finitos, entonces
\begin{eqnarray*}
\esp\left[X\left(t\right)\right]=U\star h\left(t\right)\textrm{,  donde }h\left(t\right)=\esp\left[X\left(t\right)\indora\left(T>t\right)\right].
\end{eqnarray*}
\end{Prop}

\begin{Teo}[Regeneraci\'on Cruda]
Sup\'ongase que $X\left(t\right)$ es un proceso con valores positivo crudamente regenerativo en $T$, y def\'inase $M=\sup\left\{|X\left(t\right)|:t\leq T\right\}$. Si $T$ es no aritm\'etico y $M$ y $MT$ tienen media finita, entonces
\begin{eqnarray*}
lim_{t\rightarrow\infty}\esp\left[X\left(t\right)\right]=\frac{1}{\mu}\int_{\rea_{+}}h\left(s\right)ds,
\end{eqnarray*}
donde $h\left(t\right)=\esp\left[X\left(t\right)\indora\left(T>t\right)\right]$.
\end{Teo}



%___________________________________________________________________________________________
%
\subsection{Funci\'on de Renovaci\'on}
%___________________________________________________________________________________________
%


\begin{Def}
Sea $h\left(t\right)$ funci\'on de valores reales en $\rea$ acotada en intervalos finitos e igual a cero para $t<0$ La ecuaci\'on de renovaci\'on para $h\left(t\right)$ y la distribuci\'on $F$ es

\begin{eqnarray}\label{Ec.Renovacion}
H\left(t\right)=h\left(t\right)+\int_{\left[0,t\right]}H\left(t-s\right)dF\left(s\right)\textrm{,    }t\geq0,
\end{eqnarray}
donde $H\left(t\right)$ es una funci\'on de valores reales. Esto es $H=h+F\star H$. Decimos que $H\left(t\right)$ es soluci\'on de esta ecuaci\'on si satisface la ecuaci\'on, y es acotada en intervalos finitos e iguales a cero para $t<0$.
\end{Def}

\begin{Prop}
La funci\'on $U\star h\left(t\right)$ es la \'unica soluci\'on de la ecuaci\'on de renovaci\'on (\ref{Ec.Renovacion}).
\end{Prop}

\begin{Teo}[Teorema Renovaci\'on Elemental]
\begin{eqnarray*}
t^{-1}U\left(t\right)\rightarrow 1/\mu\textrm{,    cuando }t\rightarrow\infty.
\end{eqnarray*}
\end{Teo}

%___________________________________________________________________________________________
%
\subsection{Propiedades de los Procesos de Renovaci\'on}
%___________________________________________________________________________________________
%

Los tiempos $T_{n}$ est\'an relacionados con los conteos de $N\left(t\right)$ por

\begin{eqnarray*}
\left\{N\left(t\right)\geq n\right\}&=&\left\{T_{n}\leq t\right\}\\
T_{N\left(t\right)}\leq &t&<T_{N\left(t\right)+1},
\end{eqnarray*}

adem\'as $N\left(T_{n}\right)=n$, y 

\begin{eqnarray*}
N\left(t\right)=\max\left\{n:T_{n}\leq t\right\}=\min\left\{n:T_{n+1}>t\right\}
\end{eqnarray*}

Por propiedades de la convoluci\'on se sabe que

\begin{eqnarray*}
P\left\{T_{n}\leq t\right\}=F^{n\star}\left(t\right)
\end{eqnarray*}
que es la $n$-\'esima convoluci\'on de $F$. Entonces 

\begin{eqnarray*}
\left\{N\left(t\right)\geq n\right\}&=&\left\{T_{n}\leq t\right\}\\
P\left\{N\left(t\right)\leq n\right\}&=&1-F^{\left(n+1\right)\star}\left(t\right)
\end{eqnarray*}

Adem\'as usando el hecho de que $\esp\left[N\left(t\right)\right]=\sum_{n=1}^{\infty}P\left\{N\left(t\right)\geq n\right\}$
se tiene que

\begin{eqnarray*}
\esp\left[N\left(t\right)\right]=\sum_{n=1}^{\infty}F^{n\star}\left(t\right)
\end{eqnarray*}

\begin{Prop}
Para cada $t\geq0$, la funci\'on generadora de momentos $\esp\left[e^{\alpha N\left(t\right)}\right]$ existe para alguna $\alpha$ en una vecindad del 0, y de aqu\'i que $\esp\left[N\left(t\right)^{m}\right]<\infty$, para $m\geq1$.
\end{Prop}


\begin{Note}
Si el primer tiempo de renovaci\'on $\xi_{1}$ no tiene la misma distribuci\'on que el resto de las $\xi_{n}$, para $n\geq2$, a $N\left(t\right)$ se le llama Proceso de Renovaci\'on retardado, donde si $\xi$ tiene distribuci\'on $G$, entonces el tiempo $T_{n}$ de la $n$-\'esima renovaci\'on tiene distribuci\'on $G\star F^{\left(n-1\right)\star}\left(t\right)$
\end{Note}


\begin{Teo}
Para una constante $\mu\leq\infty$ ( o variable aleatoria), las siguientes expresiones son equivalentes:

\begin{eqnarray}
lim_{n\rightarrow\infty}n^{-1}T_{n}&=&\mu,\textrm{ c.s.}\\
lim_{t\rightarrow\infty}t^{-1}N\left(t\right)&=&1/\mu,\textrm{ c.s.}
\end{eqnarray}
\end{Teo}


Es decir, $T_{n}$ satisface la Ley Fuerte de los Grandes N\'umeros s\'i y s\'olo s\'i $N\left/t\right)$ la cumple.


\begin{Coro}[Ley Fuerte de los Grandes N\'umeros para Procesos de Renovaci\'on]
Si $N\left(t\right)$ es un proceso de renovaci\'on cuyos tiempos de inter-renovaci\'on tienen media $\mu\leq\infty$, entonces
\begin{eqnarray}
t^{-1}N\left(t\right)\rightarrow 1/\mu,\textrm{ c.s. cuando }t\rightarrow\infty.
\end{eqnarray}

\end{Coro}


Considerar el proceso estoc\'astico de valores reales $\left\{Z\left(t\right):t\geq0\right\}$ en el mismo espacio de probabilidad que $N\left(t\right)$

\begin{Def}
Para el proceso $\left\{Z\left(t\right):t\geq0\right\}$ se define la fluctuaci\'on m\'axima de $Z\left(t\right)$ en el intervalo $\left(T_{n-1},T_{n}\right]$:
\begin{eqnarray*}
M_{n}=\sup_{T_{n-1}<t\leq T_{n}}|Z\left(t\right)-Z\left(T_{n-1}\right)|
\end{eqnarray*}
\end{Def}

\begin{Teo}
Sup\'ongase que $n^{-1}T_{n}\rightarrow\mu$ c.s. cuando $n\rightarrow\infty$, donde $\mu\leq\infty$ es una constante o variable aleatoria. Sea $a$ una constante o variable aleatoria que puede ser infinita cuando $\mu$ es finita, y considere las expresiones l\'imite:
\begin{eqnarray}
lim_{n\rightarrow\infty}n^{-1}Z\left(T_{n}\right)&=&a,\textrm{ c.s.}\\
lim_{t\rightarrow\infty}t^{-1}Z\left(t\right)&=&a/\mu,\textrm{ c.s.}
\end{eqnarray}
La segunda expresi\'on implica la primera. Conversamente, la primera implica la segunda si el proceso $Z\left(t\right)$ es creciente, o si $lim_{n\rightarrow\infty}n^{-1}M_{n}=0$ c.s.
\end{Teo}

\begin{Coro}
Si $N\left(t\right)$ es un proceso de renovaci\'on, y $\left(Z\left(T_{n}\right)-Z\left(T_{n-1}\right),M_{n}\right)$, para $n\geq1$, son variables aleatorias independientes e id\'enticamente distribuidas con media finita, entonces,
\begin{eqnarray}
lim_{t\rightarrow\infty}t^{-1}Z\left(t\right)\rightarrow\frac{\esp\left[Z\left(T_{1}\right)-Z\left(T_{0}\right)\right]}{\esp\left[T_{1}\right]},\textrm{ c.s. cuando  }t\rightarrow\infty.
\end{eqnarray}
\end{Coro}

%___________________________________________________________________________________________
%
\subsection{Funci\'on de Renovaci\'on}
%___________________________________________________________________________________________
%


Sup\'ongase que $N\left(t\right)$ es un proceso de renovaci\'on con distribuci\'on $F$ con media finita $\mu$.

\begin{Def}
La funci\'on de renovaci\'on asociada con la distribuci\'on $F$, del proceso $N\left(t\right)$, es
\begin{eqnarray*}
U\left(t\right)=\sum_{n=1}^{\infty}F^{n\star}\left(t\right),\textrm{   }t\geq0,
\end{eqnarray*}
donde $F^{0\star}\left(t\right)=\indora\left(t\geq0\right)$.
\end{Def}


\begin{Prop}
Sup\'ongase que la distribuci\'on de inter-renovaci\'on $F$ tiene densidad $f$. Entonces $U\left(t\right)$ tambi\'en tiene densidad, para $t>0$, y es $U^{'}\left(t\right)=\sum_{n=0}^{\infty}f^{n\star}\left(t\right)$. Adem\'as
\begin{eqnarray*}
\prob\left\{N\left(t\right)>N\left(t-\right)\right\}=0\textrm{,   }t\geq0.
\end{eqnarray*}
\end{Prop}

\begin{Def}
La Transformada de Laplace-Stieljes de $F$ est\'a dada por

\begin{eqnarray*}
\hat{F}\left(\alpha\right)=\int_{\rea_{+}}e^{-\alpha t}dF\left(t\right)\textrm{,  }\alpha\geq0.
\end{eqnarray*}
\end{Def}

Entonces

\begin{eqnarray*}
\hat{U}\left(\alpha\right)=\sum_{n=0}^{\infty}\hat{F^{n\star}}\left(\alpha\right)=\sum_{n=0}^{\infty}\hat{F}\left(\alpha\right)^{n}=\frac{1}{1-\hat{F}\left(\alpha\right)}.
\end{eqnarray*}


\begin{Prop}
La Transformada de Laplace $\hat{U}\left(\alpha\right)$ y $\hat{F}\left(\alpha\right)$ determina una a la otra de manera \'unica por la relaci\'on $\hat{U}\left(\alpha\right)=\frac{1}{1-\hat{F}\left(\alpha\right)}$.
\end{Prop}


\begin{Note}
Un proceso de renovaci\'on $N\left(t\right)$ cuyos tiempos de inter-renovaci\'on tienen media finita, es un proceso Poisson con tasa $\lambda$ si y s\'olo s\'i $\esp\left[U\left(t\right)\right]=\lambda t$, para $t\geq0$.
\end{Note}


\begin{Teo}
Sea $N\left(t\right)$ un proceso puntual simple con puntos de localizaci\'on $T_{n}$ tal que $\eta\left(t\right)=\esp\left[N\left(\right)\right]$ es finita para cada $t$. Entonces para cualquier funci\'on $f:\rea_{+}\rightarrow\rea$,
\begin{eqnarray*}
\esp\left[\sum_{n=1}^{N\left(\right)}f\left(T_{n}\right)\right]=\int_{\left(0,t\right]}f\left(s\right)d\eta\left(s\right)\textrm{,  }t\geq0,
\end{eqnarray*}
suponiendo que la integral exista. Adem\'as si $X_{1},X_{2},\ldots$ son variables aleatorias definidas en el mismo espacio de probabilidad que el proceso $N\left(t\right)$ tal que $\esp\left[X_{n}|T_{n}=s\right]=f\left(s\right)$, independiente de $n$. Entonces
\begin{eqnarray*}
\esp\left[\sum_{n=1}^{N\left(t\right)}X_{n}\right]=\int_{\left(0,t\right]}f\left(s\right)d\eta\left(s\right)\textrm{,  }t\geq0,
\end{eqnarray*} 
suponiendo que la integral exista. 
\end{Teo}

\begin{Coro}[Identidad de Wald para Renovaciones]
Para el proceso de renovaci\'on $N\left(t\right)$,
\begin{eqnarray*}
\esp\left[T_{N\left(t\right)+1}\right]=\mu\esp\left[N\left(t\right)+1\right]\textrm{,  }t\geq0,
\end{eqnarray*}  
\end{Coro}

%______________________________________________________________________
\subsection{Procesos de Renovaci\'on}
%______________________________________________________________________

\begin{Def}\label{Def.Tn}
Sean $0\leq T_{1}\leq T_{2}\leq \ldots$ son tiempos aleatorios infinitos en los cuales ocurren ciertos eventos. El n\'umero de tiempos $T_{n}$ en el intervalo $\left[0,t\right)$ es

\begin{eqnarray}
N\left(t\right)=\sum_{n=1}^{\infty}\indora\left(T_{n}\leq t\right),
\end{eqnarray}
para $t\geq0$.
\end{Def}

Si se consideran los puntos $T_{n}$ como elementos de $\rea_{+}$, y $N\left(t\right)$ es el n\'umero de puntos en $\rea$. El proceso denotado por $\left\{N\left(t\right):t\geq0\right\}$, denotado por $N\left(t\right)$, es un proceso puntual en $\rea_{+}$. Los $T_{n}$ son los tiempos de ocurrencia, el proceso puntual $N\left(t\right)$ es simple si su n\'umero de ocurrencias son distintas: $0<T_{1}<T_{2}<\ldots$ casi seguramente.

\begin{Def}
Un proceso puntual $N\left(t\right)$ es un proceso de renovaci\'on si los tiempos de interocurrencia $\xi_{n}=T_{n}-T_{n-1}$, para $n\geq1$, son independientes e identicamente distribuidos con distribuci\'on $F$, donde $F\left(0\right)=0$ y $T_{0}=0$. Los $T_{n}$ son llamados tiempos de renovaci\'on, referente a la independencia o renovaci\'on de la informaci\'on estoc\'astica en estos tiempos. Los $\xi_{n}$ son los tiempos de inter-renovaci\'on, y $N\left(t\right)$ es el n\'umero de renovaciones en el intervalo $\left[0,t\right)$
\end{Def}


\begin{Note}
Para definir un proceso de renovaci\'on para cualquier contexto, solamente hay que especificar una distribuci\'on $F$, con $F\left(0\right)=0$, para los tiempos de inter-renovaci\'on. La funci\'on $F$ en turno degune las otra variables aleatorias. De manera formal, existe un espacio de probabilidad y una sucesi\'on de variables aleatorias $\xi_{1},\xi_{2},\ldots$ definidas en este con distribuci\'on $F$. Entonces las otras cantidades son $T_{n}=\sum_{k=1}^{n}\xi_{k}$ y $N\left(t\right)=\sum_{n=1}^{\infty}\indora\left(T_{n}\leq t\right)$, donde $T_{n}\rightarrow\infty$ casi seguramente por la Ley Fuerte de los Grandes Números.
\end{Note}
%_____________________________________________________
\subsection{Puntos de Renovaci\'on}
%_____________________________________________________

Para cada cola $Q_{i}$ se tienen los procesos de arribo a la cola, para estas, los tiempos de arribo est\'an dados por $$\left\{T_{1}^{i},T_{2}^{i},\ldots,T_{k}^{i},\ldots\right\},$$ entonces, consideremos solamente los primeros tiempos de arribo a cada una de las colas, es decir, $$\left\{T_{1}^{1},T_{1}^{2},T_{1}^{3},T_{1}^{4}\right\},$$ se sabe que cada uno de estos tiempos se distribuye de manera exponencial con par\'ametro $1/mu_{i}$. Adem\'as se sabe que para $$T^{*}=\min\left\{T_{1}^{1},T_{1}^{2},T_{1}^{3},T_{1}^{4}\right\},$$ $T^{*}$ se distribuye de manera exponencial con par\'ametro $$\mu^{*}=\sum_{i=1}^{4}\mu_{i}.$$ Ahora, dado que 
\begin{center}
\begin{tabular}{lcl}
$\tilde{r}=r_{1}+r_{2}$ & y &$\hat{r}=r_{3}+r_{4}.$
\end{tabular}
\end{center}


Supongamos que $$\tilde{r},\hat{r}<\mu^{*},$$ entonces si tomamos $$r^{*}=\min\left\{\tilde{r},\hat{r}\right\},$$ se tiene que para  $$t^{*}\in\left(0,r^{*}\right)$$ se cumple que 
\begin{center}
\begin{tabular}{lcl}
$\tau_{1}\left(1\right)=0$ & y por tanto & $\overline{\tau}_{1}=0,$
\end{tabular}
\end{center}
entonces para la segunda cola en este primer ciclo se cumple que $$\tau_{2}=\overline{\tau}_{1}+r_{1}=r_{1}<\mu^{*},$$ y por tanto se tiene que  $$\overline{\tau}_{2}=\tau_{2}.$$ Por lo tanto, nuevamente para la primer cola en el segundo ciclo $$\tau_{1}\left(2\right)=\tau_{2}\left(1\right)+r_{2}=\tilde{r}<\mu^{*}.$$ An\'alogamente para el segundo sistema se tiene que ambas colas est\'an vac\'ias, es decir, existe un valor $t^{*}$ tal que en el intervalo $\left(0,t^{*}\right)$ no ha llegado ning\'un usuario, es decir, $$L_{i}\left(t^{*}\right)=0$$ para $i=1,2,3,4$.




%_______________________________________________________________________________________________________
\subsection{Ya revisado}
%_______________________________________________________________________________________________________


Def\'inanse los puntos de regenaraci\'on  en el proceso $\left[L_{1}\left(t\right),L_{2}\left(t\right),\ldots,L_{N}\left(t\right)\right]$. Los puntos cuando la cola $i$ es visitada y todos los $L_{j}\left(\tau_{i}\left(m\right)\right)=0$ para $i=1,2$  son puntos de regeneraci\'on. Se llama ciclo regenerativo al intervalo entre dos puntos regenerativos sucesivos.

Sea $M_{i}$  el n\'umero de ciclos de visita en un ciclo regenerativo, y sea $C_{i}^{(m)}$, para $m=1,2,\ldots,M_{i}$ la duraci\'on del $m$-\'esimo ciclo de visita en un ciclo regenerativo. Se define el ciclo del tiempo de visita promedio $\esp\left[C_{i}\right]$ como

\begin{eqnarray*}
\esp\left[C_{i}\right]&=&\frac{\esp\left[\sum_{m=1}^{M_{i}}C_{i}^{(m)}\right]}{\esp\left[M_{i}\right]}
\end{eqnarray*}




Sea la funci\'on generadora de momentos para $L_{i}$, el n\'umero de usuarios en la cola $Q_{i}\left(z\right)$ en cualquier momento, est\'a dada por el tiempo promedio de $z^{L_{i}\left(t\right)}$ sobre el ciclo regenerativo definido anteriormente:

\begin{eqnarray*}
Q_{i}\left(z\right)&=&\esp\left[z^{L_{i}\left(t\right)}\right]=\frac{\esp\left[\sum_{m=1}^{M_{i}}\sum_{t=\tau_{i}\left(m\right)}^{\tau_{i}\left(m+1\right)-1}z^{L_{i}\left(t\right)}\right]}{\esp\left[\sum_{m=1}^{M_{i}}\tau_{i}\left(m+1\right)-\tau_{i}\left(m\right)\right]}
\end{eqnarray*}

$M_{i}$ es un tiempo de paro en el proceso regenerativo con $\esp\left[M_{i}\right]<\infty$, se sigue del lema de Wald que:


\begin{eqnarray*}
\esp\left[\sum_{m=1}^{M_{i}}\sum_{t=\tau_{i}\left(m\right)}^{\tau_{i}\left(m+1\right)-1}z^{L_{i}\left(t\right)}\right]&=&\esp\left[M_{i}\right]\esp\left[\sum_{t=\tau_{i}\left(m\right)}^{\tau_{i}\left(m+1\right)-1}z^{L_{i}\left(t\right)}\right]\\
\esp\left[\sum_{m=1}^{M_{i}}\tau_{i}\left(m+1\right)-\tau_{i}\left(m\right)\right]&=&\esp\left[M_{i}\right]\esp\left[\tau_{i}\left(m+1\right)-\tau_{i}\left(m\right)\right]
\end{eqnarray*}

por tanto se tiene que


\begin{eqnarray*}
Q_{i}\left(z\right)&=&\frac{\esp\left[\sum_{t=\tau_{i}\left(m\right)}^{\tau_{i}\left(m+1\right)-1}z^{L_{i}\left(t\right)}\right]}{\esp\left[\tau_{i}\left(m+1\right)-\tau_{i}\left(m\right)\right]}
\end{eqnarray*}

observar que el denominador es simplemente la duraci\'on promedio del tiempo del ciclo.


Se puede demostrar (ver Hideaki Takagi 1986) que

\begin{eqnarray*}
\esp\left[\sum_{t=\tau_{i}\left(m\right)}^{\tau_{i}\left(m+1\right)-1}z^{L_{i}\left(t\right)}\right]=z\frac{F_{i}\left(z\right)-1}{z-P_{i}\left(z\right)}
\end{eqnarray*}

Durante el tiempo de intervisita para la cola $i$, $L_{i}\left(t\right)$ solamente se incrementa de manera que el incremento por intervalo de tiempo est\'a dado por la funci\'on generadora de probabilidades de $P_{i}\left(z\right)$, por tanto la suma sobre el tiempo de intervisita puede evaluarse como:

\begin{eqnarray*}
\esp\left[\sum_{t=\tau_{i}\left(m\right)}^{\tau_{i}\left(m+1\right)-1}z^{L_{i}\left(t\right)}\right]&=&\esp\left[\sum_{t=\tau_{i}\left(m\right)}^{\tau_{i}\left(m+1\right)-1}\left\{P_{i}\left(z\right)\right\}^{t-\overline{\tau}_{i}\left(m\right)}\right]=\frac{1-\esp\left[\left\{P_{i}\left(z\right)\right\}^{\tau_{i}\left(m+1\right)-\overline{\tau}_{i}\left(m\right)}\right]}{1-P_{i}\left(z\right)}\\
&=&\frac{1-I_{i}\left[P_{i}\left(z\right)\right]}{1-P_{i}\left(z\right)}
\end{eqnarray*}
por tanto

\begin{eqnarray*}
\esp\left[\sum_{t=\tau_{i}\left(m\right)}^{\tau_{i}\left(m+1\right)-1}z^{L_{i}\left(t\right)}\right]&=&\frac{1-F_{i}\left(z\right)}{1-P_{i}\left(z\right)}
\end{eqnarray*}

Haciendo uso de lo hasta ahora desarrollado se tiene que

\begin{eqnarray*}
Q_{i}\left(z\right)&=&\frac{1}{\esp\left[C_{i}\right]}\cdot\frac{1-F_{i}\left(z\right)}{P_{i}\left(z\right)-z}\cdot\frac{\left(1-z\right)P_{i}\left(z\right)}{1-P_{i}\left(z\right)}\\
&=&\frac{\mu_{i}\left(1-\mu_{i}\right)}{f_{i}\left(i\right)}\cdot\frac{1-F_{i}\left(z\right)}{P_{i}\left(z\right)-z}\cdot\frac{\left(1-z\right)P_{i}\left(z\right)}{1-P_{i}\left(z\right)}
\end{eqnarray*}

\begin{Def}
Sea $L_{i}^{*}$el n\'umero de usuarios en la cola $Q_{i}$ cuando es visitada por el servidor para dar servicio, entonces

\begin{eqnarray}
\esp\left[L_{i}^{*}\right]&=&f_{i}\left(i\right)\\
Var\left[L_{i}^{*}\right]&=&f_{i}\left(i,i\right)+\esp\left[L_{i}^{*}\right]-\esp\left[L_{i}^{*}\right]^{2}.
\end{eqnarray}

\end{Def}


\begin{Def}
El tiempo de intervisita $I_{i}$ es el periodo de tiempo que comienza cuando se ha completado el servicio en un ciclo y termina cuando es visitada nuevamente en el pr\'oximo ciclo. Su  duraci\'on del mismo est\'a dada por $\tau_{i}\left(m+1\right)-\overline{\tau}_{i}\left(m\right)$.
\end{Def}


Recordemos las siguientes expresiones:

\begin{eqnarray*}
S_{i}\left(z\right)&=&\esp\left[z^{\overline{\tau}_{i}\left(m\right)-\tau_{i}\left(m\right)}\right]=F_{i}\left(\theta\left(z\right)\right),\\
F\left(z\right)&=&\esp\left[z^{L_{0}}\right],\\
P\left(z\right)&=&\esp\left[z^{X_{n}}\right],\\
F_{i}\left(z\right)&=&\esp\left[z^{L_{i}\left(\tau_{i}\left(m\right)\right)}\right],
\theta_{i}\left(z\right)-zP_{i}
\end{eqnarray*}

entonces 

\begin{eqnarray*}
\esp\left[S_{i}\right]&=&\frac{\esp\left[L_{i}^{*}\right]}{1-\mu_{i}}=\frac{f_{i}\left(i\right)}{1-\mu_{i}},\\
Var\left[S_{i}\right]&=&\frac{Var\left[L_{i}^{*}\right]}{\left(1-\mu_{i}\right)^{2}}+\frac{\sigma^{2}\esp\left[L_{i}^{*}\right]}{\left(1-\mu_{i}\right)^{3}}
\end{eqnarray*}

donde recordemos que

\begin{eqnarray*}
Var\left[L_{i}^{*}\right]&=&f_{i}\left(i,i\right)+f_{i}\left(i\right)-f_{i}\left(i\right)^{2}.
\end{eqnarray*}

La duraci\'on del tiempo de intervisita es $\tau_{i}\left(m+1\right)-\overline{\tau}\left(m\right)$. Dado que el n\'umero de usuarios presentes en $Q_{i}$ al tiempo $t=\tau_{i}\left(m+1\right)$ es igual al n\'umero de arribos durante el intervalo de tiempo $\left[\overline{\tau}\left(m\right),\tau_{i}\left(m+1\right)\right]$ se tiene que


\begin{eqnarray*}
\esp\left[z_{i}^{L_{i}\left(\tau_{i}\left(m+1\right)\right)}\right]=\esp\left[\left\{P_{i}\left(z_{i}\right)\right\}^{\tau_{i}\left(m+1\right)-\overline{\tau}\left(m\right)}\right]
\end{eqnarray*}

entonces, si \begin{eqnarray*}I_{i}\left(z\right)&=&\esp\left[z^{\tau_{i}\left(m+1\right)-\overline{\tau}\left(m\right)}\right]\end{eqnarray*} se tienen que

\begin{eqnarray*}
F_{i}\left(z\right)=I_{i}\left[P_{i}\left(z\right)\right]
\end{eqnarray*}
para $i=1,2$, por tanto



\begin{eqnarray*}
\esp\left[L_{i}^{*}\right]&=&\mu_{i}\esp\left[I_{i}\right]\\
Var\left[L_{i}^{*}\right]&=&\mu_{i}^{2}Var\left[I_{i}\right]+\sigma^{2}\esp\left[I_{i}\right]
\end{eqnarray*}
para $i=1,2$, por tanto


\begin{eqnarray*}
\esp\left[I_{i}\right]&=&\frac{f_{i}\left(i\right)}{\mu_{i}},
\end{eqnarray*}
adem\'as

\begin{eqnarray*}
Var\left[I_{i}\right]&=&\frac{Var\left[L_{i}^{*}\right]}{\mu_{i}^{2}}-\frac{\sigma_{i}^{2}}{\mu_{i}^{2}}f_{i}\left(i\right).
\end{eqnarray*}


Si  $C_{i}\left(z\right)=\esp\left[z^{\overline{\tau}\left(m+1\right)-\overline{\tau}_{i}\left(m\right)}\right]$el tiempo de duraci\'on del ciclo, entonces, por lo hasta ahora establecido, se tiene que

\begin{eqnarray*}
C_{i}\left(z\right)=I_{i}\left[\theta_{i}\left(z\right)\right],
\end{eqnarray*}
entonces

\begin{eqnarray*}
\esp\left[C_{i}\right]&=&\esp\left[I_{i}\right]\esp\left[\theta_{i}\left(z\right)\right]=\frac{\esp\left[L_{i}^{*}\right]}{\mu_{i}}\frac{1}{1-\mu_{i}}=\frac{f_{i}\left(i\right)}{\mu_{i}\left(1-\mu_{i}\right)}\\
Var\left[C_{i}\right]&=&\frac{Var\left[L_{i}^{*}\right]}{\mu_{i}^{2}\left(1-\mu_{i}\right)^{2}}.
\end{eqnarray*}

Por tanto se tienen las siguientes igualdades


\begin{eqnarray*}
\esp\left[S_{i}\right]&=&\mu_{i}\esp\left[C_{i}\right],\\
\esp\left[I_{i}\right]&=&\left(1-\mu_{i}\right)\esp\left[C_{i}\right]\\
\end{eqnarray*}

derivando con respecto a $z$



\begin{eqnarray*}
\frac{d Q_{i}\left(z\right)}{d z}&=&\frac{\left(1-F_{i}\left(z\right)\right)P_{i}\left(z\right)}{\esp\left[C_{i}\right]\left(1-P_{i}\left(z\right)\right)\left(P_{i}\left(z\right)-z\right)}\\
&-&\frac{\left(1-z\right)P_{i}\left(z\right)F_{i}^{'}\left(z\right)}{\esp\left[C_{i}\right]\left(1-P_{i}\left(z\right)\right)\left(P_{i}\left(z\right)-z\right)}\\
&-&\frac{\left(1-z\right)\left(1-F_{i}\left(z\right)\right)P_{i}\left(z\right)\left(P_{i}^{'}\left(z\right)-1\right)}{\esp\left[C_{i}\right]\left(1-P_{i}\left(z\right)\right)\left(P_{i}\left(z\right)-z\right)^{2}}\\
&+&\frac{\left(1-z\right)\left(1-F_{i}\left(z\right)\right)P_{i}^{'}\left(z\right)}{\esp\left[C_{i}\right]\left(1-P_{i}\left(z\right)\right)\left(P_{i}\left(z\right)-z\right)}\\
&+&\frac{\left(1-z\right)\left(1-F_{i}\left(z\right)\right)P_{i}\left(z\right)P_{i}^{'}\left(z\right)}{\esp\left[C_{i}\right]\left(1-P_{i}\left(z\right)\right)^{2}\left(P_{i}\left(z\right)-z\right)}
\end{eqnarray*}

Calculando el l\'imite cuando $z\rightarrow1^{+}$:
\begin{eqnarray}
Q_{i}^{(1)}\left(z\right)=\lim_{z\rightarrow1^{+}}\frac{d Q_{i}\left(z\right)}{dz}&=&\lim_{z\rightarrow1}\frac{\left(1-F_{i}\left(z\right)\right)P_{i}\left(z\right)}{\esp\left[C_{i}\right]\left(1-P_{i}\left(z\right)\right)\left(P_{i}\left(z\right)-z\right)}\\
&-&\lim_{z\rightarrow1^{+}}\frac{\left(1-z\right)P_{i}\left(z\right)F_{i}^{'}\left(z\right)}{\esp\left[C_{i}\right]\left(1-P_{i}\left(z\right)\right)\left(P_{i}\left(z\right)-z\right)}\\
&-&\lim_{z\rightarrow1^{+}}\frac{\left(1-z\right)\left(1-F_{i}\left(z\right)\right)P_{i}\left(z\right)\left(P_{i}^{'}\left(z\right)-1\right)}{\esp\left[C_{i}\right]\left(1-P_{i}\left(z\right)\right)\left(P_{i}\left(z\right)-z\right)^{2}}\\
&+&\lim_{z\rightarrow1^{+}}\frac{\left(1-z\right)\left(1-F_{i}\left(z\right)\right)P_{i}^{'}\left(z\right)}{\esp\left[C_{i}\right]\left(1-P_{i}\left(z\right)\right)\left(P_{i}\left(z\right)-z\right)}\\
&+&\lim_{z\rightarrow1^{+}}\frac{\left(1-z\right)\left(1-F_{i}\left(z\right)\right)P_{i}\left(z\right)P_{i}^{'}\left(z\right)}{\esp\left[C_{i}\right]\left(1-P_{i}\left(z\right)\right)^{2}\left(P_{i}\left(z\right)-z\right)}
\end{eqnarray}

Entonces:
%______________________________________________________

\begin{eqnarray*}
\lim_{z\rightarrow1^{+}}\frac{\left(1-F_{i}\left(z\right)\right)P_{i}\left(z\right)}{\left(1-P_{i}\left(z\right)\right)\left(P_{i}\left(z\right)-z\right)}&=&\lim_{z\rightarrow1^{+}}\frac{\frac{d}{dz}\left[\left(1-F_{i}\left(z\right)\right)P_{i}\left(z\right)\right]}{\frac{d}{dz}\left[\left(1-P_{i}\left(z\right)\right)\left(-z+P_{i}\left(z\right)\right)\right]}\\
&=&\lim_{z\rightarrow1^{+}}\frac{-P_{i}\left(z\right)F_{i}^{'}\left(z\right)+\left(1-F_{i}\left(z\right)\right)P_{i}^{'}\left(z\right)}{\left(1-P_{i}\left(z\right)\right)\left(-1+P_{i}^{'}\left(z\right)\right)-\left(-z+P_{i}\left(z\right)\right)P_{i}^{'}\left(z\right)}
\end{eqnarray*}


%______________________________________________________


\begin{eqnarray*}
\lim_{z\rightarrow1^{+}}\frac{\left(1-z\right)P_{i}\left(z\right)F_{i}^{'}\left(z\right)}{\left(1-P_{i}\left(z\right)\right)\left(P_{i}\left(z\right)-z\right)}&=&\lim_{z\rightarrow1^{+}}\frac{\frac{d}{dz}\left[\left(1-z\right)P_{i}\left(z\right)F_{i}^{'}\left(z\right)\right]}{\frac{d}{dz}\left[\left(1-P_{i}\left(z\right)\right)\left(P_{i}\left(z\right)-z\right)\right]}\\
&=&\lim_{z\rightarrow1^{+}}\frac{-P_{i}\left(z\right) F_{i}^{'}\left(z\right)+(1-z) F_{i}^{'}\left(z\right) P_{i}^{'}\left(z\right)+(1-z) P_{i}\left(z\right)F_{i}^{''}\left(z\right)}{\left(1-P_{i}\left(z\right)\right)\left(-1+P_{i}^{'}\left(z\right)\right)-\left(-z+P_{i}\left(z\right)\right)P_{i}^{'}\left(z\right)}
\end{eqnarray*}


%______________________________________________________

\begin{eqnarray*}
&&\lim_{z\rightarrow1^{+}}\frac{\left(1-z\right)\left(1-F_{i}\left(z\right)\right)P_{i}\left(z\right)\left(P_{i}^{'}\left(z\right)-1\right)}{\left(1-P_{i}\left(z\right)\right)\left(P_{i}\left(z\right)-z\right)^{2}}=\lim_{z\rightarrow1^{+}}\frac{\frac{d}{dz}\left[\left(1-z\right)\left(1-F_{i}\left(z\right)\right)P_{i}\left(z\right)\left(P_{i}^{'}\left(z\right)-1\right)\right]}{\frac{d}{dz}\left[\left(1-P_{i}\left(z\right)\right)\left(P_{i}\left(z\right)-z\right)^{2}\right]}\\
&=&\lim_{z\rightarrow1^{+}}\frac{-\left(1-F_{i}\left(z\right)\right) P_{i}\left(z\right)\left(-1+P_{i}^{'}\left(z\right)\right)-(1-z) P_{i}\left(z\right)F_{i}^{'}\left(z\right)\left(-1+P_{i}^{'}\left(z\right)\right)}{2\left(1-P_{i}\left(z\right)\right)\left(-z+P_{i}\left(z\right)\right) \left(-1+P_{i}^{'}\left(z\right)\right)-\left(-z+P_{i}\left(z\right)\right)^2 P_{i}^{'}\left(z\right)}\\
&+&\lim_{z\rightarrow1^{+}}\frac{+(1-z) \left(1-F_{i}\left(z\right)\right) \left(-1+P_{i}^{'}\left(z\right)\right) P_{i}^{'}\left(z\right)}{{2\left(1-P_{i}\left(z\right)\right)\left(-z+P_{i}\left(z\right)\right) \left(-1+P_{i}^{'}\left(z\right)\right)-\left(-z+P_{i}\left(z\right)\right)^2 P_{i}^{'}\left(z\right)}}\\
&+&\lim_{z\rightarrow1^{+}}\frac{+(1-z) \left(1-F_{i}\left(z\right)\right) P_{i}\left(z\right)P_{i}^{''}\left(z\right)}{{2\left(1-P_{i}\left(z\right)\right)\left(-z+P_{i}\left(z\right)\right) \left(-1+P_{i}^{'}\left(z\right)\right)-\left(-z+P_{i}\left(z\right)\right)^2 P_{i}^{'}\left(z\right)}}
\end{eqnarray*}











%______________________________________________________
\begin{eqnarray*}
&&\lim_{z\rightarrow1^{+}}\frac{\left(1-z\right)\left(1-F_{i}\left(z\right)\right)P_{i}^{'}\left(z\right)}{\left(1-P_{i}\left(z\right)\right)\left(P_{i}\left(z\right)-z\right)}=\lim_{z\rightarrow1^{+}}\frac{\frac{d}{dz}\left[\left(1-z\right)\left(1-F_{i}\left(z\right)\right)P_{i}^{'}\left(z\right)\right]}{\frac{d}{dz}\left[\left(1-P_{i}\left(z\right)\right)\left(P_{i}\left(z\right)-z\right)\right]}\\
&=&\lim_{z\rightarrow1^{+}}\frac{-\left(1-F_{i}\left(z\right)\right) P_{i}^{'}\left(z\right)-(1-z) F_{i}^{'}\left(z\right) P_{i}^{'}\left(z\right)+(1-z) \left(1-F_{i}\left(z\right)\right) P_{i}^{''}\left(z\right)}{\left(1-P_{i}\left(z\right)\right) \left(-1+P_{i}^{'}\left(z\right)\right)-\left(-z+P_{i}\left(z\right)\right) P_{i}^{'}\left(z\right)}\frac{}{}
\end{eqnarray*}

%______________________________________________________
\begin{eqnarray*}
&&\lim_{z\rightarrow1^{+}}\frac{\left(1-z\right)\left(1-F_{i}\left(z\right)\right)P_{i}\left(z\right)P_{i}^{'}\left(z\right)}{\left(1-P_{i}\left(z\right)\right)^{2}\left(P_{i}\left(z\right)-z\right)}=\lim_{z\rightarrow1^{+}}\frac{\frac{d}{dz}\left[\left(1-z\right)\left(1-F_{i}\left(z\right)\right)P_{i}\left(z\right)P_{i}^{'}\left(z\right)\right]}{\frac{d}{dz}\left[\left(1-P_{i}\left(z\right)\right)^{2}\left(P_{i}\left(z\right)-z\right)\right]}\\
&=&\lim_{z\rightarrow1^{+}}\frac{-\left(1-F_{i}\left(z\right)\right) P_{i}\left(z\right) P_{i}^{'}\left(z\right)-(1-z) P_{i}\left(z\right) F_{i}^{'}\left(z\right)P_i'[z]}{\left(1-P_{i}\left(z\right)\right)^2 \left(-1+P_{i}^{'}\left(z\right)\right)-2 \left(1-P_{i}\left(z\right)\right) \left(-z+P_{i}\left(z\right)\right) P_{i}^{'}\left(z\right)}\\
&+&\lim_{z\rightarrow1^{+}}\frac{(1-z) \left(1-F_{i}\left(z\right)\right) P_{i}^{'}\left(z\right)^2+(1-z) \left(1-F_{i}\left(z\right)\right) P_{i}\left(z\right) P_{i}^{''}\left(z\right)}{\left(1-P_{i}\left(z\right)\right)^2 \left(-1+P_{i}^{'}\left(z\right)\right)-2 \left(1-P_{i}\left(z\right)\right) \left(-z+P_{i}\left(z\right)\right) P_{i}^{'}\left(z\right)}\\
\end{eqnarray*}



En nuestra notaci\'on $V\left(t\right)\equiv C_{i}$ y $X_{i}=C_{i}^{(m)}$ para nuestra segunda definici\'on, mientras que para la primera la notaci\'on es: $X\left(t\right)\equiv C_{i}$ y $R_{i}\equiv C_{i}^{(m)}$.


%___________________________________________________________________________________________
%\section{Tiempos de Ciclo e Intervisita}
%___________________________________________________________________________________________


\begin{Def}
Sea $L_{i}^{*}$el n\'umero de usuarios en la cola $Q_{i}$ cuando es visitada por el servidor para dar servicio, entonces

\begin{eqnarray}
\esp\left[L_{i}^{*}\right]&=&f_{i}\left(i\right)\\
Var\left[L_{i}^{*}\right]&=&f_{i}\left(i,i\right)+\esp\left[L_{i}^{*}\right]-\esp\left[L_{i}^{*}\right]^{2}.
\end{eqnarray}

\end{Def}

\begin{Def}
El tiempo de Ciclo $C_{i}$ es e periodo de tiempo que comienza cuando la cola $i$ es visitada por primera vez en un ciclo, y termina cuando es visitado nuevamente en el pr\'oximo ciclo. La duraci\'on del mismo est\'a dada por $\tau_{i}\left(m+1\right)-\tau_{i}\left(m\right)$, o equivalentemente $\overline{\tau}_{i}\left(m+1\right)-\overline{\tau}_{i}\left(m\right)$ bajo condiciones de estabilidad.
\end{Def}

\begin{Def}
El tiempo de intervisita $I_{i}$ es el periodo de tiempo que comienza cuando se ha completado el servicio en un ciclo y termina cuando es visitada nuevamente en el pr\'oximo ciclo. Su  duraci\'on del mismo est\'a dada por $\tau_{i}\left(m+1\right)-\overline{\tau}_{i}\left(m\right)$.
\end{Def}


Recordemos las siguientes expresiones:

\begin{eqnarray*}
S_{i}\left(z\right)&=&\esp\left[z^{\overline{\tau}_{i}\left(m\right)-\tau_{i}\left(m\right)}\right]=F_{i}\left(\theta\left(z\right)\right),\\
F\left(z\right)&=&\esp\left[z^{L_{0}}\right],\\
P\left(z\right)&=&\esp\left[z^{X_{n}}\right],\\
F_{i}\left(z\right)&=&\esp\left[z^{L_{i}\left(\tau_{i}\left(m\right)\right)}\right],
\theta_{i}\left(z\right)-zP_{i}
\end{eqnarray*}

entonces 

\begin{eqnarray*}
\esp\left[S_{i}\right]&=&\frac{\esp\left[L_{i}^{*}\right]}{1-\mu_{i}}=\frac{f_{i}\left(i\right)}{1-\mu_{i}},\\
Var\left[S_{i}\right]&=&\frac{Var\left[L_{i}^{*}\right]}{\left(1-\mu_{i}\right)^{2}}+\frac{\sigma^{2}\esp\left[L_{i}^{*}\right]}{\left(1-\mu_{i}\right)^{3}}
\end{eqnarray*}

donde recordemos que

\begin{eqnarray*}
Var\left[L_{i}^{*}\right]&=&f_{i}\left(i,i\right)+f_{i}\left(i\right)-f_{i}\left(i\right)^{2}.
\end{eqnarray*}

La duraci\'on del tiempo de intervisita es $\tau_{i}\left(m+1\right)-\overline{\tau}\left(m\right)$. Dado que el n\'umero de usuarios presentes en $Q_{i}$ al tiempo $t=\tau_{i}\left(m+1\right)$ es igual al n\'umero de arribos durante el intervalo de tiempo $\left[\overline{\tau}\left(m\right),\tau_{i}\left(m+1\right)\right]$ se tiene que


\begin{eqnarray*}
\esp\left[z_{i}^{L_{i}\left(\tau_{i}\left(m+1\right)\right)}\right]=\esp\left[\left\{P_{i}\left(z_{i}\right)\right\}^{\tau_{i}\left(m+1\right)-\overline{\tau}\left(m\right)}\right]
\end{eqnarray*}

entonces, si \begin{eqnarray*}I_{i}\left(z\right)&=&\esp\left[z^{\tau_{i}\left(m+1\right)-\overline{\tau}\left(m\right)}\right]\end{eqnarray*} se tienen que

\begin{eqnarray*}
F_{i}\left(z\right)=I_{i}\left[P_{i}\left(z\right)\right]
\end{eqnarray*}
para $i=1,2$, por tanto



\begin{eqnarray*}
\esp\left[L_{i}^{*}\right]&=&\mu_{i}\esp\left[I_{i}\right]\\
Var\left[L_{i}^{*}\right]&=&\mu_{i}^{2}Var\left[I_{i}\right]+\sigma^{2}\esp\left[I_{i}\right]
\end{eqnarray*}
para $i=1,2$, por tanto


\begin{eqnarray*}
\esp\left[I_{i}\right]&=&\frac{f_{i}\left(i\right)}{\mu_{i}},
\end{eqnarray*}
adem\'as

\begin{eqnarray*}
Var\left[I_{i}\right]&=&\frac{Var\left[L_{i}^{*}\right]}{\mu_{i}^{2}}-\frac{\sigma_{i}^{2}}{\mu_{i}^{2}}f_{i}\left(i\right).
\end{eqnarray*}


Si  $C_{i}\left(z\right)=\esp\left[z^{\overline{\tau}\left(m+1\right)-\overline{\tau}_{i}\left(m\right)}\right]$el tiempo de duraci\'on del ciclo, entonces, por lo hasta ahora establecido, se tiene que

\begin{eqnarray*}
C_{i}\left(z\right)=I_{i}\left[\theta_{i}\left(z\right)\right],
\end{eqnarray*}
entonces

\begin{eqnarray*}
\esp\left[C_{i}\right]&=&\esp\left[I_{i}\right]\esp\left[\theta_{i}\left(z\right)\right]=\frac{\esp\left[L_{i}^{*}\right]}{\mu_{i}}\frac{1}{1-\mu_{i}}=\frac{f_{i}\left(i\right)}{\mu_{i}\left(1-\mu_{i}\right)}\\
Var\left[C_{i}\right]&=&\frac{Var\left[L_{i}^{*}\right]}{\mu_{i}^{2}\left(1-\mu_{i}\right)^{2}}.
\end{eqnarray*}

Por tanto se tienen las siguientes igualdades


\begin{eqnarray*}
\esp\left[S_{i}\right]&=&\mu_{i}\esp\left[C_{i}\right],\\
\esp\left[I_{i}\right]&=&\left(1-\mu_{i}\right)\esp\left[C_{i}\right]\\
\end{eqnarray*}

Def\'inanse los puntos de regenaraci\'on  en el proceso $\left[L_{1}\left(t\right),L_{2}\left(t\right),\ldots,L_{N}\left(t\right)\right]$. Los puntos cuando la cola $i$ es visitada y todos los $L_{j}\left(\tau_{i}\left(m\right)\right)=0$ para $i=1,2$  son puntos de regeneraci\'on. Se llama ciclo regenerativo al intervalo entre dos puntos regenerativos sucesivos.

Sea $M_{i}$  el n\'umero de ciclos de visita en un ciclo regenerativo, y sea $C_{i}^{(m)}$, para $m=1,2,\ldots,M_{i}$ la duraci\'on del $m$-\'esimo ciclo de visita en un ciclo regenerativo. Se define el ciclo del tiempo de visita promedio $\esp\left[C_{i}\right]$ como

\begin{eqnarray*}
\esp\left[C_{i}\right]&=&\frac{\esp\left[\sum_{m=1}^{M_{i}}C_{i}^{(m)}\right]}{\esp\left[M_{i}\right]}
\end{eqnarray*}


En Stid72 y Heym82 se muestra que una condici\'on suficiente para que el proceso regenerativo 
estacionario sea un procesoo estacionario es que el valor esperado del tiempo del ciclo regenerativo sea finito:

\begin{eqnarray*}
\esp\left[\sum_{m=1}^{M_{i}}C_{i}^{(m)}\right]<\infty.
\end{eqnarray*}

como cada $C_{i}^{(m)}$ contiene intervalos de r\'eplica positivos, se tiene que $\esp\left[M_{i}\right]<\infty$, adem\'as, como $M_{i}>0$, se tiene que la condici\'on anterior es equivalente a tener que 

\begin{eqnarray*}
\esp\left[C_{i}\right]<\infty,
\end{eqnarray*}
por lo tanto una condici\'on suficiente para la existencia del proceso regenerativo est\'a dada por

\begin{eqnarray*}
\sum_{k=1}^{N}\mu_{k}<1.
\end{eqnarray*}



\begin{Note}\label{Cita1.Stidham}
En Stidham\cite{Stidham} y Heyman  se muestra que una condici\'on suficiente para que el proceso regenerativo 
estacionario sea un procesoo estacionario es que el valor esperado del tiempo del ciclo regenerativo sea finito:

\begin{eqnarray*}
\esp\left[\sum_{m=1}^{M_{i}}C_{i}^{(m)}\right]<\infty.
\end{eqnarray*}

como cada $C_{i}^{(m)}$ contiene intervalos de r\'eplica positivos, se tiene que $\esp\left[M_{i}\right]<\infty$, adem\'as, como $M_{i}>0$, se tiene que la condici\'on anterior es equivalente a tener que 

\begin{eqnarray*}
\esp\left[C_{i}\right]<\infty,
\end{eqnarray*}
por lo tanto una condici\'on suficiente para la existencia del proceso regenerativo est\'a dada por

\begin{eqnarray*}
\sum_{k=1}^{N}\mu_{k}<1.
\end{eqnarray*}

{\centering{\Huge{\textbf{Nota incompleta!!}}}}
\end{Note}

%_______________________________________________________________________________________
\subsection{Procesos de Renovaci\'on y Regenerativos}
%_______________________________________________________________________________________



Se puede demostrar (ver Hideaki Takagi 1986) que

\begin{eqnarray*}
\esp\left[\sum_{t=\tau_{i}\left(m\right)}^{\tau_{i}\left(m+1\right)-1}z^{L_{i}\left(t\right)}\right]=z\frac{F_{i}\left(z\right)-1}{z-P_{i}\left(z\right)}
\end{eqnarray*}

Durante el tiempo de intervisita para la cola $i$, $L_{i}\left(t\right)$ solamente se incrementa de manera que el incremento por intervalo de tiempo est\'a dado por la funci\'on generadora de probabilidades de $P_{i}\left(z\right)$, por tanto la suma sobre el tiempo de intervisita puede evaluarse como:

\begin{eqnarray*}
\esp\left[\sum_{t=\tau_{i}\left(m\right)}^{\tau_{i}\left(m+1\right)-1}z^{L_{i}\left(t\right)}\right]&=&\esp\left[\sum_{t=\tau_{i}\left(m\right)}^{\tau_{i}\left(m+1\right)-1}\left\{P_{i}\left(z\right)\right\}^{t-\overline{\tau}_{i}\left(m\right)}\right]=\frac{1-\esp\left[\left\{P_{i}\left(z\right)\right\}^{\tau_{i}\left(m+1\right)-\overline{\tau}_{i}\left(m\right)}\right]}{1-P_{i}\left(z\right)}\\
&=&\frac{1-I_{i}\left[P_{i}\left(z\right)\right]}{1-P_{i}\left(z\right)}
\end{eqnarray*}
por tanto

\begin{eqnarray*}
\esp\left[\sum_{t=\tau_{i}\left(m\right)}^{\tau_{i}\left(m+1\right)-1}z^{L_{i}\left(t\right)}\right]&=&\frac{1-F_{i}\left(z\right)}{1-P_{i}\left(z\right)}
\end{eqnarray*}

Haciendo uso de lo hasta ahora desarrollado se tiene que



%___________________________________________________________________________________________
%\subsection{Longitudes de la Cola en cualquier tiempo}
%___________________________________________________________________________________________
Sea 
\begin{eqnarray*}
Q_{i}\left(z\right)&=&\frac{1}{\esp\left[C_{i}\right]}\cdot\frac{1-F_{i}\left(z\right)}{P_{i}\left(z\right)-z}\cdot\frac{\left(1-z\right)P_{i}\left(z\right)}{1-P_{i}\left(z\right)}
\end{eqnarray*}

Consideremos una cola de la red de sistemas de visitas c\'iclicas fija, $Q_{l}$.


Conforme a la definici\'on dada al principio del cap\'itulo, definici\'on (\ref{Def.Tn}), sean $T_{1},T_{2},\ldots$ los puntos donde las longitudes de las colas de la red de sistemas de visitas c\'iclicas son cero simult\'aneamente, cuando la cola $Q_{l}$ es visitada por el servidor para dar servicio, es decir, $L_{1}\left(T_{i}\right)=0,L_{2}\left(T_{i}\right)=0,\hat{L}_{1}\left(T_{i}\right)=0$ y $\hat{L}_{2}\left(T_{i}\right)=0$, a estos puntos se les denominar\'a puntos regenerativos. Entonces, 

\begin{Def}
Al intervalo de tiempo entre dos puntos regenerativos se le llamar\'a ciclo regenerativo.
\end{Def}

\begin{Def}
Para $T_{i}$ se define, $M_{i}$, el n\'umero de ciclos de visita a la cola $Q_{l}$, durante el ciclo regenerativo, es decir, $M_{i}$ es un proceso de renovaci\'on.
\end{Def}

\begin{Def}
Para cada uno de los $M_{i}$'s, se definen a su vez la duraci\'on de cada uno de estos ciclos de visita en el ciclo regenerativo, $C_{i}^{(m)}$, para $m=1,2,\ldots,M_{i}$, que a su vez, tambi\'en es n proceso de renovaci\'on.
\end{Def}

En nuestra notaci\'on $V\left(t\right)\equiv C_{i}$ y $X_{i}=C_{i}^{(m)}$ para nuestra segunda definici\'on, mientras que para la primera la notaci\'on es: $X\left(t\right)\equiv C_{i}$ y $R_{i}\equiv C_{i}^{(m)}$.


%___________________________________________________________________________________________
%\subsection{Tiempos de Ciclo e Intervisita}
%___________________________________________________________________________________________


\begin{Def}
Sea $L_{i}^{*}$el n\'umero de usuarios en la cola $Q_{i}$ cuando es visitada por el servidor para dar servicio, entonces

\begin{eqnarray}
\esp\left[L_{i}^{*}\right]&=&f_{i}\left(i\right)\\
Var\left[L_{i}^{*}\right]&=&f_{i}\left(i,i\right)+\esp\left[L_{i}^{*}\right]-\esp\left[L_{i}^{*}\right]^{2}.
\end{eqnarray}

\end{Def}

\begin{Def}
El tiempo de Ciclo $C_{i}$ es e periodo de tiempo que comienza cuando la cola $i$ es visitada por primera vez en un ciclo, y termina cuando es visitado nuevamente en el pr\'oximo ciclo. La duraci\'on del mismo est\'a dada por $\tau_{i}\left(m+1\right)-\tau_{i}\left(m\right)$, o equivalentemente $\overline{\tau}_{i}\left(m+1\right)-\overline{\tau}_{i}\left(m\right)$ bajo condiciones de estabilidad.
\end{Def}



Recordemos las siguientes expresiones:

\begin{eqnarray*}
S_{i}\left(z\right)&=&\esp\left[z^{\overline{\tau}_{i}\left(m\right)-\tau_{i}\left(m\right)}\right]=F_{i}\left(\theta\left(z\right)\right),\\
F\left(z\right)&=&\esp\left[z^{L_{0}}\right],\\
P\left(z\right)&=&\esp\left[z^{X_{n}}\right],\\
F_{i}\left(z\right)&=&\esp\left[z^{L_{i}\left(\tau_{i}\left(m\right)\right)}\right],
\theta_{i}\left(z\right)-zP_{i}
\end{eqnarray*}

entonces 

\begin{eqnarray*}
\esp\left[S_{i}\right]&=&\frac{\esp\left[L_{i}^{*}\right]}{1-\mu_{i}}=\frac{f_{i}\left(i\right)}{1-\mu_{i}},\\
Var\left[S_{i}\right]&=&\frac{Var\left[L_{i}^{*}\right]}{\left(1-\mu_{i}\right)^{2}}+\frac{\sigma^{2}\esp\left[L_{i}^{*}\right]}{\left(1-\mu_{i}\right)^{3}}
\end{eqnarray*}

donde recordemos que

\begin{eqnarray*}
Var\left[L_{i}^{*}\right]&=&f_{i}\left(i,i\right)+f_{i}\left(i\right)-f_{i}\left(i\right)^{2}.
\end{eqnarray*}

 por tanto


\begin{eqnarray*}
\esp\left[I_{i}\right]&=&\frac{f_{i}\left(i\right)}{\mu_{i}},
\end{eqnarray*}
adem\'as

\begin{eqnarray*}
Var\left[I_{i}\right]&=&\frac{Var\left[L_{i}^{*}\right]}{\mu_{i}^{2}}-\frac{\sigma_{i}^{2}}{\mu_{i}^{2}}f_{i}\left(i\right).
\end{eqnarray*}


Si  $C_{i}\left(z\right)=\esp\left[z^{\overline{\tau}\left(m+1\right)-\overline{\tau}_{i}\left(m\right)}\right]$el tiempo de duraci\'on del ciclo, entonces, por lo hasta ahora establecido, se tiene que

\begin{eqnarray*}
C_{i}\left(z\right)=I_{i}\left[\theta_{i}\left(z\right)\right],
\end{eqnarray*}
entonces

\begin{eqnarray*}
\esp\left[C_{i}\right]&=&\esp\left[I_{i}\right]\esp\left[\theta_{i}\left(z\right)\right]=\frac{\esp\left[L_{i}^{*}\right]}{\mu_{i}}\frac{1}{1-\mu_{i}}=\frac{f_{i}\left(i\right)}{\mu_{i}\left(1-\mu_{i}\right)}\\
Var\left[C_{i}\right]&=&\frac{Var\left[L_{i}^{*}\right]}{\mu_{i}^{2}\left(1-\mu_{i}\right)^{2}}.
\end{eqnarray*}

Por tanto se tienen las siguientes igualdades


\begin{eqnarray*}
\esp\left[S_{i}\right]&=&\mu_{i}\esp\left[C_{i}\right],\\
\esp\left[I_{i}\right]&=&\left(1-\mu_{i}\right)\esp\left[C_{i}\right]\\
\end{eqnarray*}

Def\'inanse los puntos de regenaraci\'on  en el proceso $\left[L_{1}\left(t\right),L_{2}\left(t\right),\ldots,L_{N}\left(t\right)\right]$. Los puntos cuando la cola $i$ es visitada y todos los $L_{j}\left(\tau_{i}\left(m\right)\right)=0$ para $i=1,2$  son puntos de regeneraci\'on. Se llama ciclo regenerativo al intervalo entre dos puntos regenerativos sucesivos.

Sea $M_{i}$  el n\'umero de ciclos de visita en un ciclo regenerativo, y sea $C_{i}^{(m)}$, para $m=1,2,\ldots,M_{i}$ la duraci\'on del $m$-\'esimo ciclo de visita en un ciclo regenerativo. Se define el ciclo del tiempo de visita promedio $\esp\left[C_{i}\right]$ como

\begin{eqnarray*}
\esp\left[C_{i}\right]&=&\frac{\esp\left[\sum_{m=1}^{M_{i}}C_{i}^{(m)}\right]}{\esp\left[M_{i}\right]}
\end{eqnarray*}


En Stid72 y Heym82 se muestra que una condici\'on suficiente para que el proceso regenerativo 
estacionario sea un procesoo estacionario es que el valor esperado del tiempo del ciclo regenerativo sea finito:

\begin{eqnarray*}
\esp\left[\sum_{m=1}^{M_{i}}C_{i}^{(m)}\right]<\infty.
\end{eqnarray*}

como cada $C_{i}^{(m)}$ contiene intervalos de r\'eplica positivos, se tiene que $\esp\left[M_{i}\right]<\infty$, adem\'as, como $M_{i}>0$, se tiene que la condici\'on anterior es equivalente a tener que 

\begin{eqnarray*}
\esp\left[C_{i}\right]<\infty,
\end{eqnarray*}
por lo tanto una condici\'on suficiente para la existencia del proceso regenerativo est\'a dada por

\begin{eqnarray*}
\sum_{k=1}^{N}\mu_{k}<1.
\end{eqnarray*}

Sea la funci\'on generadora de momentos para $L_{i}$, el n\'umero de usuarios en la cola $Q_{i}\left(z\right)$ en cualquier momento, est\'a dada por el tiempo promedio de $z^{L_{i}\left(t\right)}$ sobre el ciclo regenerativo definido anteriormente:

\begin{eqnarray*}
Q_{i}\left(z\right)&=&\esp\left[z^{L_{i}\left(t\right)}\right]=\frac{\esp\left[\sum_{m=1}^{M_{i}}\sum_{t=\tau_{i}\left(m\right)}^{\tau_{i}\left(m+1\right)-1}z^{L_{i}\left(t\right)}\right]}{\esp\left[\sum_{m=1}^{M_{i}}\tau_{i}\left(m+1\right)-\tau_{i}\left(m\right)\right]}
\end{eqnarray*}

$M_{i}$ es un tiempo de paro en el proceso regenerativo con $\esp\left[M_{i}\right]<\infty$, se sigue del lema de Wald que:


\begin{eqnarray*}
\esp\left[\sum_{m=1}^{M_{i}}\sum_{t=\tau_{i}\left(m\right)}^{\tau_{i}\left(m+1\right)-1}z^{L_{i}\left(t\right)}\right]&=&\esp\left[M_{i}\right]\esp\left[\sum_{t=\tau_{i}\left(m\right)}^{\tau_{i}\left(m+1\right)-1}z^{L_{i}\left(t\right)}\right]\\
\esp\left[\sum_{m=1}^{M_{i}}\tau_{i}\left(m+1\right)-\tau_{i}\left(m\right)\right]&=&\esp\left[M_{i}\right]\esp\left[\tau_{i}\left(m+1\right)-\tau_{i}\left(m\right)\right]
\end{eqnarray*}

por tanto se tiene que


\begin{eqnarray*}
Q_{i}\left(z\right)&=&\frac{\esp\left[\sum_{t=\tau_{i}\left(m\right)}^{\tau_{i}\left(m+1\right)-1}z^{L_{i}\left(t\right)}\right]}{\esp\left[\tau_{i}\left(m+1\right)-\tau_{i}\left(m\right)\right]}
\end{eqnarray*}

observar que el denominador es simplemente la duraci\'on promedio del tiempo del ciclo.


Se puede demostrar (ver Hideaki Takagi 1986) que

\begin{eqnarray*}
\esp\left[\sum_{t=\tau_{i}\left(m\right)}^{\tau_{i}\left(m+1\right)-1}z^{L_{i}\left(t\right)}\right]=z\frac{F_{i}\left(z\right)-1}{z-P_{i}\left(z\right)}
\end{eqnarray*}

Durante el tiempo de intervisita para la cola $i$, $L_{i}\left(t\right)$ solamente se incrementa de manera que el incremento por intervalo de tiempo est\'a dado por la funci\'on generadora de probabilidades de $P_{i}\left(z\right)$, por tanto la suma sobre el tiempo de intervisita puede evaluarse como:

\begin{eqnarray*}
\esp\left[\sum_{t=\tau_{i}\left(m\right)}^{\tau_{i}\left(m+1\right)-1}z^{L_{i}\left(t\right)}\right]&=&\esp\left[\sum_{t=\tau_{i}\left(m\right)}^{\tau_{i}\left(m+1\right)-1}\left\{P_{i}\left(z\right)\right\}^{t-\overline{\tau}_{i}\left(m\right)}\right]=\frac{1-\esp\left[\left\{P_{i}\left(z\right)\right\}^{\tau_{i}\left(m+1\right)-\overline{\tau}_{i}\left(m\right)}\right]}{1-P_{i}\left(z\right)}\\
&=&\frac{1-I_{i}\left[P_{i}\left(z\right)\right]}{1-P_{i}\left(z\right)}
\end{eqnarray*}
por tanto

\begin{eqnarray*}
\esp\left[\sum_{t=\tau_{i}\left(m\right)}^{\tau_{i}\left(m+1\right)-1}z^{L_{i}\left(t\right)}\right]&=&\frac{1-F_{i}\left(z\right)}{1-P_{i}\left(z\right)}
\end{eqnarray*}

Haciendo uso de lo hasta ahora desarrollado se tiene que

\begin{eqnarray*}
Q_{i}\left(z\right)&=&\frac{1}{\esp\left[C_{i}\right]}\cdot\frac{1-F_{i}\left(z\right)}{P_{i}\left(z\right)-z}\cdot\frac{\left(1-z\right)P_{i}\left(z\right)}{1-P_{i}\left(z\right)}\\
&=&\frac{\mu_{i}\left(1-\mu_{i}\right)}{f_{i}\left(i\right)}\cdot\frac{1-F_{i}\left(z\right)}{P_{i}\left(z\right)-z}\cdot\frac{\left(1-z\right)P_{i}\left(z\right)}{1-P_{i}\left(z\right)}
\end{eqnarray*}

derivando con respecto a $z$



\begin{eqnarray*}
\frac{d Q_{i}\left(z\right)}{d z}&=&\frac{\left(1-F_{i}\left(z\right)\right)P_{i}\left(z\right)}{\esp\left[C_{i}\right]\left(1-P_{i}\left(z\right)\right)\left(P_{i}\left(z\right)-z\right)}\\
&-&\frac{\left(1-z\right)P_{i}\left(z\right)F_{i}^{'}\left(z\right)}{\esp\left[C_{i}\right]\left(1-P_{i}\left(z\right)\right)\left(P_{i}\left(z\right)-z\right)}\\
&-&\frac{\left(1-z\right)\left(1-F_{i}\left(z\right)\right)P_{i}\left(z\right)\left(P_{i}^{'}\left(z\right)-1\right)}{\esp\left[C_{i}\right]\left(1-P_{i}\left(z\right)\right)\left(P_{i}\left(z\right)-z\right)^{2}}\\
&+&\frac{\left(1-z\right)\left(1-F_{i}\left(z\right)\right)P_{i}^{'}\left(z\right)}{\esp\left[C_{i}\right]\left(1-P_{i}\left(z\right)\right)\left(P_{i}\left(z\right)-z\right)}\\
&+&\frac{\left(1-z\right)\left(1-F_{i}\left(z\right)\right)P_{i}\left(z\right)P_{i}^{'}\left(z\right)}{\esp\left[C_{i}\right]\left(1-P_{i}\left(z\right)\right)^{2}\left(P_{i}\left(z\right)-z\right)}
\end{eqnarray*}

Calculando el l\'imite cuando $z\rightarrow1^{+}$:
\begin{eqnarray}
Q_{i}^{(1)}\left(z\right)=\lim_{z\rightarrow1^{+}}\frac{d Q_{i}\left(z\right)}{dz}&=&\lim_{z\rightarrow1}\frac{\left(1-F_{i}\left(z\right)\right)P_{i}\left(z\right)}{\esp\left[C_{i}\right]\left(1-P_{i}\left(z\right)\right)\left(P_{i}\left(z\right)-z\right)}\\
&-&\lim_{z\rightarrow1^{+}}\frac{\left(1-z\right)P_{i}\left(z\right)F_{i}^{'}\left(z\right)}{\esp\left[C_{i}\right]\left(1-P_{i}\left(z\right)\right)\left(P_{i}\left(z\right)-z\right)}\\
&-&\lim_{z\rightarrow1^{+}}\frac{\left(1-z\right)\left(1-F_{i}\left(z\right)\right)P_{i}\left(z\right)\left(P_{i}^{'}\left(z\right)-1\right)}{\esp\left[C_{i}\right]\left(1-P_{i}\left(z\right)\right)\left(P_{i}\left(z\right)-z\right)^{2}}\\
&+&\lim_{z\rightarrow1^{+}}\frac{\left(1-z\right)\left(1-F_{i}\left(z\right)\right)P_{i}^{'}\left(z\right)}{\esp\left[C_{i}\right]\left(1-P_{i}\left(z\right)\right)\left(P_{i}\left(z\right)-z\right)}\\
&+&\lim_{z\rightarrow1^{+}}\frac{\left(1-z\right)\left(1-F_{i}\left(z\right)\right)P_{i}\left(z\right)P_{i}^{'}\left(z\right)}{\esp\left[C_{i}\right]\left(1-P_{i}\left(z\right)\right)^{2}\left(P_{i}\left(z\right)-z\right)}
\end{eqnarray}

Entonces:
%______________________________________________________

\begin{eqnarray*}
\lim_{z\rightarrow1^{+}}\frac{\left(1-F_{i}\left(z\right)\right)P_{i}\left(z\right)}{\left(1-P_{i}\left(z\right)\right)\left(P_{i}\left(z\right)-z\right)}&=&\lim_{z\rightarrow1^{+}}\frac{\frac{d}{dz}\left[\left(1-F_{i}\left(z\right)\right)P_{i}\left(z\right)\right]}{\frac{d}{dz}\left[\left(1-P_{i}\left(z\right)\right)\left(-z+P_{i}\left(z\right)\right)\right]}\\
&=&\lim_{z\rightarrow1^{+}}\frac{-P_{i}\left(z\right)F_{i}^{'}\left(z\right)+\left(1-F_{i}\left(z\right)\right)P_{i}^{'}\left(z\right)}{\left(1-P_{i}\left(z\right)\right)\left(-1+P_{i}^{'}\left(z\right)\right)-\left(-z+P_{i}\left(z\right)\right)P_{i}^{'}\left(z\right)}
\end{eqnarray*}


%______________________________________________________


\begin{eqnarray*}
\lim_{z\rightarrow1^{+}}\frac{\left(1-z\right)P_{i}\left(z\right)F_{i}^{'}\left(z\right)}{\left(1-P_{i}\left(z\right)\right)\left(P_{i}\left(z\right)-z\right)}&=&\lim_{z\rightarrow1^{+}}\frac{\frac{d}{dz}\left[\left(1-z\right)P_{i}\left(z\right)F_{i}^{'}\left(z\right)\right]}{\frac{d}{dz}\left[\left(1-P_{i}\left(z\right)\right)\left(P_{i}\left(z\right)-z\right)\right]}\\
&=&\lim_{z\rightarrow1^{+}}\frac{-P_{i}\left(z\right) F_{i}^{'}\left(z\right)+(1-z) F_{i}^{'}\left(z\right) P_{i}^{'}\left(z\right)+(1-z) P_{i}\left(z\right)F_{i}^{''}\left(z\right)}{\left(1-P_{i}\left(z\right)\right)\left(-1+P_{i}^{'}\left(z\right)\right)-\left(-z+P_{i}\left(z\right)\right)P_{i}^{'}\left(z\right)}
\end{eqnarray*}


%______________________________________________________

\begin{eqnarray*}
&&\lim_{z\rightarrow1^{+}}\frac{\left(1-z\right)\left(1-F_{i}\left(z\right)\right)P_{i}\left(z\right)\left(P_{i}^{'}\left(z\right)-1\right)}{\left(1-P_{i}\left(z\right)\right)\left(P_{i}\left(z\right)-z\right)^{2}}=\lim_{z\rightarrow1^{+}}\frac{\frac{d}{dz}\left[\left(1-z\right)\left(1-F_{i}\left(z\right)\right)P_{i}\left(z\right)\left(P_{i}^{'}\left(z\right)-1\right)\right]}{\frac{d}{dz}\left[\left(1-P_{i}\left(z\right)\right)\left(P_{i}\left(z\right)-z\right)^{2}\right]}\\
&=&\lim_{z\rightarrow1^{+}}\frac{-\left(1-F_{i}\left(z\right)\right) P_{i}\left(z\right)\left(-1+P_{i}^{'}\left(z\right)\right)-(1-z) P_{i}\left(z\right)F_{i}^{'}\left(z\right)\left(-1+P_{i}^{'}\left(z\right)\right)}{2\left(1-P_{i}\left(z\right)\right)\left(-z+P_{i}\left(z\right)\right) \left(-1+P_{i}^{'}\left(z\right)\right)-\left(-z+P_{i}\left(z\right)\right)^2 P_{i}^{'}\left(z\right)}\\
&+&\lim_{z\rightarrow1^{+}}\frac{+(1-z) \left(1-F_{i}\left(z\right)\right) \left(-1+P_{i}^{'}\left(z\right)\right) P_{i}^{'}\left(z\right)}{{2\left(1-P_{i}\left(z\right)\right)\left(-z+P_{i}\left(z\right)\right) \left(-1+P_{i}^{'}\left(z\right)\right)-\left(-z+P_{i}\left(z\right)\right)^2 P_{i}^{'}\left(z\right)}}\\
&+&\lim_{z\rightarrow1^{+}}\frac{+(1-z) \left(1-F_{i}\left(z\right)\right) P_{i}\left(z\right)P_{i}^{''}\left(z\right)}{{2\left(1-P_{i}\left(z\right)\right)\left(-z+P_{i}\left(z\right)\right) \left(-1+P_{i}^{'}\left(z\right)\right)-\left(-z+P_{i}\left(z\right)\right)^2 P_{i}^{'}\left(z\right)}}
\end{eqnarray*}











%______________________________________________________
\begin{eqnarray*}
&&\lim_{z\rightarrow1^{+}}\frac{\left(1-z\right)\left(1-F_{i}\left(z\right)\right)P_{i}^{'}\left(z\right)}{\left(1-P_{i}\left(z\right)\right)\left(P_{i}\left(z\right)-z\right)}=\lim_{z\rightarrow1^{+}}\frac{\frac{d}{dz}\left[\left(1-z\right)\left(1-F_{i}\left(z\right)\right)P_{i}^{'}\left(z\right)\right]}{\frac{d}{dz}\left[\left(1-P_{i}\left(z\right)\right)\left(P_{i}\left(z\right)-z\right)\right]}\\
&=&\lim_{z\rightarrow1^{+}}\frac{-\left(1-F_{i}\left(z\right)\right) P_{i}^{'}\left(z\right)-(1-z) F_{i}^{'}\left(z\right) P_{i}^{'}\left(z\right)+(1-z) \left(1-F_{i}\left(z\right)\right) P_{i}^{''}\left(z\right)}{\left(1-P_{i}\left(z\right)\right) \left(-1+P_{i}^{'}\left(z\right)\right)-\left(-z+P_{i}\left(z\right)\right) P_{i}^{'}\left(z\right)}\frac{}{}
\end{eqnarray*}

%______________________________________________________
\begin{eqnarray*}
&&\lim_{z\rightarrow1^{+}}\frac{\left(1-z\right)\left(1-F_{i}\left(z\right)\right)P_{i}\left(z\right)P_{i}^{'}\left(z\right)}{\left(1-P_{i}\left(z\right)\right)^{2}\left(P_{i}\left(z\right)-z\right)}=\lim_{z\rightarrow1^{+}}\frac{\frac{d}{dz}\left[\left(1-z\right)\left(1-F_{i}\left(z\right)\right)P_{i}\left(z\right)P_{i}^{'}\left(z\right)\right]}{\frac{d}{dz}\left[\left(1-P_{i}\left(z\right)\right)^{2}\left(P_{i}\left(z\right)-z\right)\right]}\\
&=&\lim_{z\rightarrow1^{+}}\frac{-\left(1-F_{i}\left(z\right)\right) P_{i}\left(z\right) P_{i}^{'}\left(z\right)-(1-z) P_{i}\left(z\right) F_{i}^{'}\left(z\right)P_i'[z]}{\left(1-P_{i}\left(z\right)\right)^2 \left(-1+P_{i}^{'}\left(z\right)\right)-2 \left(1-P_{i}\left(z\right)\right) \left(-z+P_{i}\left(z\right)\right) P_{i}^{'}\left(z\right)}\\
&+&\lim_{z\rightarrow1^{+}}\frac{(1-z) \left(1-F_{i}\left(z\right)\right) P_{i}^{'}\left(z\right)^2+(1-z) \left(1-F_{i}\left(z\right)\right) P_{i}\left(z\right) P_{i}^{''}\left(z\right)}{\left(1-P_{i}\left(z\right)\right)^2 \left(-1+P_{i}^{'}\left(z\right)\right)-2 \left(1-P_{i}\left(z\right)\right) \left(-z+P_{i}\left(z\right)\right) P_{i}^{'}\left(z\right)}\\
\end{eqnarray*}
%___________________________________________________________________________________________
%\subsection{Tiempos de Ciclo e Intervisita}
%___________________________________________________________________________________________


\begin{Def}
Sea $L_{i}^{*}$el n\'umero de usuarios en la cola $Q_{i}$ cuando es visitada por el servidor para dar servicio, entonces

\begin{eqnarray}
\esp\left[L_{i}^{*}\right]&=&f_{i}\left(i\right)\\
Var\left[L_{i}^{*}\right]&=&f_{i}\left(i,i\right)+\esp\left[L_{i}^{*}\right]-\esp\left[L_{i}^{*}\right]^{2}.
\end{eqnarray}

\end{Def}

\begin{Def}
El tiempo de Ciclo $C_{i}$ es e periodo de tiempo que comienza cuando la cola $i$ es visitada por primera vez en un ciclo, y termina cuando es visitado nuevamente en el pr\'oximo ciclo. La duraci\'on del mismo est\'a dada por $\tau_{i}\left(m+1\right)-\tau_{i}\left(m\right)$, o equivalentemente $\overline{\tau}_{i}\left(m+1\right)-\overline{\tau}_{i}\left(m\right)$ bajo condiciones de estabilidad.
\end{Def}

\begin{Def}
El tiempo de intervisita $I_{i}$ es el periodo de tiempo que comienza cuando se ha completado el servicio en un ciclo y termina cuando es visitada nuevamente en el pr\'oximo ciclo. Su  duraci\'on del mismo est\'a dada por $\tau_{i}\left(m+1\right)-\overline{\tau}_{i}\left(m\right)$.
\end{Def}


Recordemos las siguientes expresiones:

\begin{eqnarray*}
S_{i}\left(z\right)&=&\esp\left[z^{\overline{\tau}_{i}\left(m\right)-\tau_{i}\left(m\right)}\right]=F_{i}\left(\theta\left(z\right)\right),\\
F\left(z\right)&=&\esp\left[z^{L_{0}}\right],\\
P\left(z\right)&=&\esp\left[z^{X_{n}}\right],\\
F_{i}\left(z\right)&=&\esp\left[z^{L_{i}\left(\tau_{i}\left(m\right)\right)}\right],
\theta_{i}\left(z\right)-zP_{i}
\end{eqnarray*}

entonces 

\begin{eqnarray*}
\esp\left[S_{i}\right]&=&\frac{\esp\left[L_{i}^{*}\right]}{1-\mu_{i}}=\frac{f_{i}\left(i\right)}{1-\mu_{i}},\\
Var\left[S_{i}\right]&=&\frac{Var\left[L_{i}^{*}\right]}{\left(1-\mu_{i}\right)^{2}}+\frac{\sigma^{2}\esp\left[L_{i}^{*}\right]}{\left(1-\mu_{i}\right)^{3}}
\end{eqnarray*}

donde recordemos que

\begin{eqnarray*}
Var\left[L_{i}^{*}\right]&=&f_{i}\left(i,i\right)+f_{i}\left(i\right)-f_{i}\left(i\right)^{2}.
\end{eqnarray*}

La duraci\'on del tiempo de intervisita es $\tau_{i}\left(m+1\right)-\overline{\tau}\left(m\right)$. Dado que el n\'umero de usuarios presentes en $Q_{i}$ al tiempo $t=\tau_{i}\left(m+1\right)$ es igual al n\'umero de arribos durante el intervalo de tiempo $\left[\overline{\tau}\left(m\right),\tau_{i}\left(m+1\right)\right]$ se tiene que


\begin{eqnarray*}
\esp\left[z_{i}^{L_{i}\left(\tau_{i}\left(m+1\right)\right)}\right]=\esp\left[\left\{P_{i}\left(z_{i}\right)\right\}^{\tau_{i}\left(m+1\right)-\overline{\tau}\left(m\right)}\right]
\end{eqnarray*}

entonces, si \begin{eqnarray*}I_{i}\left(z\right)&=&\esp\left[z^{\tau_{i}\left(m+1\right)-\overline{\tau}\left(m\right)}\right]\end{eqnarray*} se tienen que

\begin{eqnarray*}
F_{i}\left(z\right)=I_{i}\left[P_{i}\left(z\right)\right]
\end{eqnarray*}
para $i=1,2$, por tanto



\begin{eqnarray*}
\esp\left[L_{i}^{*}\right]&=&\mu_{i}\esp\left[I_{i}\right]\\
Var\left[L_{i}^{*}\right]&=&\mu_{i}^{2}Var\left[I_{i}\right]+\sigma^{2}\esp\left[I_{i}\right]
\end{eqnarray*}
para $i=1,2$, por tanto


\begin{eqnarray*}
\esp\left[I_{i}\right]&=&\frac{f_{i}\left(i\right)}{\mu_{i}},
\end{eqnarray*}
adem\'as

\begin{eqnarray*}
Var\left[I_{i}\right]&=&\frac{Var\left[L_{i}^{*}\right]}{\mu_{i}^{2}}-\frac{\sigma_{i}^{2}}{\mu_{i}^{2}}f_{i}\left(i\right).
\end{eqnarray*}


Si  $C_{i}\left(z\right)=\esp\left[z^{\overline{\tau}\left(m+1\right)-\overline{\tau}_{i}\left(m\right)}\right]$el tiempo de duraci\'on del ciclo, entonces, por lo hasta ahora establecido, se tiene que

\begin{eqnarray*}
C_{i}\left(z\right)=I_{i}\left[\theta_{i}\left(z\right)\right],
\end{eqnarray*}
entonces

\begin{eqnarray*}
\esp\left[C_{i}\right]&=&\esp\left[I_{i}\right]\esp\left[\theta_{i}\left(z\right)\right]=\frac{\esp\left[L_{i}^{*}\right]}{\mu_{i}}\frac{1}{1-\mu_{i}}=\frac{f_{i}\left(i\right)}{\mu_{i}\left(1-\mu_{i}\right)}\\
Var\left[C_{i}\right]&=&\frac{Var\left[L_{i}^{*}\right]}{\mu_{i}^{2}\left(1-\mu_{i}\right)^{2}}.
\end{eqnarray*}

Por tanto se tienen las siguientes igualdades


\begin{eqnarray*}
\esp\left[S_{i}\right]&=&\mu_{i}\esp\left[C_{i}\right],\\
\esp\left[I_{i}\right]&=&\left(1-\mu_{i}\right)\esp\left[C_{i}\right]\\
\end{eqnarray*}

Def\'inanse los puntos de regenaraci\'on  en el proceso $\left[L_{1}\left(t\right),L_{2}\left(t\right),\ldots,L_{N}\left(t\right)\right]$. Los puntos cuando la cola $i$ es visitada y todos los $L_{j}\left(\tau_{i}\left(m\right)\right)=0$ para $i=1,2$  son puntos de regeneraci\'on. Se llama ciclo regenerativo al intervalo entre dos puntos regenerativos sucesivos.

Sea $M_{i}$  el n\'umero de ciclos de visita en un ciclo regenerativo, y sea $C_{i}^{(m)}$, para $m=1,2,\ldots,M_{i}$ la duraci\'on del $m$-\'esimo ciclo de visita en un ciclo regenerativo. Se define el ciclo del tiempo de visita promedio $\esp\left[C_{i}\right]$ como

\begin{eqnarray*}
\esp\left[C_{i}\right]&=&\frac{\esp\left[\sum_{m=1}^{M_{i}}C_{i}^{(m)}\right]}{\esp\left[M_{i}\right]}
\end{eqnarray*}


En Stid72 y Heym82 se muestra que una condici\'on suficiente para que el proceso regenerativo 
estacionario sea un procesoo estacionario es que el valor esperado del tiempo del ciclo regenerativo sea finito:

\begin{eqnarray*}
\esp\left[\sum_{m=1}^{M_{i}}C_{i}^{(m)}\right]<\infty.
\end{eqnarray*}

como cada $C_{i}^{(m)}$ contiene intervalos de r\'eplica positivos, se tiene que $\esp\left[M_{i}\right]<\infty$, adem\'as, como $M_{i}>0$, se tiene que la condici\'on anterior es equivalente a tener que 

\begin{eqnarray*}
\esp\left[C_{i}\right]<\infty,
\end{eqnarray*}
por lo tanto una condici\'on suficiente para la existencia del proceso regenerativo est\'a dada por

\begin{eqnarray*}
\sum_{k=1}^{N}\mu_{k}<1.
\end{eqnarray*}

Sea la funci\'on generadora de momentos para $L_{i}$, el n\'umero de usuarios en la cola $Q_{i}\left(z\right)$ en cualquier momento, est\'a dada por el tiempo promedio de $z^{L_{i}\left(t\right)}$ sobre el ciclo regenerativo definido anteriormente:

\begin{eqnarray*}
Q_{i}\left(z\right)&=&\esp\left[z^{L_{i}\left(t\right)}\right]=\frac{\esp\left[\sum_{m=1}^{M_{i}}\sum_{t=\tau_{i}\left(m\right)}^{\tau_{i}\left(m+1\right)-1}z^{L_{i}\left(t\right)}\right]}{\esp\left[\sum_{m=1}^{M_{i}}\tau_{i}\left(m+1\right)-\tau_{i}\left(m\right)\right]}
\end{eqnarray*}

$M_{i}$ es un tiempo de paro en el proceso regenerativo con $\esp\left[M_{i}\right]<\infty$, se sigue del lema de Wald que:


\begin{eqnarray*}
\esp\left[\sum_{m=1}^{M_{i}}\sum_{t=\tau_{i}\left(m\right)}^{\tau_{i}\left(m+1\right)-1}z^{L_{i}\left(t\right)}\right]&=&\esp\left[M_{i}\right]\esp\left[\sum_{t=\tau_{i}\left(m\right)}^{\tau_{i}\left(m+1\right)-1}z^{L_{i}\left(t\right)}\right]\\
\esp\left[\sum_{m=1}^{M_{i}}\tau_{i}\left(m+1\right)-\tau_{i}\left(m\right)\right]&=&\esp\left[M_{i}\right]\esp\left[\tau_{i}\left(m+1\right)-\tau_{i}\left(m\right)\right]
\end{eqnarray*}

por tanto se tiene que


\begin{eqnarray*}
Q_{i}\left(z\right)&=&\frac{\esp\left[\sum_{t=\tau_{i}\left(m\right)}^{\tau_{i}\left(m+1\right)-1}z^{L_{i}\left(t\right)}\right]}{\esp\left[\tau_{i}\left(m+1\right)-\tau_{i}\left(m\right)\right]}
\end{eqnarray*}

observar que el denominador es simplemente la duraci\'on promedio del tiempo del ciclo.


Se puede demostrar (ver Hideaki Takagi 1986) que

\begin{eqnarray*}
\esp\left[\sum_{t=\tau_{i}\left(m\right)}^{\tau_{i}\left(m+1\right)-1}z^{L_{i}\left(t\right)}\right]=z\frac{F_{i}\left(z\right)-1}{z-P_{i}\left(z\right)}
\end{eqnarray*}

Durante el tiempo de intervisita para la cola $i$, $L_{i}\left(t\right)$ solamente se incrementa de manera que el incremento por intervalo de tiempo est\'a dado por la funci\'on generadora de probabilidades de $P_{i}\left(z\right)$, por tanto la suma sobre el tiempo de intervisita puede evaluarse como:

\begin{eqnarray*}
\esp\left[\sum_{t=\tau_{i}\left(m\right)}^{\tau_{i}\left(m+1\right)-1}z^{L_{i}\left(t\right)}\right]&=&\esp\left[\sum_{t=\tau_{i}\left(m\right)}^{\tau_{i}\left(m+1\right)-1}\left\{P_{i}\left(z\right)\right\}^{t-\overline{\tau}_{i}\left(m\right)}\right]=\frac{1-\esp\left[\left\{P_{i}\left(z\right)\right\}^{\tau_{i}\left(m+1\right)-\overline{\tau}_{i}\left(m\right)}\right]}{1-P_{i}\left(z\right)}\\
&=&\frac{1-I_{i}\left[P_{i}\left(z\right)\right]}{1-P_{i}\left(z\right)}
\end{eqnarray*}
por tanto

\begin{eqnarray*}
\esp\left[\sum_{t=\tau_{i}\left(m\right)}^{\tau_{i}\left(m+1\right)-1}z^{L_{i}\left(t\right)}\right]&=&\frac{1-F_{i}\left(z\right)}{1-P_{i}\left(z\right)}
\end{eqnarray*}

Haciendo uso de lo hasta ahora desarrollado se tiene que

\begin{eqnarray*}
Q_{i}\left(z\right)&=&\frac{1}{\esp\left[C_{i}\right]}\cdot\frac{1-F_{i}\left(z\right)}{P_{i}\left(z\right)-z}\cdot\frac{\left(1-z\right)P_{i}\left(z\right)}{1-P_{i}\left(z\right)}\\
&=&\frac{\mu_{i}\left(1-\mu_{i}\right)}{f_{i}\left(i\right)}\cdot\frac{1-F_{i}\left(z\right)}{P_{i}\left(z\right)-z}\cdot\frac{\left(1-z\right)P_{i}\left(z\right)}{1-P_{i}\left(z\right)}
\end{eqnarray*}


%___________________________________________________________________________________________
%\subsection{Longitudes de la Cola en cualquier tiempo}
%___________________________________________________________________________________________

Sea
$V_{i}\left(z\right)=\frac{1}{\esp\left[C_{i}\right]}\frac{I_{i}\left(z\right)-1}{z-P_{i}\left(z\right)}$

%{\esp\lef[I_{i}\right]}\frac{1-\mu_{i}}{z-P_{i}\left(z\right)}

\begin{eqnarray*}
\frac{\partial V_{i}\left(z\right)}{\partial z}&=&\frac{1}{\esp\left[C_{i}\right]}\left[\frac{I_{i}{'}\left(z\right)\left(z-P_{i}\left(z\right)\right)}{z-P_{i}\left(z\right)}-\frac{\left(I_{i}\left(z\right)-1\right)\left(1-P_{i}{'}\left(z\right)\right)}{\left(z-P_{i}\left(z\right)\right)^{2}}\right]
\end{eqnarray*}


La FGP para el tiempo de espera para cualquier usuario en la cola est\'a dada por:
\[U_{i}\left(z\right)=\frac{1}{\esp\left[C_{i}\right]}\cdot\frac{1-P_{i}\left(z\right)}{z-P_{i}\left(z\right)}\cdot\frac{I_{i}\left(z\right)-1}{1-z}\]

entonces


\begin{eqnarray*}
\frac{d}{dz}V_{i}\left(z\right)&=&\frac{1}{\esp\left[C_{i}\right]}\left\{\frac{d}{dz}\left(\frac{1-P_{i}\left(z\right)}{z-P_{i}\left(z\right)}\right)\frac{I_{i}\left(z\right)-1}{1-z}+\frac{1-P_{i}\left(z\right)}{z-P_{i}\left(z\right)}\frac{d}{dz}\left(\frac{I_{i}\left(z\right)-1}{1-z}\right)\right\}\\
&=&\frac{1}{\esp\left[C_{i}\right]}\left\{\frac{-P_{i}\left(z\right)\left(z-P_{i}\left(z\right)\right)-\left(1-P_{i}\left(z\right)\right)\left(1-P_{i}^{'}\left(z\right)\right)}{\left(z-P_{i}\left(z\right)\right)^{2}}\cdot\frac{I_{i}\left(z\right)-1}{1-z}\right\}\\
&+&\frac{1}{\esp\left[C_{i}\right]}\left\{\frac{1-P_{i}\left(z\right)}{z-P_{i}\left(z\right)}\cdot\frac{I_{i}^{'}\left(z\right)\left(1-z\right)+\left(I_{i}\left(z\right)-1\right)}{\left(1-z\right)^{2}}\right\}
\end{eqnarray*}
%\frac{I_{i}\left(z\right)-1}{1-z}
%+\frac{1-P_{i}\left(z\right)}{z-P_{i}\frac{d}{dz}\left(\frac{I_{i}\left(z\right)-1}{1-z}\right)


\begin{eqnarray*}
\frac{\partial U_{i}\left(z\right)}{\partial z}&=&\frac{(-1+I_{i}[z]) (1-P_{i}[z])}{(1-z)^2 \esp[I_{i}] (z-P_{i}[z])}+\frac{(1-P_{i}[z]) I_{i}^{'}[z]}{(1-z) \esp[I_{i}] (z-P_{i}[z])}-\frac{(-1+I_{i}[z]) (1-P_{i}[z])\left(1-P{'}[z]\right)}{(1-z) \esp[I_{i}] (z-P_{i}[z])^2}\\
&-&\frac{(-1+I_{i}[z]) P_{i}{'}[z]}{(1-z) \esp[I_{i}](z-P_{i}[z])}
\end{eqnarray*}
%___________________________________________________________________________________________
%\subsection{Longitudes de la Cola en cualquier tiempo}
%___________________________________________________________________________________________
Sea 
\begin{eqnarray*}
Q_{i}\left(z\right)&=&\frac{1}{\esp\left[C_{i}\right]}\cdot\frac{1-F_{i}\left(z\right)}{P_{i}\left(z\right)-z}\cdot\frac{\left(1-z\right)P_{i}\left(z\right)}{1-P_{i}\left(z\right)}
\end{eqnarray*}

derivando con respecto a $z$



\begin{eqnarray*}
\frac{d Q_{i}\left(z\right)}{d z}&=&\frac{\left(1-F_{i}\left(z\right)\right)P_{i}\left(z\right)}{\esp\left[C_{i}\right]\left(1-P_{i}\left(z\right)\right)\left(P_{i}\left(z\right)-z\right)}\\
&-&\frac{\left(1-z\right)P_{i}\left(z\right)F_{i}^{'}\left(z\right)}{\esp\left[C_{i}\right]\left(1-P_{i}\left(z\right)\right)\left(P_{i}\left(z\right)-z\right)}\\
&-&\frac{\left(1-z\right)\left(1-F_{i}\left(z\right)\right)P_{i}\left(z\right)\left(P_{i}^{'}\left(z\right)-1\right)}{\esp\left[C_{i}\right]\left(1-P_{i}\left(z\right)\right)\left(P_{i}\left(z\right)-z\right)^{2}}\\
&+&\frac{\left(1-z\right)\left(1-F_{i}\left(z\right)\right)P_{i}^{'}\left(z\right)}{\esp\left[C_{i}\right]\left(1-P_{i}\left(z\right)\right)\left(P_{i}\left(z\right)-z\right)}\\
&+&\frac{\left(1-z\right)\left(1-F_{i}\left(z\right)\right)P_{i}\left(z\right)P_{i}^{'}\left(z\right)}{\esp\left[C_{i}\right]\left(1-P_{i}\left(z\right)\right)^{2}\left(P_{i}\left(z\right)-z\right)}
\end{eqnarray*}

Calculando el l\'imite cuando $z\rightarrow1^{+}$:
\begin{eqnarray}
Q_{i}^{(1)}\left(z\right)=\lim_{z\rightarrow1^{+}}\frac{d Q_{i}\left(z\right)}{dz}&=&\lim_{z\rightarrow1}\frac{\left(1-F_{i}\left(z\right)\right)P_{i}\left(z\right)}{\esp\left[C_{i}\right]\left(1-P_{i}\left(z\right)\right)\left(P_{i}\left(z\right)-z\right)}\\
&-&\lim_{z\rightarrow1^{+}}\frac{\left(1-z\right)P_{i}\left(z\right)F_{i}^{'}\left(z\right)}{\esp\left[C_{i}\right]\left(1-P_{i}\left(z\right)\right)\left(P_{i}\left(z\right)-z\right)}\\
&-&\lim_{z\rightarrow1^{+}}\frac{\left(1-z\right)\left(1-F_{i}\left(z\right)\right)P_{i}\left(z\right)\left(P_{i}^{'}\left(z\right)-1\right)}{\esp\left[C_{i}\right]\left(1-P_{i}\left(z\right)\right)\left(P_{i}\left(z\right)-z\right)^{2}}\\
&+&\lim_{z\rightarrow1^{+}}\frac{\left(1-z\right)\left(1-F_{i}\left(z\right)\right)P_{i}^{'}\left(z\right)}{\esp\left[C_{i}\right]\left(1-P_{i}\left(z\right)\right)\left(P_{i}\left(z\right)-z\right)}\\
&+&\lim_{z\rightarrow1^{+}}\frac{\left(1-z\right)\left(1-F_{i}\left(z\right)\right)P_{i}\left(z\right)P_{i}^{'}\left(z\right)}{\esp\left[C_{i}\right]\left(1-P_{i}\left(z\right)\right)^{2}\left(P_{i}\left(z\right)-z\right)}
\end{eqnarray}

Entonces:
%______________________________________________________

\begin{eqnarray*}
\lim_{z\rightarrow1^{+}}\frac{\left(1-F_{i}\left(z\right)\right)P_{i}\left(z\right)}{\left(1-P_{i}\left(z\right)\right)\left(P_{i}\left(z\right)-z\right)}&=&\lim_{z\rightarrow1^{+}}\frac{\frac{d}{dz}\left[\left(1-F_{i}\left(z\right)\right)P_{i}\left(z\right)\right]}{\frac{d}{dz}\left[\left(1-P_{i}\left(z\right)\right)\left(-z+P_{i}\left(z\right)\right)\right]}\\
&=&\lim_{z\rightarrow1^{+}}\frac{-P_{i}\left(z\right)F_{i}^{'}\left(z\right)+\left(1-F_{i}\left(z\right)\right)P_{i}^{'}\left(z\right)}{\left(1-P_{i}\left(z\right)\right)\left(-1+P_{i}^{'}\left(z\right)\right)-\left(-z+P_{i}\left(z\right)\right)P_{i}^{'}\left(z\right)}
\end{eqnarray*}


%______________________________________________________


\begin{eqnarray*}
\lim_{z\rightarrow1^{+}}\frac{\left(1-z\right)P_{i}\left(z\right)F_{i}^{'}\left(z\right)}{\left(1-P_{i}\left(z\right)\right)\left(P_{i}\left(z\right)-z\right)}&=&\lim_{z\rightarrow1^{+}}\frac{\frac{d}{dz}\left[\left(1-z\right)P_{i}\left(z\right)F_{i}^{'}\left(z\right)\right]}{\frac{d}{dz}\left[\left(1-P_{i}\left(z\right)\right)\left(P_{i}\left(z\right)-z\right)\right]}\\
&=&\lim_{z\rightarrow1^{+}}\frac{-P_{i}\left(z\right) F_{i}^{'}\left(z\right)+(1-z) F_{i}^{'}\left(z\right) P_{i}^{'}\left(z\right)+(1-z) P_{i}\left(z\right)F_{i}^{''}\left(z\right)}{\left(1-P_{i}\left(z\right)\right)\left(-1+P_{i}^{'}\left(z\right)\right)-\left(-z+P_{i}\left(z\right)\right)P_{i}^{'}\left(z\right)}
\end{eqnarray*}


%______________________________________________________

\begin{eqnarray*}
&&\lim_{z\rightarrow1^{+}}\frac{\left(1-z\right)\left(1-F_{i}\left(z\right)\right)P_{i}\left(z\right)\left(P_{i}^{'}\left(z\right)-1\right)}{\left(1-P_{i}\left(z\right)\right)\left(P_{i}\left(z\right)-z\right)^{2}}=\lim_{z\rightarrow1^{+}}\frac{\frac{d}{dz}\left[\left(1-z\right)\left(1-F_{i}\left(z\right)\right)P_{i}\left(z\right)\left(P_{i}^{'}\left(z\right)-1\right)\right]}{\frac{d}{dz}\left[\left(1-P_{i}\left(z\right)\right)\left(P_{i}\left(z\right)-z\right)^{2}\right]}\\
&=&\lim_{z\rightarrow1^{+}}\frac{-\left(1-F_{i}\left(z\right)\right) P_{i}\left(z\right)\left(-1+P_{i}^{'}\left(z\right)\right)-(1-z) P_{i}\left(z\right)F_{i}^{'}\left(z\right)\left(-1+P_{i}^{'}\left(z\right)\right)}{2\left(1-P_{i}\left(z\right)\right)\left(-z+P_{i}\left(z\right)\right) \left(-1+P_{i}^{'}\left(z\right)\right)-\left(-z+P_{i}\left(z\right)\right)^2 P_{i}^{'}\left(z\right)}\\
&+&\lim_{z\rightarrow1^{+}}\frac{+(1-z) \left(1-F_{i}\left(z\right)\right) \left(-1+P_{i}^{'}\left(z\right)\right) P_{i}^{'}\left(z\right)}{{2\left(1-P_{i}\left(z\right)\right)\left(-z+P_{i}\left(z\right)\right) \left(-1+P_{i}^{'}\left(z\right)\right)-\left(-z+P_{i}\left(z\right)\right)^2 P_{i}^{'}\left(z\right)}}\\
&+&\lim_{z\rightarrow1^{+}}\frac{+(1-z) \left(1-F_{i}\left(z\right)\right) P_{i}\left(z\right)P_{i}^{''}\left(z\right)}{{2\left(1-P_{i}\left(z\right)\right)\left(-z+P_{i}\left(z\right)\right) \left(-1+P_{i}^{'}\left(z\right)\right)-\left(-z+P_{i}\left(z\right)\right)^2 P_{i}^{'}\left(z\right)}}
\end{eqnarray*}











%______________________________________________________
\begin{eqnarray*}
&&\lim_{z\rightarrow1^{+}}\frac{\left(1-z\right)\left(1-F_{i}\left(z\right)\right)P_{i}^{'}\left(z\right)}{\left(1-P_{i}\left(z\right)\right)\left(P_{i}\left(z\right)-z\right)}=\lim_{z\rightarrow1^{+}}\frac{\frac{d}{dz}\left[\left(1-z\right)\left(1-F_{i}\left(z\right)\right)P_{i}^{'}\left(z\right)\right]}{\frac{d}{dz}\left[\left(1-P_{i}\left(z\right)\right)\left(P_{i}\left(z\right)-z\right)\right]}\\
&=&\lim_{z\rightarrow1^{+}}\frac{-\left(1-F_{i}\left(z\right)\right) P_{i}^{'}\left(z\right)-(1-z) F_{i}^{'}\left(z\right) P_{i}^{'}\left(z\right)+(1-z) \left(1-F_{i}\left(z\right)\right) P_{i}^{''}\left(z\right)}{\left(1-P_{i}\left(z\right)\right) \left(-1+P_{i}^{'}\left(z\right)\right)-\left(-z+P_{i}\left(z\right)\right) P_{i}^{'}\left(z\right)}\frac{}{}
\end{eqnarray*}

%______________________________________________________
\begin{eqnarray*}
&&\lim_{z\rightarrow1^{+}}\frac{\left(1-z\right)\left(1-F_{i}\left(z\right)\right)P_{i}\left(z\right)P_{i}^{'}\left(z\right)}{\left(1-P_{i}\left(z\right)\right)^{2}\left(P_{i}\left(z\right)-z\right)}=\lim_{z\rightarrow1^{+}}\frac{\frac{d}{dz}\left[\left(1-z\right)\left(1-F_{i}\left(z\right)\right)P_{i}\left(z\right)P_{i}^{'}\left(z\right)\right]}{\frac{d}{dz}\left[\left(1-P_{i}\left(z\right)\right)^{2}\left(P_{i}\left(z\right)-z\right)\right]}\\
&=&\lim_{z\rightarrow1^{+}}\frac{-\left(1-F_{i}\left(z\right)\right) P_{i}\left(z\right) P_{i}^{'}\left(z\right)-(1-z) P_{i}\left(z\right) F_{i}^{'}\left(z\right)P_i'[z]}{\left(1-P_{i}\left(z\right)\right)^2 \left(-1+P_{i}^{'}\left(z\right)\right)-2 \left(1-P_{i}\left(z\right)\right) \left(-z+P_{i}\left(z\right)\right) P_{i}^{'}\left(z\right)}\\
&+&\lim_{z\rightarrow1^{+}}\frac{(1-z) \left(1-F_{i}\left(z\right)\right) P_{i}^{'}\left(z\right)^2+(1-z) \left(1-F_{i}\left(z\right)\right) P_{i}\left(z\right) P_{i}^{''}\left(z\right)}{\left(1-P_{i}\left(z\right)\right)^2 \left(-1+P_{i}^{'}\left(z\right)\right)-2 \left(1-P_{i}\left(z\right)\right) \left(-z+P_{i}\left(z\right)\right) P_{i}^{'}\left(z\right)}\\
\end{eqnarray*}




%_______________________________________________________________________________________________________
\subsection{Tiempo de Ciclo Promedio}
%_______________________________________________________________________________________________________

Consideremos una cola de la red de sistemas de visitas c\'iclicas fija, $Q_{l}$.


Conforme a la definici\'on dada al principio del cap\'itulo, definici\'on (\ref{Def.Tn}), sean $T_{1},T_{2},\ldots$ los puntos donde las longitudes de las colas de la red de sistemas de visitas c\'iclicas son cero simult\'aneamente, cuando la cola $Q_{l}$ es visitada por el servidor para dar servicio, es decir, $L_{1}\left(T_{i}\right)=0,L_{2}\left(T_{i}\right)=0,\hat{L}_{1}\left(T_{i}\right)=0$ y $\hat{L}_{2}\left(T_{i}\right)=0$, a estos puntos se les denominar\'a puntos regenerativos. Entonces, 

\begin{Def}
Al intervalo de tiempo entre dos puntos regenerativos se le llamar\'a ciclo regenerativo.
\end{Def}

\begin{Def}
Para $T_{i}$ se define, $M_{i}$, el n\'umero de ciclos de visita a la cola $Q_{l}$, durante el ciclo regenerativo, es decir, $M_{i}$ es un proceso de renovaci\'on.
\end{Def}

\begin{Def}
Para cada uno de los $M_{i}$'s, se definen a su vez la duraci\'on de cada uno de estos ciclos de visita en el ciclo regenerativo, $C_{i}^{(m)}$, para $m=1,2,\ldots,M_{i}$, que a su vez, tambi\'en es n proceso de renovaci\'on.
\end{Def}

En nuestra notaci\'on $V\left(t\right)\equiv C_{i}$ y $X_{i}=C_{i}^{(m)}$ para nuestra segunda definici\'on, mientras que para la primera la notaci\'on es: $X\left(t\right)\equiv C_{i}$ y $R_{i}\equiv C_{i}^{(m)}$.


%___________________________________________________________________________________________
\subsection{Tiempos de Ciclo e Intervisita}
%___________________________________________________________________________________________


\begin{Def}
Sea $L_{i}^{*}$el n\'umero de usuarios en la cola $Q_{i}$ cuando es visitada por el servidor para dar servicio, entonces

\begin{eqnarray}
\esp\left[L_{i}^{*}\right]&=&f_{i}\left(i\right)\\
Var\left[L_{i}^{*}\right]&=&f_{i}\left(i,i\right)+\esp\left[L_{i}^{*}\right]-\esp\left[L_{i}^{*}\right]^{2}.
\end{eqnarray}

\end{Def}

\begin{Def}
El tiempo de Ciclo $C_{i}$ es e periodo de tiempo que comienza cuando la cola $i$ es visitada por primera vez en un ciclo, y termina cuando es visitado nuevamente en el pr\'oximo ciclo. La duraci\'on del mismo est\'a dada por $\tau_{i}\left(m+1\right)-\tau_{i}\left(m\right)$, o equivalentemente $\overline{\tau}_{i}\left(m+1\right)-\overline{\tau}_{i}\left(m\right)$ bajo condiciones de estabilidad.
\end{Def}

\begin{Def}
El tiempo de intervisita $I_{i}$ es el periodo de tiempo que comienza cuando se ha completado el servicio en un ciclo y termina cuando es visitada nuevamente en el pr\'oximo ciclo. Su  duraci\'on del mismo est\'a dada por $\tau_{i}\left(m+1\right)-\overline{\tau}_{i}\left(m\right)$.
\end{Def}


Recordemos las siguientes expresiones:

\begin{eqnarray*}
S_{i}\left(z\right)&=&\esp\left[z^{\overline{\tau}_{i}\left(m\right)-\tau_{i}\left(m\right)}\right]=F_{i}\left(\theta\left(z\right)\right),\\
F\left(z\right)&=&\esp\left[z^{L_{0}}\right],\\
P\left(z\right)&=&\esp\left[z^{X_{n}}\right],\\
F_{i}\left(z\right)&=&\esp\left[z^{L_{i}\left(\tau_{i}\left(m\right)\right)}\right],
\theta_{i}\left(z\right)-zP_{i}
\end{eqnarray*}

entonces 

\begin{eqnarray*}
\esp\left[S_{i}\right]&=&\frac{\esp\left[L_{i}^{*}\right]}{1-\mu_{i}}=\frac{f_{i}\left(i\right)}{1-\mu_{i}},\\
Var\left[S_{i}\right]&=&\frac{Var\left[L_{i}^{*}\right]}{\left(1-\mu_{i}\right)^{2}}+\frac{\sigma^{2}\esp\left[L_{i}^{*}\right]}{\left(1-\mu_{i}\right)^{3}}
\end{eqnarray*}

donde recordemos que

\begin{eqnarray*}
Var\left[L_{i}^{*}\right]&=&f_{i}\left(i,i\right)+f_{i}\left(i\right)-f_{i}\left(i\right)^{2}.
\end{eqnarray*}

La duraci\'on del tiempo de intervisita es $\tau_{i}\left(m+1\right)-\overline{\tau}\left(m\right)$. Dado que el n\'umero de usuarios presentes en $Q_{i}$ al tiempo $t=\tau_{i}\left(m+1\right)$ es igual al n\'umero de arribos durante el intervalo de tiempo $\left[\overline{\tau}\left(m\right),\tau_{i}\left(m+1\right)\right]$ se tiene que


\begin{eqnarray*}
\esp\left[z_{i}^{L_{i}\left(\tau_{i}\left(m+1\right)\right)}\right]=\esp\left[\left\{P_{i}\left(z_{i}\right)\right\}^{\tau_{i}\left(m+1\right)-\overline{\tau}\left(m\right)}\right]
\end{eqnarray*}

entonces, si \begin{eqnarray*}I_{i}\left(z\right)&=&\esp\left[z^{\tau_{i}\left(m+1\right)-\overline{\tau}\left(m\right)}\right]\end{eqnarray*} se tienen que

\begin{eqnarray*}
F_{i}\left(z\right)=I_{i}\left[P_{i}\left(z\right)\right]
\end{eqnarray*}
para $i=1,2$, por tanto



\begin{eqnarray*}
\esp\left[L_{i}^{*}\right]&=&\mu_{i}\esp\left[I_{i}\right]\\
Var\left[L_{i}^{*}\right]&=&\mu_{i}^{2}Var\left[I_{i}\right]+\sigma^{2}\esp\left[I_{i}\right]
\end{eqnarray*}
para $i=1,2$, por tanto


\begin{eqnarray*}
\esp\left[I_{i}\right]&=&\frac{f_{i}\left(i\right)}{\mu_{i}},
\end{eqnarray*}
adem\'as

\begin{eqnarray*}
Var\left[I_{i}\right]&=&\frac{Var\left[L_{i}^{*}\right]}{\mu_{i}^{2}}-\frac{\sigma_{i}^{2}}{\mu_{i}^{2}}f_{i}\left(i\right).
\end{eqnarray*}


Si  $C_{i}\left(z\right)=\esp\left[z^{\overline{\tau}\left(m+1\right)-\overline{\tau}_{i}\left(m\right)}\right]$el tiempo de duraci\'on del ciclo, entonces, por lo hasta ahora establecido, se tiene que

\begin{eqnarray*}
C_{i}\left(z\right)=I_{i}\left[\theta_{i}\left(z\right)\right],
\end{eqnarray*}
entonces

\begin{eqnarray*}
\esp\left[C_{i}\right]&=&\esp\left[I_{i}\right]\esp\left[\theta_{i}\left(z\right)\right]=\frac{\esp\left[L_{i}^{*}\right]}{\mu_{i}}\frac{1}{1-\mu_{i}}=\frac{f_{i}\left(i\right)}{\mu_{i}\left(1-\mu_{i}\right)}\\
Var\left[C_{i}\right]&=&\frac{Var\left[L_{i}^{*}\right]}{\mu_{i}^{2}\left(1-\mu_{i}\right)^{2}}.
\end{eqnarray*}

Por tanto se tienen las siguientes igualdades


\begin{eqnarray*}
\esp\left[S_{i}\right]&=&\mu_{i}\esp\left[C_{i}\right],\\
\esp\left[I_{i}\right]&=&\left(1-\mu_{i}\right)\esp\left[C_{i}\right]\\
\end{eqnarray*}

Def\'inanse los puntos de regenaraci\'on  en el proceso $\left[L_{1}\left(t\right),L_{2}\left(t\right),\ldots,L_{N}\left(t\right)\right]$. Los puntos cuando la cola $i$ es visitada y todos los $L_{j}\left(\tau_{i}\left(m\right)\right)=0$ para $i=1,2$  son puntos de regeneraci\'on. Se llama ciclo regenerativo al intervalo entre dos puntos regenerativos sucesivos.

Sea $M_{i}$  el n\'umero de ciclos de visita en un ciclo regenerativo, y sea $C_{i}^{(m)}$, para $m=1,2,\ldots,M_{i}$ la duraci\'on del $m$-\'esimo ciclo de visita en un ciclo regenerativo. Se define el ciclo del tiempo de visita promedio $\esp\left[C_{i}\right]$ como

\begin{eqnarray*}
\esp\left[C_{i}\right]&=&\frac{\esp\left[\sum_{m=1}^{M_{i}}C_{i}^{(m)}\right]}{\esp\left[M_{i}\right]}
\end{eqnarray*}


En Stid72 y Heym82 se muestra que una condici\'on suficiente para que el proceso regenerativo 
estacionario sea un procesoo estacionario es que el valor esperado del tiempo del ciclo regenerativo sea finito:

\begin{eqnarray*}
\esp\left[\sum_{m=1}^{M_{i}}C_{i}^{(m)}\right]<\infty.
\end{eqnarray*}

como cada $C_{i}^{(m)}$ contiene intervalos de r\'eplica positivos, se tiene que $\esp\left[M_{i}\right]<\infty$, adem\'as, como $M_{i}>0$, se tiene que la condici\'on anterior es equivalente a tener que 

\begin{eqnarray*}
\esp\left[C_{i}\right]<\infty,
\end{eqnarray*}
por lo tanto una condici\'on suficiente para la existencia del proceso regenerativo est\'a dada por

\begin{eqnarray*}
\sum_{k=1}^{N}\mu_{k}<1.
\end{eqnarray*}

Sea la funci\'on generadora de momentos para $L_{i}$, el n\'umero de usuarios en la cola $Q_{i}\left(z\right)$ en cualquier momento, est\'a dada por el tiempo promedio de $z^{L_{i}\left(t\right)}$ sobre el ciclo regenerativo definido anteriormente:

\begin{eqnarray*}
Q_{i}\left(z\right)&=&\esp\left[z^{L_{i}\left(t\right)}\right]=\frac{\esp\left[\sum_{m=1}^{M_{i}}\sum_{t=\tau_{i}\left(m\right)}^{\tau_{i}\left(m+1\right)-1}z^{L_{i}\left(t\right)}\right]}{\esp\left[\sum_{m=1}^{M_{i}}\tau_{i}\left(m+1\right)-\tau_{i}\left(m\right)\right]}
\end{eqnarray*}

$M_{i}$ es un tiempo de paro en el proceso regenerativo con $\esp\left[M_{i}\right]<\infty$, se sigue del lema de Wald que:


\begin{eqnarray*}
\esp\left[\sum_{m=1}^{M_{i}}\sum_{t=\tau_{i}\left(m\right)}^{\tau_{i}\left(m+1\right)-1}z^{L_{i}\left(t\right)}\right]&=&\esp\left[M_{i}\right]\esp\left[\sum_{t=\tau_{i}\left(m\right)}^{\tau_{i}\left(m+1\right)-1}z^{L_{i}\left(t\right)}\right]\\
\esp\left[\sum_{m=1}^{M_{i}}\tau_{i}\left(m+1\right)-\tau_{i}\left(m\right)\right]&=&\esp\left[M_{i}\right]\esp\left[\tau_{i}\left(m+1\right)-\tau_{i}\left(m\right)\right]
\end{eqnarray*}

por tanto se tiene que


\begin{eqnarray*}
Q_{i}\left(z\right)&=&\frac{\esp\left[\sum_{t=\tau_{i}\left(m\right)}^{\tau_{i}\left(m+1\right)-1}z^{L_{i}\left(t\right)}\right]}{\esp\left[\tau_{i}\left(m+1\right)-\tau_{i}\left(m\right)\right]}
\end{eqnarray*}

observar que el denominador es simplemente la duraci\'on promedio del tiempo del ciclo.


Se puede demostrar (ver Hideaki Takagi 1986) que

\begin{eqnarray*}
\esp\left[\sum_{t=\tau_{i}\left(m\right)}^{\tau_{i}\left(m+1\right)-1}z^{L_{i}\left(t\right)}\right]=z\frac{F_{i}\left(z\right)-1}{z-P_{i}\left(z\right)}
\end{eqnarray*}

Durante el tiempo de intervisita para la cola $i$, $L_{i}\left(t\right)$ solamente se incrementa de manera que el incremento por intervalo de tiempo est\'a dado por la funci\'on generadora de probabilidades de $P_{i}\left(z\right)$, por tanto la suma sobre el tiempo de intervisita puede evaluarse como:

\begin{eqnarray*}
\esp\left[\sum_{t=\tau_{i}\left(m\right)}^{\tau_{i}\left(m+1\right)-1}z^{L_{i}\left(t\right)}\right]&=&\esp\left[\sum_{t=\tau_{i}\left(m\right)}^{\tau_{i}\left(m+1\right)-1}\left\{P_{i}\left(z\right)\right\}^{t-\overline{\tau}_{i}\left(m\right)}\right]=\frac{1-\esp\left[\left\{P_{i}\left(z\right)\right\}^{\tau_{i}\left(m+1\right)-\overline{\tau}_{i}\left(m\right)}\right]}{1-P_{i}\left(z\right)}\\
&=&\frac{1-I_{i}\left[P_{i}\left(z\right)\right]}{1-P_{i}\left(z\right)}
\end{eqnarray*}
por tanto

\begin{eqnarray*}
\esp\left[\sum_{t=\tau_{i}\left(m\right)}^{\tau_{i}\left(m+1\right)-1}z^{L_{i}\left(t\right)}\right]&=&\frac{1-F_{i}\left(z\right)}{1-P_{i}\left(z\right)}
\end{eqnarray*}

Haciendo uso de lo hasta ahora desarrollado se tiene que

\begin{eqnarray*}
Q_{i}\left(z\right)&=&\frac{1}{\esp\left[C_{i}\right]}\cdot\frac{1-F_{i}\left(z\right)}{P_{i}\left(z\right)-z}\cdot\frac{\left(1-z\right)P_{i}\left(z\right)}{1-P_{i}\left(z\right)}\\
&=&\frac{\mu_{i}\left(1-\mu_{i}\right)}{f_{i}\left(i\right)}\cdot\frac{1-F_{i}\left(z\right)}{P_{i}\left(z\right)-z}\cdot\frac{\left(1-z\right)P_{i}\left(z\right)}{1-P_{i}\left(z\right)}
\end{eqnarray*}


%___________________________________________________________________________________________
\subsection{Longitudes de la Cola en cualquier tiempo}
%___________________________________________________________________________________________
Sea 
\begin{eqnarray*}
Q_{i}\left(z\right)&=&\frac{1}{\esp\left[C_{i}\right]}\cdot\frac{1-F_{i}\left(z\right)}{P_{i}\left(z\right)-z}\cdot\frac{\left(1-z\right)P_{i}\left(z\right)}{1-P_{i}\left(z\right)}
\end{eqnarray*}

derivando con respecto a $z$



\begin{eqnarray*}
\frac{d Q_{i}\left(z\right)}{d z}&=&\frac{\left(1-F_{i}\left(z\right)\right)P_{i}\left(z\right)}{\esp\left[C_{i}\right]\left(1-P_{i}\left(z\right)\right)\left(P_{i}\left(z\right)-z\right)}\\
&-&\frac{\left(1-z\right)P_{i}\left(z\right)F_{i}^{'}\left(z\right)}{\esp\left[C_{i}\right]\left(1-P_{i}\left(z\right)\right)\left(P_{i}\left(z\right)-z\right)}\\
&-&\frac{\left(1-z\right)\left(1-F_{i}\left(z\right)\right)P_{i}\left(z\right)\left(P_{i}^{'}\left(z\right)-1\right)}{\esp\left[C_{i}\right]\left(1-P_{i}\left(z\right)\right)\left(P_{i}\left(z\right)-z\right)^{2}}\\
&+&\frac{\left(1-z\right)\left(1-F_{i}\left(z\right)\right)P_{i}^{'}\left(z\right)}{\esp\left[C_{i}\right]\left(1-P_{i}\left(z\right)\right)\left(P_{i}\left(z\right)-z\right)}\\
&+&\frac{\left(1-z\right)\left(1-F_{i}\left(z\right)\right)P_{i}\left(z\right)P_{i}^{'}\left(z\right)}{\esp\left[C_{i}\right]\left(1-P_{i}\left(z\right)\right)^{2}\left(P_{i}\left(z\right)-z\right)}
\end{eqnarray*}

Calculando el l\'imite cuando $z\rightarrow1^{+}$:
\begin{eqnarray}
Q_{i}^{(1)}\left(z\right)=\lim_{z\rightarrow1^{+}}\frac{d Q_{i}\left(z\right)}{dz}&=&\lim_{z\rightarrow1}\frac{\left(1-F_{i}\left(z\right)\right)P_{i}\left(z\right)}{\esp\left[C_{i}\right]\left(1-P_{i}\left(z\right)\right)\left(P_{i}\left(z\right)-z\right)}\\
&-&\lim_{z\rightarrow1^{+}}\frac{\left(1-z\right)P_{i}\left(z\right)F_{i}^{'}\left(z\right)}{\esp\left[C_{i}\right]\left(1-P_{i}\left(z\right)\right)\left(P_{i}\left(z\right)-z\right)}\\
&-&\lim_{z\rightarrow1^{+}}\frac{\left(1-z\right)\left(1-F_{i}\left(z\right)\right)P_{i}\left(z\right)\left(P_{i}^{'}\left(z\right)-1\right)}{\esp\left[C_{i}\right]\left(1-P_{i}\left(z\right)\right)\left(P_{i}\left(z\right)-z\right)^{2}}\\
&+&\lim_{z\rightarrow1^{+}}\frac{\left(1-z\right)\left(1-F_{i}\left(z\right)\right)P_{i}^{'}\left(z\right)}{\esp\left[C_{i}\right]\left(1-P_{i}\left(z\right)\right)\left(P_{i}\left(z\right)-z\right)}\\
&+&\lim_{z\rightarrow1^{+}}\frac{\left(1-z\right)\left(1-F_{i}\left(z\right)\right)P_{i}\left(z\right)P_{i}^{'}\left(z\right)}{\esp\left[C_{i}\right]\left(1-P_{i}\left(z\right)\right)^{2}\left(P_{i}\left(z\right)-z\right)}
\end{eqnarray}

Entonces:
%______________________________________________________

\begin{eqnarray*}
\lim_{z\rightarrow1^{+}}\frac{\left(1-F_{i}\left(z\right)\right)P_{i}\left(z\right)}{\left(1-P_{i}\left(z\right)\right)\left(P_{i}\left(z\right)-z\right)}&=&\lim_{z\rightarrow1^{+}}\frac{\frac{d}{dz}\left[\left(1-F_{i}\left(z\right)\right)P_{i}\left(z\right)\right]}{\frac{d}{dz}\left[\left(1-P_{i}\left(z\right)\right)\left(-z+P_{i}\left(z\right)\right)\right]}\\
&=&\lim_{z\rightarrow1^{+}}\frac{-P_{i}\left(z\right)F_{i}^{'}\left(z\right)+\left(1-F_{i}\left(z\right)\right)P_{i}^{'}\left(z\right)}{\left(1-P_{i}\left(z\right)\right)\left(-1+P_{i}^{'}\left(z\right)\right)-\left(-z+P_{i}\left(z\right)\right)P_{i}^{'}\left(z\right)}
\end{eqnarray*}


%______________________________________________________


\begin{eqnarray*}
\lim_{z\rightarrow1^{+}}\frac{\left(1-z\right)P_{i}\left(z\right)F_{i}^{'}\left(z\right)}{\left(1-P_{i}\left(z\right)\right)\left(P_{i}\left(z\right)-z\right)}&=&\lim_{z\rightarrow1^{+}}\frac{\frac{d}{dz}\left[\left(1-z\right)P_{i}\left(z\right)F_{i}^{'}\left(z\right)\right]}{\frac{d}{dz}\left[\left(1-P_{i}\left(z\right)\right)\left(P_{i}\left(z\right)-z\right)\right]}\\
&=&\lim_{z\rightarrow1^{+}}\frac{-P_{i}\left(z\right) F_{i}^{'}\left(z\right)+(1-z) F_{i}^{'}\left(z\right) P_{i}^{'}\left(z\right)+(1-z) P_{i}\left(z\right)F_{i}^{''}\left(z\right)}{\left(1-P_{i}\left(z\right)\right)\left(-1+P_{i}^{'}\left(z\right)\right)-\left(-z+P_{i}\left(z\right)\right)P_{i}^{'}\left(z\right)}
\end{eqnarray*}


%______________________________________________________

\begin{eqnarray*}
&&\lim_{z\rightarrow1^{+}}\frac{\left(1-z\right)\left(1-F_{i}\left(z\right)\right)P_{i}\left(z\right)\left(P_{i}^{'}\left(z\right)-1\right)}{\left(1-P_{i}\left(z\right)\right)\left(P_{i}\left(z\right)-z\right)^{2}}=\lim_{z\rightarrow1^{+}}\frac{\frac{d}{dz}\left[\left(1-z\right)\left(1-F_{i}\left(z\right)\right)P_{i}\left(z\right)\left(P_{i}^{'}\left(z\right)-1\right)\right]}{\frac{d}{dz}\left[\left(1-P_{i}\left(z\right)\right)\left(P_{i}\left(z\right)-z\right)^{2}\right]}\\
&=&\lim_{z\rightarrow1^{+}}\frac{-\left(1-F_{i}\left(z\right)\right) P_{i}\left(z\right)\left(-1+P_{i}^{'}\left(z\right)\right)-(1-z) P_{i}\left(z\right)F_{i}^{'}\left(z\right)\left(-1+P_{i}^{'}\left(z\right)\right)}{2\left(1-P_{i}\left(z\right)\right)\left(-z+P_{i}\left(z\right)\right) \left(-1+P_{i}^{'}\left(z\right)\right)-\left(-z+P_{i}\left(z\right)\right)^2 P_{i}^{'}\left(z\right)}\\
&+&\lim_{z\rightarrow1^{+}}\frac{+(1-z) \left(1-F_{i}\left(z\right)\right) \left(-1+P_{i}^{'}\left(z\right)\right) P_{i}^{'}\left(z\right)}{{2\left(1-P_{i}\left(z\right)\right)\left(-z+P_{i}\left(z\right)\right) \left(-1+P_{i}^{'}\left(z\right)\right)-\left(-z+P_{i}\left(z\right)\right)^2 P_{i}^{'}\left(z\right)}}\\
&+&\lim_{z\rightarrow1^{+}}\frac{+(1-z) \left(1-F_{i}\left(z\right)\right) P_{i}\left(z\right)P_{i}^{''}\left(z\right)}{{2\left(1-P_{i}\left(z\right)\right)\left(-z+P_{i}\left(z\right)\right) \left(-1+P_{i}^{'}\left(z\right)\right)-\left(-z+P_{i}\left(z\right)\right)^2 P_{i}^{'}\left(z\right)}}
\end{eqnarray*}











%______________________________________________________
\begin{eqnarray*}
&&\lim_{z\rightarrow1^{+}}\frac{\left(1-z\right)\left(1-F_{i}\left(z\right)\right)P_{i}^{'}\left(z\right)}{\left(1-P_{i}\left(z\right)\right)\left(P_{i}\left(z\right)-z\right)}=\lim_{z\rightarrow1^{+}}\frac{\frac{d}{dz}\left[\left(1-z\right)\left(1-F_{i}\left(z\right)\right)P_{i}^{'}\left(z\right)\right]}{\frac{d}{dz}\left[\left(1-P_{i}\left(z\right)\right)\left(P_{i}\left(z\right)-z\right)\right]}\\
&=&\lim_{z\rightarrow1^{+}}\frac{-\left(1-F_{i}\left(z\right)\right) P_{i}^{'}\left(z\right)-(1-z) F_{i}^{'}\left(z\right) P_{i}^{'}\left(z\right)+(1-z) \left(1-F_{i}\left(z\right)\right) P_{i}^{''}\left(z\right)}{\left(1-P_{i}\left(z\right)\right) \left(-1+P_{i}^{'}\left(z\right)\right)-\left(-z+P_{i}\left(z\right)\right) P_{i}^{'}\left(z\right)}\frac{}{}
\end{eqnarray*}

%______________________________________________________
\begin{eqnarray*}
&&\lim_{z\rightarrow1^{+}}\frac{\left(1-z\right)\left(1-F_{i}\left(z\right)\right)P_{i}\left(z\right)P_{i}^{'}\left(z\right)}{\left(1-P_{i}\left(z\right)\right)^{2}\left(P_{i}\left(z\right)-z\right)}=\lim_{z\rightarrow1^{+}}\frac{\frac{d}{dz}\left[\left(1-z\right)\left(1-F_{i}\left(z\right)\right)P_{i}\left(z\right)P_{i}^{'}\left(z\right)\right]}{\frac{d}{dz}\left[\left(1-P_{i}\left(z\right)\right)^{2}\left(P_{i}\left(z\right)-z\right)\right]}\\
&=&\lim_{z\rightarrow1^{+}}\frac{-\left(1-F_{i}\left(z\right)\right) P_{i}\left(z\right) P_{i}^{'}\left(z\right)-(1-z) P_{i}\left(z\right) F_{i}^{'}\left(z\right)P_i'[z]}{\left(1-P_{i}\left(z\right)\right)^2 \left(-1+P_{i}^{'}\left(z\right)\right)-2 \left(1-P_{i}\left(z\right)\right) \left(-z+P_{i}\left(z\right)\right) P_{i}^{'}\left(z\right)}\\
&+&\lim_{z\rightarrow1^{+}}\frac{(1-z) \left(1-F_{i}\left(z\right)\right) P_{i}^{'}\left(z\right)^2+(1-z) \left(1-F_{i}\left(z\right)\right) P_{i}\left(z\right) P_{i}^{''}\left(z\right)}{\left(1-P_{i}\left(z\right)\right)^2 \left(-1+P_{i}^{'}\left(z\right)\right)-2 \left(1-P_{i}\left(z\right)\right) \left(-z+P_{i}\left(z\right)\right) P_{i}^{'}\left(z\right)}\\
\end{eqnarray*}

\subsection{Por resolver}



\begin{eqnarray*}
&&\frac{\partial Q_{i}\left(z\right)}{\partial z}=\frac{1}{\esp\left[C_{i}\right]}\frac{\partial}{\partial z}\left\{\frac{1-F_{i}\left(z\right)}{P_{i}\left(z\right)-z}\cdot\frac{\left(1-z\right)P_{i}\left(z\right)}{1-P_{i}\left(z\right)}\right\}\\
&=&\frac{1}{\esp\left[C_{i}\right]}\left\{\frac{\partial}{\partial z}\left(\frac{1-F_{i}\left(z\right)}{P_{i}\left(z\right)-z}\right)\cdot\frac{\left(1-z\right)P_{i}\left(z\right)}{1-P_{i}\left(z\right)}+\frac{1-F_{i}\left(z\right)}{P_{i}\left(z\right)-z}\cdot\frac{\partial}{\partial z}\left(\frac{\left(1-z\right)P_{i}\left(z\right)}{1-P_{i}\left(z\right)}\right)\right\}\\
&=&\frac{1}{\esp\left[C_{i}\right]}\cdot\frac{\left(1-z\right)P_{i}\left(z\right)}{1-P_{i}\left(z\right)}\cdot\frac{\partial}{\partial z}\left(\frac{1-F_{i}\left(z\right)}{P_{i}\left(z\right)-z}\right)+\frac{1}{\esp\left[C_{i}\right]}\cdot\frac{1-F_{i}\left(z\right)}{P_{i}\left(z\right)-z}\cdot\frac{\partial}{\partial z}\left(\frac{\left(1-z\right)P_{i}\left(z\right)}{1-P_{i}\left(z\right)}\right)\\
&=&\frac{1}{\esp\left[C_{i}\right]}\cdot\frac{\left(1-z\right)P_{i}\left(z\right)}{1-P_{i}\left(z\right)}\cdot\frac{-F_{i}^{'}\left(z\right)\left(P_{i}\left(z\right)-z\right)-\left(1-F_{i}\left(z\right)\right)\left(P_{i}^{'}\left(z\right)-1\right)}{\left(P_{i}\left(z\right)-z\right)^{2}}\\
&+&\frac{1}{\esp\left[C_{i}\right]}\cdot\frac{1-F_{i}\left(z\right)}{P_{i}\left(z\right)-z}\cdot\frac{\left(1-z\right)P_{i}^{'}\left(z\right)-P_{i}\left(z\right)}{\left(1-P_{i}\left(z\right)\right)^{2}}
\end{eqnarray*}



\begin{eqnarray*}
Q_{i}^{(1)}\left(z\right)&=& \frac{\left(1-F_{i}\left(z\right)\right)P_{i}\left(z\right)}{\esp\left[C_{i}\right]\left(1-P_{i}\left(z\right)\right)\left(P_{i}\left(z\right)-z\right)}
-\frac{\left(1-z\right)P_{i}\left(z\right)F_{i}^{'}\left(z\right)}{\esp\left[C_{i}\right]\left(1-P_{i}\left(z\right)\right)\left(P_{i}\left(z\right)-z\right)}\\
&-&\frac{\left(1-z\right)\left(1-F_{i}\left(z\right)\right)P_{i}\left(z\right)\left(P_{i}^{'}\left(z\right)-1\right)}{\esp\left[C_{i}\right]\left(1-P_{i}\left(z\right)\right)\left(P_{i}\left(z\right)-z\right)^{2}}+\frac{\left(1-z\right)\left(1-F_{i}\left(z\right)\right)P_{i}^{'}\left(z\right)}{\esp\left[C_{i}\right]\left(1-P_{i}\left(z\right)\right)\left(P_{i}\left(z\right)-z\right)}\\
&+&\frac{\left(1-z\right)\left(1-F_{i}\left(z\right)\right)P_{i}\left(z\right)P_{i}^{'}\left(z\right)}{\esp\left[C_{i}\right]\left(1-P_{i}\left(z\right)\right)^{2}\left(P_{i}\left(z\right)-z\right)}
\end{eqnarray*}
%___________________________________________________________________________________________
%\subsection{Operaciones Matemathica: Tiempos de Espera}
%___________________________________________________________________________________________
Sea
$V_{i}\left(z\right)=\frac{1}{\esp\left[C_{i}\right]}\frac{I_{i}\left(z\right)-1}{z-P_{i}\left(z\right)}$

%{\esp\lef[I_{i}\right]}\frac{1-\mu_{i}}{z-P_{i}\left(z\right)}

\begin{eqnarray*}
\frac{\partial V_{i}\left(z\right)}{\partial z}&=&\frac{1}{\esp\left[C_{i}\right]}\left[\frac{I_{i}{'}\left(z\right)\left(z-P_{i}\left(z\right)\right)}{z-P_{i}\left(z\right)}-\frac{\left(I_{i}\left(z\right)-1\right)\left(1-P_{i}{'}\left(z\right)\right)}{\left(z-P_{i}\left(z\right)\right)^{2}}\right]
\end{eqnarray*}


La FGP para el tiempo de espera para cualquier usuario en la cola est\'a dada por:
\[U_{i}\left(z\right)=\frac{1}{\esp\left[C_{i}\right]}\cdot\frac{1-P_{i}\left(z\right)}{z-P_{i}\left(z\right)}\cdot\frac{I_{i}\left(z\right)-1}{1-z}\]

entonces


\begin{eqnarray*}
\frac{d}{dz}V_{i}\left(z\right)&=&\frac{1}{\esp\left[C_{i}\right]}\left\{\frac{d}{dz}\left(\frac{1-P_{i}\left(z\right)}{z-P_{i}\left(z\right)}\right)\frac{I_{i}\left(z\right)-1}{1-z}+\frac{1-P_{i}\left(z\right)}{z-P_{i}\left(z\right)}\frac{d}{dz}\left(\frac{I_{i}\left(z\right)-1}{1-z}\right)\right\}\\
&=&\frac{1}{\esp\left[C_{i}\right]}\left\{\frac{-P_{i}\left(z\right)\left(z-P_{i}\left(z\right)\right)-\left(1-P_{i}\left(z\right)\right)\left(1-P_{i}^{'}\left(z\right)\right)}{\left(z-P_{i}\left(z\right)\right)^{2}}\cdot\frac{I_{i}\left(z\right)-1}{1-z}\right\}\\
&+&\frac{1}{\esp\left[C_{i}\right]}\left\{\frac{1-P_{i}\left(z\right)}{z-P_{i}\left(z\right)}\cdot\frac{I_{i}^{'}\left(z\right)\left(1-z\right)+\left(I_{i}\left(z\right)-1\right)}{\left(1-z\right)^{2}}\right\}
\end{eqnarray*}
%\frac{I_{i}\left(z\right)-1}{1-z}
%+\frac{1-P_{i}\left(z\right)}{z-P_{i}\frac{d}{dz}\left(\frac{I_{i}\left(z\right)-1}{1-z}\right)


\begin{eqnarray*}
\frac{\partial U_{i}\left(z\right)}{\partial z}&=&\frac{(-1+I_{i}[z]) (1-P_{i}[z])}{(1-z)^2 \esp[I_{i}] (z-P_{i}[z])}+\frac{(1-P_{i}[z]) I_{i}^{'}[z]}{(1-z) \esp[I_{i}] (z-P_{i}[z])}-\frac{(-1+I_{i}[z]) (1-P_{i}[z])\left(1-P{'}[z]\right)}{(1-z) \esp[I_{i}] (z-P_{i}[z])^2}\\
&-&\frac{(-1+I_{i}[z]) P_{i}{'}[z]}{(1-z) \esp[I_{i}](z-P_{i}[z])}
\end{eqnarray*}

%___________________________________________________________________________________________
\subsection{Tiempos de Ciclo e Intervisita}
%___________________________________________________________________________________________


\begin{Def}
Sea $L_{i}^{*}$el n\'umero de usuarios en la cola $Q_{i}$ cuando es visitada por el servidor para dar servicio, entonces

\begin{eqnarray}
\esp\left[L_{i}^{*}\right]&=&f_{i}\left(i\right)\\
Var\left[L_{i}^{*}\right]&=&f_{i}\left(i,i\right)+\esp\left[L_{i}^{*}\right]-\esp\left[L_{i}^{*}\right]^{2}.
\end{eqnarray}

\end{Def}

\begin{Def}
El tiempo de Ciclo $C_{i}$ es e periodo de tiempo que comienza cuando la cola $i$ es visitada por primera vez en un ciclo, y termina cuando es visitado nuevamente en el pr\'oximo ciclo. La duraci\'on del mismo est\'a dada por $\tau_{i}\left(m+1\right)-\tau_{i}\left(m\right)$, o equivalentemente $\overline{\tau}_{i}\left(m+1\right)-\overline{\tau}_{i}\left(m\right)$ bajo condiciones de estabilidad.
\end{Def}

\begin{Def}
El tiempo de intervisita $I_{i}$ es el periodo de tiempo que comienza cuando se ha completado el servicio en un ciclo y termina cuando es visitada nuevamente en el pr\'oximo ciclo. Su  duraci\'on del mismo est\'a dada por $\tau_{i}\left(m+1\right)-\overline{\tau}_{i}\left(m\right)$.
\end{Def}


Recordemos las siguientes expresiones:

\begin{eqnarray*}
S_{i}\left(z\right)&=&\esp\left[z^{\overline{\tau}_{i}\left(m\right)-\tau_{i}\left(m\right)}\right]=F_{i}\left(\theta\left(z\right)\right),\\
F\left(z\right)&=&\esp\left[z^{L_{0}}\right],\\
P\left(z\right)&=&\esp\left[z^{X_{n}}\right],\\
F_{i}\left(z\right)&=&\esp\left[z^{L_{i}\left(\tau_{i}\left(m\right)\right)}\right],
\theta_{i}\left(z\right)-zP_{i}
\end{eqnarray*}

entonces 

\begin{eqnarray*}
\esp\left[S_{i}\right]&=&\frac{\esp\left[L_{i}^{*}\right]}{1-\mu_{i}}=\frac{f_{i}\left(i\right)}{1-\mu_{i}},\\
Var\left[S_{i}\right]&=&\frac{Var\left[L_{i}^{*}\right]}{\left(1-\mu_{i}\right)^{2}}+\frac{\sigma^{2}\esp\left[L_{i}^{*}\right]}{\left(1-\mu_{i}\right)^{3}}
\end{eqnarray*}

donde recordemos que

\begin{eqnarray*}
Var\left[L_{i}^{*}\right]&=&f_{i}\left(i,i\right)+f_{i}\left(i\right)-f_{i}\left(i\right)^{2}.
\end{eqnarray*}

La duraci\'on del tiempo de intervisita es $\tau_{i}\left(m+1\right)-\overline{\tau}\left(m\right)$. Dado que el n\'umero de usuarios presentes en $Q_{i}$ al tiempo $t=\tau_{i}\left(m+1\right)$ es igual al n\'umero de arribos durante el intervalo de tiempo $\left[\overline{\tau}\left(m\right),\tau_{i}\left(m+1\right)\right]$ se tiene que


\begin{eqnarray*}
\esp\left[z_{i}^{L_{i}\left(\tau_{i}\left(m+1\right)\right)}\right]=\esp\left[\left\{P_{i}\left(z_{i}\right)\right\}^{\tau_{i}\left(m+1\right)-\overline{\tau}\left(m\right)}\right]
\end{eqnarray*}

entonces, si \begin{eqnarray*}I_{i}\left(z\right)&=&\esp\left[z^{\tau_{i}\left(m+1\right)-\overline{\tau}\left(m\right)}\right]\end{eqnarray*} se tienen que

\begin{eqnarray*}
F_{i}\left(z\right)=I_{i}\left[P_{i}\left(z\right)\right]
\end{eqnarray*}
para $i=1,2$, por tanto



\begin{eqnarray*}
\esp\left[L_{i}^{*}\right]&=&\mu_{i}\esp\left[I_{i}\right]\\
Var\left[L_{i}^{*}\right]&=&\mu_{i}^{2}Var\left[I_{i}\right]+\sigma^{2}\esp\left[I_{i}\right]
\end{eqnarray*}
para $i=1,2$, por tanto


\begin{eqnarray*}
\esp\left[I_{i}\right]&=&\frac{f_{i}\left(i\right)}{\mu_{i}},
\end{eqnarray*}
adem\'as

\begin{eqnarray*}
Var\left[I_{i}\right]&=&\frac{Var\left[L_{i}^{*}\right]}{\mu_{i}^{2}}-\frac{\sigma_{i}^{2}}{\mu_{i}^{2}}f_{i}\left(i\right).
\end{eqnarray*}


Si  $C_{i}\left(z\right)=\esp\left[z^{\overline{\tau}\left(m+1\right)-\overline{\tau}_{i}\left(m\right)}\right]$el tiempo de duraci\'on del ciclo, entonces, por lo hasta ahora establecido, se tiene que

\begin{eqnarray*}
C_{i}\left(z\right)=I_{i}\left[\theta_{i}\left(z\right)\right],
\end{eqnarray*}
entonces

\begin{eqnarray*}
\esp\left[C_{i}\right]&=&\esp\left[I_{i}\right]\esp\left[\theta_{i}\left(z\right)\right]=\frac{\esp\left[L_{i}^{*}\right]}{\mu_{i}}\frac{1}{1-\mu_{i}}=\frac{f_{i}\left(i\right)}{\mu_{i}\left(1-\mu_{i}\right)}\\
Var\left[C_{i}\right]&=&\frac{Var\left[L_{i}^{*}\right]}{\mu_{i}^{2}\left(1-\mu_{i}\right)^{2}}.
\end{eqnarray*}

Por tanto se tienen las siguientes igualdades


\begin{eqnarray*}
\esp\left[S_{i}\right]&=&\mu_{i}\esp\left[C_{i}\right],\\
\esp\left[I_{i}\right]&=&\left(1-\mu_{i}\right)\esp\left[C_{i}\right]\\
\end{eqnarray*}

Def\'inanse los puntos de regenaraci\'on  en el proceso $\left[L_{1}\left(t\right),L_{2}\left(t\right),\ldots,L_{N}\left(t\right)\right]$. Los puntos cuando la cola $i$ es visitada y todos los $L_{j}\left(\tau_{i}\left(m\right)\right)=0$ para $i=1,2$  son puntos de regeneraci\'on. Se llama ciclo regenerativo al intervalo entre dos puntos regenerativos sucesivos.

Sea $M_{i}$  el n\'umero de ciclos de visita en un ciclo regenerativo, y sea $C_{i}^{(m)}$, para $m=1,2,\ldots,M_{i}$ la duraci\'on del $m$-\'esimo ciclo de visita en un ciclo regenerativo. Se define el ciclo del tiempo de visita promedio $\esp\left[C_{i}\right]$ como

\begin{eqnarray*}
\esp\left[C_{i}\right]&=&\frac{\esp\left[\sum_{m=1}^{M_{i}}C_{i}^{(m)}\right]}{\esp\left[M_{i}\right]}
\end{eqnarray*}


En Stid72 y Heym82 se muestra que una condici\'on suficiente para que el proceso regenerativo 
estacionario sea un procesoo estacionario es que el valor esperado del tiempo del ciclo regenerativo sea finito:

\begin{eqnarray*}
\esp\left[\sum_{m=1}^{M_{i}}C_{i}^{(m)}\right]<\infty.
\end{eqnarray*}

como cada $C_{i}^{(m)}$ contiene intervalos de r\'eplica positivos, se tiene que $\esp\left[M_{i}\right]<\infty$, adem\'as, como $M_{i}>0$, se tiene que la condici\'on anterior es equivalente a tener que 

\begin{eqnarray*}
\esp\left[C_{i}\right]<\infty,
\end{eqnarray*}
por lo tanto una condici\'on suficiente para la existencia del proceso regenerativo est\'a dada por

\begin{eqnarray*}
\sum_{k=1}^{N}\mu_{k}<1.
\end{eqnarray*}

Sea la funci\'on generadora de momentos para $L_{i}$, el n\'umero de usuarios en la cola $Q_{i}\left(z\right)$ en cualquier momento, est\'a dada por el tiempo promedio de $z^{L_{i}\left(t\right)}$ sobre el ciclo regenerativo definido anteriormente:

\begin{eqnarray*}
Q_{i}\left(z\right)&=&\esp\left[z^{L_{i}\left(t\right)}\right]=\frac{\esp\left[\sum_{m=1}^{M_{i}}\sum_{t=\tau_{i}\left(m\right)}^{\tau_{i}\left(m+1\right)-1}z^{L_{i}\left(t\right)}\right]}{\esp\left[\sum_{m=1}^{M_{i}}\tau_{i}\left(m+1\right)-\tau_{i}\left(m\right)\right]}
\end{eqnarray*}

$M_{i}$ es un tiempo de paro en el proceso regenerativo con $\esp\left[M_{i}\right]<\infty$, se sigue del lema de Wald que:


\begin{eqnarray*}
\esp\left[\sum_{m=1}^{M_{i}}\sum_{t=\tau_{i}\left(m\right)}^{\tau_{i}\left(m+1\right)-1}z^{L_{i}\left(t\right)}\right]&=&\esp\left[M_{i}\right]\esp\left[\sum_{t=\tau_{i}\left(m\right)}^{\tau_{i}\left(m+1\right)-1}z^{L_{i}\left(t\right)}\right]\\
\esp\left[\sum_{m=1}^{M_{i}}\tau_{i}\left(m+1\right)-\tau_{i}\left(m\right)\right]&=&\esp\left[M_{i}\right]\esp\left[\tau_{i}\left(m+1\right)-\tau_{i}\left(m\right)\right]
\end{eqnarray*}

por tanto se tiene que


\begin{eqnarray*}
Q_{i}\left(z\right)&=&\frac{\esp\left[\sum_{t=\tau_{i}\left(m\right)}^{\tau_{i}\left(m+1\right)-1}z^{L_{i}\left(t\right)}\right]}{\esp\left[\tau_{i}\left(m+1\right)-\tau_{i}\left(m\right)\right]}
\end{eqnarray*}

observar que el denominador es simplemente la duraci\'on promedio del tiempo del ciclo.


Se puede demostrar (ver Hideaki Takagi 1986) que

\begin{eqnarray*}
\esp\left[\sum_{t=\tau_{i}\left(m\right)}^{\tau_{i}\left(m+1\right)-1}z^{L_{i}\left(t\right)}\right]=z\frac{F_{i}\left(z\right)-1}{z-P_{i}\left(z\right)}
\end{eqnarray*}

Durante el tiempo de intervisita para la cola $i$, $L_{i}\left(t\right)$ solamente se incrementa de manera que el incremento por intervalo de tiempo est\'a dado por la funci\'on generadora de probabilidades de $P_{i}\left(z\right)$, por tanto la suma sobre el tiempo de intervisita puede evaluarse como:

\begin{eqnarray*}
\esp\left[\sum_{t=\tau_{i}\left(m\right)}^{\tau_{i}\left(m+1\right)-1}z^{L_{i}\left(t\right)}\right]&=&\esp\left[\sum_{t=\tau_{i}\left(m\right)}^{\tau_{i}\left(m+1\right)-1}\left\{P_{i}\left(z\right)\right\}^{t-\overline{\tau}_{i}\left(m\right)}\right]=\frac{1-\esp\left[\left\{P_{i}\left(z\right)\right\}^{\tau_{i}\left(m+1\right)-\overline{\tau}_{i}\left(m\right)}\right]}{1-P_{i}\left(z\right)}\\
&=&\frac{1-I_{i}\left[P_{i}\left(z\right)\right]}{1-P_{i}\left(z\right)}
\end{eqnarray*}
por tanto

\begin{eqnarray*}
\esp\left[\sum_{t=\tau_{i}\left(m\right)}^{\tau_{i}\left(m+1\right)-1}z^{L_{i}\left(t\right)}\right]&=&\frac{1-F_{i}\left(z\right)}{1-P_{i}\left(z\right)}
\end{eqnarray*}

Haciendo uso de lo hasta ahora desarrollado se tiene que

\begin{eqnarray*}
Q_{i}\left(z\right)&=&\frac{1}{\esp\left[C_{i}\right]}\cdot\frac{1-F_{i}\left(z\right)}{P_{i}\left(z\right)-z}\cdot\frac{\left(1-z\right)P_{i}\left(z\right)}{1-P_{i}\left(z\right)}\\
&=&\frac{\mu_{i}\left(1-\mu_{i}\right)}{f_{i}\left(i\right)}\cdot\frac{1-F_{i}\left(z\right)}{P_{i}\left(z\right)-z}\cdot\frac{\left(1-z\right)P_{i}\left(z\right)}{1-P_{i}\left(z\right)}
\end{eqnarray*}

derivando con respecto a $z$



\begin{eqnarray*}
\frac{d Q_{i}\left(z\right)}{d z}&=&\frac{\left(1-F_{i}\left(z\right)\right)P_{i}\left(z\right)}{\esp\left[C_{i}\right]\left(1-P_{i}\left(z\right)\right)\left(P_{i}\left(z\right)-z\right)}\\
&-&\frac{\left(1-z\right)P_{i}\left(z\right)F_{i}^{'}\left(z\right)}{\esp\left[C_{i}\right]\left(1-P_{i}\left(z\right)\right)\left(P_{i}\left(z\right)-z\right)}\\
&-&\frac{\left(1-z\right)\left(1-F_{i}\left(z\right)\right)P_{i}\left(z\right)\left(P_{i}^{'}\left(z\right)-1\right)}{\esp\left[C_{i}\right]\left(1-P_{i}\left(z\right)\right)\left(P_{i}\left(z\right)-z\right)^{2}}\\
&+&\frac{\left(1-z\right)\left(1-F_{i}\left(z\right)\right)P_{i}^{'}\left(z\right)}{\esp\left[C_{i}\right]\left(1-P_{i}\left(z\right)\right)\left(P_{i}\left(z\right)-z\right)}\\
&+&\frac{\left(1-z\right)\left(1-F_{i}\left(z\right)\right)P_{i}\left(z\right)P_{i}^{'}\left(z\right)}{\esp\left[C_{i}\right]\left(1-P_{i}\left(z\right)\right)^{2}\left(P_{i}\left(z\right)-z\right)}
\end{eqnarray*}

Calculando el l\'imite cuando $z\rightarrow1^{+}$:
\begin{eqnarray}
Q_{i}^{(1)}\left(z\right)=\lim_{z\rightarrow1^{+}}\frac{d Q_{i}\left(z\right)}{dz}&=&\lim_{z\rightarrow1}\frac{\left(1-F_{i}\left(z\right)\right)P_{i}\left(z\right)}{\esp\left[C_{i}\right]\left(1-P_{i}\left(z\right)\right)\left(P_{i}\left(z\right)-z\right)}\\
&-&\lim_{z\rightarrow1^{+}}\frac{\left(1-z\right)P_{i}\left(z\right)F_{i}^{'}\left(z\right)}{\esp\left[C_{i}\right]\left(1-P_{i}\left(z\right)\right)\left(P_{i}\left(z\right)-z\right)}\\
&-&\lim_{z\rightarrow1^{+}}\frac{\left(1-z\right)\left(1-F_{i}\left(z\right)\right)P_{i}\left(z\right)\left(P_{i}^{'}\left(z\right)-1\right)}{\esp\left[C_{i}\right]\left(1-P_{i}\left(z\right)\right)\left(P_{i}\left(z\right)-z\right)^{2}}\\
&+&\lim_{z\rightarrow1^{+}}\frac{\left(1-z\right)\left(1-F_{i}\left(z\right)\right)P_{i}^{'}\left(z\right)}{\esp\left[C_{i}\right]\left(1-P_{i}\left(z\right)\right)\left(P_{i}\left(z\right)-z\right)}\\
&+&\lim_{z\rightarrow1^{+}}\frac{\left(1-z\right)\left(1-F_{i}\left(z\right)\right)P_{i}\left(z\right)P_{i}^{'}\left(z\right)}{\esp\left[C_{i}\right]\left(1-P_{i}\left(z\right)\right)^{2}\left(P_{i}\left(z\right)-z\right)}
\end{eqnarray}

Entonces:
%______________________________________________________

\begin{eqnarray*}
\lim_{z\rightarrow1^{+}}\frac{\left(1-F_{i}\left(z\right)\right)P_{i}\left(z\right)}{\left(1-P_{i}\left(z\right)\right)\left(P_{i}\left(z\right)-z\right)}&=&\lim_{z\rightarrow1^{+}}\frac{\frac{d}{dz}\left[\left(1-F_{i}\left(z\right)\right)P_{i}\left(z\right)\right]}{\frac{d}{dz}\left[\left(1-P_{i}\left(z\right)\right)\left(-z+P_{i}\left(z\right)\right)\right]}\\
&=&\lim_{z\rightarrow1^{+}}\frac{-P_{i}\left(z\right)F_{i}^{'}\left(z\right)+\left(1-F_{i}\left(z\right)\right)P_{i}^{'}\left(z\right)}{\left(1-P_{i}\left(z\right)\right)\left(-1+P_{i}^{'}\left(z\right)\right)-\left(-z+P_{i}\left(z\right)\right)P_{i}^{'}\left(z\right)}
\end{eqnarray*}


%______________________________________________________


\begin{eqnarray*}
\lim_{z\rightarrow1^{+}}\frac{\left(1-z\right)P_{i}\left(z\right)F_{i}^{'}\left(z\right)}{\left(1-P_{i}\left(z\right)\right)\left(P_{i}\left(z\right)-z\right)}&=&\lim_{z\rightarrow1^{+}}\frac{\frac{d}{dz}\left[\left(1-z\right)P_{i}\left(z\right)F_{i}^{'}\left(z\right)\right]}{\frac{d}{dz}\left[\left(1-P_{i}\left(z\right)\right)\left(P_{i}\left(z\right)-z\right)\right]}\\
&=&\lim_{z\rightarrow1^{+}}\frac{-P_{i}\left(z\right) F_{i}^{'}\left(z\right)+(1-z) F_{i}^{'}\left(z\right) P_{i}^{'}\left(z\right)+(1-z) P_{i}\left(z\right)F_{i}^{''}\left(z\right)}{\left(1-P_{i}\left(z\right)\right)\left(-1+P_{i}^{'}\left(z\right)\right)-\left(-z+P_{i}\left(z\right)\right)P_{i}^{'}\left(z\right)}
\end{eqnarray*}


%______________________________________________________

\begin{eqnarray*}
&&\lim_{z\rightarrow1^{+}}\frac{\left(1-z\right)\left(1-F_{i}\left(z\right)\right)P_{i}\left(z\right)\left(P_{i}^{'}\left(z\right)-1\right)}{\left(1-P_{i}\left(z\right)\right)\left(P_{i}\left(z\right)-z\right)^{2}}=\lim_{z\rightarrow1^{+}}\frac{\frac{d}{dz}\left[\left(1-z\right)\left(1-F_{i}\left(z\right)\right)P_{i}\left(z\right)\left(P_{i}^{'}\left(z\right)-1\right)\right]}{\frac{d}{dz}\left[\left(1-P_{i}\left(z\right)\right)\left(P_{i}\left(z\right)-z\right)^{2}\right]}\\
&=&\lim_{z\rightarrow1^{+}}\frac{-\left(1-F_{i}\left(z\right)\right) P_{i}\left(z\right)\left(-1+P_{i}^{'}\left(z\right)\right)-(1-z) P_{i}\left(z\right)F_{i}^{'}\left(z\right)\left(-1+P_{i}^{'}\left(z\right)\right)}{2\left(1-P_{i}\left(z\right)\right)\left(-z+P_{i}\left(z\right)\right) \left(-1+P_{i}^{'}\left(z\right)\right)-\left(-z+P_{i}\left(z\right)\right)^2 P_{i}^{'}\left(z\right)}\\
&+&\lim_{z\rightarrow1^{+}}\frac{+(1-z) \left(1-F_{i}\left(z\right)\right) \left(-1+P_{i}^{'}\left(z\right)\right) P_{i}^{'}\left(z\right)}{{2\left(1-P_{i}\left(z\right)\right)\left(-z+P_{i}\left(z\right)\right) \left(-1+P_{i}^{'}\left(z\right)\right)-\left(-z+P_{i}\left(z\right)\right)^2 P_{i}^{'}\left(z\right)}}\\
&+&\lim_{z\rightarrow1^{+}}\frac{+(1-z) \left(1-F_{i}\left(z\right)\right) P_{i}\left(z\right)P_{i}^{''}\left(z\right)}{{2\left(1-P_{i}\left(z\right)\right)\left(-z+P_{i}\left(z\right)\right) \left(-1+P_{i}^{'}\left(z\right)\right)-\left(-z+P_{i}\left(z\right)\right)^2 P_{i}^{'}\left(z\right)}}
\end{eqnarray*}











%______________________________________________________
\begin{eqnarray*}
&&\lim_{z\rightarrow1^{+}}\frac{\left(1-z\right)\left(1-F_{i}\left(z\right)\right)P_{i}^{'}\left(z\right)}{\left(1-P_{i}\left(z\right)\right)\left(P_{i}\left(z\right)-z\right)}=\lim_{z\rightarrow1^{+}}\frac{\frac{d}{dz}\left[\left(1-z\right)\left(1-F_{i}\left(z\right)\right)P_{i}^{'}\left(z\right)\right]}{\frac{d}{dz}\left[\left(1-P_{i}\left(z\right)\right)\left(P_{i}\left(z\right)-z\right)\right]}\\
&=&\lim_{z\rightarrow1^{+}}\frac{-\left(1-F_{i}\left(z\right)\right) P_{i}^{'}\left(z\right)-(1-z) F_{i}^{'}\left(z\right) P_{i}^{'}\left(z\right)+(1-z) \left(1-F_{i}\left(z\right)\right) P_{i}^{''}\left(z\right)}{\left(1-P_{i}\left(z\right)\right) \left(-1+P_{i}^{'}\left(z\right)\right)-\left(-z+P_{i}\left(z\right)\right) P_{i}^{'}\left(z\right)}\frac{}{}
\end{eqnarray*}

%______________________________________________________
\begin{eqnarray*}
&&\lim_{z\rightarrow1^{+}}\frac{\left(1-z\right)\left(1-F_{i}\left(z\right)\right)P_{i}\left(z\right)P_{i}^{'}\left(z\right)}{\left(1-P_{i}\left(z\right)\right)^{2}\left(P_{i}\left(z\right)-z\right)}=\lim_{z\rightarrow1^{+}}\frac{\frac{d}{dz}\left[\left(1-z\right)\left(1-F_{i}\left(z\right)\right)P_{i}\left(z\right)P_{i}^{'}\left(z\right)\right]}{\frac{d}{dz}\left[\left(1-P_{i}\left(z\right)\right)^{2}\left(P_{i}\left(z\right)-z\right)\right]}\\
&=&\lim_{z\rightarrow1^{+}}\frac{-\left(1-F_{i}\left(z\right)\right) P_{i}\left(z\right) P_{i}^{'}\left(z\right)-(1-z) P_{i}\left(z\right) F_{i}^{'}\left(z\right)P_i'[z]}{\left(1-P_{i}\left(z\right)\right)^2 \left(-1+P_{i}^{'}\left(z\right)\right)-2 \left(1-P_{i}\left(z\right)\right) \left(-z+P_{i}\left(z\right)\right) P_{i}^{'}\left(z\right)}\\
&+&\lim_{z\rightarrow1^{+}}\frac{(1-z) \left(1-F_{i}\left(z\right)\right) P_{i}^{'}\left(z\right)^2+(1-z) \left(1-F_{i}\left(z\right)\right) P_{i}\left(z\right) P_{i}^{''}\left(z\right)}{\left(1-P_{i}\left(z\right)\right)^2 \left(-1+P_{i}^{'}\left(z\right)\right)-2 \left(1-P_{i}\left(z\right)\right) \left(-z+P_{i}\left(z\right)\right) P_{i}^{'}\left(z\right)}\\
\end{eqnarray*}

%___________________________________________________________________________________________
\subsection{Longitudes de la Cola en cualquier tiempo}
%___________________________________________________________________________________________

Sea
$V_{i}\left(z\right)=\frac{1}{\esp\left[C_{i}\right]}\frac{I_{i}\left(z\right)-1}{z-P_{i}\left(z\right)}$

%{\esp\lef[I_{i}\right]}\frac{1-\mu_{i}}{z-P_{i}\left(z\right)}

\begin{eqnarray*}
\frac{\partial V_{i}\left(z\right)}{\partial z}&=&\frac{1}{\esp\left[C_{i}\right]}\left[\frac{I_{i}{'}\left(z\right)\left(z-P_{i}\left(z\right)\right)}{z-P_{i}\left(z\right)}-\frac{\left(I_{i}\left(z\right)-1\right)\left(1-P_{i}{'}\left(z\right)\right)}{\left(z-P_{i}\left(z\right)\right)^{2}}\right]
\end{eqnarray*}


La FGP para el tiempo de espera para cualquier usuario en la cola est\'a dada por:
\[U_{i}\left(z\right)=\frac{1}{\esp\left[C_{i}\right]}\cdot\frac{1-P_{i}\left(z\right)}{z-P_{i}\left(z\right)}\cdot\frac{I_{i}\left(z\right)-1}{1-z}\]

entonces


\begin{eqnarray*}
\frac{d}{dz}V_{i}\left(z\right)&=&\frac{1}{\esp\left[C_{i}\right]}\left\{\frac{d}{dz}\left(\frac{1-P_{i}\left(z\right)}{z-P_{i}\left(z\right)}\right)\frac{I_{i}\left(z\right)-1}{1-z}+\frac{1-P_{i}\left(z\right)}{z-P_{i}\left(z\right)}\frac{d}{dz}\left(\frac{I_{i}\left(z\right)-1}{1-z}\right)\right\}\\
&=&\frac{1}{\esp\left[C_{i}\right]}\left\{\frac{-P_{i}\left(z\right)\left(z-P_{i}\left(z\right)\right)-\left(1-P_{i}\left(z\right)\right)\left(1-P_{i}^{'}\left(z\right)\right)}{\left(z-P_{i}\left(z\right)\right)^{2}}\cdot\frac{I_{i}\left(z\right)-1}{1-z}\right\}\\
&+&\frac{1}{\esp\left[C_{i}\right]}\left\{\frac{1-P_{i}\left(z\right)}{z-P_{i}\left(z\right)}\cdot\frac{I_{i}^{'}\left(z\right)\left(1-z\right)+\left(I_{i}\left(z\right)-1\right)}{\left(1-z\right)^{2}}\right\}
\end{eqnarray*}
%\frac{I_{i}\left(z\right)-1}{1-z}
%+\frac{1-P_{i}\left(z\right)}{z-P_{i}\frac{d}{dz}\left(\frac{I_{i}\left(z\right)-1}{1-z}\right)


\begin{eqnarray*}
\frac{\partial U_{i}\left(z\right)}{\partial z}&=&\frac{(-1+I_{i}[z]) (1-P_{i}[z])}{(1-z)^2 \esp[I_{i}] (z-P_{i}[z])}+\frac{(1-P_{i}[z]) I_{i}^{'}[z]}{(1-z) \esp[I_{i}] (z-P_{i}[z])}-\frac{(-1+I_{i}[z]) (1-P_{i}[z])\left(1-P{'}[z]\right)}{(1-z) \esp[I_{i}] (z-P_{i}[z])^2}\\
&-&\frac{(-1+I_{i}[z]) P_{i}{'}[z]}{(1-z) \esp[I_{i}](z-P_{i}[z])}
\end{eqnarray*}


\subsection{Material por agregar}


\begin{Teo}
Dada una Red de Sistemas de Visitas C\'iclicas (RSVC), conformada por dos Sistemas de Visitas C\'iclicas (SVC), donde cada uno de ellos consta de dos colas tipo $M/M/1$. Los dos sistemas est\'an comunicados entre s\'i por medio de la transferencia de usuarios entre las colas $Q_{1}\leftrightarrow Q_{3}$ y $Q_{2}\leftrightarrow Q_{4}$. Se definen los eventos para los procesos de arribos al tiempo $t$, $A_{j}\left(t\right)=\left\{0 \textrm{ arribos en }Q_{j}\textrm{ al tiempo }t\right\}$ para alg\'un tiempo $t\geq0$ y $Q_{j}$ la cola $j$-\'esima en la RSVC, para $j=1,2,3,4$.  Existe un intervalo $I\neq\emptyset$ tal que para $T^{*}\in I$, tal que $\prob\left\{A_{1}\left(T^{*}\right),A_{2}\left(Tt^{*}\right),
A_{3}\left(T^{*}\right),A_{4}\left(T^{*}\right)|T^{*}\in I\right\}>0$.
\end{Teo}



\begin{proof}
Sin p\'erdida de generalidad podemos considerar como base del an\'alisis a la cola $Q_{1}$ del primer sistema que conforma la RSVC.\medskip 

Sea $n\geq1$, ciclo en el primer sistema en el que se sabe que $L_{j}\left(\overline{\tau}_{1}\left(n\right)\right)=0$, pues la pol\'itica de servicio con que atienden los servidores es la exhaustiva. Como es sabido, para trasladarse a la siguiente cola, el servidor incurre en un tiempo de traslado $r_{1}\left(n\right)>0$, entonces tenemos que $\tau_{2}\left(n\right)=\overline{\tau}_{1}\left(n\right)+r_{1}\left(n\right)$.\medskip 


Definamos el intervalo $I_{1}\equiv\left[\overline{\tau}_{1}\left(n\right),\tau_{2}\left(n\right)\right]$ de longitud $\xi_{1}=r_{1}\left(n\right)$.

Dado que los tiempos entre arribo son exponenciales con tasa $\tilde{\mu}_{1}=\mu_{1}+\hat{\mu}_{1}$ ($\mu_{1}$ son los arribos a $Q_{1}$ por primera vez al sistema, mientras que $\hat{\mu}_{1}$ son los arribos de traslado procedentes de $Q_{3}$) se tiene que la probabilidad del evento $A_{1}\left(t\right)$ est\'a dada por 

\begin{equation}
\prob\left\{A_{1}\left(t\right)|t\in I_{1}\left(n\right)\right\}=e^{-\tilde{\mu}_{1}\xi_{1}\left(n\right)}.
\end{equation} 


Por otra parte, para la cola $Q_{2}$ el tiempo $\overline{\tau}_{2}\left(n-1\right)$ es tal que $L_{2}\left(\overline{\tau}_{2}\left(n-1\right)\right)=0$, es decir, es el tiempo en que la cola queda totalmente vac\'ia en el ciclo anterior a $n$. \medskip 


Entonces tenemos un sgundo intervalo $I_{2}\equiv\left[\overline{\tau}_{2}\left(n-1\right),\tau_{2}\left(n\right)\right]$. Por lo tanto la probabilidad del evento $A_{2}\left(t\right)$ tiene probabilidad dada por

\begin{eqnarray}
\prob\left\{A_{2}\left(t\right)|t\in I_{2}\left(n\right)\right\}=e^{-\tilde{\mu}_{2}\xi_{2}\left(n\right)},\\
\xi_{2}\left(n\right)=\tau_{2}\left(n\right)-\overline{\tau}_{2}\left(n-1\right)
\end{eqnarray}
%\end{equation} 

%donde $$.

Ahora, dado que $I_{1}\left(n\right)\subset I_{2}\left(n\right)$, se tiene que

\begin{eqnarray*}
\xi_{1}\left(n\right)\leq\xi_{2}\left(n\right)&\Leftrightarrow& -\xi_{1}\left(n\right)\geq-\xi_{2}\left(n\right)
\\
-\tilde{\mu}_{2}\xi_{1}\left(n\right)\geq-\tilde{\mu}_{2}\xi_{2}\left(n\right)&\Leftrightarrow&
e^{-\tilde{\mu}_{2}\xi_{1}\left(n\right)}\geq e^{-\tilde{\mu}_{2}\xi_{2}\left(n\right)}\\
\prob\left\{A_{2}\left(t\right)|t\in I_{1}\left(n\right)\right\}&\geq&
\prob\left\{A_{2}\left(t\right)|t\in I_{2}\left(n\right)\right\}.
\end{eqnarray*}


Entonces se tiene que
\small{
\begin{eqnarray*}
\prob\left\{A_{1}\left(t\right),A_{2}\left(t\right)|t\in I_{1}\left(n\right)\right\}&=&
\prob\left\{A_{1}\left(t\right)|t\in I_{1}\left(n\right)\right\}
\prob\left\{A_{2}\left(t\right)|t\in I_{1}\left(n\right)\right\}\\
&\geq&
\prob\left\{A_{1}\left(t\right)|t\in I_{1}\left(n\right)\right\}
\prob\left\{A_{2}\left(t\right)|t\in I_{2}\left(n\right)\right\}\\
&=&e^{-\tilde{\mu}_{1}\xi_{1}\left(n\right)}e^{-\tilde{\mu}_{2}\xi_{2}\left(n\right)}
=e^{-\left[\tilde{\mu}_{1}\xi_{1}\left(n\right)+\tilde{\mu}_{2}\xi_{2}\left(n\right)\right]}.
\end{eqnarray*}}


Es decir, 

\begin{equation}
\prob\left\{A_{1}\left(t\right),A_{2}\left(t\right)|t\in I_{1}\left(n\right)\right\}
=e^{-\left[\tilde{\mu}_{1}\xi_{1}\left(n\right)+\tilde{\mu}_{2}\xi_{2}
\left(n\right)\right]}>0.
\end{equation}
En lo que respecta a la relaci\'on entre los dos SVC que conforman la RSVC para alg\'un $m\geq1$ se tiene que $\tau_{3}\left(m\right)<\tau_{2}\left(n\right)<\tau_{4}\left(m\right)$ por lo tanto se cumple cualquiera de los siguientes cuatro casos
\begin{itemize}
\item[a)] $\tau_{3}\left(m\right)<\tau_{2}\left(n\right)<\overline{\tau}_{3}\left(m\right)$

\item[b)] $\overline{\tau}_{3}\left(m\right)<\tau_{2}\left(n\right)
<\tau_{4}\left(m\right)$

\item[c)] $\tau_{4}\left(m\right)<\tau_{2}\left(n\right)<
\overline{\tau}_{4}\left(m\right)$

\item[d)] $\overline{\tau}_{4}\left(m\right)<\tau_{2}\left(n\right)
<\tau_{3}\left(m+1\right)$
\end{itemize}


Sea el intervalo $I_{3}\left(m\right)\equiv\left[\tau_{3}\left(m\right),\overline{\tau}_{3}\left(m\right)\right]$ tal que $\tau_{2}\left(n\right)\in I_{3}\left(m\right)$, con longitud de intervalo $\xi_{3}\equiv\overline{\tau}_{3}\left(m\right)-\tau_{3}\left(m\right)$, entonces se tiene que para $Q_{3}$
\begin{equation}
\prob\left\{A_{3}\left(t\right)|t\in I_{3}\left(m\right)\right\}=e^{-\tilde{\mu}_{3}\xi_{3}\left(m\right)}.
\end{equation} 

mientras que para $Q_{4}$ consideremos el intervalo $I_{4}\left(m\right)\equiv\left[\tau_{4}\left(m-1\right),\overline{\tau}_{3}\left(m\right)\right]$, entonces por construcci\'on  $I_{3}\left(m\right)\subset I_{4}\left(m\right)$, por lo tanto


\begin{eqnarray*}
\xi_{3}\left(m\right)\leq\xi_{4}\left(m\right)&\Leftrightarrow& -\xi_{3}\left(m\right)\geq-\xi_{4}\left(m\right)
\\
-\tilde{\mu}_{4}\xi_{3}\left(m\right)\geq-\tilde{\mu}_{4}\xi_{4}\left(m\right)&\Leftrightarrow&
e^{-\tilde{\mu}_{4}\xi_{3}\left(m\right)}\geq e^{-\tilde{\mu}_{4}\xi_{4}\left(n\right)}\\
\prob\left\{A_{4}\left(t\right)|t\in I_{3}\left(m\right)\right\}&\geq&
\prob\left\{A_{4}\left(t\right)|t\in I_{4}\left(m\right)\right\}.
\end{eqnarray*}



Entonces se tiene que
\small{
\begin{eqnarray*}
\prob\left\{A_{3}\left(t\right),A_{4}\left(t\right)|t\in I_{3}\left(m\right)\right\}&=&
\prob\left\{A_{3}\left(t\right)|t\in I_{3}\left(m\right)\right\}
\prob\left\{A_{4}\left(t\right)|t\in I_{3}\left(m\right)\right\}\\
&\geq&
\prob\left\{A_{3}\left(t\right)|t\in I_{3}\left(m\right)\right\}
\prob\left\{A_{4}\left(t\right)|t\in I_{4}\left(m\right)\right\}\\
&=&e^{-\tilde{\mu}_{3}\xi_{3}\left(m\right)}e^{-\tilde{\mu}_{4}\xi_{4}
\left(m\right)}
=e^{-\left(\tilde{\mu}_{3}\xi_{3}\left(m\right)+\tilde{\mu}_{4}\xi_{4}\left(m\right)\right)}.
\end{eqnarray*}}

Es decir, 

\begin{equation}
\prob\left\{A_{3}\left(t\right),A_{4}\left(t\right)|t\in I_{3}\left(m\right)\right\}\geq
e^{-\left(\tilde{\mu}_{3}\xi_{3}\left(m\right)+\tilde{\mu}_{4}\xi_{4}\left(m\right)\right)}>0.
\end{equation}


Sea el intervalo $I_{3}\left(m\right)\equiv\left[\overline{\tau}_{3}\left(m\right),\tau_{4}\left(m\right)\right]$ con longitud $\xi_{3}\equiv\tau_{4}\left(m\right)-\overline{\tau}_{3}\left(m\right)$, entonces se tiene que para $Q_{3}$
\begin{equation}
\prob\left\{A_{3}\left(t\right)|t\in I_{3}\left(m\right)\right\}=e^{-\tilde{\mu}_{3}\xi_{3}\left(m\right)}.
\end{equation} 

mientras que para $Q_{4}$ consideremos el intervalo $I_{4}\left(m\right)\equiv\left[\overline{\tau}_{4}\left(m-1\right),\tau_{4}\left(m\right)\right]$, entonces por construcci\'on  $I_{3}\left(m\right)\subset I_{4}\left(m\right)$, y al igual que en el caso anterior se tiene que 

\begin{eqnarray*}
\xi_{3}\left(m\right)\leq\xi_{4}\left(m\right)&\Leftrightarrow& -\xi_{3}\left(m\right)\geq-\xi_{4}\left(m\right)
\\
-\tilde{\mu}_{4}\xi_{3}\left(m\right)\geq-\tilde{\mu}_{4}\xi_{4}\left(m\right)&\Leftrightarrow&
e^{-\tilde{\mu}_{4}\xi_{3}\left(m\right)}\geq e^{-\tilde{\mu}_{4}\xi_{4}\left(n\right)}\\
\prob\left\{A_{4}\left(t\right)|t\in I_{3}\left(m\right)\right\}&\geq&
\prob\left\{A_{4}\left(t\right)|t\in I_{4}\left(m\right)\right\}.
\end{eqnarray*}


Entonces se tiene que
\small{
\begin{eqnarray*}
\prob\left\{A_{3}\left(t\right),A_{4}\left(t\right)|t\in I_{3}\left(m\right)\right\}&=&
\prob\left\{A_{3}\left(t\right)|t\in I_{3}\left(m\right)\right\}
\prob\left\{A_{4}\left(t\right)|t\in I_{3}\left(m\right)\right\}\\
&\geq&
\prob\left\{A_{3}\left(t\right)|t\in I_{3}\left(m\right)\right\}
\prob\left\{A_{4}\left(t\right)|t\in I_{4}\left(m\right)\right\}\\
&=&e^{-\tilde{\mu}_{3}\xi_{3}\left(m\right)}e^{-\tilde{\mu}_{4}\xi_{4}\left(m\right)}
=e^{-\left(\tilde{\mu}_{3}\xi_{3}\left(m\right)+\tilde{\mu}_{4}\xi_{4}\left(m\right)\right)}.
\end{eqnarray*}}

Es decir, 

\begin{equation}
\prob\left\{A_{3}\left(t\right),A_{4}\left(t\right)|t\in I_{4}\left(m\right)\right\}\geq
e^{-\left(\tilde{\mu}_{3}+\tilde{\mu}_{4}\right)\xi_{3}\left(m\right)}>0.
\end{equation}


Para el intervalo $I_{3}\left(m\right)=\left[\tau_{4}\left(m\right),\overline{\tau}_{4}\left(m\right)\right]$, se tiene que este caso es an\'alogo al caso (a).


Para el intevalo $I_{3}\left(m\right)\equiv\left[\overline{\tau}_{4}\left(m\right),\tau_{4}\left(m+1\right)\right]$, se tiene que es an\'alogo al caso (b).


Por construcci\'on se tiene que $I\left(n,m\right)\equiv I_{1}\left(n\right)\cap I_{3}\left(m\right)\neq\emptyset$,entonces en particular se tienen las contenciones $I\left(n,m\right)\subseteq I_{1}\left(n\right)$ y $I\left(n,m\right)\subseteq I_{3}\left(m\right)$, por lo tanto si definimos $\xi_{n,m}\equiv\ell\left(I\left(n,m\right)\right)$ tenemos que

\begin{eqnarray*}
\xi_{n,m}\leq\xi_{1}\left(n\right)\textrm{ y }\xi_{n,m}\leq\xi_{3}\left(m\right)\textrm{ entonces }\\
-\xi_{n,m}\geq-\xi_{1}\left(n\right)\textrm{ y }-\xi_{n,m}\leq-\xi_{3}\left(m\right)\\
\end{eqnarray*}
por lo tanto tenemos las desigualdades 


\begin{eqnarray*}
\begin{array}{ll}
-\tilde{\mu}_{1}\xi_{n,m}\geq-\tilde{\mu}_{1}\xi_{1}\left(n\right),&
-\tilde{\mu}_{2}\xi_{n,m}\geq-\tilde{\mu}_{2}\xi_{1}\left(n\right)
\geq-\tilde{\mu}_{2}\xi_{2}\left(n\right),\\
-\tilde{\mu}_{3}\xi_{n,m}\geq-\tilde{\mu}_{3}\xi_{3}\left(m\right),&
-\tilde{\mu}_{4}\xi_{n,m}\geq-\tilde{\mu}_{4}\xi_{3}\left(m\right)
\geq-\tilde{\mu}_{4}\xi_{4}\left(m\right).
\end{array}
\end{eqnarray*}

Sea $T^{*}\in I\left(n,m\right)$, entonces dado que en particular $T^{*}\in I_{1}\left(n\right)$, se cumple con probabilidad positiva que no hay arribos a las colas $Q_{1}$ y $Q_{2}$, en consecuencia, tampoco hay usuarios de transferencia para $Q_{3}$ y $Q_{4}$, es decir, $\tilde{\mu}_{1}=\mu_{1}$, $\tilde{\mu}_{2}=\mu_{2}$, $\tilde{\mu}_{3}=\mu_{3}$, $\tilde{\mu}_{4}=\mu_{4}$, es decir, los eventos $Q_{1}$ y $Q_{3}$ son condicionalmente independientes en el intervalo $I\left(n,m\right)$; lo mismo ocurre para las colas $Q_{2}$ y $Q_{4}$, por lo tanto tenemos que
%\small{
\begin{eqnarray}
\begin{array}{l}
\prob\left\{A_{1}\left(T^{*}\right),A_{2}\left(T^{*}\right),
A_{3}\left(T^{*}\right),A_{4}\left(T^{*}\right)|T^{*}\in I\left(n,m\right)\right\}\\
=\prod_{j=1}^{4}\prob\left\{A_{j}\left(T^{*}\right)|T^{*}\in I\left(n,m\right)\right\}\\
\geq\prob\left\{A_{1}\left(T^{*}\right)|T^{*}\in I_{1}\left(n\right)\right\}
\prob\left\{A_{2}\left(T^{*}\right)|T^{*}\in I_{2}\left(n\right)\right\}\\
\prob\left\{A_{3}\left(T^{*}\right)|T^{*}\in I_{3}\left(m\right)\right\}
\prob\left\{A_{4}\left(T^{*}\right)|T^{*}\in I_{4}\left(m\right)\right\}\\
=e^{-\mu_{1}\xi_{1}\left(n\right)}
e^{-\mu_{2}\xi_{2}\left(n\right)}
e^{-\mu_{3}\xi_{3}\left(m\right)}
e^{-\mu_{4}\xi_{4}\left(m\right)}\\
=e^{-\left[\tilde{\mu}_{1}\xi_{1}\left(n\right)
+\tilde{\mu}_{2}\xi_{2}\left(n\right)
+\tilde{\mu}_{3}\xi_{3}\left(m\right)
+\tilde{\mu}_{4}\xi_{4}
\left(m\right)\right]}>0.
\end{array}
\end{eqnarray}


Ahora solo resta demostrar que para $n\ge1$, existe $m\geq1$ tal que se cumplen cualquiera de los cuatro casos arriba mencionados: 

\begin{itemize}
\item[a)] $\tau_{3}\left(m\right)<\tau_{2}\left(n\right)<\overline{\tau}_{3}\left(m\right)$

\item[b)] $\overline{\tau}_{3}\left(m\right)<\tau_{2}\left(n\right)
<\tau_{4}\left(m\right)$

\item[c)] $\tau_{4}\left(m\right)<\tau_{2}\left(n\right)<
\overline{\tau}_{4}\left(m\right)$

\item[d)] $\overline{\tau}_{4}\left(m\right)<\tau_{2}\left(n\right)
<\tau_{3}\left(m+1\right)$
\end{itemize}

Consideremos nuevamente el primer caso. Supongamos que no existe $m\geq1$, tal que $I_{1}\left(n\right)\cap I_{3}\left(m\right)\neq\emptyset$, es decir, para toda $m\geq1$, $I_{1}\left(n\right)\cap I_{3}\left(m\right)=\emptyset$, entonces se tiene que ocurren cualquiera de los dos casos

\begin{itemize}
\item[a)] $\tau_{2}\left(n\right)\leq\tau_{3}\left(m\right)$: Recordemos que $\tau_{2}\left(m\right)=\overline{\tau}_{1}+r_{1}\left(m\right)$ donde cada una de las variables aleatorias son tales que $\esp\left[\overline{\tau}_{1}\left(n\right)-\tau_{1}\left(n\right)\right]<\infty$, $\esp\left[R_{1}\right]<\infty$ y $\esp\left[\tau_{3}\left(m\right)\right]<\infty$, lo cual contradice el hecho de que no exista un ciclo $m\geq1$ que satisfaga la condici\'on deseada.

\item[b)] $\tau_{2}\left(n\right)\geq\overline{\tau}_{3}\left(m\right)$: por un argumento similar al anterior se tiene que no es posible que no exista un ciclo $m\geq1$ tal que satisaface la condici\'on deseada.

\end{itemize}

Para el resto de los casos la demostraci\'on es an\'aloga. Por lo tanto, se tiene que efectivamente existe $m\geq1$ tal que $\tau_{3}\left(m\right)<\tau_{2}\left(n\right)<\tau_{4}\left(m\right)$.
\end{proof}
\newpage

En Sigman, Thorison y Wolff \cite{Sigman2} prueban que para la existencia de un una sucesi\'on infinita no decreciente de tiempos de regeneraci\'on $\tau_{1}\leq\tau_{2}\leq\cdots$ en los cuales el proceso se regenera, basta un tiempo de regeneraci\'on $R_{1}$, donde $R_{j}=\tau_{j}-\tau_{j-1}$. Para tal efecto se requiere la existencia de un espacio de probabilidad $\left(\Omega,\mathcal{F},\prob\right)$, y proceso estoc\'astico $\textit{X}=\left\{X\left(t\right):t\geq0\right\}$ con espacio de estados $\left(S,\mathcal{R}\right)$, con $\mathcal{R}$ $\sigma$-\'algebra.

\begin{Prop}
Si existe una variable aleatoria no negativa $R_{1}$ tal que $\theta_{R1}X=_{D}X$, entonces $\left(\Omega,\mathcal{F},\prob\right)$ puede extenderse para soportar una sucesi\'on estacionaria de variables aleatorias $R=\left\{R_{k}:k\geq1\right\}$, tal que para $k\geq1$,
\begin{eqnarray*}
\theta_{k}\left(X,R\right)=_{D}\left(X,R\right).
\end{eqnarray*}

Adem\'as, para $k\geq1$, $\theta_{k}R$ es condicionalmente independiente de $\left(X,R_{1},\ldots,R_{k}\right)$, dado $\theta_{\tau k}X$.

\end{Prop}


\begin{itemize}
\item Doob en 1953 demostr\'o que el estado estacionario de un proceso de partida en un sistema de espera $M/G/\infty$, es Poisson con la misma tasa que el proceso de arribos.

\item Burke en 1968, fue el primero en demostrar que el estado estacionario de un proceso de salida de una cola $M/M/s$ es un proceso Poisson.

\item Disney en 1973 obtuvo el siguiente resultado:

\begin{Teo}
Para el sistema de espera $M/G/1/L$ con disciplina FIFO, el proceso $\textbf{I}$ es un proceso de renovaci\'on si y s\'olo si el proceso denominado longitud de la cola es estacionario y se cumple cualquiera de los siguientes casos:

\begin{itemize}
\item[a)] Los tiempos de servicio son identicamente cero;
\item[b)] $L=0$, para cualquier proceso de servicio $S$;
\item[c)] $L=1$ y $G=D$;
\item[d)] $L=\infty$ y $G=M$.
\end{itemize}
En estos casos, respectivamente, las distribuciones de interpartida $P\left\{T_{n+1}-T_{n}\leq t\right\}$ son


\begin{itemize}
\item[a)] $1-e^{-\lambda t}$, $t\geq0$;
\item[b)] $1-e^{-\lambda t}*F\left(t\right)$, $t\geq0$;
\item[c)] $1-e^{-\lambda t}*\indora_{d}\left(t\right)$, $t\geq0$;
\item[d)] $1-e^{-\lambda t}*F\left(t\right)$, $t\geq0$.
\end{itemize}
\end{Teo}


\item Finch (1959) mostr\'o que para los sistemas $M/G/1/L$, con $1\leq L\leq \infty$ con distribuciones de servicio dos veces diferenciable, solamente el sistema $M/M/1/\infty$ tiene proceso de salida de renovaci\'on estacionario.

\item King (1971) demostro que un sistema de colas estacionario $M/G/1/1$ tiene sus tiempos de interpartida sucesivas $D_{n}$ y $D_{n+1}$ son independientes, si y s\'olo si, $G=D$, en cuyo caso le proceso de salida es de renovaci\'on.

\item Disney (1973) demostr\'o que el \'unico sistema estacionario $M/G/1/L$, que tiene proceso de salida de renovaci\'on  son los sistemas $M/M/1$ y $M/D/1/1$.



\item El siguiente resultado es de Disney y Koning (1985)
\begin{Teo}
En un sistema de espera $M/G/s$, el estado estacionario del proceso de salida es un proceso Poisson para cualquier distribuci\'on de los tiempos de servicio si el sistema tiene cualquiera de las siguientes cuatro propiedades.

\begin{itemize}
\item[a)] $s=\infty$
\item[b)] La disciplina de servicio es de procesador compartido.
\item[c)] La disciplina de servicio es LCFS y preemptive resume, esto se cumple para $L<\infty$
\item[d)] $G=M$.
\end{itemize}

\end{Teo}

\item El siguiente resultado es de Alamatsaz (1983)

\begin{Teo}
En cualquier sistema de colas $GI/G/1/L$ con $1\leq L<\infty$ y distribuci\'on de interarribos $A$ y distribuci\'on de los tiempos de servicio $B$, tal que $A\left(0\right)=0$, $A\left(t\right)\left(1-B\left(t\right)\right)>0$ para alguna $t>0$ y $B\left(t\right)$ para toda $t>0$, es imposible que el proceso de salida estacionario sea de renovaci\'on.
\end{Teo}

\end{itemize}



%________________________________________________________________________
%\subsection{Procesos Regenerativos Sigman, Thorisson y Wolff \cite{Sigman1}}
%________________________________________________________________________


\begin{Def}[Definici\'on Cl\'asica]
Un proceso estoc\'astico $X=\left\{X\left(t\right):t\geq0\right\}$ es llamado regenerativo is existe una variable aleatoria $R_{1}>0$ tal que
\begin{itemize}
\item[i)] $\left\{X\left(t+R_{1}\right):t\geq0\right\}$ es independiente de $\left\{\left\{X\left(t\right):t<R_{1}\right\},\right\}$
\item[ii)] $\left\{X\left(t+R_{1}\right):t\geq0\right\}$ es estoc\'asticamente equivalente a $\left\{X\left(t\right):t>0\right\}$
\end{itemize}

Llamamos a $R_{1}$ tiempo de regeneraci\'on, y decimos que $X$ se regenera en este punto.
\end{Def}

$\left\{X\left(t+R_{1}\right)\right\}$ es regenerativo con tiempo de regeneraci\'on $R_{2}$, independiente de $R_{1}$ pero con la misma distribuci\'on que $R_{1}$. Procediendo de esta manera se obtiene una secuencia de variables aleatorias independientes e id\'enticamente distribuidas $\left\{R_{n}\right\}$ llamados longitudes de ciclo. Si definimos a $Z_{k}\equiv R_{1}+R_{2}+\cdots+R_{k}$, se tiene un proceso de renovaci\'on llamado proceso de renovaci\'on encajado para $X$.


\begin{Note}
La existencia de un primer tiempo de regeneraci\'on, $R_{1}$, implica la existencia de una sucesi\'on completa de estos tiempos $R_{1},R_{2}\ldots,$ que satisfacen la propiedad deseada \cite{Sigman2}.
\end{Note}


\begin{Note} Para la cola $GI/GI/1$ los usuarios arriban con tiempos $t_{n}$ y son atendidos con tiempos de servicio $S_{n}$, los tiempos de arribo forman un proceso de renovaci\'on  con tiempos entre arribos independientes e identicamente distribuidos (\texttt{i.i.d.})$T_{n}=t_{n}-t_{n-1}$, adem\'as los tiempos de servicio son \texttt{i.i.d.} e independientes de los procesos de arribo. Por \textit{estable} se entiende que $\esp S_{n}<\esp T_{n}<\infty$.
\end{Note}
 


\begin{Def}
Para $x$ fijo y para cada $t\geq0$, sea $I_{x}\left(t\right)=1$ si $X\left(t\right)\leq x$,  $I_{x}\left(t\right)=0$ en caso contrario, y def\'inanse los tiempos promedio
\begin{eqnarray*}
\overline{X}&=&lim_{t\rightarrow\infty}\frac{1}{t}\int_{0}^{\infty}X\left(u\right)du\\
\prob\left(X_{\infty}\leq x\right)&=&lim_{t\rightarrow\infty}\frac{1}{t}\int_{0}^{\infty}I_{x}\left(u\right)du,
\end{eqnarray*}
cuando estos l\'imites existan.
\end{Def}

Como consecuencia del teorema de Renovaci\'on-Recompensa, se tiene que el primer l\'imite  existe y es igual a la constante
\begin{eqnarray*}
\overline{X}&=&\frac{\esp\left[\int_{0}^{R_{1}}X\left(t\right)dt\right]}{\esp\left[R_{1}\right]},
\end{eqnarray*}
suponiendo que ambas esperanzas son finitas.
 
\begin{Note}
Funciones de procesos regenerativos son regenerativas, es decir, si $X\left(t\right)$ es regenerativo y se define el proceso $Y\left(t\right)$ por $Y\left(t\right)=f\left(X\left(t\right)\right)$ para alguna funci\'on Borel medible $f\left(\cdot\right)$. Adem\'as $Y$ es regenerativo con los mismos tiempos de renovaci\'on que $X$. 

En general, los tiempos de renovaci\'on, $Z_{k}$ de un proceso regenerativo no requieren ser tiempos de paro con respecto a la evoluci\'on de $X\left(t\right)$.
\end{Note} 

\begin{Note}
Una funci\'on de un proceso de Markov, usualmente no ser\'a un proceso de Markov, sin embargo ser\'a regenerativo si el proceso de Markov lo es.
\end{Note}

 
\begin{Note}
Un proceso regenerativo con media de la longitud de ciclo finita es llamado positivo recurrente.
\end{Note}


\begin{Note}
\begin{itemize}
\item[a)] Si el proceso regenerativo $X$ es positivo recurrente y tiene trayectorias muestrales no negativas, entonces la ecuaci\'on anterior es v\'alida.
\item[b)] Si $X$ es positivo recurrente regenerativo, podemos construir una \'unica versi\'on estacionaria de este proceso, $X_{e}=\left\{X_{e}\left(t\right)\right\}$, donde $X_{e}$ es un proceso estoc\'astico regenerativo y estrictamente estacionario, con distribuci\'on marginal distribuida como $X_{\infty}$
\end{itemize}
\end{Note}


%__________________________________________________________________________________________
%\subsection{Procesos Regenerativos Estacionarios - Stidham \cite{Stidham}}
%__________________________________________________________________________________________


Un proceso estoc\'astico a tiempo continuo $\left\{V\left(t\right),t\geq0\right\}$ es un proceso regenerativo si existe una sucesi\'on de variables aleatorias independientes e id\'enticamente distribuidas $\left\{X_{1},X_{2},\ldots\right\}$, sucesi\'on de renovaci\'on, tal que para cualquier conjunto de Borel $A$, 

\begin{eqnarray*}
\prob\left\{V\left(t\right)\in A|X_{1}+X_{2}+\cdots+X_{R\left(t\right)}=s,\left\{V\left(\tau\right),\tau<s\right\}\right\}=\prob\left\{V\left(t-s\right)\in A|X_{1}>t-s\right\},
\end{eqnarray*}
para todo $0\leq s\leq t$, donde $R\left(t\right)=\max\left\{X_{1}+X_{2}+\cdots+X_{j}\leq t\right\}=$n\'umero de renovaciones ({\emph{puntos de regeneraci\'on}}) que ocurren en $\left[0,t\right]$. El intervalo $\left[0,X_{1}\right)$ es llamado {\emph{primer ciclo de regeneraci\'on}} de $\left\{V\left(t \right),t\geq0\right\}$, $\left[X_{1},X_{1}+X_{2}\right)$ el {\emph{segundo ciclo de regeneraci\'on}}, y as\'i sucesivamente.

Sea $X=X_{1}$ y sea $F$ la funci\'on de distrbuci\'on de $X$


\begin{Def}
Se define el proceso estacionario, $\left\{V^{*}\left(t\right),t\geq0\right\}$, para $\left\{V\left(t\right),t\geq0\right\}$ por

\begin{eqnarray*}
\prob\left\{V\left(t\right)\in A\right\}=\frac{1}{\esp\left[X\right]}\int_{0}^{\infty}\prob\left\{V\left(t+x\right)\in A|X>x\right\}\left(1-F\left(x\right)\right)dx,
\end{eqnarray*} 
para todo $t\geq0$ y todo conjunto de Borel $A$.
\end{Def}

\begin{Def}
Una distribuci\'on se dice que es {\emph{aritm\'etica}} si todos sus puntos de incremento son m\'ultiplos de la forma $0,\lambda, 2\lambda,\ldots$ para alguna $\lambda>0$ entera.
\end{Def}


\begin{Def}
Una modificaci\'on medible de un proceso $\left\{V\left(t\right),t\geq0\right\}$, es una versi\'on de este, $\left\{V\left(t,w\right)\right\}$ conjuntamente medible para $t\geq0$ y para $w\in S$, $S$ espacio de estados para $\left\{V\left(t\right),t\geq0\right\}$.
\end{Def}

\begin{Teo}
Sea $\left\{V\left(t\right),t\geq\right\}$ un proceso regenerativo no negativo con modificaci\'on medible. Sea $\esp\left[X\right]<\infty$. Entonces el proceso estacionario dado por la ecuaci\'on anterior est\'a bien definido y tiene funci\'on de distribuci\'on independiente de $t$, adem\'as
\begin{itemize}
\item[i)] \begin{eqnarray*}
\esp\left[V^{*}\left(0\right)\right]&=&\frac{\esp\left[\int_{0}^{X}V\left(s\right)ds\right]}{\esp\left[X\right]}\end{eqnarray*}
\item[ii)] Si $\esp\left[V^{*}\left(0\right)\right]<\infty$, equivalentemente, si $\esp\left[\int_{0}^{X}V\left(s\right)ds\right]<\infty$,entonces
\begin{eqnarray*}
\frac{\int_{0}^{t}V\left(s\right)ds}{t}\rightarrow\frac{\esp\left[\int_{0}^{X}V\left(s\right)ds\right]}{\esp\left[X\right]}
\end{eqnarray*}
con probabilidad 1 y en media, cuando $t\rightarrow\infty$.
\end{itemize}
\end{Teo}

\begin{Coro}
Sea $\left\{V\left(t\right),t\geq0\right\}$ un proceso regenerativo no negativo, con modificaci\'on medible. Si $\esp <\infty$, $F$ es no-aritm\'etica, y para todo $x\geq0$, $P\left\{V\left(t\right)\leq x,C>x\right\}$ es de variaci\'on acotada como funci\'on de $t$ en cada intervalo finito $\left[0,\tau\right]$, entonces $V\left(t\right)$ converge en distribuci\'on  cuando $t\rightarrow\infty$ y $$\esp V=\frac{\esp \int_{0}^{X}V\left(s\right)ds}{\esp X}$$
Donde $V$ tiene la distribuci\'on l\'imite de $V\left(t\right)$ cuando $t\rightarrow\infty$.

\end{Coro}

Para el caso discreto se tienen resultados similares.



%______________________________________________________________________
%\subsection{Procesos de Renovaci\'on}
%______________________________________________________________________

\begin{Def}%\label{Def.Tn}
Sean $0\leq T_{1}\leq T_{2}\leq \ldots$ son tiempos aleatorios infinitos en los cuales ocurren ciertos eventos. El n\'umero de tiempos $T_{n}$ en el intervalo $\left[0,t\right)$ es

\begin{eqnarray}
N\left(t\right)=\sum_{n=1}^{\infty}\indora\left(T_{n}\leq t\right),
\end{eqnarray}
para $t\geq0$.
\end{Def}

Si se consideran los puntos $T_{n}$ como elementos de $\rea_{+}$, y $N\left(t\right)$ es el n\'umero de puntos en $\rea$. El proceso denotado por $\left\{N\left(t\right):t\geq0\right\}$, denotado por $N\left(t\right)$, es un proceso puntual en $\rea_{+}$. Los $T_{n}$ son los tiempos de ocurrencia, el proceso puntual $N\left(t\right)$ es simple si su n\'umero de ocurrencias son distintas: $0<T_{1}<T_{2}<\ldots$ casi seguramente.

\begin{Def}
Un proceso puntual $N\left(t\right)$ es un proceso de renovaci\'on si los tiempos de interocurrencia $\xi_{n}=T_{n}-T_{n-1}$, para $n\geq1$, son independientes e identicamente distribuidos con distribuci\'on $F$, donde $F\left(0\right)=0$ y $T_{0}=0$. Los $T_{n}$ son llamados tiempos de renovaci\'on, referente a la independencia o renovaci\'on de la informaci\'on estoc\'astica en estos tiempos. Los $\xi_{n}$ son los tiempos de inter-renovaci\'on, y $N\left(t\right)$ es el n\'umero de renovaciones en el intervalo $\left[0,t\right)$
\end{Def}


\begin{Note}
Para definir un proceso de renovaci\'on para cualquier contexto, solamente hay que especificar una distribuci\'on $F$, con $F\left(0\right)=0$, para los tiempos de inter-renovaci\'on. La funci\'on $F$ en turno degune las otra variables aleatorias. De manera formal, existe un espacio de probabilidad y una sucesi\'on de variables aleatorias $\xi_{1},\xi_{2},\ldots$ definidas en este con distribuci\'on $F$. Entonces las otras cantidades son $T_{n}=\sum_{k=1}^{n}\xi_{k}$ y $N\left(t\right)=\sum_{n=1}^{\infty}\indora\left(T_{n}\leq t\right)$, donde $T_{n}\rightarrow\infty$ casi seguramente por la Ley Fuerte de los Grandes Números.
\end{Note}

%___________________________________________________________________________________________
%
%\subsection{Teorema Principal de Renovaci\'on}
%___________________________________________________________________________________________
%

\begin{Note} Una funci\'on $h:\rea_{+}\rightarrow\rea$ es Directamente Riemann Integrable en los siguientes casos:
\begin{itemize}
\item[a)] $h\left(t\right)\geq0$ es decreciente y Riemann Integrable.
\item[b)] $h$ es continua excepto posiblemente en un conjunto de Lebesgue de medida 0, y $|h\left(t\right)|\leq b\left(t\right)$, donde $b$ es DRI.
\end{itemize}
\end{Note}

\begin{Teo}[Teorema Principal de Renovaci\'on]
Si $F$ es no aritm\'etica y $h\left(t\right)$ es Directamente Riemann Integrable (DRI), entonces

\begin{eqnarray*}
lim_{t\rightarrow\infty}U\star h=\frac{1}{\mu}\int_{\rea_{+}}h\left(s\right)ds.
\end{eqnarray*}
\end{Teo}

\begin{Prop}
Cualquier funci\'on $H\left(t\right)$ acotada en intervalos finitos y que es 0 para $t<0$ puede expresarse como
\begin{eqnarray*}
H\left(t\right)=U\star h\left(t\right)\textrm{,  donde }h\left(t\right)=H\left(t\right)-F\star H\left(t\right)
\end{eqnarray*}
\end{Prop}

\begin{Def}
Un proceso estoc\'astico $X\left(t\right)$ es crudamente regenerativo en un tiempo aleatorio positivo $T$ si
\begin{eqnarray*}
\esp\left[X\left(T+t\right)|T\right]=\esp\left[X\left(t\right)\right]\textrm{, para }t\geq0,\end{eqnarray*}
y con las esperanzas anteriores finitas.
\end{Def}

\begin{Prop}
Sup\'ongase que $X\left(t\right)$ es un proceso crudamente regenerativo en $T$, que tiene distribuci\'on $F$. Si $\esp\left[X\left(t\right)\right]$ es acotado en intervalos finitos, entonces
\begin{eqnarray*}
\esp\left[X\left(t\right)\right]=U\star h\left(t\right)\textrm{,  donde }h\left(t\right)=\esp\left[X\left(t\right)\indora\left(T>t\right)\right].
\end{eqnarray*}
\end{Prop}

\begin{Teo}[Regeneraci\'on Cruda]
Sup\'ongase que $X\left(t\right)$ es un proceso con valores positivo crudamente regenerativo en $T$, y def\'inase $M=\sup\left\{|X\left(t\right)|:t\leq T\right\}$. Si $T$ es no aritm\'etico y $M$ y $MT$ tienen media finita, entonces
\begin{eqnarray*}
lim_{t\rightarrow\infty}\esp\left[X\left(t\right)\right]=\frac{1}{\mu}\int_{\rea_{+}}h\left(s\right)ds,
\end{eqnarray*}
donde $h\left(t\right)=\esp\left[X\left(t\right)\indora\left(T>t\right)\right]$.
\end{Teo}

%___________________________________________________________________________________________
%
%\subsection{Propiedades de los Procesos de Renovaci\'on}
%___________________________________________________________________________________________
%

Los tiempos $T_{n}$ est\'an relacionados con los conteos de $N\left(t\right)$ por

\begin{eqnarray*}
\left\{N\left(t\right)\geq n\right\}&=&\left\{T_{n}\leq t\right\}\\
T_{N\left(t\right)}\leq &t&<T_{N\left(t\right)+1},
\end{eqnarray*}

adem\'as $N\left(T_{n}\right)=n$, y 

\begin{eqnarray*}
N\left(t\right)=\max\left\{n:T_{n}\leq t\right\}=\min\left\{n:T_{n+1}>t\right\}
\end{eqnarray*}

Por propiedades de la convoluci\'on se sabe que

\begin{eqnarray*}
P\left\{T_{n}\leq t\right\}=F^{n\star}\left(t\right)
\end{eqnarray*}
que es la $n$-\'esima convoluci\'on de $F$. Entonces 

\begin{eqnarray*}
\left\{N\left(t\right)\geq n\right\}&=&\left\{T_{n}\leq t\right\}\\
P\left\{N\left(t\right)\leq n\right\}&=&1-F^{\left(n+1\right)\star}\left(t\right)
\end{eqnarray*}

Adem\'as usando el hecho de que $\esp\left[N\left(t\right)\right]=\sum_{n=1}^{\infty}P\left\{N\left(t\right)\geq n\right\}$
se tiene que

\begin{eqnarray*}
\esp\left[N\left(t\right)\right]=\sum_{n=1}^{\infty}F^{n\star}\left(t\right)
\end{eqnarray*}

\begin{Prop}
Para cada $t\geq0$, la funci\'on generadora de momentos $\esp\left[e^{\alpha N\left(t\right)}\right]$ existe para alguna $\alpha$ en una vecindad del 0, y de aqu\'i que $\esp\left[N\left(t\right)^{m}\right]<\infty$, para $m\geq1$.
\end{Prop}


\begin{Note}
Si el primer tiempo de renovaci\'on $\xi_{1}$ no tiene la misma distribuci\'on que el resto de las $\xi_{n}$, para $n\geq2$, a $N\left(t\right)$ se le llama Proceso de Renovaci\'on retardado, donde si $\xi$ tiene distribuci\'on $G$, entonces el tiempo $T_{n}$ de la $n$-\'esima renovaci\'on tiene distribuci\'on $G\star F^{\left(n-1\right)\star}\left(t\right)$
\end{Note}


\begin{Teo}
Para una constante $\mu\leq\infty$ ( o variable aleatoria), las siguientes expresiones son equivalentes:

\begin{eqnarray}
lim_{n\rightarrow\infty}n^{-1}T_{n}&=&\mu,\textrm{ c.s.}\\
lim_{t\rightarrow\infty}t^{-1}N\left(t\right)&=&1/\mu,\textrm{ c.s.}
\end{eqnarray}
\end{Teo}


Es decir, $T_{n}$ satisface la Ley Fuerte de los Grandes N\'umeros s\'i y s\'olo s\'i $N\left/t\right)$ la cumple.


\begin{Coro}[Ley Fuerte de los Grandes N\'umeros para Procesos de Renovaci\'on]
Si $N\left(t\right)$ es un proceso de renovaci\'on cuyos tiempos de inter-renovaci\'on tienen media $\mu\leq\infty$, entonces
\begin{eqnarray}
t^{-1}N\left(t\right)\rightarrow 1/\mu,\textrm{ c.s. cuando }t\rightarrow\infty.
\end{eqnarray}

\end{Coro}


Considerar el proceso estoc\'astico de valores reales $\left\{Z\left(t\right):t\geq0\right\}$ en el mismo espacio de probabilidad que $N\left(t\right)$

\begin{Def}
Para el proceso $\left\{Z\left(t\right):t\geq0\right\}$ se define la fluctuaci\'on m\'axima de $Z\left(t\right)$ en el intervalo $\left(T_{n-1},T_{n}\right]$:
\begin{eqnarray*}
M_{n}=\sup_{T_{n-1}<t\leq T_{n}}|Z\left(t\right)-Z\left(T_{n-1}\right)|
\end{eqnarray*}
\end{Def}

\begin{Teo}
Sup\'ongase que $n^{-1}T_{n}\rightarrow\mu$ c.s. cuando $n\rightarrow\infty$, donde $\mu\leq\infty$ es una constante o variable aleatoria. Sea $a$ una constante o variable aleatoria que puede ser infinita cuando $\mu$ es finita, y considere las expresiones l\'imite:
\begin{eqnarray}
lim_{n\rightarrow\infty}n^{-1}Z\left(T_{n}\right)&=&a,\textrm{ c.s.}\\
lim_{t\rightarrow\infty}t^{-1}Z\left(t\right)&=&a/\mu,\textrm{ c.s.}
\end{eqnarray}
La segunda expresi\'on implica la primera. Conversamente, la primera implica la segunda si el proceso $Z\left(t\right)$ es creciente, o si $lim_{n\rightarrow\infty}n^{-1}M_{n}=0$ c.s.
\end{Teo}

\begin{Coro}
Si $N\left(t\right)$ es un proceso de renovaci\'on, y $\left(Z\left(T_{n}\right)-Z\left(T_{n-1}\right),M_{n}\right)$, para $n\geq1$, son variables aleatorias independientes e id\'enticamente distribuidas con media finita, entonces,
\begin{eqnarray}
lim_{t\rightarrow\infty}t^{-1}Z\left(t\right)\rightarrow\frac{\esp\left[Z\left(T_{1}\right)-Z\left(T_{0}\right)\right]}{\esp\left[T_{1}\right]},\textrm{ c.s. cuando  }t\rightarrow\infty.
\end{eqnarray}
\end{Coro}



%___________________________________________________________________________________________
%
%\subsection{Propiedades de los Procesos de Renovaci\'on}
%___________________________________________________________________________________________
%

Los tiempos $T_{n}$ est\'an relacionados con los conteos de $N\left(t\right)$ por

\begin{eqnarray*}
\left\{N\left(t\right)\geq n\right\}&=&\left\{T_{n}\leq t\right\}\\
T_{N\left(t\right)}\leq &t&<T_{N\left(t\right)+1},
\end{eqnarray*}

adem\'as $N\left(T_{n}\right)=n$, y 

\begin{eqnarray*}
N\left(t\right)=\max\left\{n:T_{n}\leq t\right\}=\min\left\{n:T_{n+1}>t\right\}
\end{eqnarray*}

Por propiedades de la convoluci\'on se sabe que

\begin{eqnarray*}
P\left\{T_{n}\leq t\right\}=F^{n\star}\left(t\right)
\end{eqnarray*}
que es la $n$-\'esima convoluci\'on de $F$. Entonces 

\begin{eqnarray*}
\left\{N\left(t\right)\geq n\right\}&=&\left\{T_{n}\leq t\right\}\\
P\left\{N\left(t\right)\leq n\right\}&=&1-F^{\left(n+1\right)\star}\left(t\right)
\end{eqnarray*}

Adem\'as usando el hecho de que $\esp\left[N\left(t\right)\right]=\sum_{n=1}^{\infty}P\left\{N\left(t\right)\geq n\right\}$
se tiene que

\begin{eqnarray*}
\esp\left[N\left(t\right)\right]=\sum_{n=1}^{\infty}F^{n\star}\left(t\right)
\end{eqnarray*}

\begin{Prop}
Para cada $t\geq0$, la funci\'on generadora de momentos $\esp\left[e^{\alpha N\left(t\right)}\right]$ existe para alguna $\alpha$ en una vecindad del 0, y de aqu\'i que $\esp\left[N\left(t\right)^{m}\right]<\infty$, para $m\geq1$.
\end{Prop}


\begin{Note}
Si el primer tiempo de renovaci\'on $\xi_{1}$ no tiene la misma distribuci\'on que el resto de las $\xi_{n}$, para $n\geq2$, a $N\left(t\right)$ se le llama Proceso de Renovaci\'on retardado, donde si $\xi$ tiene distribuci\'on $G$, entonces el tiempo $T_{n}$ de la $n$-\'esima renovaci\'on tiene distribuci\'on $G\star F^{\left(n-1\right)\star}\left(t\right)$
\end{Note}


\begin{Teo}
Para una constante $\mu\leq\infty$ ( o variable aleatoria), las siguientes expresiones son equivalentes:

\begin{eqnarray}
lim_{n\rightarrow\infty}n^{-1}T_{n}&=&\mu,\textrm{ c.s.}\\
lim_{t\rightarrow\infty}t^{-1}N\left(t\right)&=&1/\mu,\textrm{ c.s.}
\end{eqnarray}
\end{Teo}


Es decir, $T_{n}$ satisface la Ley Fuerte de los Grandes N\'umeros s\'i y s\'olo s\'i $N\left/t\right)$ la cumple.


\begin{Coro}[Ley Fuerte de los Grandes N\'umeros para Procesos de Renovaci\'on]
Si $N\left(t\right)$ es un proceso de renovaci\'on cuyos tiempos de inter-renovaci\'on tienen media $\mu\leq\infty$, entonces
\begin{eqnarray}
t^{-1}N\left(t\right)\rightarrow 1/\mu,\textrm{ c.s. cuando }t\rightarrow\infty.
\end{eqnarray}

\end{Coro}


Considerar el proceso estoc\'astico de valores reales $\left\{Z\left(t\right):t\geq0\right\}$ en el mismo espacio de probabilidad que $N\left(t\right)$

\begin{Def}
Para el proceso $\left\{Z\left(t\right):t\geq0\right\}$ se define la fluctuaci\'on m\'axima de $Z\left(t\right)$ en el intervalo $\left(T_{n-1},T_{n}\right]$:
\begin{eqnarray*}
M_{n}=\sup_{T_{n-1}<t\leq T_{n}}|Z\left(t\right)-Z\left(T_{n-1}\right)|
\end{eqnarray*}
\end{Def}

\begin{Teo}
Sup\'ongase que $n^{-1}T_{n}\rightarrow\mu$ c.s. cuando $n\rightarrow\infty$, donde $\mu\leq\infty$ es una constante o variable aleatoria. Sea $a$ una constante o variable aleatoria que puede ser infinita cuando $\mu$ es finita, y considere las expresiones l\'imite:
\begin{eqnarray}
lim_{n\rightarrow\infty}n^{-1}Z\left(T_{n}\right)&=&a,\textrm{ c.s.}\\
lim_{t\rightarrow\infty}t^{-1}Z\left(t\right)&=&a/\mu,\textrm{ c.s.}
\end{eqnarray}
La segunda expresi\'on implica la primera. Conversamente, la primera implica la segunda si el proceso $Z\left(t\right)$ es creciente, o si $lim_{n\rightarrow\infty}n^{-1}M_{n}=0$ c.s.
\end{Teo}

\begin{Coro}
Si $N\left(t\right)$ es un proceso de renovaci\'on, y $\left(Z\left(T_{n}\right)-Z\left(T_{n-1}\right),M_{n}\right)$, para $n\geq1$, son variables aleatorias independientes e id\'enticamente distribuidas con media finita, entonces,
\begin{eqnarray}
lim_{t\rightarrow\infty}t^{-1}Z\left(t\right)\rightarrow\frac{\esp\left[Z\left(T_{1}\right)-Z\left(T_{0}\right)\right]}{\esp\left[T_{1}\right]},\textrm{ c.s. cuando  }t\rightarrow\infty.
\end{eqnarray}
\end{Coro}


%___________________________________________________________________________________________
%
%\subsection{Propiedades de los Procesos de Renovaci\'on}
%___________________________________________________________________________________________
%

Los tiempos $T_{n}$ est\'an relacionados con los conteos de $N\left(t\right)$ por

\begin{eqnarray*}
\left\{N\left(t\right)\geq n\right\}&=&\left\{T_{n}\leq t\right\}\\
T_{N\left(t\right)}\leq &t&<T_{N\left(t\right)+1},
\end{eqnarray*}

adem\'as $N\left(T_{n}\right)=n$, y 

\begin{eqnarray*}
N\left(t\right)=\max\left\{n:T_{n}\leq t\right\}=\min\left\{n:T_{n+1}>t\right\}
\end{eqnarray*}

Por propiedades de la convoluci\'on se sabe que

\begin{eqnarray*}
P\left\{T_{n}\leq t\right\}=F^{n\star}\left(t\right)
\end{eqnarray*}
que es la $n$-\'esima convoluci\'on de $F$. Entonces 

\begin{eqnarray*}
\left\{N\left(t\right)\geq n\right\}&=&\left\{T_{n}\leq t\right\}\\
P\left\{N\left(t\right)\leq n\right\}&=&1-F^{\left(n+1\right)\star}\left(t\right)
\end{eqnarray*}

Adem\'as usando el hecho de que $\esp\left[N\left(t\right)\right]=\sum_{n=1}^{\infty}P\left\{N\left(t\right)\geq n\right\}$
se tiene que

\begin{eqnarray*}
\esp\left[N\left(t\right)\right]=\sum_{n=1}^{\infty}F^{n\star}\left(t\right)
\end{eqnarray*}

\begin{Prop}
Para cada $t\geq0$, la funci\'on generadora de momentos $\esp\left[e^{\alpha N\left(t\right)}\right]$ existe para alguna $\alpha$ en una vecindad del 0, y de aqu\'i que $\esp\left[N\left(t\right)^{m}\right]<\infty$, para $m\geq1$.
\end{Prop}


\begin{Note}
Si el primer tiempo de renovaci\'on $\xi_{1}$ no tiene la misma distribuci\'on que el resto de las $\xi_{n}$, para $n\geq2$, a $N\left(t\right)$ se le llama Proceso de Renovaci\'on retardado, donde si $\xi$ tiene distribuci\'on $G$, entonces el tiempo $T_{n}$ de la $n$-\'esima renovaci\'on tiene distribuci\'on $G\star F^{\left(n-1\right)\star}\left(t\right)$
\end{Note}


\begin{Teo}
Para una constante $\mu\leq\infty$ ( o variable aleatoria), las siguientes expresiones son equivalentes:

\begin{eqnarray}
lim_{n\rightarrow\infty}n^{-1}T_{n}&=&\mu,\textrm{ c.s.}\\
lim_{t\rightarrow\infty}t^{-1}N\left(t\right)&=&1/\mu,\textrm{ c.s.}
\end{eqnarray}
\end{Teo}


Es decir, $T_{n}$ satisface la Ley Fuerte de los Grandes N\'umeros s\'i y s\'olo s\'i $N\left/t\right)$ la cumple.


\begin{Coro}[Ley Fuerte de los Grandes N\'umeros para Procesos de Renovaci\'on]
Si $N\left(t\right)$ es un proceso de renovaci\'on cuyos tiempos de inter-renovaci\'on tienen media $\mu\leq\infty$, entonces
\begin{eqnarray}
t^{-1}N\left(t\right)\rightarrow 1/\mu,\textrm{ c.s. cuando }t\rightarrow\infty.
\end{eqnarray}

\end{Coro}


Considerar el proceso estoc\'astico de valores reales $\left\{Z\left(t\right):t\geq0\right\}$ en el mismo espacio de probabilidad que $N\left(t\right)$

\begin{Def}
Para el proceso $\left\{Z\left(t\right):t\geq0\right\}$ se define la fluctuaci\'on m\'axima de $Z\left(t\right)$ en el intervalo $\left(T_{n-1},T_{n}\right]$:
\begin{eqnarray*}
M_{n}=\sup_{T_{n-1}<t\leq T_{n}}|Z\left(t\right)-Z\left(T_{n-1}\right)|
\end{eqnarray*}
\end{Def}

\begin{Teo}
Sup\'ongase que $n^{-1}T_{n}\rightarrow\mu$ c.s. cuando $n\rightarrow\infty$, donde $\mu\leq\infty$ es una constante o variable aleatoria. Sea $a$ una constante o variable aleatoria que puede ser infinita cuando $\mu$ es finita, y considere las expresiones l\'imite:
\begin{eqnarray}
lim_{n\rightarrow\infty}n^{-1}Z\left(T_{n}\right)&=&a,\textrm{ c.s.}\\
lim_{t\rightarrow\infty}t^{-1}Z\left(t\right)&=&a/\mu,\textrm{ c.s.}
\end{eqnarray}
La segunda expresi\'on implica la primera. Conversamente, la primera implica la segunda si el proceso $Z\left(t\right)$ es creciente, o si $lim_{n\rightarrow\infty}n^{-1}M_{n}=0$ c.s.
\end{Teo}

\begin{Coro}
Si $N\left(t\right)$ es un proceso de renovaci\'on, y $\left(Z\left(T_{n}\right)-Z\left(T_{n-1}\right),M_{n}\right)$, para $n\geq1$, son variables aleatorias independientes e id\'enticamente distribuidas con media finita, entonces,
\begin{eqnarray}
lim_{t\rightarrow\infty}t^{-1}Z\left(t\right)\rightarrow\frac{\esp\left[Z\left(T_{1}\right)-Z\left(T_{0}\right)\right]}{\esp\left[T_{1}\right]},\textrm{ c.s. cuando  }t\rightarrow\infty.
\end{eqnarray}
\end{Coro}

%___________________________________________________________________________________________
%
%\subsection{Propiedades de los Procesos de Renovaci\'on}
%___________________________________________________________________________________________
%

Los tiempos $T_{n}$ est\'an relacionados con los conteos de $N\left(t\right)$ por

\begin{eqnarray*}
\left\{N\left(t\right)\geq n\right\}&=&\left\{T_{n}\leq t\right\}\\
T_{N\left(t\right)}\leq &t&<T_{N\left(t\right)+1},
\end{eqnarray*}

adem\'as $N\left(T_{n}\right)=n$, y 

\begin{eqnarray*}
N\left(t\right)=\max\left\{n:T_{n}\leq t\right\}=\min\left\{n:T_{n+1}>t\right\}
\end{eqnarray*}

Por propiedades de la convoluci\'on se sabe que

\begin{eqnarray*}
P\left\{T_{n}\leq t\right\}=F^{n\star}\left(t\right)
\end{eqnarray*}
que es la $n$-\'esima convoluci\'on de $F$. Entonces 

\begin{eqnarray*}
\left\{N\left(t\right)\geq n\right\}&=&\left\{T_{n}\leq t\right\}\\
P\left\{N\left(t\right)\leq n\right\}&=&1-F^{\left(n+1\right)\star}\left(t\right)
\end{eqnarray*}

Adem\'as usando el hecho de que $\esp\left[N\left(t\right)\right]=\sum_{n=1}^{\infty}P\left\{N\left(t\right)\geq n\right\}$
se tiene que

\begin{eqnarray*}
\esp\left[N\left(t\right)\right]=\sum_{n=1}^{\infty}F^{n\star}\left(t\right)
\end{eqnarray*}

\begin{Prop}
Para cada $t\geq0$, la funci\'on generadora de momentos $\esp\left[e^{\alpha N\left(t\right)}\right]$ existe para alguna $\alpha$ en una vecindad del 0, y de aqu\'i que $\esp\left[N\left(t\right)^{m}\right]<\infty$, para $m\geq1$.
\end{Prop}


\begin{Note}
Si el primer tiempo de renovaci\'on $\xi_{1}$ no tiene la misma distribuci\'on que el resto de las $\xi_{n}$, para $n\geq2$, a $N\left(t\right)$ se le llama Proceso de Renovaci\'on retardado, donde si $\xi$ tiene distribuci\'on $G$, entonces el tiempo $T_{n}$ de la $n$-\'esima renovaci\'on tiene distribuci\'on $G\star F^{\left(n-1\right)\star}\left(t\right)$
\end{Note}


\begin{Teo}
Para una constante $\mu\leq\infty$ ( o variable aleatoria), las siguientes expresiones son equivalentes:

\begin{eqnarray}
lim_{n\rightarrow\infty}n^{-1}T_{n}&=&\mu,\textrm{ c.s.}\\
lim_{t\rightarrow\infty}t^{-1}N\left(t\right)&=&1/\mu,\textrm{ c.s.}
\end{eqnarray}
\end{Teo}


Es decir, $T_{n}$ satisface la Ley Fuerte de los Grandes N\'umeros s\'i y s\'olo s\'i $N\left/t\right)$ la cumple.


\begin{Coro}[Ley Fuerte de los Grandes N\'umeros para Procesos de Renovaci\'on]
Si $N\left(t\right)$ es un proceso de renovaci\'on cuyos tiempos de inter-renovaci\'on tienen media $\mu\leq\infty$, entonces
\begin{eqnarray}
t^{-1}N\left(t\right)\rightarrow 1/\mu,\textrm{ c.s. cuando }t\rightarrow\infty.
\end{eqnarray}

\end{Coro}


Considerar el proceso estoc\'astico de valores reales $\left\{Z\left(t\right):t\geq0\right\}$ en el mismo espacio de probabilidad que $N\left(t\right)$

\begin{Def}
Para el proceso $\left\{Z\left(t\right):t\geq0\right\}$ se define la fluctuaci\'on m\'axima de $Z\left(t\right)$ en el intervalo $\left(T_{n-1},T_{n}\right]$:
\begin{eqnarray*}
M_{n}=\sup_{T_{n-1}<t\leq T_{n}}|Z\left(t\right)-Z\left(T_{n-1}\right)|
\end{eqnarray*}
\end{Def}

\begin{Teo}
Sup\'ongase que $n^{-1}T_{n}\rightarrow\mu$ c.s. cuando $n\rightarrow\infty$, donde $\mu\leq\infty$ es una constante o variable aleatoria. Sea $a$ una constante o variable aleatoria que puede ser infinita cuando $\mu$ es finita, y considere las expresiones l\'imite:
\begin{eqnarray}
lim_{n\rightarrow\infty}n^{-1}Z\left(T_{n}\right)&=&a,\textrm{ c.s.}\\
lim_{t\rightarrow\infty}t^{-1}Z\left(t\right)&=&a/\mu,\textrm{ c.s.}
\end{eqnarray}
La segunda expresi\'on implica la primera. Conversamente, la primera implica la segunda si el proceso $Z\left(t\right)$ es creciente, o si $lim_{n\rightarrow\infty}n^{-1}M_{n}=0$ c.s.
\end{Teo}

\begin{Coro}
Si $N\left(t\right)$ es un proceso de renovaci\'on, y $\left(Z\left(T_{n}\right)-Z\left(T_{n-1}\right),M_{n}\right)$, para $n\geq1$, son variables aleatorias independientes e id\'enticamente distribuidas con media finita, entonces,
\begin{eqnarray}
lim_{t\rightarrow\infty}t^{-1}Z\left(t\right)\rightarrow\frac{\esp\left[Z\left(T_{1}\right)-Z\left(T_{0}\right)\right]}{\esp\left[T_{1}\right]},\textrm{ c.s. cuando  }t\rightarrow\infty.
\end{eqnarray}
\end{Coro}
%___________________________________________________________________________________________
%
%\subsection{Propiedades de los Procesos de Renovaci\'on}
%___________________________________________________________________________________________
%

Los tiempos $T_{n}$ est\'an relacionados con los conteos de $N\left(t\right)$ por

\begin{eqnarray*}
\left\{N\left(t\right)\geq n\right\}&=&\left\{T_{n}\leq t\right\}\\
T_{N\left(t\right)}\leq &t&<T_{N\left(t\right)+1},
\end{eqnarray*}

adem\'as $N\left(T_{n}\right)=n$, y 

\begin{eqnarray*}
N\left(t\right)=\max\left\{n:T_{n}\leq t\right\}=\min\left\{n:T_{n+1}>t\right\}
\end{eqnarray*}

Por propiedades de la convoluci\'on se sabe que

\begin{eqnarray*}
P\left\{T_{n}\leq t\right\}=F^{n\star}\left(t\right)
\end{eqnarray*}
que es la $n$-\'esima convoluci\'on de $F$. Entonces 

\begin{eqnarray*}
\left\{N\left(t\right)\geq n\right\}&=&\left\{T_{n}\leq t\right\}\\
P\left\{N\left(t\right)\leq n\right\}&=&1-F^{\left(n+1\right)\star}\left(t\right)
\end{eqnarray*}

Adem\'as usando el hecho de que $\esp\left[N\left(t\right)\right]=\sum_{n=1}^{\infty}P\left\{N\left(t\right)\geq n\right\}$
se tiene que

\begin{eqnarray*}
\esp\left[N\left(t\right)\right]=\sum_{n=1}^{\infty}F^{n\star}\left(t\right)
\end{eqnarray*}

\begin{Prop}
Para cada $t\geq0$, la funci\'on generadora de momentos $\esp\left[e^{\alpha N\left(t\right)}\right]$ existe para alguna $\alpha$ en una vecindad del 0, y de aqu\'i que $\esp\left[N\left(t\right)^{m}\right]<\infty$, para $m\geq1$.
\end{Prop}


\begin{Note}
Si el primer tiempo de renovaci\'on $\xi_{1}$ no tiene la misma distribuci\'on que el resto de las $\xi_{n}$, para $n\geq2$, a $N\left(t\right)$ se le llama Proceso de Renovaci\'on retardado, donde si $\xi$ tiene distribuci\'on $G$, entonces el tiempo $T_{n}$ de la $n$-\'esima renovaci\'on tiene distribuci\'on $G\star F^{\left(n-1\right)\star}\left(t\right)$
\end{Note}


\begin{Teo}
Para una constante $\mu\leq\infty$ ( o variable aleatoria), las siguientes expresiones son equivalentes:

\begin{eqnarray}
lim_{n\rightarrow\infty}n^{-1}T_{n}&=&\mu,\textrm{ c.s.}\\
lim_{t\rightarrow\infty}t^{-1}N\left(t\right)&=&1/\mu,\textrm{ c.s.}
\end{eqnarray}
\end{Teo}


Es decir, $T_{n}$ satisface la Ley Fuerte de los Grandes N\'umeros s\'i y s\'olo s\'i $N\left/t\right)$ la cumple.


\begin{Coro}[Ley Fuerte de los Grandes N\'umeros para Procesos de Renovaci\'on]
Si $N\left(t\right)$ es un proceso de renovaci\'on cuyos tiempos de inter-renovaci\'on tienen media $\mu\leq\infty$, entonces
\begin{eqnarray}
t^{-1}N\left(t\right)\rightarrow 1/\mu,\textrm{ c.s. cuando }t\rightarrow\infty.
\end{eqnarray}

\end{Coro}


Considerar el proceso estoc\'astico de valores reales $\left\{Z\left(t\right):t\geq0\right\}$ en el mismo espacio de probabilidad que $N\left(t\right)$

\begin{Def}
Para el proceso $\left\{Z\left(t\right):t\geq0\right\}$ se define la fluctuaci\'on m\'axima de $Z\left(t\right)$ en el intervalo $\left(T_{n-1},T_{n}\right]$:
\begin{eqnarray*}
M_{n}=\sup_{T_{n-1}<t\leq T_{n}}|Z\left(t\right)-Z\left(T_{n-1}\right)|
\end{eqnarray*}
\end{Def}

\begin{Teo}
Sup\'ongase que $n^{-1}T_{n}\rightarrow\mu$ c.s. cuando $n\rightarrow\infty$, donde $\mu\leq\infty$ es una constante o variable aleatoria. Sea $a$ una constante o variable aleatoria que puede ser infinita cuando $\mu$ es finita, y considere las expresiones l\'imite:
\begin{eqnarray}
lim_{n\rightarrow\infty}n^{-1}Z\left(T_{n}\right)&=&a,\textrm{ c.s.}\\
lim_{t\rightarrow\infty}t^{-1}Z\left(t\right)&=&a/\mu,\textrm{ c.s.}
\end{eqnarray}
La segunda expresi\'on implica la primera. Conversamente, la primera implica la segunda si el proceso $Z\left(t\right)$ es creciente, o si $lim_{n\rightarrow\infty}n^{-1}M_{n}=0$ c.s.
\end{Teo}

\begin{Coro}
Si $N\left(t\right)$ es un proceso de renovaci\'on, y $\left(Z\left(T_{n}\right)-Z\left(T_{n-1}\right),M_{n}\right)$, para $n\geq1$, son variables aleatorias independientes e id\'enticamente distribuidas con media finita, entonces,
\begin{eqnarray}
lim_{t\rightarrow\infty}t^{-1}Z\left(t\right)\rightarrow\frac{\esp\left[Z\left(T_{1}\right)-Z\left(T_{0}\right)\right]}{\esp\left[T_{1}\right]},\textrm{ c.s. cuando  }t\rightarrow\infty.
\end{eqnarray}
\end{Coro}


%___________________________________________________________________________________________
%
%\subsection{Funci\'on de Renovaci\'on}
%___________________________________________________________________________________________
%


\begin{Def}
Sea $h\left(t\right)$ funci\'on de valores reales en $\rea$ acotada en intervalos finitos e igual a cero para $t<0$ La ecuaci\'on de renovaci\'on para $h\left(t\right)$ y la distribuci\'on $F$ es

\begin{eqnarray}%\label{Ec.Renovacion}
H\left(t\right)=h\left(t\right)+\int_{\left[0,t\right]}H\left(t-s\right)dF\left(s\right)\textrm{,    }t\geq0,
\end{eqnarray}
donde $H\left(t\right)$ es una funci\'on de valores reales. Esto es $H=h+F\star H$. Decimos que $H\left(t\right)$ es soluci\'on de esta ecuaci\'on si satisface la ecuaci\'on, y es acotada en intervalos finitos e iguales a cero para $t<0$.
\end{Def}

\begin{Prop}
La funci\'on $U\star h\left(t\right)$ es la \'unica soluci\'on de la ecuaci\'on de renovaci\'on (\ref{Ec.Renovacion}).
\end{Prop}

\begin{Teo}[Teorema Renovaci\'on Elemental]
\begin{eqnarray*}
t^{-1}U\left(t\right)\rightarrow 1/\mu\textrm{,    cuando }t\rightarrow\infty.
\end{eqnarray*}
\end{Teo}

%___________________________________________________________________________________________
%
%\subsection{Funci\'on de Renovaci\'on}
%___________________________________________________________________________________________
%


Sup\'ongase que $N\left(t\right)$ es un proceso de renovaci\'on con distribuci\'on $F$ con media finita $\mu$.

\begin{Def}
La funci\'on de renovaci\'on asociada con la distribuci\'on $F$, del proceso $N\left(t\right)$, es
\begin{eqnarray*}
U\left(t\right)=\sum_{n=1}^{\infty}F^{n\star}\left(t\right),\textrm{   }t\geq0,
\end{eqnarray*}
donde $F^{0\star}\left(t\right)=\indora\left(t\geq0\right)$.
\end{Def}


\begin{Prop}
Sup\'ongase que la distribuci\'on de inter-renovaci\'on $F$ tiene densidad $f$. Entonces $U\left(t\right)$ tambi\'en tiene densidad, para $t>0$, y es $U^{'}\left(t\right)=\sum_{n=0}^{\infty}f^{n\star}\left(t\right)$. Adem\'as
\begin{eqnarray*}
\prob\left\{N\left(t\right)>N\left(t-\right)\right\}=0\textrm{,   }t\geq0.
\end{eqnarray*}
\end{Prop}

\begin{Def}
La Transformada de Laplace-Stieljes de $F$ est\'a dada por

\begin{eqnarray*}
\hat{F}\left(\alpha\right)=\int_{\rea_{+}}e^{-\alpha t}dF\left(t\right)\textrm{,  }\alpha\geq0.
\end{eqnarray*}
\end{Def}

Entonces

\begin{eqnarray*}
\hat{U}\left(\alpha\right)=\sum_{n=0}^{\infty}\hat{F^{n\star}}\left(\alpha\right)=\sum_{n=0}^{\infty}\hat{F}\left(\alpha\right)^{n}=\frac{1}{1-\hat{F}\left(\alpha\right)}.
\end{eqnarray*}


\begin{Prop}
La Transformada de Laplace $\hat{U}\left(\alpha\right)$ y $\hat{F}\left(\alpha\right)$ determina una a la otra de manera \'unica por la relaci\'on $\hat{U}\left(\alpha\right)=\frac{1}{1-\hat{F}\left(\alpha\right)}$.
\end{Prop}


\begin{Note}
Un proceso de renovaci\'on $N\left(t\right)$ cuyos tiempos de inter-renovaci\'on tienen media finita, es un proceso Poisson con tasa $\lambda$ si y s\'olo s\'i $\esp\left[U\left(t\right)\right]=\lambda t$, para $t\geq0$.
\end{Note}


\begin{Teo}
Sea $N\left(t\right)$ un proceso puntual simple con puntos de localizaci\'on $T_{n}$ tal que $\eta\left(t\right)=\esp\left[N\left(\right)\right]$ es finita para cada $t$. Entonces para cualquier funci\'on $f:\rea_{+}\rightarrow\rea$,
\begin{eqnarray*}
\esp\left[\sum_{n=1}^{N\left(\right)}f\left(T_{n}\right)\right]=\int_{\left(0,t\right]}f\left(s\right)d\eta\left(s\right)\textrm{,  }t\geq0,
\end{eqnarray*}
suponiendo que la integral exista. Adem\'as si $X_{1},X_{2},\ldots$ son variables aleatorias definidas en el mismo espacio de probabilidad que el proceso $N\left(t\right)$ tal que $\esp\left[X_{n}|T_{n}=s\right]=f\left(s\right)$, independiente de $n$. Entonces
\begin{eqnarray*}
\esp\left[\sum_{n=1}^{N\left(t\right)}X_{n}\right]=\int_{\left(0,t\right]}f\left(s\right)d\eta\left(s\right)\textrm{,  }t\geq0,
\end{eqnarray*} 
suponiendo que la integral exista. 
\end{Teo}

\begin{Coro}[Identidad de Wald para Renovaciones]
Para el proceso de renovaci\'on $N\left(t\right)$,
\begin{eqnarray*}
\esp\left[T_{N\left(t\right)+1}\right]=\mu\esp\left[N\left(t\right)+1\right]\textrm{,  }t\geq0,
\end{eqnarray*}  
\end{Coro}

%______________________________________________________________________
%\subsection{Procesos de Renovaci\'on}
%______________________________________________________________________

\begin{Def}%\label{Def.Tn}
Sean $0\leq T_{1}\leq T_{2}\leq \ldots$ son tiempos aleatorios infinitos en los cuales ocurren ciertos eventos. El n\'umero de tiempos $T_{n}$ en el intervalo $\left[0,t\right)$ es

\begin{eqnarray}
N\left(t\right)=\sum_{n=1}^{\infty}\indora\left(T_{n}\leq t\right),
\end{eqnarray}
para $t\geq0$.
\end{Def}

Si se consideran los puntos $T_{n}$ como elementos de $\rea_{+}$, y $N\left(t\right)$ es el n\'umero de puntos en $\rea$. El proceso denotado por $\left\{N\left(t\right):t\geq0\right\}$, denotado por $N\left(t\right)$, es un proceso puntual en $\rea_{+}$. Los $T_{n}$ son los tiempos de ocurrencia, el proceso puntual $N\left(t\right)$ es simple si su n\'umero de ocurrencias son distintas: $0<T_{1}<T_{2}<\ldots$ casi seguramente.

\begin{Def}
Un proceso puntual $N\left(t\right)$ es un proceso de renovaci\'on si los tiempos de interocurrencia $\xi_{n}=T_{n}-T_{n-1}$, para $n\geq1$, son independientes e identicamente distribuidos con distribuci\'on $F$, donde $F\left(0\right)=0$ y $T_{0}=0$. Los $T_{n}$ son llamados tiempos de renovaci\'on, referente a la independencia o renovaci\'on de la informaci\'on estoc\'astica en estos tiempos. Los $\xi_{n}$ son los tiempos de inter-renovaci\'on, y $N\left(t\right)$ es el n\'umero de renovaciones en el intervalo $\left[0,t\right)$
\end{Def}


\begin{Note}
Para definir un proceso de renovaci\'on para cualquier contexto, solamente hay que especificar una distribuci\'on $F$, con $F\left(0\right)=0$, para los tiempos de inter-renovaci\'on. La funci\'on $F$ en turno degune las otra variables aleatorias. De manera formal, existe un espacio de probabilidad y una sucesi\'on de variables aleatorias $\xi_{1},\xi_{2},\ldots$ definidas en este con distribuci\'on $F$. Entonces las otras cantidades son $T_{n}=\sum_{k=1}^{n}\xi_{k}$ y $N\left(t\right)=\sum_{n=1}^{\infty}\indora\left(T_{n}\leq t\right)$, donde $T_{n}\rightarrow\infty$ casi seguramente por la Ley Fuerte de los Grandes Números.
\end{Note}

%___________________________________________________________________________________________
%
%\subsection{Renewal and Regenerative Processes: Serfozo\cite{Serfozo}}
%___________________________________________________________________________________________
%
\begin{Def}%\label{Def.Tn}
Sean $0\leq T_{1}\leq T_{2}\leq \ldots$ son tiempos aleatorios infinitos en los cuales ocurren ciertos eventos. El n\'umero de tiempos $T_{n}$ en el intervalo $\left[0,t\right)$ es

\begin{eqnarray}
N\left(t\right)=\sum_{n=1}^{\infty}\indora\left(T_{n}\leq t\right),
\end{eqnarray}
para $t\geq0$.
\end{Def}

Si se consideran los puntos $T_{n}$ como elementos de $\rea_{+}$, y $N\left(t\right)$ es el n\'umero de puntos en $\rea$. El proceso denotado por $\left\{N\left(t\right):t\geq0\right\}$, denotado por $N\left(t\right)$, es un proceso puntual en $\rea_{+}$. Los $T_{n}$ son los tiempos de ocurrencia, el proceso puntual $N\left(t\right)$ es simple si su n\'umero de ocurrencias son distintas: $0<T_{1}<T_{2}<\ldots$ casi seguramente.

\begin{Def}
Un proceso puntual $N\left(t\right)$ es un proceso de renovaci\'on si los tiempos de interocurrencia $\xi_{n}=T_{n}-T_{n-1}$, para $n\geq1$, son independientes e identicamente distribuidos con distribuci\'on $F$, donde $F\left(0\right)=0$ y $T_{0}=0$. Los $T_{n}$ son llamados tiempos de renovaci\'on, referente a la independencia o renovaci\'on de la informaci\'on estoc\'astica en estos tiempos. Los $\xi_{n}$ son los tiempos de inter-renovaci\'on, y $N\left(t\right)$ es el n\'umero de renovaciones en el intervalo $\left[0,t\right)$
\end{Def}


\begin{Note}
Para definir un proceso de renovaci\'on para cualquier contexto, solamente hay que especificar una distribuci\'on $F$, con $F\left(0\right)=0$, para los tiempos de inter-renovaci\'on. La funci\'on $F$ en turno degune las otra variables aleatorias. De manera formal, existe un espacio de probabilidad y una sucesi\'on de variables aleatorias $\xi_{1},\xi_{2},\ldots$ definidas en este con distribuci\'on $F$. Entonces las otras cantidades son $T_{n}=\sum_{k=1}^{n}\xi_{k}$ y $N\left(t\right)=\sum_{n=1}^{\infty}\indora\left(T_{n}\leq t\right)$, donde $T_{n}\rightarrow\infty$ casi seguramente por la Ley Fuerte de los Grandes N\'umeros.
\end{Note}







Los tiempos $T_{n}$ est\'an relacionados con los conteos de $N\left(t\right)$ por

\begin{eqnarray*}
\left\{N\left(t\right)\geq n\right\}&=&\left\{T_{n}\leq t\right\}\\
T_{N\left(t\right)}\leq &t&<T_{N\left(t\right)+1},
\end{eqnarray*}

adem\'as $N\left(T_{n}\right)=n$, y 

\begin{eqnarray*}
N\left(t\right)=\max\left\{n:T_{n}\leq t\right\}=\min\left\{n:T_{n+1}>t\right\}
\end{eqnarray*}

Por propiedades de la convoluci\'on se sabe que

\begin{eqnarray*}
P\left\{T_{n}\leq t\right\}=F^{n\star}\left(t\right)
\end{eqnarray*}
que es la $n$-\'esima convoluci\'on de $F$. Entonces 

\begin{eqnarray*}
\left\{N\left(t\right)\geq n\right\}&=&\left\{T_{n}\leq t\right\}\\
P\left\{N\left(t\right)\leq n\right\}&=&1-F^{\left(n+1\right)\star}\left(t\right)
\end{eqnarray*}

Adem\'as usando el hecho de que $\esp\left[N\left(t\right)\right]=\sum_{n=1}^{\infty}P\left\{N\left(t\right)\geq n\right\}$
se tiene que

\begin{eqnarray*}
\esp\left[N\left(t\right)\right]=\sum_{n=1}^{\infty}F^{n\star}\left(t\right)
\end{eqnarray*}

\begin{Prop}
Para cada $t\geq0$, la funci\'on generadora de momentos $\esp\left[e^{\alpha N\left(t\right)}\right]$ existe para alguna $\alpha$ en una vecindad del 0, y de aqu\'i que $\esp\left[N\left(t\right)^{m}\right]<\infty$, para $m\geq1$.
\end{Prop}

\begin{Ejem}[\textbf{Proceso Poisson}]

Suponga que se tienen tiempos de inter-renovaci\'on \textit{i.i.d.} del proceso de renovaci\'on $N\left(t\right)$ tienen distribuci\'on exponencial $F\left(t\right)=q-e^{-\lambda t}$ con tasa $\lambda$. Entonces $N\left(t\right)$ es un proceso Poisson con tasa $\lambda$.

\end{Ejem}


\begin{Note}
Si el primer tiempo de renovaci\'on $\xi_{1}$ no tiene la misma distribuci\'on que el resto de las $\xi_{n}$, para $n\geq2$, a $N\left(t\right)$ se le llama Proceso de Renovaci\'on retardado, donde si $\xi$ tiene distribuci\'on $G$, entonces el tiempo $T_{n}$ de la $n$-\'esima renovaci\'on tiene distribuci\'on $G\star F^{\left(n-1\right)\star}\left(t\right)$
\end{Note}


\begin{Teo}
Para una constante $\mu\leq\infty$ ( o variable aleatoria), las siguientes expresiones son equivalentes:

\begin{eqnarray}
lim_{n\rightarrow\infty}n^{-1}T_{n}&=&\mu,\textrm{ c.s.}\\
lim_{t\rightarrow\infty}t^{-1}N\left(t\right)&=&1/\mu,\textrm{ c.s.}
\end{eqnarray}
\end{Teo}


Es decir, $T_{n}$ satisface la Ley Fuerte de los Grandes N\'umeros s\'i y s\'olo s\'i $N\left/t\right)$ la cumple.


\begin{Coro}[Ley Fuerte de los Grandes N\'umeros para Procesos de Renovaci\'on]
Si $N\left(t\right)$ es un proceso de renovaci\'on cuyos tiempos de inter-renovaci\'on tienen media $\mu\leq\infty$, entonces
\begin{eqnarray}
t^{-1}N\left(t\right)\rightarrow 1/\mu,\textrm{ c.s. cuando }t\rightarrow\infty.
\end{eqnarray}

\end{Coro}


Considerar el proceso estoc\'astico de valores reales $\left\{Z\left(t\right):t\geq0\right\}$ en el mismo espacio de probabilidad que $N\left(t\right)$

\begin{Def}
Para el proceso $\left\{Z\left(t\right):t\geq0\right\}$ se define la fluctuaci\'on m\'axima de $Z\left(t\right)$ en el intervalo $\left(T_{n-1},T_{n}\right]$:
\begin{eqnarray*}
M_{n}=\sup_{T_{n-1}<t\leq T_{n}}|Z\left(t\right)-Z\left(T_{n-1}\right)|
\end{eqnarray*}
\end{Def}

\begin{Teo}
Sup\'ongase que $n^{-1}T_{n}\rightarrow\mu$ c.s. cuando $n\rightarrow\infty$, donde $\mu\leq\infty$ es una constante o variable aleatoria. Sea $a$ una constante o variable aleatoria que puede ser infinita cuando $\mu$ es finita, y considere las expresiones l\'imite:
\begin{eqnarray}
lim_{n\rightarrow\infty}n^{-1}Z\left(T_{n}\right)&=&a,\textrm{ c.s.}\\
lim_{t\rightarrow\infty}t^{-1}Z\left(t\right)&=&a/\mu,\textrm{ c.s.}
\end{eqnarray}
La segunda expresi\'on implica la primera. Conversamente, la primera implica la segunda si el proceso $Z\left(t\right)$ es creciente, o si $lim_{n\rightarrow\infty}n^{-1}M_{n}=0$ c.s.
\end{Teo}

\begin{Coro}
Si $N\left(t\right)$ es un proceso de renovaci\'on, y $\left(Z\left(T_{n}\right)-Z\left(T_{n-1}\right),M_{n}\right)$, para $n\geq1$, son variables aleatorias independientes e id\'enticamente distribuidas con media finita, entonces,
\begin{eqnarray}
lim_{t\rightarrow\infty}t^{-1}Z\left(t\right)\rightarrow\frac{\esp\left[Z\left(T_{1}\right)-Z\left(T_{0}\right)\right]}{\esp\left[T_{1}\right]},\textrm{ c.s. cuando  }t\rightarrow\infty.
\end{eqnarray}
\end{Coro}


Sup\'ongase que $N\left(t\right)$ es un proceso de renovaci\'on con distribuci\'on $F$ con media finita $\mu$.

\begin{Def}
La funci\'on de renovaci\'on asociada con la distribuci\'on $F$, del proceso $N\left(t\right)$, es
\begin{eqnarray*}
U\left(t\right)=\sum_{n=1}^{\infty}F^{n\star}\left(t\right),\textrm{   }t\geq0,
\end{eqnarray*}
donde $F^{0\star}\left(t\right)=\indora\left(t\geq0\right)$.
\end{Def}


\begin{Prop}
Sup\'ongase que la distribuci\'on de inter-renovaci\'on $F$ tiene densidad $f$. Entonces $U\left(t\right)$ tambi\'en tiene densidad, para $t>0$, y es $U^{'}\left(t\right)=\sum_{n=0}^{\infty}f^{n\star}\left(t\right)$. Adem\'as
\begin{eqnarray*}
\prob\left\{N\left(t\right)>N\left(t-\right)\right\}=0\textrm{,   }t\geq0.
\end{eqnarray*}
\end{Prop}

\begin{Def}
La Transformada de Laplace-Stieljes de $F$ est\'a dada por

\begin{eqnarray*}
\hat{F}\left(\alpha\right)=\int_{\rea_{+}}e^{-\alpha t}dF\left(t\right)\textrm{,  }\alpha\geq0.
\end{eqnarray*}
\end{Def}

Entonces

\begin{eqnarray*}
\hat{U}\left(\alpha\right)=\sum_{n=0}^{\infty}\hat{F^{n\star}}\left(\alpha\right)=\sum_{n=0}^{\infty}\hat{F}\left(\alpha\right)^{n}=\frac{1}{1-\hat{F}\left(\alpha\right)}.
\end{eqnarray*}


\begin{Prop}
La Transformada de Laplace $\hat{U}\left(\alpha\right)$ y $\hat{F}\left(\alpha\right)$ determina una a la otra de manera \'unica por la relaci\'on $\hat{U}\left(\alpha\right)=\frac{1}{1-\hat{F}\left(\alpha\right)}$.
\end{Prop}


\begin{Note}
Un proceso de renovaci\'on $N\left(t\right)$ cuyos tiempos de inter-renovaci\'on tienen media finita, es un proceso Poisson con tasa $\lambda$ si y s\'olo s\'i $\esp\left[U\left(t\right)\right]=\lambda t$, para $t\geq0$.
\end{Note}


\begin{Teo}
Sea $N\left(t\right)$ un proceso puntual simple con puntos de localizaci\'on $T_{n}$ tal que $\eta\left(t\right)=\esp\left[N\left(\right)\right]$ es finita para cada $t$. Entonces para cualquier funci\'on $f:\rea_{+}\rightarrow\rea$,
\begin{eqnarray*}
\esp\left[\sum_{n=1}^{N\left(\right)}f\left(T_{n}\right)\right]=\int_{\left(0,t\right]}f\left(s\right)d\eta\left(s\right)\textrm{,  }t\geq0,
\end{eqnarray*}
suponiendo que la integral exista. Adem\'as si $X_{1},X_{2},\ldots$ son variables aleatorias definidas en el mismo espacio de probabilidad que el proceso $N\left(t\right)$ tal que $\esp\left[X_{n}|T_{n}=s\right]=f\left(s\right)$, independiente de $n$. Entonces
\begin{eqnarray*}
\esp\left[\sum_{n=1}^{N\left(t\right)}X_{n}\right]=\int_{\left(0,t\right]}f\left(s\right)d\eta\left(s\right)\textrm{,  }t\geq0,
\end{eqnarray*} 
suponiendo que la integral exista. 
\end{Teo}

\begin{Coro}[Identidad de Wald para Renovaciones]
Para el proceso de renovaci\'on $N\left(t\right)$,
\begin{eqnarray*}
\esp\left[T_{N\left(t\right)+1}\right]=\mu\esp\left[N\left(t\right)+1\right]\textrm{,  }t\geq0,
\end{eqnarray*}  
\end{Coro}


\begin{Def}
Sea $h\left(t\right)$ funci\'on de valores reales en $\rea$ acotada en intervalos finitos e igual a cero para $t<0$ La ecuaci\'on de renovaci\'on para $h\left(t\right)$ y la distribuci\'on $F$ es

\begin{eqnarray}%\label{Ec.Renovacion}
H\left(t\right)=h\left(t\right)+\int_{\left[0,t\right]}H\left(t-s\right)dF\left(s\right)\textrm{,    }t\geq0,
\end{eqnarray}
donde $H\left(t\right)$ es una funci\'on de valores reales. Esto es $H=h+F\star H$. Decimos que $H\left(t\right)$ es soluci\'on de esta ecuaci\'on si satisface la ecuaci\'on, y es acotada en intervalos finitos e iguales a cero para $t<0$.
\end{Def}

\begin{Prop}
La funci\'on $U\star h\left(t\right)$ es la \'unica soluci\'on de la ecuaci\'on de renovaci\'on (\ref{Ec.Renovacion}).
\end{Prop}

\begin{Teo}[Teorema Renovaci\'on Elemental]
\begin{eqnarray*}
t^{-1}U\left(t\right)\rightarrow 1/\mu\textrm{,    cuando }t\rightarrow\infty.
\end{eqnarray*}
\end{Teo}



Sup\'ongase que $N\left(t\right)$ es un proceso de renovaci\'on con distribuci\'on $F$ con media finita $\mu$.

\begin{Def}
La funci\'on de renovaci\'on asociada con la distribuci\'on $F$, del proceso $N\left(t\right)$, es
\begin{eqnarray*}
U\left(t\right)=\sum_{n=1}^{\infty}F^{n\star}\left(t\right),\textrm{   }t\geq0,
\end{eqnarray*}
donde $F^{0\star}\left(t\right)=\indora\left(t\geq0\right)$.
\end{Def}


\begin{Prop}
Sup\'ongase que la distribuci\'on de inter-renovaci\'on $F$ tiene densidad $f$. Entonces $U\left(t\right)$ tambi\'en tiene densidad, para $t>0$, y es $U^{'}\left(t\right)=\sum_{n=0}^{\infty}f^{n\star}\left(t\right)$. Adem\'as
\begin{eqnarray*}
\prob\left\{N\left(t\right)>N\left(t-\right)\right\}=0\textrm{,   }t\geq0.
\end{eqnarray*}
\end{Prop}

\begin{Def}
La Transformada de Laplace-Stieljes de $F$ est\'a dada por

\begin{eqnarray*}
\hat{F}\left(\alpha\right)=\int_{\rea_{+}}e^{-\alpha t}dF\left(t\right)\textrm{,  }\alpha\geq0.
\end{eqnarray*}
\end{Def}

Entonces

\begin{eqnarray*}
\hat{U}\left(\alpha\right)=\sum_{n=0}^{\infty}\hat{F^{n\star}}\left(\alpha\right)=\sum_{n=0}^{\infty}\hat{F}\left(\alpha\right)^{n}=\frac{1}{1-\hat{F}\left(\alpha\right)}.
\end{eqnarray*}


\begin{Prop}
La Transformada de Laplace $\hat{U}\left(\alpha\right)$ y $\hat{F}\left(\alpha\right)$ determina una a la otra de manera \'unica por la relaci\'on $\hat{U}\left(\alpha\right)=\frac{1}{1-\hat{F}\left(\alpha\right)}$.
\end{Prop}


\begin{Note}
Un proceso de renovaci\'on $N\left(t\right)$ cuyos tiempos de inter-renovaci\'on tienen media finita, es un proceso Poisson con tasa $\lambda$ si y s\'olo s\'i $\esp\left[U\left(t\right)\right]=\lambda t$, para $t\geq0$.
\end{Note}


\begin{Teo}
Sea $N\left(t\right)$ un proceso puntual simple con puntos de localizaci\'on $T_{n}$ tal que $\eta\left(t\right)=\esp\left[N\left(\right)\right]$ es finita para cada $t$. Entonces para cualquier funci\'on $f:\rea_{+}\rightarrow\rea$,
\begin{eqnarray*}
\esp\left[\sum_{n=1}^{N\left(\right)}f\left(T_{n}\right)\right]=\int_{\left(0,t\right]}f\left(s\right)d\eta\left(s\right)\textrm{,  }t\geq0,
\end{eqnarray*}
suponiendo que la integral exista. Adem\'as si $X_{1},X_{2},\ldots$ son variables aleatorias definidas en el mismo espacio de probabilidad que el proceso $N\left(t\right)$ tal que $\esp\left[X_{n}|T_{n}=s\right]=f\left(s\right)$, independiente de $n$. Entonces
\begin{eqnarray*}
\esp\left[\sum_{n=1}^{N\left(t\right)}X_{n}\right]=\int_{\left(0,t\right]}f\left(s\right)d\eta\left(s\right)\textrm{,  }t\geq0,
\end{eqnarray*} 
suponiendo que la integral exista. 
\end{Teo}

\begin{Coro}[Identidad de Wald para Renovaciones]
Para el proceso de renovaci\'on $N\left(t\right)$,
\begin{eqnarray*}
\esp\left[T_{N\left(t\right)+1}\right]=\mu\esp\left[N\left(t\right)+1\right]\textrm{,  }t\geq0,
\end{eqnarray*}  
\end{Coro}


\begin{Def}
Sea $h\left(t\right)$ funci\'on de valores reales en $\rea$ acotada en intervalos finitos e igual a cero para $t<0$ La ecuaci\'on de renovaci\'on para $h\left(t\right)$ y la distribuci\'on $F$ es

\begin{eqnarray}%\label{Ec.Renovacion}
H\left(t\right)=h\left(t\right)+\int_{\left[0,t\right]}H\left(t-s\right)dF\left(s\right)\textrm{,    }t\geq0,
\end{eqnarray}
donde $H\left(t\right)$ es una funci\'on de valores reales. Esto es $H=h+F\star H$. Decimos que $H\left(t\right)$ es soluci\'on de esta ecuaci\'on si satisface la ecuaci\'on, y es acotada en intervalos finitos e iguales a cero para $t<0$.
\end{Def}

\begin{Prop}
La funci\'on $U\star h\left(t\right)$ es la \'unica soluci\'on de la ecuaci\'on de renovaci\'on (\ref{Ec.Renovacion}).
\end{Prop}

\begin{Teo}[Teorema Renovaci\'on Elemental]
\begin{eqnarray*}
t^{-1}U\left(t\right)\rightarrow 1/\mu\textrm{,    cuando }t\rightarrow\infty.
\end{eqnarray*}
\end{Teo}


\begin{Note} Una funci\'on $h:\rea_{+}\rightarrow\rea$ es Directamente Riemann Integrable en los siguientes casos:
\begin{itemize}
\item[a)] $h\left(t\right)\geq0$ es decreciente y Riemann Integrable.
\item[b)] $h$ es continua excepto posiblemente en un conjunto de Lebesgue de medida 0, y $|h\left(t\right)|\leq b\left(t\right)$, donde $b$ es DRI.
\end{itemize}
\end{Note}

\begin{Teo}[Teorema Principal de Renovaci\'on]
Si $F$ es no aritm\'etica y $h\left(t\right)$ es Directamente Riemann Integrable (DRI), entonces

\begin{eqnarray*}
lim_{t\rightarrow\infty}U\star h=\frac{1}{\mu}\int_{\rea_{+}}h\left(s\right)ds.
\end{eqnarray*}
\end{Teo}

\begin{Prop}
Cualquier funci\'on $H\left(t\right)$ acotada en intervalos finitos y que es 0 para $t<0$ puede expresarse como
\begin{eqnarray*}
H\left(t\right)=U\star h\left(t\right)\textrm{,  donde }h\left(t\right)=H\left(t\right)-F\star H\left(t\right)
\end{eqnarray*}
\end{Prop}

\begin{Def}
Un proceso estoc\'astico $X\left(t\right)$ es crudamente regenerativo en un tiempo aleatorio positivo $T$ si
\begin{eqnarray*}
\esp\left[X\left(T+t\right)|T\right]=\esp\left[X\left(t\right)\right]\textrm{, para }t\geq0,\end{eqnarray*}
y con las esperanzas anteriores finitas.
\end{Def}

\begin{Prop}
Sup\'ongase que $X\left(t\right)$ es un proceso crudamente regenerativo en $T$, que tiene distribuci\'on $F$. Si $\esp\left[X\left(t\right)\right]$ es acotado en intervalos finitos, entonces
\begin{eqnarray*}
\esp\left[X\left(t\right)\right]=U\star h\left(t\right)\textrm{,  donde }h\left(t\right)=\esp\left[X\left(t\right)\indora\left(T>t\right)\right].
\end{eqnarray*}
\end{Prop}

\begin{Teo}[Regeneraci\'on Cruda]
Sup\'ongase que $X\left(t\right)$ es un proceso con valores positivo crudamente regenerativo en $T$, y def\'inase $M=\sup\left\{|X\left(t\right)|:t\leq T\right\}$. Si $T$ es no aritm\'etico y $M$ y $MT$ tienen media finita, entonces
\begin{eqnarray*}
lim_{t\rightarrow\infty}\esp\left[X\left(t\right)\right]=\frac{1}{\mu}\int_{\rea_{+}}h\left(s\right)ds,
\end{eqnarray*}
donde $h\left(t\right)=\esp\left[X\left(t\right)\indora\left(T>t\right)\right]$.
\end{Teo}


\begin{Note} Una funci\'on $h:\rea_{+}\rightarrow\rea$ es Directamente Riemann Integrable en los siguientes casos:
\begin{itemize}
\item[a)] $h\left(t\right)\geq0$ es decreciente y Riemann Integrable.
\item[b)] $h$ es continua excepto posiblemente en un conjunto de Lebesgue de medida 0, y $|h\left(t\right)|\leq b\left(t\right)$, donde $b$ es DRI.
\end{itemize}
\end{Note}

\begin{Teo}[Teorema Principal de Renovaci\'on]
Si $F$ es no aritm\'etica y $h\left(t\right)$ es Directamente Riemann Integrable (DRI), entonces

\begin{eqnarray*}
lim_{t\rightarrow\infty}U\star h=\frac{1}{\mu}\int_{\rea_{+}}h\left(s\right)ds.
\end{eqnarray*}
\end{Teo}

\begin{Prop}
Cualquier funci\'on $H\left(t\right)$ acotada en intervalos finitos y que es 0 para $t<0$ puede expresarse como
\begin{eqnarray*}
H\left(t\right)=U\star h\left(t\right)\textrm{,  donde }h\left(t\right)=H\left(t\right)-F\star H\left(t\right)
\end{eqnarray*}
\end{Prop}

\begin{Def}
Un proceso estoc\'astico $X\left(t\right)$ es crudamente regenerativo en un tiempo aleatorio positivo $T$ si
\begin{eqnarray*}
\esp\left[X\left(T+t\right)|T\right]=\esp\left[X\left(t\right)\right]\textrm{, para }t\geq0,\end{eqnarray*}
y con las esperanzas anteriores finitas.
\end{Def}

\begin{Prop}
Sup\'ongase que $X\left(t\right)$ es un proceso crudamente regenerativo en $T$, que tiene distribuci\'on $F$. Si $\esp\left[X\left(t\right)\right]$ es acotado en intervalos finitos, entonces
\begin{eqnarray*}
\esp\left[X\left(t\right)\right]=U\star h\left(t\right)\textrm{,  donde }h\left(t\right)=\esp\left[X\left(t\right)\indora\left(T>t\right)\right].
\end{eqnarray*}
\end{Prop}

\begin{Teo}[Regeneraci\'on Cruda]
Sup\'ongase que $X\left(t\right)$ es un proceso con valores positivo crudamente regenerativo en $T$, y def\'inase $M=\sup\left\{|X\left(t\right)|:t\leq T\right\}$. Si $T$ es no aritm\'etico y $M$ y $MT$ tienen media finita, entonces
\begin{eqnarray*}
lim_{t\rightarrow\infty}\esp\left[X\left(t\right)\right]=\frac{1}{\mu}\int_{\rea_{+}}h\left(s\right)ds,
\end{eqnarray*}
donde $h\left(t\right)=\esp\left[X\left(t\right)\indora\left(T>t\right)\right]$.
\end{Teo}

\begin{Def}
Para el proceso $\left\{\left(N\left(t\right),X\left(t\right)\right):t\geq0\right\}$, sus trayectoria muestrales en el intervalo de tiempo $\left[T_{n-1},T_{n}\right)$ est\'an descritas por
\begin{eqnarray*}
\zeta_{n}=\left(\xi_{n},\left\{X\left(T_{n-1}+t\right):0\leq t<\xi_{n}\right\}\right)
\end{eqnarray*}
Este $\zeta_{n}$ es el $n$-\'esimo segmento del proceso. El proceso es regenerativo sobre los tiempos $T_{n}$ si sus segmentos $\zeta_{n}$ son independientes e id\'enticamennte distribuidos.
\end{Def}


\begin{Note}
Si $\tilde{X}\left(t\right)$ con espacio de estados $\tilde{S}$ es regenerativo sobre $T_{n}$, entonces $X\left(t\right)=f\left(\tilde{X}\left(t\right)\right)$ tambi\'en es regenerativo sobre $T_{n}$, para cualquier funci\'on $f:\tilde{S}\rightarrow S$.
\end{Note}

\begin{Note}
Los procesos regenerativos son crudamente regenerativos, pero no al rev\'es.
\end{Note}


\begin{Note}
Un proceso estoc\'astico a tiempo continuo o discreto es regenerativo si existe un proceso de renovaci\'on  tal que los segmentos del proceso entre tiempos de renovaci\'on sucesivos son i.i.d., es decir, para $\left\{X\left(t\right):t\geq0\right\}$ proceso estoc\'astico a tiempo continuo con espacio de estados $S$, espacio m\'etrico.
\end{Note}

Para $\left\{X\left(t\right):t\geq0\right\}$ Proceso Estoc\'astico a tiempo continuo con estado de espacios $S$, que es un espacio m\'etrico, con trayectorias continuas por la derecha y con l\'imites por la izquierda c.s. Sea $N\left(t\right)$ un proceso de renovaci\'on en $\rea_{+}$ definido en el mismo espacio de probabilidad que $X\left(t\right)$, con tiempos de renovaci\'on $T$ y tiempos de inter-renovaci\'on $\xi_{n}=T_{n}-T_{n-1}$, con misma distribuci\'on $F$ de media finita $\mu$.



\begin{Def}
Para el proceso $\left\{\left(N\left(t\right),X\left(t\right)\right):t\geq0\right\}$, sus trayectoria muestrales en el intervalo de tiempo $\left[T_{n-1},T_{n}\right)$ est\'an descritas por
\begin{eqnarray*}
\zeta_{n}=\left(\xi_{n},\left\{X\left(T_{n-1}+t\right):0\leq t<\xi_{n}\right\}\right)
\end{eqnarray*}
Este $\zeta_{n}$ es el $n$-\'esimo segmento del proceso. El proceso es regenerativo sobre los tiempos $T_{n}$ si sus segmentos $\zeta_{n}$ son independientes e id\'enticamennte distribuidos.
\end{Def}

\begin{Note}
Un proceso regenerativo con media de la longitud de ciclo finita es llamado positivo recurrente.
\end{Note}

\begin{Teo}[Procesos Regenerativos]
Suponga que el proceso
\end{Teo}


\begin{Def}[Renewal Process Trinity]
Para un proceso de renovaci\'on $N\left(t\right)$, los siguientes procesos proveen de informaci\'on sobre los tiempos de renovaci\'on.
\begin{itemize}
\item $A\left(t\right)=t-T_{N\left(t\right)}$, el tiempo de recurrencia hacia atr\'as al tiempo $t$, que es el tiempo desde la \'ultima renovaci\'on para $t$.

\item $B\left(t\right)=T_{N\left(t\right)+1}-t$, el tiempo de recurrencia hacia adelante al tiempo $t$, residual del tiempo de renovaci\'on, que es el tiempo para la pr\'oxima renovaci\'on despu\'es de $t$.

\item $L\left(t\right)=\xi_{N\left(t\right)+1}=A\left(t\right)+B\left(t\right)$, la longitud del intervalo de renovaci\'on que contiene a $t$.
\end{itemize}
\end{Def}

\begin{Note}
El proceso tridimensional $\left(A\left(t\right),B\left(t\right),L\left(t\right)\right)$ es regenerativo sobre $T_{n}$, y por ende cada proceso lo es. Cada proceso $A\left(t\right)$ y $B\left(t\right)$ son procesos de MArkov a tiempo continuo con trayectorias continuas por partes en el espacio de estados $\rea_{+}$. Una expresi\'on conveniente para su distribuci\'on conjunta es, para $0\leq x<t,y\geq0$
\begin{equation}\label{NoRenovacion}
P\left\{A\left(t\right)>x,B\left(t\right)>y\right\}=
P\left\{N\left(t+y\right)-N\left((t-x)\right)=0\right\}
\end{equation}
\end{Note}

\begin{Ejem}[Tiempos de recurrencia Poisson]
Si $N\left(t\right)$ es un proceso Poisson con tasa $\lambda$, entonces de la expresi\'on (\ref{NoRenovacion}) se tiene que

\begin{eqnarray*}
\begin{array}{lc}
P\left\{A\left(t\right)>x,B\left(t\right)>y\right\}=e^{-\lambda\left(x+y\right)},&0\leq x<t,y\geq0,
\end{array}
\end{eqnarray*}
que es la probabilidad Poisson de no renovaciones en un intervalo de longitud $x+y$.

\end{Ejem}

\begin{Note}
Una cadena de Markov erg\'odica tiene la propiedad de ser estacionaria si la distribuci\'on de su estado al tiempo $0$ es su distribuci\'on estacionaria.
\end{Note}


\begin{Def}
Un proceso estoc\'astico a tiempo continuo $\left\{X\left(t\right):t\geq0\right\}$ en un espacio general es estacionario si sus distribuciones finito dimensionales son invariantes bajo cualquier  traslado: para cada $0\leq s_{1}<s_{2}<\cdots<s_{k}$ y $t\geq0$,
\begin{eqnarray*}
\left(X\left(s_{1}+t\right),\ldots,X\left(s_{k}+t\right)\right)=_{d}\left(X\left(s_{1}\right),\ldots,X\left(s_{k}\right)\right).
\end{eqnarray*}
\end{Def}

\begin{Note}
Un proceso de Markov es estacionario si $X\left(t\right)=_{d}X\left(0\right)$, $t\geq0$.
\end{Note}

Considerese el proceso $N\left(t\right)=\sum_{n}\indora\left(\tau_{n}\leq t\right)$ en $\rea_{+}$, con puntos $0<\tau_{1}<\tau_{2}<\cdots$.

\begin{Prop}
Si $N$ es un proceso puntual estacionario y $\esp\left[N\left(1\right)\right]<\infty$, entonces $\esp\left[N\left(t\right)\right]=t\esp\left[N\left(1\right)\right]$, $t\geq0$

\end{Prop}

\begin{Teo}
Los siguientes enunciados son equivalentes
\begin{itemize}
\item[i)] El proceso retardado de renovaci\'on $N$ es estacionario.

\item[ii)] EL proceso de tiempos de recurrencia hacia adelante $B\left(t\right)$ es estacionario.


\item[iii)] $\esp\left[N\left(t\right)\right]=t/\mu$,


\item[iv)] $G\left(t\right)=F_{e}\left(t\right)=\frac{1}{\mu}\int_{0}^{t}\left[1-F\left(s\right)\right]ds$
\end{itemize}
Cuando estos enunciados son ciertos, $P\left\{B\left(t\right)\leq x\right\}=F_{e}\left(x\right)$, para $t,x\geq0$.

\end{Teo}

\begin{Note}
Una consecuencia del teorema anterior es que el Proceso Poisson es el \'unico proceso sin retardo que es estacionario.
\end{Note}

\begin{Coro}
El proceso de renovaci\'on $N\left(t\right)$ sin retardo, y cuyos tiempos de inter renonaci\'on tienen media finita, es estacionario si y s\'olo si es un proceso Poisson.

\end{Coro}


%________________________________________________________________________
%\subsection{Procesos Regenerativos}
%________________________________________________________________________



\begin{Note}
Si $\tilde{X}\left(t\right)$ con espacio de estados $\tilde{S}$ es regenerativo sobre $T_{n}$, entonces $X\left(t\right)=f\left(\tilde{X}\left(t\right)\right)$ tambi\'en es regenerativo sobre $T_{n}$, para cualquier funci\'on $f:\tilde{S}\rightarrow S$.
\end{Note}

\begin{Note}
Los procesos regenerativos son crudamente regenerativos, pero no al rev\'es.
\end{Note}
%\subsection*{Procesos Regenerativos: Sigman\cite{Sigman1}}
\begin{Def}[Definici\'on Cl\'asica]
Un proceso estoc\'astico $X=\left\{X\left(t\right):t\geq0\right\}$ es llamado regenerativo is existe una variable aleatoria $R_{1}>0$ tal que
\begin{itemize}
\item[i)] $\left\{X\left(t+R_{1}\right):t\geq0\right\}$ es independiente de $\left\{\left\{X\left(t\right):t<R_{1}\right\},\right\}$
\item[ii)] $\left\{X\left(t+R_{1}\right):t\geq0\right\}$ es estoc\'asticamente equivalente a $\left\{X\left(t\right):t>0\right\}$
\end{itemize}

Llamamos a $R_{1}$ tiempo de regeneraci\'on, y decimos que $X$ se regenera en este punto.
\end{Def}

$\left\{X\left(t+R_{1}\right)\right\}$ es regenerativo con tiempo de regeneraci\'on $R_{2}$, independiente de $R_{1}$ pero con la misma distribuci\'on que $R_{1}$. Procediendo de esta manera se obtiene una secuencia de variables aleatorias independientes e id\'enticamente distribuidas $\left\{R_{n}\right\}$ llamados longitudes de ciclo. Si definimos a $Z_{k}\equiv R_{1}+R_{2}+\cdots+R_{k}$, se tiene un proceso de renovaci\'on llamado proceso de renovaci\'on encajado para $X$.




\begin{Def}
Para $x$ fijo y para cada $t\geq0$, sea $I_{x}\left(t\right)=1$ si $X\left(t\right)\leq x$,  $I_{x}\left(t\right)=0$ en caso contrario, y def\'inanse los tiempos promedio
\begin{eqnarray*}
\overline{X}&=&lim_{t\rightarrow\infty}\frac{1}{t}\int_{0}^{\infty}X\left(u\right)du\\
\prob\left(X_{\infty}\leq x\right)&=&lim_{t\rightarrow\infty}\frac{1}{t}\int_{0}^{\infty}I_{x}\left(u\right)du,
\end{eqnarray*}
cuando estos l\'imites existan.
\end{Def}

Como consecuencia del teorema de Renovaci\'on-Recompensa, se tiene que el primer l\'imite  existe y es igual a la constante
\begin{eqnarray*}
\overline{X}&=&\frac{\esp\left[\int_{0}^{R_{1}}X\left(t\right)dt\right]}{\esp\left[R_{1}\right]},
\end{eqnarray*}
suponiendo que ambas esperanzas son finitas.

\begin{Note}
\begin{itemize}
\item[a)] Si el proceso regenerativo $X$ es positivo recurrente y tiene trayectorias muestrales no negativas, entonces la ecuaci\'on anterior es v\'alida.
\item[b)] Si $X$ es positivo recurrente regenerativo, podemos construir una \'unica versi\'on estacionaria de este proceso, $X_{e}=\left\{X_{e}\left(t\right)\right\}$, donde $X_{e}$ es un proceso estoc\'astico regenerativo y estrictamente estacionario, con distribuci\'on marginal distribuida como $X_{\infty}$
\end{itemize}
\end{Note}

%________________________________________________________________________
%\subsection{Procesos Regenerativos}
%________________________________________________________________________

Para $\left\{X\left(t\right):t\geq0\right\}$ Proceso Estoc\'astico a tiempo continuo con estado de espacios $S$, que es un espacio m\'etrico, con trayectorias continuas por la derecha y con l\'imites por la izquierda c.s. Sea $N\left(t\right)$ un proceso de renovaci\'on en $\rea_{+}$ definido en el mismo espacio de probabilidad que $X\left(t\right)$, con tiempos de renovaci\'on $T$ y tiempos de inter-renovaci\'on $\xi_{n}=T_{n}-T_{n-1}$, con misma distribuci\'on $F$ de media finita $\mu$.



\begin{Def}
Para el proceso $\left\{\left(N\left(t\right),X\left(t\right)\right):t\geq0\right\}$, sus trayectoria muestrales en el intervalo de tiempo $\left[T_{n-1},T_{n}\right)$ est\'an descritas por
\begin{eqnarray*}
\zeta_{n}=\left(\xi_{n},\left\{X\left(T_{n-1}+t\right):0\leq t<\xi_{n}\right\}\right)
\end{eqnarray*}
Este $\zeta_{n}$ es el $n$-\'esimo segmento del proceso. El proceso es regenerativo sobre los tiempos $T_{n}$ si sus segmentos $\zeta_{n}$ son independientes e id\'enticamennte distribuidos.
\end{Def}


\begin{Note}
Si $\tilde{X}\left(t\right)$ con espacio de estados $\tilde{S}$ es regenerativo sobre $T_{n}$, entonces $X\left(t\right)=f\left(\tilde{X}\left(t\right)\right)$ tambi\'en es regenerativo sobre $T_{n}$, para cualquier funci\'on $f:\tilde{S}\rightarrow S$.
\end{Note}

\begin{Note}
Los procesos regenerativos son crudamente regenerativos, pero no al rev\'es.
\end{Note}

\begin{Def}[Definici\'on Cl\'asica]
Un proceso estoc\'astico $X=\left\{X\left(t\right):t\geq0\right\}$ es llamado regenerativo is existe una variable aleatoria $R_{1}>0$ tal que
\begin{itemize}
\item[i)] $\left\{X\left(t+R_{1}\right):t\geq0\right\}$ es independiente de $\left\{\left\{X\left(t\right):t<R_{1}\right\},\right\}$
\item[ii)] $\left\{X\left(t+R_{1}\right):t\geq0\right\}$ es estoc\'asticamente equivalente a $\left\{X\left(t\right):t>0\right\}$
\end{itemize}

Llamamos a $R_{1}$ tiempo de regeneraci\'on, y decimos que $X$ se regenera en este punto.
\end{Def}

$\left\{X\left(t+R_{1}\right)\right\}$ es regenerativo con tiempo de regeneraci\'on $R_{2}$, independiente de $R_{1}$ pero con la misma distribuci\'on que $R_{1}$. Procediendo de esta manera se obtiene una secuencia de variables aleatorias independientes e id\'enticamente distribuidas $\left\{R_{n}\right\}$ llamados longitudes de ciclo. Si definimos a $Z_{k}\equiv R_{1}+R_{2}+\cdots+R_{k}$, se tiene un proceso de renovaci\'on llamado proceso de renovaci\'on encajado para $X$.

\begin{Note}
Un proceso regenerativo con media de la longitud de ciclo finita es llamado positivo recurrente.
\end{Note}


\begin{Def}
Para $x$ fijo y para cada $t\geq0$, sea $I_{x}\left(t\right)=1$ si $X\left(t\right)\leq x$,  $I_{x}\left(t\right)=0$ en caso contrario, y def\'inanse los tiempos promedio
\begin{eqnarray*}
\overline{X}&=&lim_{t\rightarrow\infty}\frac{1}{t}\int_{0}^{\infty}X\left(u\right)du\\
\prob\left(X_{\infty}\leq x\right)&=&lim_{t\rightarrow\infty}\frac{1}{t}\int_{0}^{\infty}I_{x}\left(u\right)du,
\end{eqnarray*}
cuando estos l\'imites existan.
\end{Def}

Como consecuencia del teorema de Renovaci\'on-Recompensa, se tiene que el primer l\'imite  existe y es igual a la constante
\begin{eqnarray*}
\overline{X}&=&\frac{\esp\left[\int_{0}^{R_{1}}X\left(t\right)dt\right]}{\esp\left[R_{1}\right]},
\end{eqnarray*}
suponiendo que ambas esperanzas son finitas.

\begin{Note}
\begin{itemize}
\item[a)] Si el proceso regenerativo $X$ es positivo recurrente y tiene trayectorias muestrales no negativas, entonces la ecuaci\'on anterior es v\'alida.
\item[b)] Si $X$ es positivo recurrente regenerativo, podemos construir una \'unica versi\'on estacionaria de este proceso, $X_{e}=\left\{X_{e}\left(t\right)\right\}$, donde $X_{e}$ es un proceso estoc\'astico regenerativo y estrictamente estacionario, con distribuci\'on marginal distribuida como $X_{\infty}$
\end{itemize}
\end{Note}

%__________________________________________________________________________________________
%\subsection{Procesos Regenerativos Estacionarios - Stidham \cite{Stidham}}
%__________________________________________________________________________________________


Un proceso estoc\'astico a tiempo continuo $\left\{V\left(t\right),t\geq0\right\}$ es un proceso regenerativo si existe una sucesi\'on de variables aleatorias independientes e id\'enticamente distribuidas $\left\{X_{1},X_{2},\ldots\right\}$, sucesi\'on de renovaci\'on, tal que para cualquier conjunto de Borel $A$, 

\begin{eqnarray*}
\prob\left\{V\left(t\right)\in A|X_{1}+X_{2}+\cdots+X_{R\left(t\right)}=s,\left\{V\left(\tau\right),\tau<s\right\}\right\}=\prob\left\{V\left(t-s\right)\in A|X_{1}>t-s\right\},
\end{eqnarray*}
para todo $0\leq s\leq t$, donde $R\left(t\right)=\max\left\{X_{1}+X_{2}+\cdots+X_{j}\leq t\right\}=$n\'umero de renovaciones ({\emph{puntos de regeneraci\'on}}) que ocurren en $\left[0,t\right]$. El intervalo $\left[0,X_{1}\right)$ es llamado {\emph{primer ciclo de regeneraci\'on}} de $\left\{V\left(t \right),t\geq0\right\}$, $\left[X_{1},X_{1}+X_{2}\right)$ el {\emph{segundo ciclo de regeneraci\'on}}, y as\'i sucesivamente.

Sea $X=X_{1}$ y sea $F$ la funci\'on de distrbuci\'on de $X$


\begin{Def}
Se define el proceso estacionario, $\left\{V^{*}\left(t\right),t\geq0\right\}$, para $\left\{V\left(t\right),t\geq0\right\}$ por

\begin{eqnarray*}
\prob\left\{V\left(t\right)\in A\right\}=\frac{1}{\esp\left[X\right]}\int_{0}^{\infty}\prob\left\{V\left(t+x\right)\in A|X>x\right\}\left(1-F\left(x\right)\right)dx,
\end{eqnarray*} 
para todo $t\geq0$ y todo conjunto de Borel $A$.
\end{Def}

\begin{Def}
Una distribuci\'on se dice que es {\emph{aritm\'etica}} si todos sus puntos de incremento son m\'ultiplos de la forma $0,\lambda, 2\lambda,\ldots$ para alguna $\lambda>0$ entera.
\end{Def}


\begin{Def}
Una modificaci\'on medible de un proceso $\left\{V\left(t\right),t\geq0\right\}$, es una versi\'on de este, $\left\{V\left(t,w\right)\right\}$ conjuntamente medible para $t\geq0$ y para $w\in S$, $S$ espacio de estados para $\left\{V\left(t\right),t\geq0\right\}$.
\end{Def}

\begin{Teo}
Sea $\left\{V\left(t\right),t\geq\right\}$ un proceso regenerativo no negativo con modificaci\'on medible. Sea $\esp\left[X\right]<\infty$. Entonces el proceso estacionario dado por la ecuaci\'on anterior est\'a bien definido y tiene funci\'on de distribuci\'on independiente de $t$, adem\'as
\begin{itemize}
\item[i)] \begin{eqnarray*}
\esp\left[V^{*}\left(0\right)\right]&=&\frac{\esp\left[\int_{0}^{X}V\left(s\right)ds\right]}{\esp\left[X\right]}\end{eqnarray*}
\item[ii)] Si $\esp\left[V^{*}\left(0\right)\right]<\infty$, equivalentemente, si $\esp\left[\int_{0}^{X}V\left(s\right)ds\right]<\infty$,entonces
\begin{eqnarray*}
\frac{\int_{0}^{t}V\left(s\right)ds}{t}\rightarrow\frac{\esp\left[\int_{0}^{X}V\left(s\right)ds\right]}{\esp\left[X\right]}
\end{eqnarray*}
con probabilidad 1 y en media, cuando $t\rightarrow\infty$.
\end{itemize}
\end{Teo}
%
%___________________________________________________________________________________________
%\vspace{5.5cm}
%\chapter{Cadenas de Markov estacionarias}
%\vspace{-1.0cm}


%__________________________________________________________________________________________
%\subsection{Procesos Regenerativos Estacionarios - Stidham \cite{Stidham}}
%__________________________________________________________________________________________


Un proceso estoc\'astico a tiempo continuo $\left\{V\left(t\right),t\geq0\right\}$ es un proceso regenerativo si existe una sucesi\'on de variables aleatorias independientes e id\'enticamente distribuidas $\left\{X_{1},X_{2},\ldots\right\}$, sucesi\'on de renovaci\'on, tal que para cualquier conjunto de Borel $A$, 

\begin{eqnarray*}
\prob\left\{V\left(t\right)\in A|X_{1}+X_{2}+\cdots+X_{R\left(t\right)}=s,\left\{V\left(\tau\right),\tau<s\right\}\right\}=\prob\left\{V\left(t-s\right)\in A|X_{1}>t-s\right\},
\end{eqnarray*}
para todo $0\leq s\leq t$, donde $R\left(t\right)=\max\left\{X_{1}+X_{2}+\cdots+X_{j}\leq t\right\}=$n\'umero de renovaciones ({\emph{puntos de regeneraci\'on}}) que ocurren en $\left[0,t\right]$. El intervalo $\left[0,X_{1}\right)$ es llamado {\emph{primer ciclo de regeneraci\'on}} de $\left\{V\left(t \right),t\geq0\right\}$, $\left[X_{1},X_{1}+X_{2}\right)$ el {\emph{segundo ciclo de regeneraci\'on}}, y as\'i sucesivamente.

Sea $X=X_{1}$ y sea $F$ la funci\'on de distrbuci\'on de $X$


\begin{Def}
Se define el proceso estacionario, $\left\{V^{*}\left(t\right),t\geq0\right\}$, para $\left\{V\left(t\right),t\geq0\right\}$ por

\begin{eqnarray*}
\prob\left\{V\left(t\right)\in A\right\}=\frac{1}{\esp\left[X\right]}\int_{0}^{\infty}\prob\left\{V\left(t+x\right)\in A|X>x\right\}\left(1-F\left(x\right)\right)dx,
\end{eqnarray*} 
para todo $t\geq0$ y todo conjunto de Borel $A$.
\end{Def}

\begin{Def}
Una distribuci\'on se dice que es {\emph{aritm\'etica}} si todos sus puntos de incremento son m\'ultiplos de la forma $0,\lambda, 2\lambda,\ldots$ para alguna $\lambda>0$ entera.
\end{Def}


\begin{Def}
Una modificaci\'on medible de un proceso $\left\{V\left(t\right),t\geq0\right\}$, es una versi\'on de este, $\left\{V\left(t,w\right)\right\}$ conjuntamente medible para $t\geq0$ y para $w\in S$, $S$ espacio de estados para $\left\{V\left(t\right),t\geq0\right\}$.
\end{Def}

\begin{Teo}
Sea $\left\{V\left(t\right),t\geq\right\}$ un proceso regenerativo no negativo con modificaci\'on medible. Sea $\esp\left[X\right]<\infty$. Entonces el proceso estacionario dado por la ecuaci\'on anterior est\'a bien definido y tiene funci\'on de distribuci\'on independiente de $t$, adem\'as
\begin{itemize}
\item[i)] \begin{eqnarray*}
\esp\left[V^{*}\left(0\right)\right]&=&\frac{\esp\left[\int_{0}^{X}V\left(s\right)ds\right]}{\esp\left[X\right]}\end{eqnarray*}
\item[ii)] Si $\esp\left[V^{*}\left(0\right)\right]<\infty$, equivalentemente, si $\esp\left[\int_{0}^{X}V\left(s\right)ds\right]<\infty$,entonces
\begin{eqnarray*}
\frac{\int_{0}^{t}V\left(s\right)ds}{t}\rightarrow\frac{\esp\left[\int_{0}^{X}V\left(s\right)ds\right]}{\esp\left[X\right]}
\end{eqnarray*}
con probabilidad 1 y en media, cuando $t\rightarrow\infty$.
\end{itemize}
\end{Teo}

Para $\left\{X\left(t\right):t\geq0\right\}$ Proceso Estoc\'astico a tiempo continuo con estado de espacios $S$, que es un espacio m\'etrico, con trayectorias continuas por la derecha y con l\'imites por la izquierda c.s. Sea $N\left(t\right)$ un proceso de renovaci\'on en $\rea_{+}$ definido en el mismo espacio de probabilidad que $X\left(t\right)$, con tiempos de renovaci\'on $T$ y tiempos de inter-renovaci\'on $\xi_{n}=T_{n}-T_{n-1}$, con misma distribuci\'on $F$ de media finita $\mu$.


%______________________________________________________________________
%\subsection{Ejemplos, Notas importantes}


Sean $T_{1},T_{2},\ldots$ los puntos donde las longitudes de las colas de la red de sistemas de visitas c\'iclicas son cero simult\'aneamente, cuando la cola $Q_{j}$ es visitada por el servidor para dar servicio, es decir, $L_{1}\left(T_{i}\right)=0,L_{2}\left(T_{i}\right)=0,\hat{L}_{1}\left(T_{i}\right)=0$ y $\hat{L}_{2}\left(T_{i}\right)=0$, a estos puntos se les denominar\'a puntos regenerativos. Sea la funci\'on generadora de momentos para $L_{i}$, el n\'umero de usuarios en la cola $Q_{i}\left(z\right)$ en cualquier momento, est\'a dada por el tiempo promedio de $z^{L_{i}\left(t\right)}$ sobre el ciclo regenerativo definido anteriormente:

\begin{eqnarray*}
Q_{i}\left(z\right)&=&\esp\left[z^{L_{i}\left(t\right)}\right]=\frac{\esp\left[\sum_{m=1}^{M_{i}}\sum_{t=\tau_{i}\left(m\right)}^{\tau_{i}\left(m+1\right)-1}z^{L_{i}\left(t\right)}\right]}{\esp\left[\sum_{m=1}^{M_{i}}\tau_{i}\left(m+1\right)-\tau_{i}\left(m\right)\right]}
\end{eqnarray*}

$M_{i}$ es un tiempo de paro en el proceso regenerativo con $\esp\left[M_{i}\right]<\infty$\footnote{En Stidham\cite{Stidham} y Heyman  se muestra que una condici\'on suficiente para que el proceso regenerativo 
estacionario sea un procesoo estacionario es que el valor esperado del tiempo del ciclo regenerativo sea finito, es decir: $\esp\left[\sum_{m=1}^{M_{i}}C_{i}^{(m)}\right]<\infty$, como cada $C_{i}^{(m)}$ contiene intervalos de r\'eplica positivos, se tiene que $\esp\left[M_{i}\right]<\infty$, adem\'as, como $M_{i}>0$, se tiene que la condici\'on anterior es equivalente a tener que $\esp\left[C_{i}\right]<\infty$,
por lo tanto una condici\'on suficiente para la existencia del proceso regenerativo est\'a dada por $\sum_{k=1}^{N}\mu_{k}<1.$}, se sigue del lema de Wald que:


\begin{eqnarray*}
\esp\left[\sum_{m=1}^{M_{i}}\sum_{t=\tau_{i}\left(m\right)}^{\tau_{i}\left(m+1\right)-1}z^{L_{i}\left(t\right)}\right]&=&\esp\left[M_{i}\right]\esp\left[\sum_{t=\tau_{i}\left(m\right)}^{\tau_{i}\left(m+1\right)-1}z^{L_{i}\left(t\right)}\right]\\
\esp\left[\sum_{m=1}^{M_{i}}\tau_{i}\left(m+1\right)-\tau_{i}\left(m\right)\right]&=&\esp\left[M_{i}\right]\esp\left[\tau_{i}\left(m+1\right)-\tau_{i}\left(m\right)\right]
\end{eqnarray*}

por tanto se tiene que


\begin{eqnarray*}
Q_{i}\left(z\right)&=&\frac{\esp\left[\sum_{t=\tau_{i}\left(m\right)}^{\tau_{i}\left(m+1\right)-1}z^{L_{i}\left(t\right)}\right]}{\esp\left[\tau_{i}\left(m+1\right)-\tau_{i}\left(m\right)\right]}
\end{eqnarray*}

observar que el denominador es simplemente la duraci\'on promedio del tiempo del ciclo.


Haciendo las siguientes sustituciones en la ecuaci\'on (\ref{Corolario2}): $n\rightarrow t-\tau_{i}\left(m\right)$, $T \rightarrow \overline{\tau}_{i}\left(m\right)-\tau_{i}\left(m\right)$, $L_{n}\rightarrow L_{i}\left(t\right)$ y $F\left(z\right)=\esp\left[z^{L_{0}}\right]\rightarrow F_{i}\left(z\right)=\esp\left[z^{L_{i}\tau_{i}\left(m\right)}\right]$, se puede ver que

\begin{eqnarray}\label{Eq.Arribos.Primera}
\esp\left[\sum_{n=0}^{T-1}z^{L_{n}}\right]=
\esp\left[\sum_{t=\tau_{i}\left(m\right)}^{\overline{\tau}_{i}\left(m\right)-1}z^{L_{i}\left(t\right)}\right]
=z\frac{F_{i}\left(z\right)-1}{z-P_{i}\left(z\right)}
\end{eqnarray}

Por otra parte durante el tiempo de intervisita para la cola $i$, $L_{i}\left(t\right)$ solamente se incrementa de manera que el incremento por intervalo de tiempo est\'a dado por la funci\'on generadora de probabilidades de $P_{i}\left(z\right)$, por tanto la suma sobre el tiempo de intervisita puede evaluarse como:

\begin{eqnarray*}
\esp\left[\sum_{t=\tau_{i}\left(m\right)}^{\tau_{i}\left(m+1\right)-1}z^{L_{i}\left(t\right)}\right]&=&\esp\left[\sum_{t=\tau_{i}\left(m\right)}^{\tau_{i}\left(m+1\right)-1}\left\{P_{i}\left(z\right)\right\}^{t-\overline{\tau}_{i}\left(m\right)}\right]=\frac{1-\esp\left[\left\{P_{i}\left(z\right)\right\}^{\tau_{i}\left(m+1\right)-\overline{\tau}_{i}\left(m\right)}\right]}{1-P_{i}\left(z\right)}\\
&=&\frac{1-I_{i}\left[P_{i}\left(z\right)\right]}{1-P_{i}\left(z\right)}
\end{eqnarray*}
por tanto

\begin{eqnarray*}
\esp\left[\sum_{t=\tau_{i}\left(m\right)}^{\tau_{i}\left(m+1\right)-1}z^{L_{i}\left(t\right)}\right]&=&
\frac{1-F_{i}\left(z\right)}{1-P_{i}\left(z\right)}
\end{eqnarray*}

Por lo tanto

\begin{eqnarray*}
Q_{i}\left(z\right)&=&\frac{\esp\left[\sum_{t=\tau_{i}\left(m\right)}^{\tau_{i}
\left(m+1\right)-1}z^{L_{i}\left(t\right)}\right]}{\esp\left[\tau_{i}\left(m+1\right)-\tau_{i}\left(m\right)\right]}\\
&=&\frac{1}{\esp\left[\tau_{i}\left(m+1\right)-\tau_{i}\left(m\right)\right]}
\left\{
\esp\left[\sum_{t=\tau_{i}\left(m\right)}^{\overline{\tau}_{i}\left(m\right)-1}
z^{L_{i}\left(t\right)}\right]
+\esp\left[\sum_{t=\overline{\tau}_{i}\left(m\right)}^{\tau_{i}\left(m+1\right)-1}
z^{L_{i}\left(t\right)}\right]\right\}\\
&=&\frac{1}{\esp\left[\tau_{i}\left(m+1\right)-\tau_{i}\left(m\right)\right]}
\left\{
z\frac{F_{i}\left(z\right)-1}{z-P_{i}\left(z\right)}+\frac{1-F_{i}\left(z\right)}
{1-P_{i}\left(z\right)}
\right\}
\end{eqnarray*}

es decir

\begin{equation}
Q_{i}\left(z\right)=\frac{1}{\esp\left[C_{i}\right]}\cdot\frac{1-F_{i}\left(z\right)}{P_{i}\left(z\right)-z}\cdot\frac{\left(1-z\right)P_{i}\left(z\right)}{1-P_{i}\left(z\right)}
\end{equation}

\begin{Teo}
Dada una Red de Sistemas de Visitas C\'iclicas (RSVC), conformada por dos Sistemas de Visitas C\'iclicas (SVC), donde cada uno de ellos consta de dos colas tipo $M/M/1$. Los dos sistemas est\'an comunicados entre s\'i por medio de la transferencia de usuarios entre las colas $Q_{1}\leftrightarrow Q_{3}$ y $Q_{2}\leftrightarrow Q_{4}$. Se definen los eventos para los procesos de arribos al tiempo $t$, $A_{j}\left(t\right)=\left\{0 \textrm{ arribos en }Q_{j}\textrm{ al tiempo }t\right\}$ para alg\'un tiempo $t\geq0$ y $Q_{j}$ la cola $j$-\'esima en la RSVC, para $j=1,2,3,4$.  Existe un intervalo $I\neq\emptyset$ tal que para $T^{*}\in I$, tal que $\prob\left\{A_{1}\left(T^{*}\right),A_{2}\left(Tt^{*}\right),
A_{3}\left(T^{*}\right),A_{4}\left(T^{*}\right)|T^{*}\in I\right\}>0$.
\end{Teo}

\begin{proof}
Sin p\'erdida de generalidad podemos considerar como base del an\'alisis a la cola $Q_{1}$ del primer sistema que conforma la RSVC.

Sea $n>0$, ciclo en el primer sistema en el que se sabe que $L_{j}\left(\overline{\tau}_{1}\left(n\right)\right)=0$, pues la pol\'itica de servicio con que atienden los servidores es la exhaustiva. Como es sabido, para trasladarse a la siguiente cola, el servidor incurre en un tiempo de traslado $r_{1}\left(n\right)>0$, entonces tenemos que $\tau_{2}\left(n\right)=\overline{\tau}_{1}\left(n\right)+r_{1}\left(n\right)$.


Definamos el intervalo $I_{1}\equiv\left[\overline{\tau}_{1}\left(n\right),\tau_{2}\left(n\right)\right]$ de longitud $\xi_{1}=r_{1}\left(n\right)$. Dado que los tiempos entre arribo son exponenciales con tasa $\tilde{\mu}_{1}=\mu_{1}+\hat{\mu}_{1}$ ($\mu_{1}$ son los arribos a $Q_{1}$ por primera vez al sistema, mientras que $\hat{\mu}_{1}$ son los arribos de traslado procedentes de $Q_{3}$) se tiene que la probabilidad del evento $A_{1}\left(t\right)$ est\'a dada por 

\begin{equation}
\prob\left\{A_{1}\left(t\right)|t\in I_{1}\left(n\right)\right\}=e^{-\tilde{\mu}_{1}\xi_{1}\left(n\right)}.
\end{equation} 

Por otra parte, para la cola $Q_{2}$, el tiempo $\overline{\tau}_{2}\left(n-1\right)$ es tal que $L_{2}\left(\overline{\tau}_{2}\left(n-1\right)\right)=0$, es decir, es el tiempo en que la cola queda totalmente vac\'ia en el ciclo anterior a $n$. Entonces tenemos un sgundo intervalo $I_{2}\equiv\left[\overline{\tau}_{2}\left(n-1\right),\tau_{2}\left(n\right)\right]$. Por lo tanto la probabilidad del evento $A_{2}\left(t\right)$ tiene probabilidad dada por

\begin{equation}
\prob\left\{A_{2}\left(t\right)|t\in I_{2}\left(n\right)\right\}=e^{-\tilde{\mu}_{2}\xi_{2}\left(n\right)},
\end{equation} 

donde $\xi_{2}\left(n\right)=\tau_{2}\left(n\right)-\overline{\tau}_{2}\left(n-1\right)$.



Entonces, se tiene que

\begin{eqnarray*}
\prob\left\{A_{1}\left(t\right),A_{2}\left(t\right)|t\in I_{1}\left(n\right)\right\}&=&
\prob\left\{A_{1}\left(t\right)|t\in I_{1}\left(n\right)\right\}
\prob\left\{A_{2}\left(t\right)|t\in I_{1}\left(n\right)\right\}\\
&\geq&
\prob\left\{A_{1}\left(t\right)|t\in I_{1}\left(n\right)\right\}
\prob\left\{A_{2}\left(t\right)|t\in I_{2}\left(n\right)\right\}\\
&=&e^{-\tilde{\mu}_{1}\xi_{1}\left(n\right)}e^{-\tilde{\mu}_{2}\xi_{2}\left(n\right)}
=e^{-\left[\tilde{\mu}_{1}\xi_{1}\left(n\right)+\tilde{\mu}_{2}\xi_{2}\left(n\right)\right]}.
\end{eqnarray*}


es decir, 

\begin{equation}
\prob\left\{A_{1}\left(t\right),A_{2}\left(t\right)|t\in I_{1}\left(n\right)\right\}
=e^{-\left[\tilde{\mu}_{1}\xi_{1}\left(n\right)+\tilde{\mu}_{2}\xi_{2}
\left(n\right)\right]}>0.
\end{equation}

En lo que respecta a la relaci\'on entre los dos SVC que conforman la RSVC, se afirma que existe $m>0$ tal que $\overline{\tau}_{3}\left(m\right)<\tau_{2}\left(n\right)<\tau_{4}\left(m\right)$.

Para $Q_{3}$ sea $I_{3}=\left[\overline{\tau}_{3}\left(m\right),\tau_{4}\left(m\right)\right]$ con longitud  $\xi_{3}\left(m\right)=r_{3}\left(m\right)$, entonces 

\begin{equation}
\prob\left\{A_{3}\left(t\right)|t\in I_{3}\left(n\right)\right\}=e^{-\tilde{\mu}_{3}\xi_{3}\left(n\right)}.
\end{equation} 

An\'alogamente que como se hizo para $Q_{2}$, tenemos que para $Q_{4}$ se tiene el intervalo $I_{4}=\left[\overline{\tau}_{4}\left(m-1\right),\tau_{4}\left(m\right)\right]$ con longitud $\xi_{4}\left(m\right)=\tau_{4}\left(m\right)-\overline{\tau}_{4}\left(m-1\right)$, entonces


\begin{equation}
\prob\left\{A_{4}\left(t\right)|t\in I_{4}\left(m\right)\right\}=e^{-\tilde{\mu}_{4}\xi_{4}\left(n\right)}.
\end{equation} 

Al igual que para el primer sistema, dado que $I_{3}\left(m\right)\subset I_{4}\left(m\right)$, se tiene que

\begin{eqnarray*}
\xi_{3}\left(m\right)\leq\xi_{4}\left(m\right)&\Leftrightarrow& -\xi_{3}\left(m\right)\geq-\xi_{4}\left(m\right)
\\
-\tilde{\mu}_{4}\xi_{3}\left(m\right)\geq-\tilde{\mu}_{4}\xi_{4}\left(m\right)&\Leftrightarrow&
e^{-\tilde{\mu}_{4}\xi_{3}\left(m\right)}\geq e^{-\tilde{\mu}_{4}\xi_{4}\left(m\right)}\\
\prob\left\{A_{4}\left(t\right)|t\in I_{3}\left(m\right)\right\}&\geq&
\prob\left\{A_{4}\left(t\right)|t\in I_{4}\left(m\right)\right\}
\end{eqnarray*}

Entonces, dado que los eventos $A_{3}$ y $A_{4}$ son independientes, se tiene que

\begin{eqnarray*}
\prob\left\{A_{3}\left(t\right),A_{4}\left(t\right)|t\in I_{3}\left(m\right)\right\}&=&
\prob\left\{A_{3}\left(t\right)|t\in I_{3}\left(m\right)\right\}
\prob\left\{A_{4}\left(t\right)|t\in I_{3}\left(m\right)\right\}\\
&\geq&
\prob\left\{A_{3}\left(t\right)|t\in I_{3}\left(n\right)\right\}
\prob\left\{A_{4}\left(t\right)|t\in I_{4}\left(n\right)\right\}\\
&=&e^{-\tilde{\mu}_{3}\xi_{3}\left(m\right)}e^{-\tilde{\mu}_{4}\xi_{4}
\left(m\right)}
=e^{-\left[\tilde{\mu}_{3}\xi_{3}\left(m\right)+\tilde{\mu}_{4}\xi_{4}
\left(m\right)\right]}.
\end{eqnarray*}


es decir, 

\begin{equation}
\prob\left\{A_{3}\left(t\right),A_{4}\left(t\right)|t\in I_{3}\left(m\right)\right\}
=e^{-\left[\tilde{\mu}_{3}\xi_{3}\left(m\right)+\tilde{\mu}_{4}\xi_{4}
\left(m\right)\right]}>0.
\end{equation}

Por construcci\'on se tiene que $I\left(n,m\right)\equiv I_{1}\left(n\right)\cap I_{3}\left(m\right)\neq\emptyset$,entonces en particular se tienen las contenciones $I\left(n,m\right)\subseteq I_{1}\left(n\right)$ y $I\left(n,m\right)\subseteq I_{3}\left(m\right)$, por lo tanto si definimos $\xi_{n,m}\equiv\ell\left(I\left(n,m\right)\right)$ tenemos que

\begin{eqnarray*}
\xi_{n,m}\leq\xi_{1}\left(n\right)\textrm{ y }\xi_{n,m}\leq\xi_{3}\left(m\right)\textrm{ entonces }
-\xi_{n,m}\geq-\xi_{1}\left(n\right)\textrm{ y }-\xi_{n,m}\leq-\xi_{3}\left(m\right)\\
\end{eqnarray*}
por lo tanto tenemos las desigualdades 



\begin{eqnarray*}
\begin{array}{ll}
-\tilde{\mu}_{1}\xi_{n,m}\geq-\tilde{\mu}_{1}\xi_{1}\left(n\right),&
-\tilde{\mu}_{2}\xi_{n,m}\geq-\tilde{\mu}_{2}\xi_{1}\left(n\right)
\geq-\tilde{\mu}_{2}\xi_{2}\left(n\right),\\
-\tilde{\mu}_{3}\xi_{n,m}\geq-\tilde{\mu}_{3}\xi_{3}\left(m\right),&
-\tilde{\mu}_{4}\xi_{n,m}\geq-\tilde{\mu}_{4}\xi_{3}\left(m\right)
\geq-\tilde{\mu}_{4}\xi_{4}\left(m\right).
\end{array}
\end{eqnarray*}

Sea $T^{*}\in I_{n,m}$, entonces dado que en particular $T^{*}\in I_{1}\left(n\right)$ se cumple con probabilidad positiva que no hay arribos a las colas $Q_{1}$ y $Q_{2}$, en consecuencia, tampoco hay usuarios de transferencia para $Q_{3}$ y $Q_{4}$, es decir, $\tilde{\mu}_{1}=\mu_{1}$, $\tilde{\mu}_{2}=\mu_{2}$, $\tilde{\mu}_{3}=\mu_{3}$, $\tilde{\mu}_{4}=\mu_{4}$, es decir, los eventos $Q_{1}$ y $Q_{3}$ son condicionalmente independientes en el intervalo $I_{n,m}$; lo mismo ocurre para las colas $Q_{2}$ y $Q_{4}$, por lo tanto tenemos que


\begin{eqnarray}
\begin{array}{l}
\prob\left\{A_{1}\left(T^{*}\right),A_{2}\left(T^{*}\right),
A_{3}\left(T^{*}\right),A_{4}\left(T^{*}\right)|T^{*}\in I_{n,m}\right\}
=\prod_{j=1}^{4}\prob\left\{A_{j}\left(T^{*}\right)|T^{*}\in I_{n,m}\right\}\\
\geq\prob\left\{A_{1}\left(T^{*}\right)|T^{*}\in I_{1}\left(n\right)\right\}
\prob\left\{A_{2}\left(T^{*}\right)|T^{*}\in I_{2}\left(n\right)\right\}
\prob\left\{A_{3}\left(T^{*}\right)|T^{*}\in I_{3}\left(m\right)\right\}
\prob\left\{A_{4}\left(T^{*}\right)|T^{*}\in I_{4}\left(m\right)\right\}\\
=e^{-\mu_{1}\xi_{1}\left(n\right)}
e^{-\mu_{2}\xi_{2}\left(n\right)}
e^{-\mu_{3}\xi_{3}\left(m\right)}
e^{-\mu_{4}\xi_{4}\left(m\right)}
=e^{-\left[\tilde{\mu}_{1}\xi_{1}\left(n\right)
+\tilde{\mu}_{2}\xi_{2}\left(n\right)
+\tilde{\mu}_{3}\xi_{3}\left(m\right)
+\tilde{\mu}_{4}\xi_{4}
\left(m\right)\right]}>0.
\end{array}
\end{eqnarray}
\end{proof}


Estos resultados aparecen en Daley (1968) \cite{Daley68} para $\left\{T_{n}\right\}$ intervalos de inter-arribo, $\left\{D_{n}\right\}$ intervalos de inter-salida y $\left\{S_{n}\right\}$ tiempos de servicio.

\begin{itemize}
\item Si el proceso $\left\{T_{n}\right\}$ es Poisson, el proceso $\left\{D_{n}\right\}$ es no correlacionado si y s\'olo si es un proceso Poisso, lo cual ocurre si y s\'olo si $\left\{S_{n}\right\}$ son exponenciales negativas.

\item Si $\left\{S_{n}\right\}$ son exponenciales negativas, $\left\{D_{n}\right\}$ es un proceso de renovaci\'on  si y s\'olo si es un proceso Poisson, lo cual ocurre si y s\'olo si $\left\{T_{n}\right\}$ es un proceso Poisson.

\item $\esp\left(D_{n}\right)=\esp\left(T_{n}\right)$.

\item Para un sistema de visitas $GI/M/1$ se tiene el siguiente teorema:

\begin{Teo}
En un sistema estacionario $GI/M/1$ los intervalos de interpartida tienen
\begin{eqnarray*}
\esp\left(e^{-\theta D_{n}}\right)&=&\mu\left(\mu+\theta\right)^{-1}\left[\delta\theta
-\mu\left(1-\delta\right)\alpha\left(\theta\right)\right]
\left[\theta-\mu\left(1-\delta\right)^{-1}\right]\\
\alpha\left(\theta\right)&=&\esp\left[e^{-\theta T_{0}}\right]\\
var\left(D_{n}\right)&=&var\left(T_{0}\right)-\left(\tau^{-1}-\delta^{-1}\right)
2\delta\left(\esp\left(S_{0}\right)\right)^{2}\left(1-\delta\right)^{-1}.
\end{eqnarray*}
\end{Teo}



\begin{Teo}
El proceso de salida de un sistema de colas estacionario $GI/M/1$ es un proceso de renovaci\'on si y s\'olo si el proceso de entrada es un proceso Poisson, en cuyo caso el proceso de salida es un proceso Poisson.
\end{Teo}


\begin{Teo}
Los intervalos de interpartida $\left\{D_{n}\right\}$ de un sistema $M/G/1$ estacionario son no correlacionados si y s\'olo si la distribuci\'on de los tiempos de servicio es exponencial negativa, es decir, el sistema es de tipo  $M/M/1$.

\end{Teo}



\end{itemize}


%\section{Resultados para Procesos de Salida}

En Sigman, Thorison y Wolff \cite{Sigman2} prueban que para la existencia de un una sucesi\'on infinita no decreciente de tiempos de regeneraci\'on $\tau_{1}\leq\tau_{2}\leq\cdots$ en los cuales el proceso se regenera, basta un tiempo de regeneraci\'on $R_{1}$, donde $R_{j}=\tau_{j}-\tau_{j-1}$. Para tal efecto se requiere la existencia de un espacio de probabilidad $\left(\Omega,\mathcal{F},\prob\right)$, y proceso estoc\'astico $\textit{X}=\left\{X\left(t\right):t\geq0\right\}$ con espacio de estados $\left(S,\mathcal{R}\right)$, con $\mathcal{R}$ $\sigma$-\'algebra.

\begin{Prop}
Si existe una variable aleatoria no negativa $R_{1}$ tal que $\theta_{R\footnotesize{1}}X=_{D}X$, entonces $\left(\Omega,\mathcal{F},\prob\right)$ puede extenderse para soportar una sucesi\'on estacionaria de variables aleatorias $R=\left\{R_{k}:k\geq1\right\}$, tal que para $k\geq1$,
\begin{eqnarray*}
\theta_{k}\left(X,R\right)=_{D}\left(X,R\right).
\end{eqnarray*}

Adem\'as, para $k\geq1$, $\theta_{k}R$ es condicionalmente independiente de $\left(X,R_{1},\ldots,R_{k}\right)$, dado $\theta_{\tau k}X$.

\end{Prop}


\begin{itemize}
\item Doob en 1953 demostr\'o que el estado estacionario de un proceso de partida en un sistema de espera $M/G/\infty$, es Poisson con la misma tasa que el proceso de arribos.

\item Burke en 1968, fue el primero en demostrar que el estado estacionario de un proceso de salida de una cola $M/M/s$ es un proceso Poisson.

\item Disney en 1973 obtuvo el siguiente resultado:

\begin{Teo}
Para el sistema de espera $M/G/1/L$ con disciplina FIFO, el proceso $\textbf{I}$ es un proceso de renovaci\'on si y s\'olo si el proceso denominado longitud de la cola es estacionario y se cumple cualquiera de los siguientes casos:

\begin{itemize}
\item[a)] Los tiempos de servicio son identicamente cero;
\item[b)] $L=0$, para cualquier proceso de servicio $S$;
\item[c)] $L=1$ y $G=D$;
\item[d)] $L=\infty$ y $G=M$.
\end{itemize}
En estos casos, respectivamente, las distribuciones de interpartida $P\left\{T_{n+1}-T_{n}\leq t\right\}$ son


\begin{itemize}
\item[a)] $1-e^{-\lambda t}$, $t\geq0$;
\item[b)] $1-e^{-\lambda t}*F\left(t\right)$, $t\geq0$;
\item[c)] $1-e^{-\lambda t}*\indora_{d}\left(t\right)$, $t\geq0$;
\item[d)] $1-e^{-\lambda t}*F\left(t\right)$, $t\geq0$.
\end{itemize}
\end{Teo}


\item Finch (1959) mostr\'o que para los sistemas $M/G/1/L$, con $1\leq L\leq \infty$ con distribuciones de servicio dos veces diferenciable, solamente el sistema $M/M/1/\infty$ tiene proceso de salida de renovaci\'on estacionario.

\item King (1971) demostro que un sistema de colas estacionario $M/G/1/1$ tiene sus tiempos de interpartida sucesivas $D_{n}$ y $D_{n+1}$ son independientes, si y s\'olo si, $G=D$, en cuyo caso le proceso de salida es de renovaci\'on.

\item Disney (1973) demostr\'o que el \'unico sistema estacionario $M/G/1/L$, que tiene proceso de salida de renovaci\'on  son los sistemas $M/M/1$ y $M/D/1/1$.



\item El siguiente resultado es de Disney y Koning (1985)
\begin{Teo}
En un sistema de espera $M/G/s$, el estado estacionario del proceso de salida es un proceso Poisson para cualquier distribuci\'on de los tiempos de servicio si el sistema tiene cualquiera de las siguientes cuatro propiedades.

\begin{itemize}
\item[a)] $s=\infty$
\item[b)] La disciplina de servicio es de procesador compartido.
\item[c)] La disciplina de servicio es LCFS y preemptive resume, esto se cumple para $L<\infty$
\item[d)] $G=M$.
\end{itemize}

\end{Teo}

\item El siguiente resultado es de Alamatsaz (1983)

\begin{Teo}
En cualquier sistema de colas $GI/G/1/L$ con $1\leq L<\infty$ y distribuci\'on de interarribos $A$ y distribuci\'on de los tiempos de servicio $B$, tal que $A\left(0\right)=0$, $A\left(t\right)\left(1-B\left(t\right)\right)>0$ para alguna $t>0$ y $B\left(t\right)$ para toda $t>0$, es imposible que el proceso de salida estacionario sea de renovaci\'on.
\end{Teo}

\end{itemize}

Estos resultados aparecen en Daley (1968) \cite{Daley68} para $\left\{T_{n}\right\}$ intervalos de inter-arribo, $\left\{D_{n}\right\}$ intervalos de inter-salida y $\left\{S_{n}\right\}$ tiempos de servicio.

\begin{itemize}
\item Si el proceso $\left\{T_{n}\right\}$ es Poisson, el proceso $\left\{D_{n}\right\}$ es no correlacionado si y s\'olo si es un proceso Poisso, lo cual ocurre si y s\'olo si $\left\{S_{n}\right\}$ son exponenciales negativas.

\item Si $\left\{S_{n}\right\}$ son exponenciales negativas, $\left\{D_{n}\right\}$ es un proceso de renovaci\'on  si y s\'olo si es un proceso Poisson, lo cual ocurre si y s\'olo si $\left\{T_{n}\right\}$ es un proceso Poisson.

\item $\esp\left(D_{n}\right)=\esp\left(T_{n}\right)$.

\item Para un sistema de visitas $GI/M/1$ se tiene el siguiente teorema:

\begin{Teo}
En un sistema estacionario $GI/M/1$ los intervalos de interpartida tienen
\begin{eqnarray*}
\esp\left(e^{-\theta D_{n}}\right)&=&\mu\left(\mu+\theta\right)^{-1}\left[\delta\theta
-\mu\left(1-\delta\right)\alpha\left(\theta\right)\right]
\left[\theta-\mu\left(1-\delta\right)^{-1}\right]\\
\alpha\left(\theta\right)&=&\esp\left[e^{-\theta T_{0}}\right]\\
var\left(D_{n}\right)&=&var\left(T_{0}\right)-\left(\tau^{-1}-\delta^{-1}\right)
2\delta\left(\esp\left(S_{0}\right)\right)^{2}\left(1-\delta\right)^{-1}.
\end{eqnarray*}
\end{Teo}



\begin{Teo}
El proceso de salida de un sistema de colas estacionario $GI/M/1$ es un proceso de renovaci\'on si y s\'olo si el proceso de entrada es un proceso Poisson, en cuyo caso el proceso de salida es un proceso Poisson.
\end{Teo}


\begin{Teo}
Los intervalos de interpartida $\left\{D_{n}\right\}$ de un sistema $M/G/1$ estacionario son no correlacionados si y s\'olo si la distribuci\'on de los tiempos de servicio es exponencial negativa, es decir, el sistema es de tipo  $M/M/1$.

\end{Teo}



\end{itemize}
%\newpage
%________________________________________________________________________
%\section{Redes de Sistemas de Visitas C\'iclicas}
%________________________________________________________________________

Sean $Q_{1},Q_{2},Q_{3}$ y $Q_{4}$ en una Red de Sistemas de Visitas C\'iclicas (RSVC). Supongamos que cada una de las colas es del tipo $M/M/1$ con tasa de arribo $\mu_{i}$ y que la transferencia de usuarios entre los dos sistemas ocurre entre $Q_{1}\leftrightarrow Q_{3}$ y $Q_{2}\leftrightarrow Q_{4}$ con respectiva tasa de arribo igual a la tasa de salida $\hat{\mu}_{i}=\mu_{i}$, esto se sabe por lo desarrollado en la secci\'on anterior.  

Consideremos, sin p\'erdida de generalidad como base del an\'alisis, la cola $Q_{1}$ adem\'as supongamos al servidor lo comenzamos a observar una vez que termina de atender a la misma para desplazarse y llegar a $Q_{2}$, es decir al tiempo $\tau_{2}$.

Sea $n\in\nat$, $n>0$, ciclo del servidor en que regresa a $Q_{1}$ para dar servicio y atender conforme a la pol\'itica exhaustiva, entonces se tiene que $\overline{\tau}_{1}\left(n\right)$ es el tiempo del servidor en el sistema 1 en que termina de dar servicio a todos los usuarios presentes en la cola, por lo tanto se cumple que $L_{1}\left(\overline{\tau}_{1}\left(n\right)\right)=0$, entonces el servidor para llegar a $Q_{2}$ incurre en un tiempo de traslado $r_{1}$ y por tanto se cumple que $\tau_{2}\left(n\right)=\overline{\tau}_{1}\left(n\right)+r_{1}$. Dado que los tiempos entre arribos son exponenciales se cumple que 

\begin{eqnarray*}
\prob\left\{0 \textrm{ arribos en }Q_{1}\textrm{ en el intervalo }\left[\overline{\tau}_{1}\left(n\right),\overline{\tau}_{1}\left(n\right)+r_{1}\right]\right\}=e^{-\tilde{\mu}_{1}r_{1}},\\
\prob\left\{0 \textrm{ arribos en }Q_{2}\textrm{ en el intervalo }\left[\overline{\tau}_{1}\left(n\right),\overline{\tau}_{1}\left(n\right)+r_{1}\right]\right\}=e^{-\tilde{\mu}_{2}r_{1}}.
\end{eqnarray*}

El evento que nos interesa consiste en que no haya arribos desde que el servidor abandon\'o $Q_{2}$ y regresa nuevamente para dar servicio, es decir en el intervalo de tiempo $\left[\overline{\tau}_{2}\left(n-1\right),\tau_{2}\left(n\right)\right]$. Entonces, si hacemos


\begin{eqnarray*}
\varphi_{1}\left(n\right)&\equiv&\overline{\tau}_{1}\left(n\right)+r_{1}=\overline{\tau}_{2}\left(n-1\right)+r_{1}+r_{2}+\overline{\tau}_{1}\left(n\right)-\tau_{1}\left(n\right)\\
&=&\overline{\tau}_{2}\left(n-1\right)+\overline{\tau}_{1}\left(n\right)-\tau_{1}\left(n\right)+r,,
\end{eqnarray*}

y longitud del intervalo

\begin{eqnarray*}
\xi&\equiv&\overline{\tau}_{1}\left(n\right)+r_{1}-\overline{\tau}_{2}\left(n-1\right)
=\overline{\tau}_{2}\left(n-1\right)+\overline{\tau}_{1}\left(n\right)-\tau_{1}\left(n\right)+r-\overline{\tau}_{2}\left(n-1\right)\\
&=&\overline{\tau}_{1}\left(n\right)-\tau_{1}\left(n\right)+r.
\end{eqnarray*}


Entonces, determinemos la probabilidad del evento no arribos a $Q_{2}$ en $\left[\overline{\tau}_{2}\left(n-1\right),\varphi_{1}\left(n\right)\right]$:

\begin{eqnarray}
\prob\left\{0 \textrm{ arribos en }Q_{2}\textrm{ en el intervalo }\left[\overline{\tau}_{2}\left(n-1\right),\varphi_{1}\left(n\right)\right]\right\}
=e^{-\tilde{\mu}_{2}\xi}.
\end{eqnarray}

De manera an\'aloga, tenemos que la probabilidad de no arribos a $Q_{1}$ en $\left[\overline{\tau}_{2}\left(n-1\right),\varphi_{1}\left(n\right)\right]$ esta dada por

\begin{eqnarray}
\prob\left\{0 \textrm{ arribos en }Q_{1}\textrm{ en el intervalo }\left[\overline{\tau}_{2}\left(n-1\right),\varphi_{1}\left(n\right)\right]\right\}
=e^{-\tilde{\mu}_{1}\xi},
\end{eqnarray}

\begin{eqnarray}
\prob\left\{0 \textrm{ arribos en }Q_{2}\textrm{ en el intervalo }\left[\overline{\tau}_{2}\left(n-1\right),\varphi_{1}\left(n\right)\right]\right\}
=e^{-\tilde{\mu}_{2}\xi}.
\end{eqnarray}

Por tanto 

\begin{eqnarray}
\begin{array}{l}
\prob\left\{0 \textrm{ arribos en }Q_{1}\textrm{ y }Q_{2}\textrm{ en el intervalo }\left[\overline{\tau}_{2}\left(n-1\right),\varphi_{1}\left(n\right)\right]\right\}\\
=\prob\left\{0 \textrm{ arribos en }Q_{1}\textrm{ en el intervalo }\left[\overline{\tau}_{2}\left(n-1\right),\varphi_{1}\left(n\right)\right]\right\}\\
\times
\prob\left\{0 \textrm{ arribos en }Q_{2}\textrm{ en el intervalo }\left[\overline{\tau}_{2}\left(n-1\right),\varphi_{1}\left(n\right)\right]\right\}=e^{-\tilde{\mu}_{1}\xi}e^{-\tilde{\mu}_{2}\xi}
=e^{-\tilde{\mu}\xi}.
\end{array}
\end{eqnarray}

Para el segundo sistema, consideremos nuevamente $\overline{\tau}_{1}\left(n\right)+r_{1}$, sin p\'erdida de generalidad podemos suponer que existe $m>0$ tal que $\overline{\tau}_{3}\left(m\right)<\overline{\tau}_{1}+r_{1}<\tau_{4}\left(m\right)$, entonces

\begin{eqnarray}
\prob\left\{0 \textrm{ arribos en }Q_{3}\textrm{ en el intervalo }\left[\overline{\tau}_{3}\left(m\right),\overline{\tau}_{1}\left(n\right)+r_{1}\right]\right\}
=e^{-\tilde{\mu}_{3}\xi_{3}},
\end{eqnarray}
donde 
\begin{eqnarray}
\xi_{3}=\overline{\tau}_{1}\left(n\right)+r_{1}-\overline{\tau}_{3}\left(m\right)=
\overline{\tau}_{1}\left(n\right)-\overline{\tau}_{3}\left(m\right)+r_{1},
\end{eqnarray}

mientras que para $Q_{4}$ al igual que con $Q_{2}$ escribiremos $\tau_{4}\left(m\right)$ en t\'erminos de $\overline{\tau}_{4}\left(m-1\right)$:

$\varphi_{2}\equiv\tau_{4}\left(m\right)=\overline{\tau}_{4}\left(m-1\right)+r_{4}+\overline{\tau}_{3}\left(m\right)
-\tau_{3}\left(m\right)+r_{3}=\overline{\tau}_{4}\left(m-1\right)+\overline{\tau}_{3}\left(m\right)
-\tau_{3}\left(m\right)+\hat{r}$, adem\'as,

$\xi_{2}\equiv\varphi_{2}\left(m\right)-\overline{\tau}_{1}-r_{1}=\overline{\tau}_{4}\left(m-1\right)+\overline{\tau}_{3}\left(m\right)
-\tau_{3}\left(m\right)-\overline{\tau}_{1}\left(n\right)+\hat{r}-r_{1}$. 

Entonces


\begin{eqnarray}
\prob\left\{0 \textrm{ arribos en }Q_{4}\textrm{ en el intervalo }\left[\overline{\tau}_{1}\left(n\right)+r_{1},\varphi_{2}\left(m\right)\right]\right\}
=e^{-\tilde{\mu}_{4}\xi_{2}},
\end{eqnarray}

mientras que para $Q_{3}$ se tiene que 

\begin{eqnarray}
\prob\left\{0 \textrm{ arribos en }Q_{3}\textrm{ en el intervalo }\left[\overline{\tau}_{1}\left(n\right)+r_{1},\varphi_{2}\left(m\right)\right]\right\}
=e^{-\tilde{\mu}_{3}\xi_{2}}
\end{eqnarray}

Por tanto

\begin{eqnarray}
\prob\left\{0 \textrm{ arribos en }Q_{3}\wedge Q_{4}\textrm{ en el intervalo }\left[\overline{\tau}_{1}\left(n\right)+r_{1},\varphi_{2}\left(m\right)\right]\right\}
=e^{-\hat{\mu}\xi_{2}}
\end{eqnarray}
donde $\hat{\mu}=\tilde{\mu}_{3}+\tilde{\mu}_{4}$.

Ahora, definamos los intervalos $\mathcal{I}_{1}=\left[\overline{\tau}_{1}\left(n\right)+r_{1},\varphi_{1}\left(n\right)\right]$  y $\mathcal{I}_{2}=\left[\overline{\tau}_{1}\left(n\right)+r_{1},\varphi_{2}\left(m\right)\right]$, entonces, sea $\mathcal{I}=\mathcal{I}_{1}\cap\mathcal{I}_{2}$ el intervalo donde cada una de las colas se encuentran vac\'ias, es decir, si tomamos $T^{*}\in\mathcal{I}$, entonces  $L_{1}\left(T^{*}\right)=L_{2}\left(T^{*}\right)=L_{3}\left(T^{*}\right)=L_{4}\left(T^{*}\right)=0$.

Ahora, dado que por construcci\'on $\mathcal{I}\neq\emptyset$ y que para $T^{*}\in\mathcal{I}$ en ninguna de las colas han llegado usuarios, se tiene que no hay transferencia entre las colas, por lo tanto, el sistema 1 y el sistema 2 son condicionalmente independientes en $\mathcal{I}$, es decir

\begin{eqnarray}
\prob\left\{L_{1}\left(T^{*}\right),L_{2}\left(T^{*}\right),L_{3}\left(T^{*}\right),L_{4}\left(T^{*}\right)|T^{*}\in\mathcal{I}\right\}=\prod_{j=1}^{4}\prob\left\{L_{j}\left(T^{*}\right)\right\},
\end{eqnarray}

para $T^{*}\in\mathcal{I}$. 

%\newpage























%________________________________________________________________________
%\section{Procesos Regenerativos}
%________________________________________________________________________

%________________________________________________________________________
%\subsection*{Procesos Regenerativos Sigman, Thorisson y Wolff \cite{Sigman1}}
%________________________________________________________________________


\begin{Def}[Definici\'on Cl\'asica]
Un proceso estoc\'astico $X=\left\{X\left(t\right):t\geq0\right\}$ es llamado regenerativo is existe una variable aleatoria $R_{1}>0$ tal que
\begin{itemize}
\item[i)] $\left\{X\left(t+R_{1}\right):t\geq0\right\}$ es independiente de $\left\{\left\{X\left(t\right):t<R_{1}\right\},\right\}$
\item[ii)] $\left\{X\left(t+R_{1}\right):t\geq0\right\}$ es estoc\'asticamente equivalente a $\left\{X\left(t\right):t>0\right\}$
\end{itemize}

Llamamos a $R_{1}$ tiempo de regeneraci\'on, y decimos que $X$ se regenera en este punto.
\end{Def}

$\left\{X\left(t+R_{1}\right)\right\}$ es regenerativo con tiempo de regeneraci\'on $R_{2}$, independiente de $R_{1}$ pero con la misma distribuci\'on que $R_{1}$. Procediendo de esta manera se obtiene una secuencia de variables aleatorias independientes e id\'enticamente distribuidas $\left\{R_{n}\right\}$ llamados longitudes de ciclo. Si definimos a $Z_{k}\equiv R_{1}+R_{2}+\cdots+R_{k}$, se tiene un proceso de renovaci\'on llamado proceso de renovaci\'on encajado para $X$.


\begin{Note}
La existencia de un primer tiempo de regeneraci\'on, $R_{1}$, implica la existencia de una sucesi\'on completa de estos tiempos $R_{1},R_{2}\ldots,$ que satisfacen la propiedad deseada \cite{Sigman2}.
\end{Note}


\begin{Note} Para la cola $GI/GI/1$ los usuarios arriban con tiempos $t_{n}$ y son atendidos con tiempos de servicio $S_{n}$, los tiempos de arribo forman un proceso de renovaci\'on  con tiempos entre arribos independientes e identicamente distribuidos (\texttt{i.i.d.})$T_{n}=t_{n}-t_{n-1}$, adem\'as los tiempos de servicio son \texttt{i.i.d.} e independientes de los procesos de arribo. Por \textit{estable} se entiende que $\esp S_{n}<\esp T_{n}<\infty$.
\end{Note}
 


\begin{Def}
Para $x$ fijo y para cada $t\geq0$, sea $I_{x}\left(t\right)=1$ si $X\left(t\right)\leq x$,  $I_{x}\left(t\right)=0$ en caso contrario, y def\'inanse los tiempos promedio
\begin{eqnarray*}
\overline{X}&=&lim_{t\rightarrow\infty}\frac{1}{t}\int_{0}^{\infty}X\left(u\right)du\\
\prob\left(X_{\infty}\leq x\right)&=&lim_{t\rightarrow\infty}\frac{1}{t}\int_{0}^{\infty}I_{x}\left(u\right)du,
\end{eqnarray*}
cuando estos l\'imites existan.
\end{Def}

Como consecuencia del teorema de Renovaci\'on-Recompensa, se tiene que el primer l\'imite  existe y es igual a la constante
\begin{eqnarray*}
\overline{X}&=&\frac{\esp\left[\int_{0}^{R_{1}}X\left(t\right)dt\right]}{\esp\left[R_{1}\right]},
\end{eqnarray*}
suponiendo que ambas esperanzas son finitas.
 
\begin{Note}
Funciones de procesos regenerativos son regenerativas, es decir, si $X\left(t\right)$ es regenerativo y se define el proceso $Y\left(t\right)$ por $Y\left(t\right)=f\left(X\left(t\right)\right)$ para alguna funci\'on Borel medible $f\left(\cdot\right)$. Adem\'as $Y$ es regenerativo con los mismos tiempos de renovaci\'on que $X$. 

En general, los tiempos de renovaci\'on, $Z_{k}$ de un proceso regenerativo no requieren ser tiempos de paro con respecto a la evoluci\'on de $X\left(t\right)$.
\end{Note} 

\begin{Note}
Una funci\'on de un proceso de Markov, usualmente no ser\'a un proceso de Markov, sin embargo ser\'a regenerativo si el proceso de Markov lo es.
\end{Note}

 
\begin{Note}
Un proceso regenerativo con media de la longitud de ciclo finita es llamado positivo recurrente.
\end{Note}


\begin{Note}
\begin{itemize}
\item[a)] Si el proceso regenerativo $X$ es positivo recurrente y tiene trayectorias muestrales no negativas, entonces la ecuaci\'on anterior es v\'alida.
\item[b)] Si $X$ es positivo recurrente regenerativo, podemos construir una \'unica versi\'on estacionaria de este proceso, $X_{e}=\left\{X_{e}\left(t\right)\right\}$, donde $X_{e}$ es un proceso estoc\'astico regenerativo y estrictamente estacionario, con distribuci\'on marginal distribuida como $X_{\infty}$
\end{itemize}
\end{Note}


%__________________________________________________________________________________________
%\subsection*{Procesos Regenerativos Estacionarios - Stidham \cite{Stidham}}
%__________________________________________________________________________________________


Un proceso estoc\'astico a tiempo continuo $\left\{V\left(t\right),t\geq0\right\}$ es un proceso regenerativo si existe una sucesi\'on de variables aleatorias independientes e id\'enticamente distribuidas $\left\{X_{1},X_{2},\ldots\right\}$, sucesi\'on de renovaci\'on, tal que para cualquier conjunto de Borel $A$, 

\begin{eqnarray*}
\prob\left\{V\left(t\right)\in A|X_{1}+X_{2}+\cdots+X_{R\left(t\right)}=s,\left\{V\left(\tau\right),\tau<s\right\}\right\}=\prob\left\{V\left(t-s\right)\in A|X_{1}>t-s\right\},
\end{eqnarray*}
para todo $0\leq s\leq t$, donde $R\left(t\right)=\max\left\{X_{1}+X_{2}+\cdots+X_{j}\leq t\right\}=$n\'umero de renovaciones ({\emph{puntos de regeneraci\'on}}) que ocurren en $\left[0,t\right]$. El intervalo $\left[0,X_{1}\right)$ es llamado {\emph{primer ciclo de regeneraci\'on}} de $\left\{V\left(t \right),t\geq0\right\}$, $\left[X_{1},X_{1}+X_{2}\right)$ el {\emph{segundo ciclo de regeneraci\'on}}, y as\'i sucesivamente.

Sea $X=X_{1}$ y sea $F$ la funci\'on de distrbuci\'on de $X$


\begin{Def}
Se define el proceso estacionario, $\left\{V^{*}\left(t\right),t\geq0\right\}$, para $\left\{V\left(t\right),t\geq0\right\}$ por

\begin{eqnarray*}
\prob\left\{V\left(t\right)\in A\right\}=\frac{1}{\esp\left[X\right]}\int_{0}^{\infty}\prob\left\{V\left(t+x\right)\in A|X>x\right\}\left(1-F\left(x\right)\right)dx,
\end{eqnarray*} 
para todo $t\geq0$ y todo conjunto de Borel $A$.
\end{Def}

\begin{Def}
Una distribuci\'on se dice que es {\emph{aritm\'etica}} si todos sus puntos de incremento son m\'ultiplos de la forma $0,\lambda, 2\lambda,\ldots$ para alguna $\lambda>0$ entera.
\end{Def}


\begin{Def}
Una modificaci\'on medible de un proceso $\left\{V\left(t\right),t\geq0\right\}$, es una versi\'on de este, $\left\{V\left(t,w\right)\right\}$ conjuntamente medible para $t\geq0$ y para $w\in S$, $S$ espacio de estados para $\left\{V\left(t\right),t\geq0\right\}$.
\end{Def}

\begin{Teo}
Sea $\left\{V\left(t\right),t\geq\right\}$ un proceso regenerativo no negativo con modificaci\'on medible. Sea $\esp\left[X\right]<\infty$. Entonces el proceso estacionario dado por la ecuaci\'on anterior est\'a bien definido y tiene funci\'on de distribuci\'on independiente de $t$, adem\'as
\begin{itemize}
\item[i)] \begin{eqnarray*}
\esp\left[V^{*}\left(0\right)\right]&=&\frac{\esp\left[\int_{0}^{X}V\left(s\right)ds\right]}{\esp\left[X\right]}\end{eqnarray*}
\item[ii)] Si $\esp\left[V^{*}\left(0\right)\right]<\infty$, equivalentemente, si $\esp\left[\int_{0}^{X}V\left(s\right)ds\right]<\infty$,entonces
\begin{eqnarray*}
\frac{\int_{0}^{t}V\left(s\right)ds}{t}\rightarrow\frac{\esp\left[\int_{0}^{X}V\left(s\right)ds\right]}{\esp\left[X\right]}
\end{eqnarray*}
con probabilidad 1 y en media, cuando $t\rightarrow\infty$.
\end{itemize}
\end{Teo}

\begin{Coro}
Sea $\left\{V\left(t\right),t\geq0\right\}$ un proceso regenerativo no negativo, con modificaci\'on medible. Si $\esp <\infty$, $F$ es no-aritm\'etica, y para todo $x\geq0$, $P\left\{V\left(t\right)\leq x,C>x\right\}$ es de variaci\'on acotada como funci\'on de $t$ en cada intervalo finito $\left[0,\tau\right]$, entonces $V\left(t\right)$ converge en distribuci\'on  cuando $t\rightarrow\infty$ y $$\esp V=\frac{\esp \int_{0}^{X}V\left(s\right)ds}{\esp X}$$
Donde $V$ tiene la distribuci\'on l\'imite de $V\left(t\right)$ cuando $t\rightarrow\infty$.

\end{Coro}

Para el caso discreto se tienen resultados similares.



%______________________________________________________________________
%\section{Procesos de Renovaci\'on}
%______________________________________________________________________

\begin{Def}\label{Def.Tn}
Sean $0\leq T_{1}\leq T_{2}\leq \ldots$ son tiempos aleatorios infinitos en los cuales ocurren ciertos eventos. El n\'umero de tiempos $T_{n}$ en el intervalo $\left[0,t\right)$ es

\begin{eqnarray}
N\left(t\right)=\sum_{n=1}^{\infty}\indora\left(T_{n}\leq t\right),
\end{eqnarray}
para $t\geq0$.
\end{Def}

Si se consideran los puntos $T_{n}$ como elementos de $\rea_{+}$, y $N\left(t\right)$ es el n\'umero de puntos en $\rea$. El proceso denotado por $\left\{N\left(t\right):t\geq0\right\}$, denotado por $N\left(t\right)$, es un proceso puntual en $\rea_{+}$. Los $T_{n}$ son los tiempos de ocurrencia, el proceso puntual $N\left(t\right)$ es simple si su n\'umero de ocurrencias son distintas: $0<T_{1}<T_{2}<\ldots$ casi seguramente.

\begin{Def}
Un proceso puntual $N\left(t\right)$ es un proceso de renovaci\'on si los tiempos de interocurrencia $\xi_{n}=T_{n}-T_{n-1}$, para $n\geq1$, son independientes e identicamente distribuidos con distribuci\'on $F$, donde $F\left(0\right)=0$ y $T_{0}=0$. Los $T_{n}$ son llamados tiempos de renovaci\'on, referente a la independencia o renovaci\'on de la informaci\'on estoc\'astica en estos tiempos. Los $\xi_{n}$ son los tiempos de inter-renovaci\'on, y $N\left(t\right)$ es el n\'umero de renovaciones en el intervalo $\left[0,t\right)$
\end{Def}


\begin{Note}
Para definir un proceso de renovaci\'on para cualquier contexto, solamente hay que especificar una distribuci\'on $F$, con $F\left(0\right)=0$, para los tiempos de inter-renovaci\'on. La funci\'on $F$ en turno degune las otra variables aleatorias. De manera formal, existe un espacio de probabilidad y una sucesi\'on de variables aleatorias $\xi_{1},\xi_{2},\ldots$ definidas en este con distribuci\'on $F$. Entonces las otras cantidades son $T_{n}=\sum_{k=1}^{n}\xi_{k}$ y $N\left(t\right)=\sum_{n=1}^{\infty}\indora\left(T_{n}\leq t\right)$, donde $T_{n}\rightarrow\infty$ casi seguramente por la Ley Fuerte de los Grandes Números.
\end{Note}

%___________________________________________________________________________________________
%
%\subsection*{Teorema Principal de Renovaci\'on}
%___________________________________________________________________________________________
%

\begin{Note} Una funci\'on $h:\rea_{+}\rightarrow\rea$ es Directamente Riemann Integrable en los siguientes casos:
\begin{itemize}
\item[a)] $h\left(t\right)\geq0$ es decreciente y Riemann Integrable.
\item[b)] $h$ es continua excepto posiblemente en un conjunto de Lebesgue de medida 0, y $|h\left(t\right)|\leq b\left(t\right)$, donde $b$ es DRI.
\end{itemize}
\end{Note}

\begin{Teo}[Teorema Principal de Renovaci\'on]
Si $F$ es no aritm\'etica y $h\left(t\right)$ es Directamente Riemann Integrable (DRI), entonces

\begin{eqnarray*}
lim_{t\rightarrow\infty}U\star h=\frac{1}{\mu}\int_{\rea_{+}}h\left(s\right)ds.
\end{eqnarray*}
\end{Teo}

\begin{Prop}
Cualquier funci\'on $H\left(t\right)$ acotada en intervalos finitos y que es 0 para $t<0$ puede expresarse como
\begin{eqnarray*}
H\left(t\right)=U\star h\left(t\right)\textrm{,  donde }h\left(t\right)=H\left(t\right)-F\star H\left(t\right)
\end{eqnarray*}
\end{Prop}

\begin{Def}
Un proceso estoc\'astico $X\left(t\right)$ es crudamente regenerativo en un tiempo aleatorio positivo $T$ si
\begin{eqnarray*}
\esp\left[X\left(T+t\right)|T\right]=\esp\left[X\left(t\right)\right]\textrm{, para }t\geq0,\end{eqnarray*}
y con las esperanzas anteriores finitas.
\end{Def}

\begin{Prop}
Sup\'ongase que $X\left(t\right)$ es un proceso crudamente regenerativo en $T$, que tiene distribuci\'on $F$. Si $\esp\left[X\left(t\right)\right]$ es acotado en intervalos finitos, entonces
\begin{eqnarray*}
\esp\left[X\left(t\right)\right]=U\star h\left(t\right)\textrm{,  donde }h\left(t\right)=\esp\left[X\left(t\right)\indora\left(T>t\right)\right].
\end{eqnarray*}
\end{Prop}

\begin{Teo}[Regeneraci\'on Cruda]
Sup\'ongase que $X\left(t\right)$ es un proceso con valores positivo crudamente regenerativo en $T$, y def\'inase $M=\sup\left\{|X\left(t\right)|:t\leq T\right\}$. Si $T$ es no aritm\'etico y $M$ y $MT$ tienen media finita, entonces
\begin{eqnarray*}
lim_{t\rightarrow\infty}\esp\left[X\left(t\right)\right]=\frac{1}{\mu}\int_{\rea_{+}}h\left(s\right)ds,
\end{eqnarray*}
donde $h\left(t\right)=\esp\left[X\left(t\right)\indora\left(T>t\right)\right]$.
\end{Teo}

%___________________________________________________________________________________________
%
%\subsection*{Propiedades de los Procesos de Renovaci\'on}
%___________________________________________________________________________________________
%

Los tiempos $T_{n}$ est\'an relacionados con los conteos de $N\left(t\right)$ por

\begin{eqnarray*}
\left\{N\left(t\right)\geq n\right\}&=&\left\{T_{n}\leq t\right\}\\
T_{N\left(t\right)}\leq &t&<T_{N\left(t\right)+1},
\end{eqnarray*}

adem\'as $N\left(T_{n}\right)=n$, y 

\begin{eqnarray*}
N\left(t\right)=\max\left\{n:T_{n}\leq t\right\}=\min\left\{n:T_{n+1}>t\right\}
\end{eqnarray*}

Por propiedades de la convoluci\'on se sabe que

\begin{eqnarray*}
P\left\{T_{n}\leq t\right\}=F^{n\star}\left(t\right)
\end{eqnarray*}
que es la $n$-\'esima convoluci\'on de $F$. Entonces 

\begin{eqnarray*}
\left\{N\left(t\right)\geq n\right\}&=&\left\{T_{n}\leq t\right\}\\
P\left\{N\left(t\right)\leq n\right\}&=&1-F^{\left(n+1\right)\star}\left(t\right)
\end{eqnarray*}

Adem\'as usando el hecho de que $\esp\left[N\left(t\right)\right]=\sum_{n=1}^{\infty}P\left\{N\left(t\right)\geq n\right\}$
se tiene que

\begin{eqnarray*}
\esp\left[N\left(t\right)\right]=\sum_{n=1}^{\infty}F^{n\star}\left(t\right)
\end{eqnarray*}

\begin{Prop}
Para cada $t\geq0$, la funci\'on generadora de momentos $\esp\left[e^{\alpha N\left(t\right)}\right]$ existe para alguna $\alpha$ en una vecindad del 0, y de aqu\'i que $\esp\left[N\left(t\right)^{m}\right]<\infty$, para $m\geq1$.
\end{Prop}


\begin{Note}
Si el primer tiempo de renovaci\'on $\xi_{1}$ no tiene la misma distribuci\'on que el resto de las $\xi_{n}$, para $n\geq2$, a $N\left(t\right)$ se le llama Proceso de Renovaci\'on retardado, donde si $\xi$ tiene distribuci\'on $G$, entonces el tiempo $T_{n}$ de la $n$-\'esima renovaci\'on tiene distribuci\'on $G\star F^{\left(n-1\right)\star}\left(t\right)$
\end{Note}


\begin{Teo}
Para una constante $\mu\leq\infty$ ( o variable aleatoria), las siguientes expresiones son equivalentes:

\begin{eqnarray}
lim_{n\rightarrow\infty}n^{-1}T_{n}&=&\mu,\textrm{ c.s.}\\
lim_{t\rightarrow\infty}t^{-1}N\left(t\right)&=&1/\mu,\textrm{ c.s.}
\end{eqnarray}
\end{Teo}


Es decir, $T_{n}$ satisface la Ley Fuerte de los Grandes N\'umeros s\'i y s\'olo s\'i $N\left/t\right)$ la cumple.


\begin{Coro}[Ley Fuerte de los Grandes N\'umeros para Procesos de Renovaci\'on]
Si $N\left(t\right)$ es un proceso de renovaci\'on cuyos tiempos de inter-renovaci\'on tienen media $\mu\leq\infty$, entonces
\begin{eqnarray}
t^{-1}N\left(t\right)\rightarrow 1/\mu,\textrm{ c.s. cuando }t\rightarrow\infty.
\end{eqnarray}

\end{Coro}


Considerar el proceso estoc\'astico de valores reales $\left\{Z\left(t\right):t\geq0\right\}$ en el mismo espacio de probabilidad que $N\left(t\right)$

\begin{Def}
Para el proceso $\left\{Z\left(t\right):t\geq0\right\}$ se define la fluctuaci\'on m\'axima de $Z\left(t\right)$ en el intervalo $\left(T_{n-1},T_{n}\right]$:
\begin{eqnarray*}
M_{n}=\sup_{T_{n-1}<t\leq T_{n}}|Z\left(t\right)-Z\left(T_{n-1}\right)|
\end{eqnarray*}
\end{Def}

\begin{Teo}
Sup\'ongase que $n^{-1}T_{n}\rightarrow\mu$ c.s. cuando $n\rightarrow\infty$, donde $\mu\leq\infty$ es una constante o variable aleatoria. Sea $a$ una constante o variable aleatoria que puede ser infinita cuando $\mu$ es finita, y considere las expresiones l\'imite:
\begin{eqnarray}
lim_{n\rightarrow\infty}n^{-1}Z\left(T_{n}\right)&=&a,\textrm{ c.s.}\\
lim_{t\rightarrow\infty}t^{-1}Z\left(t\right)&=&a/\mu,\textrm{ c.s.}
\end{eqnarray}
La segunda expresi\'on implica la primera. Conversamente, la primera implica la segunda si el proceso $Z\left(t\right)$ es creciente, o si $lim_{n\rightarrow\infty}n^{-1}M_{n}=0$ c.s.
\end{Teo}

\begin{Coro}
Si $N\left(t\right)$ es un proceso de renovaci\'on, y $\left(Z\left(T_{n}\right)-Z\left(T_{n-1}\right),M_{n}\right)$, para $n\geq1$, son variables aleatorias independientes e id\'enticamente distribuidas con media finita, entonces,
\begin{eqnarray}
lim_{t\rightarrow\infty}t^{-1}Z\left(t\right)\rightarrow\frac{\esp\left[Z\left(T_{1}\right)-Z\left(T_{0}\right)\right]}{\esp\left[T_{1}\right]},\textrm{ c.s. cuando  }t\rightarrow\infty.
\end{eqnarray}
\end{Coro}



%___________________________________________________________________________________________
%
%\subsection{Propiedades de los Procesos de Renovaci\'on}
%___________________________________________________________________________________________
%

Los tiempos $T_{n}$ est\'an relacionados con los conteos de $N\left(t\right)$ por

\begin{eqnarray*}
\left\{N\left(t\right)\geq n\right\}&=&\left\{T_{n}\leq t\right\}\\
T_{N\left(t\right)}\leq &t&<T_{N\left(t\right)+1},
\end{eqnarray*}

adem\'as $N\left(T_{n}\right)=n$, y 

\begin{eqnarray*}
N\left(t\right)=\max\left\{n:T_{n}\leq t\right\}=\min\left\{n:T_{n+1}>t\right\}
\end{eqnarray*}

Por propiedades de la convoluci\'on se sabe que

\begin{eqnarray*}
P\left\{T_{n}\leq t\right\}=F^{n\star}\left(t\right)
\end{eqnarray*}
que es la $n$-\'esima convoluci\'on de $F$. Entonces 

\begin{eqnarray*}
\left\{N\left(t\right)\geq n\right\}&=&\left\{T_{n}\leq t\right\}\\
P\left\{N\left(t\right)\leq n\right\}&=&1-F^{\left(n+1\right)\star}\left(t\right)
\end{eqnarray*}

Adem\'as usando el hecho de que $\esp\left[N\left(t\right)\right]=\sum_{n=1}^{\infty}P\left\{N\left(t\right)\geq n\right\}$
se tiene que

\begin{eqnarray*}
\esp\left[N\left(t\right)\right]=\sum_{n=1}^{\infty}F^{n\star}\left(t\right)
\end{eqnarray*}

\begin{Prop}
Para cada $t\geq0$, la funci\'on generadora de momentos $\esp\left[e^{\alpha N\left(t\right)}\right]$ existe para alguna $\alpha$ en una vecindad del 0, y de aqu\'i que $\esp\left[N\left(t\right)^{m}\right]<\infty$, para $m\geq1$.
\end{Prop}


\begin{Note}
Si el primer tiempo de renovaci\'on $\xi_{1}$ no tiene la misma distribuci\'on que el resto de las $\xi_{n}$, para $n\geq2$, a $N\left(t\right)$ se le llama Proceso de Renovaci\'on retardado, donde si $\xi$ tiene distribuci\'on $G$, entonces el tiempo $T_{n}$ de la $n$-\'esima renovaci\'on tiene distribuci\'on $G\star F^{\left(n-1\right)\star}\left(t\right)$
\end{Note}


\begin{Teo}
Para una constante $\mu\leq\infty$ ( o variable aleatoria), las siguientes expresiones son equivalentes:

\begin{eqnarray}
lim_{n\rightarrow\infty}n^{-1}T_{n}&=&\mu,\textrm{ c.s.}\\
lim_{t\rightarrow\infty}t^{-1}N\left(t\right)&=&1/\mu,\textrm{ c.s.}
\end{eqnarray}
\end{Teo}


Es decir, $T_{n}$ satisface la Ley Fuerte de los Grandes N\'umeros s\'i y s\'olo s\'i $N\left/t\right)$ la cumple.


\begin{Coro}[Ley Fuerte de los Grandes N\'umeros para Procesos de Renovaci\'on]
Si $N\left(t\right)$ es un proceso de renovaci\'on cuyos tiempos de inter-renovaci\'on tienen media $\mu\leq\infty$, entonces
\begin{eqnarray}
t^{-1}N\left(t\right)\rightarrow 1/\mu,\textrm{ c.s. cuando }t\rightarrow\infty.
\end{eqnarray}

\end{Coro}


Considerar el proceso estoc\'astico de valores reales $\left\{Z\left(t\right):t\geq0\right\}$ en el mismo espacio de probabilidad que $N\left(t\right)$

\begin{Def}
Para el proceso $\left\{Z\left(t\right):t\geq0\right\}$ se define la fluctuaci\'on m\'axima de $Z\left(t\right)$ en el intervalo $\left(T_{n-1},T_{n}\right]$:
\begin{eqnarray*}
M_{n}=\sup_{T_{n-1}<t\leq T_{n}}|Z\left(t\right)-Z\left(T_{n-1}\right)|
\end{eqnarray*}
\end{Def}

\begin{Teo}
Sup\'ongase que $n^{-1}T_{n}\rightarrow\mu$ c.s. cuando $n\rightarrow\infty$, donde $\mu\leq\infty$ es una constante o variable aleatoria. Sea $a$ una constante o variable aleatoria que puede ser infinita cuando $\mu$ es finita, y considere las expresiones l\'imite:
\begin{eqnarray}
lim_{n\rightarrow\infty}n^{-1}Z\left(T_{n}\right)&=&a,\textrm{ c.s.}\\
lim_{t\rightarrow\infty}t^{-1}Z\left(t\right)&=&a/\mu,\textrm{ c.s.}
\end{eqnarray}
La segunda expresi\'on implica la primera. Conversamente, la primera implica la segunda si el proceso $Z\left(t\right)$ es creciente, o si $lim_{n\rightarrow\infty}n^{-1}M_{n}=0$ c.s.
\end{Teo}

\begin{Coro}
Si $N\left(t\right)$ es un proceso de renovaci\'on, y $\left(Z\left(T_{n}\right)-Z\left(T_{n-1}\right),M_{n}\right)$, para $n\geq1$, son variables aleatorias independientes e id\'enticamente distribuidas con media finita, entonces,
\begin{eqnarray}
lim_{t\rightarrow\infty}t^{-1}Z\left(t\right)\rightarrow\frac{\esp\left[Z\left(T_{1}\right)-Z\left(T_{0}\right)\right]}{\esp\left[T_{1}\right]},\textrm{ c.s. cuando  }t\rightarrow\infty.
\end{eqnarray}
\end{Coro}


%___________________________________________________________________________________________
%
%\subsection{Propiedades de los Procesos de Renovaci\'on}
%___________________________________________________________________________________________
%

Los tiempos $T_{n}$ est\'an relacionados con los conteos de $N\left(t\right)$ por

\begin{eqnarray*}
\left\{N\left(t\right)\geq n\right\}&=&\left\{T_{n}\leq t\right\}\\
T_{N\left(t\right)}\leq &t&<T_{N\left(t\right)+1},
\end{eqnarray*}

adem\'as $N\left(T_{n}\right)=n$, y 

\begin{eqnarray*}
N\left(t\right)=\max\left\{n:T_{n}\leq t\right\}=\min\left\{n:T_{n+1}>t\right\}
\end{eqnarray*}

Por propiedades de la convoluci\'on se sabe que

\begin{eqnarray*}
P\left\{T_{n}\leq t\right\}=F^{n\star}\left(t\right)
\end{eqnarray*}
que es la $n$-\'esima convoluci\'on de $F$. Entonces 

\begin{eqnarray*}
\left\{N\left(t\right)\geq n\right\}&=&\left\{T_{n}\leq t\right\}\\
P\left\{N\left(t\right)\leq n\right\}&=&1-F^{\left(n+1\right)\star}\left(t\right)
\end{eqnarray*}

Adem\'as usando el hecho de que $\esp\left[N\left(t\right)\right]=\sum_{n=1}^{\infty}P\left\{N\left(t\right)\geq n\right\}$
se tiene que

\begin{eqnarray*}
\esp\left[N\left(t\right)\right]=\sum_{n=1}^{\infty}F^{n\star}\left(t\right)
\end{eqnarray*}

\begin{Prop}
Para cada $t\geq0$, la funci\'on generadora de momentos $\esp\left[e^{\alpha N\left(t\right)}\right]$ existe para alguna $\alpha$ en una vecindad del 0, y de aqu\'i que $\esp\left[N\left(t\right)^{m}\right]<\infty$, para $m\geq1$.
\end{Prop}


\begin{Note}
Si el primer tiempo de renovaci\'on $\xi_{1}$ no tiene la misma distribuci\'on que el resto de las $\xi_{n}$, para $n\geq2$, a $N\left(t\right)$ se le llama Proceso de Renovaci\'on retardado, donde si $\xi$ tiene distribuci\'on $G$, entonces el tiempo $T_{n}$ de la $n$-\'esima renovaci\'on tiene distribuci\'on $G\star F^{\left(n-1\right)\star}\left(t\right)$
\end{Note}


\begin{Teo}
Para una constante $\mu\leq\infty$ ( o variable aleatoria), las siguientes expresiones son equivalentes:

\begin{eqnarray}
lim_{n\rightarrow\infty}n^{-1}T_{n}&=&\mu,\textrm{ c.s.}\\
lim_{t\rightarrow\infty}t^{-1}N\left(t\right)&=&1/\mu,\textrm{ c.s.}
\end{eqnarray}
\end{Teo}


Es decir, $T_{n}$ satisface la Ley Fuerte de los Grandes N\'umeros s\'i y s\'olo s\'i $N\left/t\right)$ la cumple.


\begin{Coro}[Ley Fuerte de los Grandes N\'umeros para Procesos de Renovaci\'on]
Si $N\left(t\right)$ es un proceso de renovaci\'on cuyos tiempos de inter-renovaci\'on tienen media $\mu\leq\infty$, entonces
\begin{eqnarray}
t^{-1}N\left(t\right)\rightarrow 1/\mu,\textrm{ c.s. cuando }t\rightarrow\infty.
\end{eqnarray}

\end{Coro}


Considerar el proceso estoc\'astico de valores reales $\left\{Z\left(t\right):t\geq0\right\}$ en el mismo espacio de probabilidad que $N\left(t\right)$

\begin{Def}
Para el proceso $\left\{Z\left(t\right):t\geq0\right\}$ se define la fluctuaci\'on m\'axima de $Z\left(t\right)$ en el intervalo $\left(T_{n-1},T_{n}\right]$:
\begin{eqnarray*}
M_{n}=\sup_{T_{n-1}<t\leq T_{n}}|Z\left(t\right)-Z\left(T_{n-1}\right)|
\end{eqnarray*}
\end{Def}

\begin{Teo}
Sup\'ongase que $n^{-1}T_{n}\rightarrow\mu$ c.s. cuando $n\rightarrow\infty$, donde $\mu\leq\infty$ es una constante o variable aleatoria. Sea $a$ una constante o variable aleatoria que puede ser infinita cuando $\mu$ es finita, y considere las expresiones l\'imite:
\begin{eqnarray}
lim_{n\rightarrow\infty}n^{-1}Z\left(T_{n}\right)&=&a,\textrm{ c.s.}\\
lim_{t\rightarrow\infty}t^{-1}Z\left(t\right)&=&a/\mu,\textrm{ c.s.}
\end{eqnarray}
La segunda expresi\'on implica la primera. Conversamente, la primera implica la segunda si el proceso $Z\left(t\right)$ es creciente, o si $lim_{n\rightarrow\infty}n^{-1}M_{n}=0$ c.s.
\end{Teo}

\begin{Coro}
Si $N\left(t\right)$ es un proceso de renovaci\'on, y $\left(Z\left(T_{n}\right)-Z\left(T_{n-1}\right),M_{n}\right)$, para $n\geq1$, son variables aleatorias independientes e id\'enticamente distribuidas con media finita, entonces,
\begin{eqnarray}
lim_{t\rightarrow\infty}t^{-1}Z\left(t\right)\rightarrow\frac{\esp\left[Z\left(T_{1}\right)-Z\left(T_{0}\right)\right]}{\esp\left[T_{1}\right]},\textrm{ c.s. cuando  }t\rightarrow\infty.
\end{eqnarray}
\end{Coro}

%___________________________________________________________________________________________
%
%\subsection{Propiedades de los Procesos de Renovaci\'on}
%___________________________________________________________________________________________
%

Los tiempos $T_{n}$ est\'an relacionados con los conteos de $N\left(t\right)$ por

\begin{eqnarray*}
\left\{N\left(t\right)\geq n\right\}&=&\left\{T_{n}\leq t\right\}\\
T_{N\left(t\right)}\leq &t&<T_{N\left(t\right)+1},
\end{eqnarray*}

adem\'as $N\left(T_{n}\right)=n$, y 

\begin{eqnarray*}
N\left(t\right)=\max\left\{n:T_{n}\leq t\right\}=\min\left\{n:T_{n+1}>t\right\}
\end{eqnarray*}

Por propiedades de la convoluci\'on se sabe que

\begin{eqnarray*}
P\left\{T_{n}\leq t\right\}=F^{n\star}\left(t\right)
\end{eqnarray*}
que es la $n$-\'esima convoluci\'on de $F$. Entonces 

\begin{eqnarray*}
\left\{N\left(t\right)\geq n\right\}&=&\left\{T_{n}\leq t\right\}\\
P\left\{N\left(t\right)\leq n\right\}&=&1-F^{\left(n+1\right)\star}\left(t\right)
\end{eqnarray*}

Adem\'as usando el hecho de que $\esp\left[N\left(t\right)\right]=\sum_{n=1}^{\infty}P\left\{N\left(t\right)\geq n\right\}$
se tiene que

\begin{eqnarray*}
\esp\left[N\left(t\right)\right]=\sum_{n=1}^{\infty}F^{n\star}\left(t\right)
\end{eqnarray*}

\begin{Prop}
Para cada $t\geq0$, la funci\'on generadora de momentos $\esp\left[e^{\alpha N\left(t\right)}\right]$ existe para alguna $\alpha$ en una vecindad del 0, y de aqu\'i que $\esp\left[N\left(t\right)^{m}\right]<\infty$, para $m\geq1$.
\end{Prop}


\begin{Note}
Si el primer tiempo de renovaci\'on $\xi_{1}$ no tiene la misma distribuci\'on que el resto de las $\xi_{n}$, para $n\geq2$, a $N\left(t\right)$ se le llama Proceso de Renovaci\'on retardado, donde si $\xi$ tiene distribuci\'on $G$, entonces el tiempo $T_{n}$ de la $n$-\'esima renovaci\'on tiene distribuci\'on $G\star F^{\left(n-1\right)\star}\left(t\right)$
\end{Note}


\begin{Teo}
Para una constante $\mu\leq\infty$ ( o variable aleatoria), las siguientes expresiones son equivalentes:

\begin{eqnarray}
lim_{n\rightarrow\infty}n^{-1}T_{n}&=&\mu,\textrm{ c.s.}\\
lim_{t\rightarrow\infty}t^{-1}N\left(t\right)&=&1/\mu,\textrm{ c.s.}
\end{eqnarray}
\end{Teo}


Es decir, $T_{n}$ satisface la Ley Fuerte de los Grandes N\'umeros s\'i y s\'olo s\'i $N\left/t\right)$ la cumple.


\begin{Coro}[Ley Fuerte de los Grandes N\'umeros para Procesos de Renovaci\'on]
Si $N\left(t\right)$ es un proceso de renovaci\'on cuyos tiempos de inter-renovaci\'on tienen media $\mu\leq\infty$, entonces
\begin{eqnarray}
t^{-1}N\left(t\right)\rightarrow 1/\mu,\textrm{ c.s. cuando }t\rightarrow\infty.
\end{eqnarray}

\end{Coro}


Considerar el proceso estoc\'astico de valores reales $\left\{Z\left(t\right):t\geq0\right\}$ en el mismo espacio de probabilidad que $N\left(t\right)$

\begin{Def}
Para el proceso $\left\{Z\left(t\right):t\geq0\right\}$ se define la fluctuaci\'on m\'axima de $Z\left(t\right)$ en el intervalo $\left(T_{n-1},T_{n}\right]$:
\begin{eqnarray*}
M_{n}=\sup_{T_{n-1}<t\leq T_{n}}|Z\left(t\right)-Z\left(T_{n-1}\right)|
\end{eqnarray*}
\end{Def}

\begin{Teo}
Sup\'ongase que $n^{-1}T_{n}\rightarrow\mu$ c.s. cuando $n\rightarrow\infty$, donde $\mu\leq\infty$ es una constante o variable aleatoria. Sea $a$ una constante o variable aleatoria que puede ser infinita cuando $\mu$ es finita, y considere las expresiones l\'imite:
\begin{eqnarray}
lim_{n\rightarrow\infty}n^{-1}Z\left(T_{n}\right)&=&a,\textrm{ c.s.}\\
lim_{t\rightarrow\infty}t^{-1}Z\left(t\right)&=&a/\mu,\textrm{ c.s.}
\end{eqnarray}
La segunda expresi\'on implica la primera. Conversamente, la primera implica la segunda si el proceso $Z\left(t\right)$ es creciente, o si $lim_{n\rightarrow\infty}n^{-1}M_{n}=0$ c.s.
\end{Teo}

\begin{Coro}
Si $N\left(t\right)$ es un proceso de renovaci\'on, y $\left(Z\left(T_{n}\right)-Z\left(T_{n-1}\right),M_{n}\right)$, para $n\geq1$, son variables aleatorias independientes e id\'enticamente distribuidas con media finita, entonces,
\begin{eqnarray}
lim_{t\rightarrow\infty}t^{-1}Z\left(t\right)\rightarrow\frac{\esp\left[Z\left(T_{1}\right)-Z\left(T_{0}\right)\right]}{\esp\left[T_{1}\right]},\textrm{ c.s. cuando  }t\rightarrow\infty.
\end{eqnarray}
\end{Coro}
%___________________________________________________________________________________________
%
%\subsection{Propiedades de los Procesos de Renovaci\'on}
%___________________________________________________________________________________________
%

Los tiempos $T_{n}$ est\'an relacionados con los conteos de $N\left(t\right)$ por

\begin{eqnarray*}
\left\{N\left(t\right)\geq n\right\}&=&\left\{T_{n}\leq t\right\}\\
T_{N\left(t\right)}\leq &t&<T_{N\left(t\right)+1},
\end{eqnarray*}

adem\'as $N\left(T_{n}\right)=n$, y 

\begin{eqnarray*}
N\left(t\right)=\max\left\{n:T_{n}\leq t\right\}=\min\left\{n:T_{n+1}>t\right\}
\end{eqnarray*}

Por propiedades de la convoluci\'on se sabe que

\begin{eqnarray*}
P\left\{T_{n}\leq t\right\}=F^{n\star}\left(t\right)
\end{eqnarray*}
que es la $n$-\'esima convoluci\'on de $F$. Entonces 

\begin{eqnarray*}
\left\{N\left(t\right)\geq n\right\}&=&\left\{T_{n}\leq t\right\}\\
P\left\{N\left(t\right)\leq n\right\}&=&1-F^{\left(n+1\right)\star}\left(t\right)
\end{eqnarray*}

Adem\'as usando el hecho de que $\esp\left[N\left(t\right)\right]=\sum_{n=1}^{\infty}P\left\{N\left(t\right)\geq n\right\}$
se tiene que

\begin{eqnarray*}
\esp\left[N\left(t\right)\right]=\sum_{n=1}^{\infty}F^{n\star}\left(t\right)
\end{eqnarray*}

\begin{Prop}
Para cada $t\geq0$, la funci\'on generadora de momentos $\esp\left[e^{\alpha N\left(t\right)}\right]$ existe para alguna $\alpha$ en una vecindad del 0, y de aqu\'i que $\esp\left[N\left(t\right)^{m}\right]<\infty$, para $m\geq1$.
\end{Prop}


\begin{Note}
Si el primer tiempo de renovaci\'on $\xi_{1}$ no tiene la misma distribuci\'on que el resto de las $\xi_{n}$, para $n\geq2$, a $N\left(t\right)$ se le llama Proceso de Renovaci\'on retardado, donde si $\xi$ tiene distribuci\'on $G$, entonces el tiempo $T_{n}$ de la $n$-\'esima renovaci\'on tiene distribuci\'on $G\star F^{\left(n-1\right)\star}\left(t\right)$
\end{Note}


\begin{Teo}
Para una constante $\mu\leq\infty$ ( o variable aleatoria), las siguientes expresiones son equivalentes:

\begin{eqnarray}
lim_{n\rightarrow\infty}n^{-1}T_{n}&=&\mu,\textrm{ c.s.}\\
lim_{t\rightarrow\infty}t^{-1}N\left(t\right)&=&1/\mu,\textrm{ c.s.}
\end{eqnarray}
\end{Teo}


Es decir, $T_{n}$ satisface la Ley Fuerte de los Grandes N\'umeros s\'i y s\'olo s\'i $N\left/t\right)$ la cumple.


\begin{Coro}[Ley Fuerte de los Grandes N\'umeros para Procesos de Renovaci\'on]
Si $N\left(t\right)$ es un proceso de renovaci\'on cuyos tiempos de inter-renovaci\'on tienen media $\mu\leq\infty$, entonces
\begin{eqnarray}
t^{-1}N\left(t\right)\rightarrow 1/\mu,\textrm{ c.s. cuando }t\rightarrow\infty.
\end{eqnarray}

\end{Coro}


Considerar el proceso estoc\'astico de valores reales $\left\{Z\left(t\right):t\geq0\right\}$ en el mismo espacio de probabilidad que $N\left(t\right)$

\begin{Def}
Para el proceso $\left\{Z\left(t\right):t\geq0\right\}$ se define la fluctuaci\'on m\'axima de $Z\left(t\right)$ en el intervalo $\left(T_{n-1},T_{n}\right]$:
\begin{eqnarray*}
M_{n}=\sup_{T_{n-1}<t\leq T_{n}}|Z\left(t\right)-Z\left(T_{n-1}\right)|
\end{eqnarray*}
\end{Def}

\begin{Teo}
Sup\'ongase que $n^{-1}T_{n}\rightarrow\mu$ c.s. cuando $n\rightarrow\infty$, donde $\mu\leq\infty$ es una constante o variable aleatoria. Sea $a$ una constante o variable aleatoria que puede ser infinita cuando $\mu$ es finita, y considere las expresiones l\'imite:
\begin{eqnarray}
lim_{n\rightarrow\infty}n^{-1}Z\left(T_{n}\right)&=&a,\textrm{ c.s.}\\
lim_{t\rightarrow\infty}t^{-1}Z\left(t\right)&=&a/\mu,\textrm{ c.s.}
\end{eqnarray}
La segunda expresi\'on implica la primera. Conversamente, la primera implica la segunda si el proceso $Z\left(t\right)$ es creciente, o si $lim_{n\rightarrow\infty}n^{-1}M_{n}=0$ c.s.
\end{Teo}

\begin{Coro}
Si $N\left(t\right)$ es un proceso de renovaci\'on, y $\left(Z\left(T_{n}\right)-Z\left(T_{n-1}\right),M_{n}\right)$, para $n\geq1$, son variables aleatorias independientes e id\'enticamente distribuidas con media finita, entonces,
\begin{eqnarray}
lim_{t\rightarrow\infty}t^{-1}Z\left(t\right)\rightarrow\frac{\esp\left[Z\left(T_{1}\right)-Z\left(T_{0}\right)\right]}{\esp\left[T_{1}\right]},\textrm{ c.s. cuando  }t\rightarrow\infty.
\end{eqnarray}
\end{Coro}


%___________________________________________________________________________________________
%
%\subsection*{Funci\'on de Renovaci\'on}
%___________________________________________________________________________________________
%


\begin{Def}
Sea $h\left(t\right)$ funci\'on de valores reales en $\rea$ acotada en intervalos finitos e igual a cero para $t<0$ La ecuaci\'on de renovaci\'on para $h\left(t\right)$ y la distribuci\'on $F$ es

\begin{eqnarray}\label{Ec.Renovacion}
H\left(t\right)=h\left(t\right)+\int_{\left[0,t\right]}H\left(t-s\right)dF\left(s\right)\textrm{,    }t\geq0,
\end{eqnarray}
donde $H\left(t\right)$ es una funci\'on de valores reales. Esto es $H=h+F\star H$. Decimos que $H\left(t\right)$ es soluci\'on de esta ecuaci\'on si satisface la ecuaci\'on, y es acotada en intervalos finitos e iguales a cero para $t<0$.
\end{Def}

\begin{Prop}
La funci\'on $U\star h\left(t\right)$ es la \'unica soluci\'on de la ecuaci\'on de renovaci\'on (\ref{Ec.Renovacion}).
\end{Prop}

\begin{Teo}[Teorema Renovaci\'on Elemental]
\begin{eqnarray*}
t^{-1}U\left(t\right)\rightarrow 1/\mu\textrm{,    cuando }t\rightarrow\infty.
\end{eqnarray*}
\end{Teo}

%___________________________________________________________________________________________
%
%\subsection{Funci\'on de Renovaci\'on}
%___________________________________________________________________________________________
%


Sup\'ongase que $N\left(t\right)$ es un proceso de renovaci\'on con distribuci\'on $F$ con media finita $\mu$.

\begin{Def}
La funci\'on de renovaci\'on asociada con la distribuci\'on $F$, del proceso $N\left(t\right)$, es
\begin{eqnarray*}
U\left(t\right)=\sum_{n=1}^{\infty}F^{n\star}\left(t\right),\textrm{   }t\geq0,
\end{eqnarray*}
donde $F^{0\star}\left(t\right)=\indora\left(t\geq0\right)$.
\end{Def}


\begin{Prop}
Sup\'ongase que la distribuci\'on de inter-renovaci\'on $F$ tiene densidad $f$. Entonces $U\left(t\right)$ tambi\'en tiene densidad, para $t>0$, y es $U^{'}\left(t\right)=\sum_{n=0}^{\infty}f^{n\star}\left(t\right)$. Adem\'as
\begin{eqnarray*}
\prob\left\{N\left(t\right)>N\left(t-\right)\right\}=0\textrm{,   }t\geq0.
\end{eqnarray*}
\end{Prop}

\begin{Def}
La Transformada de Laplace-Stieljes de $F$ est\'a dada por

\begin{eqnarray*}
\hat{F}\left(\alpha\right)=\int_{\rea_{+}}e^{-\alpha t}dF\left(t\right)\textrm{,  }\alpha\geq0.
\end{eqnarray*}
\end{Def}

Entonces

\begin{eqnarray*}
\hat{U}\left(\alpha\right)=\sum_{n=0}^{\infty}\hat{F^{n\star}}\left(\alpha\right)=\sum_{n=0}^{\infty}\hat{F}\left(\alpha\right)^{n}=\frac{1}{1-\hat{F}\left(\alpha\right)}.
\end{eqnarray*}


\begin{Prop}
La Transformada de Laplace $\hat{U}\left(\alpha\right)$ y $\hat{F}\left(\alpha\right)$ determina una a la otra de manera \'unica por la relaci\'on $\hat{U}\left(\alpha\right)=\frac{1}{1-\hat{F}\left(\alpha\right)}$.
\end{Prop}


\begin{Note}
Un proceso de renovaci\'on $N\left(t\right)$ cuyos tiempos de inter-renovaci\'on tienen media finita, es un proceso Poisson con tasa $\lambda$ si y s\'olo s\'i $\esp\left[U\left(t\right)\right]=\lambda t$, para $t\geq0$.
\end{Note}


\begin{Teo}
Sea $N\left(t\right)$ un proceso puntual simple con puntos de localizaci\'on $T_{n}$ tal que $\eta\left(t\right)=\esp\left[N\left(\right)\right]$ es finita para cada $t$. Entonces para cualquier funci\'on $f:\rea_{+}\rightarrow\rea$,
\begin{eqnarray*}
\esp\left[\sum_{n=1}^{N\left(\right)}f\left(T_{n}\right)\right]=\int_{\left(0,t\right]}f\left(s\right)d\eta\left(s\right)\textrm{,  }t\geq0,
\end{eqnarray*}
suponiendo que la integral exista. Adem\'as si $X_{1},X_{2},\ldots$ son variables aleatorias definidas en el mismo espacio de probabilidad que el proceso $N\left(t\right)$ tal que $\esp\left[X_{n}|T_{n}=s\right]=f\left(s\right)$, independiente de $n$. Entonces
\begin{eqnarray*}
\esp\left[\sum_{n=1}^{N\left(t\right)}X_{n}\right]=\int_{\left(0,t\right]}f\left(s\right)d\eta\left(s\right)\textrm{,  }t\geq0,
\end{eqnarray*} 
suponiendo que la integral exista. 
\end{Teo}

\begin{Coro}[Identidad de Wald para Renovaciones]
Para el proceso de renovaci\'on $N\left(t\right)$,
\begin{eqnarray*}
\esp\left[T_{N\left(t\right)+1}\right]=\mu\esp\left[N\left(t\right)+1\right]\textrm{,  }t\geq0,
\end{eqnarray*}  
\end{Coro}

%______________________________________________________________________
%\subsection{Procesos de Renovaci\'on}
%______________________________________________________________________

\begin{Def}\label{Def.Tn}
Sean $0\leq T_{1}\leq T_{2}\leq \ldots$ son tiempos aleatorios infinitos en los cuales ocurren ciertos eventos. El n\'umero de tiempos $T_{n}$ en el intervalo $\left[0,t\right)$ es

\begin{eqnarray}
N\left(t\right)=\sum_{n=1}^{\infty}\indora\left(T_{n}\leq t\right),
\end{eqnarray}
para $t\geq0$.
\end{Def}

Si se consideran los puntos $T_{n}$ como elementos de $\rea_{+}$, y $N\left(t\right)$ es el n\'umero de puntos en $\rea$. El proceso denotado por $\left\{N\left(t\right):t\geq0\right\}$, denotado por $N\left(t\right)$, es un proceso puntual en $\rea_{+}$. Los $T_{n}$ son los tiempos de ocurrencia, el proceso puntual $N\left(t\right)$ es simple si su n\'umero de ocurrencias son distintas: $0<T_{1}<T_{2}<\ldots$ casi seguramente.

\begin{Def}
Un proceso puntual $N\left(t\right)$ es un proceso de renovaci\'on si los tiempos de interocurrencia $\xi_{n}=T_{n}-T_{n-1}$, para $n\geq1$, son independientes e identicamente distribuidos con distribuci\'on $F$, donde $F\left(0\right)=0$ y $T_{0}=0$. Los $T_{n}$ son llamados tiempos de renovaci\'on, referente a la independencia o renovaci\'on de la informaci\'on estoc\'astica en estos tiempos. Los $\xi_{n}$ son los tiempos de inter-renovaci\'on, y $N\left(t\right)$ es el n\'umero de renovaciones en el intervalo $\left[0,t\right)$
\end{Def}


\begin{Note}
Para definir un proceso de renovaci\'on para cualquier contexto, solamente hay que especificar una distribuci\'on $F$, con $F\left(0\right)=0$, para los tiempos de inter-renovaci\'on. La funci\'on $F$ en turno degune las otra variables aleatorias. De manera formal, existe un espacio de probabilidad y una sucesi\'on de variables aleatorias $\xi_{1},\xi_{2},\ldots$ definidas en este con distribuci\'on $F$. Entonces las otras cantidades son $T_{n}=\sum_{k=1}^{n}\xi_{k}$ y $N\left(t\right)=\sum_{n=1}^{\infty}\indora\left(T_{n}\leq t\right)$, donde $T_{n}\rightarrow\infty$ casi seguramente por la Ley Fuerte de los Grandes Números.
\end{Note}

\begin{Def}\label{Def.Tn}
Sean $0\leq T_{1}\leq T_{2}\leq \ldots$ son tiempos aleatorios infinitos en los cuales ocurren ciertos eventos. El n\'umero de tiempos $T_{n}$ en el intervalo $\left[0,t\right)$ es

\begin{eqnarray}
N\left(t\right)=\sum_{n=1}^{\infty}\indora\left(T_{n}\leq t\right),
\end{eqnarray}
para $t\geq0$.
\end{Def}

Si se consideran los puntos $T_{n}$ como elementos de $\rea_{+}$, y $N\left(t\right)$ es el n\'umero de puntos en $\rea$. El proceso denotado por $\left\{N\left(t\right):t\geq0\right\}$, denotado por $N\left(t\right)$, es un proceso puntual en $\rea_{+}$. Los $T_{n}$ son los tiempos de ocurrencia, el proceso puntual $N\left(t\right)$ es simple si su n\'umero de ocurrencias son distintas: $0<T_{1}<T_{2}<\ldots$ casi seguramente.

\begin{Def}
Un proceso puntual $N\left(t\right)$ es un proceso de renovaci\'on si los tiempos de interocurrencia $\xi_{n}=T_{n}-T_{n-1}$, para $n\geq1$, son independientes e identicamente distribuidos con distribuci\'on $F$, donde $F\left(0\right)=0$ y $T_{0}=0$. Los $T_{n}$ son llamados tiempos de renovaci\'on, referente a la independencia o renovaci\'on de la informaci\'on estoc\'astica en estos tiempos. Los $\xi_{n}$ son los tiempos de inter-renovaci\'on, y $N\left(t\right)$ es el n\'umero de renovaciones en el intervalo $\left[0,t\right)$
\end{Def}


\begin{Note}
Para definir un proceso de renovaci\'on para cualquier contexto, solamente hay que especificar una distribuci\'on $F$, con $F\left(0\right)=0$, para los tiempos de inter-renovaci\'on. La funci\'on $F$ en turno degune las otra variables aleatorias. De manera formal, existe un espacio de probabilidad y una sucesi\'on de variables aleatorias $\xi_{1},\xi_{2},\ldots$ definidas en este con distribuci\'on $F$. Entonces las otras cantidades son $T_{n}=\sum_{k=1}^{n}\xi_{k}$ y $N\left(t\right)=\sum_{n=1}^{\infty}\indora\left(T_{n}\leq t\right)$, donde $T_{n}\rightarrow\infty$ casi seguramente por la Ley Fuerte de los Grandes N\'umeros.
\end{Note}







Los tiempos $T_{n}$ est\'an relacionados con los conteos de $N\left(t\right)$ por

\begin{eqnarray*}
\left\{N\left(t\right)\geq n\right\}&=&\left\{T_{n}\leq t\right\}\\
T_{N\left(t\right)}\leq &t&<T_{N\left(t\right)+1},
\end{eqnarray*}

adem\'as $N\left(T_{n}\right)=n$, y 

\begin{eqnarray*}
N\left(t\right)=\max\left\{n:T_{n}\leq t\right\}=\min\left\{n:T_{n+1}>t\right\}
\end{eqnarray*}

Por propiedades de la convoluci\'on se sabe que

\begin{eqnarray*}
P\left\{T_{n}\leq t\right\}=F^{n\star}\left(t\right)
\end{eqnarray*}
que es la $n$-\'esima convoluci\'on de $F$. Entonces 

\begin{eqnarray*}
\left\{N\left(t\right)\geq n\right\}&=&\left\{T_{n}\leq t\right\}\\
P\left\{N\left(t\right)\leq n\right\}&=&1-F^{\left(n+1\right)\star}\left(t\right)
\end{eqnarray*}

Adem\'as usando el hecho de que $\esp\left[N\left(t\right)\right]=\sum_{n=1}^{\infty}P\left\{N\left(t\right)\geq n\right\}$
se tiene que

\begin{eqnarray*}
\esp\left[N\left(t\right)\right]=\sum_{n=1}^{\infty}F^{n\star}\left(t\right)
\end{eqnarray*}

\begin{Prop}
Para cada $t\geq0$, la funci\'on generadora de momentos $\esp\left[e^{\alpha N\left(t\right)}\right]$ existe para alguna $\alpha$ en una vecindad del 0, y de aqu\'i que $\esp\left[N\left(t\right)^{m}\right]<\infty$, para $m\geq1$.
\end{Prop}

\begin{Ejem}[\textbf{Proceso Poisson}]

Suponga que se tienen tiempos de inter-renovaci\'on \textit{i.i.d.} del proceso de renovaci\'on $N\left(t\right)$ tienen distribuci\'on exponencial $F\left(t\right)=q-e^{-\lambda t}$ con tasa $\lambda$. Entonces $N\left(t\right)$ es un proceso Poisson con tasa $\lambda$.

\end{Ejem}


\begin{Note}
Si el primer tiempo de renovaci\'on $\xi_{1}$ no tiene la misma distribuci\'on que el resto de las $\xi_{n}$, para $n\geq2$, a $N\left(t\right)$ se le llama Proceso de Renovaci\'on retardado, donde si $\xi$ tiene distribuci\'on $G$, entonces el tiempo $T_{n}$ de la $n$-\'esima renovaci\'on tiene distribuci\'on $G\star F^{\left(n-1\right)\star}\left(t\right)$
\end{Note}


\begin{Teo}
Para una constante $\mu\leq\infty$ ( o variable aleatoria), las siguientes expresiones son equivalentes:

\begin{eqnarray}
lim_{n\rightarrow\infty}n^{-1}T_{n}&=&\mu,\textrm{ c.s.}\\
lim_{t\rightarrow\infty}t^{-1}N\left(t\right)&=&1/\mu,\textrm{ c.s.}
\end{eqnarray}
\end{Teo}


Es decir, $T_{n}$ satisface la Ley Fuerte de los Grandes N\'umeros s\'i y s\'olo s\'i $N\left/t\right)$ la cumple.


\begin{Coro}[Ley Fuerte de los Grandes N\'umeros para Procesos de Renovaci\'on]
Si $N\left(t\right)$ es un proceso de renovaci\'on cuyos tiempos de inter-renovaci\'on tienen media $\mu\leq\infty$, entonces
\begin{eqnarray}
t^{-1}N\left(t\right)\rightarrow 1/\mu,\textrm{ c.s. cuando }t\rightarrow\infty.
\end{eqnarray}

\end{Coro}


Considerar el proceso estoc\'astico de valores reales $\left\{Z\left(t\right):t\geq0\right\}$ en el mismo espacio de probabilidad que $N\left(t\right)$

\begin{Def}
Para el proceso $\left\{Z\left(t\right):t\geq0\right\}$ se define la fluctuaci\'on m\'axima de $Z\left(t\right)$ en el intervalo $\left(T_{n-1},T_{n}\right]$:
\begin{eqnarray*}
M_{n}=\sup_{T_{n-1}<t\leq T_{n}}|Z\left(t\right)-Z\left(T_{n-1}\right)|
\end{eqnarray*}
\end{Def}

\begin{Teo}
Sup\'ongase que $n^{-1}T_{n}\rightarrow\mu$ c.s. cuando $n\rightarrow\infty$, donde $\mu\leq\infty$ es una constante o variable aleatoria. Sea $a$ una constante o variable aleatoria que puede ser infinita cuando $\mu$ es finita, y considere las expresiones l\'imite:
\begin{eqnarray}
lim_{n\rightarrow\infty}n^{-1}Z\left(T_{n}\right)&=&a,\textrm{ c.s.}\\
lim_{t\rightarrow\infty}t^{-1}Z\left(t\right)&=&a/\mu,\textrm{ c.s.}
\end{eqnarray}
La segunda expresi\'on implica la primera. Conversamente, la primera implica la segunda si el proceso $Z\left(t\right)$ es creciente, o si $lim_{n\rightarrow\infty}n^{-1}M_{n}=0$ c.s.
\end{Teo}

\begin{Coro}
Si $N\left(t\right)$ es un proceso de renovaci\'on, y $\left(Z\left(T_{n}\right)-Z\left(T_{n-1}\right),M_{n}\right)$, para $n\geq1$, son variables aleatorias independientes e id\'enticamente distribuidas con media finita, entonces,
\begin{eqnarray}
lim_{t\rightarrow\infty}t^{-1}Z\left(t\right)\rightarrow\frac{\esp\left[Z\left(T_{1}\right)-Z\left(T_{0}\right)\right]}{\esp\left[T_{1}\right]},\textrm{ c.s. cuando  }t\rightarrow\infty.
\end{eqnarray}
\end{Coro}


Sup\'ongase que $N\left(t\right)$ es un proceso de renovaci\'on con distribuci\'on $F$ con media finita $\mu$.

\begin{Def}
La funci\'on de renovaci\'on asociada con la distribuci\'on $F$, del proceso $N\left(t\right)$, es
\begin{eqnarray*}
U\left(t\right)=\sum_{n=1}^{\infty}F^{n\star}\left(t\right),\textrm{   }t\geq0,
\end{eqnarray*}
donde $F^{0\star}\left(t\right)=\indora\left(t\geq0\right)$.
\end{Def}


\begin{Prop}
Sup\'ongase que la distribuci\'on de inter-renovaci\'on $F$ tiene densidad $f$. Entonces $U\left(t\right)$ tambi\'en tiene densidad, para $t>0$, y es $U^{'}\left(t\right)=\sum_{n=0}^{\infty}f^{n\star}\left(t\right)$. Adem\'as
\begin{eqnarray*}
\prob\left\{N\left(t\right)>N\left(t-\right)\right\}=0\textrm{,   }t\geq0.
\end{eqnarray*}
\end{Prop}

\begin{Def}
La Transformada de Laplace-Stieljes de $F$ est\'a dada por

\begin{eqnarray*}
\hat{F}\left(\alpha\right)=\int_{\rea_{+}}e^{-\alpha t}dF\left(t\right)\textrm{,  }\alpha\geq0.
\end{eqnarray*}
\end{Def}

Entonces

\begin{eqnarray*}
\hat{U}\left(\alpha\right)=\sum_{n=0}^{\infty}\hat{F^{n\star}}\left(\alpha\right)=\sum_{n=0}^{\infty}\hat{F}\left(\alpha\right)^{n}=\frac{1}{1-\hat{F}\left(\alpha\right)}.
\end{eqnarray*}


\begin{Prop}
La Transformada de Laplace $\hat{U}\left(\alpha\right)$ y $\hat{F}\left(\alpha\right)$ determina una a la otra de manera \'unica por la relaci\'on $\hat{U}\left(\alpha\right)=\frac{1}{1-\hat{F}\left(\alpha\right)}$.
\end{Prop}


\begin{Note}
Un proceso de renovaci\'on $N\left(t\right)$ cuyos tiempos de inter-renovaci\'on tienen media finita, es un proceso Poisson con tasa $\lambda$ si y s\'olo s\'i $\esp\left[U\left(t\right)\right]=\lambda t$, para $t\geq0$.
\end{Note}


\begin{Teo}
Sea $N\left(t\right)$ un proceso puntual simple con puntos de localizaci\'on $T_{n}$ tal que $\eta\left(t\right)=\esp\left[N\left(\right)\right]$ es finita para cada $t$. Entonces para cualquier funci\'on $f:\rea_{+}\rightarrow\rea$,
\begin{eqnarray*}
\esp\left[\sum_{n=1}^{N\left(\right)}f\left(T_{n}\right)\right]=\int_{\left(0,t\right]}f\left(s\right)d\eta\left(s\right)\textrm{,  }t\geq0,
\end{eqnarray*}
suponiendo que la integral exista. Adem\'as si $X_{1},X_{2},\ldots$ son variables aleatorias definidas en el mismo espacio de probabilidad que el proceso $N\left(t\right)$ tal que $\esp\left[X_{n}|T_{n}=s\right]=f\left(s\right)$, independiente de $n$. Entonces
\begin{eqnarray*}
\esp\left[\sum_{n=1}^{N\left(t\right)}X_{n}\right]=\int_{\left(0,t\right]}f\left(s\right)d\eta\left(s\right)\textrm{,  }t\geq0,
\end{eqnarray*} 
suponiendo que la integral exista. 
\end{Teo}

\begin{Coro}[Identidad de Wald para Renovaciones]
Para el proceso de renovaci\'on $N\left(t\right)$,
\begin{eqnarray*}
\esp\left[T_{N\left(t\right)+1}\right]=\mu\esp\left[N\left(t\right)+1\right]\textrm{,  }t\geq0,
\end{eqnarray*}  
\end{Coro}


\begin{Def}
Sea $h\left(t\right)$ funci\'on de valores reales en $\rea$ acotada en intervalos finitos e igual a cero para $t<0$ La ecuaci\'on de renovaci\'on para $h\left(t\right)$ y la distribuci\'on $F$ es

\begin{eqnarray}\label{Ec.Renovacion}
H\left(t\right)=h\left(t\right)+\int_{\left[0,t\right]}H\left(t-s\right)dF\left(s\right)\textrm{,    }t\geq0,
\end{eqnarray}
donde $H\left(t\right)$ es una funci\'on de valores reales. Esto es $H=h+F\star H$. Decimos que $H\left(t\right)$ es soluci\'on de esta ecuaci\'on si satisface la ecuaci\'on, y es acotada en intervalos finitos e iguales a cero para $t<0$.
\end{Def}

\begin{Prop}
La funci\'on $U\star h\left(t\right)$ es la \'unica soluci\'on de la ecuaci\'on de renovaci\'on (\ref{Ec.Renovacion}).
\end{Prop}

\begin{Teo}[Teorema Renovaci\'on Elemental]
\begin{eqnarray*}
t^{-1}U\left(t\right)\rightarrow 1/\mu\textrm{,    cuando }t\rightarrow\infty.
\end{eqnarray*}
\end{Teo}



Sup\'ongase que $N\left(t\right)$ es un proceso de renovaci\'on con distribuci\'on $F$ con media finita $\mu$.

\begin{Def}
La funci\'on de renovaci\'on asociada con la distribuci\'on $F$, del proceso $N\left(t\right)$, es
\begin{eqnarray*}
U\left(t\right)=\sum_{n=1}^{\infty}F^{n\star}\left(t\right),\textrm{   }t\geq0,
\end{eqnarray*}
donde $F^{0\star}\left(t\right)=\indora\left(t\geq0\right)$.
\end{Def}


\begin{Prop}
Sup\'ongase que la distribuci\'on de inter-renovaci\'on $F$ tiene densidad $f$. Entonces $U\left(t\right)$ tambi\'en tiene densidad, para $t>0$, y es $U^{'}\left(t\right)=\sum_{n=0}^{\infty}f^{n\star}\left(t\right)$. Adem\'as
\begin{eqnarray*}
\prob\left\{N\left(t\right)>N\left(t-\right)\right\}=0\textrm{,   }t\geq0.
\end{eqnarray*}
\end{Prop}

\begin{Def}
La Transformada de Laplace-Stieljes de $F$ est\'a dada por

\begin{eqnarray*}
\hat{F}\left(\alpha\right)=\int_{\rea_{+}}e^{-\alpha t}dF\left(t\right)\textrm{,  }\alpha\geq0.
\end{eqnarray*}
\end{Def}

Entonces

\begin{eqnarray*}
\hat{U}\left(\alpha\right)=\sum_{n=0}^{\infty}\hat{F^{n\star}}\left(\alpha\right)=\sum_{n=0}^{\infty}\hat{F}\left(\alpha\right)^{n}=\frac{1}{1-\hat{F}\left(\alpha\right)}.
\end{eqnarray*}


\begin{Prop}
La Transformada de Laplace $\hat{U}\left(\alpha\right)$ y $\hat{F}\left(\alpha\right)$ determina una a la otra de manera \'unica por la relaci\'on $\hat{U}\left(\alpha\right)=\frac{1}{1-\hat{F}\left(\alpha\right)}$.
\end{Prop}


\begin{Note}
Un proceso de renovaci\'on $N\left(t\right)$ cuyos tiempos de inter-renovaci\'on tienen media finita, es un proceso Poisson con tasa $\lambda$ si y s\'olo s\'i $\esp\left[U\left(t\right)\right]=\lambda t$, para $t\geq0$.
\end{Note}


\begin{Teo}
Sea $N\left(t\right)$ un proceso puntual simple con puntos de localizaci\'on $T_{n}$ tal que $\eta\left(t\right)=\esp\left[N\left(\right)\right]$ es finita para cada $t$. Entonces para cualquier funci\'on $f:\rea_{+}\rightarrow\rea$,
\begin{eqnarray*}
\esp\left[\sum_{n=1}^{N\left(\right)}f\left(T_{n}\right)\right]=\int_{\left(0,t\right]}f\left(s\right)d\eta\left(s\right)\textrm{,  }t\geq0,
\end{eqnarray*}
suponiendo que la integral exista. Adem\'as si $X_{1},X_{2},\ldots$ son variables aleatorias definidas en el mismo espacio de probabilidad que el proceso $N\left(t\right)$ tal que $\esp\left[X_{n}|T_{n}=s\right]=f\left(s\right)$, independiente de $n$. Entonces
\begin{eqnarray*}
\esp\left[\sum_{n=1}^{N\left(t\right)}X_{n}\right]=\int_{\left(0,t\right]}f\left(s\right)d\eta\left(s\right)\textrm{,  }t\geq0,
\end{eqnarray*} 
suponiendo que la integral exista. 
\end{Teo}

\begin{Coro}[Identidad de Wald para Renovaciones]
Para el proceso de renovaci\'on $N\left(t\right)$,
\begin{eqnarray*}
\esp\left[T_{N\left(t\right)+1}\right]=\mu\esp\left[N\left(t\right)+1\right]\textrm{,  }t\geq0,
\end{eqnarray*}  
\end{Coro}


\begin{Def}
Sea $h\left(t\right)$ funci\'on de valores reales en $\rea$ acotada en intervalos finitos e igual a cero para $t<0$ La ecuaci\'on de renovaci\'on para $h\left(t\right)$ y la distribuci\'on $F$ es

\begin{eqnarray}\label{Ec.Renovacion}
H\left(t\right)=h\left(t\right)+\int_{\left[0,t\right]}H\left(t-s\right)dF\left(s\right)\textrm{,    }t\geq0,
\end{eqnarray}
donde $H\left(t\right)$ es una funci\'on de valores reales. Esto es $H=h+F\star H$. Decimos que $H\left(t\right)$ es soluci\'on de esta ecuaci\'on si satisface la ecuaci\'on, y es acotada en intervalos finitos e iguales a cero para $t<0$.
\end{Def}

\begin{Prop}
La funci\'on $U\star h\left(t\right)$ es la \'unica soluci\'on de la ecuaci\'on de renovaci\'on (\ref{Ec.Renovacion}).
\end{Prop}

\begin{Teo}[Teorema Renovaci\'on Elemental]
\begin{eqnarray*}
t^{-1}U\left(t\right)\rightarrow 1/\mu\textrm{,    cuando }t\rightarrow\infty.
\end{eqnarray*}
\end{Teo}


\begin{Note} Una funci\'on $h:\rea_{+}\rightarrow\rea$ es Directamente Riemann Integrable en los siguientes casos:
\begin{itemize}
\item[a)] $h\left(t\right)\geq0$ es decreciente y Riemann Integrable.
\item[b)] $h$ es continua excepto posiblemente en un conjunto de Lebesgue de medida 0, y $|h\left(t\right)|\leq b\left(t\right)$, donde $b$ es DRI.
\end{itemize}
\end{Note}

\begin{Teo}[Teorema Principal de Renovaci\'on]
Si $F$ es no aritm\'etica y $h\left(t\right)$ es Directamente Riemann Integrable (DRI), entonces

\begin{eqnarray*}
lim_{t\rightarrow\infty}U\star h=\frac{1}{\mu}\int_{\rea_{+}}h\left(s\right)ds.
\end{eqnarray*}
\end{Teo}

\begin{Prop}
Cualquier funci\'on $H\left(t\right)$ acotada en intervalos finitos y que es 0 para $t<0$ puede expresarse como
\begin{eqnarray*}
H\left(t\right)=U\star h\left(t\right)\textrm{,  donde }h\left(t\right)=H\left(t\right)-F\star H\left(t\right)
\end{eqnarray*}
\end{Prop}

\begin{Def}
Un proceso estoc\'astico $X\left(t\right)$ es crudamente regenerativo en un tiempo aleatorio positivo $T$ si
\begin{eqnarray*}
\esp\left[X\left(T+t\right)|T\right]=\esp\left[X\left(t\right)\right]\textrm{, para }t\geq0,\end{eqnarray*}
y con las esperanzas anteriores finitas.
\end{Def}

\begin{Prop}
Sup\'ongase que $X\left(t\right)$ es un proceso crudamente regenerativo en $T$, que tiene distribuci\'on $F$. Si $\esp\left[X\left(t\right)\right]$ es acotado en intervalos finitos, entonces
\begin{eqnarray*}
\esp\left[X\left(t\right)\right]=U\star h\left(t\right)\textrm{,  donde }h\left(t\right)=\esp\left[X\left(t\right)\indora\left(T>t\right)\right].
\end{eqnarray*}
\end{Prop}

\begin{Teo}[Regeneraci\'on Cruda]
Sup\'ongase que $X\left(t\right)$ es un proceso con valores positivo crudamente regenerativo en $T$, y def\'inase $M=\sup\left\{|X\left(t\right)|:t\leq T\right\}$. Si $T$ es no aritm\'etico y $M$ y $MT$ tienen media finita, entonces
\begin{eqnarray*}
lim_{t\rightarrow\infty}\esp\left[X\left(t\right)\right]=\frac{1}{\mu}\int_{\rea_{+}}h\left(s\right)ds,
\end{eqnarray*}
donde $h\left(t\right)=\esp\left[X\left(t\right)\indora\left(T>t\right)\right]$.
\end{Teo}


\begin{Note} Una funci\'on $h:\rea_{+}\rightarrow\rea$ es Directamente Riemann Integrable en los siguientes casos:
\begin{itemize}
\item[a)] $h\left(t\right)\geq0$ es decreciente y Riemann Integrable.
\item[b)] $h$ es continua excepto posiblemente en un conjunto de Lebesgue de medida 0, y $|h\left(t\right)|\leq b\left(t\right)$, donde $b$ es DRI.
\end{itemize}
\end{Note}

\begin{Teo}[Teorema Principal de Renovaci\'on]
Si $F$ es no aritm\'etica y $h\left(t\right)$ es Directamente Riemann Integrable (DRI), entonces

\begin{eqnarray*}
lim_{t\rightarrow\infty}U\star h=\frac{1}{\mu}\int_{\rea_{+}}h\left(s\right)ds.
\end{eqnarray*}
\end{Teo}

\begin{Prop}
Cualquier funci\'on $H\left(t\right)$ acotada en intervalos finitos y que es 0 para $t<0$ puede expresarse como
\begin{eqnarray*}
H\left(t\right)=U\star h\left(t\right)\textrm{,  donde }h\left(t\right)=H\left(t\right)-F\star H\left(t\right)
\end{eqnarray*}
\end{Prop}

\begin{Def}
Un proceso estoc\'astico $X\left(t\right)$ es crudamente regenerativo en un tiempo aleatorio positivo $T$ si
\begin{eqnarray*}
\esp\left[X\left(T+t\right)|T\right]=\esp\left[X\left(t\right)\right]\textrm{, para }t\geq0,\end{eqnarray*}
y con las esperanzas anteriores finitas.
\end{Def}

\begin{Prop}
Sup\'ongase que $X\left(t\right)$ es un proceso crudamente regenerativo en $T$, que tiene distribuci\'on $F$. Si $\esp\left[X\left(t\right)\right]$ es acotado en intervalos finitos, entonces
\begin{eqnarray*}
\esp\left[X\left(t\right)\right]=U\star h\left(t\right)\textrm{,  donde }h\left(t\right)=\esp\left[X\left(t\right)\indora\left(T>t\right)\right].
\end{eqnarray*}
\end{Prop}

\begin{Teo}[Regeneraci\'on Cruda]
Sup\'ongase que $X\left(t\right)$ es un proceso con valores positivo crudamente regenerativo en $T$, y def\'inase $M=\sup\left\{|X\left(t\right)|:t\leq T\right\}$. Si $T$ es no aritm\'etico y $M$ y $MT$ tienen media finita, entonces
\begin{eqnarray*}
lim_{t\rightarrow\infty}\esp\left[X\left(t\right)\right]=\frac{1}{\mu}\int_{\rea_{+}}h\left(s\right)ds,
\end{eqnarray*}
donde $h\left(t\right)=\esp\left[X\left(t\right)\indora\left(T>t\right)\right]$.
\end{Teo}

\begin{Def}
Para el proceso $\left\{\left(N\left(t\right),X\left(t\right)\right):t\geq0\right\}$, sus trayectoria muestrales en el intervalo de tiempo $\left[T_{n-1},T_{n}\right)$ est\'an descritas por
\begin{eqnarray*}
\zeta_{n}=\left(\xi_{n},\left\{X\left(T_{n-1}+t\right):0\leq t<\xi_{n}\right\}\right)
\end{eqnarray*}
Este $\zeta_{n}$ es el $n$-\'esimo segmento del proceso. El proceso es regenerativo sobre los tiempos $T_{n}$ si sus segmentos $\zeta_{n}$ son independientes e id\'enticamennte distribuidos.
\end{Def}


\begin{Note}
Si $\tilde{X}\left(t\right)$ con espacio de estados $\tilde{S}$ es regenerativo sobre $T_{n}$, entonces $X\left(t\right)=f\left(\tilde{X}\left(t\right)\right)$ tambi\'en es regenerativo sobre $T_{n}$, para cualquier funci\'on $f:\tilde{S}\rightarrow S$.
\end{Note}

\begin{Note}
Los procesos regenerativos son crudamente regenerativos, pero no al rev\'es.
\end{Note}


\begin{Note}
Un proceso estoc\'astico a tiempo continuo o discreto es regenerativo si existe un proceso de renovaci\'on  tal que los segmentos del proceso entre tiempos de renovaci\'on sucesivos son i.i.d., es decir, para $\left\{X\left(t\right):t\geq0\right\}$ proceso estoc\'astico a tiempo continuo con espacio de estados $S$, espacio m\'etrico.
\end{Note}

Para $\left\{X\left(t\right):t\geq0\right\}$ Proceso Estoc\'astico a tiempo continuo con estado de espacios $S$, que es un espacio m\'etrico, con trayectorias continuas por la derecha y con l\'imites por la izquierda c.s. Sea $N\left(t\right)$ un proceso de renovaci\'on en $\rea_{+}$ definido en el mismo espacio de probabilidad que $X\left(t\right)$, con tiempos de renovaci\'on $T$ y tiempos de inter-renovaci\'on $\xi_{n}=T_{n}-T_{n-1}$, con misma distribuci\'on $F$ de media finita $\mu$.



\begin{Def}
Para el proceso $\left\{\left(N\left(t\right),X\left(t\right)\right):t\geq0\right\}$, sus trayectoria muestrales en el intervalo de tiempo $\left[T_{n-1},T_{n}\right)$ est\'an descritas por
\begin{eqnarray*}
\zeta_{n}=\left(\xi_{n},\left\{X\left(T_{n-1}+t\right):0\leq t<\xi_{n}\right\}\right)
\end{eqnarray*}
Este $\zeta_{n}$ es el $n$-\'esimo segmento del proceso. El proceso es regenerativo sobre los tiempos $T_{n}$ si sus segmentos $\zeta_{n}$ son independientes e id\'enticamennte distribuidos.
\end{Def}

\begin{Note}
Un proceso regenerativo con media de la longitud de ciclo finita es llamado positivo recurrente.
\end{Note}

\begin{Teo}[Procesos Regenerativos]
Suponga que el proceso
\end{Teo}


\begin{Def}[Renewal Process Trinity]
Para un proceso de renovaci\'on $N\left(t\right)$, los siguientes procesos proveen de informaci\'on sobre los tiempos de renovaci\'on.
\begin{itemize}
\item $A\left(t\right)=t-T_{N\left(t\right)}$, el tiempo de recurrencia hacia atr\'as al tiempo $t$, que es el tiempo desde la \'ultima renovaci\'on para $t$.

\item $B\left(t\right)=T_{N\left(t\right)+1}-t$, el tiempo de recurrencia hacia adelante al tiempo $t$, residual del tiempo de renovaci\'on, que es el tiempo para la pr\'oxima renovaci\'on despu\'es de $t$.

\item $L\left(t\right)=\xi_{N\left(t\right)+1}=A\left(t\right)+B\left(t\right)$, la longitud del intervalo de renovaci\'on que contiene a $t$.
\end{itemize}
\end{Def}

\begin{Note}
El proceso tridimensional $\left(A\left(t\right),B\left(t\right),L\left(t\right)\right)$ es regenerativo sobre $T_{n}$, y por ende cada proceso lo es. Cada proceso $A\left(t\right)$ y $B\left(t\right)$ son procesos de MArkov a tiempo continuo con trayectorias continuas por partes en el espacio de estados $\rea_{+}$. Una expresi\'on conveniente para su distribuci\'on conjunta es, para $0\leq x<t,y\geq0$
\begin{equation}\label{NoRenovacion}
P\left\{A\left(t\right)>x,B\left(t\right)>y\right\}=
P\left\{N\left(t+y\right)-N\left((t-x)\right)=0\right\}
\end{equation}
\end{Note}

\begin{Ejem}[Tiempos de recurrencia Poisson]
Si $N\left(t\right)$ es un proceso Poisson con tasa $\lambda$, entonces de la expresi\'on (\ref{NoRenovacion}) se tiene que

\begin{eqnarray*}
\begin{array}{lc}
P\left\{A\left(t\right)>x,B\left(t\right)>y\right\}=e^{-\lambda\left(x+y\right)},&0\leq x<t,y\geq0,
\end{array}
\end{eqnarray*}
que es la probabilidad Poisson de no renovaciones en un intervalo de longitud $x+y$.

\end{Ejem}

\begin{Note}
Una cadena de Markov erg\'odica tiene la propiedad de ser estacionaria si la distribuci\'on de su estado al tiempo $0$ es su distribuci\'on estacionaria.
\end{Note}


\begin{Def}
Un proceso estoc\'astico a tiempo continuo $\left\{X\left(t\right):t\geq0\right\}$ en un espacio general es estacionario si sus distribuciones finito dimensionales son invariantes bajo cualquier  traslado: para cada $0\leq s_{1}<s_{2}<\cdots<s_{k}$ y $t\geq0$,
\begin{eqnarray*}
\left(X\left(s_{1}+t\right),\ldots,X\left(s_{k}+t\right)\right)=_{d}\left(X\left(s_{1}\right),\ldots,X\left(s_{k}\right)\right).
\end{eqnarray*}
\end{Def}

\begin{Note}
Un proceso de Markov es estacionario si $X\left(t\right)=_{d}X\left(0\right)$, $t\geq0$.
\end{Note}

Considerese el proceso $N\left(t\right)=\sum_{n}\indora\left(\tau_{n}\leq t\right)$ en $\rea_{+}$, con puntos $0<\tau_{1}<\tau_{2}<\cdots$.

\begin{Prop}
Si $N$ es un proceso puntual estacionario y $\esp\left[N\left(1\right)\right]<\infty$, entonces $\esp\left[N\left(t\right)\right]=t\esp\left[N\left(1\right)\right]$, $t\geq0$

\end{Prop}

\begin{Teo}
Los siguientes enunciados son equivalentes
\begin{itemize}
\item[i)] El proceso retardado de renovaci\'on $N$ es estacionario.

\item[ii)] EL proceso de tiempos de recurrencia hacia adelante $B\left(t\right)$ es estacionario.


\item[iii)] $\esp\left[N\left(t\right)\right]=t/\mu$,


\item[iv)] $G\left(t\right)=F_{e}\left(t\right)=\frac{1}{\mu}\int_{0}^{t}\left[1-F\left(s\right)\right]ds$
\end{itemize}
Cuando estos enunciados son ciertos, $P\left\{B\left(t\right)\leq x\right\}=F_{e}\left(x\right)$, para $t,x\geq0$.

\end{Teo}

\begin{Note}
Una consecuencia del teorema anterior es que el Proceso Poisson es el \'unico proceso sin retardo que es estacionario.
\end{Note}

\begin{Coro}
El proceso de renovaci\'on $N\left(t\right)$ sin retardo, y cuyos tiempos de inter renonaci\'on tienen media finita, es estacionario si y s\'olo si es un proceso Poisson.

\end{Coro}

%______________________________________________________________________

%\section{Ejemplos, Notas importantes}
%______________________________________________________________________
%\section*{Ap\'endice A}
%__________________________________________________________________

%________________________________________________________________________
%\subsection*{Procesos Regenerativos}
%________________________________________________________________________



\begin{Note}
Si $\tilde{X}\left(t\right)$ con espacio de estados $\tilde{S}$ es regenerativo sobre $T_{n}$, entonces $X\left(t\right)=f\left(\tilde{X}\left(t\right)\right)$ tambi\'en es regenerativo sobre $T_{n}$, para cualquier funci\'on $f:\tilde{S}\rightarrow S$.
\end{Note}

\begin{Note}
Los procesos regenerativos son crudamente regenerativos, pero no al rev\'es.
\end{Note}
%\subsection*{Procesos Regenerativos: Sigman\cite{Sigman1}}
\begin{Def}[Definici\'on Cl\'asica]
Un proceso estoc\'astico $X=\left\{X\left(t\right):t\geq0\right\}$ es llamado regenerativo is existe una variable aleatoria $R_{1}>0$ tal que
\begin{itemize}
\item[i)] $\left\{X\left(t+R_{1}\right):t\geq0\right\}$ es independiente de $\left\{\left\{X\left(t\right):t<R_{1}\right\},\right\}$
\item[ii)] $\left\{X\left(t+R_{1}\right):t\geq0\right\}$ es estoc\'asticamente equivalente a $\left\{X\left(t\right):t>0\right\}$
\end{itemize}

Llamamos a $R_{1}$ tiempo de regeneraci\'on, y decimos que $X$ se regenera en este punto.
\end{Def}

$\left\{X\left(t+R_{1}\right)\right\}$ es regenerativo con tiempo de regeneraci\'on $R_{2}$, independiente de $R_{1}$ pero con la misma distribuci\'on que $R_{1}$. Procediendo de esta manera se obtiene una secuencia de variables aleatorias independientes e id\'enticamente distribuidas $\left\{R_{n}\right\}$ llamados longitudes de ciclo. Si definimos a $Z_{k}\equiv R_{1}+R_{2}+\cdots+R_{k}$, se tiene un proceso de renovaci\'on llamado proceso de renovaci\'on encajado para $X$.




\begin{Def}
Para $x$ fijo y para cada $t\geq0$, sea $I_{x}\left(t\right)=1$ si $X\left(t\right)\leq x$,  $I_{x}\left(t\right)=0$ en caso contrario, y def\'inanse los tiempos promedio
\begin{eqnarray*}
\overline{X}&=&lim_{t\rightarrow\infty}\frac{1}{t}\int_{0}^{\infty}X\left(u\right)du\\
\prob\left(X_{\infty}\leq x\right)&=&lim_{t\rightarrow\infty}\frac{1}{t}\int_{0}^{\infty}I_{x}\left(u\right)du,
\end{eqnarray*}
cuando estos l\'imites existan.
\end{Def}

Como consecuencia del teorema de Renovaci\'on-Recompensa, se tiene que el primer l\'imite  existe y es igual a la constante
\begin{eqnarray*}
\overline{X}&=&\frac{\esp\left[\int_{0}^{R_{1}}X\left(t\right)dt\right]}{\esp\left[R_{1}\right]},
\end{eqnarray*}
suponiendo que ambas esperanzas son finitas.

\begin{Note}
\begin{itemize}
\item[a)] Si el proceso regenerativo $X$ es positivo recurrente y tiene trayectorias muestrales no negativas, entonces la ecuaci\'on anterior es v\'alida.
\item[b)] Si $X$ es positivo recurrente regenerativo, podemos construir una \'unica versi\'on estacionaria de este proceso, $X_{e}=\left\{X_{e}\left(t\right)\right\}$, donde $X_{e}$ es un proceso estoc\'astico regenerativo y estrictamente estacionario, con distribuci\'on marginal distribuida como $X_{\infty}$
\end{itemize}
\end{Note}

Para $\left\{X\left(t\right):t\geq0\right\}$ Proceso Estoc\'astico a tiempo continuo con estado de espacios $S$, que es un espacio m\'etrico, con trayectorias continuas por la derecha y con l\'imites por la izquierda c.s. Sea $N\left(t\right)$ un proceso de renovaci\'on en $\rea_{+}$ definido en el mismo espacio de probabilidad que $X\left(t\right)$, con tiempos de renovaci\'on $T$ y tiempos de inter-renovaci\'on $\xi_{n}=T_{n}-T_{n-1}$, con misma distribuci\'on $F$ de media finita $\mu$.


\begin{Def}
Para el proceso $\left\{\left(N\left(t\right),X\left(t\right)\right):t\geq0\right\}$, sus trayectoria muestrales en el intervalo de tiempo $\left[T_{n-1},T_{n}\right)$ est\'an descritas por
\begin{eqnarray*}
\zeta_{n}=\left(\xi_{n},\left\{X\left(T_{n-1}+t\right):0\leq t<\xi_{n}\right\}\right)
\end{eqnarray*}
Este $\zeta_{n}$ es el $n$-\'esimo segmento del proceso. El proceso es regenerativo sobre los tiempos $T_{n}$ si sus segmentos $\zeta_{n}$ son independientes e id\'enticamennte distribuidos.
\end{Def}


\begin{Note}
Si $\tilde{X}\left(t\right)$ con espacio de estados $\tilde{S}$ es regenerativo sobre $T_{n}$, entonces $X\left(t\right)=f\left(\tilde{X}\left(t\right)\right)$ tambi\'en es regenerativo sobre $T_{n}$, para cualquier funci\'on $f:\tilde{S}\rightarrow S$.
\end{Note}

\begin{Note}
Los procesos regenerativos son crudamente regenerativos, pero no al rev\'es.
\end{Note}

\begin{Def}[Definici\'on Cl\'asica]
Un proceso estoc\'astico $X=\left\{X\left(t\right):t\geq0\right\}$ es llamado regenerativo is existe una variable aleatoria $R_{1}>0$ tal que
\begin{itemize}
\item[i)] $\left\{X\left(t+R_{1}\right):t\geq0\right\}$ es independiente de $\left\{\left\{X\left(t\right):t<R_{1}\right\},\right\}$
\item[ii)] $\left\{X\left(t+R_{1}\right):t\geq0\right\}$ es estoc\'asticamente equivalente a $\left\{X\left(t\right):t>0\right\}$
\end{itemize}

Llamamos a $R_{1}$ tiempo de regeneraci\'on, y decimos que $X$ se regenera en este punto.
\end{Def}

$\left\{X\left(t+R_{1}\right)\right\}$ es regenerativo con tiempo de regeneraci\'on $R_{2}$, independiente de $R_{1}$ pero con la misma distribuci\'on que $R_{1}$. Procediendo de esta manera se obtiene una secuencia de variables aleatorias independientes e id\'enticamente distribuidas $\left\{R_{n}\right\}$ llamados longitudes de ciclo. Si definimos a $Z_{k}\equiv R_{1}+R_{2}+\cdots+R_{k}$, se tiene un proceso de renovaci\'on llamado proceso de renovaci\'on encajado para $X$.

\begin{Note}
Un proceso regenerativo con media de la longitud de ciclo finita es llamado positivo recurrente.
\end{Note}


\begin{Def}
Para $x$ fijo y para cada $t\geq0$, sea $I_{x}\left(t\right)=1$ si $X\left(t\right)\leq x$,  $I_{x}\left(t\right)=0$ en caso contrario, y def\'inanse los tiempos promedio
\begin{eqnarray*}
\overline{X}&=&lim_{t\rightarrow\infty}\frac{1}{t}\int_{0}^{\infty}X\left(u\right)du\\
\prob\left(X_{\infty}\leq x\right)&=&lim_{t\rightarrow\infty}\frac{1}{t}\int_{0}^{\infty}I_{x}\left(u\right)du,
\end{eqnarray*}
cuando estos l\'imites existan.
\end{Def}

Como consecuencia del teorema de Renovaci\'on-Recompensa, se tiene que el primer l\'imite  existe y es igual a la constante
\begin{eqnarray*}
\overline{X}&=&\frac{\esp\left[\int_{0}^{R_{1}}X\left(t\right)dt\right]}{\esp\left[R_{1}\right]},
\end{eqnarray*}
suponiendo que ambas esperanzas son finitas.

\begin{Note}
\begin{itemize}
\item[a)] Si el proceso regenerativo $X$ es positivo recurrente y tiene trayectorias muestrales no negativas, entonces la ecuaci\'on anterior es v\'alida.
\item[b)] Si $X$ es positivo recurrente regenerativo, podemos construir una \'unica versi\'on estacionaria de este proceso, $X_{e}=\left\{X_{e}\left(t\right)\right\}$, donde $X_{e}$ es un proceso estoc\'astico regenerativo y estrictamente estacionario, con distribuci\'on marginal distribuida como $X_{\infty}$
\end{itemize}
\end{Note}

%__________________________________________________________________________________________
%\subsection{Procesos Regenerativos Estacionarios - Stidham \cite{Stidham}}
%__________________________________________________________________________________________


Un proceso estoc\'astico a tiempo continuo $\left\{V\left(t\right),t\geq0\right\}$ es un proceso regenerativo si existe una sucesi\'on de variables aleatorias independientes e id\'enticamente distribuidas $\left\{X_{1},X_{2},\ldots\right\}$, sucesi\'on de renovaci\'on, tal que para cualquier conjunto de Borel $A$, 

\begin{eqnarray*}
\prob\left\{V\left(t\right)\in A|X_{1}+X_{2}+\cdots+X_{R\left(t\right)}=s,\left\{V\left(\tau\right),\tau<s\right\}\right\}=\prob\left\{V\left(t-s\right)\in A|X_{1}>t-s\right\},
\end{eqnarray*}
para todo $0\leq s\leq t$, donde $R\left(t\right)=\max\left\{X_{1}+X_{2}+\cdots+X_{j}\leq t\right\}=$n\'umero de renovaciones ({\emph{puntos de regeneraci\'on}}) que ocurren en $\left[0,t\right]$. El intervalo $\left[0,X_{1}\right)$ es llamado {\emph{primer ciclo de regeneraci\'on}} de $\left\{V\left(t \right),t\geq0\right\}$, $\left[X_{1},X_{1}+X_{2}\right)$ el {\emph{segundo ciclo de regeneraci\'on}}, y as\'i sucesivamente.

Sea $X=X_{1}$ y sea $F$ la funci\'on de distrbuci\'on de $X$


\begin{Def}
Se define el proceso estacionario, $\left\{V^{*}\left(t\right),t\geq0\right\}$, para $\left\{V\left(t\right),t\geq0\right\}$ por

\begin{eqnarray*}
\prob\left\{V\left(t\right)\in A\right\}=\frac{1}{\esp\left[X\right]}\int_{0}^{\infty}\prob\left\{V\left(t+x\right)\in A|X>x\right\}\left(1-F\left(x\right)\right)dx,
\end{eqnarray*} 
para todo $t\geq0$ y todo conjunto de Borel $A$.
\end{Def}

\begin{Def}
Una distribuci\'on se dice que es {\emph{aritm\'etica}} si todos sus puntos de incremento son m\'ultiplos de la forma $0,\lambda, 2\lambda,\ldots$ para alguna $\lambda>0$ entera.
\end{Def}


\begin{Def}
Una modificaci\'on medible de un proceso $\left\{V\left(t\right),t\geq0\right\}$, es una versi\'on de este, $\left\{V\left(t,w\right)\right\}$ conjuntamente medible para $t\geq0$ y para $w\in S$, $S$ espacio de estados para $\left\{V\left(t\right),t\geq0\right\}$.
\end{Def}

\begin{Teo}
Sea $\left\{V\left(t\right),t\geq\right\}$ un proceso regenerativo no negativo con modificaci\'on medible. Sea $\esp\left[X\right]<\infty$. Entonces el proceso estacionario dado por la ecuaci\'on anterior est\'a bien definido y tiene funci\'on de distribuci\'on independiente de $t$, adem\'as
\begin{itemize}
\item[i)] \begin{eqnarray*}
\esp\left[V^{*}\left(0\right)\right]&=&\frac{\esp\left[\int_{0}^{X}V\left(s\right)ds\right]}{\esp\left[X\right]}\end{eqnarray*}
\item[ii)] Si $\esp\left[V^{*}\left(0\right)\right]<\infty$, equivalentemente, si $\esp\left[\int_{0}^{X}V\left(s\right)ds\right]<\infty$,entonces
\begin{eqnarray*}
\frac{\int_{0}^{t}V\left(s\right)ds}{t}\rightarrow\frac{\esp\left[\int_{0}^{X}V\left(s\right)ds\right]}{\esp\left[X\right]}
\end{eqnarray*}
con probabilidad 1 y en media, cuando $t\rightarrow\infty$.
\end{itemize}
\end{Teo}
%
%___________________________________________________________________________________________
%\vspace{5.5cm}
%\chapter{Cadenas de Markov estacionarias}
%\vspace{-1.0cm}


%__________________________________________________________________________________________
%\subsection{Procesos Regenerativos Estacionarios - Stidham \cite{Stidham}}
%__________________________________________________________________________________________


Un proceso estoc\'astico a tiempo continuo $\left\{V\left(t\right),t\geq0\right\}$ es un proceso regenerativo si existe una sucesi\'on de variables aleatorias independientes e id\'enticamente distribuidas $\left\{X_{1},X_{2},\ldots\right\}$, sucesi\'on de renovaci\'on, tal que para cualquier conjunto de Borel $A$, 

\begin{eqnarray*}
\prob\left\{V\left(t\right)\in A|X_{1}+X_{2}+\cdots+X_{R\left(t\right)}=s,\left\{V\left(\tau\right),\tau<s\right\}\right\}=\prob\left\{V\left(t-s\right)\in A|X_{1}>t-s\right\},
\end{eqnarray*}
para todo $0\leq s\leq t$, donde $R\left(t\right)=\max\left\{X_{1}+X_{2}+\cdots+X_{j}\leq t\right\}=$n\'umero de renovaciones ({\emph{puntos de regeneraci\'on}}) que ocurren en $\left[0,t\right]$. El intervalo $\left[0,X_{1}\right)$ es llamado {\emph{primer ciclo de regeneraci\'on}} de $\left\{V\left(t \right),t\geq0\right\}$, $\left[X_{1},X_{1}+X_{2}\right)$ el {\emph{segundo ciclo de regeneraci\'on}}, y as\'i sucesivamente.

Sea $X=X_{1}$ y sea $F$ la funci\'on de distrbuci\'on de $X$


\begin{Def}
Se define el proceso estacionario, $\left\{V^{*}\left(t\right),t\geq0\right\}$, para $\left\{V\left(t\right),t\geq0\right\}$ por

\begin{eqnarray*}
\prob\left\{V\left(t\right)\in A\right\}=\frac{1}{\esp\left[X\right]}\int_{0}^{\infty}\prob\left\{V\left(t+x\right)\in A|X>x\right\}\left(1-F\left(x\right)\right)dx,
\end{eqnarray*} 
para todo $t\geq0$ y todo conjunto de Borel $A$.
\end{Def}

\begin{Def}
Una distribuci\'on se dice que es {\emph{aritm\'etica}} si todos sus puntos de incremento son m\'ultiplos de la forma $0,\lambda, 2\lambda,\ldots$ para alguna $\lambda>0$ entera.
\end{Def}


\begin{Def}
Una modificaci\'on medible de un proceso $\left\{V\left(t\right),t\geq0\right\}$, es una versi\'on de este, $\left\{V\left(t,w\right)\right\}$ conjuntamente medible para $t\geq0$ y para $w\in S$, $S$ espacio de estados para $\left\{V\left(t\right),t\geq0\right\}$.
\end{Def}

\begin{Teo}
Sea $\left\{V\left(t\right),t\geq\right\}$ un proceso regenerativo no negativo con modificaci\'on medible. Sea $\esp\left[X\right]<\infty$. Entonces el proceso estacionario dado por la ecuaci\'on anterior est\'a bien definido y tiene funci\'on de distribuci\'on independiente de $t$, adem\'as
\begin{itemize}
\item[i)] \begin{eqnarray*}
\esp\left[V^{*}\left(0\right)\right]&=&\frac{\esp\left[\int_{0}^{X}V\left(s\right)ds\right]}{\esp\left[X\right]}\end{eqnarray*}
\item[ii)] Si $\esp\left[V^{*}\left(0\right)\right]<\infty$, equivalentemente, si $\esp\left[\int_{0}^{X}V\left(s\right)ds\right]<\infty$,entonces
\begin{eqnarray*}
\frac{\int_{0}^{t}V\left(s\right)ds}{t}\rightarrow\frac{\esp\left[\int_{0}^{X}V\left(s\right)ds\right]}{\esp\left[X\right]}
\end{eqnarray*}
con probabilidad 1 y en media, cuando $t\rightarrow\infty$.
\end{itemize}
\end{Teo}

Sea la funci\'on generadora de momentos para $L_{i}$, el n\'umero de usuarios en la cola $Q_{i}\left(z\right)$ en cualquier momento, est\'a dada por el tiempo promedio de $z^{L_{i}\left(t\right)}$ sobre el ciclo regenerativo definido anteriormente. Entonces 



Es decir, es posible determinar las longitudes de las colas a cualquier tiempo $t$. Entonces, determinando el primer momento es posible ver que


\begin{Def}
El tiempo de Ciclo $C_{i}$ es el periodo de tiempo que comienza cuando la cola $i$ es visitada por primera vez en un ciclo, y termina cuando es visitado nuevamente en el pr\'oximo ciclo. La duraci\'on del mismo est\'a dada por $\tau_{i}\left(m+1\right)-\tau_{i}\left(m\right)$, o equivalentemente $\overline{\tau}_{i}\left(m+1\right)-\overline{\tau}_{i}\left(m\right)$ bajo condiciones de estabilidad.
\end{Def}


\begin{Def}
El tiempo de intervisita $I_{i}$ es el periodo de tiempo que comienza cuando se ha completado el servicio en un ciclo y termina cuando es visitada nuevamente en el pr\'oximo ciclo. Su  duraci\'on del mismo est\'a dada por $\tau_{i}\left(m+1\right)-\overline{\tau}_{i}\left(m\right)$.
\end{Def}

La duraci\'on del tiempo de intervisita es $\tau_{i}\left(m+1\right)-\overline{\tau}\left(m\right)$. Dado que el n\'umero de usuarios presentes en $Q_{i}$ al tiempo $t=\tau_{i}\left(m+1\right)$ es igual al n\'umero de arribos durante el intervalo de tiempo $\left[\overline{\tau}\left(m\right),\tau_{i}\left(m+1\right)\right]$ se tiene que


\begin{eqnarray*}
\esp\left[z_{i}^{L_{i}\left(\tau_{i}\left(m+1\right)\right)}\right]=\esp\left[\left\{P_{i}\left(z_{i}\right)\right\}^{\tau_{i}\left(m+1\right)-\overline{\tau}\left(m\right)}\right]
\end{eqnarray*}

entonces, si $I_{i}\left(z\right)=\esp\left[z^{\tau_{i}\left(m+1\right)-\overline{\tau}\left(m\right)}\right]$
se tiene que $F_{i}\left(z\right)=I_{i}\left[P_{i}\left(z\right)\right]$
para $i=1,2$.

Conforme a la definici\'on dada al principio del cap\'itulo, definici\'on (\ref{Def.Tn}), sean $T_{1},T_{2},\ldots$ los puntos donde las longitudes de las colas de la red de sistemas de visitas c\'iclicas son cero simult\'aneamente, cuando la cola $Q_{j}$ es visitada por el servidor para dar servicio, es decir, $L_{1}\left(T_{i}\right)=0,L_{2}\left(T_{i}\right)=0,\hat{L}_{1}\left(T_{i}\right)=0$ y $\hat{L}_{2}\left(T_{i}\right)=0$, a estos puntos se les denominar\'a puntos regenerativos. Entonces, 

\begin{Def}
Al intervalo de tiempo entre dos puntos regenerativos se le llamar\'a ciclo regenerativo.
\end{Def}

\begin{Def}
Para $T_{i}$ se define, $M_{i}$, el n\'umero de ciclos de visita a la cola $Q_{l}$, durante el ciclo regenerativo, es decir, $M_{i}$ es un proceso de renovaci\'on.
\end{Def}

\begin{Def}
Para cada uno de los $M_{i}$'s, se definen a su vez la duraci\'on de cada uno de estos ciclos de visita en el ciclo regenerativo, $C_{i}^{(m)}$, para $m=1,2,\ldots,M_{i}$, que a su vez, tambi\'en es n proceso de renovaci\'on.
\end{Def}

\footnote{In Stidham and  Heyman \cite{Stidham} shows that is sufficient for the regenerative process to be stationary that the mean regenerative cycle time is finite: $\esp\left[\sum_{m=1}^{M_{i}}C_{i}^{(m)}\right]<\infty$, 


 como cada $C_{i}^{(m)}$ contiene intervalos de r\'eplica positivos, se tiene que $\esp\left[M_{i}\right]<\infty$, adem\'as, como $M_{i}>0$, se tiene que la condici\'on anterior es equivalente a tener que $\esp\left[C_{i}\right]<\infty$,
por lo tanto una condici\'on suficiente para la existencia del proceso regenerativo est\'a dada por $\sum_{k=1}^{N}\mu_{k}<1.$}

Para $\left\{X\left(t\right):t\geq0\right\}$ Proceso Estoc\'astico a tiempo continuo con estado de espacios $S$, que es un espacio m\'etrico, con trayectorias continuas por la derecha y con l\'imites por la izquierda c.s. Sea $N\left(t\right)$ un proceso de renovaci\'on en $\rea_{+}$ definido en el mismo espacio de probabilidad que $X\left(t\right)$, con tiempos de renovaci\'on $T$ y tiempos de inter-renovaci\'on $\xi_{n}=T_{n}-T_{n-1}$, con misma distribuci\'on $F$ de media finita $\mu$.

\begin{Def}
Un elemento aleatorio en un espacio medible $\left(E,\mathcal{E}\right)$ en un espacio de probabilidad $\left(\Omega,\mathcal{F},\prob\right)$ a $\left(E,\mathcal{E}\right)$, es decir,
para $A\in \mathcal{E}$,  se tiene que $\left\{Y\in A\right\}\in\mathcal{F}$, donde $\left\{Y\in A\right\}:=\left\{w\in\Omega:Y\left(w\right)\in A\right\}=:Y^{-1}A$.
\end{Def}

\begin{Note}
Tambi\'en se dice que $Y$ est\'a soportado por el espacio de probabilidad $\left(\Omega,\mathcal{F},\prob\right)$ y que $Y$ es un mapeo medible de $\Omega$ en $E$, es decir, es $\mathcal{F}/\mathcal{E}$ medible.
\end{Note}

\begin{Def}
Para cada $i\in \mathbb{I}$ sea $P_{i}$ una medida de probabilidad en un espacio medible $\left(E_{i},\mathcal{E}_{i}\right)$. Se define el espacio producto
$\otimes_{i\in\mathbb{I}}\left(E_{i},\mathcal{E}_{i}\right):=\left(\prod_{i\in\mathbb{I}}E_{i},\otimes_{i\in\mathbb{I}}\mathcal{E}_{i}\right)$, donde $\prod_{i\in\mathbb{I}}E_{i}$ es el producto cartesiano de los $E_{i}$'s, y $\otimes_{i\in\mathbb{I}}\mathcal{E}_{i}$ es la $\sigma$-\'algebra producto, es decir, es la $\sigma$-\'algebra m\'as peque\~na en $\prod_{i\in\mathbb{I}}E_{i}$ que hace al $i$-\'esimo mapeo proyecci\'on en $E_{i}$ medible para toda $i\in\mathbb{I}$ es la $\sigma$-\'algebra inducida por los mapeos proyecci\'on. $$\otimes_{i\in\mathbb{I}}\mathcal{E}_{i}:=\sigma\left\{\left\{y:y_{i}\in A\right\}:i\in\mathbb{I}\textrm{ y }A\in\mathcal{E}_{i}\right\}.$$
\end{Def}

\begin{Def}
Un espacio de probabilidad $\left(\tilde{\Omega},\tilde{\mathcal{F}},\tilde{\prob}\right)$ es una extensi\'on de otro espacio de probabilidad $\left(\Omega,\mathcal{F},\prob\right)$ si $\left(\tilde{\Omega},\tilde{\mathcal{F}},\tilde{\prob}\right)$ soporta un elemento aleatorio $\xi\in\left(\Omega,\mathcal{F}\right)$ que tienen a $\prob$ como distribuci\'on.
\end{Def}

\begin{Teo}
Sea $\mathbb{I}$ un conjunto de \'indices arbitrario. Para cada $i\in\mathbb{I}$ sea $P_{i}$ una medida de probabilidad en un espacio medible $\left(E_{i},\mathcal{E}_{i}\right)$. Entonces existe una \'unica medida de probabilidad $\otimes_{i\in\mathbb{I}}P_{i}$ en $\otimes_{i\in\mathbb{I}}\left(E_{i},\mathcal{E}_{i}\right)$ tal que 

\begin{eqnarray*}
\otimes_{i\in\mathbb{I}}P_{i}\left(y\in\prod_{i\in\mathbb{I}}E_{i}:y_{i}\in A_{i_{1}},\ldots,y_{n}\in A_{i_{n}}\right)=P_{i_{1}}\left(A_{i_{n}}\right)\cdots P_{i_{n}}\left(A_{i_{n}}\right)
\end{eqnarray*}
para todos los enteros $n>0$, toda $i_{1},\ldots,i_{n}\in\mathbb{I}$ y todo $A_{i_{1}}\in\mathcal{E}_{i_{1}},\ldots,A_{i_{n}}\in\mathcal{E}_{i_{n}}$
\end{Teo}

La medida $\otimes_{i\in\mathbb{I}}P_{i}$ es llamada la medida producto y $\otimes_{i\in\mathbb{I}}\left(E_{i},\mathcal{E}_{i},P_{i}\right):=\left(\prod_{i\in\mathbb{I}},E_{i},\otimes_{i\in\mathbb{I}}\mathcal{E}_{i},\otimes_{i\in\mathbb{I}}P_{i}\right)$, es llamado espacio de probabilidad producto.


\begin{Def}
Un espacio medible $\left(E,\mathcal{E}\right)$ es \textit{Polaco} si existe una m\'etrica en $E$ tal que $E$ es completo, es decir cada sucesi\'on de Cauchy converge a un l\'imite en $E$, y \textit{separable}, $E$ tienen un subconjunto denso numerable, y tal que $\mathcal{E}$ es generado por conjuntos abiertos.
\end{Def}


\begin{Def}
Dos espacios medibles $\left(E,\mathcal{E}\right)$ y $\left(G,\mathcal{G}\right)$ son Borel equivalentes \textit{isomorfos} si existe una biyecci\'on $f:E\rightarrow G$ tal que $f$ es $\mathcal{E}/\mathcal{G}$ medible y su inversa $f^{-1}$ es $\mathcal{G}/\mathcal{E}$ medible. La biyecci\'on es una equivalencia de Borel.
\end{Def}

\begin{Def}
Un espacio medible  $\left(E,\mathcal{E}\right)$ es un \textit{espacio est\'andar} si es Borel equivalente a $\left(G,\mathcal{G}\right)$, donde $G$ es un subconjunto de Borel de $\left[0,1\right]$ y $\mathcal{G}$ son los subconjuntos de Borel de $G$.
\end{Def}

\begin{Note}
Cualquier espacio Polaco es un espacio est\'andar.
\end{Note}


\begin{Def}
Un proceso estoc\'astico con conjunto de \'indices $\mathbb{I}$ y espacio de estados $\left(E,\mathcal{E}\right)$ es una familia $Z=\left(\mathbb{Z}_{s}\right)_{s\in\mathbb{I}}$ donde $\mathbb{Z}_{s}$ son elementos aleatorios definidos en un espacio de probabilidad com\'un $\left(\Omega,\mathcal{F},\prob\right)$ y todos toman valores en $\left(E,\mathcal{E}\right)$.
\end{Def}

\begin{Def}
Un proceso estoc\'astico \textit{one-sided contiuous time} (\textbf{PEOSCT}) es un proceso estoc\'astico con conjunto de \'indices $\mathbb{I}=\left[0,\infty\right)$.
\end{Def}


Sea $\left(E^{\mathbb{I}},\mathcal{E}^{\mathbb{I}}\right)$ denota el espacio producto $\left(E^{\mathbb{I}},\mathcal{E}^{\mathbb{I}}\right):=\otimes_{s\in\mathbb{I}}\left(E,\mathcal{E}\right)$. Vamos a considerar $\mathbb{Z}$ como un mapeo aleatorio, es decir, como un elemento aleatorio en $\left(E^{\mathbb{I}},\mathcal{E}^{\mathbb{I}}\right)$ definido por $Z\left(w\right)=\left(Z_{s}\left(w\right)\right)_{s\in\mathbb{I}}$ y $w\in\Omega$.

\begin{Note}
La distribuci\'on de un proceso estoc\'astico $Z$ es la distribuci\'on de $Z$ como un elemento aleatorio en $\left(E^{\mathbb{I}},\mathcal{E}^{\mathbb{I}}\right)$. La distribuci\'on de $Z$ esta determinada de manera \'unica por las distribuciones finito dimensionales.
\end{Note}

\begin{Note}
En particular cuando $Z$ toma valores reales, es decir, $\left(E,\mathcal{E}\right)=\left(\mathbb{R},\mathcal{B}\right)$ las distribuciones finito dimensionales est\'an determinadas por las funciones de distribuci\'on finito dimensionales

\begin{eqnarray}
\prob\left(Z_{t_{1}}\leq x_{1},\ldots,Z_{t_{n}}\leq x_{n}\right),x_{1},\ldots,x_{n}\in\mathbb{R},t_{1},\ldots,t_{n}\in\mathbb{I},n\geq1.
\end{eqnarray}
\end{Note}

\begin{Note}
Para espacios polacos $\left(E,\mathcal{E}\right)$ el Teorema de Consistencia de Kolmogorov asegura que dada una colecci\'on de distribuciones finito dimensionales consistentes, siempre existe un proceso estoc\'astico que posee tales distribuciones finito dimensionales.
\end{Note}


\begin{Def}
Las trayectorias de $Z$ son las realizaciones $Z\left(w\right)$ para $w\in\Omega$ del mapeo aleatorio $Z$.
\end{Def}

\begin{Note}
Algunas restricciones se imponen sobre las trayectorias, por ejemplo que sean continuas por la derecha, o continuas por la derecha con l\'imites por la izquierda, o de manera m\'as general, se pedir\'a que caigan en alg\'un subconjunto $H$ de $E^{\mathbb{I}}$. En este caso es natural considerar a $Z$ como un elemento aleatorio que no est\'a en $\left(E^{\mathbb{I}},\mathcal{E}^{\mathbb{I}}\right)$ sino en $\left(H,\mathcal{H}\right)$, donde $\mathcal{H}$ es la $\sigma$-\'algebra generada por los mapeos proyecci\'on que toman a $z\in H$ a $z_{t}\in E$ para $t\in\mathbb{I}$. A $\mathcal{H}$ se le conoce como la traza de $H$ en $E^{\mathbb{I}}$, es decir,
\begin{eqnarray}
\mathcal{H}:=E^{\mathbb{I}}\cap H:=\left\{A\cap H:A\in E^{\mathbb{I}}\right\}.
\end{eqnarray}
\end{Note}


\begin{Note}
$Z$ tiene trayectorias con valores en $H$ y cada $Z_{t}$ es un mapeo medible de $\left(\Omega,\mathcal{F}\right)$ a $\left(H,\mathcal{H}\right)$. Cuando se considera un espacio de trayectorias en particular $H$, al espacio $\left(H,\mathcal{H}\right)$ se le llama el espacio de trayectorias de $Z$.
\end{Note}

\begin{Note}
La distribuci\'on del proceso estoc\'astico $Z$ con espacio de trayectorias $\left(H,\mathcal{H}\right)$ es la distribuci\'on de $Z$ como  un elemento aleatorio en $\left(H,\mathcal{H}\right)$. La distribuci\'on, nuevemente, est\'a determinada de manera \'unica por las distribuciones finito dimensionales.
\end{Note}


\begin{Def}
Sea $Z$ un PEOSCT  con espacio de estados $\left(E,\mathcal{E}\right)$ y sea $T$ un tiempo aleatorio en $\left[0,\infty\right)$. Por $Z_{T}$ se entiende el mapeo con valores en $E$ definido en $\Omega$ en la manera obvia:
\begin{eqnarray*}
Z_{T}\left(w\right):=Z_{T\left(w\right)}\left(w\right). w\in\Omega.
\end{eqnarray*}
\end{Def}

\begin{Def}
Un PEOSCT $Z$ es conjuntamente medible (\textbf{CM}) si el mapeo que toma $\left(w,t\right)\in\Omega\times\left[0,\infty\right)$ a $Z_{t}\left(w\right)\in E$ es $\mathcal{F}\otimes\mathcal{B}\left[0,\infty\right)/\mathcal{E}$ medible.
\end{Def}

\begin{Note}
Un PEOSCT-CM implica que el proceso es medible, dado que $Z_{T}$ es una composici\'on  de dos mapeos continuos: el primero que toma $w$ en $\left(w,T\left(w\right)\right)$ es $\mathcal{F}/\mathcal{F}\otimes\mathcal{B}\left[0,\infty\right)$ medible, mientras que el segundo toma $\left(w,T\left(w\right)\right)$ en $Z_{T\left(w\right)}\left(w\right)$ es $\mathcal{F}\otimes\mathcal{B}\left[0,\infty\right)/\mathcal{E}$ medible.
\end{Note}


\begin{Def}
Un PEOSCT con espacio de estados $\left(H,\mathcal{H}\right)$ es can\'onicamente conjuntamente medible (\textbf{CCM}) si el mapeo $\left(z,t\right)\in H\times\left[0,\infty\right)$ en $Z_{t}\in E$ es $\mathcal{H}\otimes\mathcal{B}\left[0,\infty\right)/\mathcal{E}$ medible.
\end{Def}

\begin{Note}
Un PEOSCTCCM implica que el proceso es CM, dado que un PECCM $Z$ es un mapeo de $\Omega\times\left[0,\infty\right)$ a $E$, es la composici\'on de dos mapeos medibles: el primero, toma $\left(w,t\right)$ en $\left(Z\left(w\right),t\right)$ es $\mathcal{F}\otimes\mathcal{B}\left[0,\infty\right)/\mathcal{H}\otimes\mathcal{B}\left[0,\infty\right)$ medible, y el segundo que toma $\left(Z\left(w\right),t\right)$  en $Z_{t}\left(w\right)$ es $\mathcal{H}\otimes\mathcal{B}\left[0,\infty\right)/\mathcal{E}$ medible. Por tanto CCM es una condici\'on m\'as fuerte que CM.
\end{Note}

\begin{Def}
Un conjunto de trayectorias $H$ de un PEOSCT $Z$, es internamente shift-invariante (\textbf{ISI}) si 
\begin{eqnarray*}
\left\{\left(z_{t+s}\right)_{s\in\left[0,\infty\right)}:z\in H\right\}=H\textrm{, }t\in\left[0,\infty\right).
\end{eqnarray*}
\end{Def}


\begin{Def}
Dado un PEOSCTISI, se define el mapeo-shift $\theta_{t}$, $t\in\left[0,\infty\right)$, de $H$ a $H$ por 
\begin{eqnarray*}
\theta_{t}z=\left(z_{t+s}\right)_{s\in\left[0,\infty\right)}\textrm{, }z\in H.
\end{eqnarray*}
\end{Def}

\begin{Def}
Se dice que un proceso $Z$ es shift-medible (\textbf{SM}) si $Z$ tiene un conjunto de trayectorias $H$ que es ISI y adem\'as el mapeo que toma $\left(z,t\right)\in H\times\left[0,\infty\right)$ en $\theta_{t}z\in H$ es $\mathcal{H}\otimes\mathcal{B}\left[0,\infty\right)/\mathcal{H}$ medible.
\end{Def}

\begin{Note}
Un proceso estoc\'astico con conjunto de trayectorias $H$ ISI es shift-medible si y s\'olo si es CCM
\end{Note}

\begin{Note}
\begin{itemize}
\item Dado el espacio polaco $\left(E,\mathcal{E}\right)$ se tiene el  conjunto de trayectorias $D_{E}\left[0,\infty\right)$ que es ISI, entonces cumpe con ser CCM.

\item Si $G$ es abierto, podemos cubrirlo por bolas abiertas cuay cerradura este contenida en $G$, y como $G$ es segundo numerable como subespacio de $E$, lo podemos cubrir por una cantidad numerable de bolas abiertas.

\end{itemize}
\end{Note}


\begin{Note}
Los procesos estoc\'asticos $Z$ a tiempo discreto con espacio de estados polaco, tambi\'en tiene un espacio de trayectorias polaco y por tanto tiene distribuciones condicionales regulares.
\end{Note}

\begin{Teo}
El producto numerable de espacios polacos es polaco.
\end{Teo}


\begin{Def}
Sea $\left(\Omega,\mathcal{F},\prob\right)$ espacio de probabilidad que soporta al proceso $Z=\left(Z_{s}\right)_{s\in\left[0,\infty\right)}$ y $S=\left(S_{k}\right)_{0}^{\infty}$ donde $Z$ es un PEOSCTM con espacio de estados $\left(E,\mathcal{E}\right)$  y espacio de trayectorias $\left(H,\mathcal{H}\right)$  y adem\'as $S$ es una sucesi\'on de tiempos aleatorios one-sided que satisfacen la condici\'on $0\leq S_{0}<S_{1}<\cdots\rightarrow\infty$. Considerando $S$ como un mapeo medible de $\left(\Omega,\mathcal{F}\right)$ al espacio sucesi\'on $\left(L,\mathcal{L}\right)$, donde 
\begin{eqnarray*}
L=\left\{\left(s_{k}\right)_{0}^{\infty}\in\left[0,\infty\right)^{\left\{0,1,\ldots\right\}}:s_{0}<s_{1}<\cdots\rightarrow\infty\right\},
\end{eqnarray*}
donde $\mathcal{L}$ son los subconjuntos de Borel de $L$, es decir, $\mathcal{L}=L\cap\mathcal{B}^{\left\{0,1,\ldots\right\}}$.

As\'i el par $\left(Z,S\right)$ es un mapeo medible de  $\left(\Omega,\mathcal{F}\right)$ en $\left(H\times L,\mathcal{H}\otimes\mathcal{L}\right)$. El par $\mathcal{H}\otimes\mathcal{L}^{+}$ denotar\'a la clase de todas las funciones medibles de $\left(H\times L,\mathcal{H}\otimes\mathcal{L}\right)$ en $\left(\left[0,\infty\right),\mathcal{B}\left[0,\infty\right)\right)$.
\end{Def}


\begin{Def}
Sea $\theta_{t}$ el mapeo-shift conjunto de $H\times L$ en $H\times L$ dado por
\begin{eqnarray*}
\theta_{t}\left(z,\left(s_{k}\right)_{0}^{\infty}\right)=\theta_{t}\left(z,\left(s_{n_{t-}+k}-t\right)_{0}^{\infty}\right)
\end{eqnarray*}
donde 
$n_{t-}=inf\left\{n\geq1:s_{n}\geq t\right\}$.
\end{Def}

\begin{Note}
Con la finalidad de poder realizar los shift's sin complicaciones de medibilidad, se supondr\'a que $Z$ es shit-medible, es decir, el conjunto de trayectorias $H$ es invariante bajo shifts del tiempo y el mapeo que toma $\left(z,t\right)\in H\times\left[0,\infty\right)$ en $z_{t}\in E$ es $\mathcal{H}\otimes\mathcal{B}\left[0,\infty\right)/\mathcal{E}$ medible.
\end{Note}

\begin{Def}
Dado un proceso \textbf{PEOSSM} (Proceso Estoc\'astico One Side Shift Medible) $Z$, se dice regenerativo cl\'asico con tiempos de regeneraci\'on $S$ si 

\begin{eqnarray*}
\theta_{S_{n}}\left(Z,S\right)=\left(Z^{0},S^{0}\right),n\geq0
\end{eqnarray*}
y adem\'as $\theta_{S_{n}}\left(Z,S\right)$ es independiente de $\left(\left(Z_{s}\right)s\in\left[0,S_{n}\right),S_{0},\ldots,S_{n}\right)$
Si lo anterior se cumple, al par $\left(Z,S\right)$ se le llama regenerativo cl\'asico.
\end{Def}

\begin{Note}
Si el par $\left(Z,S\right)$ es regenerativo cl\'asico, entonces las longitudes de los ciclos $X_{1},X_{2},\ldots,$ son i.i.d. e independientes de la longitud del retraso $S_{0}$, es decir, $S$ es un proceso de renovaci\'on. Las longitudes de los ciclos tambi\'en son llamados tiempos de inter-regeneraci\'on y tiempos de ocurrencia.

\end{Note}

\begin{Teo}
Sup\'ongase que el par $\left(Z,S\right)$ es regenerativo cl\'asico con $\esp\left[X_{1}\right]<\infty$. Entonces $\left(Z^{*},S^{*}\right)$ en el teorema 2.1 es una versi\'on estacionaria de $\left(Z,S\right)$. Adem\'as, si $X_{1}$ es lattice con span $d$, entonces $\left(Z^{**},S^{**}\right)$ en el teorema 2.2 es una versi\'on periodicamente estacionaria de $\left(Z,S\right)$ con periodo $d$.

\end{Teo}

\begin{Def}
Una variable aleatoria $X_{1}$ es \textit{spread out} si existe una $n\geq1$ y una  funci\'on $f\in\mathcal{B}^{+}$ tal que $\int_{\rea}f\left(x\right)dx>0$ con $X_{2},X_{3},\ldots,X_{n}$ copias i.i.d  de $X_{1}$, $$\prob\left(X_{1}+\cdots+X_{n}\in B\right)\geq\int_{B}f\left(x\right)dx$$ para $B\in\mathcal{B}$.

\end{Def}



\begin{Def}
Dado un proceso estoc\'astico $Z$ se le llama \textit{wide-sense regenerative} (\textbf{WSR}) con tiempos de regeneraci\'on $S$ si $\theta_{S_{n}}\left(Z,S\right)=\left(Z^{0},S^{0}\right)$ para $n\geq0$ en distribuci\'on y $\theta_{S_{n}}\left(Z,S\right)$ es independiente de $\left(S_{0},S_{1},\ldots,S_{n}\right)$ para $n\geq0$.
Se dice que el par $\left(Z,S\right)$ es WSR si lo anterior se cumple.
\end{Def}


\begin{Note}
\begin{itemize}
\item El proceso de trayectorias $\left(\theta_{s}Z\right)_{s\in\left[0,\infty\right)}$ es WSR con tiempos de regeneraci\'on $S$ pero no es regenerativo cl\'asico.

\item Si $Z$ es cualquier proceso estacionario y $S$ es un proceso de renovaci\'on que es independiente de $Z$, entonces $\left(Z,S\right)$ es WSR pero en general no es regenerativo cl\'asico

\end{itemize}

\end{Note}


\begin{Note}
Para cualquier proceso estoc\'astico $Z$, el proceso de trayectorias $\left(\theta_{s}Z\right)_{s\in\left[0,\infty\right)}$ es siempre un proceso de Markov.
\end{Note}



\begin{Teo}
Supongase que el par $\left(Z,S\right)$ es WSR con $\esp\left[X_{1}\right]<\infty$. Entonces $\left(Z^{*},S^{*}\right)$ en el teorema 2.1 es una versi\'on estacionaria de 
$\left(Z,S\right)$.
\end{Teo}


\begin{Teo}
Supongase que $\left(Z,S\right)$ es cycle-stationary con $\esp\left[X_{1}\right]<\infty$. Sea $U$ distribuida uniformemente en $\left[0,1\right)$ e independiente de $\left(Z^{0},S^{0}\right)$ y sea $\prob^{*}$ la medida de probabilidad en $\left(\Omega,\prob\right)$ definida por $$d\prob^{*}=\frac{X_{1}}{\esp\left[X_{1}\right]}d\prob$$. Sea $\left(Z^{*},S^{*}\right)$ con distribuci\'on $\prob^{*}\left(\theta_{UX_{1}}\left(Z^{0},S^{0}\right)\in\cdot\right)$. Entonces $\left(Z^{}*,S^{*}\right)$ es estacionario,
\begin{eqnarray*}
\esp\left[f\left(Z^{*},S^{*}\right)\right]=\esp\left[\int_{0}^{X_{1}}f\left(\theta_{s}\left(Z^{0},S^{0}\right)\right)ds\right]/\esp\left[X_{1}\right]
\end{eqnarray*}
$f\in\mathcal{H}\otimes\mathcal{L}^{+}$, and $S_{0}^{*}$ es continuo con funci\'on distribuci\'on $G_{\infty}$ definida por $$G_{\infty}\left(x\right):=\frac{\esp\left[X_{1}\right]\wedge x}{\esp\left[X_{1}\right]}$$ para $x\geq0$ y densidad $\prob\left[X_{1}>x\right]/\esp\left[X_{1}\right]$, con $x\geq0$.

\end{Teo}


\begin{Teo}
Sea $Z$ un Proceso Estoc\'astico un lado shift-medible \textit{one-sided shift-measurable stochastic process}, (PEOSSM),
y $S_{0}$ y $S_{1}$ tiempos aleatorios tales que $0\leq S_{0}<S_{1}$ y
\begin{equation}
\theta_{S_{1}}Z=\theta_{S_{0}}Z\textrm{ en distribuci\'on}.
\end{equation}

Entonces el espacio de probabilidad subyacente $\left(\Omega,\mathcal{F},\prob\right)$ puede extenderse para soportar una sucesi\'on de tiempos aleatorios $S$ tales que

\begin{eqnarray}
\theta_{S_{n}}\left(Z,S\right)=\left(Z^{0},S^{0}\right),n\geq0,\textrm{ en distribuci\'on},\\
\left(Z,S_{0},S_{1}\right)\textrm{ depende de }\left(X_{2},X_{3},\ldots\right)\textrm{ solamente a traves de }\theta_{S_{1}}Z.
\end{eqnarray}
\end{Teo}





%_________________________________________________________________________
%
%\subsection{Una vez que se tiene estabilidad}
%_________________________________________________________________________
%

Also the intervisit time $I_{i}$ is defined as the period beginning at the time of its service completion in a cycle and ending at the time when it is polled in the next cycle; its duration is given by $\tau_{i}\left(m+1\right)-\overline{\tau}_{i}\left(m\right)$.

So we the following are still true 

\begin{eqnarray}
\begin{array}{ll}
\esp\left[L_{i}\right]=\mu_{i}\esp\left[I_{i}\right], &
\esp\left[C_{i}\right]=\frac{f_{i}\left(i\right)}{\mu_{i}\left(1-\mu_{i}\right)},\\
\esp\left[S_{i}\right]=\mu_{i}\esp\left[C_{i}\right],&
\esp\left[I_{i}\right]=\left(1-\mu_{i}\right)\esp\left[C_{i}\right],\\
Var\left[L_{i}\right]= \mu_{i}^{2}Var\left[I_{i}\right]+\sigma^{2}\esp\left[I_{i}\right],& 
Var\left[C_{i}\right]=\frac{Var\left[L_{i}^{*}\right]}{\mu_{i}^{2}\left(1-\mu_{i}\right)^{2}},\\
Var\left[S_{i}\right]= \frac{Var\left[L_{i}^{*}\right]}{\left(1-\mu_{i}\right)^{2}}+\frac{\sigma^{2}\esp\left[L_{i}^{*}\right]}{\left(1-\mu_{i}\right)^{3}},&
Var\left[I_{i}\right]= \frac{Var\left[L_{i}^{*}\right]}{\mu_{i}^{2}}-\frac{\sigma_{i}^{2}}{\mu_{i}^{2}}f_{i}\left(i\right).
\end{array}
\end{eqnarray}
\begin{Def}
El tiempo de Ciclo $C_{i}$ es el periodo de tiempo que comienza cuando la cola $i$ es visitada por primera vez en un ciclo, y termina cuando es visitado nuevamente en el pr\'oximo ciclo. La duraci\'on del mismo est\'a dada por $\tau_{i}\left(m+1\right)-\tau_{i}\left(m\right)$, o equivalentemente $\overline{\tau}_{i}\left(m+1\right)-\overline{\tau}_{i}\left(m\right)$ bajo condiciones de estabilidad.
\end{Def}


\begin{Def}
El tiempo de intervisita $I_{i}$ es el periodo de tiempo que comienza cuando se ha completado el servicio en un ciclo y termina cuando es visitada nuevamente en el pr\'oximo ciclo. Su  duraci\'on del mismo est\'a dada por $\tau_{i}\left(m+1\right)-\overline{\tau}_{i}\left(m\right)$.
\end{Def}

La duraci\'on del tiempo de intervisita es $\tau_{i}\left(m+1\right)-\overline{\tau}\left(m\right)$. Dado que el n\'umero de usuarios presentes en $Q_{i}$ al tiempo $t=\tau_{i}\left(m+1\right)$ es igual al n\'umero de arribos durante el intervalo de tiempo $\left[\overline{\tau}\left(m\right),\tau_{i}\left(m+1\right)\right]$ se tiene que


\begin{eqnarray*}
\esp\left[z_{i}^{L_{i}\left(\tau_{i}\left(m+1\right)\right)}\right]=\esp\left[\left\{P_{i}\left(z_{i}\right)\right\}^{\tau_{i}\left(m+1\right)-\overline{\tau}\left(m\right)}\right]
\end{eqnarray*}

entonces, si $I_{i}\left(z\right)=\esp\left[z^{\tau_{i}\left(m+1\right)-\overline{\tau}\left(m\right)}\right]$
se tiene que $F_{i}\left(z\right)=I_{i}\left[P_{i}\left(z\right)\right]$
para $i=1,2$.

Conforme a la definici\'on dada al principio del cap\'itulo, definici\'on (\ref{Def.Tn}), sean $T_{1},T_{2},\ldots$ los puntos donde las longitudes de las colas de la red de sistemas de visitas c\'iclicas son cero simult\'aneamente, cuando la cola $Q_{j}$ es visitada por el servidor para dar servicio, es decir, $L_{1}\left(T_{i}\right)=0,L_{2}\left(T_{i}\right)=0,\hat{L}_{1}\left(T_{i}\right)=0$ y $\hat{L}_{2}\left(T_{i}\right)=0$, a estos puntos se les denominar\'a puntos regenerativos. Entonces, 

\begin{Def}
Al intervalo de tiempo entre dos puntos regenerativos se le llamar\'a ciclo regenerativo.
\end{Def}

\begin{Def}
Para $T_{i}$ se define, $M_{i}$, el n\'umero de ciclos de visita a la cola $Q_{l}$, durante el ciclo regenerativo, es decir, $M_{i}$ es un proceso de renovaci\'on.
\end{Def}

\begin{Def}
Para cada uno de los $M_{i}$'s, se definen a su vez la duraci\'on de cada uno de estos ciclos de visita en el ciclo regenerativo, $C_{i}^{(m)}$, para $m=1,2,\ldots,M_{i}$, que a su vez, tambi\'en es n proceso de renovaci\'on.
\end{Def}


Sea la funci\'on generadora de momentos para $L_{i}$, el n\'umero de usuarios en la cola $Q_{i}\left(z\right)$ en cualquier momento, est\'a dada por el tiempo promedio de $z^{L_{i}\left(t\right)}$ sobre el ciclo regenerativo definido anteriormente:

\begin{eqnarray*}
Q_{i}\left(z\right)&=&\esp\left[z^{L_{i}\left(t\right)}\right]=\frac{\esp\left[\sum_{m=1}^{M_{i}}\sum_{t=\tau_{i}\left(m\right)}^{\tau_{i}\left(m+1\right)-1}z^{L_{i}\left(t\right)}\right]}{\esp\left[\sum_{m=1}^{M_{i}}\tau_{i}\left(m+1\right)-\tau_{i}\left(m\right)\right]}
\end{eqnarray*}

$M_{i}$ es un tiempo de paro en el proceso regenerativo con $\esp\left[M_{i}\right]<\infty$\footnote{En Stidham\cite{Stidham} y Heyman  se muestra que una condici\'on suficiente para que el proceso regenerativo 
estacionario sea un procesoo estacionario es que el valor esperado del tiempo del ciclo regenerativo sea finito, es decir: $\esp\left[\sum_{m=1}^{M_{i}}C_{i}^{(m)}\right]<\infty$, como cada $C_{i}^{(m)}$ contiene intervalos de r\'eplica positivos, se tiene que $\esp\left[M_{i}\right]<\infty$, adem\'as, como $M_{i}>0$, se tiene que la condici\'on anterior es equivalente a tener que $\esp\left[C_{i}\right]<\infty$,
por lo tanto una condici\'on suficiente para la existencia del proceso regenerativo est\'a dada por $\sum_{k=1}^{N}\mu_{k}<1.$}, se sigue del lema de Wald que:


\begin{eqnarray*}
\esp\left[\sum_{m=1}^{M_{i}}\sum_{t=\tau_{i}\left(m\right)}^{\tau_{i}\left(m+1\right)-1}z^{L_{i}\left(t\right)}\right]&=&\esp\left[M_{i}\right]\esp\left[\sum_{t=\tau_{i}\left(m\right)}^{\tau_{i}\left(m+1\right)-1}z^{L_{i}\left(t\right)}\right]\\
\esp\left[\sum_{m=1}^{M_{i}}\tau_{i}\left(m+1\right)-\tau_{i}\left(m\right)\right]&=&\esp\left[M_{i}\right]\esp\left[\tau_{i}\left(m+1\right)-\tau_{i}\left(m\right)\right]
\end{eqnarray*}

por tanto se tiene que


\begin{eqnarray*}
Q_{i}\left(z\right)&=&\frac{\esp\left[\sum_{t=\tau_{i}\left(m\right)}^{\tau_{i}\left(m+1\right)-1}z^{L_{i}\left(t\right)}\right]}{\esp\left[\tau_{i}\left(m+1\right)-\tau_{i}\left(m\right)\right]}
\end{eqnarray*}

observar que el denominador es simplemente la duraci\'on promedio del tiempo del ciclo.


Haciendo las siguientes sustituciones en la ecuaci\'on (\ref{Corolario2}): $n\rightarrow t-\tau_{i}\left(m\right)$, $T \rightarrow \overline{\tau}_{i}\left(m\right)-\tau_{i}\left(m\right)$, $L_{n}\rightarrow L_{i}\left(t\right)$ y $F\left(z\right)=\esp\left[z^{L_{0}}\right]\rightarrow F_{i}\left(z\right)=\esp\left[z^{L_{i}\tau_{i}\left(m\right)}\right]$, se puede ver que

\begin{eqnarray}\label{Eq.Arribos.Primera}
\esp\left[\sum_{n=0}^{T-1}z^{L_{n}}\right]=
\esp\left[\sum_{t=\tau_{i}\left(m\right)}^{\overline{\tau}_{i}\left(m\right)-1}z^{L_{i}\left(t\right)}\right]
=z\frac{F_{i}\left(z\right)-1}{z-P_{i}\left(z\right)}
\end{eqnarray}

Por otra parte durante el tiempo de intervisita para la cola $i$, $L_{i}\left(t\right)$ solamente se incrementa de manera que el incremento por intervalo de tiempo est\'a dado por la funci\'on generadora de probabilidades de $P_{i}\left(z\right)$, por tanto la suma sobre el tiempo de intervisita puede evaluarse como:

\begin{eqnarray*}
\esp\left[\sum_{t=\tau_{i}\left(m\right)}^{\tau_{i}\left(m+1\right)-1}z^{L_{i}\left(t\right)}\right]&=&\esp\left[\sum_{t=\tau_{i}\left(m\right)}^{\tau_{i}\left(m+1\right)-1}\left\{P_{i}\left(z\right)\right\}^{t-\overline{\tau}_{i}\left(m\right)}\right]=\frac{1-\esp\left[\left\{P_{i}\left(z\right)\right\}^{\tau_{i}\left(m+1\right)-\overline{\tau}_{i}\left(m\right)}\right]}{1-P_{i}\left(z\right)}\\
&=&\frac{1-I_{i}\left[P_{i}\left(z\right)\right]}{1-P_{i}\left(z\right)}
\end{eqnarray*}
por tanto

\begin{eqnarray*}
\esp\left[\sum_{t=\tau_{i}\left(m\right)}^{\tau_{i}\left(m+1\right)-1}z^{L_{i}\left(t\right)}\right]&=&
\frac{1-F_{i}\left(z\right)}{1-P_{i}\left(z\right)}
\end{eqnarray*}

Por lo tanto

\begin{eqnarray*}
Q_{i}\left(z\right)&=&\frac{\esp\left[\sum_{t=\tau_{i}\left(m\right)}^{\tau_{i}\left(m+1\right)-1}z^{L_{i}\left(t\right)}\right]}{\esp\left[\tau_{i}\left(m+1\right)-\tau_{i}\left(m\right)\right]}
=\frac{1}{\esp\left[\tau_{i}\left(m+1\right)-\tau_{i}\left(m\right)\right]}
\esp\left[\sum_{t=\tau_{i}\left(m\right)}^{\tau_{i}\left(m+1\right)-1}z^{L_{i}\left(t\right)}\right]\\
&=&\frac{1}{\esp\left[\tau_{i}\left(m+1\right)-\tau_{i}\left(m\right)\right]}
\esp\left[\sum_{t=\tau_{i}\left(m\right)}^{\overline{\tau}_{i}\left(m\right)-1}z^{L_{i}\left(t\right)}
+\sum_{t=\overline{\tau}_{i}\left(m\right)}^{\tau_{i}\left(m+1\right)-1}z^{L_{i}\left(t\right)}\right]\\
&=&\frac{1}{\esp\left[\tau_{i}\left(m+1\right)-\tau_{i}\left(m\right)\right]}\left\{
\esp\left[\sum_{t=\tau_{i}\left(m\right)}^{\overline{\tau}_{i}\left(m\right)-1}z^{L_{i}\left(t\right)}\right]
+\esp\left[\sum_{t=\overline{\tau}_{i}\left(m\right)}^{\tau_{i}\left(m+1\right)-1}z^{L_{i}\left(t\right)}\right]\right\}\\
&=&\frac{1}{\esp\left[\tau_{i}\left(m+1\right)-\tau_{i}\left(m\right)\right]}\left\{
z\frac{F_{i}\left(z\right)-1}{z-P_{i}\left(z\right)}+\frac{1-F_{i}\left(z\right)}{1-P_{i}\left(z\right)}
\right\}\\
&=&\frac{1}{\esp\left[C_{i}\right]}\cdot\frac{1-F_{i}\left(z\right)}{P_{i}\left(z\right)-z}\cdot\frac{\left(1-z\right)P_{i}\left(z\right)}{1-P_{i}\left(z\right)}
\end{eqnarray*}

es decir

\begin{equation}
Q_{i}\left(z\right)=\frac{1}{\esp\left[C_{i}\right]}\cdot\frac{1-F_{i}\left(z\right)}{P_{i}\left(z\right)-z}\cdot\frac{\left(1-z\right)P_{i}\left(z\right)}{1-P_{i}\left(z\right)}
\end{equation}


Si hacemos:

\begin{eqnarray}
S\left(z\right)&=&1-F\left(z\right)\\
T\left(z\right)&=&z-P\left(z\right)\\
U\left(z\right)&=&1-P\left(z\right)
\end{eqnarray}
entonces 

\begin{eqnarray}
\esp\left[C_{i}\right]Q\left(z\right)=\frac{\left(z-1\right)S\left(z\right)P\left(z\right)}{T\left(z\right)U\left(z\right)}
\end{eqnarray}

A saber, si $a_{k}=P\left\{L\left(t\right)=k\right\}$
\begin{eqnarray*}
S\left(z\right)=1-F\left(z\right)=1-\sum_{k=0}^{+\infty}a_{k}z^{k}
\end{eqnarray*}
entonces

%\begin{eqnarray}
%\begin{array}{ll}
%S^{'}\left(z\right)=-\sum_{k=1}^{+\infty}ka_{k}z^{k-1},& %S^{(1)}\left(1\right)=-\sum_{k=1}^{+\infty}ka_{k}=-\esp\left[L\left(t\right)\right],\\
%S^{''}\left(z\right)=-\sum_{k=2}^{+\infty}k(k-1)a_{k}z^{k-2},& S^{(2)}\left(1\right)=-\sum_{k=2}^{+\infty}k(k-1)a_{k}=\esp\left[L\left(L-1\right)\right],\\
%S^{'''}\left(z\right)=-\sum_{k=3}^{+\infty}k(k-1)(k-2)a_{k}z^{k-3},&
%S^{(3)}\left(1\right)=-\sum_{k=3}^{+\infty}k(k-1)(k-2)a_{k}\\
%&=-\esp\left[L\left(L-1\right)\left(L-2\right)\right]\\
%&=-\esp\left[L^{3}\right]+3-\esp\left[L^{2}\right]-2-\esp\left[L\right];
%\end{array}
%\end{eqnarray}

$S^{'}\left(z\right)=-\sum_{k=1}^{+\infty}ka_{k}z^{k-1}$, por tanto $S^{(1)}\left(1\right)=-\sum_{k=1}^{+\infty}ka_{k}=-\esp\left[L\left(t\right)\right]$,
luego $S^{''}\left(z\right)=-\sum_{k=2}^{+\infty}k(k-1)a_{k}z^{k-2}$ y $S^{(2)}\left(1\right)=-\sum_{k=2}^{+\infty}k(k-1)a_{k}=\esp\left[L\left(L-1\right)\right]$;
de la misma manera $S^{'''}\left(z\right)=-\sum_{k=3}^{+\infty}k(k-1)(k-2)a_{k}z^{k-3}$ y $S^{(3)}\left(1\right)=-\sum_{k=3}^{+\infty}k(k-1)(k-2)a_{k}=-\esp\left[L\left(L-1\right)\left(L-2\right)\right]
=-\esp\left[L^{3}\right]+3-\esp\left[L^{2}\right]-2-\esp\left[L\right]$. 

Es decir

\begin{eqnarray*}
S^{(1)}\left(1\right)&=&-\esp\left[L\left(t\right)\right],\\ S^{(2)}\left(1\right)&=&-\esp\left[L\left(L-1\right)\right]
=-\esp\left[L^{2}\right]+\esp\left[L\right],\\
S^{(3)}\left(1\right)&=&-\esp\left[L\left(L-1\right)\left(L-2\right)\right]
=-\esp\left[L^{3}\right]+3\esp\left[L^{2}\right]-2\esp\left[L\right].
\end{eqnarray*}


Expandiendo alrededor de $z=1$

\begin{eqnarray*}
S\left(z\right)&=&S\left(1\right)+\frac{S^{'}\left(1\right)}{1!}\left(z-1\right)+\frac{S^{''}\left(1\right)}{2!}\left(z-1\right)^{2}+\frac{S^{'''}\left(1\right)}{3!}\left(z-1\right)^{3}+\ldots+\\
&=&\left(z-1\right)\left\{S^{'}\left(1\right)+\frac{S^{''}\left(1\right)}{2!}\left(z-1\right)+\frac{S^{'''}\left(1\right)}{3!}\left(z-1\right)^{2}+\ldots+\right\}\\
&=&\left(z-1\right)R_{1}\left(z\right)
\end{eqnarray*}
con $R_{1}\left(z\right)\neq0$, pues

\begin{eqnarray}\label{Eq.R1}
R_{1}\left(z\right)=-\esp\left[L\right]
\end{eqnarray}
entonces

\begin{eqnarray}
R_{1}\left(z\right)&=&S^{'}\left(1\right)+\frac{S^{''}\left(1\right)}{2!}\left(z-1\right)+\frac{S^{'''}\left(1\right)}{3!}\left(z-1\right)^{2}+\frac{S^{iv}\left(1\right)}{4!}\left(z-1\right)^{3}+\ldots+
\end{eqnarray}
Calculando las derivadas y evaluando en $z=1$

\begin{eqnarray}
R_{1}\left(1\right)&=&S^{(1)}\left(1\right)=-\esp\left[L\right]\\
R_{1}^{(1)}\left(1\right)&=&\frac{1}{2}S^{(2)}\left(1\right)=-\frac{1}{2}\esp\left[L^{2}\right]+\frac{1}{2}\esp\left[L\right]\\
R_{1}^{(2)}\left(1\right)&=&\frac{2}{3!}S^{(3)}\left(1\right)
=-\frac{1}{3}\esp\left[L^{3}\right]+\esp\left[L^{2}\right]-\frac{2}{3}\esp\left[L\right]
\end{eqnarray}

De manera an\'aloga se puede ver que para $T\left(z\right)=z-P\left(z\right)$ se puede encontrar una expanci\'on alrededor de $z=1$

Expandiendo alrededor de $z=1$

\begin{eqnarray*}
T\left(z\right)&=&T\left(1\right)+\frac{T^{'}\left(1\right)}{1!}\left(z-1\right)+\frac{T^{''}\left(1\right)}{2!}\left(z-1\right)^{2}+\frac{T^{'''}\left(1\right)}{3!}\left(z-1\right)^{3}+\ldots+\\
&=&\left(z-1\right)\left\{T^{'}\left(1\right)+\frac{T^{''}\left(1\right)}{2!}\left(z-1\right)+\frac{T^{'''}\left(1\right)}{3!}\left(z-1\right)^{2}+\ldots+\right\}\\
&=&\left(z-1\right)R_{2}\left(z\right)
\end{eqnarray*}

donde 
\begin{eqnarray*}
T^{(1)}\left(1\right)&=&-\esp\left[X\left(t\right)\right]=-\mu,\\ T^{(2)}\left(1\right)&=&-\esp\left[X\left(X-1\right)\right]
=-\esp\left[X^{2}\right]+\esp\left[X\right]=-\esp\left[X^{2}\right]+\mu,\\
T^{(3)}\left(1\right)&=&-\esp\left[X\left(X-1\right)\left(X-2\right)\right]
=-\esp\left[X^{3}\right]+3\esp\left[X^{2}\right]-2\esp\left[X\right]\\
&=&-\esp\left[X^{3}\right]+3\esp\left[X^{2}\right]-2\mu.
\end{eqnarray*}

Por lo tanto $R_{2}\left(1\right)\neq0$, pues

\begin{eqnarray}\label{Eq.R2}
R_{2}\left(1\right)=1-\esp\left[X\right]=1-\mu
\end{eqnarray}
entonces

\begin{eqnarray}
R_{2}\left(z\right)&=&T^{'}\left(1\right)+\frac{T^{''}\left(1\right)}{2!}\left(z-1\right)+\frac{T^{'''}\left(1\right)}{3!}\left(z-1\right)^{2}+\frac{T^{(iv)}\left(1\right)}{4!}\left(z-1\right)^{3}+\ldots+
\end{eqnarray}
Calculando las derivadas y evaluando en $z=1$

\begin{eqnarray}
R_{2}\left(1\right)&=&T^{(1)}\left(1\right)=1-\mu\\
R_{2}^{(1)}\left(1\right)&=&\frac{1}{2}T^{(2)}\left(1\right)=-\frac{1}{2}\esp\left[X^{2}\right]+\frac{1}{2}\mu\\
R_{2}^{(2)}\left(1\right)&=&\frac{2}{3!}T^{(3)}\left(1\right)
=-\frac{1}{3}\esp\left[X^{3}\right]+\esp\left[X^{2}\right]-\frac{2}{3}\mu
\end{eqnarray}

Finalmente para de manera an\'aloga se puede ver que para $U\left(z\right)=1-P\left(z\right)$ se puede encontrar una expanci\'on alrededor de $z=1$

\begin{eqnarray*}
U\left(z\right)&=&U\left(1\right)+\frac{U^{'}\left(1\right)}{1!}\left(z-1\right)+\frac{U^{''}\left(1\right)}{2!}\left(z-1\right)^{2}+\frac{U^{'''}\left(1\right)}{3!}\left(z-1\right)^{3}+\ldots+\\
&=&\left(z-1\right)\left\{U^{'}\left(1\right)+\frac{U^{''}\left(1\right)}{2!}\left(z-1\right)+\frac{U^{'''}\left(1\right)}{3!}\left(z-1\right)^{2}+\ldots+\right\}\\
&=&\left(z-1\right)R_{3}\left(z\right)
\end{eqnarray*}

donde 
\begin{eqnarray*}
U^{(1)}\left(1\right)&=&-\esp\left[X\left(t\right)\right]=-\mu,\\ U^{(2)}\left(1\right)&=&-\esp\left[X\left(X-1\right)\right]
=-\esp\left[X^{2}\right]+\esp\left[X\right]=-\esp\left[X^{2}\right]+\mu,\\
U^{(3)}\left(1\right)&=&-\esp\left[X\left(X-1\right)\left(X-2\right)\right]
=-\esp\left[X^{3}\right]+3\esp\left[X^{2}\right]-2\esp\left[X\right]\\
&=&-\esp\left[X^{3}\right]+3\esp\left[X^{2}\right]-2\mu.
\end{eqnarray*}

Por lo tanto $R_{3}\left(1\right)\neq0$, pues

\begin{eqnarray}\label{Eq.R2}
R_{3}\left(1\right)=-\esp\left[X\right]=-\mu
\end{eqnarray}
entonces

\begin{eqnarray}
R_{3}\left(z\right)&=&U^{'}\left(1\right)+\frac{U^{''}\left(1\right)}{2!}\left(z-1\right)+\frac{U^{'''}\left(1\right)}{3!}\left(z-1\right)^{2}+\frac{U^{(iv)}\left(1\right)}{4!}\left(z-1\right)^{3}+\ldots+
\end{eqnarray}

Calculando las derivadas y evaluando en $z=1$

\begin{eqnarray}
R_{3}\left(1\right)&=&U^{(1)}\left(1\right)=-\mu\\
R_{3}^{(1)}\left(1\right)&=&\frac{1}{2}U^{(2)}\left(1\right)=-\frac{1}{2}\esp\left[X^{2}\right]+\frac{1}{2}\mu\\
R_{3}^{(2)}\left(1\right)&=&\frac{2}{3!}U^{(3)}\left(1\right)
=-\frac{1}{3}\esp\left[X^{3}\right]+\esp\left[X^{2}\right]-\frac{2}{3}\mu
\end{eqnarray}

Por lo tanto

\begin{eqnarray}
\esp\left[C_{i}\right]Q\left(z\right)&=&\frac{\left(z-1\right)\left(z-1\right)R_{1}\left(z\right)P\left(z\right)}{\left(z-1\right)R_{2}\left(z\right)\left(z-1\right)R_{3}\left(z\right)}
=\frac{R_{1}\left(z\right)P\left(z\right)}{R_{2}\left(z\right)R_{3}\left(z\right)}\equiv\frac{R_{1}P}{R_{2}R_{3}}
\end{eqnarray}

Entonces

\begin{eqnarray}\label{Eq.Primer.Derivada.Q}
\left[\frac{R_{1}\left(z\right)P\left(z\right)}{R_{2}\left(z\right)R_{3}\left(z\right)}\right]^{'}&=&\frac{PR_{2}R_{3}R_{1}^{'}
+R_{1}R_{2}R_{3}P^{'}-R_{3}R_{1}PR_{2}-R_{2}R_{1}PR_{3}^{'}}{\left(R_{2}R_{3}\right)^{2}}
\end{eqnarray}
Evaluando en $z=1$
\begin{eqnarray*}
&=&\frac{R_{2}(1)R_{3}(1)R_{1}^{(1)}(1)+R_{1}(1)R_{2}(1)R_{3}(1)P^{'}(1)-R_{3}(1)R_{1}(1)R_{2}(1)^{(1)}-R_{2}(1)R_{1}(1)R_{3}^{'}(1)}{\left(R_{2}(1)R_{3}(1)\right)^{2}}\\
&=&\frac{1}{\left(1-\mu\right)^{2}\mu^{2}}\left\{\left(-\frac{1}{2}\esp L^{2}+\frac{1}{2}\esp L\right)\left(1-\mu\right)\left(-\mu\right)+\left(-\esp L\right)\left(1-\mu\right)\left(-\mu\right)\mu\right.\\
&&\left.-\left(-\frac{1}{2}\esp X^{2}+\frac{1}{2}\mu\right)\left(-\mu\right)\left(-\esp L\right)-\left(1-\mu\right)\left(-\esp L\right)\left(-\frac{1}{2}\esp X^{2}+\frac{1}{2}\mu\right)\right\}\\
&=&\frac{1}{\left(1-\mu\right)^{2}\mu^{2}}\left\{\left(-\frac{1}{2}\esp L^{2}+\frac{1}{2}\esp L\right)\left(\mu^{2}-\mu\right)
+\left(\mu^{2}-\mu^{3}\right)\esp L\right.\\
&&\left.-\mu\esp L\left(-\frac{1}{2}\esp X^{2}+\frac{1}{2}\mu\right)
+\left(\esp L-\mu\esp L\right)\left(-\frac{1}{2}\esp X^{2}+\frac{1}{2}\mu\right)\right\}\\
&=&\frac{1}{\left(1-\mu\right)^{2}\mu^{2}}\left\{-\frac{1}{2}\mu^{2}\esp L^{2}
+\frac{1}{2}\mu\esp L^{2}
+\frac{1}{2}\mu^{2}\esp L
-\mu^{3}\esp L
+\mu\esp L\esp X^{2}
-\frac{1}{2}\esp L\esp X^{2}\right\}\\
&=&\frac{1}{\left(1-\mu\right)^{2}\mu^{2}}\left\{
\frac{1}{2}\mu\esp L^{2}\left(1-\mu\right)
+\esp L\left(\frac{1}{2}-\mu\right)\left(\mu^{2}-\esp X^{2}\right)\right\}\\
&=&\frac{1}{2\mu\left(1-\mu\right)}\esp L^{2}-\frac{\frac{1}{2}-\mu}{\left(1-\mu\right)^{2}\mu^{2}}\sigma^{2}\esp L
\end{eqnarray*}

por lo tanto (para Takagi)

\begin{eqnarray*}
Q^{(1)}=\frac{1}{\esp C}\left\{\frac{1}{2\mu\left(1-\mu\right)}\esp L^{2}-\frac{\frac{1}{2}-\mu}{\left(1-\mu\right)^{2}\mu^{2}}\sigma^{2}\esp L\right\}
\end{eqnarray*}
donde 

\begin{eqnarray*}
\esp C = \frac{\esp L}{\mu\left(1-\mu\right)}
\end{eqnarray*}
entonces

\begin{eqnarray*}
Q^{(1)}&=&\frac{1}{2}\frac{\esp L^{2}}{\esp L}-\frac{\frac{1}{2}-\mu}{\left(1-\mu\right)\mu}\sigma^{2}
=\frac{\esp L^{2}}{2\esp L}-\frac{\sigma^{2}}{2}\left\{\frac{2\mu-1}{\left(1-\mu\right)\mu}\right\}\\
&=&\frac{\esp L^{2}}{2\esp L}+\frac{\sigma^{2}}{2}\left\{\frac{1}{1-\mu}+\frac{1}{\mu}\right\}
\end{eqnarray*}

Mientras que para nosotros

\begin{eqnarray*}
Q^{(1)}=\frac{1}{\mu\left(1-\mu\right)}\frac{\esp L^{2}}{2\esp C}
-\sigma^{2}\frac{\esp L}{2\esp C}\cdot\frac{1-2\mu}{\left(1-\mu\right)^{2}\mu^{2}}
\end{eqnarray*}

Retomando la ecuaci\'on (\ref{Eq.Primer.Derivada.Q})

\begin{eqnarray*}
\left[\frac{R_{1}\left(z\right)P\left(z\right)}{R_{2}\left(z\right)R_{3}\left(z\right)}\right]^{'}&=&\frac{PR_{2}R_{3}R_{1}^{'}
+R_{1}R_{2}R_{3}P^{'}-R_{3}R_{1}PR_{2}-R_{2}R_{1}PR_{3}^{'}}{\left(R_{2}R_{3}\right)^{2}}
=\frac{F\left(z\right)}{G\left(z\right)}
\end{eqnarray*}

donde 

\begin{eqnarray*}
F\left(z\right)&=&PR_{2}R_{3}R_{1}^{'}
+R_{1}R_{2}R_{3}P^{'}-R_{3}R_{1}PR_{2}^{'}-R_{2}R_{1}PR_{3}^{'}\\
G\left(z\right)&=&R_{2}^{2}R_{3}^{2}\\
G^{2}\left(z\right)&=&R_{2}^{4}R_{3}^{4}=\left(1-\mu\right)^{4}\mu^{4}
\end{eqnarray*}
y por tanto

\begin{eqnarray*}
G^{'}\left(z\right)&=&2R_{2}R_{3}\left[R_{2}^{'}R_{3}+R_{2}R_{3}^{'}\right]\\
G^{'}\left(1\right)&=&-2\left(1-\mu\right)\mu\left[\left(-\frac{1}{2}\esp\left[X^{2}\right]+\frac{1}{2}\mu\right)\left(-\mu\right)+\left(1-\mu\right)\left(-\frac{1}{2}\esp\left[X^{2}\right]+\frac{1}{2}\mu\right)\right]
\end{eqnarray*}


\begin{eqnarray*}
F^{'}\left(z\right)&=&\left[\left(R_{2}R_{3}\right)R_{1}^{''}
-\left(R_{1}R_{3}\right)R_{2}^{''}
-\left(R_{1}R_{2}\right)R_{3}^{''}
-2\left(R_{2}^{'}R_{3}^{'}\right)R_{1}\right]P
+2\left(R_{2}R_{3}\right)R_{1}^{'}P^{'}
+\left(R_{1}R_{2}R_{3}\right)P^{''}
\end{eqnarray*}

Por lo tanto, encontremos $F^{'}\left(z\right)G\left(z\right)+F\left(z\right)G^{'}\left(z\right)$:

\begin{eqnarray*}
&&F^{'}\left(z\right)G\left(z\right)+F\left(z\right)G^{'}\left(z\right)=
\left\{\left[\left(R_{2}R_{3}\right)R_{1}^{''}
-\left(R_{1}R_{3}\right)R_{2}^{''}
-\left(R_{1}R_{2}\right)R_{3}^{''}
-2\left(R_{2}^{'}R_{3}^{'}\right)R_{1}\right]P\right.\\
&&\left.+2\left(R_{2}R_{3}\right)R_{1}^{'}P^{'}
+\left(R_{1}R_{2}R_{3}\right)P^{''}\right\}R_{2}^{2}R_{3}^{2}
-\left\{\left[PR_{2}R_{3}R_{1}^{'}+R_{1}R_{2}R_{3}P^{'}
-R_{3}R_{1}PR_{2}^{'}\right.\right.\\
&&\left.\left.
-R_{2}R_{1}PR_{3}^{'}\right]\left[2R_{2}R_{3}\left(R_{2}^{'}R_{3}+R_{2}R_{3}^{'}\right)\right]\right\}
\end{eqnarray*}
Evaluando en $z=1$

\begin{eqnarray*}
&=&\left(1+R_{3}\right)^{3}R_{3}^{3}R_{1}^{''}-\left(1+R_{3}\right)^{2}R_{1}R_{3}^{3}R_{3}^{''}
-\left(1+R_{3}\right)^{3}R_{3}^{2}R_{1}R_{3}^{''}-2\left(1+R_{3}\right)^{2}R_{3}^{2}
\left(R_{3}^{'}\right)^{2}\\
&+&2\left(1+R_{3}\right)^{3}R_{3}^{3}R_{1}^{'}P^{'}
+\left(1+R_{3}\right)^{3}R_{3}^{3}R_{1}P^{''}
-2\left(1+R_{3}\right)^{2}R_{3}^{2}\left(1+2R_{3}\right)R_{3}^{'}R_{1}^{'}\\
&-&2\left(1+R_{3}\right)^{2}R_{3}^{2}R_{1}R_{3}^{'}\left(1+2R_{3}\right)P^{'}
+2\left(1+R_{3}\right)\left(1+2R_{3}\right)R_{3}^{3}R_{1}\left(R_{3}^{'}\right)^{2}\\
&+&2\left(1+R_{3}\right)^{2}\left(1+2R_{3}\right)R_{1}R_{3}R_{3}^{'}\\
&=&-\left(1-\mu\right)^{3}\mu^{3}R_{1}^{''}-\left(1-\mu\right)^{2}\mu^{2}R_{1}\left(1-2\mu\right)R_{3}^{''}
-\left(1-\mu\right)^{3}\mu^{3}R_{1}P^{''}\\
&+&2\left(1-\mu\right)\mu^{2}\left[\left(1-2\mu\right)R_{1}-\left(1-\mu\right)\right]\left(R_{3}^{'}\right)^{2}
-2\left(1-\mu\right)^{2}\mu R_{1}\left(1-2\mu\right)R_{3}^{'}\\
&-&2\left(1-\mu\right)^{3}\mu^{4}R_{1}^{'}-2\mu\left(1-\mu\right)\left(1-2\mu\right)R_{3}^{'}R_{1}^{'}
-2\mu^{3}\left(1-\mu\right)^{2}\left(1-2\mu\right)R_{1}R_{1}^{'}
\end{eqnarray*}

por tanto

\begin{eqnarray*}
\left[\frac{F\left(z\right)}{G\left(z\right)}\right]^{'}&=&\frac{1}{\mu^{3}\left(1-\mu\right)^{3}}\left\{
-\left(1-\mu\right)^{2}\mu^{2}R_{1}^{''}-\mu\left(1-\mu\right)\left(1-2\mu\right)R_{1}R_{3}^{''}
-\mu^{2}\left(1-\mu\right)^{2}R_{1}P^{''}\right.\\
&&\left.+2\mu\left[\left(1-2\mu\right)R_{1}-\left(1-\mu\right)\right]\left(R_{3}^{'}\right)^{2}
-2\left(1-\mu\right)\left(1-2\mu\right)R_{1}R_{3}^{'}-2\mu^{3}\left(1-\mu\right)^{2}R_{1}^{'}\right.\\
&&\left.-2\left(1-2\mu\right)R_{3}^{'}R_{1}^{'}-2\mu^{2}\left(1-\mu\right)\left(1-2\mu\right)R_{1}R_{1}^{'}\right\}
\end{eqnarray*}

recordemos que


\begin{eqnarray*}
R_{1}&=&-\esp L\\
R_{3}&=& -\mu\\
R_{1}^{'}&=&-\frac{1}{2}\esp L^{2}+\frac{1}{2}\esp L\\
R_{3}^{'}&=&-\frac{1}{2}\esp X^{2}+\frac{1}{2}\mu\\
R_{1}^{''}&=&-\frac{1}{3}\esp L^{3}+\esp L^{2}-\frac{2}{3}\esp L\\
R_{3}^{''}&=&-\frac{1}{3}\esp X^{3}+\esp X^{2}-\frac{2}{3}\mu\\
R_{1}R_{3}^{'}&=&\frac{1}{2}\esp X^{2}\esp L-\frac{1}{2}\esp X\esp L\\
R_{1}R_{1}^{'}&=&\frac{1}{2}\esp L^{2}\esp L+\frac{1}{2}\esp^{2}L\\
R_{3}^{'}R_{1}^{'}&=&\frac{1}{4}\esp X^{2}\esp L^{2}-\frac{1}{4}\esp X^{2}\esp L-\frac{1}{4}\esp L^{2}\esp X+\frac{1}{4}\esp X\esp L\\
R_{1}R_{3}^{''}&=&\frac{1}{6}\esp X^{3}\esp L^{2}-\frac{1}{6}\esp X^{3}\esp L-\frac{1}{2}\esp L^{2}\esp X^{2}+\frac{1}{2}\esp X^{2}\esp L+\frac{1}{3}\esp X\esp L^{2}-\frac{1}{3}\esp X\esp L\\
R_{1}P^{''}&=&-\esp X^{2}\esp L\\
\left(R_{3}^{'}\right)^{2}&=&\frac{1}{4}\esp^{2}X^{2}-\frac{1}{2}\esp X^{2}\esp X+\frac{1}{4}\esp^{2} X
\end{eqnarray*}




\begin{Def}
Let $L_{i}^{*}$ be the number of users at queue $Q_{i}$ when it is polled, then
\begin{eqnarray}
\begin{array}{cc}
\esp\left[L_{i}^{*}\right]=f_{i}\left(i\right), &
Var\left[L_{i}^{*}\right]=f_{i}\left(i,i\right)+\esp\left[L_{i}^{*}\right]-\esp\left[L_{i}^{*}\right]^{2}.
\end{array}
\end{eqnarray}
\end{Def}

\begin{Def}
The cycle time $C_{i}$ for the queue $Q_{i}$ is the period beginning at the time when it is polled in a cycle and ending at the time when it is polled in the next cycle; it's duration is given by $\tau_{i}\left(m+1\right)-\tau_{i}\left(m\right)$, equivalently $\overline{\tau}_{i}\left(m+1\right)-\overline{\tau}_{i}\left(m\right)$ under steady state assumption.
\end{Def}

\begin{Def}
The intervisit time $I_{i}$ is defined as the period beginning at the time of its service completion in a cycle and ending at the time when it is polled in the next cycle; its duration is given by $\tau_{i}\left(m+1\right)-\overline{\tau}_{i}\left(m\right)$.
\end{Def}

The intervisit time duration $\tau_{i}\left(m+1\right)-\overline{\tau}\left(m\right)$ given the number of users found at queue $Q_{i}$ at time $t=\tau_{i}\left(m+1\right)$ is equal to the number of arrivals during the preceding intervisit time $\left[\overline{\tau}\left(m\right),\tau_{i}\left(m+1\right)\right]$. 

So we have



\begin{eqnarray*}
\esp\left[z_{i}^{L_{i}\left(\tau_{i}\left(m+1\right)\right)}\right]=\esp\left[\left\{P_{i}\left(z_{i}\right)\right\}^{\tau_{i}\left(m+1\right)-\overline{\tau}\left(m\right)}\right]
\end{eqnarray*}

if $I_{i}\left(z\right)=\esp\left[z^{\tau_{i}\left(m+1\right)-\overline{\tau}\left(m\right)}\right]$
we have $F_{i}\left(z\right)=I_{i}\left[P_{i}\left(z\right)\right]$
for $i=1,2$. Futhermore can be proved that

\begin{eqnarray}
\begin{array}{ll}
\esp\left[L_{i}\right]=\mu_{i}\esp\left[I_{i}\right], &
\esp\left[C_{i}\right]=\frac{f_{i}\left(i\right)}{\mu_{i}\left(1-\mu_{i}\right)},\\
\esp\left[S_{i}\right]=\mu_{i}\esp\left[C_{i}\right],&
\esp\left[I_{i}\right]=\left(1-\mu_{i}\right)\esp\left[C_{i}\right],\\
Var\left[L_{i}\right]= \mu_{i}^{2}Var\left[I_{i}\right]+\sigma^{2}\esp\left[I_{i}\right],& 
Var\left[C_{i}\right]=\frac{Var\left[L_{i}^{*}\right]}{\mu_{i}^{2}\left(1-\mu_{i}\right)^{2}},\\
Var\left[S_{i}\right]= \frac{Var\left[L_{i}^{*}\right]}{\left(1-\mu_{i}\right)^{2}}+\frac{\sigma^{2}\esp\left[L_{i}^{*}\right]}{\left(1-\mu_{i}\right)^{3}},&
Var\left[I_{i}\right]= \frac{Var\left[L_{i}^{*}\right]}{\mu_{i}^{2}}-\frac{\sigma_{i}^{2}}{\mu_{i}^{2}}f_{i}\left(i\right).
\end{array}
\end{eqnarray}

Let consider the points when the process $\left[L_{1}\left(1\right),L_{2}\left(1\right),L_{3}\left(1\right),L_{4}\left(1\right)
\right]$ becomes zero at the same time, this points, $T_{1},T_{2},\ldots$ will be denoted as regeneration points, then we have that

\begin{Def}
the interval between two such succesive regeneration points will be called regenerative cycle.
\end{Def}

\begin{Def}
Para $T_{i}$ se define, $M_{i}$, el n\'umero de ciclos de visita a la cola $Q_{l}$, durante el ciclo regenerativo, es decir, $M_{i}$ es un proceso de renovaci\'on.
\end{Def}

\begin{Def}
Para cada uno de los $M_{i}$'s, se definen a su vez la duraci\'on de cada uno de estos ciclos de visita en el ciclo regenerativo, $C_{i}^{(m)}$, para $m=1,2,\ldots,M_{i}$, que a su vez, tambi\'en es n proceso de renovaci\'on.
\end{Def}



Sea la funci\'on generadora de momentos para $L_{i}$, el n\'umero de usuarios en la cola $Q_{i}\left(z\right)$ en cualquier momento, est\'a dada por el tiempo promedio de $z^{L_{i}\left(t\right)}$ sobre el ciclo regenerativo definido anteriormente. Entonces 

\begin{equation}\label{Eq.Longitud.Tiempo.t}
Q_{i}\left(z\right)=\frac{1}{\esp\left[C_{i}\right]}\cdot\frac{1-F_{i}\left(z\right)}{P_{i}\left(z\right)-z}\cdot\frac{\left(1-z\right)P_{i}\left(z\right)}{1-P_{i}\left(z\right)}.
\end{equation}

Es decir, es posible determinar las longitudes de las colas a cualquier tiempo $t$. Entonces, determinando el primer momento es posible ver que


$M_{i}$ is an stopping time for the regenerative process with $\esp\left[M_{i}\right]<\infty$, from Wald's lemma follows that:


\begin{eqnarray*}
\esp\left[\sum_{m=1}^{M_{i}}\sum_{t=\tau_{i}\left(m\right)}^{\tau_{i}\left(m+1\right)-1}z^{L_{i}\left(t\right)}\right]&=&\esp\left[M_{i}\right]\esp\left[\sum_{t=\tau_{i}\left(m\right)}^{\tau_{i}\left(m+1\right)-1}z^{L_{i}\left(t\right)}\right]\\
\esp\left[\sum_{m=1}^{M_{i}}\tau_{i}\left(m+1\right)-\tau_{i}\left(m\right)\right]&=&\esp\left[M_{i}\right]\esp\left[\tau_{i}\left(m+1\right)-\tau_{i}\left(m\right)\right]
\end{eqnarray*}
therefore 

\begin{eqnarray*}
Q_{i}\left(z\right)&=&\frac{\esp\left[\sum_{t=\tau_{i}\left(m\right)}^{\tau_{i}\left(m+1\right)-1}z^{L_{i}\left(t\right)}\right]}{\esp\left[\tau_{i}\left(m+1\right)-\tau_{i}\left(m\right)\right]}
\end{eqnarray*}

Doing the following substitutions en (\ref{Corolario2}): $n\rightarrow t-\tau_{i}\left(m\right)$, $T \rightarrow \overline{\tau}_{i}\left(m\right)-\tau_{i}\left(m\right)$, $L_{n}\rightarrow L_{i}\left(t\right)$ and $F\left(z\right)=\esp\left[z^{L_{0}}\right]\rightarrow F_{i}\left(z\right)=\esp\left[z^{L_{i}\tau_{i}\left(m\right)}\right]$, 
we obtain

\begin{eqnarray}\label{Eq.Arribos.Primera}
\esp\left[\sum_{n=0}^{T-1}z^{L_{n}}\right]=
\esp\left[\sum_{t=\tau_{i}\left(m\right)}^{\overline{\tau}_{i}\left(m\right)-1}z^{L_{i}\left(t\right)}\right]
=z\frac{F_{i}\left(z\right)-1}{z-P_{i}\left(z\right)}
\end{eqnarray}



Por otra parte durante el tiempo de intervisita para la cola $i$, $L_{i}\left(t\right)$ solamente se incrementa de manera que el incremento por intervalo de tiempo est\'a dado por la funci\'on generadora de probabilidades de $P_{i}\left(z\right)$, por tanto la suma sobre el tiempo de intervisita puede evaluarse como:

\begin{eqnarray*}
\esp\left[\sum_{t=\tau_{i}\left(m\right)}^{\tau_{i}\left(m+1\right)-1}z^{L_{i}\left(t\right)}\right]&=&\esp\left[\sum_{t=\tau_{i}\left(m\right)}^{\tau_{i}\left(m+1\right)-1}\left\{P_{i}\left(z\right)\right\}^{t-\overline{\tau}_{i}\left(m\right)}\right]=\frac{1-\esp\left[\left\{P_{i}\left(z\right)\right\}^{\tau_{i}\left(m+1\right)-\overline{\tau}_{i}\left(m\right)}\right]}{1-P_{i}\left(z\right)}\\
&=&\frac{1-I_{i}\left[P_{i}\left(z\right)\right]}{1-P_{i}\left(z\right)}
\end{eqnarray*}
por tanto

\begin{eqnarray*}
\esp\left[\sum_{t=\tau_{i}\left(m\right)}^{\tau_{i}\left(m+1\right)-1}z^{L_{i}\left(t\right)}\right]&=&
\frac{1-F_{i}\left(z\right)}{1-P_{i}\left(z\right)}
\end{eqnarray*}

Por lo tanto

\begin{eqnarray*}
Q_{i}\left(z\right)&=&\frac{\esp\left[\sum_{t=\tau_{i}\left(m\right)}^{\tau_{i}\left(m+1\right)-1}z^{L_{i}\left(t\right)}\right]}{\esp\left[\tau_{i}\left(m+1\right)-\tau_{i}\left(m\right)\right]}
=\frac{1}{\esp\left[\tau_{i}\left(m+1\right)-\tau_{i}\left(m\right)\right]}
\esp\left[\sum_{t=\tau_{i}\left(m\right)}^{\tau_{i}\left(m+1\right)-1}z^{L_{i}\left(t\right)}\right]\\
&=&\frac{1}{\esp\left[\tau_{i}\left(m+1\right)-\tau_{i}\left(m\right)\right]}
\esp\left[\sum_{t=\tau_{i}\left(m\right)}^{\overline{\tau}_{i}\left(m\right)-1}z^{L_{i}\left(t\right)}
+\sum_{t=\overline{\tau}_{i}\left(m\right)}^{\tau_{i}\left(m+1\right)-1}z^{L_{i}\left(t\right)}\right]\\
&=&\frac{1}{\esp\left[\tau_{i}\left(m+1\right)-\tau_{i}\left(m\right)\right]}\left\{
\esp\left[\sum_{t=\tau_{i}\left(m\right)}^{\overline{\tau}_{i}\left(m\right)-1}z^{L_{i}\left(t\right)}\right]
+\esp\left[\sum_{t=\overline{\tau}_{i}\left(m\right)}^{\tau_{i}\left(m+1\right)-1}z^{L_{i}\left(t\right)}\right]\right\}\\
&=&\frac{1}{\esp\left[\tau_{i}\left(m+1\right)-\tau_{i}\left(m\right)\right]}\left\{
z\frac{F_{i}\left(z\right)-1}{z-P_{i}\left(z\right)}+\frac{1-F_{i}\left(z\right)}{1-P_{i}\left(z\right)}
\right\}\\
&=&\frac{1}{\esp\left[C_{i}\right]}\cdot\frac{1-F_{i}\left(z\right)}{P_{i}\left(z\right)-z}\cdot\frac{\left(1-z\right)P_{i}\left(z\right)}{1-P_{i}\left(z\right)}
\end{eqnarray*}

es decir

\begin{eqnarray}
\begin{array}{ll}
S^{'}\left(z\right)=-\sum_{k=1}^{+\infty}ka_{k}z^{k-1},& S^{(1)}\left(1\right)=-\sum_{k=1}^{+\infty}ka_{k}=-\esp\left[L\left(t\right)\right],\\
S^{''}\left(z\right)=-\sum_{k=2}^{+\infty}k(k-1)a_{k}z^{k-2},& S^{(2)}\left(1\right)=-\sum_{k=2}^{+\infty}k(k-1)a_{k}=\esp\left[L\left(L-1\right)\right],\\
S^{'''}\left(z\right)=-\sum_{k=3}^{+\infty}k(k-1)(k-2)a_{k}z^{k-3},&
S^{(3)}\left(1\right)=-\sum_{k=3}^{+\infty}k(k-1)(k-2)a_{k}\\
&=-\esp\left[L\left(L-1\right)\left(L-2\right)\right]\\
&=-\esp\left[L^{3}\right]+3-\esp\left[L^{2}\right]-2-\esp\left[L\right];
\end{array}
\end{eqnarray}







%________________________________________________________________________
\subsection{Procesos Regenerativos Sigman, Thorisson y Wolff \cite{Sigman1}}
%________________________________________________________________________


\begin{Def}[Definici\'on Cl\'asica]
Un proceso estoc\'astico $X=\left\{X\left(t\right):t\geq0\right\}$ es llamado regenerativo is existe una variable aleatoria $R_{1}>0$ tal que
\begin{itemize}
\item[i)] $\left\{X\left(t+R_{1}\right):t\geq0\right\}$ es independiente de $\left\{\left\{X\left(t\right):t<R_{1}\right\},\right\}$
\item[ii)] $\left\{X\left(t+R_{1}\right):t\geq0\right\}$ es estoc\'asticamente equivalente a $\left\{X\left(t\right):t>0\right\}$
\end{itemize}

Llamamos a $R_{1}$ tiempo de regeneraci\'on, y decimos que $X$ se regenera en este punto.
\end{Def}

$\left\{X\left(t+R_{1}\right)\right\}$ es regenerativo con tiempo de regeneraci\'on $R_{2}$, independiente de $R_{1}$ pero con la misma distribuci\'on que $R_{1}$. Procediendo de esta manera se obtiene una secuencia de variables aleatorias independientes e id\'enticamente distribuidas $\left\{R_{n}\right\}$ llamados longitudes de ciclo. Si definimos a $Z_{k}\equiv R_{1}+R_{2}+\cdots+R_{k}$, se tiene un proceso de renovaci\'on llamado proceso de renovaci\'on encajado para $X$.


\begin{Note}
La existencia de un primer tiempo de regeneraci\'on, $R_{1}$, implica la existencia de una sucesi\'on completa de estos tiempos $R_{1},R_{2}\ldots,$ que satisfacen la propiedad deseada \cite{Sigman2}.
\end{Note}


\begin{Note} Para la cola $GI/GI/1$ los usuarios arriban con tiempos $t_{n}$ y son atendidos con tiempos de servicio $S_{n}$, los tiempos de arribo forman un proceso de renovaci\'on  con tiempos entre arribos independientes e identicamente distribuidos (\texttt{i.i.d.})$T_{n}=t_{n}-t_{n-1}$, adem\'as los tiempos de servicio son \texttt{i.i.d.} e independientes de los procesos de arribo. Por \textit{estable} se entiende que $\esp S_{n}<\esp T_{n}<\infty$.
\end{Note}
 


\begin{Def}
Para $x$ fijo y para cada $t\geq0$, sea $I_{x}\left(t\right)=1$ si $X\left(t\right)\leq x$,  $I_{x}\left(t\right)=0$ en caso contrario, y def\'inanse los tiempos promedio
\begin{eqnarray*}
\overline{X}&=&lim_{t\rightarrow\infty}\frac{1}{t}\int_{0}^{\infty}X\left(u\right)du\\
\prob\left(X_{\infty}\leq x\right)&=&lim_{t\rightarrow\infty}\frac{1}{t}\int_{0}^{\infty}I_{x}\left(u\right)du,
\end{eqnarray*}
cuando estos l\'imites existan.
\end{Def}

Como consecuencia del teorema de Renovaci\'on-Recompensa, se tiene que el primer l\'imite  existe y es igual a la constante
\begin{eqnarray*}
\overline{X}&=&\frac{\esp\left[\int_{0}^{R_{1}}X\left(t\right)dt\right]}{\esp\left[R_{1}\right]},
\end{eqnarray*}
suponiendo que ambas esperanzas son finitas.
 
\begin{Note}
Funciones de procesos regenerativos son regenerativas, es decir, si $X\left(t\right)$ es regenerativo y se define el proceso $Y\left(t\right)$ por $Y\left(t\right)=f\left(X\left(t\right)\right)$ para alguna funci\'on Borel medible $f\left(\cdot\right)$. Adem\'as $Y$ es regenerativo con los mismos tiempos de renovaci\'on que $X$. 

En general, los tiempos de renovaci\'on, $Z_{k}$ de un proceso regenerativo no requieren ser tiempos de paro con respecto a la evoluci\'on de $X\left(t\right)$.
\end{Note} 

\begin{Note}
Una funci\'on de un proceso de Markov, usualmente no ser\'a un proceso de Markov, sin embargo ser\'a regenerativo si el proceso de Markov lo es.
\end{Note}

 
\begin{Note}
Un proceso regenerativo con media de la longitud de ciclo finita es llamado positivo recurrente.
\end{Note}


\begin{Note}
\begin{itemize}
\item[a)] Si el proceso regenerativo $X$ es positivo recurrente y tiene trayectorias muestrales no negativas, entonces la ecuaci\'on anterior es v\'alida.
\item[b)] Si $X$ es positivo recurrente regenerativo, podemos construir una \'unica versi\'on estacionaria de este proceso, $X_{e}=\left\{X_{e}\left(t\right)\right\}$, donde $X_{e}$ es un proceso estoc\'astico regenerativo y estrictamente estacionario, con distribuci\'on marginal distribuida como $X_{\infty}$
\end{itemize}
\end{Note}


%__________________________________________________________________________________________
%\subsection{Procesos Regenerativos Estacionarios - Stidham \cite{Stidham}}
%__________________________________________________________________________________________


Un proceso estoc\'astico a tiempo continuo $\left\{V\left(t\right),t\geq0\right\}$ es un proceso regenerativo si existe una sucesi\'on de variables aleatorias independientes e id\'enticamente distribuidas $\left\{X_{1},X_{2},\ldots\right\}$, sucesi\'on de renovaci\'on, tal que para cualquier conjunto de Borel $A$, 

\begin{eqnarray*}
\prob\left\{V\left(t\right)\in A|X_{1}+X_{2}+\cdots+X_{R\left(t\right)}=s,\left\{V\left(\tau\right),\tau<s\right\}\right\}=\prob\left\{V\left(t-s\right)\in A|X_{1}>t-s\right\},
\end{eqnarray*}
para todo $0\leq s\leq t$, donde $R\left(t\right)=\max\left\{X_{1}+X_{2}+\cdots+X_{j}\leq t\right\}=$n\'umero de renovaciones ({\emph{puntos de regeneraci\'on}}) que ocurren en $\left[0,t\right]$. El intervalo $\left[0,X_{1}\right)$ es llamado {\emph{primer ciclo de regeneraci\'on}} de $\left\{V\left(t \right),t\geq0\right\}$, $\left[X_{1},X_{1}+X_{2}\right)$ el {\emph{segundo ciclo de regeneraci\'on}}, y as\'i sucesivamente.

Sea $X=X_{1}$ y sea $F$ la funci\'on de distrbuci\'on de $X$


\begin{Def}
Se define el proceso estacionario, $\left\{V^{*}\left(t\right),t\geq0\right\}$, para $\left\{V\left(t\right),t\geq0\right\}$ por

\begin{eqnarray*}
\prob\left\{V\left(t\right)\in A\right\}=\frac{1}{\esp\left[X\right]}\int_{0}^{\infty}\prob\left\{V\left(t+x\right)\in A|X>x\right\}\left(1-F\left(x\right)\right)dx,
\end{eqnarray*} 
para todo $t\geq0$ y todo conjunto de Borel $A$.
\end{Def}

\begin{Def}
Una distribuci\'on se dice que es {\emph{aritm\'etica}} si todos sus puntos de incremento son m\'ultiplos de la forma $0,\lambda, 2\lambda,\ldots$ para alguna $\lambda>0$ entera.
\end{Def}


\begin{Def}
Una modificaci\'on medible de un proceso $\left\{V\left(t\right),t\geq0\right\}$, es una versi\'on de este, $\left\{V\left(t,w\right)\right\}$ conjuntamente medible para $t\geq0$ y para $w\in S$, $S$ espacio de estados para $\left\{V\left(t\right),t\geq0\right\}$.
\end{Def}

\begin{Teo}
Sea $\left\{V\left(t\right),t\geq\right\}$ un proceso regenerativo no negativo con modificaci\'on medible. Sea $\esp\left[X\right]<\infty$. Entonces el proceso estacionario dado por la ecuaci\'on anterior est\'a bien definido y tiene funci\'on de distribuci\'on independiente de $t$, adem\'as
\begin{itemize}
\item[i)] \begin{eqnarray*}
\esp\left[V^{*}\left(0\right)\right]&=&\frac{\esp\left[\int_{0}^{X}V\left(s\right)ds\right]}{\esp\left[X\right]}\end{eqnarray*}
\item[ii)] Si $\esp\left[V^{*}\left(0\right)\right]<\infty$, equivalentemente, si $\esp\left[\int_{0}^{X}V\left(s\right)ds\right]<\infty$,entonces
\begin{eqnarray*}
\frac{\int_{0}^{t}V\left(s\right)ds}{t}\rightarrow\frac{\esp\left[\int_{0}^{X}V\left(s\right)ds\right]}{\esp\left[X\right]}
\end{eqnarray*}
con probabilidad 1 y en media, cuando $t\rightarrow\infty$.
\end{itemize}
\end{Teo}

\begin{Coro}
Sea $\left\{V\left(t\right),t\geq0\right\}$ un proceso regenerativo no negativo, con modificaci\'on medible. Si $\esp <\infty$, $F$ es no-aritm\'etica, y para todo $x\geq0$, $P\left\{V\left(t\right)\leq x,C>x\right\}$ es de variaci\'on acotada como funci\'on de $t$ en cada intervalo finito $\left[0,\tau\right]$, entonces $V\left(t\right)$ converge en distribuci\'on  cuando $t\rightarrow\infty$ y $$\esp V=\frac{\esp \int_{0}^{X}V\left(s\right)ds}{\esp X}$$
Donde $V$ tiene la distribuci\'on l\'imite de $V\left(t\right)$ cuando $t\rightarrow\infty$.

\end{Coro}

Para el caso discreto se tienen resultados similares.



%______________________________________________________________________
%\subsection{Procesos de Renovaci\'on}
%______________________________________________________________________

\begin{Def}%\label{Def.Tn}
Sean $0\leq T_{1}\leq T_{2}\leq \ldots$ son tiempos aleatorios infinitos en los cuales ocurren ciertos eventos. El n\'umero de tiempos $T_{n}$ en el intervalo $\left[0,t\right)$ es

\begin{eqnarray}
N\left(t\right)=\sum_{n=1}^{\infty}\indora\left(T_{n}\leq t\right),
\end{eqnarray}
para $t\geq0$.
\end{Def}

Si se consideran los puntos $T_{n}$ como elementos de $\rea_{+}$, y $N\left(t\right)$ es el n\'umero de puntos en $\rea$. El proceso denotado por $\left\{N\left(t\right):t\geq0\right\}$, denotado por $N\left(t\right)$, es un proceso puntual en $\rea_{+}$. Los $T_{n}$ son los tiempos de ocurrencia, el proceso puntual $N\left(t\right)$ es simple si su n\'umero de ocurrencias son distintas: $0<T_{1}<T_{2}<\ldots$ casi seguramente.

\begin{Def}
Un proceso puntual $N\left(t\right)$ es un proceso de renovaci\'on si los tiempos de interocurrencia $\xi_{n}=T_{n}-T_{n-1}$, para $n\geq1$, son independientes e identicamente distribuidos con distribuci\'on $F$, donde $F\left(0\right)=0$ y $T_{0}=0$. Los $T_{n}$ son llamados tiempos de renovaci\'on, referente a la independencia o renovaci\'on de la informaci\'on estoc\'astica en estos tiempos. Los $\xi_{n}$ son los tiempos de inter-renovaci\'on, y $N\left(t\right)$ es el n\'umero de renovaciones en el intervalo $\left[0,t\right)$
\end{Def}


\begin{Note}
Para definir un proceso de renovaci\'on para cualquier contexto, solamente hay que especificar una distribuci\'on $F$, con $F\left(0\right)=0$, para los tiempos de inter-renovaci\'on. La funci\'on $F$ en turno degune las otra variables aleatorias. De manera formal, existe un espacio de probabilidad y una sucesi\'on de variables aleatorias $\xi_{1},\xi_{2},\ldots$ definidas en este con distribuci\'on $F$. Entonces las otras cantidades son $T_{n}=\sum_{k=1}^{n}\xi_{k}$ y $N\left(t\right)=\sum_{n=1}^{\infty}\indora\left(T_{n}\leq t\right)$, donde $T_{n}\rightarrow\infty$ casi seguramente por la Ley Fuerte de los Grandes Números.
\end{Note}

%___________________________________________________________________________________________
%
%\subsection{Teorema Principal de Renovaci\'on}
%___________________________________________________________________________________________
%

\begin{Note} Una funci\'on $h:\rea_{+}\rightarrow\rea$ es Directamente Riemann Integrable en los siguientes casos:
\begin{itemize}
\item[a)] $h\left(t\right)\geq0$ es decreciente y Riemann Integrable.
\item[b)] $h$ es continua excepto posiblemente en un conjunto de Lebesgue de medida 0, y $|h\left(t\right)|\leq b\left(t\right)$, donde $b$ es DRI.
\end{itemize}
\end{Note}

\begin{Teo}[Teorema Principal de Renovaci\'on]
Si $F$ es no aritm\'etica y $h\left(t\right)$ es Directamente Riemann Integrable (DRI), entonces

\begin{eqnarray*}
lim_{t\rightarrow\infty}U\star h=\frac{1}{\mu}\int_{\rea_{+}}h\left(s\right)ds.
\end{eqnarray*}
\end{Teo}

\begin{Prop}
Cualquier funci\'on $H\left(t\right)$ acotada en intervalos finitos y que es 0 para $t<0$ puede expresarse como
\begin{eqnarray*}
H\left(t\right)=U\star h\left(t\right)\textrm{,  donde }h\left(t\right)=H\left(t\right)-F\star H\left(t\right)
\end{eqnarray*}
\end{Prop}

\begin{Def}
Un proceso estoc\'astico $X\left(t\right)$ es crudamente regenerativo en un tiempo aleatorio positivo $T$ si
\begin{eqnarray*}
\esp\left[X\left(T+t\right)|T\right]=\esp\left[X\left(t\right)\right]\textrm{, para }t\geq0,\end{eqnarray*}
y con las esperanzas anteriores finitas.
\end{Def}

\begin{Prop}
Sup\'ongase que $X\left(t\right)$ es un proceso crudamente regenerativo en $T$, que tiene distribuci\'on $F$. Si $\esp\left[X\left(t\right)\right]$ es acotado en intervalos finitos, entonces
\begin{eqnarray*}
\esp\left[X\left(t\right)\right]=U\star h\left(t\right)\textrm{,  donde }h\left(t\right)=\esp\left[X\left(t\right)\indora\left(T>t\right)\right].
\end{eqnarray*}
\end{Prop}

\begin{Teo}[Regeneraci\'on Cruda]
Sup\'ongase que $X\left(t\right)$ es un proceso con valores positivo crudamente regenerativo en $T$, y def\'inase $M=\sup\left\{|X\left(t\right)|:t\leq T\right\}$. Si $T$ es no aritm\'etico y $M$ y $MT$ tienen media finita, entonces
\begin{eqnarray*}
lim_{t\rightarrow\infty}\esp\left[X\left(t\right)\right]=\frac{1}{\mu}\int_{\rea_{+}}h\left(s\right)ds,
\end{eqnarray*}
donde $h\left(t\right)=\esp\left[X\left(t\right)\indora\left(T>t\right)\right]$.
\end{Teo}

%___________________________________________________________________________________________
%
%\subsection{Propiedades de los Procesos de Renovaci\'on}
%___________________________________________________________________________________________
%

Los tiempos $T_{n}$ est\'an relacionados con los conteos de $N\left(t\right)$ por

\begin{eqnarray*}
\left\{N\left(t\right)\geq n\right\}&=&\left\{T_{n}\leq t\right\}\\
T_{N\left(t\right)}\leq &t&<T_{N\left(t\right)+1},
\end{eqnarray*}

adem\'as $N\left(T_{n}\right)=n$, y 

\begin{eqnarray*}
N\left(t\right)=\max\left\{n:T_{n}\leq t\right\}=\min\left\{n:T_{n+1}>t\right\}
\end{eqnarray*}

Por propiedades de la convoluci\'on se sabe que

\begin{eqnarray*}
P\left\{T_{n}\leq t\right\}=F^{n\star}\left(t\right)
\end{eqnarray*}
que es la $n$-\'esima convoluci\'on de $F$. Entonces 

\begin{eqnarray*}
\left\{N\left(t\right)\geq n\right\}&=&\left\{T_{n}\leq t\right\}\\
P\left\{N\left(t\right)\leq n\right\}&=&1-F^{\left(n+1\right)\star}\left(t\right)
\end{eqnarray*}

Adem\'as usando el hecho de que $\esp\left[N\left(t\right)\right]=\sum_{n=1}^{\infty}P\left\{N\left(t\right)\geq n\right\}$
se tiene que

\begin{eqnarray*}
\esp\left[N\left(t\right)\right]=\sum_{n=1}^{\infty}F^{n\star}\left(t\right)
\end{eqnarray*}

\begin{Prop}
Para cada $t\geq0$, la funci\'on generadora de momentos $\esp\left[e^{\alpha N\left(t\right)}\right]$ existe para alguna $\alpha$ en una vecindad del 0, y de aqu\'i que $\esp\left[N\left(t\right)^{m}\right]<\infty$, para $m\geq1$.
\end{Prop}


\begin{Note}
Si el primer tiempo de renovaci\'on $\xi_{1}$ no tiene la misma distribuci\'on que el resto de las $\xi_{n}$, para $n\geq2$, a $N\left(t\right)$ se le llama Proceso de Renovaci\'on retardado, donde si $\xi$ tiene distribuci\'on $G$, entonces el tiempo $T_{n}$ de la $n$-\'esima renovaci\'on tiene distribuci\'on $G\star F^{\left(n-1\right)\star}\left(t\right)$
\end{Note}


\begin{Teo}
Para una constante $\mu\leq\infty$ ( o variable aleatoria), las siguientes expresiones son equivalentes:

\begin{eqnarray}
lim_{n\rightarrow\infty}n^{-1}T_{n}&=&\mu,\textrm{ c.s.}\\
lim_{t\rightarrow\infty}t^{-1}N\left(t\right)&=&1/\mu,\textrm{ c.s.}
\end{eqnarray}
\end{Teo}


Es decir, $T_{n}$ satisface la Ley Fuerte de los Grandes N\'umeros s\'i y s\'olo s\'i $N\left/t\right)$ la cumple.


\begin{Coro}[Ley Fuerte de los Grandes N\'umeros para Procesos de Renovaci\'on]
Si $N\left(t\right)$ es un proceso de renovaci\'on cuyos tiempos de inter-renovaci\'on tienen media $\mu\leq\infty$, entonces
\begin{eqnarray}
t^{-1}N\left(t\right)\rightarrow 1/\mu,\textrm{ c.s. cuando }t\rightarrow\infty.
\end{eqnarray}

\end{Coro}


Considerar el proceso estoc\'astico de valores reales $\left\{Z\left(t\right):t\geq0\right\}$ en el mismo espacio de probabilidad que $N\left(t\right)$

\begin{Def}
Para el proceso $\left\{Z\left(t\right):t\geq0\right\}$ se define la fluctuaci\'on m\'axima de $Z\left(t\right)$ en el intervalo $\left(T_{n-1},T_{n}\right]$:
\begin{eqnarray*}
M_{n}=\sup_{T_{n-1}<t\leq T_{n}}|Z\left(t\right)-Z\left(T_{n-1}\right)|
\end{eqnarray*}
\end{Def}

\begin{Teo}
Sup\'ongase que $n^{-1}T_{n}\rightarrow\mu$ c.s. cuando $n\rightarrow\infty$, donde $\mu\leq\infty$ es una constante o variable aleatoria. Sea $a$ una constante o variable aleatoria que puede ser infinita cuando $\mu$ es finita, y considere las expresiones l\'imite:
\begin{eqnarray}
lim_{n\rightarrow\infty}n^{-1}Z\left(T_{n}\right)&=&a,\textrm{ c.s.}\\
lim_{t\rightarrow\infty}t^{-1}Z\left(t\right)&=&a/\mu,\textrm{ c.s.}
\end{eqnarray}
La segunda expresi\'on implica la primera. Conversamente, la primera implica la segunda si el proceso $Z\left(t\right)$ es creciente, o si $lim_{n\rightarrow\infty}n^{-1}M_{n}=0$ c.s.
\end{Teo}

\begin{Coro}
Si $N\left(t\right)$ es un proceso de renovaci\'on, y $\left(Z\left(T_{n}\right)-Z\left(T_{n-1}\right),M_{n}\right)$, para $n\geq1$, son variables aleatorias independientes e id\'enticamente distribuidas con media finita, entonces,
\begin{eqnarray}
lim_{t\rightarrow\infty}t^{-1}Z\left(t\right)\rightarrow\frac{\esp\left[Z\left(T_{1}\right)-Z\left(T_{0}\right)\right]}{\esp\left[T_{1}\right]},\textrm{ c.s. cuando  }t\rightarrow\infty.
\end{eqnarray}
\end{Coro}



%___________________________________________________________________________________________
%
%\subsection{Propiedades de los Procesos de Renovaci\'on}
%___________________________________________________________________________________________
%

Los tiempos $T_{n}$ est\'an relacionados con los conteos de $N\left(t\right)$ por

\begin{eqnarray*}
\left\{N\left(t\right)\geq n\right\}&=&\left\{T_{n}\leq t\right\}\\
T_{N\left(t\right)}\leq &t&<T_{N\left(t\right)+1},
\end{eqnarray*}

adem\'as $N\left(T_{n}\right)=n$, y 

\begin{eqnarray*}
N\left(t\right)=\max\left\{n:T_{n}\leq t\right\}=\min\left\{n:T_{n+1}>t\right\}
\end{eqnarray*}

Por propiedades de la convoluci\'on se sabe que

\begin{eqnarray*}
P\left\{T_{n}\leq t\right\}=F^{n\star}\left(t\right)
\end{eqnarray*}
que es la $n$-\'esima convoluci\'on de $F$. Entonces 

\begin{eqnarray*}
\left\{N\left(t\right)\geq n\right\}&=&\left\{T_{n}\leq t\right\}\\
P\left\{N\left(t\right)\leq n\right\}&=&1-F^{\left(n+1\right)\star}\left(t\right)
\end{eqnarray*}

Adem\'as usando el hecho de que $\esp\left[N\left(t\right)\right]=\sum_{n=1}^{\infty}P\left\{N\left(t\right)\geq n\right\}$
se tiene que

\begin{eqnarray*}
\esp\left[N\left(t\right)\right]=\sum_{n=1}^{\infty}F^{n\star}\left(t\right)
\end{eqnarray*}

\begin{Prop}
Para cada $t\geq0$, la funci\'on generadora de momentos $\esp\left[e^{\alpha N\left(t\right)}\right]$ existe para alguna $\alpha$ en una vecindad del 0, y de aqu\'i que $\esp\left[N\left(t\right)^{m}\right]<\infty$, para $m\geq1$.
\end{Prop}


\begin{Note}
Si el primer tiempo de renovaci\'on $\xi_{1}$ no tiene la misma distribuci\'on que el resto de las $\xi_{n}$, para $n\geq2$, a $N\left(t\right)$ se le llama Proceso de Renovaci\'on retardado, donde si $\xi$ tiene distribuci\'on $G$, entonces el tiempo $T_{n}$ de la $n$-\'esima renovaci\'on tiene distribuci\'on $G\star F^{\left(n-1\right)\star}\left(t\right)$
\end{Note}


\begin{Teo}
Para una constante $\mu\leq\infty$ ( o variable aleatoria), las siguientes expresiones son equivalentes:

\begin{eqnarray}
lim_{n\rightarrow\infty}n^{-1}T_{n}&=&\mu,\textrm{ c.s.}\\
lim_{t\rightarrow\infty}t^{-1}N\left(t\right)&=&1/\mu,\textrm{ c.s.}
\end{eqnarray}
\end{Teo}


Es decir, $T_{n}$ satisface la Ley Fuerte de los Grandes N\'umeros s\'i y s\'olo s\'i $N\left/t\right)$ la cumple.


\begin{Coro}[Ley Fuerte de los Grandes N\'umeros para Procesos de Renovaci\'on]
Si $N\left(t\right)$ es un proceso de renovaci\'on cuyos tiempos de inter-renovaci\'on tienen media $\mu\leq\infty$, entonces
\begin{eqnarray}
t^{-1}N\left(t\right)\rightarrow 1/\mu,\textrm{ c.s. cuando }t\rightarrow\infty.
\end{eqnarray}

\end{Coro}


Considerar el proceso estoc\'astico de valores reales $\left\{Z\left(t\right):t\geq0\right\}$ en el mismo espacio de probabilidad que $N\left(t\right)$

\begin{Def}
Para el proceso $\left\{Z\left(t\right):t\geq0\right\}$ se define la fluctuaci\'on m\'axima de $Z\left(t\right)$ en el intervalo $\left(T_{n-1},T_{n}\right]$:
\begin{eqnarray*}
M_{n}=\sup_{T_{n-1}<t\leq T_{n}}|Z\left(t\right)-Z\left(T_{n-1}\right)|
\end{eqnarray*}
\end{Def}

\begin{Teo}
Sup\'ongase que $n^{-1}T_{n}\rightarrow\mu$ c.s. cuando $n\rightarrow\infty$, donde $\mu\leq\infty$ es una constante o variable aleatoria. Sea $a$ una constante o variable aleatoria que puede ser infinita cuando $\mu$ es finita, y considere las expresiones l\'imite:
\begin{eqnarray}
lim_{n\rightarrow\infty}n^{-1}Z\left(T_{n}\right)&=&a,\textrm{ c.s.}\\
lim_{t\rightarrow\infty}t^{-1}Z\left(t\right)&=&a/\mu,\textrm{ c.s.}
\end{eqnarray}
La segunda expresi\'on implica la primera. Conversamente, la primera implica la segunda si el proceso $Z\left(t\right)$ es creciente, o si $lim_{n\rightarrow\infty}n^{-1}M_{n}=0$ c.s.
\end{Teo}

\begin{Coro}
Si $N\left(t\right)$ es un proceso de renovaci\'on, y $\left(Z\left(T_{n}\right)-Z\left(T_{n-1}\right),M_{n}\right)$, para $n\geq1$, son variables aleatorias independientes e id\'enticamente distribuidas con media finita, entonces,
\begin{eqnarray}
lim_{t\rightarrow\infty}t^{-1}Z\left(t\right)\rightarrow\frac{\esp\left[Z\left(T_{1}\right)-Z\left(T_{0}\right)\right]}{\esp\left[T_{1}\right]},\textrm{ c.s. cuando  }t\rightarrow\infty.
\end{eqnarray}
\end{Coro}


%___________________________________________________________________________________________
%
%\subsection{Propiedades de los Procesos de Renovaci\'on}
%___________________________________________________________________________________________
%

Los tiempos $T_{n}$ est\'an relacionados con los conteos de $N\left(t\right)$ por

\begin{eqnarray*}
\left\{N\left(t\right)\geq n\right\}&=&\left\{T_{n}\leq t\right\}\\
T_{N\left(t\right)}\leq &t&<T_{N\left(t\right)+1},
\end{eqnarray*}

adem\'as $N\left(T_{n}\right)=n$, y 

\begin{eqnarray*}
N\left(t\right)=\max\left\{n:T_{n}\leq t\right\}=\min\left\{n:T_{n+1}>t\right\}
\end{eqnarray*}

Por propiedades de la convoluci\'on se sabe que

\begin{eqnarray*}
P\left\{T_{n}\leq t\right\}=F^{n\star}\left(t\right)
\end{eqnarray*}
que es la $n$-\'esima convoluci\'on de $F$. Entonces 

\begin{eqnarray*}
\left\{N\left(t\right)\geq n\right\}&=&\left\{T_{n}\leq t\right\}\\
P\left\{N\left(t\right)\leq n\right\}&=&1-F^{\left(n+1\right)\star}\left(t\right)
\end{eqnarray*}

Adem\'as usando el hecho de que $\esp\left[N\left(t\right)\right]=\sum_{n=1}^{\infty}P\left\{N\left(t\right)\geq n\right\}$
se tiene que

\begin{eqnarray*}
\esp\left[N\left(t\right)\right]=\sum_{n=1}^{\infty}F^{n\star}\left(t\right)
\end{eqnarray*}

\begin{Prop}
Para cada $t\geq0$, la funci\'on generadora de momentos $\esp\left[e^{\alpha N\left(t\right)}\right]$ existe para alguna $\alpha$ en una vecindad del 0, y de aqu\'i que $\esp\left[N\left(t\right)^{m}\right]<\infty$, para $m\geq1$.
\end{Prop}


\begin{Note}
Si el primer tiempo de renovaci\'on $\xi_{1}$ no tiene la misma distribuci\'on que el resto de las $\xi_{n}$, para $n\geq2$, a $N\left(t\right)$ se le llama Proceso de Renovaci\'on retardado, donde si $\xi$ tiene distribuci\'on $G$, entonces el tiempo $T_{n}$ de la $n$-\'esima renovaci\'on tiene distribuci\'on $G\star F^{\left(n-1\right)\star}\left(t\right)$
\end{Note}


\begin{Teo}
Para una constante $\mu\leq\infty$ ( o variable aleatoria), las siguientes expresiones son equivalentes:

\begin{eqnarray}
lim_{n\rightarrow\infty}n^{-1}T_{n}&=&\mu,\textrm{ c.s.}\\
lim_{t\rightarrow\infty}t^{-1}N\left(t\right)&=&1/\mu,\textrm{ c.s.}
\end{eqnarray}
\end{Teo}


Es decir, $T_{n}$ satisface la Ley Fuerte de los Grandes N\'umeros s\'i y s\'olo s\'i $N\left/t\right)$ la cumple.


\begin{Coro}[Ley Fuerte de los Grandes N\'umeros para Procesos de Renovaci\'on]
Si $N\left(t\right)$ es un proceso de renovaci\'on cuyos tiempos de inter-renovaci\'on tienen media $\mu\leq\infty$, entonces
\begin{eqnarray}
t^{-1}N\left(t\right)\rightarrow 1/\mu,\textrm{ c.s. cuando }t\rightarrow\infty.
\end{eqnarray}

\end{Coro}


Considerar el proceso estoc\'astico de valores reales $\left\{Z\left(t\right):t\geq0\right\}$ en el mismo espacio de probabilidad que $N\left(t\right)$

\begin{Def}
Para el proceso $\left\{Z\left(t\right):t\geq0\right\}$ se define la fluctuaci\'on m\'axima de $Z\left(t\right)$ en el intervalo $\left(T_{n-1},T_{n}\right]$:
\begin{eqnarray*}
M_{n}=\sup_{T_{n-1}<t\leq T_{n}}|Z\left(t\right)-Z\left(T_{n-1}\right)|
\end{eqnarray*}
\end{Def}

\begin{Teo}
Sup\'ongase que $n^{-1}T_{n}\rightarrow\mu$ c.s. cuando $n\rightarrow\infty$, donde $\mu\leq\infty$ es una constante o variable aleatoria. Sea $a$ una constante o variable aleatoria que puede ser infinita cuando $\mu$ es finita, y considere las expresiones l\'imite:
\begin{eqnarray}
lim_{n\rightarrow\infty}n^{-1}Z\left(T_{n}\right)&=&a,\textrm{ c.s.}\\
lim_{t\rightarrow\infty}t^{-1}Z\left(t\right)&=&a/\mu,\textrm{ c.s.}
\end{eqnarray}
La segunda expresi\'on implica la primera. Conversamente, la primera implica la segunda si el proceso $Z\left(t\right)$ es creciente, o si $lim_{n\rightarrow\infty}n^{-1}M_{n}=0$ c.s.
\end{Teo}

\begin{Coro}
Si $N\left(t\right)$ es un proceso de renovaci\'on, y $\left(Z\left(T_{n}\right)-Z\left(T_{n-1}\right),M_{n}\right)$, para $n\geq1$, son variables aleatorias independientes e id\'enticamente distribuidas con media finita, entonces,
\begin{eqnarray}
lim_{t\rightarrow\infty}t^{-1}Z\left(t\right)\rightarrow\frac{\esp\left[Z\left(T_{1}\right)-Z\left(T_{0}\right)\right]}{\esp\left[T_{1}\right]},\textrm{ c.s. cuando  }t\rightarrow\infty.
\end{eqnarray}
\end{Coro}

%___________________________________________________________________________________________
%
%\subsection{Propiedades de los Procesos de Renovaci\'on}
%___________________________________________________________________________________________
%

Los tiempos $T_{n}$ est\'an relacionados con los conteos de $N\left(t\right)$ por

\begin{eqnarray*}
\left\{N\left(t\right)\geq n\right\}&=&\left\{T_{n}\leq t\right\}\\
T_{N\left(t\right)}\leq &t&<T_{N\left(t\right)+1},
\end{eqnarray*}

adem\'as $N\left(T_{n}\right)=n$, y 

\begin{eqnarray*}
N\left(t\right)=\max\left\{n:T_{n}\leq t\right\}=\min\left\{n:T_{n+1}>t\right\}
\end{eqnarray*}

Por propiedades de la convoluci\'on se sabe que

\begin{eqnarray*}
P\left\{T_{n}\leq t\right\}=F^{n\star}\left(t\right)
\end{eqnarray*}
que es la $n$-\'esima convoluci\'on de $F$. Entonces 

\begin{eqnarray*}
\left\{N\left(t\right)\geq n\right\}&=&\left\{T_{n}\leq t\right\}\\
P\left\{N\left(t\right)\leq n\right\}&=&1-F^{\left(n+1\right)\star}\left(t\right)
\end{eqnarray*}

Adem\'as usando el hecho de que $\esp\left[N\left(t\right)\right]=\sum_{n=1}^{\infty}P\left\{N\left(t\right)\geq n\right\}$
se tiene que

\begin{eqnarray*}
\esp\left[N\left(t\right)\right]=\sum_{n=1}^{\infty}F^{n\star}\left(t\right)
\end{eqnarray*}

\begin{Prop}
Para cada $t\geq0$, la funci\'on generadora de momentos $\esp\left[e^{\alpha N\left(t\right)}\right]$ existe para alguna $\alpha$ en una vecindad del 0, y de aqu\'i que $\esp\left[N\left(t\right)^{m}\right]<\infty$, para $m\geq1$.
\end{Prop}


\begin{Note}
Si el primer tiempo de renovaci\'on $\xi_{1}$ no tiene la misma distribuci\'on que el resto de las $\xi_{n}$, para $n\geq2$, a $N\left(t\right)$ se le llama Proceso de Renovaci\'on retardado, donde si $\xi$ tiene distribuci\'on $G$, entonces el tiempo $T_{n}$ de la $n$-\'esima renovaci\'on tiene distribuci\'on $G\star F^{\left(n-1\right)\star}\left(t\right)$
\end{Note}


\begin{Teo}
Para una constante $\mu\leq\infty$ ( o variable aleatoria), las siguientes expresiones son equivalentes:

\begin{eqnarray}
lim_{n\rightarrow\infty}n^{-1}T_{n}&=&\mu,\textrm{ c.s.}\\
lim_{t\rightarrow\infty}t^{-1}N\left(t\right)&=&1/\mu,\textrm{ c.s.}
\end{eqnarray}
\end{Teo}


Es decir, $T_{n}$ satisface la Ley Fuerte de los Grandes N\'umeros s\'i y s\'olo s\'i $N\left/t\right)$ la cumple.


\begin{Coro}[Ley Fuerte de los Grandes N\'umeros para Procesos de Renovaci\'on]
Si $N\left(t\right)$ es un proceso de renovaci\'on cuyos tiempos de inter-renovaci\'on tienen media $\mu\leq\infty$, entonces
\begin{eqnarray}
t^{-1}N\left(t\right)\rightarrow 1/\mu,\textrm{ c.s. cuando }t\rightarrow\infty.
\end{eqnarray}

\end{Coro}


Considerar el proceso estoc\'astico de valores reales $\left\{Z\left(t\right):t\geq0\right\}$ en el mismo espacio de probabilidad que $N\left(t\right)$

\begin{Def}
Para el proceso $\left\{Z\left(t\right):t\geq0\right\}$ se define la fluctuaci\'on m\'axima de $Z\left(t\right)$ en el intervalo $\left(T_{n-1},T_{n}\right]$:
\begin{eqnarray*}
M_{n}=\sup_{T_{n-1}<t\leq T_{n}}|Z\left(t\right)-Z\left(T_{n-1}\right)|
\end{eqnarray*}
\end{Def}

\begin{Teo}
Sup\'ongase que $n^{-1}T_{n}\rightarrow\mu$ c.s. cuando $n\rightarrow\infty$, donde $\mu\leq\infty$ es una constante o variable aleatoria. Sea $a$ una constante o variable aleatoria que puede ser infinita cuando $\mu$ es finita, y considere las expresiones l\'imite:
\begin{eqnarray}
lim_{n\rightarrow\infty}n^{-1}Z\left(T_{n}\right)&=&a,\textrm{ c.s.}\\
lim_{t\rightarrow\infty}t^{-1}Z\left(t\right)&=&a/\mu,\textrm{ c.s.}
\end{eqnarray}
La segunda expresi\'on implica la primera. Conversamente, la primera implica la segunda si el proceso $Z\left(t\right)$ es creciente, o si $lim_{n\rightarrow\infty}n^{-1}M_{n}=0$ c.s.
\end{Teo}

\begin{Coro}
Si $N\left(t\right)$ es un proceso de renovaci\'on, y $\left(Z\left(T_{n}\right)-Z\left(T_{n-1}\right),M_{n}\right)$, para $n\geq1$, son variables aleatorias independientes e id\'enticamente distribuidas con media finita, entonces,
\begin{eqnarray}
lim_{t\rightarrow\infty}t^{-1}Z\left(t\right)\rightarrow\frac{\esp\left[Z\left(T_{1}\right)-Z\left(T_{0}\right)\right]}{\esp\left[T_{1}\right]},\textrm{ c.s. cuando  }t\rightarrow\infty.
\end{eqnarray}
\end{Coro}
%___________________________________________________________________________________________
%
%\subsection{Propiedades de los Procesos de Renovaci\'on}
%___________________________________________________________________________________________
%

Los tiempos $T_{n}$ est\'an relacionados con los conteos de $N\left(t\right)$ por

\begin{eqnarray*}
\left\{N\left(t\right)\geq n\right\}&=&\left\{T_{n}\leq t\right\}\\
T_{N\left(t\right)}\leq &t&<T_{N\left(t\right)+1},
\end{eqnarray*}

adem\'as $N\left(T_{n}\right)=n$, y 

\begin{eqnarray*}
N\left(t\right)=\max\left\{n:T_{n}\leq t\right\}=\min\left\{n:T_{n+1}>t\right\}
\end{eqnarray*}

Por propiedades de la convoluci\'on se sabe que

\begin{eqnarray*}
P\left\{T_{n}\leq t\right\}=F^{n\star}\left(t\right)
\end{eqnarray*}
que es la $n$-\'esima convoluci\'on de $F$. Entonces 

\begin{eqnarray*}
\left\{N\left(t\right)\geq n\right\}&=&\left\{T_{n}\leq t\right\}\\
P\left\{N\left(t\right)\leq n\right\}&=&1-F^{\left(n+1\right)\star}\left(t\right)
\end{eqnarray*}

Adem\'as usando el hecho de que $\esp\left[N\left(t\right)\right]=\sum_{n=1}^{\infty}P\left\{N\left(t\right)\geq n\right\}$
se tiene que

\begin{eqnarray*}
\esp\left[N\left(t\right)\right]=\sum_{n=1}^{\infty}F^{n\star}\left(t\right)
\end{eqnarray*}

\begin{Prop}
Para cada $t\geq0$, la funci\'on generadora de momentos $\esp\left[e^{\alpha N\left(t\right)}\right]$ existe para alguna $\alpha$ en una vecindad del 0, y de aqu\'i que $\esp\left[N\left(t\right)^{m}\right]<\infty$, para $m\geq1$.
\end{Prop}


\begin{Note}
Si el primer tiempo de renovaci\'on $\xi_{1}$ no tiene la misma distribuci\'on que el resto de las $\xi_{n}$, para $n\geq2$, a $N\left(t\right)$ se le llama Proceso de Renovaci\'on retardado, donde si $\xi$ tiene distribuci\'on $G$, entonces el tiempo $T_{n}$ de la $n$-\'esima renovaci\'on tiene distribuci\'on $G\star F^{\left(n-1\right)\star}\left(t\right)$
\end{Note}


\begin{Teo}
Para una constante $\mu\leq\infty$ ( o variable aleatoria), las siguientes expresiones son equivalentes:

\begin{eqnarray}
lim_{n\rightarrow\infty}n^{-1}T_{n}&=&\mu,\textrm{ c.s.}\\
lim_{t\rightarrow\infty}t^{-1}N\left(t\right)&=&1/\mu,\textrm{ c.s.}
\end{eqnarray}
\end{Teo}


Es decir, $T_{n}$ satisface la Ley Fuerte de los Grandes N\'umeros s\'i y s\'olo s\'i $N\left/t\right)$ la cumple.


\begin{Coro}[Ley Fuerte de los Grandes N\'umeros para Procesos de Renovaci\'on]
Si $N\left(t\right)$ es un proceso de renovaci\'on cuyos tiempos de inter-renovaci\'on tienen media $\mu\leq\infty$, entonces
\begin{eqnarray}
t^{-1}N\left(t\right)\rightarrow 1/\mu,\textrm{ c.s. cuando }t\rightarrow\infty.
\end{eqnarray}

\end{Coro}


Considerar el proceso estoc\'astico de valores reales $\left\{Z\left(t\right):t\geq0\right\}$ en el mismo espacio de probabilidad que $N\left(t\right)$

\begin{Def}
Para el proceso $\left\{Z\left(t\right):t\geq0\right\}$ se define la fluctuaci\'on m\'axima de $Z\left(t\right)$ en el intervalo $\left(T_{n-1},T_{n}\right]$:
\begin{eqnarray*}
M_{n}=\sup_{T_{n-1}<t\leq T_{n}}|Z\left(t\right)-Z\left(T_{n-1}\right)|
\end{eqnarray*}
\end{Def}

\begin{Teo}
Sup\'ongase que $n^{-1}T_{n}\rightarrow\mu$ c.s. cuando $n\rightarrow\infty$, donde $\mu\leq\infty$ es una constante o variable aleatoria. Sea $a$ una constante o variable aleatoria que puede ser infinita cuando $\mu$ es finita, y considere las expresiones l\'imite:
\begin{eqnarray}
lim_{n\rightarrow\infty}n^{-1}Z\left(T_{n}\right)&=&a,\textrm{ c.s.}\\
lim_{t\rightarrow\infty}t^{-1}Z\left(t\right)&=&a/\mu,\textrm{ c.s.}
\end{eqnarray}
La segunda expresi\'on implica la primera. Conversamente, la primera implica la segunda si el proceso $Z\left(t\right)$ es creciente, o si $lim_{n\rightarrow\infty}n^{-1}M_{n}=0$ c.s.
\end{Teo}

\begin{Coro}
Si $N\left(t\right)$ es un proceso de renovaci\'on, y $\left(Z\left(T_{n}\right)-Z\left(T_{n-1}\right),M_{n}\right)$, para $n\geq1$, son variables aleatorias independientes e id\'enticamente distribuidas con media finita, entonces,
\begin{eqnarray}
lim_{t\rightarrow\infty}t^{-1}Z\left(t\right)\rightarrow\frac{\esp\left[Z\left(T_{1}\right)-Z\left(T_{0}\right)\right]}{\esp\left[T_{1}\right]},\textrm{ c.s. cuando  }t\rightarrow\infty.
\end{eqnarray}
\end{Coro}


%___________________________________________________________________________________________
%
%\subsection{Funci\'on de Renovaci\'on}
%___________________________________________________________________________________________
%


\begin{Def}
Sea $h\left(t\right)$ funci\'on de valores reales en $\rea$ acotada en intervalos finitos e igual a cero para $t<0$ La ecuaci\'on de renovaci\'on para $h\left(t\right)$ y la distribuci\'on $F$ es

\begin{eqnarray}%\label{Ec.Renovacion}
H\left(t\right)=h\left(t\right)+\int_{\left[0,t\right]}H\left(t-s\right)dF\left(s\right)\textrm{,    }t\geq0,
\end{eqnarray}
donde $H\left(t\right)$ es una funci\'on de valores reales. Esto es $H=h+F\star H$. Decimos que $H\left(t\right)$ es soluci\'on de esta ecuaci\'on si satisface la ecuaci\'on, y es acotada en intervalos finitos e iguales a cero para $t<0$.
\end{Def}

\begin{Prop}
La funci\'on $U\star h\left(t\right)$ es la \'unica soluci\'on de la ecuaci\'on de renovaci\'on (\ref{Ec.Renovacion}).
\end{Prop}

\begin{Teo}[Teorema Renovaci\'on Elemental]
\begin{eqnarray*}
t^{-1}U\left(t\right)\rightarrow 1/\mu\textrm{,    cuando }t\rightarrow\infty.
\end{eqnarray*}
\end{Teo}

%___________________________________________________________________________________________
%
%\subsection{Funci\'on de Renovaci\'on}
%___________________________________________________________________________________________
%


Sup\'ongase que $N\left(t\right)$ es un proceso de renovaci\'on con distribuci\'on $F$ con media finita $\mu$.

\begin{Def}
La funci\'on de renovaci\'on asociada con la distribuci\'on $F$, del proceso $N\left(t\right)$, es
\begin{eqnarray*}
U\left(t\right)=\sum_{n=1}^{\infty}F^{n\star}\left(t\right),\textrm{   }t\geq0,
\end{eqnarray*}
donde $F^{0\star}\left(t\right)=\indora\left(t\geq0\right)$.
\end{Def}


\begin{Prop}
Sup\'ongase que la distribuci\'on de inter-renovaci\'on $F$ tiene densidad $f$. Entonces $U\left(t\right)$ tambi\'en tiene densidad, para $t>0$, y es $U^{'}\left(t\right)=\sum_{n=0}^{\infty}f^{n\star}\left(t\right)$. Adem\'as
\begin{eqnarray*}
\prob\left\{N\left(t\right)>N\left(t-\right)\right\}=0\textrm{,   }t\geq0.
\end{eqnarray*}
\end{Prop}

\begin{Def}
La Transformada de Laplace-Stieljes de $F$ est\'a dada por

\begin{eqnarray*}
\hat{F}\left(\alpha\right)=\int_{\rea_{+}}e^{-\alpha t}dF\left(t\right)\textrm{,  }\alpha\geq0.
\end{eqnarray*}
\end{Def}

Entonces

\begin{eqnarray*}
\hat{U}\left(\alpha\right)=\sum_{n=0}^{\infty}\hat{F^{n\star}}\left(\alpha\right)=\sum_{n=0}^{\infty}\hat{F}\left(\alpha\right)^{n}=\frac{1}{1-\hat{F}\left(\alpha\right)}.
\end{eqnarray*}


\begin{Prop}
La Transformada de Laplace $\hat{U}\left(\alpha\right)$ y $\hat{F}\left(\alpha\right)$ determina una a la otra de manera \'unica por la relaci\'on $\hat{U}\left(\alpha\right)=\frac{1}{1-\hat{F}\left(\alpha\right)}$.
\end{Prop}


\begin{Note}
Un proceso de renovaci\'on $N\left(t\right)$ cuyos tiempos de inter-renovaci\'on tienen media finita, es un proceso Poisson con tasa $\lambda$ si y s\'olo s\'i $\esp\left[U\left(t\right)\right]=\lambda t$, para $t\geq0$.
\end{Note}


\begin{Teo}
Sea $N\left(t\right)$ un proceso puntual simple con puntos de localizaci\'on $T_{n}$ tal que $\eta\left(t\right)=\esp\left[N\left(\right)\right]$ es finita para cada $t$. Entonces para cualquier funci\'on $f:\rea_{+}\rightarrow\rea$,
\begin{eqnarray*}
\esp\left[\sum_{n=1}^{N\left(\right)}f\left(T_{n}\right)\right]=\int_{\left(0,t\right]}f\left(s\right)d\eta\left(s\right)\textrm{,  }t\geq0,
\end{eqnarray*}
suponiendo que la integral exista. Adem\'as si $X_{1},X_{2},\ldots$ son variables aleatorias definidas en el mismo espacio de probabilidad que el proceso $N\left(t\right)$ tal que $\esp\left[X_{n}|T_{n}=s\right]=f\left(s\right)$, independiente de $n$. Entonces
\begin{eqnarray*}
\esp\left[\sum_{n=1}^{N\left(t\right)}X_{n}\right]=\int_{\left(0,t\right]}f\left(s\right)d\eta\left(s\right)\textrm{,  }t\geq0,
\end{eqnarray*} 
suponiendo que la integral exista. 
\end{Teo}

\begin{Coro}[Identidad de Wald para Renovaciones]
Para el proceso de renovaci\'on $N\left(t\right)$,
\begin{eqnarray*}
\esp\left[T_{N\left(t\right)+1}\right]=\mu\esp\left[N\left(t\right)+1\right]\textrm{,  }t\geq0,
\end{eqnarray*}  
\end{Coro}

%______________________________________________________________________
%\subsection{Procesos de Renovaci\'on}
%______________________________________________________________________

\begin{Def}%\label{Def.Tn}
Sean $0\leq T_{1}\leq T_{2}\leq \ldots$ son tiempos aleatorios infinitos en los cuales ocurren ciertos eventos. El n\'umero de tiempos $T_{n}$ en el intervalo $\left[0,t\right)$ es

\begin{eqnarray}
N\left(t\right)=\sum_{n=1}^{\infty}\indora\left(T_{n}\leq t\right),
\end{eqnarray}
para $t\geq0$.
\end{Def}

Si se consideran los puntos $T_{n}$ como elementos de $\rea_{+}$, y $N\left(t\right)$ es el n\'umero de puntos en $\rea$. El proceso denotado por $\left\{N\left(t\right):t\geq0\right\}$, denotado por $N\left(t\right)$, es un proceso puntual en $\rea_{+}$. Los $T_{n}$ son los tiempos de ocurrencia, el proceso puntual $N\left(t\right)$ es simple si su n\'umero de ocurrencias son distintas: $0<T_{1}<T_{2}<\ldots$ casi seguramente.

\begin{Def}
Un proceso puntual $N\left(t\right)$ es un proceso de renovaci\'on si los tiempos de interocurrencia $\xi_{n}=T_{n}-T_{n-1}$, para $n\geq1$, son independientes e identicamente distribuidos con distribuci\'on $F$, donde $F\left(0\right)=0$ y $T_{0}=0$. Los $T_{n}$ son llamados tiempos de renovaci\'on, referente a la independencia o renovaci\'on de la informaci\'on estoc\'astica en estos tiempos. Los $\xi_{n}$ son los tiempos de inter-renovaci\'on, y $N\left(t\right)$ es el n\'umero de renovaciones en el intervalo $\left[0,t\right)$
\end{Def}


\begin{Note}
Para definir un proceso de renovaci\'on para cualquier contexto, solamente hay que especificar una distribuci\'on $F$, con $F\left(0\right)=0$, para los tiempos de inter-renovaci\'on. La funci\'on $F$ en turno degune las otra variables aleatorias. De manera formal, existe un espacio de probabilidad y una sucesi\'on de variables aleatorias $\xi_{1},\xi_{2},\ldots$ definidas en este con distribuci\'on $F$. Entonces las otras cantidades son $T_{n}=\sum_{k=1}^{n}\xi_{k}$ y $N\left(t\right)=\sum_{n=1}^{\infty}\indora\left(T_{n}\leq t\right)$, donde $T_{n}\rightarrow\infty$ casi seguramente por la Ley Fuerte de los Grandes Números.
\end{Note}

%___________________________________________________________________________________________
%
%\subsection{Renewal and Regenerative Processes: Serfozo\cite{Serfozo}}
%___________________________________________________________________________________________
%
\begin{Def}%\label{Def.Tn}
Sean $0\leq T_{1}\leq T_{2}\leq \ldots$ son tiempos aleatorios infinitos en los cuales ocurren ciertos eventos. El n\'umero de tiempos $T_{n}$ en el intervalo $\left[0,t\right)$ es

\begin{eqnarray}
N\left(t\right)=\sum_{n=1}^{\infty}\indora\left(T_{n}\leq t\right),
\end{eqnarray}
para $t\geq0$.
\end{Def}

Si se consideran los puntos $T_{n}$ como elementos de $\rea_{+}$, y $N\left(t\right)$ es el n\'umero de puntos en $\rea$. El proceso denotado por $\left\{N\left(t\right):t\geq0\right\}$, denotado por $N\left(t\right)$, es un proceso puntual en $\rea_{+}$. Los $T_{n}$ son los tiempos de ocurrencia, el proceso puntual $N\left(t\right)$ es simple si su n\'umero de ocurrencias son distintas: $0<T_{1}<T_{2}<\ldots$ casi seguramente.

\begin{Def}
Un proceso puntual $N\left(t\right)$ es un proceso de renovaci\'on si los tiempos de interocurrencia $\xi_{n}=T_{n}-T_{n-1}$, para $n\geq1$, son independientes e identicamente distribuidos con distribuci\'on $F$, donde $F\left(0\right)=0$ y $T_{0}=0$. Los $T_{n}$ son llamados tiempos de renovaci\'on, referente a la independencia o renovaci\'on de la informaci\'on estoc\'astica en estos tiempos. Los $\xi_{n}$ son los tiempos de inter-renovaci\'on, y $N\left(t\right)$ es el n\'umero de renovaciones en el intervalo $\left[0,t\right)$
\end{Def}


\begin{Note}
Para definir un proceso de renovaci\'on para cualquier contexto, solamente hay que especificar una distribuci\'on $F$, con $F\left(0\right)=0$, para los tiempos de inter-renovaci\'on. La funci\'on $F$ en turno degune las otra variables aleatorias. De manera formal, existe un espacio de probabilidad y una sucesi\'on de variables aleatorias $\xi_{1},\xi_{2},\ldots$ definidas en este con distribuci\'on $F$. Entonces las otras cantidades son $T_{n}=\sum_{k=1}^{n}\xi_{k}$ y $N\left(t\right)=\sum_{n=1}^{\infty}\indora\left(T_{n}\leq t\right)$, donde $T_{n}\rightarrow\infty$ casi seguramente por la Ley Fuerte de los Grandes N\'umeros.
\end{Note}







Los tiempos $T_{n}$ est\'an relacionados con los conteos de $N\left(t\right)$ por

\begin{eqnarray*}
\left\{N\left(t\right)\geq n\right\}&=&\left\{T_{n}\leq t\right\}\\
T_{N\left(t\right)}\leq &t&<T_{N\left(t\right)+1},
\end{eqnarray*}

adem\'as $N\left(T_{n}\right)=n$, y 

\begin{eqnarray*}
N\left(t\right)=\max\left\{n:T_{n}\leq t\right\}=\min\left\{n:T_{n+1}>t\right\}
\end{eqnarray*}

Por propiedades de la convoluci\'on se sabe que

\begin{eqnarray*}
P\left\{T_{n}\leq t\right\}=F^{n\star}\left(t\right)
\end{eqnarray*}
que es la $n$-\'esima convoluci\'on de $F$. Entonces 

\begin{eqnarray*}
\left\{N\left(t\right)\geq n\right\}&=&\left\{T_{n}\leq t\right\}\\
P\left\{N\left(t\right)\leq n\right\}&=&1-F^{\left(n+1\right)\star}\left(t\right)
\end{eqnarray*}

Adem\'as usando el hecho de que $\esp\left[N\left(t\right)\right]=\sum_{n=1}^{\infty}P\left\{N\left(t\right)\geq n\right\}$
se tiene que

\begin{eqnarray*}
\esp\left[N\left(t\right)\right]=\sum_{n=1}^{\infty}F^{n\star}\left(t\right)
\end{eqnarray*}

\begin{Prop}
Para cada $t\geq0$, la funci\'on generadora de momentos $\esp\left[e^{\alpha N\left(t\right)}\right]$ existe para alguna $\alpha$ en una vecindad del 0, y de aqu\'i que $\esp\left[N\left(t\right)^{m}\right]<\infty$, para $m\geq1$.
\end{Prop}

\begin{Ejem}[\textbf{Proceso Poisson}]

Suponga que se tienen tiempos de inter-renovaci\'on \textit{i.i.d.} del proceso de renovaci\'on $N\left(t\right)$ tienen distribuci\'on exponencial $F\left(t\right)=q-e^{-\lambda t}$ con tasa $\lambda$. Entonces $N\left(t\right)$ es un proceso Poisson con tasa $\lambda$.

\end{Ejem}


\begin{Note}
Si el primer tiempo de renovaci\'on $\xi_{1}$ no tiene la misma distribuci\'on que el resto de las $\xi_{n}$, para $n\geq2$, a $N\left(t\right)$ se le llama Proceso de Renovaci\'on retardado, donde si $\xi$ tiene distribuci\'on $G$, entonces el tiempo $T_{n}$ de la $n$-\'esima renovaci\'on tiene distribuci\'on $G\star F^{\left(n-1\right)\star}\left(t\right)$
\end{Note}


\begin{Teo}
Para una constante $\mu\leq\infty$ ( o variable aleatoria), las siguientes expresiones son equivalentes:

\begin{eqnarray}
lim_{n\rightarrow\infty}n^{-1}T_{n}&=&\mu,\textrm{ c.s.}\\
lim_{t\rightarrow\infty}t^{-1}N\left(t\right)&=&1/\mu,\textrm{ c.s.}
\end{eqnarray}
\end{Teo}


Es decir, $T_{n}$ satisface la Ley Fuerte de los Grandes N\'umeros s\'i y s\'olo s\'i $N\left/t\right)$ la cumple.


\begin{Coro}[Ley Fuerte de los Grandes N\'umeros para Procesos de Renovaci\'on]
Si $N\left(t\right)$ es un proceso de renovaci\'on cuyos tiempos de inter-renovaci\'on tienen media $\mu\leq\infty$, entonces
\begin{eqnarray}
t^{-1}N\left(t\right)\rightarrow 1/\mu,\textrm{ c.s. cuando }t\rightarrow\infty.
\end{eqnarray}

\end{Coro}


Considerar el proceso estoc\'astico de valores reales $\left\{Z\left(t\right):t\geq0\right\}$ en el mismo espacio de probabilidad que $N\left(t\right)$

\begin{Def}
Para el proceso $\left\{Z\left(t\right):t\geq0\right\}$ se define la fluctuaci\'on m\'axima de $Z\left(t\right)$ en el intervalo $\left(T_{n-1},T_{n}\right]$:
\begin{eqnarray*}
M_{n}=\sup_{T_{n-1}<t\leq T_{n}}|Z\left(t\right)-Z\left(T_{n-1}\right)|
\end{eqnarray*}
\end{Def}

\begin{Teo}
Sup\'ongase que $n^{-1}T_{n}\rightarrow\mu$ c.s. cuando $n\rightarrow\infty$, donde $\mu\leq\infty$ es una constante o variable aleatoria. Sea $a$ una constante o variable aleatoria que puede ser infinita cuando $\mu$ es finita, y considere las expresiones l\'imite:
\begin{eqnarray}
lim_{n\rightarrow\infty}n^{-1}Z\left(T_{n}\right)&=&a,\textrm{ c.s.}\\
lim_{t\rightarrow\infty}t^{-1}Z\left(t\right)&=&a/\mu,\textrm{ c.s.}
\end{eqnarray}
La segunda expresi\'on implica la primera. Conversamente, la primera implica la segunda si el proceso $Z\left(t\right)$ es creciente, o si $lim_{n\rightarrow\infty}n^{-1}M_{n}=0$ c.s.
\end{Teo}

\begin{Coro}
Si $N\left(t\right)$ es un proceso de renovaci\'on, y $\left(Z\left(T_{n}\right)-Z\left(T_{n-1}\right),M_{n}\right)$, para $n\geq1$, son variables aleatorias independientes e id\'enticamente distribuidas con media finita, entonces,
\begin{eqnarray}
lim_{t\rightarrow\infty}t^{-1}Z\left(t\right)\rightarrow\frac{\esp\left[Z\left(T_{1}\right)-Z\left(T_{0}\right)\right]}{\esp\left[T_{1}\right]},\textrm{ c.s. cuando  }t\rightarrow\infty.
\end{eqnarray}
\end{Coro}


Sup\'ongase que $N\left(t\right)$ es un proceso de renovaci\'on con distribuci\'on $F$ con media finita $\mu$.

\begin{Def}
La funci\'on de renovaci\'on asociada con la distribuci\'on $F$, del proceso $N\left(t\right)$, es
\begin{eqnarray*}
U\left(t\right)=\sum_{n=1}^{\infty}F^{n\star}\left(t\right),\textrm{   }t\geq0,
\end{eqnarray*}
donde $F^{0\star}\left(t\right)=\indora\left(t\geq0\right)$.
\end{Def}


\begin{Prop}
Sup\'ongase que la distribuci\'on de inter-renovaci\'on $F$ tiene densidad $f$. Entonces $U\left(t\right)$ tambi\'en tiene densidad, para $t>0$, y es $U^{'}\left(t\right)=\sum_{n=0}^{\infty}f^{n\star}\left(t\right)$. Adem\'as
\begin{eqnarray*}
\prob\left\{N\left(t\right)>N\left(t-\right)\right\}=0\textrm{,   }t\geq0.
\end{eqnarray*}
\end{Prop}

\begin{Def}
La Transformada de Laplace-Stieljes de $F$ est\'a dada por

\begin{eqnarray*}
\hat{F}\left(\alpha\right)=\int_{\rea_{+}}e^{-\alpha t}dF\left(t\right)\textrm{,  }\alpha\geq0.
\end{eqnarray*}
\end{Def}

Entonces

\begin{eqnarray*}
\hat{U}\left(\alpha\right)=\sum_{n=0}^{\infty}\hat{F^{n\star}}\left(\alpha\right)=\sum_{n=0}^{\infty}\hat{F}\left(\alpha\right)^{n}=\frac{1}{1-\hat{F}\left(\alpha\right)}.
\end{eqnarray*}


\begin{Prop}
La Transformada de Laplace $\hat{U}\left(\alpha\right)$ y $\hat{F}\left(\alpha\right)$ determina una a la otra de manera \'unica por la relaci\'on $\hat{U}\left(\alpha\right)=\frac{1}{1-\hat{F}\left(\alpha\right)}$.
\end{Prop}


\begin{Note}
Un proceso de renovaci\'on $N\left(t\right)$ cuyos tiempos de inter-renovaci\'on tienen media finita, es un proceso Poisson con tasa $\lambda$ si y s\'olo s\'i $\esp\left[U\left(t\right)\right]=\lambda t$, para $t\geq0$.
\end{Note}


\begin{Teo}
Sea $N\left(t\right)$ un proceso puntual simple con puntos de localizaci\'on $T_{n}$ tal que $\eta\left(t\right)=\esp\left[N\left(\right)\right]$ es finita para cada $t$. Entonces para cualquier funci\'on $f:\rea_{+}\rightarrow\rea$,
\begin{eqnarray*}
\esp\left[\sum_{n=1}^{N\left(\right)}f\left(T_{n}\right)\right]=\int_{\left(0,t\right]}f\left(s\right)d\eta\left(s\right)\textrm{,  }t\geq0,
\end{eqnarray*}
suponiendo que la integral exista. Adem\'as si $X_{1},X_{2},\ldots$ son variables aleatorias definidas en el mismo espacio de probabilidad que el proceso $N\left(t\right)$ tal que $\esp\left[X_{n}|T_{n}=s\right]=f\left(s\right)$, independiente de $n$. Entonces
\begin{eqnarray*}
\esp\left[\sum_{n=1}^{N\left(t\right)}X_{n}\right]=\int_{\left(0,t\right]}f\left(s\right)d\eta\left(s\right)\textrm{,  }t\geq0,
\end{eqnarray*} 
suponiendo que la integral exista. 
\end{Teo}

\begin{Coro}[Identidad de Wald para Renovaciones]
Para el proceso de renovaci\'on $N\left(t\right)$,
\begin{eqnarray*}
\esp\left[T_{N\left(t\right)+1}\right]=\mu\esp\left[N\left(t\right)+1\right]\textrm{,  }t\geq0,
\end{eqnarray*}  
\end{Coro}


\begin{Def}
Sea $h\left(t\right)$ funci\'on de valores reales en $\rea$ acotada en intervalos finitos e igual a cero para $t<0$ La ecuaci\'on de renovaci\'on para $h\left(t\right)$ y la distribuci\'on $F$ es

\begin{eqnarray}%\label{Ec.Renovacion}
H\left(t\right)=h\left(t\right)+\int_{\left[0,t\right]}H\left(t-s\right)dF\left(s\right)\textrm{,    }t\geq0,
\end{eqnarray}
donde $H\left(t\right)$ es una funci\'on de valores reales. Esto es $H=h+F\star H$. Decimos que $H\left(t\right)$ es soluci\'on de esta ecuaci\'on si satisface la ecuaci\'on, y es acotada en intervalos finitos e iguales a cero para $t<0$.
\end{Def}

\begin{Prop}
La funci\'on $U\star h\left(t\right)$ es la \'unica soluci\'on de la ecuaci\'on de renovaci\'on (\ref{Ec.Renovacion}).
\end{Prop}

\begin{Teo}[Teorema Renovaci\'on Elemental]
\begin{eqnarray*}
t^{-1}U\left(t\right)\rightarrow 1/\mu\textrm{,    cuando }t\rightarrow\infty.
\end{eqnarray*}
\end{Teo}



Sup\'ongase que $N\left(t\right)$ es un proceso de renovaci\'on con distribuci\'on $F$ con media finita $\mu$.

\begin{Def}
La funci\'on de renovaci\'on asociada con la distribuci\'on $F$, del proceso $N\left(t\right)$, es
\begin{eqnarray*}
U\left(t\right)=\sum_{n=1}^{\infty}F^{n\star}\left(t\right),\textrm{   }t\geq0,
\end{eqnarray*}
donde $F^{0\star}\left(t\right)=\indora\left(t\geq0\right)$.
\end{Def}


\begin{Prop}
Sup\'ongase que la distribuci\'on de inter-renovaci\'on $F$ tiene densidad $f$. Entonces $U\left(t\right)$ tambi\'en tiene densidad, para $t>0$, y es $U^{'}\left(t\right)=\sum_{n=0}^{\infty}f^{n\star}\left(t\right)$. Adem\'as
\begin{eqnarray*}
\prob\left\{N\left(t\right)>N\left(t-\right)\right\}=0\textrm{,   }t\geq0.
\end{eqnarray*}
\end{Prop}

\begin{Def}
La Transformada de Laplace-Stieljes de $F$ est\'a dada por

\begin{eqnarray*}
\hat{F}\left(\alpha\right)=\int_{\rea_{+}}e^{-\alpha t}dF\left(t\right)\textrm{,  }\alpha\geq0.
\end{eqnarray*}
\end{Def}

Entonces

\begin{eqnarray*}
\hat{U}\left(\alpha\right)=\sum_{n=0}^{\infty}\hat{F^{n\star}}\left(\alpha\right)=\sum_{n=0}^{\infty}\hat{F}\left(\alpha\right)^{n}=\frac{1}{1-\hat{F}\left(\alpha\right)}.
\end{eqnarray*}


\begin{Prop}
La Transformada de Laplace $\hat{U}\left(\alpha\right)$ y $\hat{F}\left(\alpha\right)$ determina una a la otra de manera \'unica por la relaci\'on $\hat{U}\left(\alpha\right)=\frac{1}{1-\hat{F}\left(\alpha\right)}$.
\end{Prop}


\begin{Note}
Un proceso de renovaci\'on $N\left(t\right)$ cuyos tiempos de inter-renovaci\'on tienen media finita, es un proceso Poisson con tasa $\lambda$ si y s\'olo s\'i $\esp\left[U\left(t\right)\right]=\lambda t$, para $t\geq0$.
\end{Note}


\begin{Teo}
Sea $N\left(t\right)$ un proceso puntual simple con puntos de localizaci\'on $T_{n}$ tal que $\eta\left(t\right)=\esp\left[N\left(\right)\right]$ es finita para cada $t$. Entonces para cualquier funci\'on $f:\rea_{+}\rightarrow\rea$,
\begin{eqnarray*}
\esp\left[\sum_{n=1}^{N\left(\right)}f\left(T_{n}\right)\right]=\int_{\left(0,t\right]}f\left(s\right)d\eta\left(s\right)\textrm{,  }t\geq0,
\end{eqnarray*}
suponiendo que la integral exista. Adem\'as si $X_{1},X_{2},\ldots$ son variables aleatorias definidas en el mismo espacio de probabilidad que el proceso $N\left(t\right)$ tal que $\esp\left[X_{n}|T_{n}=s\right]=f\left(s\right)$, independiente de $n$. Entonces
\begin{eqnarray*}
\esp\left[\sum_{n=1}^{N\left(t\right)}X_{n}\right]=\int_{\left(0,t\right]}f\left(s\right)d\eta\left(s\right)\textrm{,  }t\geq0,
\end{eqnarray*} 
suponiendo que la integral exista. 
\end{Teo}

\begin{Coro}[Identidad de Wald para Renovaciones]
Para el proceso de renovaci\'on $N\left(t\right)$,
\begin{eqnarray*}
\esp\left[T_{N\left(t\right)+1}\right]=\mu\esp\left[N\left(t\right)+1\right]\textrm{,  }t\geq0,
\end{eqnarray*}  
\end{Coro}


\begin{Def}
Sea $h\left(t\right)$ funci\'on de valores reales en $\rea$ acotada en intervalos finitos e igual a cero para $t<0$ La ecuaci\'on de renovaci\'on para $h\left(t\right)$ y la distribuci\'on $F$ es

\begin{eqnarray}%\label{Ec.Renovacion}
H\left(t\right)=h\left(t\right)+\int_{\left[0,t\right]}H\left(t-s\right)dF\left(s\right)\textrm{,    }t\geq0,
\end{eqnarray}
donde $H\left(t\right)$ es una funci\'on de valores reales. Esto es $H=h+F\star H$. Decimos que $H\left(t\right)$ es soluci\'on de esta ecuaci\'on si satisface la ecuaci\'on, y es acotada en intervalos finitos e iguales a cero para $t<0$.
\end{Def}

\begin{Prop}
La funci\'on $U\star h\left(t\right)$ es la \'unica soluci\'on de la ecuaci\'on de renovaci\'on (\ref{Ec.Renovacion}).
\end{Prop}

\begin{Teo}[Teorema Renovaci\'on Elemental]
\begin{eqnarray*}
t^{-1}U\left(t\right)\rightarrow 1/\mu\textrm{,    cuando }t\rightarrow\infty.
\end{eqnarray*}
\end{Teo}


\begin{Note} Una funci\'on $h:\rea_{+}\rightarrow\rea$ es Directamente Riemann Integrable en los siguientes casos:
\begin{itemize}
\item[a)] $h\left(t\right)\geq0$ es decreciente y Riemann Integrable.
\item[b)] $h$ es continua excepto posiblemente en un conjunto de Lebesgue de medida 0, y $|h\left(t\right)|\leq b\left(t\right)$, donde $b$ es DRI.
\end{itemize}
\end{Note}

\begin{Teo}[Teorema Principal de Renovaci\'on]
Si $F$ es no aritm\'etica y $h\left(t\right)$ es Directamente Riemann Integrable (DRI), entonces

\begin{eqnarray*}
lim_{t\rightarrow\infty}U\star h=\frac{1}{\mu}\int_{\rea_{+}}h\left(s\right)ds.
\end{eqnarray*}
\end{Teo}

\begin{Prop}
Cualquier funci\'on $H\left(t\right)$ acotada en intervalos finitos y que es 0 para $t<0$ puede expresarse como
\begin{eqnarray*}
H\left(t\right)=U\star h\left(t\right)\textrm{,  donde }h\left(t\right)=H\left(t\right)-F\star H\left(t\right)
\end{eqnarray*}
\end{Prop}

\begin{Def}
Un proceso estoc\'astico $X\left(t\right)$ es crudamente regenerativo en un tiempo aleatorio positivo $T$ si
\begin{eqnarray*}
\esp\left[X\left(T+t\right)|T\right]=\esp\left[X\left(t\right)\right]\textrm{, para }t\geq0,\end{eqnarray*}
y con las esperanzas anteriores finitas.
\end{Def}

\begin{Prop}
Sup\'ongase que $X\left(t\right)$ es un proceso crudamente regenerativo en $T$, que tiene distribuci\'on $F$. Si $\esp\left[X\left(t\right)\right]$ es acotado en intervalos finitos, entonces
\begin{eqnarray*}
\esp\left[X\left(t\right)\right]=U\star h\left(t\right)\textrm{,  donde }h\left(t\right)=\esp\left[X\left(t\right)\indora\left(T>t\right)\right].
\end{eqnarray*}
\end{Prop}

\begin{Teo}[Regeneraci\'on Cruda]
Sup\'ongase que $X\left(t\right)$ es un proceso con valores positivo crudamente regenerativo en $T$, y def\'inase $M=\sup\left\{|X\left(t\right)|:t\leq T\right\}$. Si $T$ es no aritm\'etico y $M$ y $MT$ tienen media finita, entonces
\begin{eqnarray*}
lim_{t\rightarrow\infty}\esp\left[X\left(t\right)\right]=\frac{1}{\mu}\int_{\rea_{+}}h\left(s\right)ds,
\end{eqnarray*}
donde $h\left(t\right)=\esp\left[X\left(t\right)\indora\left(T>t\right)\right]$.
\end{Teo}


\begin{Note} Una funci\'on $h:\rea_{+}\rightarrow\rea$ es Directamente Riemann Integrable en los siguientes casos:
\begin{itemize}
\item[a)] $h\left(t\right)\geq0$ es decreciente y Riemann Integrable.
\item[b)] $h$ es continua excepto posiblemente en un conjunto de Lebesgue de medida 0, y $|h\left(t\right)|\leq b\left(t\right)$, donde $b$ es DRI.
\end{itemize}
\end{Note}

\begin{Teo}[Teorema Principal de Renovaci\'on]
Si $F$ es no aritm\'etica y $h\left(t\right)$ es Directamente Riemann Integrable (DRI), entonces

\begin{eqnarray*}
lim_{t\rightarrow\infty}U\star h=\frac{1}{\mu}\int_{\rea_{+}}h\left(s\right)ds.
\end{eqnarray*}
\end{Teo}

\begin{Prop}
Cualquier funci\'on $H\left(t\right)$ acotada en intervalos finitos y que es 0 para $t<0$ puede expresarse como
\begin{eqnarray*}
H\left(t\right)=U\star h\left(t\right)\textrm{,  donde }h\left(t\right)=H\left(t\right)-F\star H\left(t\right)
\end{eqnarray*}
\end{Prop}

\begin{Def}
Un proceso estoc\'astico $X\left(t\right)$ es crudamente regenerativo en un tiempo aleatorio positivo $T$ si
\begin{eqnarray*}
\esp\left[X\left(T+t\right)|T\right]=\esp\left[X\left(t\right)\right]\textrm{, para }t\geq0,\end{eqnarray*}
y con las esperanzas anteriores finitas.
\end{Def}

\begin{Prop}
Sup\'ongase que $X\left(t\right)$ es un proceso crudamente regenerativo en $T$, que tiene distribuci\'on $F$. Si $\esp\left[X\left(t\right)\right]$ es acotado en intervalos finitos, entonces
\begin{eqnarray*}
\esp\left[X\left(t\right)\right]=U\star h\left(t\right)\textrm{,  donde }h\left(t\right)=\esp\left[X\left(t\right)\indora\left(T>t\right)\right].
\end{eqnarray*}
\end{Prop}

\begin{Teo}[Regeneraci\'on Cruda]
Sup\'ongase que $X\left(t\right)$ es un proceso con valores positivo crudamente regenerativo en $T$, y def\'inase $M=\sup\left\{|X\left(t\right)|:t\leq T\right\}$. Si $T$ es no aritm\'etico y $M$ y $MT$ tienen media finita, entonces
\begin{eqnarray*}
lim_{t\rightarrow\infty}\esp\left[X\left(t\right)\right]=\frac{1}{\mu}\int_{\rea_{+}}h\left(s\right)ds,
\end{eqnarray*}
donde $h\left(t\right)=\esp\left[X\left(t\right)\indora\left(T>t\right)\right]$.
\end{Teo}

\begin{Def}
Para el proceso $\left\{\left(N\left(t\right),X\left(t\right)\right):t\geq0\right\}$, sus trayectoria muestrales en el intervalo de tiempo $\left[T_{n-1},T_{n}\right)$ est\'an descritas por
\begin{eqnarray*}
\zeta_{n}=\left(\xi_{n},\left\{X\left(T_{n-1}+t\right):0\leq t<\xi_{n}\right\}\right)
\end{eqnarray*}
Este $\zeta_{n}$ es el $n$-\'esimo segmento del proceso. El proceso es regenerativo sobre los tiempos $T_{n}$ si sus segmentos $\zeta_{n}$ son independientes e id\'enticamennte distribuidos.
\end{Def}


\begin{Note}
Si $\tilde{X}\left(t\right)$ con espacio de estados $\tilde{S}$ es regenerativo sobre $T_{n}$, entonces $X\left(t\right)=f\left(\tilde{X}\left(t\right)\right)$ tambi\'en es regenerativo sobre $T_{n}$, para cualquier funci\'on $f:\tilde{S}\rightarrow S$.
\end{Note}

\begin{Note}
Los procesos regenerativos son crudamente regenerativos, pero no al rev\'es.
\end{Note}


\begin{Note}
Un proceso estoc\'astico a tiempo continuo o discreto es regenerativo si existe un proceso de renovaci\'on  tal que los segmentos del proceso entre tiempos de renovaci\'on sucesivos son i.i.d., es decir, para $\left\{X\left(t\right):t\geq0\right\}$ proceso estoc\'astico a tiempo continuo con espacio de estados $S$, espacio m\'etrico.
\end{Note}

Para $\left\{X\left(t\right):t\geq0\right\}$ Proceso Estoc\'astico a tiempo continuo con estado de espacios $S$, que es un espacio m\'etrico, con trayectorias continuas por la derecha y con l\'imites por la izquierda c.s. Sea $N\left(t\right)$ un proceso de renovaci\'on en $\rea_{+}$ definido en el mismo espacio de probabilidad que $X\left(t\right)$, con tiempos de renovaci\'on $T$ y tiempos de inter-renovaci\'on $\xi_{n}=T_{n}-T_{n-1}$, con misma distribuci\'on $F$ de media finita $\mu$.



\begin{Def}
Para el proceso $\left\{\left(N\left(t\right),X\left(t\right)\right):t\geq0\right\}$, sus trayectoria muestrales en el intervalo de tiempo $\left[T_{n-1},T_{n}\right)$ est\'an descritas por
\begin{eqnarray*}
\zeta_{n}=\left(\xi_{n},\left\{X\left(T_{n-1}+t\right):0\leq t<\xi_{n}\right\}\right)
\end{eqnarray*}
Este $\zeta_{n}$ es el $n$-\'esimo segmento del proceso. El proceso es regenerativo sobre los tiempos $T_{n}$ si sus segmentos $\zeta_{n}$ son independientes e id\'enticamennte distribuidos.
\end{Def}

\begin{Note}
Un proceso regenerativo con media de la longitud de ciclo finita es llamado positivo recurrente.
\end{Note}

\begin{Teo}[Procesos Regenerativos]
Suponga que el proceso
\end{Teo}


\begin{Def}[Renewal Process Trinity]
Para un proceso de renovaci\'on $N\left(t\right)$, los siguientes procesos proveen de informaci\'on sobre los tiempos de renovaci\'on.
\begin{itemize}
\item $A\left(t\right)=t-T_{N\left(t\right)}$, el tiempo de recurrencia hacia atr\'as al tiempo $t$, que es el tiempo desde la \'ultima renovaci\'on para $t$.

\item $B\left(t\right)=T_{N\left(t\right)+1}-t$, el tiempo de recurrencia hacia adelante al tiempo $t$, residual del tiempo de renovaci\'on, que es el tiempo para la pr\'oxima renovaci\'on despu\'es de $t$.

\item $L\left(t\right)=\xi_{N\left(t\right)+1}=A\left(t\right)+B\left(t\right)$, la longitud del intervalo de renovaci\'on que contiene a $t$.
\end{itemize}
\end{Def}

\begin{Note}
El proceso tridimensional $\left(A\left(t\right),B\left(t\right),L\left(t\right)\right)$ es regenerativo sobre $T_{n}$, y por ende cada proceso lo es. Cada proceso $A\left(t\right)$ y $B\left(t\right)$ son procesos de MArkov a tiempo continuo con trayectorias continuas por partes en el espacio de estados $\rea_{+}$. Una expresi\'on conveniente para su distribuci\'on conjunta es, para $0\leq x<t,y\geq0$
\begin{equation}\label{NoRenovacion}
P\left\{A\left(t\right)>x,B\left(t\right)>y\right\}=
P\left\{N\left(t+y\right)-N\left((t-x)\right)=0\right\}
\end{equation}
\end{Note}

\begin{Ejem}[Tiempos de recurrencia Poisson]
Si $N\left(t\right)$ es un proceso Poisson con tasa $\lambda$, entonces de la expresi\'on (\ref{NoRenovacion}) se tiene que

\begin{eqnarray*}
\begin{array}{lc}
P\left\{A\left(t\right)>x,B\left(t\right)>y\right\}=e^{-\lambda\left(x+y\right)},&0\leq x<t,y\geq0,
\end{array}
\end{eqnarray*}
que es la probabilidad Poisson de no renovaciones en un intervalo de longitud $x+y$.

\end{Ejem}

\begin{Note}
Una cadena de Markov erg\'odica tiene la propiedad de ser estacionaria si la distribuci\'on de su estado al tiempo $0$ es su distribuci\'on estacionaria.
\end{Note}


\begin{Def}
Un proceso estoc\'astico a tiempo continuo $\left\{X\left(t\right):t\geq0\right\}$ en un espacio general es estacionario si sus distribuciones finito dimensionales son invariantes bajo cualquier  traslado: para cada $0\leq s_{1}<s_{2}<\cdots<s_{k}$ y $t\geq0$,
\begin{eqnarray*}
\left(X\left(s_{1}+t\right),\ldots,X\left(s_{k}+t\right)\right)=_{d}\left(X\left(s_{1}\right),\ldots,X\left(s_{k}\right)\right).
\end{eqnarray*}
\end{Def}

\begin{Note}
Un proceso de Markov es estacionario si $X\left(t\right)=_{d}X\left(0\right)$, $t\geq0$.
\end{Note}

Considerese el proceso $N\left(t\right)=\sum_{n}\indora\left(\tau_{n}\leq t\right)$ en $\rea_{+}$, con puntos $0<\tau_{1}<\tau_{2}<\cdots$.

\begin{Prop}
Si $N$ es un proceso puntual estacionario y $\esp\left[N\left(1\right)\right]<\infty$, entonces $\esp\left[N\left(t\right)\right]=t\esp\left[N\left(1\right)\right]$, $t\geq0$

\end{Prop}

\begin{Teo}
Los siguientes enunciados son equivalentes
\begin{itemize}
\item[i)] El proceso retardado de renovaci\'on $N$ es estacionario.

\item[ii)] EL proceso de tiempos de recurrencia hacia adelante $B\left(t\right)$ es estacionario.


\item[iii)] $\esp\left[N\left(t\right)\right]=t/\mu$,


\item[iv)] $G\left(t\right)=F_{e}\left(t\right)=\frac{1}{\mu}\int_{0}^{t}\left[1-F\left(s\right)\right]ds$
\end{itemize}
Cuando estos enunciados son ciertos, $P\left\{B\left(t\right)\leq x\right\}=F_{e}\left(x\right)$, para $t,x\geq0$.

\end{Teo}

\begin{Note}
Una consecuencia del teorema anterior es que el Proceso Poisson es el \'unico proceso sin retardo que es estacionario.
\end{Note}

\begin{Coro}
El proceso de renovaci\'on $N\left(t\right)$ sin retardo, y cuyos tiempos de inter renonaci\'on tienen media finita, es estacionario si y s\'olo si es un proceso Poisson.

\end{Coro}


%________________________________________________________________________
%\subsection{Procesos Regenerativos}
%________________________________________________________________________



\begin{Note}
Si $\tilde{X}\left(t\right)$ con espacio de estados $\tilde{S}$ es regenerativo sobre $T_{n}$, entonces $X\left(t\right)=f\left(\tilde{X}\left(t\right)\right)$ tambi\'en es regenerativo sobre $T_{n}$, para cualquier funci\'on $f:\tilde{S}\rightarrow S$.
\end{Note}

\begin{Note}
Los procesos regenerativos son crudamente regenerativos, pero no al rev\'es.
\end{Note}
%\subsection*{Procesos Regenerativos: Sigman\cite{Sigman1}}
\begin{Def}[Definici\'on Cl\'asica]
Un proceso estoc\'astico $X=\left\{X\left(t\right):t\geq0\right\}$ es llamado regenerativo is existe una variable aleatoria $R_{1}>0$ tal que
\begin{itemize}
\item[i)] $\left\{X\left(t+R_{1}\right):t\geq0\right\}$ es independiente de $\left\{\left\{X\left(t\right):t<R_{1}\right\},\right\}$
\item[ii)] $\left\{X\left(t+R_{1}\right):t\geq0\right\}$ es estoc\'asticamente equivalente a $\left\{X\left(t\right):t>0\right\}$
\end{itemize}

Llamamos a $R_{1}$ tiempo de regeneraci\'on, y decimos que $X$ se regenera en este punto.
\end{Def}

$\left\{X\left(t+R_{1}\right)\right\}$ es regenerativo con tiempo de regeneraci\'on $R_{2}$, independiente de $R_{1}$ pero con la misma distribuci\'on que $R_{1}$. Procediendo de esta manera se obtiene una secuencia de variables aleatorias independientes e id\'enticamente distribuidas $\left\{R_{n}\right\}$ llamados longitudes de ciclo. Si definimos a $Z_{k}\equiv R_{1}+R_{2}+\cdots+R_{k}$, se tiene un proceso de renovaci\'on llamado proceso de renovaci\'on encajado para $X$.




\begin{Def}
Para $x$ fijo y para cada $t\geq0$, sea $I_{x}\left(t\right)=1$ si $X\left(t\right)\leq x$,  $I_{x}\left(t\right)=0$ en caso contrario, y def\'inanse los tiempos promedio
\begin{eqnarray*}
\overline{X}&=&lim_{t\rightarrow\infty}\frac{1}{t}\int_{0}^{\infty}X\left(u\right)du\\
\prob\left(X_{\infty}\leq x\right)&=&lim_{t\rightarrow\infty}\frac{1}{t}\int_{0}^{\infty}I_{x}\left(u\right)du,
\end{eqnarray*}
cuando estos l\'imites existan.
\end{Def}

Como consecuencia del teorema de Renovaci\'on-Recompensa, se tiene que el primer l\'imite  existe y es igual a la constante
\begin{eqnarray*}
\overline{X}&=&\frac{\esp\left[\int_{0}^{R_{1}}X\left(t\right)dt\right]}{\esp\left[R_{1}\right]},
\end{eqnarray*}
suponiendo que ambas esperanzas son finitas.

\begin{Note}
\begin{itemize}
\item[a)] Si el proceso regenerativo $X$ es positivo recurrente y tiene trayectorias muestrales no negativas, entonces la ecuaci\'on anterior es v\'alida.
\item[b)] Si $X$ es positivo recurrente regenerativo, podemos construir una \'unica versi\'on estacionaria de este proceso, $X_{e}=\left\{X_{e}\left(t\right)\right\}$, donde $X_{e}$ es un proceso estoc\'astico regenerativo y estrictamente estacionario, con distribuci\'on marginal distribuida como $X_{\infty}$
\end{itemize}
\end{Note}

%________________________________________________________________________
%\subsection{Procesos Regenerativos}
%________________________________________________________________________

Para $\left\{X\left(t\right):t\geq0\right\}$ Proceso Estoc\'astico a tiempo continuo con estado de espacios $S$, que es un espacio m\'etrico, con trayectorias continuas por la derecha y con l\'imites por la izquierda c.s. Sea $N\left(t\right)$ un proceso de renovaci\'on en $\rea_{+}$ definido en el mismo espacio de probabilidad que $X\left(t\right)$, con tiempos de renovaci\'on $T$ y tiempos de inter-renovaci\'on $\xi_{n}=T_{n}-T_{n-1}$, con misma distribuci\'on $F$ de media finita $\mu$.



\begin{Def}
Para el proceso $\left\{\left(N\left(t\right),X\left(t\right)\right):t\geq0\right\}$, sus trayectoria muestrales en el intervalo de tiempo $\left[T_{n-1},T_{n}\right)$ est\'an descritas por
\begin{eqnarray*}
\zeta_{n}=\left(\xi_{n},\left\{X\left(T_{n-1}+t\right):0\leq t<\xi_{n}\right\}\right)
\end{eqnarray*}
Este $\zeta_{n}$ es el $n$-\'esimo segmento del proceso. El proceso es regenerativo sobre los tiempos $T_{n}$ si sus segmentos $\zeta_{n}$ son independientes e id\'enticamennte distribuidos.
\end{Def}


\begin{Note}
Si $\tilde{X}\left(t\right)$ con espacio de estados $\tilde{S}$ es regenerativo sobre $T_{n}$, entonces $X\left(t\right)=f\left(\tilde{X}\left(t\right)\right)$ tambi\'en es regenerativo sobre $T_{n}$, para cualquier funci\'on $f:\tilde{S}\rightarrow S$.
\end{Note}

\begin{Note}
Los procesos regenerativos son crudamente regenerativos, pero no al rev\'es.
\end{Note}

\begin{Def}[Definici\'on Cl\'asica]
Un proceso estoc\'astico $X=\left\{X\left(t\right):t\geq0\right\}$ es llamado regenerativo is existe una variable aleatoria $R_{1}>0$ tal que
\begin{itemize}
\item[i)] $\left\{X\left(t+R_{1}\right):t\geq0\right\}$ es independiente de $\left\{\left\{X\left(t\right):t<R_{1}\right\},\right\}$
\item[ii)] $\left\{X\left(t+R_{1}\right):t\geq0\right\}$ es estoc\'asticamente equivalente a $\left\{X\left(t\right):t>0\right\}$
\end{itemize}

Llamamos a $R_{1}$ tiempo de regeneraci\'on, y decimos que $X$ se regenera en este punto.
\end{Def}

$\left\{X\left(t+R_{1}\right)\right\}$ es regenerativo con tiempo de regeneraci\'on $R_{2}$, independiente de $R_{1}$ pero con la misma distribuci\'on que $R_{1}$. Procediendo de esta manera se obtiene una secuencia de variables aleatorias independientes e id\'enticamente distribuidas $\left\{R_{n}\right\}$ llamados longitudes de ciclo. Si definimos a $Z_{k}\equiv R_{1}+R_{2}+\cdots+R_{k}$, se tiene un proceso de renovaci\'on llamado proceso de renovaci\'on encajado para $X$.

\begin{Note}
Un proceso regenerativo con media de la longitud de ciclo finita es llamado positivo recurrente.
\end{Note}


\begin{Def}
Para $x$ fijo y para cada $t\geq0$, sea $I_{x}\left(t\right)=1$ si $X\left(t\right)\leq x$,  $I_{x}\left(t\right)=0$ en caso contrario, y def\'inanse los tiempos promedio
\begin{eqnarray*}
\overline{X}&=&lim_{t\rightarrow\infty}\frac{1}{t}\int_{0}^{\infty}X\left(u\right)du\\
\prob\left(X_{\infty}\leq x\right)&=&lim_{t\rightarrow\infty}\frac{1}{t}\int_{0}^{\infty}I_{x}\left(u\right)du,
\end{eqnarray*}
cuando estos l\'imites existan.
\end{Def}

Como consecuencia del teorema de Renovaci\'on-Recompensa, se tiene que el primer l\'imite  existe y es igual a la constante
\begin{eqnarray*}
\overline{X}&=&\frac{\esp\left[\int_{0}^{R_{1}}X\left(t\right)dt\right]}{\esp\left[R_{1}\right]},
\end{eqnarray*}
suponiendo que ambas esperanzas son finitas.

\begin{Note}
\begin{itemize}
\item[a)] Si el proceso regenerativo $X$ es positivo recurrente y tiene trayectorias muestrales no negativas, entonces la ecuaci\'on anterior es v\'alida.
\item[b)] Si $X$ es positivo recurrente regenerativo, podemos construir una \'unica versi\'on estacionaria de este proceso, $X_{e}=\left\{X_{e}\left(t\right)\right\}$, donde $X_{e}$ es un proceso estoc\'astico regenerativo y estrictamente estacionario, con distribuci\'on marginal distribuida como $X_{\infty}$
\end{itemize}
\end{Note}

%__________________________________________________________________________________________
%\subsection{Procesos Regenerativos Estacionarios - Stidham \cite{Stidham}}
%__________________________________________________________________________________________


Un proceso estoc\'astico a tiempo continuo $\left\{V\left(t\right),t\geq0\right\}$ es un proceso regenerativo si existe una sucesi\'on de variables aleatorias independientes e id\'enticamente distribuidas $\left\{X_{1},X_{2},\ldots\right\}$, sucesi\'on de renovaci\'on, tal que para cualquier conjunto de Borel $A$, 

\begin{eqnarray*}
\prob\left\{V\left(t\right)\in A|X_{1}+X_{2}+\cdots+X_{R\left(t\right)}=s,\left\{V\left(\tau\right),\tau<s\right\}\right\}=\prob\left\{V\left(t-s\right)\in A|X_{1}>t-s\right\},
\end{eqnarray*}
para todo $0\leq s\leq t$, donde $R\left(t\right)=\max\left\{X_{1}+X_{2}+\cdots+X_{j}\leq t\right\}=$n\'umero de renovaciones ({\emph{puntos de regeneraci\'on}}) que ocurren en $\left[0,t\right]$. El intervalo $\left[0,X_{1}\right)$ es llamado {\emph{primer ciclo de regeneraci\'on}} de $\left\{V\left(t \right),t\geq0\right\}$, $\left[X_{1},X_{1}+X_{2}\right)$ el {\emph{segundo ciclo de regeneraci\'on}}, y as\'i sucesivamente.

Sea $X=X_{1}$ y sea $F$ la funci\'on de distrbuci\'on de $X$


\begin{Def}
Se define el proceso estacionario, $\left\{V^{*}\left(t\right),t\geq0\right\}$, para $\left\{V\left(t\right),t\geq0\right\}$ por

\begin{eqnarray*}
\prob\left\{V\left(t\right)\in A\right\}=\frac{1}{\esp\left[X\right]}\int_{0}^{\infty}\prob\left\{V\left(t+x\right)\in A|X>x\right\}\left(1-F\left(x\right)\right)dx,
\end{eqnarray*} 
para todo $t\geq0$ y todo conjunto de Borel $A$.
\end{Def}

\begin{Def}
Una distribuci\'on se dice que es {\emph{aritm\'etica}} si todos sus puntos de incremento son m\'ultiplos de la forma $0,\lambda, 2\lambda,\ldots$ para alguna $\lambda>0$ entera.
\end{Def}


\begin{Def}
Una modificaci\'on medible de un proceso $\left\{V\left(t\right),t\geq0\right\}$, es una versi\'on de este, $\left\{V\left(t,w\right)\right\}$ conjuntamente medible para $t\geq0$ y para $w\in S$, $S$ espacio de estados para $\left\{V\left(t\right),t\geq0\right\}$.
\end{Def}

\begin{Teo}
Sea $\left\{V\left(t\right),t\geq\right\}$ un proceso regenerativo no negativo con modificaci\'on medible. Sea $\esp\left[X\right]<\infty$. Entonces el proceso estacionario dado por la ecuaci\'on anterior est\'a bien definido y tiene funci\'on de distribuci\'on independiente de $t$, adem\'as
\begin{itemize}
\item[i)] \begin{eqnarray*}
\esp\left[V^{*}\left(0\right)\right]&=&\frac{\esp\left[\int_{0}^{X}V\left(s\right)ds\right]}{\esp\left[X\right]}\end{eqnarray*}
\item[ii)] Si $\esp\left[V^{*}\left(0\right)\right]<\infty$, equivalentemente, si $\esp\left[\int_{0}^{X}V\left(s\right)ds\right]<\infty$,entonces
\begin{eqnarray*}
\frac{\int_{0}^{t}V\left(s\right)ds}{t}\rightarrow\frac{\esp\left[\int_{0}^{X}V\left(s\right)ds\right]}{\esp\left[X\right]}
\end{eqnarray*}
con probabilidad 1 y en media, cuando $t\rightarrow\infty$.
\end{itemize}
\end{Teo}
%
%___________________________________________________________________________________________
%\vspace{5.5cm}
%\chapter{Cadenas de Markov estacionarias}
%\vspace{-1.0cm}


%__________________________________________________________________________________________
%\subsection{Procesos Regenerativos Estacionarios - Stidham \cite{Stidham}}
%__________________________________________________________________________________________


Un proceso estoc\'astico a tiempo continuo $\left\{V\left(t\right),t\geq0\right\}$ es un proceso regenerativo si existe una sucesi\'on de variables aleatorias independientes e id\'enticamente distribuidas $\left\{X_{1},X_{2},\ldots\right\}$, sucesi\'on de renovaci\'on, tal que para cualquier conjunto de Borel $A$, 

\begin{eqnarray*}
\prob\left\{V\left(t\right)\in A|X_{1}+X_{2}+\cdots+X_{R\left(t\right)}=s,\left\{V\left(\tau\right),\tau<s\right\}\right\}=\prob\left\{V\left(t-s\right)\in A|X_{1}>t-s\right\},
\end{eqnarray*}
para todo $0\leq s\leq t$, donde $R\left(t\right)=\max\left\{X_{1}+X_{2}+\cdots+X_{j}\leq t\right\}=$n\'umero de renovaciones ({\emph{puntos de regeneraci\'on}}) que ocurren en $\left[0,t\right]$. El intervalo $\left[0,X_{1}\right)$ es llamado {\emph{primer ciclo de regeneraci\'on}} de $\left\{V\left(t \right),t\geq0\right\}$, $\left[X_{1},X_{1}+X_{2}\right)$ el {\emph{segundo ciclo de regeneraci\'on}}, y as\'i sucesivamente.

Sea $X=X_{1}$ y sea $F$ la funci\'on de distrbuci\'on de $X$


\begin{Def}
Se define el proceso estacionario, $\left\{V^{*}\left(t\right),t\geq0\right\}$, para $\left\{V\left(t\right),t\geq0\right\}$ por

\begin{eqnarray*}
\prob\left\{V\left(t\right)\in A\right\}=\frac{1}{\esp\left[X\right]}\int_{0}^{\infty}\prob\left\{V\left(t+x\right)\in A|X>x\right\}\left(1-F\left(x\right)\right)dx,
\end{eqnarray*} 
para todo $t\geq0$ y todo conjunto de Borel $A$.
\end{Def}

\begin{Def}
Una distribuci\'on se dice que es {\emph{aritm\'etica}} si todos sus puntos de incremento son m\'ultiplos de la forma $0,\lambda, 2\lambda,\ldots$ para alguna $\lambda>0$ entera.
\end{Def}


\begin{Def}
Una modificaci\'on medible de un proceso $\left\{V\left(t\right),t\geq0\right\}$, es una versi\'on de este, $\left\{V\left(t,w\right)\right\}$ conjuntamente medible para $t\geq0$ y para $w\in S$, $S$ espacio de estados para $\left\{V\left(t\right),t\geq0\right\}$.
\end{Def}

\begin{Teo}
Sea $\left\{V\left(t\right),t\geq\right\}$ un proceso regenerativo no negativo con modificaci\'on medible. Sea $\esp\left[X\right]<\infty$. Entonces el proceso estacionario dado por la ecuaci\'on anterior est\'a bien definido y tiene funci\'on de distribuci\'on independiente de $t$, adem\'as
\begin{itemize}
\item[i)] \begin{eqnarray*}
\esp\left[V^{*}\left(0\right)\right]&=&\frac{\esp\left[\int_{0}^{X}V\left(s\right)ds\right]}{\esp\left[X\right]}\end{eqnarray*}
\item[ii)] Si $\esp\left[V^{*}\left(0\right)\right]<\infty$, equivalentemente, si $\esp\left[\int_{0}^{X}V\left(s\right)ds\right]<\infty$,entonces
\begin{eqnarray*}
\frac{\int_{0}^{t}V\left(s\right)ds}{t}\rightarrow\frac{\esp\left[\int_{0}^{X}V\left(s\right)ds\right]}{\esp\left[X\right]}
\end{eqnarray*}
con probabilidad 1 y en media, cuando $t\rightarrow\infty$.
\end{itemize}
\end{Teo}

Para $\left\{X\left(t\right):t\geq0\right\}$ Proceso Estoc\'astico a tiempo continuo con estado de espacios $S$, que es un espacio m\'etrico, con trayectorias continuas por la derecha y con l\'imites por la izquierda c.s. Sea $N\left(t\right)$ un proceso de renovaci\'on en $\rea_{+}$ definido en el mismo espacio de probabilidad que $X\left(t\right)$, con tiempos de renovaci\'on $T$ y tiempos de inter-renovaci\'on $\xi_{n}=T_{n}-T_{n-1}$, con misma distribuci\'on $F$ de media finita $\mu$.


%______________________________________________________________________
%\subsection{Ejemplos, Notas importantes}


Sean $T_{1},T_{2},\ldots$ los puntos donde las longitudes de las colas de la red de sistemas de visitas c\'iclicas son cero simult\'aneamente, cuando la cola $Q_{j}$ es visitada por el servidor para dar servicio, es decir, $L_{1}\left(T_{i}\right)=0,L_{2}\left(T_{i}\right)=0,\hat{L}_{1}\left(T_{i}\right)=0$ y $\hat{L}_{2}\left(T_{i}\right)=0$, a estos puntos se les denominar\'a puntos regenerativos. Sea la funci\'on generadora de momentos para $L_{i}$, el n\'umero de usuarios en la cola $Q_{i}\left(z\right)$ en cualquier momento, est\'a dada por el tiempo promedio de $z^{L_{i}\left(t\right)}$ sobre el ciclo regenerativo definido anteriormente:

\begin{eqnarray*}
Q_{i}\left(z\right)&=&\esp\left[z^{L_{i}\left(t\right)}\right]=\frac{\esp\left[\sum_{m=1}^{M_{i}}\sum_{t=\tau_{i}\left(m\right)}^{\tau_{i}\left(m+1\right)-1}z^{L_{i}\left(t\right)}\right]}{\esp\left[\sum_{m=1}^{M_{i}}\tau_{i}\left(m+1\right)-\tau_{i}\left(m\right)\right]}
\end{eqnarray*}

$M_{i}$ es un tiempo de paro en el proceso regenerativo con $\esp\left[M_{i}\right]<\infty$\footnote{En Stidham\cite{Stidham} y Heyman  se muestra que una condici\'on suficiente para que el proceso regenerativo 
estacionario sea un procesoo estacionario es que el valor esperado del tiempo del ciclo regenerativo sea finito, es decir: $\esp\left[\sum_{m=1}^{M_{i}}C_{i}^{(m)}\right]<\infty$, como cada $C_{i}^{(m)}$ contiene intervalos de r\'eplica positivos, se tiene que $\esp\left[M_{i}\right]<\infty$, adem\'as, como $M_{i}>0$, se tiene que la condici\'on anterior es equivalente a tener que $\esp\left[C_{i}\right]<\infty$,
por lo tanto una condici\'on suficiente para la existencia del proceso regenerativo est\'a dada por $\sum_{k=1}^{N}\mu_{k}<1.$}, se sigue del lema de Wald que:


\begin{eqnarray*}
\esp\left[\sum_{m=1}^{M_{i}}\sum_{t=\tau_{i}\left(m\right)}^{\tau_{i}\left(m+1\right)-1}z^{L_{i}\left(t\right)}\right]&=&\esp\left[M_{i}\right]\esp\left[\sum_{t=\tau_{i}\left(m\right)}^{\tau_{i}\left(m+1\right)-1}z^{L_{i}\left(t\right)}\right]\\
\esp\left[\sum_{m=1}^{M_{i}}\tau_{i}\left(m+1\right)-\tau_{i}\left(m\right)\right]&=&\esp\left[M_{i}\right]\esp\left[\tau_{i}\left(m+1\right)-\tau_{i}\left(m\right)\right]
\end{eqnarray*}

por tanto se tiene que


\begin{eqnarray*}
Q_{i}\left(z\right)&=&\frac{\esp\left[\sum_{t=\tau_{i}\left(m\right)}^{\tau_{i}\left(m+1\right)-1}z^{L_{i}\left(t\right)}\right]}{\esp\left[\tau_{i}\left(m+1\right)-\tau_{i}\left(m\right)\right]}
\end{eqnarray*}

observar que el denominador es simplemente la duraci\'on promedio del tiempo del ciclo.


Haciendo las siguientes sustituciones en la ecuaci\'on (\ref{Corolario2}): $n\rightarrow t-\tau_{i}\left(m\right)$, $T \rightarrow \overline{\tau}_{i}\left(m\right)-\tau_{i}\left(m\right)$, $L_{n}\rightarrow L_{i}\left(t\right)$ y $F\left(z\right)=\esp\left[z^{L_{0}}\right]\rightarrow F_{i}\left(z\right)=\esp\left[z^{L_{i}\tau_{i}\left(m\right)}\right]$, se puede ver que

\begin{eqnarray}\label{Eq.Arribos.Primera}
\esp\left[\sum_{n=0}^{T-1}z^{L_{n}}\right]=
\esp\left[\sum_{t=\tau_{i}\left(m\right)}^{\overline{\tau}_{i}\left(m\right)-1}z^{L_{i}\left(t\right)}\right]
=z\frac{F_{i}\left(z\right)-1}{z-P_{i}\left(z\right)}
\end{eqnarray}

Por otra parte durante el tiempo de intervisita para la cola $i$, $L_{i}\left(t\right)$ solamente se incrementa de manera que el incremento por intervalo de tiempo est\'a dado por la funci\'on generadora de probabilidades de $P_{i}\left(z\right)$, por tanto la suma sobre el tiempo de intervisita puede evaluarse como:

\begin{eqnarray*}
\esp\left[\sum_{t=\tau_{i}\left(m\right)}^{\tau_{i}\left(m+1\right)-1}z^{L_{i}\left(t\right)}\right]&=&\esp\left[\sum_{t=\tau_{i}\left(m\right)}^{\tau_{i}\left(m+1\right)-1}\left\{P_{i}\left(z\right)\right\}^{t-\overline{\tau}_{i}\left(m\right)}\right]=\frac{1-\esp\left[\left\{P_{i}\left(z\right)\right\}^{\tau_{i}\left(m+1\right)-\overline{\tau}_{i}\left(m\right)}\right]}{1-P_{i}\left(z\right)}\\
&=&\frac{1-I_{i}\left[P_{i}\left(z\right)\right]}{1-P_{i}\left(z\right)}
\end{eqnarray*}
por tanto

\begin{eqnarray*}
\esp\left[\sum_{t=\tau_{i}\left(m\right)}^{\tau_{i}\left(m+1\right)-1}z^{L_{i}\left(t\right)}\right]&=&
\frac{1-F_{i}\left(z\right)}{1-P_{i}\left(z\right)}
\end{eqnarray*}

Por lo tanto

\begin{eqnarray*}
Q_{i}\left(z\right)&=&\frac{\esp\left[\sum_{t=\tau_{i}\left(m\right)}^{\tau_{i}
\left(m+1\right)-1}z^{L_{i}\left(t\right)}\right]}{\esp\left[\tau_{i}\left(m+1\right)-\tau_{i}\left(m\right)\right]}\\
&=&\frac{1}{\esp\left[\tau_{i}\left(m+1\right)-\tau_{i}\left(m\right)\right]}
\left\{
\esp\left[\sum_{t=\tau_{i}\left(m\right)}^{\overline{\tau}_{i}\left(m\right)-1}
z^{L_{i}\left(t\right)}\right]
+\esp\left[\sum_{t=\overline{\tau}_{i}\left(m\right)}^{\tau_{i}\left(m+1\right)-1}
z^{L_{i}\left(t\right)}\right]\right\}\\
&=&\frac{1}{\esp\left[\tau_{i}\left(m+1\right)-\tau_{i}\left(m\right)\right]}
\left\{
z\frac{F_{i}\left(z\right)-1}{z-P_{i}\left(z\right)}+\frac{1-F_{i}\left(z\right)}
{1-P_{i}\left(z\right)}
\right\}
\end{eqnarray*}

es decir

\begin{equation}
Q_{i}\left(z\right)=\frac{1}{\esp\left[C_{i}\right]}\cdot\frac{1-F_{i}\left(z\right)}{P_{i}\left(z\right)-z}\cdot\frac{\left(1-z\right)P_{i}\left(z\right)}{1-P_{i}\left(z\right)}
\end{equation}

\begin{Teo}
Dada una Red de Sistemas de Visitas C\'iclicas (RSVC), conformada por dos Sistemas de Visitas C\'iclicas (SVC), donde cada uno de ellos consta de dos colas tipo $M/M/1$. Los dos sistemas est\'an comunicados entre s\'i por medio de la transferencia de usuarios entre las colas $Q_{1}\leftrightarrow Q_{3}$ y $Q_{2}\leftrightarrow Q_{4}$. Se definen los eventos para los procesos de arribos al tiempo $t$, $A_{j}\left(t\right)=\left\{0 \textrm{ arribos en }Q_{j}\textrm{ al tiempo }t\right\}$ para alg\'un tiempo $t\geq0$ y $Q_{j}$ la cola $j$-\'esima en la RSVC, para $j=1,2,3,4$.  Existe un intervalo $I\neq\emptyset$ tal que para $T^{*}\in I$, tal que $\prob\left\{A_{1}\left(T^{*}\right),A_{2}\left(Tt^{*}\right),
A_{3}\left(T^{*}\right),A_{4}\left(T^{*}\right)|T^{*}\in I\right\}>0$.
\end{Teo}

\begin{proof}
Sin p\'erdida de generalidad podemos considerar como base del an\'alisis a la cola $Q_{1}$ del primer sistema que conforma la RSVC.

Sea $n>0$, ciclo en el primer sistema en el que se sabe que $L_{j}\left(\overline{\tau}_{1}\left(n\right)\right)=0$, pues la pol\'itica de servicio con que atienden los servidores es la exhaustiva. Como es sabido, para trasladarse a la siguiente cola, el servidor incurre en un tiempo de traslado $r_{1}\left(n\right)>0$, entonces tenemos que $\tau_{2}\left(n\right)=\overline{\tau}_{1}\left(n\right)+r_{1}\left(n\right)$.


Definamos el intervalo $I_{1}\equiv\left[\overline{\tau}_{1}\left(n\right),\tau_{2}\left(n\right)\right]$ de longitud $\xi_{1}=r_{1}\left(n\right)$. Dado que los tiempos entre arribo son exponenciales con tasa $\tilde{\mu}_{1}=\mu_{1}+\hat{\mu}_{1}$ ($\mu_{1}$ son los arribos a $Q_{1}$ por primera vez al sistema, mientras que $\hat{\mu}_{1}$ son los arribos de traslado procedentes de $Q_{3}$) se tiene que la probabilidad del evento $A_{1}\left(t\right)$ est\'a dada por 

\begin{equation}
\prob\left\{A_{1}\left(t\right)|t\in I_{1}\left(n\right)\right\}=e^{-\tilde{\mu}_{1}\xi_{1}\left(n\right)}.
\end{equation} 

Por otra parte, para la cola $Q_{2}$, el tiempo $\overline{\tau}_{2}\left(n-1\right)$ es tal que $L_{2}\left(\overline{\tau}_{2}\left(n-1\right)\right)=0$, es decir, es el tiempo en que la cola queda totalmente vac\'ia en el ciclo anterior a $n$. Entonces tenemos un sgundo intervalo $I_{2}\equiv\left[\overline{\tau}_{2}\left(n-1\right),\tau_{2}\left(n\right)\right]$. Por lo tanto la probabilidad del evento $A_{2}\left(t\right)$ tiene probabilidad dada por

\begin{equation}
\prob\left\{A_{2}\left(t\right)|t\in I_{2}\left(n\right)\right\}=e^{-\tilde{\mu}_{2}\xi_{2}\left(n\right)},
\end{equation} 

donde $\xi_{2}\left(n\right)=\tau_{2}\left(n\right)-\overline{\tau}_{2}\left(n-1\right)$.



Entonces, se tiene que

\begin{eqnarray*}
\prob\left\{A_{1}\left(t\right),A_{2}\left(t\right)|t\in I_{1}\left(n\right)\right\}&=&
\prob\left\{A_{1}\left(t\right)|t\in I_{1}\left(n\right)\right\}
\prob\left\{A_{2}\left(t\right)|t\in I_{1}\left(n\right)\right\}\\
&\geq&
\prob\left\{A_{1}\left(t\right)|t\in I_{1}\left(n\right)\right\}
\prob\left\{A_{2}\left(t\right)|t\in I_{2}\left(n\right)\right\}\\
&=&e^{-\tilde{\mu}_{1}\xi_{1}\left(n\right)}e^{-\tilde{\mu}_{2}\xi_{2}\left(n\right)}
=e^{-\left[\tilde{\mu}_{1}\xi_{1}\left(n\right)+\tilde{\mu}_{2}\xi_{2}\left(n\right)\right]}.
\end{eqnarray*}


es decir, 

\begin{equation}
\prob\left\{A_{1}\left(t\right),A_{2}\left(t\right)|t\in I_{1}\left(n\right)\right\}
=e^{-\left[\tilde{\mu}_{1}\xi_{1}\left(n\right)+\tilde{\mu}_{2}\xi_{2}
\left(n\right)\right]}>0.
\end{equation}

En lo que respecta a la relaci\'on entre los dos SVC que conforman la RSVC, se afirma que existe $m>0$ tal que $\overline{\tau}_{3}\left(m\right)<\tau_{2}\left(n\right)<\tau_{4}\left(m\right)$.

Para $Q_{3}$ sea $I_{3}=\left[\overline{\tau}_{3}\left(m\right),\tau_{4}\left(m\right)\right]$ con longitud  $\xi_{3}\left(m\right)=r_{3}\left(m\right)$, entonces 

\begin{equation}
\prob\left\{A_{3}\left(t\right)|t\in I_{3}\left(n\right)\right\}=e^{-\tilde{\mu}_{3}\xi_{3}\left(n\right)}.
\end{equation} 

An\'alogamente que como se hizo para $Q_{2}$, tenemos que para $Q_{4}$ se tiene el intervalo $I_{4}=\left[\overline{\tau}_{4}\left(m-1\right),\tau_{4}\left(m\right)\right]$ con longitud $\xi_{4}\left(m\right)=\tau_{4}\left(m\right)-\overline{\tau}_{4}\left(m-1\right)$, entonces


\begin{equation}
\prob\left\{A_{4}\left(t\right)|t\in I_{4}\left(m\right)\right\}=e^{-\tilde{\mu}_{4}\xi_{4}\left(n\right)}.
\end{equation} 

Al igual que para el primer sistema, dado que $I_{3}\left(m\right)\subset I_{4}\left(m\right)$, se tiene que

\begin{eqnarray*}
\xi_{3}\left(m\right)\leq\xi_{4}\left(m\right)&\Leftrightarrow& -\xi_{3}\left(m\right)\geq-\xi_{4}\left(m\right)
\\
-\tilde{\mu}_{4}\xi_{3}\left(m\right)\geq-\tilde{\mu}_{4}\xi_{4}\left(m\right)&\Leftrightarrow&
e^{-\tilde{\mu}_{4}\xi_{3}\left(m\right)}\geq e^{-\tilde{\mu}_{4}\xi_{4}\left(m\right)}\\
\prob\left\{A_{4}\left(t\right)|t\in I_{3}\left(m\right)\right\}&\geq&
\prob\left\{A_{4}\left(t\right)|t\in I_{4}\left(m\right)\right\}
\end{eqnarray*}

Entonces, dado que los eventos $A_{3}$ y $A_{4}$ son independientes, se tiene que

\begin{eqnarray*}
\prob\left\{A_{3}\left(t\right),A_{4}\left(t\right)|t\in I_{3}\left(m\right)\right\}&=&
\prob\left\{A_{3}\left(t\right)|t\in I_{3}\left(m\right)\right\}
\prob\left\{A_{4}\left(t\right)|t\in I_{3}\left(m\right)\right\}\\
&\geq&
\prob\left\{A_{3}\left(t\right)|t\in I_{3}\left(n\right)\right\}
\prob\left\{A_{4}\left(t\right)|t\in I_{4}\left(n\right)\right\}\\
&=&e^{-\tilde{\mu}_{3}\xi_{3}\left(m\right)}e^{-\tilde{\mu}_{4}\xi_{4}
\left(m\right)}
=e^{-\left[\tilde{\mu}_{3}\xi_{3}\left(m\right)+\tilde{\mu}_{4}\xi_{4}
\left(m\right)\right]}.
\end{eqnarray*}


es decir, 

\begin{equation}
\prob\left\{A_{3}\left(t\right),A_{4}\left(t\right)|t\in I_{3}\left(m\right)\right\}
=e^{-\left[\tilde{\mu}_{3}\xi_{3}\left(m\right)+\tilde{\mu}_{4}\xi_{4}
\left(m\right)\right]}>0.
\end{equation}

Por construcci\'on se tiene que $I\left(n,m\right)\equiv I_{1}\left(n\right)\cap I_{3}\left(m\right)\neq\emptyset$,entonces en particular se tienen las contenciones $I\left(n,m\right)\subseteq I_{1}\left(n\right)$ y $I\left(n,m\right)\subseteq I_{3}\left(m\right)$, por lo tanto si definimos $\xi_{n,m}\equiv\ell\left(I\left(n,m\right)\right)$ tenemos que

\begin{eqnarray*}
\xi_{n,m}\leq\xi_{1}\left(n\right)\textrm{ y }\xi_{n,m}\leq\xi_{3}\left(m\right)\textrm{ entonces }
-\xi_{n,m}\geq-\xi_{1}\left(n\right)\textrm{ y }-\xi_{n,m}\leq-\xi_{3}\left(m\right)\\
\end{eqnarray*}
por lo tanto tenemos las desigualdades 



\begin{eqnarray*}
\begin{array}{ll}
-\tilde{\mu}_{1}\xi_{n,m}\geq-\tilde{\mu}_{1}\xi_{1}\left(n\right),&
-\tilde{\mu}_{2}\xi_{n,m}\geq-\tilde{\mu}_{2}\xi_{1}\left(n\right)
\geq-\tilde{\mu}_{2}\xi_{2}\left(n\right),\\
-\tilde{\mu}_{3}\xi_{n,m}\geq-\tilde{\mu}_{3}\xi_{3}\left(m\right),&
-\tilde{\mu}_{4}\xi_{n,m}\geq-\tilde{\mu}_{4}\xi_{3}\left(m\right)
\geq-\tilde{\mu}_{4}\xi_{4}\left(m\right).
\end{array}
\end{eqnarray*}

Sea $T^{*}\in I_{n,m}$, entonces dado que en particular $T^{*}\in I_{1}\left(n\right)$ se cumple con probabilidad positiva que no hay arribos a las colas $Q_{1}$ y $Q_{2}$, en consecuencia, tampoco hay usuarios de transferencia para $Q_{3}$ y $Q_{4}$, es decir, $\tilde{\mu}_{1}=\mu_{1}$, $\tilde{\mu}_{2}=\mu_{2}$, $\tilde{\mu}_{3}=\mu_{3}$, $\tilde{\mu}_{4}=\mu_{4}$, es decir, los eventos $Q_{1}$ y $Q_{3}$ son condicionalmente independientes en el intervalo $I_{n,m}$; lo mismo ocurre para las colas $Q_{2}$ y $Q_{4}$, por lo tanto tenemos que


\begin{eqnarray}
\begin{array}{l}
\prob\left\{A_{1}\left(T^{*}\right),A_{2}\left(T^{*}\right),
A_{3}\left(T^{*}\right),A_{4}\left(T^{*}\right)|T^{*}\in I_{n,m}\right\}
=\prod_{j=1}^{4}\prob\left\{A_{j}\left(T^{*}\right)|T^{*}\in I_{n,m}\right\}\\
\geq\prob\left\{A_{1}\left(T^{*}\right)|T^{*}\in I_{1}\left(n\right)\right\}
\prob\left\{A_{2}\left(T^{*}\right)|T^{*}\in I_{2}\left(n\right)\right\}
\prob\left\{A_{3}\left(T^{*}\right)|T^{*}\in I_{3}\left(m\right)\right\}
\prob\left\{A_{4}\left(T^{*}\right)|T^{*}\in I_{4}\left(m\right)\right\}\\
=e^{-\mu_{1}\xi_{1}\left(n\right)}
e^{-\mu_{2}\xi_{2}\left(n\right)}
e^{-\mu_{3}\xi_{3}\left(m\right)}
e^{-\mu_{4}\xi_{4}\left(m\right)}
=e^{-\left[\tilde{\mu}_{1}\xi_{1}\left(n\right)
+\tilde{\mu}_{2}\xi_{2}\left(n\right)
+\tilde{\mu}_{3}\xi_{3}\left(m\right)
+\tilde{\mu}_{4}\xi_{4}
\left(m\right)\right]}>0.
\end{array}
\end{eqnarray}
\end{proof}


Estos resultados aparecen en Daley (1968) \cite{Daley68} para $\left\{T_{n}\right\}$ intervalos de inter-arribo, $\left\{D_{n}\right\}$ intervalos de inter-salida y $\left\{S_{n}\right\}$ tiempos de servicio.

\begin{itemize}
\item Si el proceso $\left\{T_{n}\right\}$ es Poisson, el proceso $\left\{D_{n}\right\}$ es no correlacionado si y s\'olo si es un proceso Poisso, lo cual ocurre si y s\'olo si $\left\{S_{n}\right\}$ son exponenciales negativas.

\item Si $\left\{S_{n}\right\}$ son exponenciales negativas, $\left\{D_{n}\right\}$ es un proceso de renovaci\'on  si y s\'olo si es un proceso Poisson, lo cual ocurre si y s\'olo si $\left\{T_{n}\right\}$ es un proceso Poisson.

\item $\esp\left(D_{n}\right)=\esp\left(T_{n}\right)$.

\item Para un sistema de visitas $GI/M/1$ se tiene el siguiente teorema:

\begin{Teo}
En un sistema estacionario $GI/M/1$ los intervalos de interpartida tienen
\begin{eqnarray*}
\esp\left(e^{-\theta D_{n}}\right)&=&\mu\left(\mu+\theta\right)^{-1}\left[\delta\theta
-\mu\left(1-\delta\right)\alpha\left(\theta\right)\right]
\left[\theta-\mu\left(1-\delta\right)^{-1}\right]\\
\alpha\left(\theta\right)&=&\esp\left[e^{-\theta T_{0}}\right]\\
var\left(D_{n}\right)&=&var\left(T_{0}\right)-\left(\tau^{-1}-\delta^{-1}\right)
2\delta\left(\esp\left(S_{0}\right)\right)^{2}\left(1-\delta\right)^{-1}.
\end{eqnarray*}
\end{Teo}



\begin{Teo}
El proceso de salida de un sistema de colas estacionario $GI/M/1$ es un proceso de renovaci\'on si y s\'olo si el proceso de entrada es un proceso Poisson, en cuyo caso el proceso de salida es un proceso Poisson.
\end{Teo}


\begin{Teo}
Los intervalos de interpartida $\left\{D_{n}\right\}$ de un sistema $M/G/1$ estacionario son no correlacionados si y s\'olo si la distribuci\'on de los tiempos de servicio es exponencial negativa, es decir, el sistema es de tipo  $M/M/1$.

\end{Teo}



\end{itemize}


%\section{Resultados para Procesos de Salida}

En Sigman, Thorison y Wolff \cite{Sigman2} prueban que para la existencia de un una sucesi\'on infinita no decreciente de tiempos de regeneraci\'on $\tau_{1}\leq\tau_{2}\leq\cdots$ en los cuales el proceso se regenera, basta un tiempo de regeneraci\'on $R_{1}$, donde $R_{j}=\tau_{j}-\tau_{j-1}$. Para tal efecto se requiere la existencia de un espacio de probabilidad $\left(\Omega,\mathcal{F},\prob\right)$, y proceso estoc\'astico $\textit{X}=\left\{X\left(t\right):t\geq0\right\}$ con espacio de estados $\left(S,\mathcal{R}\right)$, con $\mathcal{R}$ $\sigma$-\'algebra.

\begin{Prop}
Si existe una variable aleatoria no negativa $R_{1}$ tal que $\theta_{R\footnotesize{1}}X=_{D}X$, entonces $\left(\Omega,\mathcal{F},\prob\right)$ puede extenderse para soportar una sucesi\'on estacionaria de variables aleatorias $R=\left\{R_{k}:k\geq1\right\}$, tal que para $k\geq1$,
\begin{eqnarray*}
\theta_{k}\left(X,R\right)=_{D}\left(X,R\right).
\end{eqnarray*}

Adem\'as, para $k\geq1$, $\theta_{k}R$ es condicionalmente independiente de $\left(X,R_{1},\ldots,R_{k}\right)$, dado $\theta_{\tau k}X$.

\end{Prop}


\begin{itemize}
\item Doob en 1953 demostr\'o que el estado estacionario de un proceso de partida en un sistema de espera $M/G/\infty$, es Poisson con la misma tasa que el proceso de arribos.

\item Burke en 1968, fue el primero en demostrar que el estado estacionario de un proceso de salida de una cola $M/M/s$ es un proceso Poisson.

\item Disney en 1973 obtuvo el siguiente resultado:

\begin{Teo}
Para el sistema de espera $M/G/1/L$ con disciplina FIFO, el proceso $\textbf{I}$ es un proceso de renovaci\'on si y s\'olo si el proceso denominado longitud de la cola es estacionario y se cumple cualquiera de los siguientes casos:

\begin{itemize}
\item[a)] Los tiempos de servicio son identicamente cero;
\item[b)] $L=0$, para cualquier proceso de servicio $S$;
\item[c)] $L=1$ y $G=D$;
\item[d)] $L=\infty$ y $G=M$.
\end{itemize}
En estos casos, respectivamente, las distribuciones de interpartida $P\left\{T_{n+1}-T_{n}\leq t\right\}$ son


\begin{itemize}
\item[a)] $1-e^{-\lambda t}$, $t\geq0$;
\item[b)] $1-e^{-\lambda t}*F\left(t\right)$, $t\geq0$;
\item[c)] $1-e^{-\lambda t}*\indora_{d}\left(t\right)$, $t\geq0$;
\item[d)] $1-e^{-\lambda t}*F\left(t\right)$, $t\geq0$.
\end{itemize}
\end{Teo}


\item Finch (1959) mostr\'o que para los sistemas $M/G/1/L$, con $1\leq L\leq \infty$ con distribuciones de servicio dos veces diferenciable, solamente el sistema $M/M/1/\infty$ tiene proceso de salida de renovaci\'on estacionario.

\item King (1971) demostro que un sistema de colas estacionario $M/G/1/1$ tiene sus tiempos de interpartida sucesivas $D_{n}$ y $D_{n+1}$ son independientes, si y s\'olo si, $G=D$, en cuyo caso le proceso de salida es de renovaci\'on.

\item Disney (1973) demostr\'o que el \'unico sistema estacionario $M/G/1/L$, que tiene proceso de salida de renovaci\'on  son los sistemas $M/M/1$ y $M/D/1/1$.



\item El siguiente resultado es de Disney y Koning (1985)
\begin{Teo}
En un sistema de espera $M/G/s$, el estado estacionario del proceso de salida es un proceso Poisson para cualquier distribuci\'on de los tiempos de servicio si el sistema tiene cualquiera de las siguientes cuatro propiedades.

\begin{itemize}
\item[a)] $s=\infty$
\item[b)] La disciplina de servicio es de procesador compartido.
\item[c)] La disciplina de servicio es LCFS y preemptive resume, esto se cumple para $L<\infty$
\item[d)] $G=M$.
\end{itemize}

\end{Teo}

\item El siguiente resultado es de Alamatsaz (1983)

\begin{Teo}
En cualquier sistema de colas $GI/G/1/L$ con $1\leq L<\infty$ y distribuci\'on de interarribos $A$ y distribuci\'on de los tiempos de servicio $B$, tal que $A\left(0\right)=0$, $A\left(t\right)\left(1-B\left(t\right)\right)>0$ para alguna $t>0$ y $B\left(t\right)$ para toda $t>0$, es imposible que el proceso de salida estacionario sea de renovaci\'on.
\end{Teo}

\end{itemize}

Estos resultados aparecen en Daley (1968) \cite{Daley68} para $\left\{T_{n}\right\}$ intervalos de inter-arribo, $\left\{D_{n}\right\}$ intervalos de inter-salida y $\left\{S_{n}\right\}$ tiempos de servicio.

\begin{itemize}
\item Si el proceso $\left\{T_{n}\right\}$ es Poisson, el proceso $\left\{D_{n}\right\}$ es no correlacionado si y s\'olo si es un proceso Poisso, lo cual ocurre si y s\'olo si $\left\{S_{n}\right\}$ son exponenciales negativas.

\item Si $\left\{S_{n}\right\}$ son exponenciales negativas, $\left\{D_{n}\right\}$ es un proceso de renovaci\'on  si y s\'olo si es un proceso Poisson, lo cual ocurre si y s\'olo si $\left\{T_{n}\right\}$ es un proceso Poisson.

\item $\esp\left(D_{n}\right)=\esp\left(T_{n}\right)$.

\item Para un sistema de visitas $GI/M/1$ se tiene el siguiente teorema:

\begin{Teo}
En un sistema estacionario $GI/M/1$ los intervalos de interpartida tienen
\begin{eqnarray*}
\esp\left(e^{-\theta D_{n}}\right)&=&\mu\left(\mu+\theta\right)^{-1}\left[\delta\theta
-\mu\left(1-\delta\right)\alpha\left(\theta\right)\right]
\left[\theta-\mu\left(1-\delta\right)^{-1}\right]\\
\alpha\left(\theta\right)&=&\esp\left[e^{-\theta T_{0}}\right]\\
var\left(D_{n}\right)&=&var\left(T_{0}\right)-\left(\tau^{-1}-\delta^{-1}\right)
2\delta\left(\esp\left(S_{0}\right)\right)^{2}\left(1-\delta\right)^{-1}.
\end{eqnarray*}
\end{Teo}



\begin{Teo}
El proceso de salida de un sistema de colas estacionario $GI/M/1$ es un proceso de renovaci\'on si y s\'olo si el proceso de entrada es un proceso Poisson, en cuyo caso el proceso de salida es un proceso Poisson.
\end{Teo}


\begin{Teo}
Los intervalos de interpartida $\left\{D_{n}\right\}$ de un sistema $M/G/1$ estacionario son no correlacionados si y s\'olo si la distribuci\'on de los tiempos de servicio es exponencial negativa, es decir, el sistema es de tipo  $M/M/1$.

\end{Teo}



\end{itemize}
%\newpage
%________________________________________________________________________
%\section{Redes de Sistemas de Visitas C\'iclicas}
%________________________________________________________________________

Sean $Q_{1},Q_{2},Q_{3}$ y $Q_{4}$ en una Red de Sistemas de Visitas C\'iclicas (RSVC). Supongamos que cada una de las colas es del tipo $M/M/1$ con tasa de arribo $\mu_{i}$ y que la transferencia de usuarios entre los dos sistemas ocurre entre $Q_{1}\leftrightarrow Q_{3}$ y $Q_{2}\leftrightarrow Q_{4}$ con respectiva tasa de arribo igual a la tasa de salida $\hat{\mu}_{i}=\mu_{i}$, esto se sabe por lo desarrollado en la secci\'on anterior.  

Consideremos, sin p\'erdida de generalidad como base del an\'alisis, la cola $Q_{1}$ adem\'as supongamos al servidor lo comenzamos a observar una vez que termina de atender a la misma para desplazarse y llegar a $Q_{2}$, es decir al tiempo $\tau_{2}$.

Sea $n\in\nat$, $n>0$, ciclo del servidor en que regresa a $Q_{1}$ para dar servicio y atender conforme a la pol\'itica exhaustiva, entonces se tiene que $\overline{\tau}_{1}\left(n\right)$ es el tiempo del servidor en el sistema 1 en que termina de dar servicio a todos los usuarios presentes en la cola, por lo tanto se cumple que $L_{1}\left(\overline{\tau}_{1}\left(n\right)\right)=0$, entonces el servidor para llegar a $Q_{2}$ incurre en un tiempo de traslado $r_{1}$ y por tanto se cumple que $\tau_{2}\left(n\right)=\overline{\tau}_{1}\left(n\right)+r_{1}$. Dado que los tiempos entre arribos son exponenciales se cumple que 

\begin{eqnarray*}
\prob\left\{0 \textrm{ arribos en }Q_{1}\textrm{ en el intervalo }\left[\overline{\tau}_{1}\left(n\right),\overline{\tau}_{1}\left(n\right)+r_{1}\right]\right\}=e^{-\tilde{\mu}_{1}r_{1}},\\
\prob\left\{0 \textrm{ arribos en }Q_{2}\textrm{ en el intervalo }\left[\overline{\tau}_{1}\left(n\right),\overline{\tau}_{1}\left(n\right)+r_{1}\right]\right\}=e^{-\tilde{\mu}_{2}r_{1}}.
\end{eqnarray*}

El evento que nos interesa consiste en que no haya arribos desde que el servidor abandon\'o $Q_{2}$ y regresa nuevamente para dar servicio, es decir en el intervalo de tiempo $\left[\overline{\tau}_{2}\left(n-1\right),\tau_{2}\left(n\right)\right]$. Entonces, si hacemos


\begin{eqnarray*}
\varphi_{1}\left(n\right)&\equiv&\overline{\tau}_{1}\left(n\right)+r_{1}=\overline{\tau}_{2}\left(n-1\right)+r_{1}+r_{2}+\overline{\tau}_{1}\left(n\right)-\tau_{1}\left(n\right)\\
&=&\overline{\tau}_{2}\left(n-1\right)+\overline{\tau}_{1}\left(n\right)-\tau_{1}\left(n\right)+r,,
\end{eqnarray*}

y longitud del intervalo

\begin{eqnarray*}
\xi&\equiv&\overline{\tau}_{1}\left(n\right)+r_{1}-\overline{\tau}_{2}\left(n-1\right)
=\overline{\tau}_{2}\left(n-1\right)+\overline{\tau}_{1}\left(n\right)-\tau_{1}\left(n\right)+r-\overline{\tau}_{2}\left(n-1\right)\\
&=&\overline{\tau}_{1}\left(n\right)-\tau_{1}\left(n\right)+r.
\end{eqnarray*}


Entonces, determinemos la probabilidad del evento no arribos a $Q_{2}$ en $\left[\overline{\tau}_{2}\left(n-1\right),\varphi_{1}\left(n\right)\right]$:

\begin{eqnarray}
\prob\left\{0 \textrm{ arribos en }Q_{2}\textrm{ en el intervalo }\left[\overline{\tau}_{2}\left(n-1\right),\varphi_{1}\left(n\right)\right]\right\}
=e^{-\tilde{\mu}_{2}\xi}.
\end{eqnarray}

De manera an\'aloga, tenemos que la probabilidad de no arribos a $Q_{1}$ en $\left[\overline{\tau}_{2}\left(n-1\right),\varphi_{1}\left(n\right)\right]$ esta dada por

\begin{eqnarray}
\prob\left\{0 \textrm{ arribos en }Q_{1}\textrm{ en el intervalo }\left[\overline{\tau}_{2}\left(n-1\right),\varphi_{1}\left(n\right)\right]\right\}
=e^{-\tilde{\mu}_{1}\xi},
\end{eqnarray}

\begin{eqnarray}
\prob\left\{0 \textrm{ arribos en }Q_{2}\textrm{ en el intervalo }\left[\overline{\tau}_{2}\left(n-1\right),\varphi_{1}\left(n\right)\right]\right\}
=e^{-\tilde{\mu}_{2}\xi}.
\end{eqnarray}

Por tanto 

\begin{eqnarray}
\begin{array}{l}
\prob\left\{0 \textrm{ arribos en }Q_{1}\textrm{ y }Q_{2}\textrm{ en el intervalo }\left[\overline{\tau}_{2}\left(n-1\right),\varphi_{1}\left(n\right)\right]\right\}\\
=\prob\left\{0 \textrm{ arribos en }Q_{1}\textrm{ en el intervalo }\left[\overline{\tau}_{2}\left(n-1\right),\varphi_{1}\left(n\right)\right]\right\}\\
\times
\prob\left\{0 \textrm{ arribos en }Q_{2}\textrm{ en el intervalo }\left[\overline{\tau}_{2}\left(n-1\right),\varphi_{1}\left(n\right)\right]\right\}=e^{-\tilde{\mu}_{1}\xi}e^{-\tilde{\mu}_{2}\xi}
=e^{-\tilde{\mu}\xi}.
\end{array}
\end{eqnarray}

Para el segundo sistema, consideremos nuevamente $\overline{\tau}_{1}\left(n\right)+r_{1}$, sin p\'erdida de generalidad podemos suponer que existe $m>0$ tal que $\overline{\tau}_{3}\left(m\right)<\overline{\tau}_{1}+r_{1}<\tau_{4}\left(m\right)$, entonces

\begin{eqnarray}
\prob\left\{0 \textrm{ arribos en }Q_{3}\textrm{ en el intervalo }\left[\overline{\tau}_{3}\left(m\right),\overline{\tau}_{1}\left(n\right)+r_{1}\right]\right\}
=e^{-\tilde{\mu}_{3}\xi_{3}},
\end{eqnarray}
donde 
\begin{eqnarray}
\xi_{3}=\overline{\tau}_{1}\left(n\right)+r_{1}-\overline{\tau}_{3}\left(m\right)=
\overline{\tau}_{1}\left(n\right)-\overline{\tau}_{3}\left(m\right)+r_{1},
\end{eqnarray}

mientras que para $Q_{4}$ al igual que con $Q_{2}$ escribiremos $\tau_{4}\left(m\right)$ en t\'erminos de $\overline{\tau}_{4}\left(m-1\right)$:

$\varphi_{2}\equiv\tau_{4}\left(m\right)=\overline{\tau}_{4}\left(m-1\right)+r_{4}+\overline{\tau}_{3}\left(m\right)
-\tau_{3}\left(m\right)+r_{3}=\overline{\tau}_{4}\left(m-1\right)+\overline{\tau}_{3}\left(m\right)
-\tau_{3}\left(m\right)+\hat{r}$, adem\'as,

$\xi_{2}\equiv\varphi_{2}\left(m\right)-\overline{\tau}_{1}-r_{1}=\overline{\tau}_{4}\left(m-1\right)+\overline{\tau}_{3}\left(m\right)
-\tau_{3}\left(m\right)-\overline{\tau}_{1}\left(n\right)+\hat{r}-r_{1}$. 

Entonces


\begin{eqnarray}
\prob\left\{0 \textrm{ arribos en }Q_{4}\textrm{ en el intervalo }\left[\overline{\tau}_{1}\left(n\right)+r_{1},\varphi_{2}\left(m\right)\right]\right\}
=e^{-\tilde{\mu}_{4}\xi_{2}},
\end{eqnarray}

mientras que para $Q_{3}$ se tiene que 

\begin{eqnarray}
\prob\left\{0 \textrm{ arribos en }Q_{3}\textrm{ en el intervalo }\left[\overline{\tau}_{1}\left(n\right)+r_{1},\varphi_{2}\left(m\right)\right]\right\}
=e^{-\tilde{\mu}_{3}\xi_{2}}
\end{eqnarray}

Por tanto

\begin{eqnarray}
\prob\left\{0 \textrm{ arribos en }Q_{3}\wedge Q_{4}\textrm{ en el intervalo }\left[\overline{\tau}_{1}\left(n\right)+r_{1},\varphi_{2}\left(m\right)\right]\right\}
=e^{-\hat{\mu}\xi_{2}}
\end{eqnarray}
donde $\hat{\mu}=\tilde{\mu}_{3}+\tilde{\mu}_{4}$.

Ahora, definamos los intervalos $\mathcal{I}_{1}=\left[\overline{\tau}_{1}\left(n\right)+r_{1},\varphi_{1}\left(n\right)\right]$  y $\mathcal{I}_{2}=\left[\overline{\tau}_{1}\left(n\right)+r_{1},\varphi_{2}\left(m\right)\right]$, entonces, sea $\mathcal{I}=\mathcal{I}_{1}\cap\mathcal{I}_{2}$ el intervalo donde cada una de las colas se encuentran vac\'ias, es decir, si tomamos $T^{*}\in\mathcal{I}$, entonces  $L_{1}\left(T^{*}\right)=L_{2}\left(T^{*}\right)=L_{3}\left(T^{*}\right)=L_{4}\left(T^{*}\right)=0$.

Ahora, dado que por construcci\'on $\mathcal{I}\neq\emptyset$ y que para $T^{*}\in\mathcal{I}$ en ninguna de las colas han llegado usuarios, se tiene que no hay transferencia entre las colas, por lo tanto, el sistema 1 y el sistema 2 son condicionalmente independientes en $\mathcal{I}$, es decir

\begin{eqnarray}
\prob\left\{L_{1}\left(T^{*}\right),L_{2}\left(T^{*}\right),L_{3}\left(T^{*}\right),L_{4}\left(T^{*}\right)|T^{*}\in\mathcal{I}\right\}=\prod_{j=1}^{4}\prob\left\{L_{j}\left(T^{*}\right)\right\},
\end{eqnarray}

para $T^{*}\in\mathcal{I}$. 

%\newpage























%________________________________________________________________________
%\section{Procesos Regenerativos}
%________________________________________________________________________

%________________________________________________________________________
%\subsection*{Procesos Regenerativos Sigman, Thorisson y Wolff \cite{Sigman1}}
%________________________________________________________________________


\begin{Def}[Definici\'on Cl\'asica]
Un proceso estoc\'astico $X=\left\{X\left(t\right):t\geq0\right\}$ es llamado regenerativo is existe una variable aleatoria $R_{1}>0$ tal que
\begin{itemize}
\item[i)] $\left\{X\left(t+R_{1}\right):t\geq0\right\}$ es independiente de $\left\{\left\{X\left(t\right):t<R_{1}\right\},\right\}$
\item[ii)] $\left\{X\left(t+R_{1}\right):t\geq0\right\}$ es estoc\'asticamente equivalente a $\left\{X\left(t\right):t>0\right\}$
\end{itemize}

Llamamos a $R_{1}$ tiempo de regeneraci\'on, y decimos que $X$ se regenera en este punto.
\end{Def}

$\left\{X\left(t+R_{1}\right)\right\}$ es regenerativo con tiempo de regeneraci\'on $R_{2}$, independiente de $R_{1}$ pero con la misma distribuci\'on que $R_{1}$. Procediendo de esta manera se obtiene una secuencia de variables aleatorias independientes e id\'enticamente distribuidas $\left\{R_{n}\right\}$ llamados longitudes de ciclo. Si definimos a $Z_{k}\equiv R_{1}+R_{2}+\cdots+R_{k}$, se tiene un proceso de renovaci\'on llamado proceso de renovaci\'on encajado para $X$.


\begin{Note}
La existencia de un primer tiempo de regeneraci\'on, $R_{1}$, implica la existencia de una sucesi\'on completa de estos tiempos $R_{1},R_{2}\ldots,$ que satisfacen la propiedad deseada \cite{Sigman2}.
\end{Note}


\begin{Note} Para la cola $GI/GI/1$ los usuarios arriban con tiempos $t_{n}$ y son atendidos con tiempos de servicio $S_{n}$, los tiempos de arribo forman un proceso de renovaci\'on  con tiempos entre arribos independientes e identicamente distribuidos (\texttt{i.i.d.})$T_{n}=t_{n}-t_{n-1}$, adem\'as los tiempos de servicio son \texttt{i.i.d.} e independientes de los procesos de arribo. Por \textit{estable} se entiende que $\esp S_{n}<\esp T_{n}<\infty$.
\end{Note}
 


\begin{Def}
Para $x$ fijo y para cada $t\geq0$, sea $I_{x}\left(t\right)=1$ si $X\left(t\right)\leq x$,  $I_{x}\left(t\right)=0$ en caso contrario, y def\'inanse los tiempos promedio
\begin{eqnarray*}
\overline{X}&=&lim_{t\rightarrow\infty}\frac{1}{t}\int_{0}^{\infty}X\left(u\right)du\\
\prob\left(X_{\infty}\leq x\right)&=&lim_{t\rightarrow\infty}\frac{1}{t}\int_{0}^{\infty}I_{x}\left(u\right)du,
\end{eqnarray*}
cuando estos l\'imites existan.
\end{Def}

Como consecuencia del teorema de Renovaci\'on-Recompensa, se tiene que el primer l\'imite  existe y es igual a la constante
\begin{eqnarray*}
\overline{X}&=&\frac{\esp\left[\int_{0}^{R_{1}}X\left(t\right)dt\right]}{\esp\left[R_{1}\right]},
\end{eqnarray*}
suponiendo que ambas esperanzas son finitas.
 
\begin{Note}
Funciones de procesos regenerativos son regenerativas, es decir, si $X\left(t\right)$ es regenerativo y se define el proceso $Y\left(t\right)$ por $Y\left(t\right)=f\left(X\left(t\right)\right)$ para alguna funci\'on Borel medible $f\left(\cdot\right)$. Adem\'as $Y$ es regenerativo con los mismos tiempos de renovaci\'on que $X$. 

En general, los tiempos de renovaci\'on, $Z_{k}$ de un proceso regenerativo no requieren ser tiempos de paro con respecto a la evoluci\'on de $X\left(t\right)$.
\end{Note} 

\begin{Note}
Una funci\'on de un proceso de Markov, usualmente no ser\'a un proceso de Markov, sin embargo ser\'a regenerativo si el proceso de Markov lo es.
\end{Note}

 
\begin{Note}
Un proceso regenerativo con media de la longitud de ciclo finita es llamado positivo recurrente.
\end{Note}


\begin{Note}
\begin{itemize}
\item[a)] Si el proceso regenerativo $X$ es positivo recurrente y tiene trayectorias muestrales no negativas, entonces la ecuaci\'on anterior es v\'alida.
\item[b)] Si $X$ es positivo recurrente regenerativo, podemos construir una \'unica versi\'on estacionaria de este proceso, $X_{e}=\left\{X_{e}\left(t\right)\right\}$, donde $X_{e}$ es un proceso estoc\'astico regenerativo y estrictamente estacionario, con distribuci\'on marginal distribuida como $X_{\infty}$
\end{itemize}
\end{Note}


%__________________________________________________________________________________________
%\subsection*{Procesos Regenerativos Estacionarios - Stidham \cite{Stidham}}
%__________________________________________________________________________________________


Un proceso estoc\'astico a tiempo continuo $\left\{V\left(t\right),t\geq0\right\}$ es un proceso regenerativo si existe una sucesi\'on de variables aleatorias independientes e id\'enticamente distribuidas $\left\{X_{1},X_{2},\ldots\right\}$, sucesi\'on de renovaci\'on, tal que para cualquier conjunto de Borel $A$, 

\begin{eqnarray*}
\prob\left\{V\left(t\right)\in A|X_{1}+X_{2}+\cdots+X_{R\left(t\right)}=s,\left\{V\left(\tau\right),\tau<s\right\}\right\}=\prob\left\{V\left(t-s\right)\in A|X_{1}>t-s\right\},
\end{eqnarray*}
para todo $0\leq s\leq t$, donde $R\left(t\right)=\max\left\{X_{1}+X_{2}+\cdots+X_{j}\leq t\right\}=$n\'umero de renovaciones ({\emph{puntos de regeneraci\'on}}) que ocurren en $\left[0,t\right]$. El intervalo $\left[0,X_{1}\right)$ es llamado {\emph{primer ciclo de regeneraci\'on}} de $\left\{V\left(t \right),t\geq0\right\}$, $\left[X_{1},X_{1}+X_{2}\right)$ el {\emph{segundo ciclo de regeneraci\'on}}, y as\'i sucesivamente.

Sea $X=X_{1}$ y sea $F$ la funci\'on de distrbuci\'on de $X$


\begin{Def}
Se define el proceso estacionario, $\left\{V^{*}\left(t\right),t\geq0\right\}$, para $\left\{V\left(t\right),t\geq0\right\}$ por

\begin{eqnarray*}
\prob\left\{V\left(t\right)\in A\right\}=\frac{1}{\esp\left[X\right]}\int_{0}^{\infty}\prob\left\{V\left(t+x\right)\in A|X>x\right\}\left(1-F\left(x\right)\right)dx,
\end{eqnarray*} 
para todo $t\geq0$ y todo conjunto de Borel $A$.
\end{Def}

\begin{Def}
Una distribuci\'on se dice que es {\emph{aritm\'etica}} si todos sus puntos de incremento son m\'ultiplos de la forma $0,\lambda, 2\lambda,\ldots$ para alguna $\lambda>0$ entera.
\end{Def}


\begin{Def}
Una modificaci\'on medible de un proceso $\left\{V\left(t\right),t\geq0\right\}$, es una versi\'on de este, $\left\{V\left(t,w\right)\right\}$ conjuntamente medible para $t\geq0$ y para $w\in S$, $S$ espacio de estados para $\left\{V\left(t\right),t\geq0\right\}$.
\end{Def}

\begin{Teo}
Sea $\left\{V\left(t\right),t\geq\right\}$ un proceso regenerativo no negativo con modificaci\'on medible. Sea $\esp\left[X\right]<\infty$. Entonces el proceso estacionario dado por la ecuaci\'on anterior est\'a bien definido y tiene funci\'on de distribuci\'on independiente de $t$, adem\'as
\begin{itemize}
\item[i)] \begin{eqnarray*}
\esp\left[V^{*}\left(0\right)\right]&=&\frac{\esp\left[\int_{0}^{X}V\left(s\right)ds\right]}{\esp\left[X\right]}\end{eqnarray*}
\item[ii)] Si $\esp\left[V^{*}\left(0\right)\right]<\infty$, equivalentemente, si $\esp\left[\int_{0}^{X}V\left(s\right)ds\right]<\infty$,entonces
\begin{eqnarray*}
\frac{\int_{0}^{t}V\left(s\right)ds}{t}\rightarrow\frac{\esp\left[\int_{0}^{X}V\left(s\right)ds\right]}{\esp\left[X\right]}
\end{eqnarray*}
con probabilidad 1 y en media, cuando $t\rightarrow\infty$.
\end{itemize}
\end{Teo}

\begin{Coro}
Sea $\left\{V\left(t\right),t\geq0\right\}$ un proceso regenerativo no negativo, con modificaci\'on medible. Si $\esp <\infty$, $F$ es no-aritm\'etica, y para todo $x\geq0$, $P\left\{V\left(t\right)\leq x,C>x\right\}$ es de variaci\'on acotada como funci\'on de $t$ en cada intervalo finito $\left[0,\tau\right]$, entonces $V\left(t\right)$ converge en distribuci\'on  cuando $t\rightarrow\infty$ y $$\esp V=\frac{\esp \int_{0}^{X}V\left(s\right)ds}{\esp X}$$
Donde $V$ tiene la distribuci\'on l\'imite de $V\left(t\right)$ cuando $t\rightarrow\infty$.

\end{Coro}

Para el caso discreto se tienen resultados similares.



%______________________________________________________________________
%\section{Procesos de Renovaci\'on}
%______________________________________________________________________

\begin{Def}\label{Def.Tn}
Sean $0\leq T_{1}\leq T_{2}\leq \ldots$ son tiempos aleatorios infinitos en los cuales ocurren ciertos eventos. El n\'umero de tiempos $T_{n}$ en el intervalo $\left[0,t\right)$ es

\begin{eqnarray}
N\left(t\right)=\sum_{n=1}^{\infty}\indora\left(T_{n}\leq t\right),
\end{eqnarray}
para $t\geq0$.
\end{Def}

Si se consideran los puntos $T_{n}$ como elementos de $\rea_{+}$, y $N\left(t\right)$ es el n\'umero de puntos en $\rea$. El proceso denotado por $\left\{N\left(t\right):t\geq0\right\}$, denotado por $N\left(t\right)$, es un proceso puntual en $\rea_{+}$. Los $T_{n}$ son los tiempos de ocurrencia, el proceso puntual $N\left(t\right)$ es simple si su n\'umero de ocurrencias son distintas: $0<T_{1}<T_{2}<\ldots$ casi seguramente.

\begin{Def}
Un proceso puntual $N\left(t\right)$ es un proceso de renovaci\'on si los tiempos de interocurrencia $\xi_{n}=T_{n}-T_{n-1}$, para $n\geq1$, son independientes e identicamente distribuidos con distribuci\'on $F$, donde $F\left(0\right)=0$ y $T_{0}=0$. Los $T_{n}$ son llamados tiempos de renovaci\'on, referente a la independencia o renovaci\'on de la informaci\'on estoc\'astica en estos tiempos. Los $\xi_{n}$ son los tiempos de inter-renovaci\'on, y $N\left(t\right)$ es el n\'umero de renovaciones en el intervalo $\left[0,t\right)$
\end{Def}


\begin{Note}
Para definir un proceso de renovaci\'on para cualquier contexto, solamente hay que especificar una distribuci\'on $F$, con $F\left(0\right)=0$, para los tiempos de inter-renovaci\'on. La funci\'on $F$ en turno degune las otra variables aleatorias. De manera formal, existe un espacio de probabilidad y una sucesi\'on de variables aleatorias $\xi_{1},\xi_{2},\ldots$ definidas en este con distribuci\'on $F$. Entonces las otras cantidades son $T_{n}=\sum_{k=1}^{n}\xi_{k}$ y $N\left(t\right)=\sum_{n=1}^{\infty}\indora\left(T_{n}\leq t\right)$, donde $T_{n}\rightarrow\infty$ casi seguramente por la Ley Fuerte de los Grandes Números.
\end{Note}

%___________________________________________________________________________________________
%
%\subsection*{Teorema Principal de Renovaci\'on}
%___________________________________________________________________________________________
%

\begin{Note} Una funci\'on $h:\rea_{+}\rightarrow\rea$ es Directamente Riemann Integrable en los siguientes casos:
\begin{itemize}
\item[a)] $h\left(t\right)\geq0$ es decreciente y Riemann Integrable.
\item[b)] $h$ es continua excepto posiblemente en un conjunto de Lebesgue de medida 0, y $|h\left(t\right)|\leq b\left(t\right)$, donde $b$ es DRI.
\end{itemize}
\end{Note}

\begin{Teo}[Teorema Principal de Renovaci\'on]
Si $F$ es no aritm\'etica y $h\left(t\right)$ es Directamente Riemann Integrable (DRI), entonces

\begin{eqnarray*}
lim_{t\rightarrow\infty}U\star h=\frac{1}{\mu}\int_{\rea_{+}}h\left(s\right)ds.
\end{eqnarray*}
\end{Teo}

\begin{Prop}
Cualquier funci\'on $H\left(t\right)$ acotada en intervalos finitos y que es 0 para $t<0$ puede expresarse como
\begin{eqnarray*}
H\left(t\right)=U\star h\left(t\right)\textrm{,  donde }h\left(t\right)=H\left(t\right)-F\star H\left(t\right)
\end{eqnarray*}
\end{Prop}

\begin{Def}
Un proceso estoc\'astico $X\left(t\right)$ es crudamente regenerativo en un tiempo aleatorio positivo $T$ si
\begin{eqnarray*}
\esp\left[X\left(T+t\right)|T\right]=\esp\left[X\left(t\right)\right]\textrm{, para }t\geq0,\end{eqnarray*}
y con las esperanzas anteriores finitas.
\end{Def}

\begin{Prop}
Sup\'ongase que $X\left(t\right)$ es un proceso crudamente regenerativo en $T$, que tiene distribuci\'on $F$. Si $\esp\left[X\left(t\right)\right]$ es acotado en intervalos finitos, entonces
\begin{eqnarray*}
\esp\left[X\left(t\right)\right]=U\star h\left(t\right)\textrm{,  donde }h\left(t\right)=\esp\left[X\left(t\right)\indora\left(T>t\right)\right].
\end{eqnarray*}
\end{Prop}

\begin{Teo}[Regeneraci\'on Cruda]
Sup\'ongase que $X\left(t\right)$ es un proceso con valores positivo crudamente regenerativo en $T$, y def\'inase $M=\sup\left\{|X\left(t\right)|:t\leq T\right\}$. Si $T$ es no aritm\'etico y $M$ y $MT$ tienen media finita, entonces
\begin{eqnarray*}
lim_{t\rightarrow\infty}\esp\left[X\left(t\right)\right]=\frac{1}{\mu}\int_{\rea_{+}}h\left(s\right)ds,
\end{eqnarray*}
donde $h\left(t\right)=\esp\left[X\left(t\right)\indora\left(T>t\right)\right]$.
\end{Teo}

%___________________________________________________________________________________________
%
%\subsection*{Propiedades de los Procesos de Renovaci\'on}
%___________________________________________________________________________________________
%

Los tiempos $T_{n}$ est\'an relacionados con los conteos de $N\left(t\right)$ por

\begin{eqnarray*}
\left\{N\left(t\right)\geq n\right\}&=&\left\{T_{n}\leq t\right\}\\
T_{N\left(t\right)}\leq &t&<T_{N\left(t\right)+1},
\end{eqnarray*}

adem\'as $N\left(T_{n}\right)=n$, y 

\begin{eqnarray*}
N\left(t\right)=\max\left\{n:T_{n}\leq t\right\}=\min\left\{n:T_{n+1}>t\right\}
\end{eqnarray*}

Por propiedades de la convoluci\'on se sabe que

\begin{eqnarray*}
P\left\{T_{n}\leq t\right\}=F^{n\star}\left(t\right)
\end{eqnarray*}
que es la $n$-\'esima convoluci\'on de $F$. Entonces 

\begin{eqnarray*}
\left\{N\left(t\right)\geq n\right\}&=&\left\{T_{n}\leq t\right\}\\
P\left\{N\left(t\right)\leq n\right\}&=&1-F^{\left(n+1\right)\star}\left(t\right)
\end{eqnarray*}

Adem\'as usando el hecho de que $\esp\left[N\left(t\right)\right]=\sum_{n=1}^{\infty}P\left\{N\left(t\right)\geq n\right\}$
se tiene que

\begin{eqnarray*}
\esp\left[N\left(t\right)\right]=\sum_{n=1}^{\infty}F^{n\star}\left(t\right)
\end{eqnarray*}

\begin{Prop}
Para cada $t\geq0$, la funci\'on generadora de momentos $\esp\left[e^{\alpha N\left(t\right)}\right]$ existe para alguna $\alpha$ en una vecindad del 0, y de aqu\'i que $\esp\left[N\left(t\right)^{m}\right]<\infty$, para $m\geq1$.
\end{Prop}


\begin{Note}
Si el primer tiempo de renovaci\'on $\xi_{1}$ no tiene la misma distribuci\'on que el resto de las $\xi_{n}$, para $n\geq2$, a $N\left(t\right)$ se le llama Proceso de Renovaci\'on retardado, donde si $\xi$ tiene distribuci\'on $G$, entonces el tiempo $T_{n}$ de la $n$-\'esima renovaci\'on tiene distribuci\'on $G\star F^{\left(n-1\right)\star}\left(t\right)$
\end{Note}


\begin{Teo}
Para una constante $\mu\leq\infty$ ( o variable aleatoria), las siguientes expresiones son equivalentes:

\begin{eqnarray}
lim_{n\rightarrow\infty}n^{-1}T_{n}&=&\mu,\textrm{ c.s.}\\
lim_{t\rightarrow\infty}t^{-1}N\left(t\right)&=&1/\mu,\textrm{ c.s.}
\end{eqnarray}
\end{Teo}


Es decir, $T_{n}$ satisface la Ley Fuerte de los Grandes N\'umeros s\'i y s\'olo s\'i $N\left/t\right)$ la cumple.


\begin{Coro}[Ley Fuerte de los Grandes N\'umeros para Procesos de Renovaci\'on]
Si $N\left(t\right)$ es un proceso de renovaci\'on cuyos tiempos de inter-renovaci\'on tienen media $\mu\leq\infty$, entonces
\begin{eqnarray}
t^{-1}N\left(t\right)\rightarrow 1/\mu,\textrm{ c.s. cuando }t\rightarrow\infty.
\end{eqnarray}

\end{Coro}


Considerar el proceso estoc\'astico de valores reales $\left\{Z\left(t\right):t\geq0\right\}$ en el mismo espacio de probabilidad que $N\left(t\right)$

\begin{Def}
Para el proceso $\left\{Z\left(t\right):t\geq0\right\}$ se define la fluctuaci\'on m\'axima de $Z\left(t\right)$ en el intervalo $\left(T_{n-1},T_{n}\right]$:
\begin{eqnarray*}
M_{n}=\sup_{T_{n-1}<t\leq T_{n}}|Z\left(t\right)-Z\left(T_{n-1}\right)|
\end{eqnarray*}
\end{Def}

\begin{Teo}
Sup\'ongase que $n^{-1}T_{n}\rightarrow\mu$ c.s. cuando $n\rightarrow\infty$, donde $\mu\leq\infty$ es una constante o variable aleatoria. Sea $a$ una constante o variable aleatoria que puede ser infinita cuando $\mu$ es finita, y considere las expresiones l\'imite:
\begin{eqnarray}
lim_{n\rightarrow\infty}n^{-1}Z\left(T_{n}\right)&=&a,\textrm{ c.s.}\\
lim_{t\rightarrow\infty}t^{-1}Z\left(t\right)&=&a/\mu,\textrm{ c.s.}
\end{eqnarray}
La segunda expresi\'on implica la primera. Conversamente, la primera implica la segunda si el proceso $Z\left(t\right)$ es creciente, o si $lim_{n\rightarrow\infty}n^{-1}M_{n}=0$ c.s.
\end{Teo}

\begin{Coro}
Si $N\left(t\right)$ es un proceso de renovaci\'on, y $\left(Z\left(T_{n}\right)-Z\left(T_{n-1}\right),M_{n}\right)$, para $n\geq1$, son variables aleatorias independientes e id\'enticamente distribuidas con media finita, entonces,
\begin{eqnarray}
lim_{t\rightarrow\infty}t^{-1}Z\left(t\right)\rightarrow\frac{\esp\left[Z\left(T_{1}\right)-Z\left(T_{0}\right)\right]}{\esp\left[T_{1}\right]},\textrm{ c.s. cuando  }t\rightarrow\infty.
\end{eqnarray}
\end{Coro}



%___________________________________________________________________________________________
%
%\subsection{Propiedades de los Procesos de Renovaci\'on}
%___________________________________________________________________________________________
%

Los tiempos $T_{n}$ est\'an relacionados con los conteos de $N\left(t\right)$ por

\begin{eqnarray*}
\left\{N\left(t\right)\geq n\right\}&=&\left\{T_{n}\leq t\right\}\\
T_{N\left(t\right)}\leq &t&<T_{N\left(t\right)+1},
\end{eqnarray*}

adem\'as $N\left(T_{n}\right)=n$, y 

\begin{eqnarray*}
N\left(t\right)=\max\left\{n:T_{n}\leq t\right\}=\min\left\{n:T_{n+1}>t\right\}
\end{eqnarray*}

Por propiedades de la convoluci\'on se sabe que

\begin{eqnarray*}
P\left\{T_{n}\leq t\right\}=F^{n\star}\left(t\right)
\end{eqnarray*}
que es la $n$-\'esima convoluci\'on de $F$. Entonces 

\begin{eqnarray*}
\left\{N\left(t\right)\geq n\right\}&=&\left\{T_{n}\leq t\right\}\\
P\left\{N\left(t\right)\leq n\right\}&=&1-F^{\left(n+1\right)\star}\left(t\right)
\end{eqnarray*}

Adem\'as usando el hecho de que $\esp\left[N\left(t\right)\right]=\sum_{n=1}^{\infty}P\left\{N\left(t\right)\geq n\right\}$
se tiene que

\begin{eqnarray*}
\esp\left[N\left(t\right)\right]=\sum_{n=1}^{\infty}F^{n\star}\left(t\right)
\end{eqnarray*}

\begin{Prop}
Para cada $t\geq0$, la funci\'on generadora de momentos $\esp\left[e^{\alpha N\left(t\right)}\right]$ existe para alguna $\alpha$ en una vecindad del 0, y de aqu\'i que $\esp\left[N\left(t\right)^{m}\right]<\infty$, para $m\geq1$.
\end{Prop}


\begin{Note}
Si el primer tiempo de renovaci\'on $\xi_{1}$ no tiene la misma distribuci\'on que el resto de las $\xi_{n}$, para $n\geq2$, a $N\left(t\right)$ se le llama Proceso de Renovaci\'on retardado, donde si $\xi$ tiene distribuci\'on $G$, entonces el tiempo $T_{n}$ de la $n$-\'esima renovaci\'on tiene distribuci\'on $G\star F^{\left(n-1\right)\star}\left(t\right)$
\end{Note}


\begin{Teo}
Para una constante $\mu\leq\infty$ ( o variable aleatoria), las siguientes expresiones son equivalentes:

\begin{eqnarray}
lim_{n\rightarrow\infty}n^{-1}T_{n}&=&\mu,\textrm{ c.s.}\\
lim_{t\rightarrow\infty}t^{-1}N\left(t\right)&=&1/\mu,\textrm{ c.s.}
\end{eqnarray}
\end{Teo}


Es decir, $T_{n}$ satisface la Ley Fuerte de los Grandes N\'umeros s\'i y s\'olo s\'i $N\left/t\right)$ la cumple.


\begin{Coro}[Ley Fuerte de los Grandes N\'umeros para Procesos de Renovaci\'on]
Si $N\left(t\right)$ es un proceso de renovaci\'on cuyos tiempos de inter-renovaci\'on tienen media $\mu\leq\infty$, entonces
\begin{eqnarray}
t^{-1}N\left(t\right)\rightarrow 1/\mu,\textrm{ c.s. cuando }t\rightarrow\infty.
\end{eqnarray}

\end{Coro}


Considerar el proceso estoc\'astico de valores reales $\left\{Z\left(t\right):t\geq0\right\}$ en el mismo espacio de probabilidad que $N\left(t\right)$

\begin{Def}
Para el proceso $\left\{Z\left(t\right):t\geq0\right\}$ se define la fluctuaci\'on m\'axima de $Z\left(t\right)$ en el intervalo $\left(T_{n-1},T_{n}\right]$:
\begin{eqnarray*}
M_{n}=\sup_{T_{n-1}<t\leq T_{n}}|Z\left(t\right)-Z\left(T_{n-1}\right)|
\end{eqnarray*}
\end{Def}

\begin{Teo}
Sup\'ongase que $n^{-1}T_{n}\rightarrow\mu$ c.s. cuando $n\rightarrow\infty$, donde $\mu\leq\infty$ es una constante o variable aleatoria. Sea $a$ una constante o variable aleatoria que puede ser infinita cuando $\mu$ es finita, y considere las expresiones l\'imite:
\begin{eqnarray}
lim_{n\rightarrow\infty}n^{-1}Z\left(T_{n}\right)&=&a,\textrm{ c.s.}\\
lim_{t\rightarrow\infty}t^{-1}Z\left(t\right)&=&a/\mu,\textrm{ c.s.}
\end{eqnarray}
La segunda expresi\'on implica la primera. Conversamente, la primera implica la segunda si el proceso $Z\left(t\right)$ es creciente, o si $lim_{n\rightarrow\infty}n^{-1}M_{n}=0$ c.s.
\end{Teo}

\begin{Coro}
Si $N\left(t\right)$ es un proceso de renovaci\'on, y $\left(Z\left(T_{n}\right)-Z\left(T_{n-1}\right),M_{n}\right)$, para $n\geq1$, son variables aleatorias independientes e id\'enticamente distribuidas con media finita, entonces,
\begin{eqnarray}
lim_{t\rightarrow\infty}t^{-1}Z\left(t\right)\rightarrow\frac{\esp\left[Z\left(T_{1}\right)-Z\left(T_{0}\right)\right]}{\esp\left[T_{1}\right]},\textrm{ c.s. cuando  }t\rightarrow\infty.
\end{eqnarray}
\end{Coro}


%___________________________________________________________________________________________
%
%\subsection{Propiedades de los Procesos de Renovaci\'on}
%___________________________________________________________________________________________
%

Los tiempos $T_{n}$ est\'an relacionados con los conteos de $N\left(t\right)$ por

\begin{eqnarray*}
\left\{N\left(t\right)\geq n\right\}&=&\left\{T_{n}\leq t\right\}\\
T_{N\left(t\right)}\leq &t&<T_{N\left(t\right)+1},
\end{eqnarray*}

adem\'as $N\left(T_{n}\right)=n$, y 

\begin{eqnarray*}
N\left(t\right)=\max\left\{n:T_{n}\leq t\right\}=\min\left\{n:T_{n+1}>t\right\}
\end{eqnarray*}

Por propiedades de la convoluci\'on se sabe que

\begin{eqnarray*}
P\left\{T_{n}\leq t\right\}=F^{n\star}\left(t\right)
\end{eqnarray*}
que es la $n$-\'esima convoluci\'on de $F$. Entonces 

\begin{eqnarray*}
\left\{N\left(t\right)\geq n\right\}&=&\left\{T_{n}\leq t\right\}\\
P\left\{N\left(t\right)\leq n\right\}&=&1-F^{\left(n+1\right)\star}\left(t\right)
\end{eqnarray*}

Adem\'as usando el hecho de que $\esp\left[N\left(t\right)\right]=\sum_{n=1}^{\infty}P\left\{N\left(t\right)\geq n\right\}$
se tiene que

\begin{eqnarray*}
\esp\left[N\left(t\right)\right]=\sum_{n=1}^{\infty}F^{n\star}\left(t\right)
\end{eqnarray*}

\begin{Prop}
Para cada $t\geq0$, la funci\'on generadora de momentos $\esp\left[e^{\alpha N\left(t\right)}\right]$ existe para alguna $\alpha$ en una vecindad del 0, y de aqu\'i que $\esp\left[N\left(t\right)^{m}\right]<\infty$, para $m\geq1$.
\end{Prop}


\begin{Note}
Si el primer tiempo de renovaci\'on $\xi_{1}$ no tiene la misma distribuci\'on que el resto de las $\xi_{n}$, para $n\geq2$, a $N\left(t\right)$ se le llama Proceso de Renovaci\'on retardado, donde si $\xi$ tiene distribuci\'on $G$, entonces el tiempo $T_{n}$ de la $n$-\'esima renovaci\'on tiene distribuci\'on $G\star F^{\left(n-1\right)\star}\left(t\right)$
\end{Note}


\begin{Teo}
Para una constante $\mu\leq\infty$ ( o variable aleatoria), las siguientes expresiones son equivalentes:

\begin{eqnarray}
lim_{n\rightarrow\infty}n^{-1}T_{n}&=&\mu,\textrm{ c.s.}\\
lim_{t\rightarrow\infty}t^{-1}N\left(t\right)&=&1/\mu,\textrm{ c.s.}
\end{eqnarray}
\end{Teo}


Es decir, $T_{n}$ satisface la Ley Fuerte de los Grandes N\'umeros s\'i y s\'olo s\'i $N\left/t\right)$ la cumple.


\begin{Coro}[Ley Fuerte de los Grandes N\'umeros para Procesos de Renovaci\'on]
Si $N\left(t\right)$ es un proceso de renovaci\'on cuyos tiempos de inter-renovaci\'on tienen media $\mu\leq\infty$, entonces
\begin{eqnarray}
t^{-1}N\left(t\right)\rightarrow 1/\mu,\textrm{ c.s. cuando }t\rightarrow\infty.
\end{eqnarray}

\end{Coro}


Considerar el proceso estoc\'astico de valores reales $\left\{Z\left(t\right):t\geq0\right\}$ en el mismo espacio de probabilidad que $N\left(t\right)$

\begin{Def}
Para el proceso $\left\{Z\left(t\right):t\geq0\right\}$ se define la fluctuaci\'on m\'axima de $Z\left(t\right)$ en el intervalo $\left(T_{n-1},T_{n}\right]$:
\begin{eqnarray*}
M_{n}=\sup_{T_{n-1}<t\leq T_{n}}|Z\left(t\right)-Z\left(T_{n-1}\right)|
\end{eqnarray*}
\end{Def}

\begin{Teo}
Sup\'ongase que $n^{-1}T_{n}\rightarrow\mu$ c.s. cuando $n\rightarrow\infty$, donde $\mu\leq\infty$ es una constante o variable aleatoria. Sea $a$ una constante o variable aleatoria que puede ser infinita cuando $\mu$ es finita, y considere las expresiones l\'imite:
\begin{eqnarray}
lim_{n\rightarrow\infty}n^{-1}Z\left(T_{n}\right)&=&a,\textrm{ c.s.}\\
lim_{t\rightarrow\infty}t^{-1}Z\left(t\right)&=&a/\mu,\textrm{ c.s.}
\end{eqnarray}
La segunda expresi\'on implica la primera. Conversamente, la primera implica la segunda si el proceso $Z\left(t\right)$ es creciente, o si $lim_{n\rightarrow\infty}n^{-1}M_{n}=0$ c.s.
\end{Teo}

\begin{Coro}
Si $N\left(t\right)$ es un proceso de renovaci\'on, y $\left(Z\left(T_{n}\right)-Z\left(T_{n-1}\right),M_{n}\right)$, para $n\geq1$, son variables aleatorias independientes e id\'enticamente distribuidas con media finita, entonces,
\begin{eqnarray}
lim_{t\rightarrow\infty}t^{-1}Z\left(t\right)\rightarrow\frac{\esp\left[Z\left(T_{1}\right)-Z\left(T_{0}\right)\right]}{\esp\left[T_{1}\right]},\textrm{ c.s. cuando  }t\rightarrow\infty.
\end{eqnarray}
\end{Coro}

%___________________________________________________________________________________________
%
%\subsection{Propiedades de los Procesos de Renovaci\'on}
%___________________________________________________________________________________________
%

Los tiempos $T_{n}$ est\'an relacionados con los conteos de $N\left(t\right)$ por

\begin{eqnarray*}
\left\{N\left(t\right)\geq n\right\}&=&\left\{T_{n}\leq t\right\}\\
T_{N\left(t\right)}\leq &t&<T_{N\left(t\right)+1},
\end{eqnarray*}

adem\'as $N\left(T_{n}\right)=n$, y 

\begin{eqnarray*}
N\left(t\right)=\max\left\{n:T_{n}\leq t\right\}=\min\left\{n:T_{n+1}>t\right\}
\end{eqnarray*}

Por propiedades de la convoluci\'on se sabe que

\begin{eqnarray*}
P\left\{T_{n}\leq t\right\}=F^{n\star}\left(t\right)
\end{eqnarray*}
que es la $n$-\'esima convoluci\'on de $F$. Entonces 

\begin{eqnarray*}
\left\{N\left(t\right)\geq n\right\}&=&\left\{T_{n}\leq t\right\}\\
P\left\{N\left(t\right)\leq n\right\}&=&1-F^{\left(n+1\right)\star}\left(t\right)
\end{eqnarray*}

Adem\'as usando el hecho de que $\esp\left[N\left(t\right)\right]=\sum_{n=1}^{\infty}P\left\{N\left(t\right)\geq n\right\}$
se tiene que

\begin{eqnarray*}
\esp\left[N\left(t\right)\right]=\sum_{n=1}^{\infty}F^{n\star}\left(t\right)
\end{eqnarray*}

\begin{Prop}
Para cada $t\geq0$, la funci\'on generadora de momentos $\esp\left[e^{\alpha N\left(t\right)}\right]$ existe para alguna $\alpha$ en una vecindad del 0, y de aqu\'i que $\esp\left[N\left(t\right)^{m}\right]<\infty$, para $m\geq1$.
\end{Prop}


\begin{Note}
Si el primer tiempo de renovaci\'on $\xi_{1}$ no tiene la misma distribuci\'on que el resto de las $\xi_{n}$, para $n\geq2$, a $N\left(t\right)$ se le llama Proceso de Renovaci\'on retardado, donde si $\xi$ tiene distribuci\'on $G$, entonces el tiempo $T_{n}$ de la $n$-\'esima renovaci\'on tiene distribuci\'on $G\star F^{\left(n-1\right)\star}\left(t\right)$
\end{Note}


\begin{Teo}
Para una constante $\mu\leq\infty$ ( o variable aleatoria), las siguientes expresiones son equivalentes:

\begin{eqnarray}
lim_{n\rightarrow\infty}n^{-1}T_{n}&=&\mu,\textrm{ c.s.}\\
lim_{t\rightarrow\infty}t^{-1}N\left(t\right)&=&1/\mu,\textrm{ c.s.}
\end{eqnarray}
\end{Teo}


Es decir, $T_{n}$ satisface la Ley Fuerte de los Grandes N\'umeros s\'i y s\'olo s\'i $N\left/t\right)$ la cumple.


\begin{Coro}[Ley Fuerte de los Grandes N\'umeros para Procesos de Renovaci\'on]
Si $N\left(t\right)$ es un proceso de renovaci\'on cuyos tiempos de inter-renovaci\'on tienen media $\mu\leq\infty$, entonces
\begin{eqnarray}
t^{-1}N\left(t\right)\rightarrow 1/\mu,\textrm{ c.s. cuando }t\rightarrow\infty.
\end{eqnarray}

\end{Coro}


Considerar el proceso estoc\'astico de valores reales $\left\{Z\left(t\right):t\geq0\right\}$ en el mismo espacio de probabilidad que $N\left(t\right)$

\begin{Def}
Para el proceso $\left\{Z\left(t\right):t\geq0\right\}$ se define la fluctuaci\'on m\'axima de $Z\left(t\right)$ en el intervalo $\left(T_{n-1},T_{n}\right]$:
\begin{eqnarray*}
M_{n}=\sup_{T_{n-1}<t\leq T_{n}}|Z\left(t\right)-Z\left(T_{n-1}\right)|
\end{eqnarray*}
\end{Def}

\begin{Teo}
Sup\'ongase que $n^{-1}T_{n}\rightarrow\mu$ c.s. cuando $n\rightarrow\infty$, donde $\mu\leq\infty$ es una constante o variable aleatoria. Sea $a$ una constante o variable aleatoria que puede ser infinita cuando $\mu$ es finita, y considere las expresiones l\'imite:
\begin{eqnarray}
lim_{n\rightarrow\infty}n^{-1}Z\left(T_{n}\right)&=&a,\textrm{ c.s.}\\
lim_{t\rightarrow\infty}t^{-1}Z\left(t\right)&=&a/\mu,\textrm{ c.s.}
\end{eqnarray}
La segunda expresi\'on implica la primera. Conversamente, la primera implica la segunda si el proceso $Z\left(t\right)$ es creciente, o si $lim_{n\rightarrow\infty}n^{-1}M_{n}=0$ c.s.
\end{Teo}

\begin{Coro}
Si $N\left(t\right)$ es un proceso de renovaci\'on, y $\left(Z\left(T_{n}\right)-Z\left(T_{n-1}\right),M_{n}\right)$, para $n\geq1$, son variables aleatorias independientes e id\'enticamente distribuidas con media finita, entonces,
\begin{eqnarray}
lim_{t\rightarrow\infty}t^{-1}Z\left(t\right)\rightarrow\frac{\esp\left[Z\left(T_{1}\right)-Z\left(T_{0}\right)\right]}{\esp\left[T_{1}\right]},\textrm{ c.s. cuando  }t\rightarrow\infty.
\end{eqnarray}
\end{Coro}
%___________________________________________________________________________________________
%
%\subsection{Propiedades de los Procesos de Renovaci\'on}
%___________________________________________________________________________________________
%

Los tiempos $T_{n}$ est\'an relacionados con los conteos de $N\left(t\right)$ por

\begin{eqnarray*}
\left\{N\left(t\right)\geq n\right\}&=&\left\{T_{n}\leq t\right\}\\
T_{N\left(t\right)}\leq &t&<T_{N\left(t\right)+1},
\end{eqnarray*}

adem\'as $N\left(T_{n}\right)=n$, y 

\begin{eqnarray*}
N\left(t\right)=\max\left\{n:T_{n}\leq t\right\}=\min\left\{n:T_{n+1}>t\right\}
\end{eqnarray*}

Por propiedades de la convoluci\'on se sabe que

\begin{eqnarray*}
P\left\{T_{n}\leq t\right\}=F^{n\star}\left(t\right)
\end{eqnarray*}
que es la $n$-\'esima convoluci\'on de $F$. Entonces 

\begin{eqnarray*}
\left\{N\left(t\right)\geq n\right\}&=&\left\{T_{n}\leq t\right\}\\
P\left\{N\left(t\right)\leq n\right\}&=&1-F^{\left(n+1\right)\star}\left(t\right)
\end{eqnarray*}

Adem\'as usando el hecho de que $\esp\left[N\left(t\right)\right]=\sum_{n=1}^{\infty}P\left\{N\left(t\right)\geq n\right\}$
se tiene que

\begin{eqnarray*}
\esp\left[N\left(t\right)\right]=\sum_{n=1}^{\infty}F^{n\star}\left(t\right)
\end{eqnarray*}

\begin{Prop}
Para cada $t\geq0$, la funci\'on generadora de momentos $\esp\left[e^{\alpha N\left(t\right)}\right]$ existe para alguna $\alpha$ en una vecindad del 0, y de aqu\'i que $\esp\left[N\left(t\right)^{m}\right]<\infty$, para $m\geq1$.
\end{Prop}


\begin{Note}
Si el primer tiempo de renovaci\'on $\xi_{1}$ no tiene la misma distribuci\'on que el resto de las $\xi_{n}$, para $n\geq2$, a $N\left(t\right)$ se le llama Proceso de Renovaci\'on retardado, donde si $\xi$ tiene distribuci\'on $G$, entonces el tiempo $T_{n}$ de la $n$-\'esima renovaci\'on tiene distribuci\'on $G\star F^{\left(n-1\right)\star}\left(t\right)$
\end{Note}


\begin{Teo}
Para una constante $\mu\leq\infty$ ( o variable aleatoria), las siguientes expresiones son equivalentes:

\begin{eqnarray}
lim_{n\rightarrow\infty}n^{-1}T_{n}&=&\mu,\textrm{ c.s.}\\
lim_{t\rightarrow\infty}t^{-1}N\left(t\right)&=&1/\mu,\textrm{ c.s.}
\end{eqnarray}
\end{Teo}


Es decir, $T_{n}$ satisface la Ley Fuerte de los Grandes N\'umeros s\'i y s\'olo s\'i $N\left/t\right)$ la cumple.


\begin{Coro}[Ley Fuerte de los Grandes N\'umeros para Procesos de Renovaci\'on]
Si $N\left(t\right)$ es un proceso de renovaci\'on cuyos tiempos de inter-renovaci\'on tienen media $\mu\leq\infty$, entonces
\begin{eqnarray}
t^{-1}N\left(t\right)\rightarrow 1/\mu,\textrm{ c.s. cuando }t\rightarrow\infty.
\end{eqnarray}

\end{Coro}


Considerar el proceso estoc\'astico de valores reales $\left\{Z\left(t\right):t\geq0\right\}$ en el mismo espacio de probabilidad que $N\left(t\right)$

\begin{Def}
Para el proceso $\left\{Z\left(t\right):t\geq0\right\}$ se define la fluctuaci\'on m\'axima de $Z\left(t\right)$ en el intervalo $\left(T_{n-1},T_{n}\right]$:
\begin{eqnarray*}
M_{n}=\sup_{T_{n-1}<t\leq T_{n}}|Z\left(t\right)-Z\left(T_{n-1}\right)|
\end{eqnarray*}
\end{Def}

\begin{Teo}
Sup\'ongase que $n^{-1}T_{n}\rightarrow\mu$ c.s. cuando $n\rightarrow\infty$, donde $\mu\leq\infty$ es una constante o variable aleatoria. Sea $a$ una constante o variable aleatoria que puede ser infinita cuando $\mu$ es finita, y considere las expresiones l\'imite:
\begin{eqnarray}
lim_{n\rightarrow\infty}n^{-1}Z\left(T_{n}\right)&=&a,\textrm{ c.s.}\\
lim_{t\rightarrow\infty}t^{-1}Z\left(t\right)&=&a/\mu,\textrm{ c.s.}
\end{eqnarray}
La segunda expresi\'on implica la primera. Conversamente, la primera implica la segunda si el proceso $Z\left(t\right)$ es creciente, o si $lim_{n\rightarrow\infty}n^{-1}M_{n}=0$ c.s.
\end{Teo}

\begin{Coro}
Si $N\left(t\right)$ es un proceso de renovaci\'on, y $\left(Z\left(T_{n}\right)-Z\left(T_{n-1}\right),M_{n}\right)$, para $n\geq1$, son variables aleatorias independientes e id\'enticamente distribuidas con media finita, entonces,
\begin{eqnarray}
lim_{t\rightarrow\infty}t^{-1}Z\left(t\right)\rightarrow\frac{\esp\left[Z\left(T_{1}\right)-Z\left(T_{0}\right)\right]}{\esp\left[T_{1}\right]},\textrm{ c.s. cuando  }t\rightarrow\infty.
\end{eqnarray}
\end{Coro}


%___________________________________________________________________________________________
%
%\subsection*{Funci\'on de Renovaci\'on}
%___________________________________________________________________________________________
%


\begin{Def}
Sea $h\left(t\right)$ funci\'on de valores reales en $\rea$ acotada en intervalos finitos e igual a cero para $t<0$ La ecuaci\'on de renovaci\'on para $h\left(t\right)$ y la distribuci\'on $F$ es

\begin{eqnarray}\label{Ec.Renovacion}
H\left(t\right)=h\left(t\right)+\int_{\left[0,t\right]}H\left(t-s\right)dF\left(s\right)\textrm{,    }t\geq0,
\end{eqnarray}
donde $H\left(t\right)$ es una funci\'on de valores reales. Esto es $H=h+F\star H$. Decimos que $H\left(t\right)$ es soluci\'on de esta ecuaci\'on si satisface la ecuaci\'on, y es acotada en intervalos finitos e iguales a cero para $t<0$.
\end{Def}

\begin{Prop}
La funci\'on $U\star h\left(t\right)$ es la \'unica soluci\'on de la ecuaci\'on de renovaci\'on (\ref{Ec.Renovacion}).
\end{Prop}

\begin{Teo}[Teorema Renovaci\'on Elemental]
\begin{eqnarray*}
t^{-1}U\left(t\right)\rightarrow 1/\mu\textrm{,    cuando }t\rightarrow\infty.
\end{eqnarray*}
\end{Teo}

%___________________________________________________________________________________________
%
%\subsection{Funci\'on de Renovaci\'on}
%___________________________________________________________________________________________
%


Sup\'ongase que $N\left(t\right)$ es un proceso de renovaci\'on con distribuci\'on $F$ con media finita $\mu$.

\begin{Def}
La funci\'on de renovaci\'on asociada con la distribuci\'on $F$, del proceso $N\left(t\right)$, es
\begin{eqnarray*}
U\left(t\right)=\sum_{n=1}^{\infty}F^{n\star}\left(t\right),\textrm{   }t\geq0,
\end{eqnarray*}
donde $F^{0\star}\left(t\right)=\indora\left(t\geq0\right)$.
\end{Def}


\begin{Prop}
Sup\'ongase que la distribuci\'on de inter-renovaci\'on $F$ tiene densidad $f$. Entonces $U\left(t\right)$ tambi\'en tiene densidad, para $t>0$, y es $U^{'}\left(t\right)=\sum_{n=0}^{\infty}f^{n\star}\left(t\right)$. Adem\'as
\begin{eqnarray*}
\prob\left\{N\left(t\right)>N\left(t-\right)\right\}=0\textrm{,   }t\geq0.
\end{eqnarray*}
\end{Prop}

\begin{Def}
La Transformada de Laplace-Stieljes de $F$ est\'a dada por

\begin{eqnarray*}
\hat{F}\left(\alpha\right)=\int_{\rea_{+}}e^{-\alpha t}dF\left(t\right)\textrm{,  }\alpha\geq0.
\end{eqnarray*}
\end{Def}

Entonces

\begin{eqnarray*}
\hat{U}\left(\alpha\right)=\sum_{n=0}^{\infty}\hat{F^{n\star}}\left(\alpha\right)=\sum_{n=0}^{\infty}\hat{F}\left(\alpha\right)^{n}=\frac{1}{1-\hat{F}\left(\alpha\right)}.
\end{eqnarray*}


\begin{Prop}
La Transformada de Laplace $\hat{U}\left(\alpha\right)$ y $\hat{F}\left(\alpha\right)$ determina una a la otra de manera \'unica por la relaci\'on $\hat{U}\left(\alpha\right)=\frac{1}{1-\hat{F}\left(\alpha\right)}$.
\end{Prop}


\begin{Note}
Un proceso de renovaci\'on $N\left(t\right)$ cuyos tiempos de inter-renovaci\'on tienen media finita, es un proceso Poisson con tasa $\lambda$ si y s\'olo s\'i $\esp\left[U\left(t\right)\right]=\lambda t$, para $t\geq0$.
\end{Note}


\begin{Teo}
Sea $N\left(t\right)$ un proceso puntual simple con puntos de localizaci\'on $T_{n}$ tal que $\eta\left(t\right)=\esp\left[N\left(\right)\right]$ es finita para cada $t$. Entonces para cualquier funci\'on $f:\rea_{+}\rightarrow\rea$,
\begin{eqnarray*}
\esp\left[\sum_{n=1}^{N\left(\right)}f\left(T_{n}\right)\right]=\int_{\left(0,t\right]}f\left(s\right)d\eta\left(s\right)\textrm{,  }t\geq0,
\end{eqnarray*}
suponiendo que la integral exista. Adem\'as si $X_{1},X_{2},\ldots$ son variables aleatorias definidas en el mismo espacio de probabilidad que el proceso $N\left(t\right)$ tal que $\esp\left[X_{n}|T_{n}=s\right]=f\left(s\right)$, independiente de $n$. Entonces
\begin{eqnarray*}
\esp\left[\sum_{n=1}^{N\left(t\right)}X_{n}\right]=\int_{\left(0,t\right]}f\left(s\right)d\eta\left(s\right)\textrm{,  }t\geq0,
\end{eqnarray*} 
suponiendo que la integral exista. 
\end{Teo}

\begin{Coro}[Identidad de Wald para Renovaciones]
Para el proceso de renovaci\'on $N\left(t\right)$,
\begin{eqnarray*}
\esp\left[T_{N\left(t\right)+1}\right]=\mu\esp\left[N\left(t\right)+1\right]\textrm{,  }t\geq0,
\end{eqnarray*}  
\end{Coro}

%______________________________________________________________________
%\subsection{Procesos de Renovaci\'on}
%______________________________________________________________________

\begin{Def}\label{Def.Tn}
Sean $0\leq T_{1}\leq T_{2}\leq \ldots$ son tiempos aleatorios infinitos en los cuales ocurren ciertos eventos. El n\'umero de tiempos $T_{n}$ en el intervalo $\left[0,t\right)$ es

\begin{eqnarray}
N\left(t\right)=\sum_{n=1}^{\infty}\indora\left(T_{n}\leq t\right),
\end{eqnarray}
para $t\geq0$.
\end{Def}

Si se consideran los puntos $T_{n}$ como elementos de $\rea_{+}$, y $N\left(t\right)$ es el n\'umero de puntos en $\rea$. El proceso denotado por $\left\{N\left(t\right):t\geq0\right\}$, denotado por $N\left(t\right)$, es un proceso puntual en $\rea_{+}$. Los $T_{n}$ son los tiempos de ocurrencia, el proceso puntual $N\left(t\right)$ es simple si su n\'umero de ocurrencias son distintas: $0<T_{1}<T_{2}<\ldots$ casi seguramente.

\begin{Def}
Un proceso puntual $N\left(t\right)$ es un proceso de renovaci\'on si los tiempos de interocurrencia $\xi_{n}=T_{n}-T_{n-1}$, para $n\geq1$, son independientes e identicamente distribuidos con distribuci\'on $F$, donde $F\left(0\right)=0$ y $T_{0}=0$. Los $T_{n}$ son llamados tiempos de renovaci\'on, referente a la independencia o renovaci\'on de la informaci\'on estoc\'astica en estos tiempos. Los $\xi_{n}$ son los tiempos de inter-renovaci\'on, y $N\left(t\right)$ es el n\'umero de renovaciones en el intervalo $\left[0,t\right)$
\end{Def}


\begin{Note}
Para definir un proceso de renovaci\'on para cualquier contexto, solamente hay que especificar una distribuci\'on $F$, con $F\left(0\right)=0$, para los tiempos de inter-renovaci\'on. La funci\'on $F$ en turno degune las otra variables aleatorias. De manera formal, existe un espacio de probabilidad y una sucesi\'on de variables aleatorias $\xi_{1},\xi_{2},\ldots$ definidas en este con distribuci\'on $F$. Entonces las otras cantidades son $T_{n}=\sum_{k=1}^{n}\xi_{k}$ y $N\left(t\right)=\sum_{n=1}^{\infty}\indora\left(T_{n}\leq t\right)$, donde $T_{n}\rightarrow\infty$ casi seguramente por la Ley Fuerte de los Grandes Números.
\end{Note}

%___________________________________________________________________________________________
%
\subsection{Renewal and Regenerative Processes: Serfozo\cite{Serfozo}}
%___________________________________________________________________________________________
%
\begin{Def}\label{Def.Tn}
Sean $0\leq T_{1}\leq T_{2}\leq \ldots$ son tiempos aleatorios infinitos en los cuales ocurren ciertos eventos. El n\'umero de tiempos $T_{n}$ en el intervalo $\left[0,t\right)$ es

\begin{eqnarray}
N\left(t\right)=\sum_{n=1}^{\infty}\indora\left(T_{n}\leq t\right),
\end{eqnarray}
para $t\geq0$.
\end{Def}

Si se consideran los puntos $T_{n}$ como elementos de $\rea_{+}$, y $N\left(t\right)$ es el n\'umero de puntos en $\rea$. El proceso denotado por $\left\{N\left(t\right):t\geq0\right\}$, denotado por $N\left(t\right)$, es un proceso puntual en $\rea_{+}$. Los $T_{n}$ son los tiempos de ocurrencia, el proceso puntual $N\left(t\right)$ es simple si su n\'umero de ocurrencias son distintas: $0<T_{1}<T_{2}<\ldots$ casi seguramente.

\begin{Def}
Un proceso puntual $N\left(t\right)$ es un proceso de renovaci\'on si los tiempos de interocurrencia $\xi_{n}=T_{n}-T_{n-1}$, para $n\geq1$, son independientes e identicamente distribuidos con distribuci\'on $F$, donde $F\left(0\right)=0$ y $T_{0}=0$. Los $T_{n}$ son llamados tiempos de renovaci\'on, referente a la independencia o renovaci\'on de la informaci\'on estoc\'astica en estos tiempos. Los $\xi_{n}$ son los tiempos de inter-renovaci\'on, y $N\left(t\right)$ es el n\'umero de renovaciones en el intervalo $\left[0,t\right)$
\end{Def}


\begin{Note}
Para definir un proceso de renovaci\'on para cualquier contexto, solamente hay que especificar una distribuci\'on $F$, con $F\left(0\right)=0$, para los tiempos de inter-renovaci\'on. La funci\'on $F$ en turno degune las otra variables aleatorias. De manera formal, existe un espacio de probabilidad y una sucesi\'on de variables aleatorias $\xi_{1},\xi_{2},\ldots$ definidas en este con distribuci\'on $F$. Entonces las otras cantidades son $T_{n}=\sum_{k=1}^{n}\xi_{k}$ y $N\left(t\right)=\sum_{n=1}^{\infty}\indora\left(T_{n}\leq t\right)$, donde $T_{n}\rightarrow\infty$ casi seguramente por la Ley Fuerte de los Grandes N\'umeros.
\end{Note}







Los tiempos $T_{n}$ est\'an relacionados con los conteos de $N\left(t\right)$ por

\begin{eqnarray*}
\left\{N\left(t\right)\geq n\right\}&=&\left\{T_{n}\leq t\right\}\\
T_{N\left(t\right)}\leq &t&<T_{N\left(t\right)+1},
\end{eqnarray*}

adem\'as $N\left(T_{n}\right)=n$, y 

\begin{eqnarray*}
N\left(t\right)=\max\left\{n:T_{n}\leq t\right\}=\min\left\{n:T_{n+1}>t\right\}
\end{eqnarray*}

Por propiedades de la convoluci\'on se sabe que

\begin{eqnarray*}
P\left\{T_{n}\leq t\right\}=F^{n\star}\left(t\right)
\end{eqnarray*}
que es la $n$-\'esima convoluci\'on de $F$. Entonces 

\begin{eqnarray*}
\left\{N\left(t\right)\geq n\right\}&=&\left\{T_{n}\leq t\right\}\\
P\left\{N\left(t\right)\leq n\right\}&=&1-F^{\left(n+1\right)\star}\left(t\right)
\end{eqnarray*}

Adem\'as usando el hecho de que $\esp\left[N\left(t\right)\right]=\sum_{n=1}^{\infty}P\left\{N\left(t\right)\geq n\right\}$
se tiene que

\begin{eqnarray*}
\esp\left[N\left(t\right)\right]=\sum_{n=1}^{\infty}F^{n\star}\left(t\right)
\end{eqnarray*}

\begin{Prop}
Para cada $t\geq0$, la funci\'on generadora de momentos $\esp\left[e^{\alpha N\left(t\right)}\right]$ existe para alguna $\alpha$ en una vecindad del 0, y de aqu\'i que $\esp\left[N\left(t\right)^{m}\right]<\infty$, para $m\geq1$.
\end{Prop}

\begin{Ejem}[\textbf{Proceso Poisson}]

Suponga que se tienen tiempos de inter-renovaci\'on \textit{i.i.d.} del proceso de renovaci\'on $N\left(t\right)$ tienen distribuci\'on exponencial $F\left(t\right)=q-e^{-\lambda t}$ con tasa $\lambda$. Entonces $N\left(t\right)$ es un proceso Poisson con tasa $\lambda$.

\end{Ejem}


\begin{Note}
Si el primer tiempo de renovaci\'on $\xi_{1}$ no tiene la misma distribuci\'on que el resto de las $\xi_{n}$, para $n\geq2$, a $N\left(t\right)$ se le llama Proceso de Renovaci\'on retardado, donde si $\xi$ tiene distribuci\'on $G$, entonces el tiempo $T_{n}$ de la $n$-\'esima renovaci\'on tiene distribuci\'on $G\star F^{\left(n-1\right)\star}\left(t\right)$
\end{Note}


\begin{Teo}
Para una constante $\mu\leq\infty$ ( o variable aleatoria), las siguientes expresiones son equivalentes:

\begin{eqnarray}
lim_{n\rightarrow\infty}n^{-1}T_{n}&=&\mu,\textrm{ c.s.}\\
lim_{t\rightarrow\infty}t^{-1}N\left(t\right)&=&1/\mu,\textrm{ c.s.}
\end{eqnarray}
\end{Teo}


Es decir, $T_{n}$ satisface la Ley Fuerte de los Grandes N\'umeros s\'i y s\'olo s\'i $N\left/t\right)$ la cumple.


\begin{Coro}[Ley Fuerte de los Grandes N\'umeros para Procesos de Renovaci\'on]
Si $N\left(t\right)$ es un proceso de renovaci\'on cuyos tiempos de inter-renovaci\'on tienen media $\mu\leq\infty$, entonces
\begin{eqnarray}
t^{-1}N\left(t\right)\rightarrow 1/\mu,\textrm{ c.s. cuando }t\rightarrow\infty.
\end{eqnarray}

\end{Coro}


Considerar el proceso estoc\'astico de valores reales $\left\{Z\left(t\right):t\geq0\right\}$ en el mismo espacio de probabilidad que $N\left(t\right)$

\begin{Def}
Para el proceso $\left\{Z\left(t\right):t\geq0\right\}$ se define la fluctuaci\'on m\'axima de $Z\left(t\right)$ en el intervalo $\left(T_{n-1},T_{n}\right]$:
\begin{eqnarray*}
M_{n}=\sup_{T_{n-1}<t\leq T_{n}}|Z\left(t\right)-Z\left(T_{n-1}\right)|
\end{eqnarray*}
\end{Def}

\begin{Teo}
Sup\'ongase que $n^{-1}T_{n}\rightarrow\mu$ c.s. cuando $n\rightarrow\infty$, donde $\mu\leq\infty$ es una constante o variable aleatoria. Sea $a$ una constante o variable aleatoria que puede ser infinita cuando $\mu$ es finita, y considere las expresiones l\'imite:
\begin{eqnarray}
lim_{n\rightarrow\infty}n^{-1}Z\left(T_{n}\right)&=&a,\textrm{ c.s.}\\
lim_{t\rightarrow\infty}t^{-1}Z\left(t\right)&=&a/\mu,\textrm{ c.s.}
\end{eqnarray}
La segunda expresi\'on implica la primera. Conversamente, la primera implica la segunda si el proceso $Z\left(t\right)$ es creciente, o si $lim_{n\rightarrow\infty}n^{-1}M_{n}=0$ c.s.
\end{Teo}

\begin{Coro}
Si $N\left(t\right)$ es un proceso de renovaci\'on, y $\left(Z\left(T_{n}\right)-Z\left(T_{n-1}\right),M_{n}\right)$, para $n\geq1$, son variables aleatorias independientes e id\'enticamente distribuidas con media finita, entonces,
\begin{eqnarray}
lim_{t\rightarrow\infty}t^{-1}Z\left(t\right)\rightarrow\frac{\esp\left[Z\left(T_{1}\right)-Z\left(T_{0}\right)\right]}{\esp\left[T_{1}\right]},\textrm{ c.s. cuando  }t\rightarrow\infty.
\end{eqnarray}
\end{Coro}


Sup\'ongase que $N\left(t\right)$ es un proceso de renovaci\'on con distribuci\'on $F$ con media finita $\mu$.

\begin{Def}
La funci\'on de renovaci\'on asociada con la distribuci\'on $F$, del proceso $N\left(t\right)$, es
\begin{eqnarray*}
U\left(t\right)=\sum_{n=1}^{\infty}F^{n\star}\left(t\right),\textrm{   }t\geq0,
\end{eqnarray*}
donde $F^{0\star}\left(t\right)=\indora\left(t\geq0\right)$.
\end{Def}


\begin{Prop}
Sup\'ongase que la distribuci\'on de inter-renovaci\'on $F$ tiene densidad $f$. Entonces $U\left(t\right)$ tambi\'en tiene densidad, para $t>0$, y es $U^{'}\left(t\right)=\sum_{n=0}^{\infty}f^{n\star}\left(t\right)$. Adem\'as
\begin{eqnarray*}
\prob\left\{N\left(t\right)>N\left(t-\right)\right\}=0\textrm{,   }t\geq0.
\end{eqnarray*}
\end{Prop}

\begin{Def}
La Transformada de Laplace-Stieljes de $F$ est\'a dada por

\begin{eqnarray*}
\hat{F}\left(\alpha\right)=\int_{\rea_{+}}e^{-\alpha t}dF\left(t\right)\textrm{,  }\alpha\geq0.
\end{eqnarray*}
\end{Def}

Entonces

\begin{eqnarray*}
\hat{U}\left(\alpha\right)=\sum_{n=0}^{\infty}\hat{F^{n\star}}\left(\alpha\right)=\sum_{n=0}^{\infty}\hat{F}\left(\alpha\right)^{n}=\frac{1}{1-\hat{F}\left(\alpha\right)}.
\end{eqnarray*}


\begin{Prop}
La Transformada de Laplace $\hat{U}\left(\alpha\right)$ y $\hat{F}\left(\alpha\right)$ determina una a la otra de manera \'unica por la relaci\'on $\hat{U}\left(\alpha\right)=\frac{1}{1-\hat{F}\left(\alpha\right)}$.
\end{Prop}


\begin{Note}
Un proceso de renovaci\'on $N\left(t\right)$ cuyos tiempos de inter-renovaci\'on tienen media finita, es un proceso Poisson con tasa $\lambda$ si y s\'olo s\'i $\esp\left[U\left(t\right)\right]=\lambda t$, para $t\geq0$.
\end{Note}


\begin{Teo}
Sea $N\left(t\right)$ un proceso puntual simple con puntos de localizaci\'on $T_{n}$ tal que $\eta\left(t\right)=\esp\left[N\left(\right)\right]$ es finita para cada $t$. Entonces para cualquier funci\'on $f:\rea_{+}\rightarrow\rea$,
\begin{eqnarray*}
\esp\left[\sum_{n=1}^{N\left(\right)}f\left(T_{n}\right)\right]=\int_{\left(0,t\right]}f\left(s\right)d\eta\left(s\right)\textrm{,  }t\geq0,
\end{eqnarray*}
suponiendo que la integral exista. Adem\'as si $X_{1},X_{2},\ldots$ son variables aleatorias definidas en el mismo espacio de probabilidad que el proceso $N\left(t\right)$ tal que $\esp\left[X_{n}|T_{n}=s\right]=f\left(s\right)$, independiente de $n$. Entonces
\begin{eqnarray*}
\esp\left[\sum_{n=1}^{N\left(t\right)}X_{n}\right]=\int_{\left(0,t\right]}f\left(s\right)d\eta\left(s\right)\textrm{,  }t\geq0,
\end{eqnarray*} 
suponiendo que la integral exista. 
\end{Teo}

\begin{Coro}[Identidad de Wald para Renovaciones]
Para el proceso de renovaci\'on $N\left(t\right)$,
\begin{eqnarray*}
\esp\left[T_{N\left(t\right)+1}\right]=\mu\esp\left[N\left(t\right)+1\right]\textrm{,  }t\geq0,
\end{eqnarray*}  
\end{Coro}


\begin{Def}
Sea $h\left(t\right)$ funci\'on de valores reales en $\rea$ acotada en intervalos finitos e igual a cero para $t<0$ La ecuaci\'on de renovaci\'on para $h\left(t\right)$ y la distribuci\'on $F$ es

\begin{eqnarray}\label{Ec.Renovacion}
H\left(t\right)=h\left(t\right)+\int_{\left[0,t\right]}H\left(t-s\right)dF\left(s\right)\textrm{,    }t\geq0,
\end{eqnarray}
donde $H\left(t\right)$ es una funci\'on de valores reales. Esto es $H=h+F\star H$. Decimos que $H\left(t\right)$ es soluci\'on de esta ecuaci\'on si satisface la ecuaci\'on, y es acotada en intervalos finitos e iguales a cero para $t<0$.
\end{Def}

\begin{Prop}
La funci\'on $U\star h\left(t\right)$ es la \'unica soluci\'on de la ecuaci\'on de renovaci\'on (\ref{Ec.Renovacion}).
\end{Prop}

\begin{Teo}[Teorema Renovaci\'on Elemental]
\begin{eqnarray*}
t^{-1}U\left(t\right)\rightarrow 1/\mu\textrm{,    cuando }t\rightarrow\infty.
\end{eqnarray*}
\end{Teo}



Sup\'ongase que $N\left(t\right)$ es un proceso de renovaci\'on con distribuci\'on $F$ con media finita $\mu$.

\begin{Def}
La funci\'on de renovaci\'on asociada con la distribuci\'on $F$, del proceso $N\left(t\right)$, es
\begin{eqnarray*}
U\left(t\right)=\sum_{n=1}^{\infty}F^{n\star}\left(t\right),\textrm{   }t\geq0,
\end{eqnarray*}
donde $F^{0\star}\left(t\right)=\indora\left(t\geq0\right)$.
\end{Def}


\begin{Prop}
Sup\'ongase que la distribuci\'on de inter-renovaci\'on $F$ tiene densidad $f$. Entonces $U\left(t\right)$ tambi\'en tiene densidad, para $t>0$, y es $U^{'}\left(t\right)=\sum_{n=0}^{\infty}f^{n\star}\left(t\right)$. Adem\'as
\begin{eqnarray*}
\prob\left\{N\left(t\right)>N\left(t-\right)\right\}=0\textrm{,   }t\geq0.
\end{eqnarray*}
\end{Prop}

\begin{Def}
La Transformada de Laplace-Stieljes de $F$ est\'a dada por

\begin{eqnarray*}
\hat{F}\left(\alpha\right)=\int_{\rea_{+}}e^{-\alpha t}dF\left(t\right)\textrm{,  }\alpha\geq0.
\end{eqnarray*}
\end{Def}

Entonces

\begin{eqnarray*}
\hat{U}\left(\alpha\right)=\sum_{n=0}^{\infty}\hat{F^{n\star}}\left(\alpha\right)=\sum_{n=0}^{\infty}\hat{F}\left(\alpha\right)^{n}=\frac{1}{1-\hat{F}\left(\alpha\right)}.
\end{eqnarray*}


\begin{Prop}
La Transformada de Laplace $\hat{U}\left(\alpha\right)$ y $\hat{F}\left(\alpha\right)$ determina una a la otra de manera \'unica por la relaci\'on $\hat{U}\left(\alpha\right)=\frac{1}{1-\hat{F}\left(\alpha\right)}$.
\end{Prop}


\begin{Note}
Un proceso de renovaci\'on $N\left(t\right)$ cuyos tiempos de inter-renovaci\'on tienen media finita, es un proceso Poisson con tasa $\lambda$ si y s\'olo s\'i $\esp\left[U\left(t\right)\right]=\lambda t$, para $t\geq0$.
\end{Note}


\begin{Teo}
Sea $N\left(t\right)$ un proceso puntual simple con puntos de localizaci\'on $T_{n}$ tal que $\eta\left(t\right)=\esp\left[N\left(\right)\right]$ es finita para cada $t$. Entonces para cualquier funci\'on $f:\rea_{+}\rightarrow\rea$,
\begin{eqnarray*}
\esp\left[\sum_{n=1}^{N\left(\right)}f\left(T_{n}\right)\right]=\int_{\left(0,t\right]}f\left(s\right)d\eta\left(s\right)\textrm{,  }t\geq0,
\end{eqnarray*}
suponiendo que la integral exista. Adem\'as si $X_{1},X_{2},\ldots$ son variables aleatorias definidas en el mismo espacio de probabilidad que el proceso $N\left(t\right)$ tal que $\esp\left[X_{n}|T_{n}=s\right]=f\left(s\right)$, independiente de $n$. Entonces
\begin{eqnarray*}
\esp\left[\sum_{n=1}^{N\left(t\right)}X_{n}\right]=\int_{\left(0,t\right]}f\left(s\right)d\eta\left(s\right)\textrm{,  }t\geq0,
\end{eqnarray*} 
suponiendo que la integral exista. 
\end{Teo}

\begin{Coro}[Identidad de Wald para Renovaciones]
Para el proceso de renovaci\'on $N\left(t\right)$,
\begin{eqnarray*}
\esp\left[T_{N\left(t\right)+1}\right]=\mu\esp\left[N\left(t\right)+1\right]\textrm{,  }t\geq0,
\end{eqnarray*}  
\end{Coro}


\begin{Def}
Sea $h\left(t\right)$ funci\'on de valores reales en $\rea$ acotada en intervalos finitos e igual a cero para $t<0$ La ecuaci\'on de renovaci\'on para $h\left(t\right)$ y la distribuci\'on $F$ es

\begin{eqnarray}\label{Ec.Renovacion}
H\left(t\right)=h\left(t\right)+\int_{\left[0,t\right]}H\left(t-s\right)dF\left(s\right)\textrm{,    }t\geq0,
\end{eqnarray}
donde $H\left(t\right)$ es una funci\'on de valores reales. Esto es $H=h+F\star H$. Decimos que $H\left(t\right)$ es soluci\'on de esta ecuaci\'on si satisface la ecuaci\'on, y es acotada en intervalos finitos e iguales a cero para $t<0$.
\end{Def}

\begin{Prop}
La funci\'on $U\star h\left(t\right)$ es la \'unica soluci\'on de la ecuaci\'on de renovaci\'on (\ref{Ec.Renovacion}).
\end{Prop}

\begin{Teo}[Teorema Renovaci\'on Elemental]
\begin{eqnarray*}
t^{-1}U\left(t\right)\rightarrow 1/\mu\textrm{,    cuando }t\rightarrow\infty.
\end{eqnarray*}
\end{Teo}


\begin{Note} Una funci\'on $h:\rea_{+}\rightarrow\rea$ es Directamente Riemann Integrable en los siguientes casos:
\begin{itemize}
\item[a)] $h\left(t\right)\geq0$ es decreciente y Riemann Integrable.
\item[b)] $h$ es continua excepto posiblemente en un conjunto de Lebesgue de medida 0, y $|h\left(t\right)|\leq b\left(t\right)$, donde $b$ es DRI.
\end{itemize}
\end{Note}

\begin{Teo}[Teorema Principal de Renovaci\'on]
Si $F$ es no aritm\'etica y $h\left(t\right)$ es Directamente Riemann Integrable (DRI), entonces

\begin{eqnarray*}
lim_{t\rightarrow\infty}U\star h=\frac{1}{\mu}\int_{\rea_{+}}h\left(s\right)ds.
\end{eqnarray*}
\end{Teo}

\begin{Prop}
Cualquier funci\'on $H\left(t\right)$ acotada en intervalos finitos y que es 0 para $t<0$ puede expresarse como
\begin{eqnarray*}
H\left(t\right)=U\star h\left(t\right)\textrm{,  donde }h\left(t\right)=H\left(t\right)-F\star H\left(t\right)
\end{eqnarray*}
\end{Prop}

\begin{Def}
Un proceso estoc\'astico $X\left(t\right)$ es crudamente regenerativo en un tiempo aleatorio positivo $T$ si
\begin{eqnarray*}
\esp\left[X\left(T+t\right)|T\right]=\esp\left[X\left(t\right)\right]\textrm{, para }t\geq0,\end{eqnarray*}
y con las esperanzas anteriores finitas.
\end{Def}

\begin{Prop}
Sup\'ongase que $X\left(t\right)$ es un proceso crudamente regenerativo en $T$, que tiene distribuci\'on $F$. Si $\esp\left[X\left(t\right)\right]$ es acotado en intervalos finitos, entonces
\begin{eqnarray*}
\esp\left[X\left(t\right)\right]=U\star h\left(t\right)\textrm{,  donde }h\left(t\right)=\esp\left[X\left(t\right)\indora\left(T>t\right)\right].
\end{eqnarray*}
\end{Prop}

\begin{Teo}[Regeneraci\'on Cruda]
Sup\'ongase que $X\left(t\right)$ es un proceso con valores positivo crudamente regenerativo en $T$, y def\'inase $M=\sup\left\{|X\left(t\right)|:t\leq T\right\}$. Si $T$ es no aritm\'etico y $M$ y $MT$ tienen media finita, entonces
\begin{eqnarray*}
lim_{t\rightarrow\infty}\esp\left[X\left(t\right)\right]=\frac{1}{\mu}\int_{\rea_{+}}h\left(s\right)ds,
\end{eqnarray*}
donde $h\left(t\right)=\esp\left[X\left(t\right)\indora\left(T>t\right)\right]$.
\end{Teo}


\begin{Note} Una funci\'on $h:\rea_{+}\rightarrow\rea$ es Directamente Riemann Integrable en los siguientes casos:
\begin{itemize}
\item[a)] $h\left(t\right)\geq0$ es decreciente y Riemann Integrable.
\item[b)] $h$ es continua excepto posiblemente en un conjunto de Lebesgue de medida 0, y $|h\left(t\right)|\leq b\left(t\right)$, donde $b$ es DRI.
\end{itemize}
\end{Note}

\begin{Teo}[Teorema Principal de Renovaci\'on]
Si $F$ es no aritm\'etica y $h\left(t\right)$ es Directamente Riemann Integrable (DRI), entonces

\begin{eqnarray*}
lim_{t\rightarrow\infty}U\star h=\frac{1}{\mu}\int_{\rea_{+}}h\left(s\right)ds.
\end{eqnarray*}
\end{Teo}

\begin{Prop}
Cualquier funci\'on $H\left(t\right)$ acotada en intervalos finitos y que es 0 para $t<0$ puede expresarse como
\begin{eqnarray*}
H\left(t\right)=U\star h\left(t\right)\textrm{,  donde }h\left(t\right)=H\left(t\right)-F\star H\left(t\right)
\end{eqnarray*}
\end{Prop}

\begin{Def}
Un proceso estoc\'astico $X\left(t\right)$ es crudamente regenerativo en un tiempo aleatorio positivo $T$ si
\begin{eqnarray*}
\esp\left[X\left(T+t\right)|T\right]=\esp\left[X\left(t\right)\right]\textrm{, para }t\geq0,\end{eqnarray*}
y con las esperanzas anteriores finitas.
\end{Def}

\begin{Prop}
Sup\'ongase que $X\left(t\right)$ es un proceso crudamente regenerativo en $T$, que tiene distribuci\'on $F$. Si $\esp\left[X\left(t\right)\right]$ es acotado en intervalos finitos, entonces
\begin{eqnarray*}
\esp\left[X\left(t\right)\right]=U\star h\left(t\right)\textrm{,  donde }h\left(t\right)=\esp\left[X\left(t\right)\indora\left(T>t\right)\right].
\end{eqnarray*}
\end{Prop}

\begin{Teo}[Regeneraci\'on Cruda]
Sup\'ongase que $X\left(t\right)$ es un proceso con valores positivo crudamente regenerativo en $T$, y def\'inase $M=\sup\left\{|X\left(t\right)|:t\leq T\right\}$. Si $T$ es no aritm\'etico y $M$ y $MT$ tienen media finita, entonces
\begin{eqnarray*}
lim_{t\rightarrow\infty}\esp\left[X\left(t\right)\right]=\frac{1}{\mu}\int_{\rea_{+}}h\left(s\right)ds,
\end{eqnarray*}
donde $h\left(t\right)=\esp\left[X\left(t\right)\indora\left(T>t\right)\right]$.
\end{Teo}

\begin{Def}
Para el proceso $\left\{\left(N\left(t\right),X\left(t\right)\right):t\geq0\right\}$, sus trayectoria muestrales en el intervalo de tiempo $\left[T_{n-1},T_{n}\right)$ est\'an descritas por
\begin{eqnarray*}
\zeta_{n}=\left(\xi_{n},\left\{X\left(T_{n-1}+t\right):0\leq t<\xi_{n}\right\}\right)
\end{eqnarray*}
Este $\zeta_{n}$ es el $n$-\'esimo segmento del proceso. El proceso es regenerativo sobre los tiempos $T_{n}$ si sus segmentos $\zeta_{n}$ son independientes e id\'enticamennte distribuidos.
\end{Def}


\begin{Note}
Si $\tilde{X}\left(t\right)$ con espacio de estados $\tilde{S}$ es regenerativo sobre $T_{n}$, entonces $X\left(t\right)=f\left(\tilde{X}\left(t\right)\right)$ tambi\'en es regenerativo sobre $T_{n}$, para cualquier funci\'on $f:\tilde{S}\rightarrow S$.
\end{Note}

\begin{Note}
Los procesos regenerativos son crudamente regenerativos, pero no al rev\'es.
\end{Note}


\begin{Note}
Un proceso estoc\'astico a tiempo continuo o discreto es regenerativo si existe un proceso de renovaci\'on  tal que los segmentos del proceso entre tiempos de renovaci\'on sucesivos son i.i.d., es decir, para $\left\{X\left(t\right):t\geq0\right\}$ proceso estoc\'astico a tiempo continuo con espacio de estados $S$, espacio m\'etrico.
\end{Note}

Para $\left\{X\left(t\right):t\geq0\right\}$ Proceso Estoc\'astico a tiempo continuo con estado de espacios $S$, que es un espacio m\'etrico, con trayectorias continuas por la derecha y con l\'imites por la izquierda c.s. Sea $N\left(t\right)$ un proceso de renovaci\'on en $\rea_{+}$ definido en el mismo espacio de probabilidad que $X\left(t\right)$, con tiempos de renovaci\'on $T$ y tiempos de inter-renovaci\'on $\xi_{n}=T_{n}-T_{n-1}$, con misma distribuci\'on $F$ de media finita $\mu$.



\begin{Def}
Para el proceso $\left\{\left(N\left(t\right),X\left(t\right)\right):t\geq0\right\}$, sus trayectoria muestrales en el intervalo de tiempo $\left[T_{n-1},T_{n}\right)$ est\'an descritas por
\begin{eqnarray*}
\zeta_{n}=\left(\xi_{n},\left\{X\left(T_{n-1}+t\right):0\leq t<\xi_{n}\right\}\right)
\end{eqnarray*}
Este $\zeta_{n}$ es el $n$-\'esimo segmento del proceso. El proceso es regenerativo sobre los tiempos $T_{n}$ si sus segmentos $\zeta_{n}$ son independientes e id\'enticamennte distribuidos.
\end{Def}

\begin{Note}
Un proceso regenerativo con media de la longitud de ciclo finita es llamado positivo recurrente.
\end{Note}

\begin{Teo}[Procesos Regenerativos]
Suponga que el proceso
\end{Teo}


\begin{Def}[Renewal Process Trinity]
Para un proceso de renovaci\'on $N\left(t\right)$, los siguientes procesos proveen de informaci\'on sobre los tiempos de renovaci\'on.
\begin{itemize}
\item $A\left(t\right)=t-T_{N\left(t\right)}$, el tiempo de recurrencia hacia atr\'as al tiempo $t$, que es el tiempo desde la \'ultima renovaci\'on para $t$.

\item $B\left(t\right)=T_{N\left(t\right)+1}-t$, el tiempo de recurrencia hacia adelante al tiempo $t$, residual del tiempo de renovaci\'on, que es el tiempo para la pr\'oxima renovaci\'on despu\'es de $t$.

\item $L\left(t\right)=\xi_{N\left(t\right)+1}=A\left(t\right)+B\left(t\right)$, la longitud del intervalo de renovaci\'on que contiene a $t$.
\end{itemize}
\end{Def}

\begin{Note}
El proceso tridimensional $\left(A\left(t\right),B\left(t\right),L\left(t\right)\right)$ es regenerativo sobre $T_{n}$, y por ende cada proceso lo es. Cada proceso $A\left(t\right)$ y $B\left(t\right)$ son procesos de MArkov a tiempo continuo con trayectorias continuas por partes en el espacio de estados $\rea_{+}$. Una expresi\'on conveniente para su distribuci\'on conjunta es, para $0\leq x<t,y\geq0$
\begin{equation}\label{NoRenovacion}
P\left\{A\left(t\right)>x,B\left(t\right)>y\right\}=
P\left\{N\left(t+y\right)-N\left((t-x)\right)=0\right\}
\end{equation}
\end{Note}

\begin{Ejem}[Tiempos de recurrencia Poisson]
Si $N\left(t\right)$ es un proceso Poisson con tasa $\lambda$, entonces de la expresi\'on (\ref{NoRenovacion}) se tiene que

\begin{eqnarray*}
\begin{array}{lc}
P\left\{A\left(t\right)>x,B\left(t\right)>y\right\}=e^{-\lambda\left(x+y\right)},&0\leq x<t,y\geq0,
\end{array}
\end{eqnarray*}
que es la probabilidad Poisson de no renovaciones en un intervalo de longitud $x+y$.

\end{Ejem}

\begin{Note}
Una cadena de Markov erg\'odica tiene la propiedad de ser estacionaria si la distribuci\'on de su estado al tiempo $0$ es su distribuci\'on estacionaria.
\end{Note}


\begin{Def}
Un proceso estoc\'astico a tiempo continuo $\left\{X\left(t\right):t\geq0\right\}$ en un espacio general es estacionario si sus distribuciones finito dimensionales son invariantes bajo cualquier  traslado: para cada $0\leq s_{1}<s_{2}<\cdots<s_{k}$ y $t\geq0$,
\begin{eqnarray*}
\left(X\left(s_{1}+t\right),\ldots,X\left(s_{k}+t\right)\right)=_{d}\left(X\left(s_{1}\right),\ldots,X\left(s_{k}\right)\right).
\end{eqnarray*}
\end{Def}

\begin{Note}
Un proceso de Markov es estacionario si $X\left(t\right)=_{d}X\left(0\right)$, $t\geq0$.
\end{Note}

Considerese el proceso $N\left(t\right)=\sum_{n}\indora\left(\tau_{n}\leq t\right)$ en $\rea_{+}$, con puntos $0<\tau_{1}<\tau_{2}<\cdots$.

\begin{Prop}
Si $N$ es un proceso puntual estacionario y $\esp\left[N\left(1\right)\right]<\infty$, entonces $\esp\left[N\left(t\right)\right]=t\esp\left[N\left(1\right)\right]$, $t\geq0$

\end{Prop}

\begin{Teo}
Los siguientes enunciados son equivalentes
\begin{itemize}
\item[i)] El proceso retardado de renovaci\'on $N$ es estacionario.

\item[ii)] EL proceso de tiempos de recurrencia hacia adelante $B\left(t\right)$ es estacionario.


\item[iii)] $\esp\left[N\left(t\right)\right]=t/\mu$,


\item[iv)] $G\left(t\right)=F_{e}\left(t\right)=\frac{1}{\mu}\int_{0}^{t}\left[1-F\left(s\right)\right]ds$
\end{itemize}
Cuando estos enunciados son ciertos, $P\left\{B\left(t\right)\leq x\right\}=F_{e}\left(x\right)$, para $t,x\geq0$.

\end{Teo}

\begin{Note}
Una consecuencia del teorema anterior es que el Proceso Poisson es el \'unico proceso sin retardo que es estacionario.
\end{Note}

\begin{Coro}
El proceso de renovaci\'on $N\left(t\right)$ sin retardo, y cuyos tiempos de inter renonaci\'on tienen media finita, es estacionario si y s\'olo si es un proceso Poisson.

\end{Coro}

%______________________________________________________________________

%\section{Ejemplos, Notas importantes}
%______________________________________________________________________
%\section*{Ap\'endice A}
%__________________________________________________________________

%________________________________________________________________________
%\subsection*{Procesos Regenerativos}
%________________________________________________________________________



\begin{Note}
Si $\tilde{X}\left(t\right)$ con espacio de estados $\tilde{S}$ es regenerativo sobre $T_{n}$, entonces $X\left(t\right)=f\left(\tilde{X}\left(t\right)\right)$ tambi\'en es regenerativo sobre $T_{n}$, para cualquier funci\'on $f:\tilde{S}\rightarrow S$.
\end{Note}

\begin{Note}
Los procesos regenerativos son crudamente regenerativos, pero no al rev\'es.
\end{Note}
%\subsection*{Procesos Regenerativos: Sigman\cite{Sigman1}}
\begin{Def}[Definici\'on Cl\'asica]
Un proceso estoc\'astico $X=\left\{X\left(t\right):t\geq0\right\}$ es llamado regenerativo is existe una variable aleatoria $R_{1}>0$ tal que
\begin{itemize}
\item[i)] $\left\{X\left(t+R_{1}\right):t\geq0\right\}$ es independiente de $\left\{\left\{X\left(t\right):t<R_{1}\right\},\right\}$
\item[ii)] $\left\{X\left(t+R_{1}\right):t\geq0\right\}$ es estoc\'asticamente equivalente a $\left\{X\left(t\right):t>0\right\}$
\end{itemize}

Llamamos a $R_{1}$ tiempo de regeneraci\'on, y decimos que $X$ se regenera en este punto.
\end{Def}

$\left\{X\left(t+R_{1}\right)\right\}$ es regenerativo con tiempo de regeneraci\'on $R_{2}$, independiente de $R_{1}$ pero con la misma distribuci\'on que $R_{1}$. Procediendo de esta manera se obtiene una secuencia de variables aleatorias independientes e id\'enticamente distribuidas $\left\{R_{n}\right\}$ llamados longitudes de ciclo. Si definimos a $Z_{k}\equiv R_{1}+R_{2}+\cdots+R_{k}$, se tiene un proceso de renovaci\'on llamado proceso de renovaci\'on encajado para $X$.




\begin{Def}
Para $x$ fijo y para cada $t\geq0$, sea $I_{x}\left(t\right)=1$ si $X\left(t\right)\leq x$,  $I_{x}\left(t\right)=0$ en caso contrario, y def\'inanse los tiempos promedio
\begin{eqnarray*}
\overline{X}&=&lim_{t\rightarrow\infty}\frac{1}{t}\int_{0}^{\infty}X\left(u\right)du\\
\prob\left(X_{\infty}\leq x\right)&=&lim_{t\rightarrow\infty}\frac{1}{t}\int_{0}^{\infty}I_{x}\left(u\right)du,
\end{eqnarray*}
cuando estos l\'imites existan.
\end{Def}

Como consecuencia del teorema de Renovaci\'on-Recompensa, se tiene que el primer l\'imite  existe y es igual a la constante
\begin{eqnarray*}
\overline{X}&=&\frac{\esp\left[\int_{0}^{R_{1}}X\left(t\right)dt\right]}{\esp\left[R_{1}\right]},
\end{eqnarray*}
suponiendo que ambas esperanzas son finitas.

\begin{Note}
\begin{itemize}
\item[a)] Si el proceso regenerativo $X$ es positivo recurrente y tiene trayectorias muestrales no negativas, entonces la ecuaci\'on anterior es v\'alida.
\item[b)] Si $X$ es positivo recurrente regenerativo, podemos construir una \'unica versi\'on estacionaria de este proceso, $X_{e}=\left\{X_{e}\left(t\right)\right\}$, donde $X_{e}$ es un proceso estoc\'astico regenerativo y estrictamente estacionario, con distribuci\'on marginal distribuida como $X_{\infty}$
\end{itemize}
\end{Note}

Para $\left\{X\left(t\right):t\geq0\right\}$ Proceso Estoc\'astico a tiempo continuo con estado de espacios $S$, que es un espacio m\'etrico, con trayectorias continuas por la derecha y con l\'imites por la izquierda c.s. Sea $N\left(t\right)$ un proceso de renovaci\'on en $\rea_{+}$ definido en el mismo espacio de probabilidad que $X\left(t\right)$, con tiempos de renovaci\'on $T$ y tiempos de inter-renovaci\'on $\xi_{n}=T_{n}-T_{n-1}$, con misma distribuci\'on $F$ de media finita $\mu$.


\begin{Def}
Para el proceso $\left\{\left(N\left(t\right),X\left(t\right)\right):t\geq0\right\}$, sus trayectoria muestrales en el intervalo de tiempo $\left[T_{n-1},T_{n}\right)$ est\'an descritas por
\begin{eqnarray*}
\zeta_{n}=\left(\xi_{n},\left\{X\left(T_{n-1}+t\right):0\leq t<\xi_{n}\right\}\right)
\end{eqnarray*}
Este $\zeta_{n}$ es el $n$-\'esimo segmento del proceso. El proceso es regenerativo sobre los tiempos $T_{n}$ si sus segmentos $\zeta_{n}$ son independientes e id\'enticamennte distribuidos.
\end{Def}


\begin{Note}
Si $\tilde{X}\left(t\right)$ con espacio de estados $\tilde{S}$ es regenerativo sobre $T_{n}$, entonces $X\left(t\right)=f\left(\tilde{X}\left(t\right)\right)$ tambi\'en es regenerativo sobre $T_{n}$, para cualquier funci\'on $f:\tilde{S}\rightarrow S$.
\end{Note}

\begin{Note}
Los procesos regenerativos son crudamente regenerativos, pero no al rev\'es.
\end{Note}

\begin{Def}[Definici\'on Cl\'asica]
Un proceso estoc\'astico $X=\left\{X\left(t\right):t\geq0\right\}$ es llamado regenerativo is existe una variable aleatoria $R_{1}>0$ tal que
\begin{itemize}
\item[i)] $\left\{X\left(t+R_{1}\right):t\geq0\right\}$ es independiente de $\left\{\left\{X\left(t\right):t<R_{1}\right\},\right\}$
\item[ii)] $\left\{X\left(t+R_{1}\right):t\geq0\right\}$ es estoc\'asticamente equivalente a $\left\{X\left(t\right):t>0\right\}$
\end{itemize}

Llamamos a $R_{1}$ tiempo de regeneraci\'on, y decimos que $X$ se regenera en este punto.
\end{Def}

$\left\{X\left(t+R_{1}\right)\right\}$ es regenerativo con tiempo de regeneraci\'on $R_{2}$, independiente de $R_{1}$ pero con la misma distribuci\'on que $R_{1}$. Procediendo de esta manera se obtiene una secuencia de variables aleatorias independientes e id\'enticamente distribuidas $\left\{R_{n}\right\}$ llamados longitudes de ciclo. Si definimos a $Z_{k}\equiv R_{1}+R_{2}+\cdots+R_{k}$, se tiene un proceso de renovaci\'on llamado proceso de renovaci\'on encajado para $X$.

\begin{Note}
Un proceso regenerativo con media de la longitud de ciclo finita es llamado positivo recurrente.
\end{Note}


\begin{Def}
Para $x$ fijo y para cada $t\geq0$, sea $I_{x}\left(t\right)=1$ si $X\left(t\right)\leq x$,  $I_{x}\left(t\right)=0$ en caso contrario, y def\'inanse los tiempos promedio
\begin{eqnarray*}
\overline{X}&=&lim_{t\rightarrow\infty}\frac{1}{t}\int_{0}^{\infty}X\left(u\right)du\\
\prob\left(X_{\infty}\leq x\right)&=&lim_{t\rightarrow\infty}\frac{1}{t}\int_{0}^{\infty}I_{x}\left(u\right)du,
\end{eqnarray*}
cuando estos l\'imites existan.
\end{Def}

Como consecuencia del teorema de Renovaci\'on-Recompensa, se tiene que el primer l\'imite  existe y es igual a la constante
\begin{eqnarray*}
\overline{X}&=&\frac{\esp\left[\int_{0}^{R_{1}}X\left(t\right)dt\right]}{\esp\left[R_{1}\right]},
\end{eqnarray*}
suponiendo que ambas esperanzas son finitas.

\begin{Note}
\begin{itemize}
\item[a)] Si el proceso regenerativo $X$ es positivo recurrente y tiene trayectorias muestrales no negativas, entonces la ecuaci\'on anterior es v\'alida.
\item[b)] Si $X$ es positivo recurrente regenerativo, podemos construir una \'unica versi\'on estacionaria de este proceso, $X_{e}=\left\{X_{e}\left(t\right)\right\}$, donde $X_{e}$ es un proceso estoc\'astico regenerativo y estrictamente estacionario, con distribuci\'on marginal distribuida como $X_{\infty}$
\end{itemize}
\end{Note}

%__________________________________________________________________________________________
%\subsection{Procesos Regenerativos Estacionarios - Stidham \cite{Stidham}}
%__________________________________________________________________________________________


Un proceso estoc\'astico a tiempo continuo $\left\{V\left(t\right),t\geq0\right\}$ es un proceso regenerativo si existe una sucesi\'on de variables aleatorias independientes e id\'enticamente distribuidas $\left\{X_{1},X_{2},\ldots\right\}$, sucesi\'on de renovaci\'on, tal que para cualquier conjunto de Borel $A$, 

\begin{eqnarray*}
\prob\left\{V\left(t\right)\in A|X_{1}+X_{2}+\cdots+X_{R\left(t\right)}=s,\left\{V\left(\tau\right),\tau<s\right\}\right\}=\prob\left\{V\left(t-s\right)\in A|X_{1}>t-s\right\},
\end{eqnarray*}
para todo $0\leq s\leq t$, donde $R\left(t\right)=\max\left\{X_{1}+X_{2}+\cdots+X_{j}\leq t\right\}=$n\'umero de renovaciones ({\emph{puntos de regeneraci\'on}}) que ocurren en $\left[0,t\right]$. El intervalo $\left[0,X_{1}\right)$ es llamado {\emph{primer ciclo de regeneraci\'on}} de $\left\{V\left(t \right),t\geq0\right\}$, $\left[X_{1},X_{1}+X_{2}\right)$ el {\emph{segundo ciclo de regeneraci\'on}}, y as\'i sucesivamente.

Sea $X=X_{1}$ y sea $F$ la funci\'on de distrbuci\'on de $X$


\begin{Def}
Se define el proceso estacionario, $\left\{V^{*}\left(t\right),t\geq0\right\}$, para $\left\{V\left(t\right),t\geq0\right\}$ por

\begin{eqnarray*}
\prob\left\{V\left(t\right)\in A\right\}=\frac{1}{\esp\left[X\right]}\int_{0}^{\infty}\prob\left\{V\left(t+x\right)\in A|X>x\right\}\left(1-F\left(x\right)\right)dx,
\end{eqnarray*} 
para todo $t\geq0$ y todo conjunto de Borel $A$.
\end{Def}

\begin{Def}
Una distribuci\'on se dice que es {\emph{aritm\'etica}} si todos sus puntos de incremento son m\'ultiplos de la forma $0,\lambda, 2\lambda,\ldots$ para alguna $\lambda>0$ entera.
\end{Def}


\begin{Def}
Una modificaci\'on medible de un proceso $\left\{V\left(t\right),t\geq0\right\}$, es una versi\'on de este, $\left\{V\left(t,w\right)\right\}$ conjuntamente medible para $t\geq0$ y para $w\in S$, $S$ espacio de estados para $\left\{V\left(t\right),t\geq0\right\}$.
\end{Def}

\begin{Teo}
Sea $\left\{V\left(t\right),t\geq\right\}$ un proceso regenerativo no negativo con modificaci\'on medible. Sea $\esp\left[X\right]<\infty$. Entonces el proceso estacionario dado por la ecuaci\'on anterior est\'a bien definido y tiene funci\'on de distribuci\'on independiente de $t$, adem\'as
\begin{itemize}
\item[i)] \begin{eqnarray*}
\esp\left[V^{*}\left(0\right)\right]&=&\frac{\esp\left[\int_{0}^{X}V\left(s\right)ds\right]}{\esp\left[X\right]}\end{eqnarray*}
\item[ii)] Si $\esp\left[V^{*}\left(0\right)\right]<\infty$, equivalentemente, si $\esp\left[\int_{0}^{X}V\left(s\right)ds\right]<\infty$,entonces
\begin{eqnarray*}
\frac{\int_{0}^{t}V\left(s\right)ds}{t}\rightarrow\frac{\esp\left[\int_{0}^{X}V\left(s\right)ds\right]}{\esp\left[X\right]}
\end{eqnarray*}
con probabilidad 1 y en media, cuando $t\rightarrow\infty$.
\end{itemize}
\end{Teo}



%_______________________________________________________________________________________
\subsection*{Procesos Regenerativos: Sigman\cite{Sigman1}}
\begin{Def}[Definici\'on Cl\'asica]
Un proceso estoc\'astico $X=\left\{X\left(t\right):t\geq0\right\}$ es llamado regenerativo is existe una variable aleatoria $R_{1}>0$ tal que
\begin{itemize}
\item[i)] $\left\{X\left(t+R_{1}\right):t\geq0\right\}$ es independiente de $\left\{\left\{X\left(t\right):t<R_{1}\right\},\right\}$
\item[ii)] $\left\{X\left(t+R_{1}\right):t\geq0\right\}$ es estoc\'asticamente equivalente a $\left\{X\left(t\right):t>0\right\}$
\end{itemize}

Llamamos a $R_{1}$ tiempo de regeneraci\'on, y decimos que $X$ se regenera en este punto.
\end{Def}

$\left\{X\left(t+R_{1}\right)\right\}$ es regenerativo con tiempo de regeneraci\'on $R_{2}$, independiente de $R_{1}$ pero con la misma distribuci\'on que $R_{1}$. Procediendo de esta manera se obtiene una secuencia de variables aleatorias independientes e id\'enticamente distribuidas $\left\{R_{n}\right\}$ llamados longitudes de ciclo. Si definimos a $Z_{k}\equiv R_{1}+R_{2}+\cdots+R_{k}$, se tiene un proceso de renovaci\'on llamado proceso de renovaci\'on encajado para $X$.




\begin{Def}
Para $x$ fijo y para cada $t\geq0$, sea $I_{x}\left(t\right)=1$ si $X\left(t\right)\leq x$,  $I_{x}\left(t\right)=0$ en caso contrario, y def\'inanse los tiempos promedio
\begin{eqnarray*}
\overline{X}&=&lim_{t\rightarrow\infty}\frac{1}{t}\int_{0}^{\infty}X\left(u\right)du\\
\prob\left(X_{\infty}\leq x\right)&=&lim_{t\rightarrow\infty}\frac{1}{t}\int_{0}^{\infty}I_{x}\left(u\right)du,
\end{eqnarray*}
cuando estos l\'imites existan.
\end{Def}

Como consecuencia del teorema de Renovaci\'on-Recompensa, se tiene que el primer l\'imite  existe y es igual a la constante
\begin{eqnarray*}
\overline{X}&=&\frac{\esp\left[\int_{0}^{R_{1}}X\left(t\right)dt\right]}{\esp\left[R_{1}\right]},
\end{eqnarray*}
suponiendo que ambas esperanzas son finitas.

\begin{Note}
\begin{itemize}
\item[a)] Si el proceso regenerativo $X$ es positivo recurrente y tiene trayectorias muestrales no negativas, entonces la ecuaci\'on anterior es v\'alida.
\item[b)] Si $X$ es positivo recurrente regenerativo, podemos construir una \'unica versi\'on estacionaria de este proceso, $X_{e}=\left\{X_{e}\left(t\right)\right\}$, donde $X_{e}$ es un proceso estoc\'astico regenerativo y estrictamente estacionario, con distribuci\'on marginal distribuida como $X_{\infty}$
\end{itemize}
\end{Note}


%______________________________________________________________________
\subsection{Procesos de Renovaci\'on}
%______________________________________________________________________

\begin{Def}\label{Def.Tn}
Sean $0\leq T_{1}\leq T_{2}\leq \ldots$ son tiempos aleatorios infinitos en los cuales ocurren ciertos eventos. El n\'umero de tiempos $T_{n}$ en el intervalo $\left[0,t\right)$ es

\begin{eqnarray}
N\left(t\right)=\sum_{n=1}^{\infty}\indora\left(T_{n}\leq t\right),
\end{eqnarray}
para $t\geq0$.
\end{Def}

Si se consideran los puntos $T_{n}$ como elementos de $\rea_{+}$, y $N\left(t\right)$ es el n\'umero de puntos en $\rea$. El proceso denotado por $\left\{N\left(t\right):t\geq0\right\}$, denotado por $N\left(t\right)$, es un proceso puntual en $\rea_{+}$. Los $T_{n}$ son los tiempos de ocurrencia, el proceso puntual $N\left(t\right)$ es simple si su n\'umero de ocurrencias son distintas: $0<T_{1}<T_{2}<\ldots$ casi seguramente.

\begin{Def}
Un proceso puntual $N\left(t\right)$ es un proceso de renovaci\'on si los tiempos de interocurrencia $\xi_{n}=T_{n}-T_{n-1}$, para $n\geq1$, son independientes e identicamente distribuidos con distribuci\'on $F$, donde $F\left(0\right)=0$ y $T_{0}=0$. Los $T_{n}$ son llamados tiempos de renovaci\'on, referente a la independencia o renovaci\'on de la informaci\'on estoc\'astica en estos tiempos. Los $\xi_{n}$ son los tiempos de inter-renovaci\'on, y $N\left(t\right)$ es el n\'umero de renovaciones en el intervalo $\left[0,t\right)$
\end{Def}


\begin{Note}
Para definir un proceso de renovaci\'on para cualquier contexto, solamente hay que especificar una distribuci\'on $F$, con $F\left(0\right)=0$, para los tiempos de inter-renovaci\'on. La funci\'on $F$ en turno degune las otra variables aleatorias. De manera formal, existe un espacio de probabilidad y una sucesi\'on de variables aleatorias $\xi_{1},\xi_{2},\ldots$ definidas en este con distribuci\'on $F$. Entonces las otras cantidades son $T_{n}=\sum_{k=1}^{n}\xi_{k}$ y $N\left(t\right)=\sum_{n=1}^{\infty}\indora\left(T_{n}\leq t\right)$, donde $T_{n}\rightarrow\infty$ casi seguramente por la Ley Fuerte de los Grandes Números.
\end{Note}



%______________________________________________________________________
\subsection{Procesos de Renovaci\'on}
%______________________________________________________________________

\begin{Def}\label{Def.Tn}
Sean $0\leq T_{1}\leq T_{2}\leq \ldots$ son tiempos aleatorios infinitos en los cuales ocurren ciertos eventos. El n\'umero de tiempos $T_{n}$ en el intervalo $\left[0,t\right)$ es

\begin{eqnarray}
N\left(t\right)=\sum_{n=1}^{\infty}\indora\left(T_{n}\leq t\right),
\end{eqnarray}
para $t\geq0$.
\end{Def}

Si se consideran los puntos $T_{n}$ como elementos de $\rea_{+}$, y $N\left(t\right)$ es el n\'umero de puntos en $\rea$. El proceso denotado por $\left\{N\left(t\right):t\geq0\right\}$, denotado por $N\left(t\right)$, es un proceso puntual en $\rea_{+}$. Los $T_{n}$ son los tiempos de ocurrencia, el proceso puntual $N\left(t\right)$ es simple si su n\'umero de ocurrencias son distintas: $0<T_{1}<T_{2}<\ldots$ casi seguramente.

\begin{Def}
Un proceso puntual $N\left(t\right)$ es un proceso de renovaci\'on si los tiempos de interocurrencia $\xi_{n}=T_{n}-T_{n-1}$, para $n\geq1$, son independientes e identicamente distribuidos con distribuci\'on $F$, donde $F\left(0\right)=0$ y $T_{0}=0$. Los $T_{n}$ son llamados tiempos de renovaci\'on, referente a la independencia o renovaci\'on de la informaci\'on estoc\'astica en estos tiempos. Los $\xi_{n}$ son los tiempos de inter-renovaci\'on, y $N\left(t\right)$ es el n\'umero de renovaciones en el intervalo $\left[0,t\right)$
\end{Def}


\begin{Note}
Para definir un proceso de renovaci\'on para cualquier contexto, solamente hay que especificar una distribuci\'on $F$, con $F\left(0\right)=0$, para los tiempos de inter-renovaci\'on. La funci\'on $F$ en turno degune las otra variables aleatorias. De manera formal, existe un espacio de probabilidad y una sucesi\'on de variables aleatorias $\xi_{1},\xi_{2},\ldots$ definidas en este con distribuci\'on $F$. Entonces las otras cantidades son $T_{n}=\sum_{k=1}^{n}\xi_{k}$ y $N\left(t\right)=\sum_{n=1}^{\infty}\indora\left(T_{n}\leq t\right)$, donde $T_{n}\rightarrow\infty$ casi seguramente por la Ley Fuerte de los Grandes Números.
\end{Note}

%______________________________________________________________________
\subsection{Procesos de Renovaci\'on}
%______________________________________________________________________

\begin{Def}%\label{Def.Tn}
Sean $0\leq T_{1}\leq T_{2}\leq \ldots$ son tiempos aleatorios infinitos en los cuales ocurren ciertos eventos. El n\'umero de tiempos $T_{n}$ en el intervalo $\left[0,t\right)$ es

\begin{eqnarray}
N\left(t\right)=\sum_{n=1}^{\infty}\indora\left(T_{n}\leq t\right),
\end{eqnarray}
para $t\geq0$.
\end{Def}

Si se consideran los puntos $T_{n}$ como elementos de $\rea_{+}$, y $N\left(t\right)$ es el n\'umero de puntos en $\rea$. El proceso denotado por $\left\{N\left(t\right):t\geq0\right\}$, denotado por $N\left(t\right)$, es un proceso puntual en $\rea_{+}$. Los $T_{n}$ son los tiempos de ocurrencia, el proceso puntual $N\left(t\right)$ es simple si su n\'umero de ocurrencias son distintas: $0<T_{1}<T_{2}<\ldots$ casi seguramente.

\begin{Def}
Un proceso puntual $N\left(t\right)$ es un proceso de renovaci\'on si los tiempos de interocurrencia $\xi_{n}=T_{n}-T_{n-1}$, para $n\geq1$, son independientes e identicamente distribuidos con distribuci\'on $F$, donde $F\left(0\right)=0$ y $T_{0}=0$. Los $T_{n}$ son llamados tiempos de renovaci\'on, referente a la independencia o renovaci\'on de la informaci\'on estoc\'astica en estos tiempos. Los $\xi_{n}$ son los tiempos de inter-renovaci\'on, y $N\left(t\right)$ es el n\'umero de renovaciones en el intervalo $\left[0,t\right)$
\end{Def}


\begin{Note}
Para definir un proceso de renovaci\'on para cualquier contexto, solamente hay que especificar una distribuci\'on $F$, con $F\left(0\right)=0$, para los tiempos de inter-renovaci\'on. La funci\'on $F$ en turno degune las otra variables aleatorias. De manera formal, existe un espacio de probabilidad y una sucesi\'on de variables aleatorias $\xi_{1},\xi_{2},\ldots$ definidas en este con distribuci\'on $F$. Entonces las otras cantidades son $T_{n}=\sum_{k=1}^{n}\xi_{k}$ y $N\left(t\right)=\sum_{n=1}^{\infty}\indora\left(T_{n}\leq t\right)$, donde $T_{n}\rightarrow\infty$ casi seguramente por la Ley Fuerte de los Grandes Números.
\end{Note}
%______________________________________________________________________
\subsection{Procesos de Renovaci\'on}
%______________________________________________________________________

\begin{Def}\label{Def.Tn}
Sean $0\leq T_{1}\leq T_{2}\leq \ldots$ son tiempos aleatorios infinitos en los cuales ocurren ciertos eventos. El n\'umero de tiempos $T_{n}$ en el intervalo $\left[0,t\right)$ es

\begin{eqnarray}
N\left(t\right)=\sum_{n=1}^{\infty}\indora\left(T_{n}\leq t\right),
\end{eqnarray}
para $t\geq0$.
\end{Def}

Si se consideran los puntos $T_{n}$ como elementos de $\rea_{+}$, y $N\left(t\right)$ es el n\'umero de puntos en $\rea$. El proceso denotado por $\left\{N\left(t\right):t\geq0\right\}$, denotado por $N\left(t\right)$, es un proceso puntual en $\rea_{+}$. Los $T_{n}$ son los tiempos de ocurrencia, el proceso puntual $N\left(t\right)$ es simple si su n\'umero de ocurrencias son distintas: $0<T_{1}<T_{2}<\ldots$ casi seguramente.

\begin{Def}
Un proceso puntual $N\left(t\right)$ es un proceso de renovaci\'on si los tiempos de interocurrencia $\xi_{n}=T_{n}-T_{n-1}$, para $n\geq1$, son independientes e identicamente distribuidos con distribuci\'on $F$, donde $F\left(0\right)=0$ y $T_{0}=0$. Los $T_{n}$ son llamados tiempos de renovaci\'on, referente a la independencia o renovaci\'on de la informaci\'on estoc\'astica en estos tiempos. Los $\xi_{n}$ son los tiempos de inter-renovaci\'on, y $N\left(t\right)$ es el n\'umero de renovaciones en el intervalo $\left[0,t\right)$
\end{Def}


\begin{Note}
Para definir un proceso de renovaci\'on para cualquier contexto, solamente hay que especificar una distribuci\'on $F$, con $F\left(0\right)=0$, para los tiempos de inter-renovaci\'on. La funci\'on $F$ en turno degune las otra variables aleatorias. De manera formal, existe un espacio de probabilidad y una sucesi\'on de variables aleatorias $\xi_{1},\xi_{2},\ldots$ definidas en este con distribuci\'on $F$. Entonces las otras cantidades son $T_{n}=\sum_{k=1}^{n}\xi_{k}$ y $N\left(t\right)=\sum_{n=1}^{\infty}\indora\left(T_{n}\leq t\right)$, donde $T_{n}\rightarrow\infty$ casi seguramente por la Ley Fuerte de los Grandes Números.
\end{Note}

%______________________________________________________________________
\subsection{Procesos de Renovaci\'on}
%______________________________________________________________________

\begin{Def}%\label{Def.Tn}
Sean $0\leq T_{1}\leq T_{2}\leq \ldots$ son tiempos aleatorios infinitos en los cuales ocurren ciertos eventos. El n\'umero de tiempos $T_{n}$ en el intervalo $\left[0,t\right)$ es

\begin{eqnarray}
N\left(t\right)=\sum_{n=1}^{\infty}\indora\left(T_{n}\leq t\right),
\end{eqnarray}
para $t\geq0$.
\end{Def}

Si se consideran los puntos $T_{n}$ como elementos de $\rea_{+}$, y $N\left(t\right)$ es el n\'umero de puntos en $\rea$. El proceso denotado por $\left\{N\left(t\right):t\geq0\right\}$, denotado por $N\left(t\right)$, es un proceso puntual en $\rea_{+}$. Los $T_{n}$ son los tiempos de ocurrencia, el proceso puntual $N\left(t\right)$ es simple si su n\'umero de ocurrencias son distintas: $0<T_{1}<T_{2}<\ldots$ casi seguramente.

\begin{Def}
Un proceso puntual $N\left(t\right)$ es un proceso de renovaci\'on si los tiempos de interocurrencia $\xi_{n}=T_{n}-T_{n-1}$, para $n\geq1$, son independientes e identicamente distribuidos con distribuci\'on $F$, donde $F\left(0\right)=0$ y $T_{0}=0$. Los $T_{n}$ son llamados tiempos de renovaci\'on, referente a la independencia o renovaci\'on de la informaci\'on estoc\'astica en estos tiempos. Los $\xi_{n}$ son los tiempos de inter-renovaci\'on, y $N\left(t\right)$ es el n\'umero de renovaciones en el intervalo $\left[0,t\right)$
\end{Def}


\begin{Note}
Para definir un proceso de renovaci\'on para cualquier contexto, solamente hay que especificar una distribuci\'on $F$, con $F\left(0\right)=0$, para los tiempos de inter-renovaci\'on. La funci\'on $F$ en turno degune las otra variables aleatorias. De manera formal, existe un espacio de probabilidad y una sucesi\'on de variables aleatorias $\xi_{1},\xi_{2},\ldots$ definidas en este con distribuci\'on $F$. Entonces las otras cantidades son $T_{n}=\sum_{k=1}^{n}\xi_{k}$ y $N\left(t\right)=\sum_{n=1}^{\infty}\indora\left(T_{n}\leq t\right)$, donde $T_{n}\rightarrow\infty$ casi seguramente por la Ley Fuerte de los Grandes Números.
\end{Note}
%______________________________________________________________________
\subsection{Procesos Regenerativos Estacionarios: Visi\'on cl\'asica}
%______________________________________________________________________

\begin{Def}\label{Def.Tn}
Sean $0\leq T_{1}\leq T_{2}\leq \ldots$ son tiempos aleatorios infinitos en los cuales ocurren ciertos eventos. El n\'umero de tiempos $T_{n}$ en el intervalo $\left[0,t\right)$ es

\begin{eqnarray}
N\left(t\right)=\sum_{n=1}^{\infty}\indora\left(T_{n}\leq t\right),
\end{eqnarray}
para $t\geq0$.
\end{Def}

Si se consideran los puntos $T_{n}$ como elementos de $\rea_{+}$, y $N\left(t\right)$ es el n\'umero de puntos en $\rea$. El proceso denotado por $\left\{N\left(t\right):t\geq0\right\}$, denotado por $N\left(t\right)$, es un proceso puntual en $\rea_{+}$. Los $T_{n}$ son los tiempos de ocurrencia, el proceso puntual $N\left(t\right)$ es simple si su n\'umero de ocurrencias son distintas: $0<T_{1}<T_{2}<\ldots$ casi seguramente.

\begin{Def}
Un proceso puntual $N\left(t\right)$ es un proceso de renovaci\'on si los tiempos de interocurrencia $\xi_{n}=T_{n}-T_{n-1}$, para $n\geq1$, son independientes e identicamente distribuidos con distribuci\'on $F$, donde $F\left(0\right)=0$ y $T_{0}=0$. Los $T_{n}$ son llamados tiempos de renovaci\'on, referente a la independencia o renovaci\'on de la informaci\'on estoc\'astica en estos tiempos. Los $\xi_{n}$ son los tiempos de inter-renovaci\'on, y $N\left(t\right)$ es el n\'umero de renovaciones en el intervalo $\left[0,t\right)$
\end{Def}


\begin{Note}
Para definir un proceso de renovaci\'on para cualquier contexto, solamente hay que especificar una distribuci\'on $F$, con $F\left(0\right)=0$, para los tiempos de inter-renovaci\'on. La funci\'on $F$ en turno degune las otra variables aleatorias. De manera formal, existe un espacio de probabilidad y una sucesi\'on de variables aleatorias $\xi_{1},\xi_{2},\ldots$ definidas en este con distribuci\'on $F$. Entonces las otras cantidades son $T_{n}=\sum_{k=1}^{n}\xi_{k}$ y $N\left(t\right)=\sum_{n=1}^{\infty}\indora\left(T_{n}\leq t\right)$, donde $T_{n}\rightarrow\infty$ casi seguramente por la Ley Fuerte de los Grandes Números.
\end{Note}

%___________________________________________________________________________________________
%
\subsection{Teorema Principal de Renovaci\'on}
%___________________________________________________________________________________________
%

\begin{Note} Una funci\'on $h:\rea_{+}\rightarrow\rea$ es Directamente Riemann Integrable en los siguientes casos:
\begin{itemize}
\item[a)] $h\left(t\right)\geq0$ es decreciente y Riemann Integrable.
\item[b)] $h$ es continua excepto posiblemente en un conjunto de Lebesgue de medida 0, y $|h\left(t\right)|\leq b\left(t\right)$, donde $b$ es DRI.
\end{itemize}
\end{Note}

\begin{Teo}[Teorema Principal de Renovaci\'on]
Si $F$ es no aritm\'etica y $h\left(t\right)$ es Directamente Riemann Integrable (DRI), entonces

\begin{eqnarray*}
lim_{t\rightarrow\infty}U\star h=\frac{1}{\mu}\int_{\rea_{+}}h\left(s\right)ds.
\end{eqnarray*}
\end{Teo}

\begin{Prop}
Cualquier funci\'on $H\left(t\right)$ acotada en intervalos finitos y que es 0 para $t<0$ puede expresarse como
\begin{eqnarray*}
H\left(t\right)=U\star h\left(t\right)\textrm{,  donde }h\left(t\right)=H\left(t\right)-F\star H\left(t\right)
\end{eqnarray*}
\end{Prop}

\begin{Def}
Un proceso estoc\'astico $X\left(t\right)$ es crudamente regenerativo en un tiempo aleatorio positivo $T$ si
\begin{eqnarray*}
\esp\left[X\left(T+t\right)|T\right]=\esp\left[X\left(t\right)\right]\textrm{, para }t\geq0,\end{eqnarray*}
y con las esperanzas anteriores finitas.
\end{Def}

\begin{Prop}
Sup\'ongase que $X\left(t\right)$ es un proceso crudamente regenerativo en $T$, que tiene distribuci\'on $F$. Si $\esp\left[X\left(t\right)\right]$ es acotado en intervalos finitos, entonces
\begin{eqnarray*}
\esp\left[X\left(t\right)\right]=U\star h\left(t\right)\textrm{,  donde }h\left(t\right)=\esp\left[X\left(t\right)\indora\left(T>t\right)\right].
\end{eqnarray*}
\end{Prop}

\begin{Teo}[Regeneraci\'on Cruda]
Sup\'ongase que $X\left(t\right)$ es un proceso con valores positivo crudamente regenerativo en $T$, y def\'inase $M=\sup\left\{|X\left(t\right)|:t\leq T\right\}$. Si $T$ es no aritm\'etico y $M$ y $MT$ tienen media finita, entonces
\begin{eqnarray*}
lim_{t\rightarrow\infty}\esp\left[X\left(t\right)\right]=\frac{1}{\mu}\int_{\rea_{+}}h\left(s\right)ds,
\end{eqnarray*}
donde $h\left(t\right)=\esp\left[X\left(t\right)\indora\left(T>t\right)\right]$.
\end{Teo}

%___________________________________________________________________________________________
%
\subsection{Propiedades de los Procesos de Renovaci\'on}
%___________________________________________________________________________________________
%

Los tiempos $T_{n}$ est\'an relacionados con los conteos de $N\left(t\right)$ por

\begin{eqnarray*}
\left\{N\left(t\right)\geq n\right\}&=&\left\{T_{n}\leq t\right\}\\
T_{N\left(t\right)}\leq &t&<T_{N\left(t\right)+1},
\end{eqnarray*}

adem\'as $N\left(T_{n}\right)=n$, y 

\begin{eqnarray*}
N\left(t\right)=\max\left\{n:T_{n}\leq t\right\}=\min\left\{n:T_{n+1}>t\right\}
\end{eqnarray*}

Por propiedades de la convoluci\'on se sabe que

\begin{eqnarray*}
P\left\{T_{n}\leq t\right\}=F^{n\star}\left(t\right)
\end{eqnarray*}
que es la $n$-\'esima convoluci\'on de $F$. Entonces 

\begin{eqnarray*}
\left\{N\left(t\right)\geq n\right\}&=&\left\{T_{n}\leq t\right\}\\
P\left\{N\left(t\right)\leq n\right\}&=&1-F^{\left(n+1\right)\star}\left(t\right)
\end{eqnarray*}

Adem\'as usando el hecho de que $\esp\left[N\left(t\right)\right]=\sum_{n=1}^{\infty}P\left\{N\left(t\right)\geq n\right\}$
se tiene que

\begin{eqnarray*}
\esp\left[N\left(t\right)\right]=\sum_{n=1}^{\infty}F^{n\star}\left(t\right)
\end{eqnarray*}

\begin{Prop}
Para cada $t\geq0$, la funci\'on generadora de momentos $\esp\left[e^{\alpha N\left(t\right)}\right]$ existe para alguna $\alpha$ en una vecindad del 0, y de aqu\'i que $\esp\left[N\left(t\right)^{m}\right]<\infty$, para $m\geq1$.
\end{Prop}


\begin{Note}
Si el primer tiempo de renovaci\'on $\xi_{1}$ no tiene la misma distribuci\'on que el resto de las $\xi_{n}$, para $n\geq2$, a $N\left(t\right)$ se le llama Proceso de Renovaci\'on retardado, donde si $\xi$ tiene distribuci\'on $G$, entonces el tiempo $T_{n}$ de la $n$-\'esima renovaci\'on tiene distribuci\'on $G\star F^{\left(n-1\right)\star}\left(t\right)$
\end{Note}


\begin{Teo}
Para una constante $\mu\leq\infty$ ( o variable aleatoria), las siguientes expresiones son equivalentes:

\begin{eqnarray}
lim_{n\rightarrow\infty}n^{-1}T_{n}&=&\mu,\textrm{ c.s.}\\
lim_{t\rightarrow\infty}t^{-1}N\left(t\right)&=&1/\mu,\textrm{ c.s.}
\end{eqnarray}
\end{Teo}


Es decir, $T_{n}$ satisface la Ley Fuerte de los Grandes N\'umeros s\'i y s\'olo s\'i $N\left/t\right)$ la cumple.


\begin{Coro}[Ley Fuerte de los Grandes N\'umeros para Procesos de Renovaci\'on]
Si $N\left(t\right)$ es un proceso de renovaci\'on cuyos tiempos de inter-renovaci\'on tienen media $\mu\leq\infty$, entonces
\begin{eqnarray}
t^{-1}N\left(t\right)\rightarrow 1/\mu,\textrm{ c.s. cuando }t\rightarrow\infty.
\end{eqnarray}

\end{Coro}


Considerar el proceso estoc\'astico de valores reales $\left\{Z\left(t\right):t\geq0\right\}$ en el mismo espacio de probabilidad que $N\left(t\right)$

\begin{Def}
Para el proceso $\left\{Z\left(t\right):t\geq0\right\}$ se define la fluctuaci\'on m\'axima de $Z\left(t\right)$ en el intervalo $\left(T_{n-1},T_{n}\right]$:
\begin{eqnarray*}
M_{n}=\sup_{T_{n-1}<t\leq T_{n}}|Z\left(t\right)-Z\left(T_{n-1}\right)|
\end{eqnarray*}
\end{Def}

\begin{Teo}
Sup\'ongase que $n^{-1}T_{n}\rightarrow\mu$ c.s. cuando $n\rightarrow\infty$, donde $\mu\leq\infty$ es una constante o variable aleatoria. Sea $a$ una constante o variable aleatoria que puede ser infinita cuando $\mu$ es finita, y considere las expresiones l\'imite:
\begin{eqnarray}
lim_{n\rightarrow\infty}n^{-1}Z\left(T_{n}\right)&=&a,\textrm{ c.s.}\\
lim_{t\rightarrow\infty}t^{-1}Z\left(t\right)&=&a/\mu,\textrm{ c.s.}
\end{eqnarray}
La segunda expresi\'on implica la primera. Conversamente, la primera implica la segunda si el proceso $Z\left(t\right)$ es creciente, o si $lim_{n\rightarrow\infty}n^{-1}M_{n}=0$ c.s.
\end{Teo}

\begin{Coro}
Si $N\left(t\right)$ es un proceso de renovaci\'on, y $\left(Z\left(T_{n}\right)-Z\left(T_{n-1}\right),M_{n}\right)$, para $n\geq1$, son variables aleatorias independientes e id\'enticamente distribuidas con media finita, entonces,
\begin{eqnarray}
lim_{t\rightarrow\infty}t^{-1}Z\left(t\right)\rightarrow\frac{\esp\left[Z\left(T_{1}\right)-Z\left(T_{0}\right)\right]}{\esp\left[T_{1}\right]},\textrm{ c.s. cuando  }t\rightarrow\infty.
\end{eqnarray}
\end{Coro}

%___________________________________________________________________________________________
%
\subsection{Funci\'on de Renovaci\'on}
%___________________________________________________________________________________________
%


\begin{Def}
Sea $h\left(t\right)$ funci\'on de valores reales en $\rea$ acotada en intervalos finitos e igual a cero para $t<0$ La ecuaci\'on de renovaci\'on para $h\left(t\right)$ y la distribuci\'on $F$ es

\begin{eqnarray}\label{Ec.Renovacion}
H\left(t\right)=h\left(t\right)+\int_{\left[0,t\right]}H\left(t-s\right)dF\left(s\right)\textrm{,    }t\geq0,
\end{eqnarray}
donde $H\left(t\right)$ es una funci\'on de valores reales. Esto es $H=h+F\star H$. Decimos que $H\left(t\right)$ es soluci\'on de esta ecuaci\'on si satisface la ecuaci\'on, y es acotada en intervalos finitos e iguales a cero para $t<0$.
\end{Def}

\begin{Prop}
La funci\'on $U\star h\left(t\right)$ es la \'unica soluci\'on de la ecuaci\'on de renovaci\'on (\ref{Ec.Renovacion}).
\end{Prop}

\begin{Teo}[Teorema Renovaci\'on Elemental]
\begin{eqnarray*}
t^{-1}U\left(t\right)\rightarrow 1/\mu\textrm{,    cuando }t\rightarrow\infty.
\end{eqnarray*}
\end{Teo}

%______________________________________________________________________
\subsection{Procesos de Renovaci\'on}
%______________________________________________________________________

\begin{Def}\label{Def.Tn}
Sean $0\leq T_{1}\leq T_{2}\leq \ldots$ son tiempos aleatorios infinitos en los cuales ocurren ciertos eventos. El n\'umero de tiempos $T_{n}$ en el intervalo $\left[0,t\right)$ es

\begin{eqnarray}
N\left(t\right)=\sum_{n=1}^{\infty}\indora\left(T_{n}\leq t\right),
\end{eqnarray}
para $t\geq0$.
\end{Def}

Si se consideran los puntos $T_{n}$ como elementos de $\rea_{+}$, y $N\left(t\right)$ es el n\'umero de puntos en $\rea$. El proceso denotado por $\left\{N\left(t\right):t\geq0\right\}$, denotado por $N\left(t\right)$, es un proceso puntual en $\rea_{+}$. Los $T_{n}$ son los tiempos de ocurrencia, el proceso puntual $N\left(t\right)$ es simple si su n\'umero de ocurrencias son distintas: $0<T_{1}<T_{2}<\ldots$ casi seguramente.

\begin{Def}
Un proceso puntual $N\left(t\right)$ es un proceso de renovaci\'on si los tiempos de interocurrencia $\xi_{n}=T_{n}-T_{n-1}$, para $n\geq1$, son independientes e identicamente distribuidos con distribuci\'on $F$, donde $F\left(0\right)=0$ y $T_{0}=0$. Los $T_{n}$ son llamados tiempos de renovaci\'on, referente a la independencia o renovaci\'on de la informaci\'on estoc\'astica en estos tiempos. Los $\xi_{n}$ son los tiempos de inter-renovaci\'on, y $N\left(t\right)$ es el n\'umero de renovaciones en el intervalo $\left[0,t\right)$
\end{Def}


\begin{Note}
Para definir un proceso de renovaci\'on para cualquier contexto, solamente hay que especificar una distribuci\'on $F$, con $F\left(0\right)=0$, para los tiempos de inter-renovaci\'on. La funci\'on $F$ en turno degune las otra variables aleatorias. De manera formal, existe un espacio de probabilidad y una sucesi\'on de variables aleatorias $\xi_{1},\xi_{2},\ldots$ definidas en este con distribuci\'on $F$. Entonces las otras cantidades son $T_{n}=\sum_{k=1}^{n}\xi_{k}$ y $N\left(t\right)=\sum_{n=1}^{\infty}\indora\left(T_{n}\leq t\right)$, donde $T_{n}\rightarrow\infty$ casi seguramente por la Ley Fuerte de los Grandes Números.
\end{Note}

%___________________________________________________________________________________________
%
\subsection{Renewal and Regenerative Processes: Serfozo\cite{Serfozo}}
%___________________________________________________________________________________________
%
\begin{Def}\label{Def.Tn}
Sean $0\leq T_{1}\leq T_{2}\leq \ldots$ son tiempos aleatorios infinitos en los cuales ocurren ciertos eventos. El n\'umero de tiempos $T_{n}$ en el intervalo $\left[0,t\right)$ es

\begin{eqnarray}
N\left(t\right)=\sum_{n=1}^{\infty}\indora\left(T_{n}\leq t\right),
\end{eqnarray}
para $t\geq0$.
\end{Def}

Si se consideran los puntos $T_{n}$ como elementos de $\rea_{+}$, y $N\left(t\right)$ es el n\'umero de puntos en $\rea$. El proceso denotado por $\left\{N\left(t\right):t\geq0\right\}$, denotado por $N\left(t\right)$, es un proceso puntual en $\rea_{+}$. Los $T_{n}$ son los tiempos de ocurrencia, el proceso puntual $N\left(t\right)$ es simple si su n\'umero de ocurrencias son distintas: $0<T_{1}<T_{2}<\ldots$ casi seguramente.

\begin{Def}
Un proceso puntual $N\left(t\right)$ es un proceso de renovaci\'on si los tiempos de interocurrencia $\xi_{n}=T_{n}-T_{n-1}$, para $n\geq1$, son independientes e identicamente distribuidos con distribuci\'on $F$, donde $F\left(0\right)=0$ y $T_{0}=0$. Los $T_{n}$ son llamados tiempos de renovaci\'on, referente a la independencia o renovaci\'on de la informaci\'on estoc\'astica en estos tiempos. Los $\xi_{n}$ son los tiempos de inter-renovaci\'on, y $N\left(t\right)$ es el n\'umero de renovaciones en el intervalo $\left[0,t\right)$
\end{Def}


\begin{Note}
Para definir un proceso de renovaci\'on para cualquier contexto, solamente hay que especificar una distribuci\'on $F$, con $F\left(0\right)=0$, para los tiempos de inter-renovaci\'on. La funci\'on $F$ en turno degune las otra variables aleatorias. De manera formal, existe un espacio de probabilidad y una sucesi\'on de variables aleatorias $\xi_{1},\xi_{2},\ldots$ definidas en este con distribuci\'on $F$. Entonces las otras cantidades son $T_{n}=\sum_{k=1}^{n}\xi_{k}$ y $N\left(t\right)=\sum_{n=1}^{\infty}\indora\left(T_{n}\leq t\right)$, donde $T_{n}\rightarrow\infty$ casi seguramente por la Ley Fuerte de los Grandes N\'umeros.
\end{Note}







Los tiempos $T_{n}$ est\'an relacionados con los conteos de $N\left(t\right)$ por

\begin{eqnarray*}
\left\{N\left(t\right)\geq n\right\}&=&\left\{T_{n}\leq t\right\}\\
T_{N\left(t\right)}\leq &t&<T_{N\left(t\right)+1},
\end{eqnarray*}

adem\'as $N\left(T_{n}\right)=n$, y 

\begin{eqnarray*}
N\left(t\right)=\max\left\{n:T_{n}\leq t\right\}=\min\left\{n:T_{n+1}>t\right\}
\end{eqnarray*}

Por propiedades de la convoluci\'on se sabe que

\begin{eqnarray*}
P\left\{T_{n}\leq t\right\}=F^{n\star}\left(t\right)
\end{eqnarray*}
que es la $n$-\'esima convoluci\'on de $F$. Entonces 

\begin{eqnarray*}
\left\{N\left(t\right)\geq n\right\}&=&\left\{T_{n}\leq t\right\}\\
P\left\{N\left(t\right)\leq n\right\}&=&1-F^{\left(n+1\right)\star}\left(t\right)
\end{eqnarray*}

Adem\'as usando el hecho de que $\esp\left[N\left(t\right)\right]=\sum_{n=1}^{\infty}P\left\{N\left(t\right)\geq n\right\}$
se tiene que

\begin{eqnarray*}
\esp\left[N\left(t\right)\right]=\sum_{n=1}^{\infty}F^{n\star}\left(t\right)
\end{eqnarray*}

\begin{Prop}
Para cada $t\geq0$, la funci\'on generadora de momentos $\esp\left[e^{\alpha N\left(t\right)}\right]$ existe para alguna $\alpha$ en una vecindad del 0, y de aqu\'i que $\esp\left[N\left(t\right)^{m}\right]<\infty$, para $m\geq1$.
\end{Prop}

\begin{Ejem}[\textbf{Proceso Poisson}]

Suponga que se tienen tiempos de inter-renovaci\'on \textit{i.i.d.} del proceso de renovaci\'on $N\left(t\right)$ tienen distribuci\'on exponencial $F\left(t\right)=q-e^{-\lambda t}$ con tasa $\lambda$. Entonces $N\left(t\right)$ es un proceso Poisson con tasa $\lambda$.

\end{Ejem}


\begin{Note}
Si el primer tiempo de renovaci\'on $\xi_{1}$ no tiene la misma distribuci\'on que el resto de las $\xi_{n}$, para $n\geq2$, a $N\left(t\right)$ se le llama Proceso de Renovaci\'on retardado, donde si $\xi$ tiene distribuci\'on $G$, entonces el tiempo $T_{n}$ de la $n$-\'esima renovaci\'on tiene distribuci\'on $G\star F^{\left(n-1\right)\star}\left(t\right)$
\end{Note}


\begin{Teo}
Para una constante $\mu\leq\infty$ ( o variable aleatoria), las siguientes expresiones son equivalentes:

\begin{eqnarray}
lim_{n\rightarrow\infty}n^{-1}T_{n}&=&\mu,\textrm{ c.s.}\\
lim_{t\rightarrow\infty}t^{-1}N\left(t\right)&=&1/\mu,\textrm{ c.s.}
\end{eqnarray}
\end{Teo}


Es decir, $T_{n}$ satisface la Ley Fuerte de los Grandes N\'umeros s\'i y s\'olo s\'i $N\left/t\right)$ la cumple.


\begin{Coro}[Ley Fuerte de los Grandes N\'umeros para Procesos de Renovaci\'on]
Si $N\left(t\right)$ es un proceso de renovaci\'on cuyos tiempos de inter-renovaci\'on tienen media $\mu\leq\infty$, entonces
\begin{eqnarray}
t^{-1}N\left(t\right)\rightarrow 1/\mu,\textrm{ c.s. cuando }t\rightarrow\infty.
\end{eqnarray}

\end{Coro}


Considerar el proceso estoc\'astico de valores reales $\left\{Z\left(t\right):t\geq0\right\}$ en el mismo espacio de probabilidad que $N\left(t\right)$

\begin{Def}
Para el proceso $\left\{Z\left(t\right):t\geq0\right\}$ se define la fluctuaci\'on m\'axima de $Z\left(t\right)$ en el intervalo $\left(T_{n-1},T_{n}\right]$:
\begin{eqnarray*}
M_{n}=\sup_{T_{n-1}<t\leq T_{n}}|Z\left(t\right)-Z\left(T_{n-1}\right)|
\end{eqnarray*}
\end{Def}

\begin{Teo}
Sup\'ongase que $n^{-1}T_{n}\rightarrow\mu$ c.s. cuando $n\rightarrow\infty$, donde $\mu\leq\infty$ es una constante o variable aleatoria. Sea $a$ una constante o variable aleatoria que puede ser infinita cuando $\mu$ es finita, y considere las expresiones l\'imite:
\begin{eqnarray}
lim_{n\rightarrow\infty}n^{-1}Z\left(T_{n}\right)&=&a,\textrm{ c.s.}\\
lim_{t\rightarrow\infty}t^{-1}Z\left(t\right)&=&a/\mu,\textrm{ c.s.}
\end{eqnarray}
La segunda expresi\'on implica la primera. Conversamente, la primera implica la segunda si el proceso $Z\left(t\right)$ es creciente, o si $lim_{n\rightarrow\infty}n^{-1}M_{n}=0$ c.s.
\end{Teo}

\begin{Coro}
Si $N\left(t\right)$ es un proceso de renovaci\'on, y $\left(Z\left(T_{n}\right)-Z\left(T_{n-1}\right),M_{n}\right)$, para $n\geq1$, son variables aleatorias independientes e id\'enticamente distribuidas con media finita, entonces,
\begin{eqnarray}
lim_{t\rightarrow\infty}t^{-1}Z\left(t\right)\rightarrow\frac{\esp\left[Z\left(T_{1}\right)-Z\left(T_{0}\right)\right]}{\esp\left[T_{1}\right]},\textrm{ c.s. cuando  }t\rightarrow\infty.
\end{eqnarray}
\end{Coro}


Sup\'ongase que $N\left(t\right)$ es un proceso de renovaci\'on con distribuci\'on $F$ con media finita $\mu$.

\begin{Def}
La funci\'on de renovaci\'on asociada con la distribuci\'on $F$, del proceso $N\left(t\right)$, es
\begin{eqnarray*}
U\left(t\right)=\sum_{n=1}^{\infty}F^{n\star}\left(t\right),\textrm{   }t\geq0,
\end{eqnarray*}
donde $F^{0\star}\left(t\right)=\indora\left(t\geq0\right)$.
\end{Def}


\begin{Prop}
Sup\'ongase que la distribuci\'on de inter-renovaci\'on $F$ tiene densidad $f$. Entonces $U\left(t\right)$ tambi\'en tiene densidad, para $t>0$, y es $U^{'}\left(t\right)=\sum_{n=0}^{\infty}f^{n\star}\left(t\right)$. Adem\'as
\begin{eqnarray*}
\prob\left\{N\left(t\right)>N\left(t-\right)\right\}=0\textrm{,   }t\geq0.
\end{eqnarray*}
\end{Prop}

\begin{Def}
La Transformada de Laplace-Stieljes de $F$ est\'a dada por

\begin{eqnarray*}
\hat{F}\left(\alpha\right)=\int_{\rea_{+}}e^{-\alpha t}dF\left(t\right)\textrm{,  }\alpha\geq0.
\end{eqnarray*}
\end{Def}

Entonces

\begin{eqnarray*}
\hat{U}\left(\alpha\right)=\sum_{n=0}^{\infty}\hat{F^{n\star}}\left(\alpha\right)=\sum_{n=0}^{\infty}\hat{F}\left(\alpha\right)^{n}=\frac{1}{1-\hat{F}\left(\alpha\right)}.
\end{eqnarray*}


\begin{Prop}
La Transformada de Laplace $\hat{U}\left(\alpha\right)$ y $\hat{F}\left(\alpha\right)$ determina una a la otra de manera \'unica por la relaci\'on $\hat{U}\left(\alpha\right)=\frac{1}{1-\hat{F}\left(\alpha\right)}$.
\end{Prop}


\begin{Note}
Un proceso de renovaci\'on $N\left(t\right)$ cuyos tiempos de inter-renovaci\'on tienen media finita, es un proceso Poisson con tasa $\lambda$ si y s\'olo s\'i $\esp\left[U\left(t\right)\right]=\lambda t$, para $t\geq0$.
\end{Note}


\begin{Teo}
Sea $N\left(t\right)$ un proceso puntual simple con puntos de localizaci\'on $T_{n}$ tal que $\eta\left(t\right)=\esp\left[N\left(\right)\right]$ es finita para cada $t$. Entonces para cualquier funci\'on $f:\rea_{+}\rightarrow\rea$,
\begin{eqnarray*}
\esp\left[\sum_{n=1}^{N\left(\right)}f\left(T_{n}\right)\right]=\int_{\left(0,t\right]}f\left(s\right)d\eta\left(s\right)\textrm{,  }t\geq0,
\end{eqnarray*}
suponiendo que la integral exista. Adem\'as si $X_{1},X_{2},\ldots$ son variables aleatorias definidas en el mismo espacio de probabilidad que el proceso $N\left(t\right)$ tal que $\esp\left[X_{n}|T_{n}=s\right]=f\left(s\right)$, independiente de $n$. Entonces
\begin{eqnarray*}
\esp\left[\sum_{n=1}^{N\left(t\right)}X_{n}\right]=\int_{\left(0,t\right]}f\left(s\right)d\eta\left(s\right)\textrm{,  }t\geq0,
\end{eqnarray*} 
suponiendo que la integral exista. 
\end{Teo}

\begin{Coro}[Identidad de Wald para Renovaciones]
Para el proceso de renovaci\'on $N\left(t\right)$,
\begin{eqnarray*}
\esp\left[T_{N\left(t\right)+1}\right]=\mu\esp\left[N\left(t\right)+1\right]\textrm{,  }t\geq0,
\end{eqnarray*}  
\end{Coro}


\begin{Def}
Sea $h\left(t\right)$ funci\'on de valores reales en $\rea$ acotada en intervalos finitos e igual a cero para $t<0$ La ecuaci\'on de renovaci\'on para $h\left(t\right)$ y la distribuci\'on $F$ es

\begin{eqnarray}\label{Ec.Renovacion}
H\left(t\right)=h\left(t\right)+\int_{\left[0,t\right]}H\left(t-s\right)dF\left(s\right)\textrm{,    }t\geq0,
\end{eqnarray}
donde $H\left(t\right)$ es una funci\'on de valores reales. Esto es $H=h+F\star H$. Decimos que $H\left(t\right)$ es soluci\'on de esta ecuaci\'on si satisface la ecuaci\'on, y es acotada en intervalos finitos e iguales a cero para $t<0$.
\end{Def}

\begin{Prop}
La funci\'on $U\star h\left(t\right)$ es la \'unica soluci\'on de la ecuaci\'on de renovaci\'on (\ref{Ec.Renovacion}).
\end{Prop}

\begin{Teo}[Teorema Renovaci\'on Elemental]
\begin{eqnarray*}
t^{-1}U\left(t\right)\rightarrow 1/\mu\textrm{,    cuando }t\rightarrow\infty.
\end{eqnarray*}
\end{Teo}



Sup\'ongase que $N\left(t\right)$ es un proceso de renovaci\'on con distribuci\'on $F$ con media finita $\mu$.

\begin{Def}
La funci\'on de renovaci\'on asociada con la distribuci\'on $F$, del proceso $N\left(t\right)$, es
\begin{eqnarray*}
U\left(t\right)=\sum_{n=1}^{\infty}F^{n\star}\left(t\right),\textrm{   }t\geq0,
\end{eqnarray*}
donde $F^{0\star}\left(t\right)=\indora\left(t\geq0\right)$.
\end{Def}


\begin{Prop}
Sup\'ongase que la distribuci\'on de inter-renovaci\'on $F$ tiene densidad $f$. Entonces $U\left(t\right)$ tambi\'en tiene densidad, para $t>0$, y es $U^{'}\left(t\right)=\sum_{n=0}^{\infty}f^{n\star}\left(t\right)$. Adem\'as
\begin{eqnarray*}
\prob\left\{N\left(t\right)>N\left(t-\right)\right\}=0\textrm{,   }t\geq0.
\end{eqnarray*}
\end{Prop}

\begin{Def}
La Transformada de Laplace-Stieljes de $F$ est\'a dada por

\begin{eqnarray*}
\hat{F}\left(\alpha\right)=\int_{\rea_{+}}e^{-\alpha t}dF\left(t\right)\textrm{,  }\alpha\geq0.
\end{eqnarray*}
\end{Def}

Entonces

\begin{eqnarray*}
\hat{U}\left(\alpha\right)=\sum_{n=0}^{\infty}\hat{F^{n\star}}\left(\alpha\right)=\sum_{n=0}^{\infty}\hat{F}\left(\alpha\right)^{n}=\frac{1}{1-\hat{F}\left(\alpha\right)}.
\end{eqnarray*}


\begin{Prop}
La Transformada de Laplace $\hat{U}\left(\alpha\right)$ y $\hat{F}\left(\alpha\right)$ determina una a la otra de manera \'unica por la relaci\'on $\hat{U}\left(\alpha\right)=\frac{1}{1-\hat{F}\left(\alpha\right)}$.
\end{Prop}


\begin{Note}
Un proceso de renovaci\'on $N\left(t\right)$ cuyos tiempos de inter-renovaci\'on tienen media finita, es un proceso Poisson con tasa $\lambda$ si y s\'olo s\'i $\esp\left[U\left(t\right)\right]=\lambda t$, para $t\geq0$.
\end{Note}


\begin{Teo}
Sea $N\left(t\right)$ un proceso puntual simple con puntos de localizaci\'on $T_{n}$ tal que $\eta\left(t\right)=\esp\left[N\left(\right)\right]$ es finita para cada $t$. Entonces para cualquier funci\'on $f:\rea_{+}\rightarrow\rea$,
\begin{eqnarray*}
\esp\left[\sum_{n=1}^{N\left(\right)}f\left(T_{n}\right)\right]=\int_{\left(0,t\right]}f\left(s\right)d\eta\left(s\right)\textrm{,  }t\geq0,
\end{eqnarray*}
suponiendo que la integral exista. Adem\'as si $X_{1},X_{2},\ldots$ son variables aleatorias definidas en el mismo espacio de probabilidad que el proceso $N\left(t\right)$ tal que $\esp\left[X_{n}|T_{n}=s\right]=f\left(s\right)$, independiente de $n$. Entonces
\begin{eqnarray*}
\esp\left[\sum_{n=1}^{N\left(t\right)}X_{n}\right]=\int_{\left(0,t\right]}f\left(s\right)d\eta\left(s\right)\textrm{,  }t\geq0,
\end{eqnarray*} 
suponiendo que la integral exista. 
\end{Teo}

\begin{Coro}[Identidad de Wald para Renovaciones]
Para el proceso de renovaci\'on $N\left(t\right)$,
\begin{eqnarray*}
\esp\left[T_{N\left(t\right)+1}\right]=\mu\esp\left[N\left(t\right)+1\right]\textrm{,  }t\geq0,
\end{eqnarray*}  
\end{Coro}


\begin{Def}
Sea $h\left(t\right)$ funci\'on de valores reales en $\rea$ acotada en intervalos finitos e igual a cero para $t<0$ La ecuaci\'on de renovaci\'on para $h\left(t\right)$ y la distribuci\'on $F$ es

\begin{eqnarray}\label{Ec.Renovacion}
H\left(t\right)=h\left(t\right)+\int_{\left[0,t\right]}H\left(t-s\right)dF\left(s\right)\textrm{,    }t\geq0,
\end{eqnarray}
donde $H\left(t\right)$ es una funci\'on de valores reales. Esto es $H=h+F\star H$. Decimos que $H\left(t\right)$ es soluci\'on de esta ecuaci\'on si satisface la ecuaci\'on, y es acotada en intervalos finitos e iguales a cero para $t<0$.
\end{Def}

\begin{Prop}
La funci\'on $U\star h\left(t\right)$ es la \'unica soluci\'on de la ecuaci\'on de renovaci\'on (\ref{Ec.Renovacion}).
\end{Prop}

\begin{Teo}[Teorema Renovaci\'on Elemental]
\begin{eqnarray*}
t^{-1}U\left(t\right)\rightarrow 1/\mu\textrm{,    cuando }t\rightarrow\infty.
\end{eqnarray*}
\end{Teo}


\begin{Note} Una funci\'on $h:\rea_{+}\rightarrow\rea$ es Directamente Riemann Integrable en los siguientes casos:
\begin{itemize}
\item[a)] $h\left(t\right)\geq0$ es decreciente y Riemann Integrable.
\item[b)] $h$ es continua excepto posiblemente en un conjunto de Lebesgue de medida 0, y $|h\left(t\right)|\leq b\left(t\right)$, donde $b$ es DRI.
\end{itemize}
\end{Note}

\begin{Teo}[Teorema Principal de Renovaci\'on]
Si $F$ es no aritm\'etica y $h\left(t\right)$ es Directamente Riemann Integrable (DRI), entonces

\begin{eqnarray*}
lim_{t\rightarrow\infty}U\star h=\frac{1}{\mu}\int_{\rea_{+}}h\left(s\right)ds.
\end{eqnarray*}
\end{Teo}

\begin{Prop}
Cualquier funci\'on $H\left(t\right)$ acotada en intervalos finitos y que es 0 para $t<0$ puede expresarse como
\begin{eqnarray*}
H\left(t\right)=U\star h\left(t\right)\textrm{,  donde }h\left(t\right)=H\left(t\right)-F\star H\left(t\right)
\end{eqnarray*}
\end{Prop}

\begin{Def}
Un proceso estoc\'astico $X\left(t\right)$ es crudamente regenerativo en un tiempo aleatorio positivo $T$ si
\begin{eqnarray*}
\esp\left[X\left(T+t\right)|T\right]=\esp\left[X\left(t\right)\right]\textrm{, para }t\geq0,\end{eqnarray*}
y con las esperanzas anteriores finitas.
\end{Def}

\begin{Prop}
Sup\'ongase que $X\left(t\right)$ es un proceso crudamente regenerativo en $T$, que tiene distribuci\'on $F$. Si $\esp\left[X\left(t\right)\right]$ es acotado en intervalos finitos, entonces
\begin{eqnarray*}
\esp\left[X\left(t\right)\right]=U\star h\left(t\right)\textrm{,  donde }h\left(t\right)=\esp\left[X\left(t\right)\indora\left(T>t\right)\right].
\end{eqnarray*}
\end{Prop}

\begin{Teo}[Regeneraci\'on Cruda]
Sup\'ongase que $X\left(t\right)$ es un proceso con valores positivo crudamente regenerativo en $T$, y def\'inase $M=\sup\left\{|X\left(t\right)|:t\leq T\right\}$. Si $T$ es no aritm\'etico y $M$ y $MT$ tienen media finita, entonces
\begin{eqnarray*}
lim_{t\rightarrow\infty}\esp\left[X\left(t\right)\right]=\frac{1}{\mu}\int_{\rea_{+}}h\left(s\right)ds,
\end{eqnarray*}
donde $h\left(t\right)=\esp\left[X\left(t\right)\indora\left(T>t\right)\right]$.
\end{Teo}


\begin{Note} Una funci\'on $h:\rea_{+}\rightarrow\rea$ es Directamente Riemann Integrable en los siguientes casos:
\begin{itemize}
\item[a)] $h\left(t\right)\geq0$ es decreciente y Riemann Integrable.
\item[b)] $h$ es continua excepto posiblemente en un conjunto de Lebesgue de medida 0, y $|h\left(t\right)|\leq b\left(t\right)$, donde $b$ es DRI.
\end{itemize}
\end{Note}

\begin{Teo}[Teorema Principal de Renovaci\'on]
Si $F$ es no aritm\'etica y $h\left(t\right)$ es Directamente Riemann Integrable (DRI), entonces

\begin{eqnarray*}
lim_{t\rightarrow\infty}U\star h=\frac{1}{\mu}\int_{\rea_{+}}h\left(s\right)ds.
\end{eqnarray*}
\end{Teo}

\begin{Prop}
Cualquier funci\'on $H\left(t\right)$ acotada en intervalos finitos y que es 0 para $t<0$ puede expresarse como
\begin{eqnarray*}
H\left(t\right)=U\star h\left(t\right)\textrm{,  donde }h\left(t\right)=H\left(t\right)-F\star H\left(t\right)
\end{eqnarray*}
\end{Prop}

\begin{Def}
Un proceso estoc\'astico $X\left(t\right)$ es crudamente regenerativo en un tiempo aleatorio positivo $T$ si
\begin{eqnarray*}
\esp\left[X\left(T+t\right)|T\right]=\esp\left[X\left(t\right)\right]\textrm{, para }t\geq0,\end{eqnarray*}
y con las esperanzas anteriores finitas.
\end{Def}

\begin{Prop}
Sup\'ongase que $X\left(t\right)$ es un proceso crudamente regenerativo en $T$, que tiene distribuci\'on $F$. Si $\esp\left[X\left(t\right)\right]$ es acotado en intervalos finitos, entonces
\begin{eqnarray*}
\esp\left[X\left(t\right)\right]=U\star h\left(t\right)\textrm{,  donde }h\left(t\right)=\esp\left[X\left(t\right)\indora\left(T>t\right)\right].
\end{eqnarray*}
\end{Prop}

\begin{Teo}[Regeneraci\'on Cruda]
Sup\'ongase que $X\left(t\right)$ es un proceso con valores positivo crudamente regenerativo en $T$, y def\'inase $M=\sup\left\{|X\left(t\right)|:t\leq T\right\}$. Si $T$ es no aritm\'etico y $M$ y $MT$ tienen media finita, entonces
\begin{eqnarray*}
lim_{t\rightarrow\infty}\esp\left[X\left(t\right)\right]=\frac{1}{\mu}\int_{\rea_{+}}h\left(s\right)ds,
\end{eqnarray*}
donde $h\left(t\right)=\esp\left[X\left(t\right)\indora\left(T>t\right)\right]$.
\end{Teo}

\begin{Def}
Para el proceso $\left\{\left(N\left(t\right),X\left(t\right)\right):t\geq0\right\}$, sus trayectoria muestrales en el intervalo de tiempo $\left[T_{n-1},T_{n}\right)$ est\'an descritas por
\begin{eqnarray*}
\zeta_{n}=\left(\xi_{n},\left\{X\left(T_{n-1}+t\right):0\leq t<\xi_{n}\right\}\right)
\end{eqnarray*}
Este $\zeta_{n}$ es el $n$-\'esimo segmento del proceso. El proceso es regenerativo sobre los tiempos $T_{n}$ si sus segmentos $\zeta_{n}$ son independientes e id\'enticamennte distribuidos.
\end{Def}


\begin{Note}
Si $\tilde{X}\left(t\right)$ con espacio de estados $\tilde{S}$ es regenerativo sobre $T_{n}$, entonces $X\left(t\right)=f\left(\tilde{X}\left(t\right)\right)$ tambi\'en es regenerativo sobre $T_{n}$, para cualquier funci\'on $f:\tilde{S}\rightarrow S$.
\end{Note}

\begin{Note}
Los procesos regenerativos son crudamente regenerativos, pero no al rev\'es.
\end{Note}


\begin{Note}
Un proceso estoc\'astico a tiempo continuo o discreto es regenerativo si existe un proceso de renovaci\'on  tal que los segmentos del proceso entre tiempos de renovaci\'on sucesivos son i.i.d., es decir, para $\left\{X\left(t\right):t\geq0\right\}$ proceso estoc\'astico a tiempo continuo con espacio de estados $S$, espacio m\'etrico.
\end{Note}

Para $\left\{X\left(t\right):t\geq0\right\}$ Proceso Estoc\'astico a tiempo continuo con estado de espacios $S$, que es un espacio m\'etrico, con trayectorias continuas por la derecha y con l\'imites por la izquierda c.s. Sea $N\left(t\right)$ un proceso de renovaci\'on en $\rea_{+}$ definido en el mismo espacio de probabilidad que $X\left(t\right)$, con tiempos de renovaci\'on $T$ y tiempos de inter-renovaci\'on $\xi_{n}=T_{n}-T_{n-1}$, con misma distribuci\'on $F$ de media finita $\mu$.



\begin{Def}
Para el proceso $\left\{\left(N\left(t\right),X\left(t\right)\right):t\geq0\right\}$, sus trayectoria muestrales en el intervalo de tiempo $\left[T_{n-1},T_{n}\right)$ est\'an descritas por
\begin{eqnarray*}
\zeta_{n}=\left(\xi_{n},\left\{X\left(T_{n-1}+t\right):0\leq t<\xi_{n}\right\}\right)
\end{eqnarray*}
Este $\zeta_{n}$ es el $n$-\'esimo segmento del proceso. El proceso es regenerativo sobre los tiempos $T_{n}$ si sus segmentos $\zeta_{n}$ son independientes e id\'enticamennte distribuidos.
\end{Def}

\begin{Note}
Un proceso regenerativo con media de la longitud de ciclo finita es llamado positivo recurrente.
\end{Note}

\begin{Teo}[Procesos Regenerativos]
Suponga que el proceso
\end{Teo}


\begin{Def}[Renewal Process Trinity]
Para un proceso de renovaci\'on $N\left(t\right)$, los siguientes procesos proveen de informaci\'on sobre los tiempos de renovaci\'on.
\begin{itemize}
\item $A\left(t\right)=t-T_{N\left(t\right)}$, el tiempo de recurrencia hacia atr\'as al tiempo $t$, que es el tiempo desde la \'ultima renovaci\'on para $t$.

\item $B\left(t\right)=T_{N\left(t\right)+1}-t$, el tiempo de recurrencia hacia adelante al tiempo $t$, residual del tiempo de renovaci\'on, que es el tiempo para la pr\'oxima renovaci\'on despu\'es de $t$.

\item $L\left(t\right)=\xi_{N\left(t\right)+1}=A\left(t\right)+B\left(t\right)$, la longitud del intervalo de renovaci\'on que contiene a $t$.
\end{itemize}
\end{Def}

\begin{Note}
El proceso tridimensional $\left(A\left(t\right),B\left(t\right),L\left(t\right)\right)$ es regenerativo sobre $T_{n}$, y por ende cada proceso lo es. Cada proceso $A\left(t\right)$ y $B\left(t\right)$ son procesos de MArkov a tiempo continuo con trayectorias continuas por partes en el espacio de estados $\rea_{+}$. Una expresi\'on conveniente para su distribuci\'on conjunta es, para $0\leq x<t,y\geq0$
\begin{equation}\label{NoRenovacion}
P\left\{A\left(t\right)>x,B\left(t\right)>y\right\}=
P\left\{N\left(t+y\right)-N\left((t-x)\right)=0\right\}
\end{equation}
\end{Note}

\begin{Ejem}[Tiempos de recurrencia Poisson]
Si $N\left(t\right)$ es un proceso Poisson con tasa $\lambda$, entonces de la expresi\'on (\ref{NoRenovacion}) se tiene que

\begin{eqnarray*}
\begin{array}{lc}
P\left\{A\left(t\right)>x,B\left(t\right)>y\right\}=e^{-\lambda\left(x+y\right)},&0\leq x<t,y\geq0,
\end{array}
\end{eqnarray*}
que es la probabilidad Poisson de no renovaciones en un intervalo de longitud $x+y$.

\end{Ejem}

\begin{Note}
Una cadena de Markov erg\'odica tiene la propiedad de ser estacionaria si la distribuci\'on de su estado al tiempo $0$ es su distribuci\'on estacionaria.
\end{Note}


\begin{Def}
Un proceso estoc\'astico a tiempo continuo $\left\{X\left(t\right):t\geq0\right\}$ en un espacio general es estacionario si sus distribuciones finito dimensionales son invariantes bajo cualquier  traslado: para cada $0\leq s_{1}<s_{2}<\cdots<s_{k}$ y $t\geq0$,
\begin{eqnarray*}
\left(X\left(s_{1}+t\right),\ldots,X\left(s_{k}+t\right)\right)=_{d}\left(X\left(s_{1}\right),\ldots,X\left(s_{k}\right)\right).
\end{eqnarray*}
\end{Def}

\begin{Note}
Un proceso de Markov es estacionario si $X\left(t\right)=_{d}X\left(0\right)$, $t\geq0$.
\end{Note}

Considerese el proceso $N\left(t\right)=\sum_{n}\indora\left(\tau_{n}\leq t\right)$ en $\rea_{+}$, con puntos $0<\tau_{1}<\tau_{2}<\cdots$.

\begin{Prop}
Si $N$ es un proceso puntual estacionario y $\esp\left[N\left(1\right)\right]<\infty$, entonces $\esp\left[N\left(t\right)\right]=t\esp\left[N\left(1\right)\right]$, $t\geq0$

\end{Prop}

\begin{Teo}
Los siguientes enunciados son equivalentes
\begin{itemize}
\item[i)] El proceso retardado de renovaci\'on $N$ es estacionario.

\item[ii)] EL proceso de tiempos de recurrencia hacia adelante $B\left(t\right)$ es estacionario.


\item[iii)] $\esp\left[N\left(t\right)\right]=t/\mu$,


\item[iv)] $G\left(t\right)=F_{e}\left(t\right)=\frac{1}{\mu}\int_{0}^{t}\left[1-F\left(s\right)\right]ds$
\end{itemize}
Cuando estos enunciados son ciertos, $P\left\{B\left(t\right)\leq x\right\}=F_{e}\left(x\right)$, para $t,x\geq0$.

\end{Teo}

\begin{Note}
Una consecuencia del teorema anterior es que el Proceso Poisson es el \'unico proceso sin retardo que es estacionario.
\end{Note}

\begin{Coro}
El proceso de renovaci\'on $N\left(t\right)$ sin retardo, y cuyos tiempos de inter renonaci\'on tienen media finita, es estacionario si y s\'olo si es un proceso Poisson.

\end{Coro}

%______________________________________________________________________

%\section{Ejemplos, Notas importantes}
%______________________________________________________________________
%\section*{Ap\'endice A}
%__________________________________________________________________

%________________________________________________________________________
%\subsection*{Procesos Regenerativos}
%________________________________________________________________________



\begin{Note}
Si $\tilde{X}\left(t\right)$ con espacio de estados $\tilde{S}$ es regenerativo sobre $T_{n}$, entonces $X\left(t\right)=f\left(\tilde{X}\left(t\right)\right)$ tambi\'en es regenerativo sobre $T_{n}$, para cualquier funci\'on $f:\tilde{S}\rightarrow S$.
\end{Note}

\begin{Note}
Los procesos regenerativos son crudamente regenerativos, pero no al rev\'es.
\end{Note}
%\subsection*{Procesos Regenerativos: Sigman\cite{Sigman1}}
\begin{Def}[Definici\'on Cl\'asica]
Un proceso estoc\'astico $X=\left\{X\left(t\right):t\geq0\right\}$ es llamado regenerativo is existe una variable aleatoria $R_{1}>0$ tal que
\begin{itemize}
\item[i)] $\left\{X\left(t+R_{1}\right):t\geq0\right\}$ es independiente de $\left\{\left\{X\left(t\right):t<R_{1}\right\},\right\}$
\item[ii)] $\left\{X\left(t+R_{1}\right):t\geq0\right\}$ es estoc\'asticamente equivalente a $\left\{X\left(t\right):t>0\right\}$
\end{itemize}

Llamamos a $R_{1}$ tiempo de regeneraci\'on, y decimos que $X$ se regenera en este punto.
\end{Def}

$\left\{X\left(t+R_{1}\right)\right\}$ es regenerativo con tiempo de regeneraci\'on $R_{2}$, independiente de $R_{1}$ pero con la misma distribuci\'on que $R_{1}$. Procediendo de esta manera se obtiene una secuencia de variables aleatorias independientes e id\'enticamente distribuidas $\left\{R_{n}\right\}$ llamados longitudes de ciclo. Si definimos a $Z_{k}\equiv R_{1}+R_{2}+\cdots+R_{k}$, se tiene un proceso de renovaci\'on llamado proceso de renovaci\'on encajado para $X$.




\begin{Def}
Para $x$ fijo y para cada $t\geq0$, sea $I_{x}\left(t\right)=1$ si $X\left(t\right)\leq x$,  $I_{x}\left(t\right)=0$ en caso contrario, y def\'inanse los tiempos promedio
\begin{eqnarray*}
\overline{X}&=&lim_{t\rightarrow\infty}\frac{1}{t}\int_{0}^{\infty}X\left(u\right)du\\
\prob\left(X_{\infty}\leq x\right)&=&lim_{t\rightarrow\infty}\frac{1}{t}\int_{0}^{\infty}I_{x}\left(u\right)du,
\end{eqnarray*}
cuando estos l\'imites existan.
\end{Def}

Como consecuencia del teorema de Renovaci\'on-Recompensa, se tiene que el primer l\'imite  existe y es igual a la constante
\begin{eqnarray*}
\overline{X}&=&\frac{\esp\left[\int_{0}^{R_{1}}X\left(t\right)dt\right]}{\esp\left[R_{1}\right]},
\end{eqnarray*}
suponiendo que ambas esperanzas son finitas.

\begin{Note}
\begin{itemize}
\item[a)] Si el proceso regenerativo $X$ es positivo recurrente y tiene trayectorias muestrales no negativas, entonces la ecuaci\'on anterior es v\'alida.
\item[b)] Si $X$ es positivo recurrente regenerativo, podemos construir una \'unica versi\'on estacionaria de este proceso, $X_{e}=\left\{X_{e}\left(t\right)\right\}$, donde $X_{e}$ es un proceso estoc\'astico regenerativo y estrictamente estacionario, con distribuci\'on marginal distribuida como $X_{\infty}$
\end{itemize}
\end{Note}

Para $\left\{X\left(t\right):t\geq0\right\}$ Proceso Estoc\'astico a tiempo continuo con estado de espacios $S$, que es un espacio m\'etrico, con trayectorias continuas por la derecha y con l\'imites por la izquierda c.s. Sea $N\left(t\right)$ un proceso de renovaci\'on en $\rea_{+}$ definido en el mismo espacio de probabilidad que $X\left(t\right)$, con tiempos de renovaci\'on $T$ y tiempos de inter-renovaci\'on $\xi_{n}=T_{n}-T_{n-1}$, con misma distribuci\'on $F$ de media finita $\mu$.


\begin{Def}
Para el proceso $\left\{\left(N\left(t\right),X\left(t\right)\right):t\geq0\right\}$, sus trayectoria muestrales en el intervalo de tiempo $\left[T_{n-1},T_{n}\right)$ est\'an descritas por
\begin{eqnarray*}
\zeta_{n}=\left(\xi_{n},\left\{X\left(T_{n-1}+t\right):0\leq t<\xi_{n}\right\}\right)
\end{eqnarray*}
Este $\zeta_{n}$ es el $n$-\'esimo segmento del proceso. El proceso es regenerativo sobre los tiempos $T_{n}$ si sus segmentos $\zeta_{n}$ son independientes e id\'enticamennte distribuidos.
\end{Def}


\begin{Note}
Si $\tilde{X}\left(t\right)$ con espacio de estados $\tilde{S}$ es regenerativo sobre $T_{n}$, entonces $X\left(t\right)=f\left(\tilde{X}\left(t\right)\right)$ tambi\'en es regenerativo sobre $T_{n}$, para cualquier funci\'on $f:\tilde{S}\rightarrow S$.
\end{Note}

\begin{Note}
Los procesos regenerativos son crudamente regenerativos, pero no al rev\'es.
\end{Note}

\begin{Def}[Definici\'on Cl\'asica]
Un proceso estoc\'astico $X=\left\{X\left(t\right):t\geq0\right\}$ es llamado regenerativo is existe una variable aleatoria $R_{1}>0$ tal que
\begin{itemize}
\item[i)] $\left\{X\left(t+R_{1}\right):t\geq0\right\}$ es independiente de $\left\{\left\{X\left(t\right):t<R_{1}\right\},\right\}$
\item[ii)] $\left\{X\left(t+R_{1}\right):t\geq0\right\}$ es estoc\'asticamente equivalente a $\left\{X\left(t\right):t>0\right\}$
\end{itemize}

Llamamos a $R_{1}$ tiempo de regeneraci\'on, y decimos que $X$ se regenera en este punto.
\end{Def}

$\left\{X\left(t+R_{1}\right)\right\}$ es regenerativo con tiempo de regeneraci\'on $R_{2}$, independiente de $R_{1}$ pero con la misma distribuci\'on que $R_{1}$. Procediendo de esta manera se obtiene una secuencia de variables aleatorias independientes e id\'enticamente distribuidas $\left\{R_{n}\right\}$ llamados longitudes de ciclo. Si definimos a $Z_{k}\equiv R_{1}+R_{2}+\cdots+R_{k}$, se tiene un proceso de renovaci\'on llamado proceso de renovaci\'on encajado para $X$.

\begin{Note}
Un proceso regenerativo con media de la longitud de ciclo finita es llamado positivo recurrente.
\end{Note}


\begin{Def}
Para $x$ fijo y para cada $t\geq0$, sea $I_{x}\left(t\right)=1$ si $X\left(t\right)\leq x$,  $I_{x}\left(t\right)=0$ en caso contrario, y def\'inanse los tiempos promedio
\begin{eqnarray*}
\overline{X}&=&lim_{t\rightarrow\infty}\frac{1}{t}\int_{0}^{\infty}X\left(u\right)du\\
\prob\left(X_{\infty}\leq x\right)&=&lim_{t\rightarrow\infty}\frac{1}{t}\int_{0}^{\infty}I_{x}\left(u\right)du,
\end{eqnarray*}
cuando estos l\'imites existan.
\end{Def}

Como consecuencia del teorema de Renovaci\'on-Recompensa, se tiene que el primer l\'imite  existe y es igual a la constante
\begin{eqnarray*}
\overline{X}&=&\frac{\esp\left[\int_{0}^{R_{1}}X\left(t\right)dt\right]}{\esp\left[R_{1}\right]},
\end{eqnarray*}
suponiendo que ambas esperanzas son finitas.

\begin{Note}
\begin{itemize}
\item[a)] Si el proceso regenerativo $X$ es positivo recurrente y tiene trayectorias muestrales no negativas, entonces la ecuaci\'on anterior es v\'alida.
\item[b)] Si $X$ es positivo recurrente regenerativo, podemos construir una \'unica versi\'on estacionaria de este proceso, $X_{e}=\left\{X_{e}\left(t\right)\right\}$, donde $X_{e}$ es un proceso estoc\'astico regenerativo y estrictamente estacionario, con distribuci\'on marginal distribuida como $X_{\infty}$
\end{itemize}
\end{Note}

%__________________________________________________________________________________________
%\subsection{Procesos Regenerativos Estacionarios - Stidham \cite{Stidham}}
%__________________________________________________________________________________________


Un proceso estoc\'astico a tiempo continuo $\left\{V\left(t\right),t\geq0\right\}$ es un proceso regenerativo si existe una sucesi\'on de variables aleatorias independientes e id\'enticamente distribuidas $\left\{X_{1},X_{2},\ldots\right\}$, sucesi\'on de renovaci\'on, tal que para cualquier conjunto de Borel $A$, 

\begin{eqnarray*}
\prob\left\{V\left(t\right)\in A|X_{1}+X_{2}+\cdots+X_{R\left(t\right)}=s,\left\{V\left(\tau\right),\tau<s\right\}\right\}=\prob\left\{V\left(t-s\right)\in A|X_{1}>t-s\right\},
\end{eqnarray*}
para todo $0\leq s\leq t$, donde $R\left(t\right)=\max\left\{X_{1}+X_{2}+\cdots+X_{j}\leq t\right\}=$n\'umero de renovaciones ({\emph{puntos de regeneraci\'on}}) que ocurren en $\left[0,t\right]$. El intervalo $\left[0,X_{1}\right)$ es llamado {\emph{primer ciclo de regeneraci\'on}} de $\left\{V\left(t \right),t\geq0\right\}$, $\left[X_{1},X_{1}+X_{2}\right)$ el {\emph{segundo ciclo de regeneraci\'on}}, y as\'i sucesivamente.

Sea $X=X_{1}$ y sea $F$ la funci\'on de distrbuci\'on de $X$


\begin{Def}
Se define el proceso estacionario, $\left\{V^{*}\left(t\right),t\geq0\right\}$, para $\left\{V\left(t\right),t\geq0\right\}$ por

\begin{eqnarray*}
\prob\left\{V\left(t\right)\in A\right\}=\frac{1}{\esp\left[X\right]}\int_{0}^{\infty}\prob\left\{V\left(t+x\right)\in A|X>x\right\}\left(1-F\left(x\right)\right)dx,
\end{eqnarray*} 
para todo $t\geq0$ y todo conjunto de Borel $A$.
\end{Def}

\begin{Def}
Una distribuci\'on se dice que es {\emph{aritm\'etica}} si todos sus puntos de incremento son m\'ultiplos de la forma $0,\lambda, 2\lambda,\ldots$ para alguna $\lambda>0$ entera.
\end{Def}


\begin{Def}
Una modificaci\'on medible de un proceso $\left\{V\left(t\right),t\geq0\right\}$, es una versi\'on de este, $\left\{V\left(t,w\right)\right\}$ conjuntamente medible para $t\geq0$ y para $w\in S$, $S$ espacio de estados para $\left\{V\left(t\right),t\geq0\right\}$.
\end{Def}

\begin{Teo}
Sea $\left\{V\left(t\right),t\geq\right\}$ un proceso regenerativo no negativo con modificaci\'on medible. Sea $\esp\left[X\right]<\infty$. Entonces el proceso estacionario dado por la ecuaci\'on anterior est\'a bien definido y tiene funci\'on de distribuci\'on independiente de $t$, adem\'as
\begin{itemize}
\item[i)] \begin{eqnarray*}
\esp\left[V^{*}\left(0\right)\right]&=&\frac{\esp\left[\int_{0}^{X}V\left(s\right)ds\right]}{\esp\left[X\right]}\end{eqnarray*}
\item[ii)] Si $\esp\left[V^{*}\left(0\right)\right]<\infty$, equivalentemente, si $\esp\left[\int_{0}^{X}V\left(s\right)ds\right]<\infty$,entonces
\begin{eqnarray*}
\frac{\int_{0}^{t}V\left(s\right)ds}{t}\rightarrow\frac{\esp\left[\int_{0}^{X}V\left(s\right)ds\right]}{\esp\left[X\right]}
\end{eqnarray*}
con probabilidad 1 y en media, cuando $t\rightarrow\infty$.
\end{itemize}
\end{Teo}
%
%___________________________________________________________________________________________
%\vspace{5.5cm}
%\chapter{Cadenas de Markov estacionarias}
%\vspace{-1.0cm}


%__________________________________________________________________________________________
%\subsection{Procesos Regenerativos Estacionarios - Stidham \cite{Stidham}}
%__________________________________________________________________________________________


Un proceso estoc\'astico a tiempo continuo $\left\{V\left(t\right),t\geq0\right\}$ es un proceso regenerativo si existe una sucesi\'on de variables aleatorias independientes e id\'enticamente distribuidas $\left\{X_{1},X_{2},\ldots\right\}$, sucesi\'on de renovaci\'on, tal que para cualquier conjunto de Borel $A$, 

\begin{eqnarray*}
\prob\left\{V\left(t\right)\in A|X_{1}+X_{2}+\cdots+X_{R\left(t\right)}=s,\left\{V\left(\tau\right),\tau<s\right\}\right\}=\prob\left\{V\left(t-s\right)\in A|X_{1}>t-s\right\},
\end{eqnarray*}
para todo $0\leq s\leq t$, donde $R\left(t\right)=\max\left\{X_{1}+X_{2}+\cdots+X_{j}\leq t\right\}=$n\'umero de renovaciones ({\emph{puntos de regeneraci\'on}}) que ocurren en $\left[0,t\right]$. El intervalo $\left[0,X_{1}\right)$ es llamado {\emph{primer ciclo de regeneraci\'on}} de $\left\{V\left(t \right),t\geq0\right\}$, $\left[X_{1},X_{1}+X_{2}\right)$ el {\emph{segundo ciclo de regeneraci\'on}}, y as\'i sucesivamente.

Sea $X=X_{1}$ y sea $F$ la funci\'on de distrbuci\'on de $X$


\begin{Def}
Se define el proceso estacionario, $\left\{V^{*}\left(t\right),t\geq0\right\}$, para $\left\{V\left(t\right),t\geq0\right\}$ por

\begin{eqnarray*}
\prob\left\{V\left(t\right)\in A\right\}=\frac{1}{\esp\left[X\right]}\int_{0}^{\infty}\prob\left\{V\left(t+x\right)\in A|X>x\right\}\left(1-F\left(x\right)\right)dx,
\end{eqnarray*} 
para todo $t\geq0$ y todo conjunto de Borel $A$.
\end{Def}

\begin{Def}
Una distribuci\'on se dice que es {\emph{aritm\'etica}} si todos sus puntos de incremento son m\'ultiplos de la forma $0,\lambda, 2\lambda,\ldots$ para alguna $\lambda>0$ entera.
\end{Def}


\begin{Def}
Una modificaci\'on medible de un proceso $\left\{V\left(t\right),t\geq0\right\}$, es una versi\'on de este, $\left\{V\left(t,w\right)\right\}$ conjuntamente medible para $t\geq0$ y para $w\in S$, $S$ espacio de estados para $\left\{V\left(t\right),t\geq0\right\}$.
\end{Def}

\begin{Teo}
Sea $\left\{V\left(t\right),t\geq\right\}$ un proceso regenerativo no negativo con modificaci\'on medible. Sea $\esp\left[X\right]<\infty$. Entonces el proceso estacionario dado por la ecuaci\'on anterior est\'a bien definido y tiene funci\'on de distribuci\'on independiente de $t$, adem\'as
\begin{itemize}
\item[i)] \begin{eqnarray*}
\esp\left[V^{*}\left(0\right)\right]&=&\frac{\esp\left[\int_{0}^{X}V\left(s\right)ds\right]}{\esp\left[X\right]}\end{eqnarray*}
\item[ii)] Si $\esp\left[V^{*}\left(0\right)\right]<\infty$, equivalentemente, si $\esp\left[\int_{0}^{X}V\left(s\right)ds\right]<\infty$,entonces
\begin{eqnarray*}
\frac{\int_{0}^{t}V\left(s\right)ds}{t}\rightarrow\frac{\esp\left[\int_{0}^{X}V\left(s\right)ds\right]}{\esp\left[X\right]}
\end{eqnarray*}
con probabilidad 1 y en media, cuando $t\rightarrow\infty$.
\end{itemize}
\end{Teo}

Sea la funci\'on generadora de momentos para $L_{i}$, el n\'umero de usuarios en la cola $Q_{i}\left(z\right)$ en cualquier momento, est\'a dada por el tiempo promedio de $z^{L_{i}\left(t\right)}$ sobre el ciclo regenerativo definido anteriormente. Entonces 



Es decir, es posible determinar las longitudes de las colas a cualquier tiempo $t$. Entonces, determinando el primer momento es posible ver que


\begin{Def}
El tiempo de Ciclo $C_{i}$ es el periodo de tiempo que comienza cuando la cola $i$ es visitada por primera vez en un ciclo, y termina cuando es visitado nuevamente en el pr\'oximo ciclo. La duraci\'on del mismo est\'a dada por $\tau_{i}\left(m+1\right)-\tau_{i}\left(m\right)$, o equivalentemente $\overline{\tau}_{i}\left(m+1\right)-\overline{\tau}_{i}\left(m\right)$ bajo condiciones de estabilidad.
\end{Def}


\begin{Def}
El tiempo de intervisita $I_{i}$ es el periodo de tiempo que comienza cuando se ha completado el servicio en un ciclo y termina cuando es visitada nuevamente en el pr\'oximo ciclo. Su  duraci\'on del mismo est\'a dada por $\tau_{i}\left(m+1\right)-\overline{\tau}_{i}\left(m\right)$.
\end{Def}

La duraci\'on del tiempo de intervisita es $\tau_{i}\left(m+1\right)-\overline{\tau}\left(m\right)$. Dado que el n\'umero de usuarios presentes en $Q_{i}$ al tiempo $t=\tau_{i}\left(m+1\right)$ es igual al n\'umero de arribos durante el intervalo de tiempo $\left[\overline{\tau}\left(m\right),\tau_{i}\left(m+1\right)\right]$ se tiene que


\begin{eqnarray*}
\esp\left[z_{i}^{L_{i}\left(\tau_{i}\left(m+1\right)\right)}\right]=\esp\left[\left\{P_{i}\left(z_{i}\right)\right\}^{\tau_{i}\left(m+1\right)-\overline{\tau}\left(m\right)}\right]
\end{eqnarray*}

entonces, si $I_{i}\left(z\right)=\esp\left[z^{\tau_{i}\left(m+1\right)-\overline{\tau}\left(m\right)}\right]$
se tiene que $F_{i}\left(z\right)=I_{i}\left[P_{i}\left(z\right)\right]$
para $i=1,2$.

Conforme a la definici\'on dada al principio del cap\'itulo, definici\'on (\ref{Def.Tn}), sean $T_{1},T_{2},\ldots$ los puntos donde las longitudes de las colas de la red de sistemas de visitas c\'iclicas son cero simult\'aneamente, cuando la cola $Q_{j}$ es visitada por el servidor para dar servicio, es decir, $L_{1}\left(T_{i}\right)=0,L_{2}\left(T_{i}\right)=0,\hat{L}_{1}\left(T_{i}\right)=0$ y $\hat{L}_{2}\left(T_{i}\right)=0$, a estos puntos se les denominar\'a puntos regenerativos. Entonces, 

\begin{Def}
Al intervalo de tiempo entre dos puntos regenerativos se le llamar\'a ciclo regenerativo.
\end{Def}

\begin{Def}
Para $T_{i}$ se define, $M_{i}$, el n\'umero de ciclos de visita a la cola $Q_{l}$, durante el ciclo regenerativo, es decir, $M_{i}$ es un proceso de renovaci\'on.
\end{Def}

\begin{Def}
Para cada uno de los $M_{i}$'s, se definen a su vez la duraci\'on de cada uno de estos ciclos de visita en el ciclo regenerativo, $C_{i}^{(m)}$, para $m=1,2,\ldots,M_{i}$, que a su vez, tambi\'en es n proceso de renovaci\'on.
\end{Def}

\footnote{In Stidham and  Heyman \cite{Stidham} shows that is sufficient for the regenerative process to be stationary that the mean regenerative cycle time is finite: $\esp\left[\sum_{m=1}^{M_{i}}C_{i}^{(m)}\right]<\infty$, 


 como cada $C_{i}^{(m)}$ contiene intervalos de r\'eplica positivos, se tiene que $\esp\left[M_{i}\right]<\infty$, adem\'as, como $M_{i}>0$, se tiene que la condici\'on anterior es equivalente a tener que $\esp\left[C_{i}\right]<\infty$,
por lo tanto una condici\'on suficiente para la existencia del proceso regenerativo est\'a dada por $\sum_{k=1}^{N}\mu_{k}<1.$}
%________________________________________________________________________
\subsection{Procesos Regenerativos Sigman, Thorisson y Wolff \cite{Sigman2}}
%________________________________________________________________________


\begin{Def}[Definici\'on Cl\'asica]
Un proceso estoc\'astico $X=\left\{X\left(t\right):t\geq0\right\}$ es llamado regenerativo is existe una variable aleatoria $R_{1}>0$ tal que
\begin{itemize}
\item[i)] $\left\{X\left(t+R_{1}\right):t\geq0\right\}$ es independiente de $\left\{\left\{X\left(t\right):t<R_{1}\right\},\right\}$
\item[ii)] $\left\{X\left(t+R_{1}\right):t\geq0\right\}$ es estoc\'asticamente equivalente a $\left\{X\left(t\right):t>0\right\}$
\end{itemize}

Llamamos a $R_{1}$ tiempo de regeneraci\'on, y decimos que $X$ se regenera en este punto.
\end{Def}

$\left\{X\left(t+R_{1}\right)\right\}$ es regenerativo con tiempo de regeneraci\'on $R_{2}$, independiente de $R_{1}$ pero con la misma distribuci\'on que $R_{1}$. Procediendo de esta manera se obtiene una secuencia de variables aleatorias independientes e id\'enticamente distribuidas $\left\{R_{n}\right\}$ llamados longitudes de ciclo. Si definimos a $Z_{k}\equiv R_{1}+R_{2}+\cdots+R_{k}$, se tiene un proceso de renovaci\'on llamado proceso de renovaci\'on encajado para $X$.


\begin{Note}
La existencia de un primer tiempo de regeneraci\'on, $R_{1}$, implica la existencia de una sucesi\'on completa de estos tiempos $R_{1},R_{2}\ldots,$ que satisfacen la propiedad deseada \cite{Sigman2}.
\end{Note}


\begin{Note} Para la cola $GI/GI/1$ los usuarios arriban con tiempos $t_{n}$ y son atendidos con tiempos de servicio $S_{n}$, los tiempos de arribo forman un proceso de renovaci\'on  con tiempos entre arribos independientes e identicamente distribuidos (\texttt{i.i.d.})$T_{n}=t_{n}-t_{n-1}$, adem\'as los tiempos de servicio son \texttt{i.i.d.} e independientes de los procesos de arribo. Por \textit{estable} se entiende que $\esp S_{n}<\esp T_{n}<\infty$.
\end{Note}
 


\begin{Def}
Para $x$ fijo y para cada $t\geq0$, sea $I_{x}\left(t\right)=1$ si $X\left(t\right)\leq x$,  $I_{x}\left(t\right)=0$ en caso contrario, y def\'inanse los tiempos promedio
\begin{eqnarray*}
\overline{X}&=&lim_{t\rightarrow\infty}\frac{1}{t}\int_{0}^{\infty}X\left(u\right)du\\
\prob\left(X_{\infty}\leq x\right)&=&lim_{t\rightarrow\infty}\frac{1}{t}\int_{0}^{\infty}I_{x}\left(u\right)du,
\end{eqnarray*}
cuando estos l\'imites existan.
\end{Def}

Como consecuencia del teorema de Renovaci\'on-Recompensa, se tiene que el primer l\'imite  existe y es igual a la constante
\begin{eqnarray*}
\overline{X}&=&\frac{\esp\left[\int_{0}^{R_{1}}X\left(t\right)dt\right]}{\esp\left[R_{1}\right]},
\end{eqnarray*}
suponiendo que ambas esperanzas son finitas.
 
\begin{Note}
Funciones de procesos regenerativos son regenerativas, es decir, si $X\left(t\right)$ es regenerativo y se define el proceso $Y\left(t\right)$ por $Y\left(t\right)=f\left(X\left(t\right)\right)$ para alguna funci\'on Borel medible $f\left(\cdot\right)$. Adem\'as $Y$ es regenerativo con los mismos tiempos de renovaci\'on que $X$. 

En general, los tiempos de renovaci\'on, $Z_{k}$ de un proceso regenerativo no requieren ser tiempos de paro con respecto a la evoluci\'on de $X\left(t\right)$.
\end{Note} 

\begin{Note}
Una funci\'on de un proceso de Markov, usualmente no ser\'a un proceso de Markov, sin embargo ser\'a regenerativo si el proceso de Markov lo es.
\end{Note}

 
\begin{Note}
Un proceso regenerativo con media de la longitud de ciclo finita es llamado positivo recurrente.
\end{Note}


\begin{Note}
\begin{itemize}
\item[a)] Si el proceso regenerativo $X$ es positivo recurrente y tiene trayectorias muestrales no negativas, entonces la ecuaci\'on anterior es v\'alida.
\item[b)] Si $X$ es positivo recurrente regenerativo, podemos construir una \'unica versi\'on estacionaria de este proceso, $X_{e}=\left\{X_{e}\left(t\right)\right\}$, donde $X_{e}$ es un proceso estoc\'astico regenerativo y estrictamente estacionario, con distribuci\'on marginal distribuida como $X_{\infty}$
\end{itemize}
\end{Note}



%________________________________________________________________________
\subsection{Procesos Regenerativos}
%________________________________________________________________________



\begin{Note}
Si $\tilde{X}\left(t\right)$ con espacio de estados $\tilde{S}$ es regenerativo sobre $T_{n}$, entonces $X\left(t\right)=f\left(\tilde{X}\left(t\right)\right)$ tambi\'en es regenerativo sobre $T_{n}$, para cualquier funci\'on $f:\tilde{S}\rightarrow S$.
\end{Note}

\begin{Note}
Los procesos regenerativos son crudamente regenerativos, pero no al rev\'es.
\end{Note}

%______________________________________________________________________
\subsection*{Procesos Regenerativos: Sigman\cite{Sigman1}}
%______________________________________________________________________
\begin{Def}[Definici\'on Cl\'asica]
Un proceso estoc\'astico $X=\left\{X\left(t\right):t\geq0\right\}$ es llamado regenerativo is existe una variable aleatoria $R_{1}>0$ tal que
\begin{itemize}
\item[i)] $\left\{X\left(t+R_{1}\right):t\geq0\right\}$ es independiente de $\left\{\left\{X\left(t\right):t<R_{1}\right\},\right\}$
\item[ii)] $\left\{X\left(t+R_{1}\right):t\geq0\right\}$ es estoc\'asticamente equivalente a $\left\{X\left(t\right):t>0\right\}$
\end{itemize}

Llamamos a $R_{1}$ tiempo de regeneraci\'on, y decimos que $X$ se regenera en este punto.
\end{Def}

$\left\{X\left(t+R_{1}\right)\right\}$ es regenerativo con tiempo de regeneraci\'on $R_{2}$, independiente de $R_{1}$ pero con la misma distribuci\'on que $R_{1}$. Procediendo de esta manera se obtiene una secuencia de variables aleatorias independientes e id\'enticamente distribuidas $\left\{R_{n}\right\}$ llamados longitudes de ciclo. Si definimos a $Z_{k}\equiv R_{1}+R_{2}+\cdots+R_{k}$, se tiene un proceso de renovaci\'on llamado proceso de renovaci\'on encajado para $X$.




\begin{Def}
Para $x$ fijo y para cada $t\geq0$, sea $I_{x}\left(t\right)=1$ si $X\left(t\right)\leq x$,  $I_{x}\left(t\right)=0$ en caso contrario, y def\'inanse los tiempos promedio
\begin{eqnarray*}
\overline{X}&=&lim_{t\rightarrow\infty}\frac{1}{t}\int_{0}^{\infty}X\left(u\right)du\\
\prob\left(X_{\infty}\leq x\right)&=&lim_{t\rightarrow\infty}\frac{1}{t}\int_{0}^{\infty}I_{x}\left(u\right)du,
\end{eqnarray*}
cuando estos l\'imites existan.
\end{Def}

Como consecuencia del teorema de Renovaci\'on-Recompensa, se tiene que el primer l\'imite  existe y es igual a la constante
\begin{eqnarray*}
\overline{X}&=&\frac{\esp\left[\int_{0}^{R_{1}}X\left(t\right)dt\right]}{\esp\left[R_{1}\right]},
\end{eqnarray*}
suponiendo que ambas esperanzas son finitas.

\begin{Note}
\begin{itemize}
\item[a)] Si el proceso regenerativo $X$ es positivo recurrente y tiene trayectorias muestrales no negativas, entonces la ecuaci\'on anterior es v\'alida.
\item[b)] Si $X$ es positivo recurrente regenerativo, podemos construir una \'unica versi\'on estacionaria de este proceso, $X_{e}=\left\{X_{e}\left(t\right)\right\}$, donde $X_{e}$ es un proceso estoc\'astico regenerativo y estrictamente estacionario, con distribuci\'on marginal distribuida como $X_{\infty}$
\end{itemize}
\end{Note}


%__________________________________________________________________________________________
\subsection{Procesos Regenerativos Estacionarios - Stidham \cite{Stidham}}
%__________________________________________________________________________________________


Un proceso estoc\'astico a tiempo continuo $\left\{V\left(t\right),t\geq0\right\}$ es un proceso regenerativo si existe una sucesi\'on de variables aleatorias independientes e id\'enticamente distribuidas $\left\{X_{1},X_{2},\ldots\right\}$, sucesi\'on de renovaci\'on, tal que para cualquier conjunto de Borel $A$, 

\begin{eqnarray*}
\prob\left\{V\left(t\right)\in A|X_{1}+X_{2}+\cdots+X_{R\left(t\right)}=s,\left\{V\left(\tau\right),\tau<s\right\}\right\}=\prob\left\{V\left(t-s\right)\in A|X_{1}>t-s\right\},
\end{eqnarray*}
para todo $0\leq s\leq t$, donde $R\left(t\right)=\max\left\{X_{1}+X_{2}+\cdots+X_{j}\leq t\right\}=$n\'umero de renovaciones ({\emph{puntos de regeneraci\'on}}) que ocurren en $\left[0,t\right]$. El intervalo $\left[0,X_{1}\right)$ es llamado {\emph{primer ciclo de regeneraci\'on}} de $\left\{V\left(t \right),t\geq0\right\}$, $\left[X_{1},X_{1}+X_{2}\right)$ el {\emph{segundo ciclo de regeneraci\'on}}, y as\'i sucesivamente.

Sea $X=X_{1}$ y sea $F$ la funci\'on de distrbuci\'on de $X$


\begin{Def}
Se define el proceso estacionario, $\left\{V^{*}\left(t\right),t\geq0\right\}$, para $\left\{V\left(t\right),t\geq0\right\}$ por

\begin{eqnarray*}
\prob\left\{V\left(t\right)\in A\right\}=\frac{1}{\esp\left[X\right]}\int_{0}^{\infty}\prob\left\{V\left(t+x\right)\in A|X>x\right\}\left(1-F\left(x\right)\right)dx,
\end{eqnarray*} 
para todo $t\geq0$ y todo conjunto de Borel $A$.
\end{Def}

\begin{Def}
Una distribuci\'on se dice que es {\emph{aritm\'etica}} si todos sus puntos de incremento son m\'ultiplos de la forma $0,\lambda, 2\lambda,\ldots$ para alguna $\lambda>0$ entera.
\end{Def}


\begin{Def}
Una modificaci\'on medible de un proceso $\left\{V\left(t\right),t\geq0\right\}$, es una versi\'on de este, $\left\{V\left(t,w\right)\right\}$ conjuntamente medible para $t\geq0$ y para $w\in S$, $S$ espacio de estados para $\left\{V\left(t\right),t\geq0\right\}$.
\end{Def}

\begin{Teo}
Sea $\left\{V\left(t\right),t\geq\right\}$ un proceso regenerativo no negativo con modificaci\'on medible. Sea $\esp\left[X\right]<\infty$. Entonces el proceso estacionario dado por la ecuaci\'on anterior est\'a bien definido y tiene funci\'on de distribuci\'on independiente de $t$, adem\'as
\begin{itemize}
\item[i)] \begin{eqnarray*}
\esp\left[V^{*}\left(0\right)\right]&=&\frac{\esp\left[\int_{0}^{X}V\left(s\right)ds\right]}{\esp\left[X\right]}\end{eqnarray*}
\item[ii)] Si $\esp\left[V^{*}\left(0\right)\right]<\infty$, equivalentemente, si $\esp\left[\int_{0}^{X}V\left(s\right)ds\right]<\infty$,entonces
\begin{eqnarray*}
\frac{\int_{0}^{t}V\left(s\right)ds}{t}\rightarrow\frac{\esp\left[\int_{0}^{X}V\left(s\right)ds\right]}{\esp\left[X\right]}
\end{eqnarray*}
con probabilidad 1 y en media, cuando $t\rightarrow\infty$.
\end{itemize}
\end{Teo}

%________________________________________________________________________
\subsection{Procesos Regenerativos}
%________________________________________________________________________

Para $\left\{X\left(t\right):t\geq0\right\}$ Proceso Estoc\'astico a tiempo continuo con estado de espacios $S$, que es un espacio m\'etrico, con trayectorias continuas por la derecha y con l\'imites por la izquierda c.s. Sea $N\left(t\right)$ un proceso de renovaci\'on en $\rea_{+}$ definido en el mismo espacio de probabilidad que $X\left(t\right)$, con tiempos de renovaci\'on $T$ y tiempos de inter-renovaci\'on $\xi_{n}=T_{n}-T_{n-1}$, con misma distribuci\'on $F$ de media finita $\mu$.



\begin{Def}
Para el proceso $\left\{\left(N\left(t\right),X\left(t\right)\right):t\geq0\right\}$, sus trayectoria muestrales en el intervalo de tiempo $\left[T_{n-1},T_{n}\right)$ est\'an descritas por
\begin{eqnarray*}
\zeta_{n}=\left(\xi_{n},\left\{X\left(T_{n-1}+t\right):0\leq t<\xi_{n}\right\}\right)
\end{eqnarray*}
Este $\zeta_{n}$ es el $n$-\'esimo segmento del proceso. El proceso es regenerativo sobre los tiempos $T_{n}$ si sus segmentos $\zeta_{n}$ son independientes e id\'enticamennte distribuidos.
\end{Def}


\begin{Obs}
Si $\tilde{X}\left(t\right)$ con espacio de estados $\tilde{S}$ es regenerativo sobre $T_{n}$, entonces $X\left(t\right)=f\left(\tilde{X}\left(t\right)\right)$ tambi\'en es regenerativo sobre $T_{n}$, para cualquier funci\'on $f:\tilde{S}\rightarrow S$.
\end{Obs}

\begin{Obs}
Los procesos regenerativos son crudamente regenerativos, pero no al rev\'es.
\end{Obs}

\begin{Def}[Definici\'on Cl\'asica]
Un proceso estoc\'astico $X=\left\{X\left(t\right):t\geq0\right\}$ es llamado regenerativo is existe una variable aleatoria $R_{1}>0$ tal que
\begin{itemize}
\item[i)] $\left\{X\left(t+R_{1}\right):t\geq0\right\}$ es independiente de $\left\{\left\{X\left(t\right):t<R_{1}\right\},\right\}$
\item[ii)] $\left\{X\left(t+R_{1}\right):t\geq0\right\}$ es estoc\'asticamente equivalente a $\left\{X\left(t\right):t>0\right\}$
\end{itemize}

Llamamos a $R_{1}$ tiempo de regeneraci\'on, y decimos que $X$ se regenera en este punto.
\end{Def}

$\left\{X\left(t+R_{1}\right)\right\}$ es regenerativo con tiempo de regeneraci\'on $R_{2}$, independiente de $R_{1}$ pero con la misma distribuci\'on que $R_{1}$. Procediendo de esta manera se obtiene una secuencia de variables aleatorias independientes e id\'enticamente distribuidas $\left\{R_{n}\right\}$ llamados longitudes de ciclo. Si definimos a $Z_{k}\equiv R_{1}+R_{2}+\cdots+R_{k}$, se tiene un proceso de renovaci\'on llamado proceso de renovaci\'on encajado para $X$.

\begin{Note}
Un proceso regenerativo con media de la longitud de ciclo finita es llamado positivo recurrente.
\end{Note}


\begin{Def}
Para $x$ fijo y para cada $t\geq0$, sea $I_{x}\left(t\right)=1$ si $X\left(t\right)\leq x$,  $I_{x}\left(t\right)=0$ en caso contrario, y def\'inanse los tiempos promedio
\begin{eqnarray*}
\overline{X}&=&lim_{t\rightarrow\infty}\frac{1}{t}\int_{0}^{\infty}X\left(u\right)du\\
\prob\left(X_{\infty}\leq x\right)&=&lim_{t\rightarrow\infty}\frac{1}{t}\int_{0}^{\infty}I_{x}\left(u\right)du,
\end{eqnarray*}
cuando estos l\'imites existan.
\end{Def}

Como consecuencia del teorema de Renovaci\'on-Recompensa, se tiene que el primer l\'imite  existe y es igual a la constante
\begin{eqnarray*}
\overline{X}&=&\frac{\esp\left[\int_{0}^{R_{1}}X\left(t\right)dt\right]}{\esp\left[R_{1}\right]},
\end{eqnarray*}
suponiendo que ambas esperanzas son finitas.

\begin{Note}
\begin{itemize}
\item[a)] Si el proceso regenerativo $X$ es positivo recurrente y tiene trayectorias muestrales no negativas, entonces la ecuaci\'on anterior es v\'alida.
\item[b)] Si $X$ es positivo recurrente regenerativo, podemos construir una \'unica versi\'on estacionaria de este proceso, $X_{e}=\left\{X_{e}\left(t\right)\right\}$, donde $X_{e}$ es un proceso estoc\'astico regenerativo y estrictamente estacionario, con distribuci\'on marginal distribuida como $X_{\infty}$
\end{itemize}
\end{Note}


%______________________________________________________________________
\subsection{Procesos Regenerativos Estacionarios: Visi\'on cl\'asica}
%______________________________________________________________________

\begin{Def}\label{Def.Tn}
Sean $0\leq T_{1}\leq T_{2}\leq \ldots$ son tiempos aleatorios infinitos en los cuales ocurren ciertos eventos. El n\'umero de tiempos $T_{n}$ en el intervalo $\left[0,t\right)$ es

\begin{eqnarray}
N\left(t\right)=\sum_{n=1}^{\infty}\indora\left(T_{n}\leq t\right),
\end{eqnarray}
para $t\geq0$.
\end{Def}

Si se consideran los puntos $T_{n}$ como elementos de $\rea_{+}$, y $N\left(t\right)$ es el n\'umero de puntos en $\rea$. El proceso denotado por $\left\{N\left(t\right):t\geq0\right\}$, denotado por $N\left(t\right)$, es un proceso puntual en $\rea_{+}$. Los $T_{n}$ son los tiempos de ocurrencia, el proceso puntual $N\left(t\right)$ es simple si su n\'umero de ocurrencias son distintas: $0<T_{1}<T_{2}<\ldots$ casi seguramente.

\begin{Def}
Un proceso puntual $N\left(t\right)$ es un proceso de renovaci\'on si los tiempos de interocurrencia $\xi_{n}=T_{n}-T_{n-1}$, para $n\geq1$, son independientes e identicamente distribuidos con distribuci\'on $F$, donde $F\left(0\right)=0$ y $T_{0}=0$. Los $T_{n}$ son llamados tiempos de renovaci\'on, referente a la independencia o renovaci\'on de la informaci\'on estoc\'astica en estos tiempos. Los $\xi_{n}$ son los tiempos de inter-renovaci\'on, y $N\left(t\right)$ es el n\'umero de renovaciones en el intervalo $\left[0,t\right)$
\end{Def}


\begin{Note}
Para definir un proceso de renovaci\'on para cualquier contexto, solamente hay que especificar una distribuci\'on $F$, con $F\left(0\right)=0$, para los tiempos de inter-renovaci\'on. La funci\'on $F$ en turno degune las otra variables aleatorias. De manera formal, existe un espacio de probabilidad y una sucesi\'on de variables aleatorias $\xi_{1},\xi_{2},\ldots$ definidas en este con distribuci\'on $F$. Entonces las otras cantidades son $T_{n}=\sum_{k=1}^{n}\xi_{k}$ y $N\left(t\right)=\sum_{n=1}^{\infty}\indora\left(T_{n}\leq t\right)$, donde $T_{n}\rightarrow\infty$ casi seguramente por la Ley Fuerte de los Grandes Números.
\end{Note}

%___________________________________________________________________________________________
%
\subsection{Teorema Principal de Renovaci\'on}
%___________________________________________________________________________________________
%

\begin{Note} Una funci\'on $h:\rea_{+}\rightarrow\rea$ es Directamente Riemann Integrable en los siguientes casos:
\begin{itemize}
\item[a)] $h\left(t\right)\geq0$ es decreciente y Riemann Integrable.
\item[b)] $h$ es continua excepto posiblemente en un conjunto de Lebesgue de medida 0, y $|h\left(t\right)|\leq b\left(t\right)$, donde $b$ es DRI.
\end{itemize}
\end{Note}

\begin{Teo}[Teorema Principal de Renovaci\'on]
Si $F$ es no aritm\'etica y $h\left(t\right)$ es Directamente Riemann Integrable (DRI), entonces

\begin{eqnarray*}
lim_{t\rightarrow\infty}U\star h=\frac{1}{\mu}\int_{\rea_{+}}h\left(s\right)ds.
\end{eqnarray*}
\end{Teo}

\begin{Prop}
Cualquier funci\'on $H\left(t\right)$ acotada en intervalos finitos y que es 0 para $t<0$ puede expresarse como
\begin{eqnarray*}
H\left(t\right)=U\star h\left(t\right)\textrm{,  donde }h\left(t\right)=H\left(t\right)-F\star H\left(t\right)
\end{eqnarray*}
\end{Prop}

\begin{Def}
Un proceso estoc\'astico $X\left(t\right)$ es crudamente regenerativo en un tiempo aleatorio positivo $T$ si
\begin{eqnarray*}
\esp\left[X\left(T+t\right)|T\right]=\esp\left[X\left(t\right)\right]\textrm{, para }t\geq0,\end{eqnarray*}
y con las esperanzas anteriores finitas.
\end{Def}

\begin{Prop}
Sup\'ongase que $X\left(t\right)$ es un proceso crudamente regenerativo en $T$, que tiene distribuci\'on $F$. Si $\esp\left[X\left(t\right)\right]$ es acotado en intervalos finitos, entonces
\begin{eqnarray*}
\esp\left[X\left(t\right)\right]=U\star h\left(t\right)\textrm{,  donde }h\left(t\right)=\esp\left[X\left(t\right)\indora\left(T>t\right)\right].
\end{eqnarray*}
\end{Prop}

\begin{Teo}[Regeneraci\'on Cruda]
Sup\'ongase que $X\left(t\right)$ es un proceso con valores positivo crudamente regenerativo en $T$, y def\'inase $M=\sup\left\{|X\left(t\right)|:t\leq T\right\}$. Si $T$ es no aritm\'etico y $M$ y $MT$ tienen media finita, entonces
\begin{eqnarray*}
lim_{t\rightarrow\infty}\esp\left[X\left(t\right)\right]=\frac{1}{\mu}\int_{\rea_{+}}h\left(s\right)ds,
\end{eqnarray*}
donde $h\left(t\right)=\esp\left[X\left(t\right)\indora\left(T>t\right)\right]$.
\end{Teo}

%___________________________________________________________________________________________
%
\subsection{Propiedades de los Procesos de Renovaci\'on}
%___________________________________________________________________________________________
%

Los tiempos $T_{n}$ est\'an relacionados con los conteos de $N\left(t\right)$ por

\begin{eqnarray*}
\left\{N\left(t\right)\geq n\right\}&=&\left\{T_{n}\leq t\right\}\\
T_{N\left(t\right)}\leq &t&<T_{N\left(t\right)+1},
\end{eqnarray*}

adem\'as $N\left(T_{n}\right)=n$, y 

\begin{eqnarray*}
N\left(t\right)=\max\left\{n:T_{n}\leq t\right\}=\min\left\{n:T_{n+1}>t\right\}
\end{eqnarray*}

Por propiedades de la convoluci\'on se sabe que

\begin{eqnarray*}
P\left\{T_{n}\leq t\right\}=F^{n\star}\left(t\right)
\end{eqnarray*}
que es la $n$-\'esima convoluci\'on de $F$. Entonces 

\begin{eqnarray*}
\left\{N\left(t\right)\geq n\right\}&=&\left\{T_{n}\leq t\right\}\\
P\left\{N\left(t\right)\leq n\right\}&=&1-F^{\left(n+1\right)\star}\left(t\right)
\end{eqnarray*}

Adem\'as usando el hecho de que $\esp\left[N\left(t\right)\right]=\sum_{n=1}^{\infty}P\left\{N\left(t\right)\geq n\right\}$
se tiene que

\begin{eqnarray*}
\esp\left[N\left(t\right)\right]=\sum_{n=1}^{\infty}F^{n\star}\left(t\right)
\end{eqnarray*}

\begin{Prop}
Para cada $t\geq0$, la funci\'on generadora de momentos $\esp\left[e^{\alpha N\left(t\right)}\right]$ existe para alguna $\alpha$ en una vecindad del 0, y de aqu\'i que $\esp\left[N\left(t\right)^{m}\right]<\infty$, para $m\geq1$.
\end{Prop}


\begin{Note}
Si el primer tiempo de renovaci\'on $\xi_{1}$ no tiene la misma distribuci\'on que el resto de las $\xi_{n}$, para $n\geq2$, a $N\left(t\right)$ se le llama Proceso de Renovaci\'on retardado, donde si $\xi$ tiene distribuci\'on $G$, entonces el tiempo $T_{n}$ de la $n$-\'esima renovaci\'on tiene distribuci\'on $G\star F^{\left(n-1\right)\star}\left(t\right)$
\end{Note}


\begin{Teo}
Para una constante $\mu\leq\infty$ ( o variable aleatoria), las siguientes expresiones son equivalentes:

\begin{eqnarray}
lim_{n\rightarrow\infty}n^{-1}T_{n}&=&\mu,\textrm{ c.s.}\\
lim_{t\rightarrow\infty}t^{-1}N\left(t\right)&=&1/\mu,\textrm{ c.s.}
\end{eqnarray}
\end{Teo}


Es decir, $T_{n}$ satisface la Ley Fuerte de los Grandes N\'umeros s\'i y s\'olo s\'i $N\left/t\right)$ la cumple.


\begin{Coro}[Ley Fuerte de los Grandes N\'umeros para Procesos de Renovaci\'on]
Si $N\left(t\right)$ es un proceso de renovaci\'on cuyos tiempos de inter-renovaci\'on tienen media $\mu\leq\infty$, entonces
\begin{eqnarray}
t^{-1}N\left(t\right)\rightarrow 1/\mu,\textrm{ c.s. cuando }t\rightarrow\infty.
\end{eqnarray}

\end{Coro}


Considerar el proceso estoc\'astico de valores reales $\left\{Z\left(t\right):t\geq0\right\}$ en el mismo espacio de probabilidad que $N\left(t\right)$

\begin{Def}
Para el proceso $\left\{Z\left(t\right):t\geq0\right\}$ se define la fluctuaci\'on m\'axima de $Z\left(t\right)$ en el intervalo $\left(T_{n-1},T_{n}\right]$:
\begin{eqnarray*}
M_{n}=\sup_{T_{n-1}<t\leq T_{n}}|Z\left(t\right)-Z\left(T_{n-1}\right)|
\end{eqnarray*}
\end{Def}

\begin{Teo}
Sup\'ongase que $n^{-1}T_{n}\rightarrow\mu$ c.s. cuando $n\rightarrow\infty$, donde $\mu\leq\infty$ es una constante o variable aleatoria. Sea $a$ una constante o variable aleatoria que puede ser infinita cuando $\mu$ es finita, y considere las expresiones l\'imite:
\begin{eqnarray}
lim_{n\rightarrow\infty}n^{-1}Z\left(T_{n}\right)&=&a,\textrm{ c.s.}\\
lim_{t\rightarrow\infty}t^{-1}Z\left(t\right)&=&a/\mu,\textrm{ c.s.}
\end{eqnarray}
La segunda expresi\'on implica la primera. Conversamente, la primera implica la segunda si el proceso $Z\left(t\right)$ es creciente, o si $lim_{n\rightarrow\infty}n^{-1}M_{n}=0$ c.s.
\end{Teo}

\begin{Coro}
Si $N\left(t\right)$ es un proceso de renovaci\'on, y $\left(Z\left(T_{n}\right)-Z\left(T_{n-1}\right),M_{n}\right)$, para $n\geq1$, son variables aleatorias independientes e id\'enticamente distribuidas con media finita, entonces,
\begin{eqnarray}
lim_{t\rightarrow\infty}t^{-1}Z\left(t\right)\rightarrow\frac{\esp\left[Z\left(T_{1}\right)-Z\left(T_{0}\right)\right]}{\esp\left[T_{1}\right]},\textrm{ c.s. cuando  }t\rightarrow\infty.
\end{eqnarray}
\end{Coro}

%___________________________________________________________________________________________
%
\subsection{Funci\'on de Renovaci\'on}
%___________________________________________________________________________________________
%


\begin{Def}
Sea $h\left(t\right)$ funci\'on de valores reales en $\rea$ acotada en intervalos finitos e igual a cero para $t<0$ La ecuaci\'on de renovaci\'on para $h\left(t\right)$ y la distribuci\'on $F$ es

\begin{eqnarray}\label{Ec.Renovacion}
H\left(t\right)=h\left(t\right)+\int_{\left[0,t\right]}H\left(t-s\right)dF\left(s\right)\textrm{,    }t\geq0,
\end{eqnarray}
donde $H\left(t\right)$ es una funci\'on de valores reales. Esto es $H=h+F\star H$. Decimos que $H\left(t\right)$ es soluci\'on de esta ecuaci\'on si satisface la ecuaci\'on, y es acotada en intervalos finitos e iguales a cero para $t<0$.
\end{Def}

\begin{Prop}
La funci\'on $U\star h\left(t\right)$ es la \'unica soluci\'on de la ecuaci\'on de renovaci\'on (\ref{Ec.Renovacion}).
\end{Prop}

\begin{Teo}[Teorema Renovaci\'on Elemental]
\begin{eqnarray*}
t^{-1}U\left(t\right)\rightarrow 1/\mu\textrm{,    cuando }t\rightarrow\infty.
\end{eqnarray*}
\end{Teo}

%______________________________________________________________________
\subsection{Procesos de Renovaci\'on}
%______________________________________________________________________

\begin{Def}\label{Def.Tn}
Sean $0\leq T_{1}\leq T_{2}\leq \ldots$ son tiempos aleatorios infinitos en los cuales ocurren ciertos eventos. El n\'umero de tiempos $T_{n}$ en el intervalo $\left[0,t\right)$ es

\begin{eqnarray}
N\left(t\right)=\sum_{n=1}^{\infty}\indora\left(T_{n}\leq t\right),
\end{eqnarray}
para $t\geq0$.
\end{Def}

Si se consideran los puntos $T_{n}$ como elementos de $\rea_{+}$, y $N\left(t\right)$ es el n\'umero de puntos en $\rea$. El proceso denotado por $\left\{N\left(t\right):t\geq0\right\}$, denotado por $N\left(t\right)$, es un proceso puntual en $\rea_{+}$. Los $T_{n}$ son los tiempos de ocurrencia, el proceso puntual $N\left(t\right)$ es simple si su n\'umero de ocurrencias son distintas: $0<T_{1}<T_{2}<\ldots$ casi seguramente.

\begin{Def}
Un proceso puntual $N\left(t\right)$ es un proceso de renovaci\'on si los tiempos de interocurrencia $\xi_{n}=T_{n}-T_{n-1}$, para $n\geq1$, son independientes e identicamente distribuidos con distribuci\'on $F$, donde $F\left(0\right)=0$ y $T_{0}=0$. Los $T_{n}$ son llamados tiempos de renovaci\'on, referente a la independencia o renovaci\'on de la informaci\'on estoc\'astica en estos tiempos. Los $\xi_{n}$ son los tiempos de inter-renovaci\'on, y $N\left(t\right)$ es el n\'umero de renovaciones en el intervalo $\left[0,t\right)$
\end{Def}


\begin{Note}
Para definir un proceso de renovaci\'on para cualquier contexto, solamente hay que especificar una distribuci\'on $F$, con $F\left(0\right)=0$, para los tiempos de inter-renovaci\'on. La funci\'on $F$ en turno degune las otra variables aleatorias. De manera formal, existe un espacio de probabilidad y una sucesi\'on de variables aleatorias $\xi_{1},\xi_{2},\ldots$ definidas en este con distribuci\'on $F$. Entonces las otras cantidades son $T_{n}=\sum_{k=1}^{n}\xi_{k}$ y $N\left(t\right)=\sum_{n=1}^{\infty}\indora\left(T_{n}\leq t\right)$, donde $T_{n}\rightarrow\infty$ casi seguramente por la Ley Fuerte de los Grandes Números.
\end{Note}

%___________________________________________________________________________________________
%
\subsection{Renewal and Regenerative Processes: Serfozo\cite{Serfozo}}
%___________________________________________________________________________________________
%
\begin{Def}\label{Def.Tn}
Sean $0\leq T_{1}\leq T_{2}\leq \ldots$ son tiempos aleatorios infinitos en los cuales ocurren ciertos eventos. El n\'umero de tiempos $T_{n}$ en el intervalo $\left[0,t\right)$ es

\begin{eqnarray}
N\left(t\right)=\sum_{n=1}^{\infty}\indora\left(T_{n}\leq t\right),
\end{eqnarray}
para $t\geq0$.
\end{Def}

Si se consideran los puntos $T_{n}$ como elementos de $\rea_{+}$, y $N\left(t\right)$ es el n\'umero de puntos en $\rea$. El proceso denotado por $\left\{N\left(t\right):t\geq0\right\}$, denotado por $N\left(t\right)$, es un proceso puntual en $\rea_{+}$. Los $T_{n}$ son los tiempos de ocurrencia, el proceso puntual $N\left(t\right)$ es simple si su n\'umero de ocurrencias son distintas: $0<T_{1}<T_{2}<\ldots$ casi seguramente.

\begin{Def}
Un proceso puntual $N\left(t\right)$ es un proceso de renovaci\'on si los tiempos de interocurrencia $\xi_{n}=T_{n}-T_{n-1}$, para $n\geq1$, son independientes e identicamente distribuidos con distribuci\'on $F$, donde $F\left(0\right)=0$ y $T_{0}=0$. Los $T_{n}$ son llamados tiempos de renovaci\'on, referente a la independencia o renovaci\'on de la informaci\'on estoc\'astica en estos tiempos. Los $\xi_{n}$ son los tiempos de inter-renovaci\'on, y $N\left(t\right)$ es el n\'umero de renovaciones en el intervalo $\left[0,t\right)$
\end{Def}


\begin{Note}
Para definir un proceso de renovaci\'on para cualquier contexto, solamente hay que especificar una distribuci\'on $F$, con $F\left(0\right)=0$, para los tiempos de inter-renovaci\'on. La funci\'on $F$ en turno degune las otra variables aleatorias. De manera formal, existe un espacio de probabilidad y una sucesi\'on de variables aleatorias $\xi_{1},\xi_{2},\ldots$ definidas en este con distribuci\'on $F$. Entonces las otras cantidades son $T_{n}=\sum_{k=1}^{n}\xi_{k}$ y $N\left(t\right)=\sum_{n=1}^{\infty}\indora\left(T_{n}\leq t\right)$, donde $T_{n}\rightarrow\infty$ casi seguramente por la Ley Fuerte de los Grandes N\'umeros.
\end{Note}







Los tiempos $T_{n}$ est\'an relacionados con los conteos de $N\left(t\right)$ por

\begin{eqnarray*}
\left\{N\left(t\right)\geq n\right\}&=&\left\{T_{n}\leq t\right\}\\
T_{N\left(t\right)}\leq &t&<T_{N\left(t\right)+1},
\end{eqnarray*}

adem\'as $N\left(T_{n}\right)=n$, y 

\begin{eqnarray*}
N\left(t\right)=\max\left\{n:T_{n}\leq t\right\}=\min\left\{n:T_{n+1}>t\right\}
\end{eqnarray*}

Por propiedades de la convoluci\'on se sabe que

\begin{eqnarray*}
P\left\{T_{n}\leq t\right\}=F^{n\star}\left(t\right)
\end{eqnarray*}
que es la $n$-\'esima convoluci\'on de $F$. Entonces 

\begin{eqnarray*}
\left\{N\left(t\right)\geq n\right\}&=&\left\{T_{n}\leq t\right\}\\
P\left\{N\left(t\right)\leq n\right\}&=&1-F^{\left(n+1\right)\star}\left(t\right)
\end{eqnarray*}

Adem\'as usando el hecho de que $\esp\left[N\left(t\right)\right]=\sum_{n=1}^{\infty}P\left\{N\left(t\right)\geq n\right\}$
se tiene que

\begin{eqnarray*}
\esp\left[N\left(t\right)\right]=\sum_{n=1}^{\infty}F^{n\star}\left(t\right)
\end{eqnarray*}

\begin{Prop}
Para cada $t\geq0$, la funci\'on generadora de momentos $\esp\left[e^{\alpha N\left(t\right)}\right]$ existe para alguna $\alpha$ en una vecindad del 0, y de aqu\'i que $\esp\left[N\left(t\right)^{m}\right]<\infty$, para $m\geq1$.
\end{Prop}

\begin{Ejem}[\textbf{Proceso Poisson}]

Suponga que se tienen tiempos de inter-renovaci\'on \textit{i.i.d.} del proceso de renovaci\'on $N\left(t\right)$ tienen distribuci\'on exponencial $F\left(t\right)=q-e^{-\lambda t}$ con tasa $\lambda$. Entonces $N\left(t\right)$ es un proceso Poisson con tasa $\lambda$.

\end{Ejem}


\begin{Note}
Si el primer tiempo de renovaci\'on $\xi_{1}$ no tiene la misma distribuci\'on que el resto de las $\xi_{n}$, para $n\geq2$, a $N\left(t\right)$ se le llama Proceso de Renovaci\'on retardado, donde si $\xi$ tiene distribuci\'on $G$, entonces el tiempo $T_{n}$ de la $n$-\'esima renovaci\'on tiene distribuci\'on $G\star F^{\left(n-1\right)\star}\left(t\right)$
\end{Note}


\begin{Teo}
Para una constante $\mu\leq\infty$ ( o variable aleatoria), las siguientes expresiones son equivalentes:

\begin{eqnarray}
lim_{n\rightarrow\infty}n^{-1}T_{n}&=&\mu,\textrm{ c.s.}\\
lim_{t\rightarrow\infty}t^{-1}N\left(t\right)&=&1/\mu,\textrm{ c.s.}
\end{eqnarray}
\end{Teo}


Es decir, $T_{n}$ satisface la Ley Fuerte de los Grandes N\'umeros s\'i y s\'olo s\'i $N\left/t\right)$ la cumple.


\begin{Coro}[Ley Fuerte de los Grandes N\'umeros para Procesos de Renovaci\'on]
Si $N\left(t\right)$ es un proceso de renovaci\'on cuyos tiempos de inter-renovaci\'on tienen media $\mu\leq\infty$, entonces
\begin{eqnarray}
t^{-1}N\left(t\right)\rightarrow 1/\mu,\textrm{ c.s. cuando }t\rightarrow\infty.
\end{eqnarray}

\end{Coro}


Considerar el proceso estoc\'astico de valores reales $\left\{Z\left(t\right):t\geq0\right\}$ en el mismo espacio de probabilidad que $N\left(t\right)$

\begin{Def}
Para el proceso $\left\{Z\left(t\right):t\geq0\right\}$ se define la fluctuaci\'on m\'axima de $Z\left(t\right)$ en el intervalo $\left(T_{n-1},T_{n}\right]$:
\begin{eqnarray*}
M_{n}=\sup_{T_{n-1}<t\leq T_{n}}|Z\left(t\right)-Z\left(T_{n-1}\right)|
\end{eqnarray*}
\end{Def}

\begin{Teo}
Sup\'ongase que $n^{-1}T_{n}\rightarrow\mu$ c.s. cuando $n\rightarrow\infty$, donde $\mu\leq\infty$ es una constante o variable aleatoria. Sea $a$ una constante o variable aleatoria que puede ser infinita cuando $\mu$ es finita, y considere las expresiones l\'imite:
\begin{eqnarray}
lim_{n\rightarrow\infty}n^{-1}Z\left(T_{n}\right)&=&a,\textrm{ c.s.}\\
lim_{t\rightarrow\infty}t^{-1}Z\left(t\right)&=&a/\mu,\textrm{ c.s.}
\end{eqnarray}
La segunda expresi\'on implica la primera. Conversamente, la primera implica la segunda si el proceso $Z\left(t\right)$ es creciente, o si $lim_{n\rightarrow\infty}n^{-1}M_{n}=0$ c.s.
\end{Teo}

\begin{Coro}
Si $N\left(t\right)$ es un proceso de renovaci\'on, y $\left(Z\left(T_{n}\right)-Z\left(T_{n-1}\right),M_{n}\right)$, para $n\geq1$, son variables aleatorias independientes e id\'enticamente distribuidas con media finita, entonces,
\begin{eqnarray}
lim_{t\rightarrow\infty}t^{-1}Z\left(t\right)\rightarrow\frac{\esp\left[Z\left(T_{1}\right)-Z\left(T_{0}\right)\right]}{\esp\left[T_{1}\right]},\textrm{ c.s. cuando  }t\rightarrow\infty.
\end{eqnarray}
\end{Coro}


Sup\'ongase que $N\left(t\right)$ es un proceso de renovaci\'on con distribuci\'on $F$ con media finita $\mu$.

\begin{Def}
La funci\'on de renovaci\'on asociada con la distribuci\'on $F$, del proceso $N\left(t\right)$, es
\begin{eqnarray*}
U\left(t\right)=\sum_{n=1}^{\infty}F^{n\star}\left(t\right),\textrm{   }t\geq0,
\end{eqnarray*}
donde $F^{0\star}\left(t\right)=\indora\left(t\geq0\right)$.
\end{Def}


\begin{Prop}
Sup\'ongase que la distribuci\'on de inter-renovaci\'on $F$ tiene densidad $f$. Entonces $U\left(t\right)$ tambi\'en tiene densidad, para $t>0$, y es $U^{'}\left(t\right)=\sum_{n=0}^{\infty}f^{n\star}\left(t\right)$. Adem\'as
\begin{eqnarray*}
\prob\left\{N\left(t\right)>N\left(t-\right)\right\}=0\textrm{,   }t\geq0.
\end{eqnarray*}
\end{Prop}

\begin{Def}
La Transformada de Laplace-Stieljes de $F$ est\'a dada por

\begin{eqnarray*}
\hat{F}\left(\alpha\right)=\int_{\rea_{+}}e^{-\alpha t}dF\left(t\right)\textrm{,  }\alpha\geq0.
\end{eqnarray*}
\end{Def}

Entonces

\begin{eqnarray*}
\hat{U}\left(\alpha\right)=\sum_{n=0}^{\infty}\hat{F^{n\star}}\left(\alpha\right)=\sum_{n=0}^{\infty}\hat{F}\left(\alpha\right)^{n}=\frac{1}{1-\hat{F}\left(\alpha\right)}.
\end{eqnarray*}


\begin{Prop}
La Transformada de Laplace $\hat{U}\left(\alpha\right)$ y $\hat{F}\left(\alpha\right)$ determina una a la otra de manera \'unica por la relaci\'on $\hat{U}\left(\alpha\right)=\frac{1}{1-\hat{F}\left(\alpha\right)}$.
\end{Prop}


\begin{Note}
Un proceso de renovaci\'on $N\left(t\right)$ cuyos tiempos de inter-renovaci\'on tienen media finita, es un proceso Poisson con tasa $\lambda$ si y s\'olo s\'i $\esp\left[U\left(t\right)\right]=\lambda t$, para $t\geq0$.
\end{Note}


\begin{Teo}
Sea $N\left(t\right)$ un proceso puntual simple con puntos de localizaci\'on $T_{n}$ tal que $\eta\left(t\right)=\esp\left[N\left(\right)\right]$ es finita para cada $t$. Entonces para cualquier funci\'on $f:\rea_{+}\rightarrow\rea$,
\begin{eqnarray*}
\esp\left[\sum_{n=1}^{N\left(\right)}f\left(T_{n}\right)\right]=\int_{\left(0,t\right]}f\left(s\right)d\eta\left(s\right)\textrm{,  }t\geq0,
\end{eqnarray*}
suponiendo que la integral exista. Adem\'as si $X_{1},X_{2},\ldots$ son variables aleatorias definidas en el mismo espacio de probabilidad que el proceso $N\left(t\right)$ tal que $\esp\left[X_{n}|T_{n}=s\right]=f\left(s\right)$, independiente de $n$. Entonces
\begin{eqnarray*}
\esp\left[\sum_{n=1}^{N\left(t\right)}X_{n}\right]=\int_{\left(0,t\right]}f\left(s\right)d\eta\left(s\right)\textrm{,  }t\geq0,
\end{eqnarray*} 
suponiendo que la integral exista. 
\end{Teo}

\begin{Coro}[Identidad de Wald para Renovaciones]
Para el proceso de renovaci\'on $N\left(t\right)$,
\begin{eqnarray*}
\esp\left[T_{N\left(t\right)+1}\right]=\mu\esp\left[N\left(t\right)+1\right]\textrm{,  }t\geq0,
\end{eqnarray*}  
\end{Coro}


\begin{Def}
Sea $h\left(t\right)$ funci\'on de valores reales en $\rea$ acotada en intervalos finitos e igual a cero para $t<0$ La ecuaci\'on de renovaci\'on para $h\left(t\right)$ y la distribuci\'on $F$ es

\begin{eqnarray}\label{Ec.Renovacion}
H\left(t\right)=h\left(t\right)+\int_{\left[0,t\right]}H\left(t-s\right)dF\left(s\right)\textrm{,    }t\geq0,
\end{eqnarray}
donde $H\left(t\right)$ es una funci\'on de valores reales. Esto es $H=h+F\star H$. Decimos que $H\left(t\right)$ es soluci\'on de esta ecuaci\'on si satisface la ecuaci\'on, y es acotada en intervalos finitos e iguales a cero para $t<0$.
\end{Def}

\begin{Prop}
La funci\'on $U\star h\left(t\right)$ es la \'unica soluci\'on de la ecuaci\'on de renovaci\'on (\ref{Ec.Renovacion}).
\end{Prop}

\begin{Teo}[Teorema Renovaci\'on Elemental]
\begin{eqnarray*}
t^{-1}U\left(t\right)\rightarrow 1/\mu\textrm{,    cuando }t\rightarrow\infty.
\end{eqnarray*}
\end{Teo}



Sup\'ongase que $N\left(t\right)$ es un proceso de renovaci\'on con distribuci\'on $F$ con media finita $\mu$.

\begin{Def}
La funci\'on de renovaci\'on asociada con la distribuci\'on $F$, del proceso $N\left(t\right)$, es
\begin{eqnarray*}
U\left(t\right)=\sum_{n=1}^{\infty}F^{n\star}\left(t\right),\textrm{   }t\geq0,
\end{eqnarray*}
donde $F^{0\star}\left(t\right)=\indora\left(t\geq0\right)$.
\end{Def}


\begin{Prop}
Sup\'ongase que la distribuci\'on de inter-renovaci\'on $F$ tiene densidad $f$. Entonces $U\left(t\right)$ tambi\'en tiene densidad, para $t>0$, y es $U^{'}\left(t\right)=\sum_{n=0}^{\infty}f^{n\star}\left(t\right)$. Adem\'as
\begin{eqnarray*}
\prob\left\{N\left(t\right)>N\left(t-\right)\right\}=0\textrm{,   }t\geq0.
\end{eqnarray*}
\end{Prop}

\begin{Def}
La Transformada de Laplace-Stieljes de $F$ est\'a dada por

\begin{eqnarray*}
\hat{F}\left(\alpha\right)=\int_{\rea_{+}}e^{-\alpha t}dF\left(t\right)\textrm{,  }\alpha\geq0.
\end{eqnarray*}
\end{Def}

Entonces

\begin{eqnarray*}
\hat{U}\left(\alpha\right)=\sum_{n=0}^{\infty}\hat{F^{n\star}}\left(\alpha\right)=\sum_{n=0}^{\infty}\hat{F}\left(\alpha\right)^{n}=\frac{1}{1-\hat{F}\left(\alpha\right)}.
\end{eqnarray*}


\begin{Prop}
La Transformada de Laplace $\hat{U}\left(\alpha\right)$ y $\hat{F}\left(\alpha\right)$ determina una a la otra de manera \'unica por la relaci\'on $\hat{U}\left(\alpha\right)=\frac{1}{1-\hat{F}\left(\alpha\right)}$.
\end{Prop}


\begin{Note}
Un proceso de renovaci\'on $N\left(t\right)$ cuyos tiempos de inter-renovaci\'on tienen media finita, es un proceso Poisson con tasa $\lambda$ si y s\'olo s\'i $\esp\left[U\left(t\right)\right]=\lambda t$, para $t\geq0$.
\end{Note}


\begin{Teo}
Sea $N\left(t\right)$ un proceso puntual simple con puntos de localizaci\'on $T_{n}$ tal que $\eta\left(t\right)=\esp\left[N\left(\right)\right]$ es finita para cada $t$. Entonces para cualquier funci\'on $f:\rea_{+}\rightarrow\rea$,
\begin{eqnarray*}
\esp\left[\sum_{n=1}^{N\left(\right)}f\left(T_{n}\right)\right]=\int_{\left(0,t\right]}f\left(s\right)d\eta\left(s\right)\textrm{,  }t\geq0,
\end{eqnarray*}
suponiendo que la integral exista. Adem\'as si $X_{1},X_{2},\ldots$ son variables aleatorias definidas en el mismo espacio de probabilidad que el proceso $N\left(t\right)$ tal que $\esp\left[X_{n}|T_{n}=s\right]=f\left(s\right)$, independiente de $n$. Entonces
\begin{eqnarray*}
\esp\left[\sum_{n=1}^{N\left(t\right)}X_{n}\right]=\int_{\left(0,t\right]}f\left(s\right)d\eta\left(s\right)\textrm{,  }t\geq0,
\end{eqnarray*} 
suponiendo que la integral exista. 
\end{Teo}

\begin{Coro}[Identidad de Wald para Renovaciones]
Para el proceso de renovaci\'on $N\left(t\right)$,
\begin{eqnarray*}
\esp\left[T_{N\left(t\right)+1}\right]=\mu\esp\left[N\left(t\right)+1\right]\textrm{,  }t\geq0,
\end{eqnarray*}  
\end{Coro}


\begin{Def}
Sea $h\left(t\right)$ funci\'on de valores reales en $\rea$ acotada en intervalos finitos e igual a cero para $t<0$ La ecuaci\'on de renovaci\'on para $h\left(t\right)$ y la distribuci\'on $F$ es

\begin{eqnarray}\label{Ec.Renovacion}
H\left(t\right)=h\left(t\right)+\int_{\left[0,t\right]}H\left(t-s\right)dF\left(s\right)\textrm{,    }t\geq0,
\end{eqnarray}
donde $H\left(t\right)$ es una funci\'on de valores reales. Esto es $H=h+F\star H$. Decimos que $H\left(t\right)$ es soluci\'on de esta ecuaci\'on si satisface la ecuaci\'on, y es acotada en intervalos finitos e iguales a cero para $t<0$.
\end{Def}

\begin{Prop}
La funci\'on $U\star h\left(t\right)$ es la \'unica soluci\'on de la ecuaci\'on de renovaci\'on (\ref{Ec.Renovacion}).
\end{Prop}

\begin{Teo}[Teorema Renovaci\'on Elemental]
\begin{eqnarray*}
t^{-1}U\left(t\right)\rightarrow 1/\mu\textrm{,    cuando }t\rightarrow\infty.
\end{eqnarray*}
\end{Teo}


\begin{Note} Una funci\'on $h:\rea_{+}\rightarrow\rea$ es Directamente Riemann Integrable en los siguientes casos:
\begin{itemize}
\item[a)] $h\left(t\right)\geq0$ es decreciente y Riemann Integrable.
\item[b)] $h$ es continua excepto posiblemente en un conjunto de Lebesgue de medida 0, y $|h\left(t\right)|\leq b\left(t\right)$, donde $b$ es DRI.
\end{itemize}
\end{Note}

\begin{Teo}[Teorema Principal de Renovaci\'on]
Si $F$ es no aritm\'etica y $h\left(t\right)$ es Directamente Riemann Integrable (DRI), entonces

\begin{eqnarray*}
lim_{t\rightarrow\infty}U\star h=\frac{1}{\mu}\int_{\rea_{+}}h\left(s\right)ds.
\end{eqnarray*}
\end{Teo}

\begin{Prop}
Cualquier funci\'on $H\left(t\right)$ acotada en intervalos finitos y que es 0 para $t<0$ puede expresarse como
\begin{eqnarray*}
H\left(t\right)=U\star h\left(t\right)\textrm{,  donde }h\left(t\right)=H\left(t\right)-F\star H\left(t\right)
\end{eqnarray*}
\end{Prop}

\begin{Def}
Un proceso estoc\'astico $X\left(t\right)$ es crudamente regenerativo en un tiempo aleatorio positivo $T$ si
\begin{eqnarray*}
\esp\left[X\left(T+t\right)|T\right]=\esp\left[X\left(t\right)\right]\textrm{, para }t\geq0,\end{eqnarray*}
y con las esperanzas anteriores finitas.
\end{Def}

\begin{Prop}
Sup\'ongase que $X\left(t\right)$ es un proceso crudamente regenerativo en $T$, que tiene distribuci\'on $F$. Si $\esp\left[X\left(t\right)\right]$ es acotado en intervalos finitos, entonces
\begin{eqnarray*}
\esp\left[X\left(t\right)\right]=U\star h\left(t\right)\textrm{,  donde }h\left(t\right)=\esp\left[X\left(t\right)\indora\left(T>t\right)\right].
\end{eqnarray*}
\end{Prop}

\begin{Teo}[Regeneraci\'on Cruda]
Sup\'ongase que $X\left(t\right)$ es un proceso con valores positivo crudamente regenerativo en $T$, y def\'inase $M=\sup\left\{|X\left(t\right)|:t\leq T\right\}$. Si $T$ es no aritm\'etico y $M$ y $MT$ tienen media finita, entonces
\begin{eqnarray*}
lim_{t\rightarrow\infty}\esp\left[X\left(t\right)\right]=\frac{1}{\mu}\int_{\rea_{+}}h\left(s\right)ds,
\end{eqnarray*}
donde $h\left(t\right)=\esp\left[X\left(t\right)\indora\left(T>t\right)\right]$.
\end{Teo}


\begin{Note} Una funci\'on $h:\rea_{+}\rightarrow\rea$ es Directamente Riemann Integrable en los siguientes casos:
\begin{itemize}
\item[a)] $h\left(t\right)\geq0$ es decreciente y Riemann Integrable.
\item[b)] $h$ es continua excepto posiblemente en un conjunto de Lebesgue de medida 0, y $|h\left(t\right)|\leq b\left(t\right)$, donde $b$ es DRI.
\end{itemize}
\end{Note}

\begin{Teo}[Teorema Principal de Renovaci\'on]
Si $F$ es no aritm\'etica y $h\left(t\right)$ es Directamente Riemann Integrable (DRI), entonces

\begin{eqnarray*}
lim_{t\rightarrow\infty}U\star h=\frac{1}{\mu}\int_{\rea_{+}}h\left(s\right)ds.
\end{eqnarray*}
\end{Teo}

\begin{Prop}
Cualquier funci\'on $H\left(t\right)$ acotada en intervalos finitos y que es 0 para $t<0$ puede expresarse como
\begin{eqnarray*}
H\left(t\right)=U\star h\left(t\right)\textrm{,  donde }h\left(t\right)=H\left(t\right)-F\star H\left(t\right)
\end{eqnarray*}
\end{Prop}

\begin{Def}
Un proceso estoc\'astico $X\left(t\right)$ es crudamente regenerativo en un tiempo aleatorio positivo $T$ si
\begin{eqnarray*}
\esp\left[X\left(T+t\right)|T\right]=\esp\left[X\left(t\right)\right]\textrm{, para }t\geq0,\end{eqnarray*}
y con las esperanzas anteriores finitas.
\end{Def}

\begin{Prop}
Sup\'ongase que $X\left(t\right)$ es un proceso crudamente regenerativo en $T$, que tiene distribuci\'on $F$. Si $\esp\left[X\left(t\right)\right]$ es acotado en intervalos finitos, entonces
\begin{eqnarray*}
\esp\left[X\left(t\right)\right]=U\star h\left(t\right)\textrm{,  donde }h\left(t\right)=\esp\left[X\left(t\right)\indora\left(T>t\right)\right].
\end{eqnarray*}
\end{Prop}

\begin{Teo}[Regeneraci\'on Cruda]
Sup\'ongase que $X\left(t\right)$ es un proceso con valores positivo crudamente regenerativo en $T$, y def\'inase $M=\sup\left\{|X\left(t\right)|:t\leq T\right\}$. Si $T$ es no aritm\'etico y $M$ y $MT$ tienen media finita, entonces
\begin{eqnarray*}
lim_{t\rightarrow\infty}\esp\left[X\left(t\right)\right]=\frac{1}{\mu}\int_{\rea_{+}}h\left(s\right)ds,
\end{eqnarray*}
donde $h\left(t\right)=\esp\left[X\left(t\right)\indora\left(T>t\right)\right]$.
\end{Teo}

\begin{Def}
Para el proceso $\left\{\left(N\left(t\right),X\left(t\right)\right):t\geq0\right\}$, sus trayectoria muestrales en el intervalo de tiempo $\left[T_{n-1},T_{n}\right)$ est\'an descritas por
\begin{eqnarray*}
\zeta_{n}=\left(\xi_{n},\left\{X\left(T_{n-1}+t\right):0\leq t<\xi_{n}\right\}\right)
\end{eqnarray*}
Este $\zeta_{n}$ es el $n$-\'esimo segmento del proceso. El proceso es regenerativo sobre los tiempos $T_{n}$ si sus segmentos $\zeta_{n}$ son independientes e id\'enticamennte distribuidos.
\end{Def}


\begin{Note}
Si $\tilde{X}\left(t\right)$ con espacio de estados $\tilde{S}$ es regenerativo sobre $T_{n}$, entonces $X\left(t\right)=f\left(\tilde{X}\left(t\right)\right)$ tambi\'en es regenerativo sobre $T_{n}$, para cualquier funci\'on $f:\tilde{S}\rightarrow S$.
\end{Note}

\begin{Note}
Los procesos regenerativos son crudamente regenerativos, pero no al rev\'es.
\end{Note}


\begin{Note}
Un proceso estoc\'astico a tiempo continuo o discreto es regenerativo si existe un proceso de renovaci\'on  tal que los segmentos del proceso entre tiempos de renovaci\'on sucesivos son i.i.d., es decir, para $\left\{X\left(t\right):t\geq0\right\}$ proceso estoc\'astico a tiempo continuo con espacio de estados $S$, espacio m\'etrico.
\end{Note}

Para $\left\{X\left(t\right):t\geq0\right\}$ Proceso Estoc\'astico a tiempo continuo con estado de espacios $S$, que es un espacio m\'etrico, con trayectorias continuas por la derecha y con l\'imites por la izquierda c.s. Sea $N\left(t\right)$ un proceso de renovaci\'on en $\rea_{+}$ definido en el mismo espacio de probabilidad que $X\left(t\right)$, con tiempos de renovaci\'on $T$ y tiempos de inter-renovaci\'on $\xi_{n}=T_{n}-T_{n-1}$, con misma distribuci\'on $F$ de media finita $\mu$.



\begin{Def}
Para el proceso $\left\{\left(N\left(t\right),X\left(t\right)\right):t\geq0\right\}$, sus trayectoria muestrales en el intervalo de tiempo $\left[T_{n-1},T_{n}\right)$ est\'an descritas por
\begin{eqnarray*}
\zeta_{n}=\left(\xi_{n},\left\{X\left(T_{n-1}+t\right):0\leq t<\xi_{n}\right\}\right)
\end{eqnarray*}
Este $\zeta_{n}$ es el $n$-\'esimo segmento del proceso. El proceso es regenerativo sobre los tiempos $T_{n}$ si sus segmentos $\zeta_{n}$ son independientes e id\'enticamennte distribuidos.
\end{Def}

\begin{Note}
Un proceso regenerativo con media de la longitud de ciclo finita es llamado positivo recurrente.
\end{Note}

\begin{Teo}[Procesos Regenerativos]
Suponga que el proceso
\end{Teo}


\begin{Def}[Renewal Process Trinity]
Para un proceso de renovaci\'on $N\left(t\right)$, los siguientes procesos proveen de informaci\'on sobre los tiempos de renovaci\'on.
\begin{itemize}
\item $A\left(t\right)=t-T_{N\left(t\right)}$, el tiempo de recurrencia hacia atr\'as al tiempo $t$, que es el tiempo desde la \'ultima renovaci\'on para $t$.

\item $B\left(t\right)=T_{N\left(t\right)+1}-t$, el tiempo de recurrencia hacia adelante al tiempo $t$, residual del tiempo de renovaci\'on, que es el tiempo para la pr\'oxima renovaci\'on despu\'es de $t$.

\item $L\left(t\right)=\xi_{N\left(t\right)+1}=A\left(t\right)+B\left(t\right)$, la longitud del intervalo de renovaci\'on que contiene a $t$.
\end{itemize}
\end{Def}

\begin{Note}
El proceso tridimensional $\left(A\left(t\right),B\left(t\right),L\left(t\right)\right)$ es regenerativo sobre $T_{n}$, y por ende cada proceso lo es. Cada proceso $A\left(t\right)$ y $B\left(t\right)$ son procesos de MArkov a tiempo continuo con trayectorias continuas por partes en el espacio de estados $\rea_{+}$. Una expresi\'on conveniente para su distribuci\'on conjunta es, para $0\leq x<t,y\geq0$
\begin{equation}\label{NoRenovacion}
P\left\{A\left(t\right)>x,B\left(t\right)>y\right\}=
P\left\{N\left(t+y\right)-N\left((t-x)\right)=0\right\}
\end{equation}
\end{Note}

\begin{Ejem}[Tiempos de recurrencia Poisson]
Si $N\left(t\right)$ es un proceso Poisson con tasa $\lambda$, entonces de la expresi\'on (\ref{NoRenovacion}) se tiene que

\begin{eqnarray*}
\begin{array}{lc}
P\left\{A\left(t\right)>x,B\left(t\right)>y\right\}=e^{-\lambda\left(x+y\right)},&0\leq x<t,y\geq0,
\end{array}
\end{eqnarray*}
que es la probabilidad Poisson de no renovaciones en un intervalo de longitud $x+y$.

\end{Ejem}

\begin{Note}
Una cadena de Markov erg\'odica tiene la propiedad de ser estacionaria si la distribuci\'on de su estado al tiempo $0$ es su distribuci\'on estacionaria.
\end{Note}


\begin{Def}
Un proceso estoc\'astico a tiempo continuo $\left\{X\left(t\right):t\geq0\right\}$ en un espacio general es estacionario si sus distribuciones finito dimensionales son invariantes bajo cualquier  traslado: para cada $0\leq s_{1}<s_{2}<\cdots<s_{k}$ y $t\geq0$,
\begin{eqnarray*}
\left(X\left(s_{1}+t\right),\ldots,X\left(s_{k}+t\right)\right)=_{d}\left(X\left(s_{1}\right),\ldots,X\left(s_{k}\right)\right).
\end{eqnarray*}
\end{Def}

\begin{Note}
Un proceso de Markov es estacionario si $X\left(t\right)=_{d}X\left(0\right)$, $t\geq0$.
\end{Note}

Considerese el proceso $N\left(t\right)=\sum_{n}\indora\left(\tau_{n}\leq t\right)$ en $\rea_{+}$, con puntos $0<\tau_{1}<\tau_{2}<\cdots$.

\begin{Prop}
Si $N$ es un proceso puntual estacionario y $\esp\left[N\left(1\right)\right]<\infty$, entonces $\esp\left[N\left(t\right)\right]=t\esp\left[N\left(1\right)\right]$, $t\geq0$

\end{Prop}

\begin{Teo}
Los siguientes enunciados son equivalentes
\begin{itemize}
\item[i)] El proceso retardado de renovaci\'on $N$ es estacionario.

\item[ii)] EL proceso de tiempos de recurrencia hacia adelante $B\left(t\right)$ es estacionario.


\item[iii)] $\esp\left[N\left(t\right)\right]=t/\mu$,


\item[iv)] $G\left(t\right)=F_{e}\left(t\right)=\frac{1}{\mu}\int_{0}^{t}\left[1-F\left(s\right)\right]ds$
\end{itemize}
Cuando estos enunciados son ciertos, $P\left\{B\left(t\right)\leq x\right\}=F_{e}\left(x\right)$, para $t,x\geq0$.

\end{Teo}

\begin{Note}
Una consecuencia del teorema anterior es que el Proceso Poisson es el \'unico proceso sin retardo que es estacionario.
\end{Note}

\begin{Coro}
El proceso de renovaci\'on $N\left(t\right)$ sin retardo, y cuyos tiempos de inter renonaci\'on tienen media finita, es estacionario si y s\'olo si es un proceso Poisson.

\end{Coro}

%______________________________________________________________________

%\section{Ejemplos, Notas importantes}
%______________________________________________________________________
%\section*{Ap\'endice A}
%__________________________________________________________________

%________________________________________________________________________
%\subsection*{Procesos Regenerativos}
%________________________________________________________________________



\begin{Note}
Si $\tilde{X}\left(t\right)$ con espacio de estados $\tilde{S}$ es regenerativo sobre $T_{n}$, entonces $X\left(t\right)=f\left(\tilde{X}\left(t\right)\right)$ tambi\'en es regenerativo sobre $T_{n}$, para cualquier funci\'on $f:\tilde{S}\rightarrow S$.
\end{Note}

\begin{Note}
Los procesos regenerativos son crudamente regenerativos, pero no al rev\'es.
\end{Note}
%\subsection*{Procesos Regenerativos: Sigman\cite{Sigman1}}
\begin{Def}[Definici\'on Cl\'asica]
Un proceso estoc\'astico $X=\left\{X\left(t\right):t\geq0\right\}$ es llamado regenerativo is existe una variable aleatoria $R_{1}>0$ tal que
\begin{itemize}
\item[i)] $\left\{X\left(t+R_{1}\right):t\geq0\right\}$ es independiente de $\left\{\left\{X\left(t\right):t<R_{1}\right\},\right\}$
\item[ii)] $\left\{X\left(t+R_{1}\right):t\geq0\right\}$ es estoc\'asticamente equivalente a $\left\{X\left(t\right):t>0\right\}$
\end{itemize}

Llamamos a $R_{1}$ tiempo de regeneraci\'on, y decimos que $X$ se regenera en este punto.
\end{Def}

$\left\{X\left(t+R_{1}\right)\right\}$ es regenerativo con tiempo de regeneraci\'on $R_{2}$, independiente de $R_{1}$ pero con la misma distribuci\'on que $R_{1}$. Procediendo de esta manera se obtiene una secuencia de variables aleatorias independientes e id\'enticamente distribuidas $\left\{R_{n}\right\}$ llamados longitudes de ciclo. Si definimos a $Z_{k}\equiv R_{1}+R_{2}+\cdots+R_{k}$, se tiene un proceso de renovaci\'on llamado proceso de renovaci\'on encajado para $X$.




\begin{Def}
Para $x$ fijo y para cada $t\geq0$, sea $I_{x}\left(t\right)=1$ si $X\left(t\right)\leq x$,  $I_{x}\left(t\right)=0$ en caso contrario, y def\'inanse los tiempos promedio
\begin{eqnarray*}
\overline{X}&=&lim_{t\rightarrow\infty}\frac{1}{t}\int_{0}^{\infty}X\left(u\right)du\\
\prob\left(X_{\infty}\leq x\right)&=&lim_{t\rightarrow\infty}\frac{1}{t}\int_{0}^{\infty}I_{x}\left(u\right)du,
\end{eqnarray*}
cuando estos l\'imites existan.
\end{Def}

Como consecuencia del teorema de Renovaci\'on-Recompensa, se tiene que el primer l\'imite  existe y es igual a la constante
\begin{eqnarray*}
\overline{X}&=&\frac{\esp\left[\int_{0}^{R_{1}}X\left(t\right)dt\right]}{\esp\left[R_{1}\right]},
\end{eqnarray*}
suponiendo que ambas esperanzas son finitas.

\begin{Note}
\begin{itemize}
\item[a)] Si el proceso regenerativo $X$ es positivo recurrente y tiene trayectorias muestrales no negativas, entonces la ecuaci\'on anterior es v\'alida.
\item[b)] Si $X$ es positivo recurrente regenerativo, podemos construir una \'unica versi\'on estacionaria de este proceso, $X_{e}=\left\{X_{e}\left(t\right)\right\}$, donde $X_{e}$ es un proceso estoc\'astico regenerativo y estrictamente estacionario, con distribuci\'on marginal distribuida como $X_{\infty}$
\end{itemize}
\end{Note}

Para $\left\{X\left(t\right):t\geq0\right\}$ Proceso Estoc\'astico a tiempo continuo con estado de espacios $S$, que es un espacio m\'etrico, con trayectorias continuas por la derecha y con l\'imites por la izquierda c.s. Sea $N\left(t\right)$ un proceso de renovaci\'on en $\rea_{+}$ definido en el mismo espacio de probabilidad que $X\left(t\right)$, con tiempos de renovaci\'on $T$ y tiempos de inter-renovaci\'on $\xi_{n}=T_{n}-T_{n-1}$, con misma distribuci\'on $F$ de media finita $\mu$.


\begin{Def}
Para el proceso $\left\{\left(N\left(t\right),X\left(t\right)\right):t\geq0\right\}$, sus trayectoria muestrales en el intervalo de tiempo $\left[T_{n-1},T_{n}\right)$ est\'an descritas por
\begin{eqnarray*}
\zeta_{n}=\left(\xi_{n},\left\{X\left(T_{n-1}+t\right):0\leq t<\xi_{n}\right\}\right)
\end{eqnarray*}
Este $\zeta_{n}$ es el $n$-\'esimo segmento del proceso. El proceso es regenerativo sobre los tiempos $T_{n}$ si sus segmentos $\zeta_{n}$ son independientes e id\'enticamennte distribuidos.
\end{Def}


\begin{Note}
Si $\tilde{X}\left(t\right)$ con espacio de estados $\tilde{S}$ es regenerativo sobre $T_{n}$, entonces $X\left(t\right)=f\left(\tilde{X}\left(t\right)\right)$ tambi\'en es regenerativo sobre $T_{n}$, para cualquier funci\'on $f:\tilde{S}\rightarrow S$.
\end{Note}

\begin{Note}
Los procesos regenerativos son crudamente regenerativos, pero no al rev\'es.
\end{Note}

\begin{Def}[Definici\'on Cl\'asica]
Un proceso estoc\'astico $X=\left\{X\left(t\right):t\geq0\right\}$ es llamado regenerativo is existe una variable aleatoria $R_{1}>0$ tal que
\begin{itemize}
\item[i)] $\left\{X\left(t+R_{1}\right):t\geq0\right\}$ es independiente de $\left\{\left\{X\left(t\right):t<R_{1}\right\},\right\}$
\item[ii)] $\left\{X\left(t+R_{1}\right):t\geq0\right\}$ es estoc\'asticamente equivalente a $\left\{X\left(t\right):t>0\right\}$
\end{itemize}

Llamamos a $R_{1}$ tiempo de regeneraci\'on, y decimos que $X$ se regenera en este punto.
\end{Def}

$\left\{X\left(t+R_{1}\right)\right\}$ es regenerativo con tiempo de regeneraci\'on $R_{2}$, independiente de $R_{1}$ pero con la misma distribuci\'on que $R_{1}$. Procediendo de esta manera se obtiene una secuencia de variables aleatorias independientes e id\'enticamente distribuidas $\left\{R_{n}\right\}$ llamados longitudes de ciclo. Si definimos a $Z_{k}\equiv R_{1}+R_{2}+\cdots+R_{k}$, se tiene un proceso de renovaci\'on llamado proceso de renovaci\'on encajado para $X$.

\begin{Note}
Un proceso regenerativo con media de la longitud de ciclo finita es llamado positivo recurrente.
\end{Note}


\begin{Def}
Para $x$ fijo y para cada $t\geq0$, sea $I_{x}\left(t\right)=1$ si $X\left(t\right)\leq x$,  $I_{x}\left(t\right)=0$ en caso contrario, y def\'inanse los tiempos promedio
\begin{eqnarray*}
\overline{X}&=&lim_{t\rightarrow\infty}\frac{1}{t}\int_{0}^{\infty}X\left(u\right)du\\
\prob\left(X_{\infty}\leq x\right)&=&lim_{t\rightarrow\infty}\frac{1}{t}\int_{0}^{\infty}I_{x}\left(u\right)du,
\end{eqnarray*}
cuando estos l\'imites existan.
\end{Def}

Como consecuencia del teorema de Renovaci\'on-Recompensa, se tiene que el primer l\'imite  existe y es igual a la constante
\begin{eqnarray*}
\overline{X}&=&\frac{\esp\left[\int_{0}^{R_{1}}X\left(t\right)dt\right]}{\esp\left[R_{1}\right]},
\end{eqnarray*}
suponiendo que ambas esperanzas son finitas.

\begin{Note}
\begin{itemize}
\item[a)] Si el proceso regenerativo $X$ es positivo recurrente y tiene trayectorias muestrales no negativas, entonces la ecuaci\'on anterior es v\'alida.
\item[b)] Si $X$ es positivo recurrente regenerativo, podemos construir una \'unica versi\'on estacionaria de este proceso, $X_{e}=\left\{X_{e}\left(t\right)\right\}$, donde $X_{e}$ es un proceso estoc\'astico regenerativo y estrictamente estacionario, con distribuci\'on marginal distribuida como $X_{\infty}$
\end{itemize}
\end{Note}

%__________________________________________________________________________________________
%\subsection{Procesos Regenerativos Estacionarios - Stidham \cite{Stidham}}
%__________________________________________________________________________________________


Un proceso estoc\'astico a tiempo continuo $\left\{V\left(t\right),t\geq0\right\}$ es un proceso regenerativo si existe una sucesi\'on de variables aleatorias independientes e id\'enticamente distribuidas $\left\{X_{1},X_{2},\ldots\right\}$, sucesi\'on de renovaci\'on, tal que para cualquier conjunto de Borel $A$, 

\begin{eqnarray*}
\prob\left\{V\left(t\right)\in A|X_{1}+X_{2}+\cdots+X_{R\left(t\right)}=s,\left\{V\left(\tau\right),\tau<s\right\}\right\}=\prob\left\{V\left(t-s\right)\in A|X_{1}>t-s\right\},
\end{eqnarray*}
para todo $0\leq s\leq t$, donde $R\left(t\right)=\max\left\{X_{1}+X_{2}+\cdots+X_{j}\leq t\right\}=$n\'umero de renovaciones ({\emph{puntos de regeneraci\'on}}) que ocurren en $\left[0,t\right]$. El intervalo $\left[0,X_{1}\right)$ es llamado {\emph{primer ciclo de regeneraci\'on}} de $\left\{V\left(t \right),t\geq0\right\}$, $\left[X_{1},X_{1}+X_{2}\right)$ el {\emph{segundo ciclo de regeneraci\'on}}, y as\'i sucesivamente.

Sea $X=X_{1}$ y sea $F$ la funci\'on de distrbuci\'on de $X$


\begin{Def}
Se define el proceso estacionario, $\left\{V^{*}\left(t\right),t\geq0\right\}$, para $\left\{V\left(t\right),t\geq0\right\}$ por

\begin{eqnarray*}
\prob\left\{V\left(t\right)\in A\right\}=\frac{1}{\esp\left[X\right]}\int_{0}^{\infty}\prob\left\{V\left(t+x\right)\in A|X>x\right\}\left(1-F\left(x\right)\right)dx,
\end{eqnarray*} 
para todo $t\geq0$ y todo conjunto de Borel $A$.
\end{Def}

\begin{Def}
Una distribuci\'on se dice que es {\emph{aritm\'etica}} si todos sus puntos de incremento son m\'ultiplos de la forma $0,\lambda, 2\lambda,\ldots$ para alguna $\lambda>0$ entera.
\end{Def}


\begin{Def}
Una modificaci\'on medible de un proceso $\left\{V\left(t\right),t\geq0\right\}$, es una versi\'on de este, $\left\{V\left(t,w\right)\right\}$ conjuntamente medible para $t\geq0$ y para $w\in S$, $S$ espacio de estados para $\left\{V\left(t\right),t\geq0\right\}$.
\end{Def}

\begin{Teo}
Sea $\left\{V\left(t\right),t\geq\right\}$ un proceso regenerativo no negativo con modificaci\'on medible. Sea $\esp\left[X\right]<\infty$. Entonces el proceso estacionario dado por la ecuaci\'on anterior est\'a bien definido y tiene funci\'on de distribuci\'on independiente de $t$, adem\'as
\begin{itemize}
\item[i)] \begin{eqnarray*}
\esp\left[V^{*}\left(0\right)\right]&=&\frac{\esp\left[\int_{0}^{X}V\left(s\right)ds\right]}{\esp\left[X\right]}\end{eqnarray*}
\item[ii)] Si $\esp\left[V^{*}\left(0\right)\right]<\infty$, equivalentemente, si $\esp\left[\int_{0}^{X}V\left(s\right)ds\right]<\infty$,entonces
\begin{eqnarray*}
\frac{\int_{0}^{t}V\left(s\right)ds}{t}\rightarrow\frac{\esp\left[\int_{0}^{X}V\left(s\right)ds\right]}{\esp\left[X\right]}
\end{eqnarray*}
con probabilidad 1 y en media, cuando $t\rightarrow\infty$.
\end{itemize}
\end{Teo}
%
%___________________________________________________________________________________________
%\vspace{5.5cm}
%\chapter{Cadenas de Markov estacionarias}
%\vspace{-1.0cm}


%__________________________________________________________________________________________
%\subsection{Procesos Regenerativos Estacionarios - Stidham \cite{Stidham}}
%__________________________________________________________________________________________


Un proceso estoc\'astico a tiempo continuo $\left\{V\left(t\right),t\geq0\right\}$ es un proceso regenerativo si existe una sucesi\'on de variables aleatorias independientes e id\'enticamente distribuidas $\left\{X_{1},X_{2},\ldots\right\}$, sucesi\'on de renovaci\'on, tal que para cualquier conjunto de Borel $A$, 

\begin{eqnarray*}
\prob\left\{V\left(t\right)\in A|X_{1}+X_{2}+\cdots+X_{R\left(t\right)}=s,\left\{V\left(\tau\right),\tau<s\right\}\right\}=\prob\left\{V\left(t-s\right)\in A|X_{1}>t-s\right\},
\end{eqnarray*}
para todo $0\leq s\leq t$, donde $R\left(t\right)=\max\left\{X_{1}+X_{2}+\cdots+X_{j}\leq t\right\}=$n\'umero de renovaciones ({\emph{puntos de regeneraci\'on}}) que ocurren en $\left[0,t\right]$. El intervalo $\left[0,X_{1}\right)$ es llamado {\emph{primer ciclo de regeneraci\'on}} de $\left\{V\left(t \right),t\geq0\right\}$, $\left[X_{1},X_{1}+X_{2}\right)$ el {\emph{segundo ciclo de regeneraci\'on}}, y as\'i sucesivamente.

Sea $X=X_{1}$ y sea $F$ la funci\'on de distrbuci\'on de $X$


\begin{Def}
Se define el proceso estacionario, $\left\{V^{*}\left(t\right),t\geq0\right\}$, para $\left\{V\left(t\right),t\geq0\right\}$ por

\begin{eqnarray*}
\prob\left\{V\left(t\right)\in A\right\}=\frac{1}{\esp\left[X\right]}\int_{0}^{\infty}\prob\left\{V\left(t+x\right)\in A|X>x\right\}\left(1-F\left(x\right)\right)dx,
\end{eqnarray*} 
para todo $t\geq0$ y todo conjunto de Borel $A$.
\end{Def}

\begin{Def}
Una distribuci\'on se dice que es {\emph{aritm\'etica}} si todos sus puntos de incremento son m\'ultiplos de la forma $0,\lambda, 2\lambda,\ldots$ para alguna $\lambda>0$ entera.
\end{Def}


\begin{Def}
Una modificaci\'on medible de un proceso $\left\{V\left(t\right),t\geq0\right\}$, es una versi\'on de este, $\left\{V\left(t,w\right)\right\}$ conjuntamente medible para $t\geq0$ y para $w\in S$, $S$ espacio de estados para $\left\{V\left(t\right),t\geq0\right\}$.
\end{Def}

\begin{Teo}
Sea $\left\{V\left(t\right),t\geq\right\}$ un proceso regenerativo no negativo con modificaci\'on medible. Sea $\esp\left[X\right]<\infty$. Entonces el proceso estacionario dado por la ecuaci\'on anterior est\'a bien definido y tiene funci\'on de distribuci\'on independiente de $t$, adem\'as
\begin{itemize}
\item[i)] \begin{eqnarray*}
\esp\left[V^{*}\left(0\right)\right]&=&\frac{\esp\left[\int_{0}^{X}V\left(s\right)ds\right]}{\esp\left[X\right]}\end{eqnarray*}
\item[ii)] Si $\esp\left[V^{*}\left(0\right)\right]<\infty$, equivalentemente, si $\esp\left[\int_{0}^{X}V\left(s\right)ds\right]<\infty$,entonces
\begin{eqnarray*}
\frac{\int_{0}^{t}V\left(s\right)ds}{t}\rightarrow\frac{\esp\left[\int_{0}^{X}V\left(s\right)ds\right]}{\esp\left[X\right]}
\end{eqnarray*}
con probabilidad 1 y en media, cuando $t\rightarrow\infty$.
\end{itemize}
\end{Teo}

Sea la funci\'on generadora de momentos para $L_{i}$, el n\'umero de usuarios en la cola $Q_{i}\left(z\right)$ en cualquier momento, est\'a dada por el tiempo promedio de $z^{L_{i}\left(t\right)}$ sobre el ciclo regenerativo definido anteriormente. Entonces 



Es decir, es posible determinar las longitudes de las colas a cualquier tiempo $t$. Entonces, determinando el primer momento es posible ver que


\begin{Def}
El tiempo de Ciclo $C_{i}$ es el periodo de tiempo que comienza cuando la cola $i$ es visitada por primera vez en un ciclo, y termina cuando es visitado nuevamente en el pr\'oximo ciclo. La duraci\'on del mismo est\'a dada por $\tau_{i}\left(m+1\right)-\tau_{i}\left(m\right)$, o equivalentemente $\overline{\tau}_{i}\left(m+1\right)-\overline{\tau}_{i}\left(m\right)$ bajo condiciones de estabilidad.
\end{Def}


\begin{Def}
El tiempo de intervisita $I_{i}$ es el periodo de tiempo que comienza cuando se ha completado el servicio en un ciclo y termina cuando es visitada nuevamente en el pr\'oximo ciclo. Su  duraci\'on del mismo est\'a dada por $\tau_{i}\left(m+1\right)-\overline{\tau}_{i}\left(m\right)$.
\end{Def}

La duraci\'on del tiempo de intervisita es $\tau_{i}\left(m+1\right)-\overline{\tau}\left(m\right)$. Dado que el n\'umero de usuarios presentes en $Q_{i}$ al tiempo $t=\tau_{i}\left(m+1\right)$ es igual al n\'umero de arribos durante el intervalo de tiempo $\left[\overline{\tau}\left(m\right),\tau_{i}\left(m+1\right)\right]$ se tiene que


\begin{eqnarray*}
\esp\left[z_{i}^{L_{i}\left(\tau_{i}\left(m+1\right)\right)}\right]=\esp\left[\left\{P_{i}\left(z_{i}\right)\right\}^{\tau_{i}\left(m+1\right)-\overline{\tau}\left(m\right)}\right]
\end{eqnarray*}

entonces, si $I_{i}\left(z\right)=\esp\left[z^{\tau_{i}\left(m+1\right)-\overline{\tau}\left(m\right)}\right]$
se tiene que $F_{i}\left(z\right)=I_{i}\left[P_{i}\left(z\right)\right]$
para $i=1,2$.

Conforme a la definici\'on dada al principio del cap\'itulo, definici\'on (\ref{Def.Tn}), sean $T_{1},T_{2},\ldots$ los puntos donde las longitudes de las colas de la red de sistemas de visitas c\'iclicas son cero simult\'aneamente, cuando la cola $Q_{j}$ es visitada por el servidor para dar servicio, es decir, $L_{1}\left(T_{i}\right)=0,L_{2}\left(T_{i}\right)=0,\hat{L}_{1}\left(T_{i}\right)=0$ y $\hat{L}_{2}\left(T_{i}\right)=0$, a estos puntos se les denominar\'a puntos regenerativos. Entonces, 

\begin{Def}
Al intervalo de tiempo entre dos puntos regenerativos se le llamar\'a ciclo regenerativo.
\end{Def}

\begin{Def}
Para $T_{i}$ se define, $M_{i}$, el n\'umero de ciclos de visita a la cola $Q_{l}$, durante el ciclo regenerativo, es decir, $M_{i}$ es un proceso de renovaci\'on.
\end{Def}

\begin{Def}
Para cada uno de los $M_{i}$'s, se definen a su vez la duraci\'on de cada uno de estos ciclos de visita en el ciclo regenerativo, $C_{i}^{(m)}$, para $m=1,2,\ldots,M_{i}$, que a su vez, tambi\'en es n proceso de renovaci\'on.
\end{Def}

\footnote{In Stidham and  Heyman \cite{Stidham} shows that is sufficient for the regenerative process to be stationary that the mean regenerative cycle time is finite: $\esp\left[\sum_{m=1}^{M_{i}}C_{i}^{(m)}\right]<\infty$, 


 como cada $C_{i}^{(m)}$ contiene intervalos de r\'eplica positivos, se tiene que $\esp\left[M_{i}\right]<\infty$, adem\'as, como $M_{i}>0$, se tiene que la condici\'on anterior es equivalente a tener que $\esp\left[C_{i}\right]<\infty$,
por lo tanto una condici\'on suficiente para la existencia del proceso regenerativo est\'a dada por $\sum_{k=1}^{N}\mu_{k}<1.$}
%________________________________________________________________________
\subsection{Procesos Regenerativos Sigman, Thorisson y Wolff \cite{Sigman2}}
%________________________________________________________________________


\begin{Def}[Definici\'on Cl\'asica]
Un proceso estoc\'astico $X=\left\{X\left(t\right):t\geq0\right\}$ es llamado regenerativo is existe una variable aleatoria $R_{1}>0$ tal que
\begin{itemize}
\item[i)] $\left\{X\left(t+R_{1}\right):t\geq0\right\}$ es independiente de $\left\{\left\{X\left(t\right):t<R_{1}\right\},\right\}$
\item[ii)] $\left\{X\left(t+R_{1}\right):t\geq0\right\}$ es estoc\'asticamente equivalente a $\left\{X\left(t\right):t>0\right\}$
\end{itemize}

Llamamos a $R_{1}$ tiempo de regeneraci\'on, y decimos que $X$ se regenera en este punto.
\end{Def}

$\left\{X\left(t+R_{1}\right)\right\}$ es regenerativo con tiempo de regeneraci\'on $R_{2}$, independiente de $R_{1}$ pero con la misma distribuci\'on que $R_{1}$. Procediendo de esta manera se obtiene una secuencia de variables aleatorias independientes e id\'enticamente distribuidas $\left\{R_{n}\right\}$ llamados longitudes de ciclo. Si definimos a $Z_{k}\equiv R_{1}+R_{2}+\cdots+R_{k}$, se tiene un proceso de renovaci\'on llamado proceso de renovaci\'on encajado para $X$.


\begin{Note}
La existencia de un primer tiempo de regeneraci\'on, $R_{1}$, implica la existencia de una sucesi\'on completa de estos tiempos $R_{1},R_{2}\ldots,$ que satisfacen la propiedad deseada \cite{Sigman2}.
\end{Note}


\begin{Note} Para la cola $GI/GI/1$ los usuarios arriban con tiempos $t_{n}$ y son atendidos con tiempos de servicio $S_{n}$, los tiempos de arribo forman un proceso de renovaci\'on  con tiempos entre arribos independientes e identicamente distribuidos (\texttt{i.i.d.})$T_{n}=t_{n}-t_{n-1}$, adem\'as los tiempos de servicio son \texttt{i.i.d.} e independientes de los procesos de arribo. Por \textit{estable} se entiende que $\esp S_{n}<\esp T_{n}<\infty$.
\end{Note}
 


\begin{Def}
Para $x$ fijo y para cada $t\geq0$, sea $I_{x}\left(t\right)=1$ si $X\left(t\right)\leq x$,  $I_{x}\left(t\right)=0$ en caso contrario, y def\'inanse los tiempos promedio
\begin{eqnarray*}
\overline{X}&=&lim_{t\rightarrow\infty}\frac{1}{t}\int_{0}^{\infty}X\left(u\right)du\\
\prob\left(X_{\infty}\leq x\right)&=&lim_{t\rightarrow\infty}\frac{1}{t}\int_{0}^{\infty}I_{x}\left(u\right)du,
\end{eqnarray*}
cuando estos l\'imites existan.
\end{Def}

Como consecuencia del teorema de Renovaci\'on-Recompensa, se tiene que el primer l\'imite  existe y es igual a la constante
\begin{eqnarray*}
\overline{X}&=&\frac{\esp\left[\int_{0}^{R_{1}}X\left(t\right)dt\right]}{\esp\left[R_{1}\right]},
\end{eqnarray*}
suponiendo que ambas esperanzas son finitas.
 
\begin{Note}
Funciones de procesos regenerativos son regenerativas, es decir, si $X\left(t\right)$ es regenerativo y se define el proceso $Y\left(t\right)$ por $Y\left(t\right)=f\left(X\left(t\right)\right)$ para alguna funci\'on Borel medible $f\left(\cdot\right)$. Adem\'as $Y$ es regenerativo con los mismos tiempos de renovaci\'on que $X$. 

En general, los tiempos de renovaci\'on, $Z_{k}$ de un proceso regenerativo no requieren ser tiempos de paro con respecto a la evoluci\'on de $X\left(t\right)$.
\end{Note} 

\begin{Note}
Una funci\'on de un proceso de Markov, usualmente no ser\'a un proceso de Markov, sin embargo ser\'a regenerativo si el proceso de Markov lo es.
\end{Note}

 
\begin{Note}
Un proceso regenerativo con media de la longitud de ciclo finita es llamado positivo recurrente.
\end{Note}


\begin{Note}
\begin{itemize}
\item[a)] Si el proceso regenerativo $X$ es positivo recurrente y tiene trayectorias muestrales no negativas, entonces la ecuaci\'on anterior es v\'alida.
\item[b)] Si $X$ es positivo recurrente regenerativo, podemos construir una \'unica versi\'on estacionaria de este proceso, $X_{e}=\left\{X_{e}\left(t\right)\right\}$, donde $X_{e}$ es un proceso estoc\'astico regenerativo y estrictamente estacionario, con distribuci\'on marginal distribuida como $X_{\infty}$
\end{itemize}
\end{Note}



%________________________________________________________________________
\subsection{Procesos Regenerativos}
%________________________________________________________________________



\begin{Note}
Si $\tilde{X}\left(t\right)$ con espacio de estados $\tilde{S}$ es regenerativo sobre $T_{n}$, entonces $X\left(t\right)=f\left(\tilde{X}\left(t\right)\right)$ tambi\'en es regenerativo sobre $T_{n}$, para cualquier funci\'on $f:\tilde{S}\rightarrow S$.
\end{Note}

\begin{Note}
Los procesos regenerativos son crudamente regenerativos, pero no al rev\'es.
\end{Note}

%______________________________________________________________________
\subsection*{Procesos Regenerativos: Sigman\cite{Sigman1}}
%______________________________________________________________________
\begin{Def}[Definici\'on Cl\'asica]
Un proceso estoc\'astico $X=\left\{X\left(t\right):t\geq0\right\}$ es llamado regenerativo is existe una variable aleatoria $R_{1}>0$ tal que
\begin{itemize}
\item[i)] $\left\{X\left(t+R_{1}\right):t\geq0\right\}$ es independiente de $\left\{\left\{X\left(t\right):t<R_{1}\right\},\right\}$
\item[ii)] $\left\{X\left(t+R_{1}\right):t\geq0\right\}$ es estoc\'asticamente equivalente a $\left\{X\left(t\right):t>0\right\}$
\end{itemize}

Llamamos a $R_{1}$ tiempo de regeneraci\'on, y decimos que $X$ se regenera en este punto.
\end{Def}

$\left\{X\left(t+R_{1}\right)\right\}$ es regenerativo con tiempo de regeneraci\'on $R_{2}$, independiente de $R_{1}$ pero con la misma distribuci\'on que $R_{1}$. Procediendo de esta manera se obtiene una secuencia de variables aleatorias independientes e id\'enticamente distribuidas $\left\{R_{n}\right\}$ llamados longitudes de ciclo. Si definimos a $Z_{k}\equiv R_{1}+R_{2}+\cdots+R_{k}$, se tiene un proceso de renovaci\'on llamado proceso de renovaci\'on encajado para $X$.




\begin{Def}
Para $x$ fijo y para cada $t\geq0$, sea $I_{x}\left(t\right)=1$ si $X\left(t\right)\leq x$,  $I_{x}\left(t\right)=0$ en caso contrario, y def\'inanse los tiempos promedio
\begin{eqnarray*}
\overline{X}&=&lim_{t\rightarrow\infty}\frac{1}{t}\int_{0}^{\infty}X\left(u\right)du\\
\prob\left(X_{\infty}\leq x\right)&=&lim_{t\rightarrow\infty}\frac{1}{t}\int_{0}^{\infty}I_{x}\left(u\right)du,
\end{eqnarray*}
cuando estos l\'imites existan.
\end{Def}

Como consecuencia del teorema de Renovaci\'on-Recompensa, se tiene que el primer l\'imite  existe y es igual a la constante
\begin{eqnarray*}
\overline{X}&=&\frac{\esp\left[\int_{0}^{R_{1}}X\left(t\right)dt\right]}{\esp\left[R_{1}\right]},
\end{eqnarray*}
suponiendo que ambas esperanzas son finitas.

\begin{Note}
\begin{itemize}
\item[a)] Si el proceso regenerativo $X$ es positivo recurrente y tiene trayectorias muestrales no negativas, entonces la ecuaci\'on anterior es v\'alida.
\item[b)] Si $X$ es positivo recurrente regenerativo, podemos construir una \'unica versi\'on estacionaria de este proceso, $X_{e}=\left\{X_{e}\left(t\right)\right\}$, donde $X_{e}$ es un proceso estoc\'astico regenerativo y estrictamente estacionario, con distribuci\'on marginal distribuida como $X_{\infty}$
\end{itemize}
\end{Note}


%__________________________________________________________________________________________
\subsection{Procesos Regenerativos Estacionarios - Stidham \cite{Stidham}}
%__________________________________________________________________________________________


Un proceso estoc\'astico a tiempo continuo $\left\{V\left(t\right),t\geq0\right\}$ es un proceso regenerativo si existe una sucesi\'on de variables aleatorias independientes e id\'enticamente distribuidas $\left\{X_{1},X_{2},\ldots\right\}$, sucesi\'on de renovaci\'on, tal que para cualquier conjunto de Borel $A$, 

\begin{eqnarray*}
\prob\left\{V\left(t\right)\in A|X_{1}+X_{2}+\cdots+X_{R\left(t\right)}=s,\left\{V\left(\tau\right),\tau<s\right\}\right\}=\prob\left\{V\left(t-s\right)\in A|X_{1}>t-s\right\},
\end{eqnarray*}
para todo $0\leq s\leq t$, donde $R\left(t\right)=\max\left\{X_{1}+X_{2}+\cdots+X_{j}\leq t\right\}=$n\'umero de renovaciones ({\emph{puntos de regeneraci\'on}}) que ocurren en $\left[0,t\right]$. El intervalo $\left[0,X_{1}\right)$ es llamado {\emph{primer ciclo de regeneraci\'on}} de $\left\{V\left(t \right),t\geq0\right\}$, $\left[X_{1},X_{1}+X_{2}\right)$ el {\emph{segundo ciclo de regeneraci\'on}}, y as\'i sucesivamente.

Sea $X=X_{1}$ y sea $F$ la funci\'on de distrbuci\'on de $X$


\begin{Def}
Se define el proceso estacionario, $\left\{V^{*}\left(t\right),t\geq0\right\}$, para $\left\{V\left(t\right),t\geq0\right\}$ por

\begin{eqnarray*}
\prob\left\{V\left(t\right)\in A\right\}=\frac{1}{\esp\left[X\right]}\int_{0}^{\infty}\prob\left\{V\left(t+x\right)\in A|X>x\right\}\left(1-F\left(x\right)\right)dx,
\end{eqnarray*} 
para todo $t\geq0$ y todo conjunto de Borel $A$.
\end{Def}

\begin{Def}
Una distribuci\'on se dice que es {\emph{aritm\'etica}} si todos sus puntos de incremento son m\'ultiplos de la forma $0,\lambda, 2\lambda,\ldots$ para alguna $\lambda>0$ entera.
\end{Def}


\begin{Def}
Una modificaci\'on medible de un proceso $\left\{V\left(t\right),t\geq0\right\}$, es una versi\'on de este, $\left\{V\left(t,w\right)\right\}$ conjuntamente medible para $t\geq0$ y para $w\in S$, $S$ espacio de estados para $\left\{V\left(t\right),t\geq0\right\}$.
\end{Def}

\begin{Teo}
Sea $\left\{V\left(t\right),t\geq\right\}$ un proceso regenerativo no negativo con modificaci\'on medible. Sea $\esp\left[X\right]<\infty$. Entonces el proceso estacionario dado por la ecuaci\'on anterior est\'a bien definido y tiene funci\'on de distribuci\'on independiente de $t$, adem\'as
\begin{itemize}
\item[i)] \begin{eqnarray*}
\esp\left[V^{*}\left(0\right)\right]&=&\frac{\esp\left[\int_{0}^{X}V\left(s\right)ds\right]}{\esp\left[X\right]}\end{eqnarray*}
\item[ii)] Si $\esp\left[V^{*}\left(0\right)\right]<\infty$, equivalentemente, si $\esp\left[\int_{0}^{X}V\left(s\right)ds\right]<\infty$,entonces
\begin{eqnarray*}
\frac{\int_{0}^{t}V\left(s\right)ds}{t}\rightarrow\frac{\esp\left[\int_{0}^{X}V\left(s\right)ds\right]}{\esp\left[X\right]}
\end{eqnarray*}
con probabilidad 1 y en media, cuando $t\rightarrow\infty$.
\end{itemize}
\end{Teo}

%________________________________________________________________________
\subsection{Procesos Regenerativos}
%________________________________________________________________________

Para $\left\{X\left(t\right):t\geq0\right\}$ Proceso Estoc\'astico a tiempo continuo con estado de espacios $S$, que es un espacio m\'etrico, con trayectorias continuas por la derecha y con l\'imites por la izquierda c.s. Sea $N\left(t\right)$ un proceso de renovaci\'on en $\rea_{+}$ definido en el mismo espacio de probabilidad que $X\left(t\right)$, con tiempos de renovaci\'on $T$ y tiempos de inter-renovaci\'on $\xi_{n}=T_{n}-T_{n-1}$, con misma distribuci\'on $F$ de media finita $\mu$.



\begin{Def}
Para el proceso $\left\{\left(N\left(t\right),X\left(t\right)\right):t\geq0\right\}$, sus trayectoria muestrales en el intervalo de tiempo $\left[T_{n-1},T_{n}\right)$ est\'an descritas por
\begin{eqnarray*}
\zeta_{n}=\left(\xi_{n},\left\{X\left(T_{n-1}+t\right):0\leq t<\xi_{n}\right\}\right)
\end{eqnarray*}
Este $\zeta_{n}$ es el $n$-\'esimo segmento del proceso. El proceso es regenerativo sobre los tiempos $T_{n}$ si sus segmentos $\zeta_{n}$ son independientes e id\'enticamennte distribuidos.
\end{Def}


\begin{Obs}
Si $\tilde{X}\left(t\right)$ con espacio de estados $\tilde{S}$ es regenerativo sobre $T_{n}$, entonces $X\left(t\right)=f\left(\tilde{X}\left(t\right)\right)$ tambi\'en es regenerativo sobre $T_{n}$, para cualquier funci\'on $f:\tilde{S}\rightarrow S$.
\end{Obs}

\begin{Obs}
Los procesos regenerativos son crudamente regenerativos, pero no al rev\'es.
\end{Obs}

\begin{Def}[Definici\'on Cl\'asica]
Un proceso estoc\'astico $X=\left\{X\left(t\right):t\geq0\right\}$ es llamado regenerativo is existe una variable aleatoria $R_{1}>0$ tal que
\begin{itemize}
\item[i)] $\left\{X\left(t+R_{1}\right):t\geq0\right\}$ es independiente de $\left\{\left\{X\left(t\right):t<R_{1}\right\},\right\}$
\item[ii)] $\left\{X\left(t+R_{1}\right):t\geq0\right\}$ es estoc\'asticamente equivalente a $\left\{X\left(t\right):t>0\right\}$
\end{itemize}

Llamamos a $R_{1}$ tiempo de regeneraci\'on, y decimos que $X$ se regenera en este punto.
\end{Def}

$\left\{X\left(t+R_{1}\right)\right\}$ es regenerativo con tiempo de regeneraci\'on $R_{2}$, independiente de $R_{1}$ pero con la misma distribuci\'on que $R_{1}$. Procediendo de esta manera se obtiene una secuencia de variables aleatorias independientes e id\'enticamente distribuidas $\left\{R_{n}\right\}$ llamados longitudes de ciclo. Si definimos a $Z_{k}\equiv R_{1}+R_{2}+\cdots+R_{k}$, se tiene un proceso de renovaci\'on llamado proceso de renovaci\'on encajado para $X$.

\begin{Note}
Un proceso regenerativo con media de la longitud de ciclo finita es llamado positivo recurrente.
\end{Note}


\begin{Def}
Para $x$ fijo y para cada $t\geq0$, sea $I_{x}\left(t\right)=1$ si $X\left(t\right)\leq x$,  $I_{x}\left(t\right)=0$ en caso contrario, y def\'inanse los tiempos promedio
\begin{eqnarray*}
\overline{X}&=&lim_{t\rightarrow\infty}\frac{1}{t}\int_{0}^{\infty}X\left(u\right)du\\
\prob\left(X_{\infty}\leq x\right)&=&lim_{t\rightarrow\infty}\frac{1}{t}\int_{0}^{\infty}I_{x}\left(u\right)du,
\end{eqnarray*}
cuando estos l\'imites existan.
\end{Def}

Como consecuencia del teorema de Renovaci\'on-Recompensa, se tiene que el primer l\'imite  existe y es igual a la constante
\begin{eqnarray*}
\overline{X}&=&\frac{\esp\left[\int_{0}^{R_{1}}X\left(t\right)dt\right]}{\esp\left[R_{1}\right]},
\end{eqnarray*}
suponiendo que ambas esperanzas son finitas.

\begin{Note}
\begin{itemize}
\item[a)] Si el proceso regenerativo $X$ es positivo recurrente y tiene trayectorias muestrales no negativas, entonces la ecuaci\'on anterior es v\'alida.
\item[b)] Si $X$ es positivo recurrente regenerativo, podemos construir una \'unica versi\'on estacionaria de este proceso, $X_{e}=\left\{X_{e}\left(t\right)\right\}$, donde $X_{e}$ es un proceso estoc\'astico regenerativo y estrictamente estacionario, con distribuci\'on marginal distribuida como $X_{\infty}$
\end{itemize}
\end{Note}




%________________________________________________________________________
\subsection{Procesos Regenerativos Sigman, Thorisson y Wolff \cite{Sigman2}}
%________________________________________________________________________


\begin{Def}[Definici\'on Cl\'asica]
Un proceso estoc\'astico $X=\left\{X\left(t\right):t\geq0\right\}$ es llamado regenerativo is existe una variable aleatoria $R_{1}>0$ tal que
\begin{itemize}
\item[i)] $\left\{X\left(t+R_{1}\right):t\geq0\right\}$ es independiente de $\left\{\left\{X\left(t\right):t<R_{1}\right\},\right\}$
\item[ii)] $\left\{X\left(t+R_{1}\right):t\geq0\right\}$ es estoc\'asticamente equivalente a $\left\{X\left(t\right):t>0\right\}$
\end{itemize}

Llamamos a $R_{1}$ tiempo de regeneraci\'on, y decimos que $X$ se regenera en este punto.
\end{Def}

$\left\{X\left(t+R_{1}\right)\right\}$ es regenerativo con tiempo de regeneraci\'on $R_{2}$, independiente de $R_{1}$ pero con la misma distribuci\'on que $R_{1}$. Procediendo de esta manera se obtiene una secuencia de variables aleatorias independientes e id\'enticamente distribuidas $\left\{R_{n}\right\}$ llamados longitudes de ciclo. Si definimos a $Z_{k}\equiv R_{1}+R_{2}+\cdots+R_{k}$, se tiene un proceso de renovaci\'on llamado proceso de renovaci\'on encajado para $X$.


\begin{Note}
La existencia de un primer tiempo de regeneraci\'on, $R_{1}$, implica la existencia de una sucesi\'on completa de estos tiempos $R_{1},R_{2}\ldots,$ que satisfacen la propiedad deseada \cite{Sigman2}.
\end{Note}


\begin{Note} Para la cola $GI/GI/1$ los usuarios arriban con tiempos $t_{n}$ y son atendidos con tiempos de servicio $S_{n}$, los tiempos de arribo forman un proceso de renovaci\'on  con tiempos entre arribos independientes e identicamente distribuidos (\texttt{i.i.d.})$T_{n}=t_{n}-t_{n-1}$, adem\'as los tiempos de servicio son \texttt{i.i.d.} e independientes de los procesos de arribo. Por \textit{estable} se entiende que $\esp S_{n}<\esp T_{n}<\infty$.
\end{Note}
 


\begin{Def}
Para $x$ fijo y para cada $t\geq0$, sea $I_{x}\left(t\right)=1$ si $X\left(t\right)\leq x$,  $I_{x}\left(t\right)=0$ en caso contrario, y def\'inanse los tiempos promedio
\begin{eqnarray*}
\overline{X}&=&lim_{t\rightarrow\infty}\frac{1}{t}\int_{0}^{\infty}X\left(u\right)du\\
\prob\left(X_{\infty}\leq x\right)&=&lim_{t\rightarrow\infty}\frac{1}{t}\int_{0}^{\infty}I_{x}\left(u\right)du,
\end{eqnarray*}
cuando estos l\'imites existan.
\end{Def}

Como consecuencia del teorema de Renovaci\'on-Recompensa, se tiene que el primer l\'imite  existe y es igual a la constante
\begin{eqnarray*}
\overline{X}&=&\frac{\esp\left[\int_{0}^{R_{1}}X\left(t\right)dt\right]}{\esp\left[R_{1}\right]},
\end{eqnarray*}
suponiendo que ambas esperanzas son finitas.
 
\begin{Note}
Funciones de procesos regenerativos son regenerativas, es decir, si $X\left(t\right)$ es regenerativo y se define el proceso $Y\left(t\right)$ por $Y\left(t\right)=f\left(X\left(t\right)\right)$ para alguna funci\'on Borel medible $f\left(\cdot\right)$. Adem\'as $Y$ es regenerativo con los mismos tiempos de renovaci\'on que $X$. 

En general, los tiempos de renovaci\'on, $Z_{k}$ de un proceso regenerativo no requieren ser tiempos de paro con respecto a la evoluci\'on de $X\left(t\right)$.
\end{Note} 

\begin{Note}
Una funci\'on de un proceso de Markov, usualmente no ser\'a un proceso de Markov, sin embargo ser\'a regenerativo si el proceso de Markov lo es.
\end{Note}

 
\begin{Note}
Un proceso regenerativo con media de la longitud de ciclo finita es llamado positivo recurrente.
\end{Note}


\begin{Note}
\begin{itemize}
\item[a)] Si el proceso regenerativo $X$ es positivo recurrente y tiene trayectorias muestrales no negativas, entonces la ecuaci\'on anterior es v\'alida.
\item[b)] Si $X$ es positivo recurrente regenerativo, podemos construir una \'unica versi\'on estacionaria de este proceso, $X_{e}=\left\{X_{e}\left(t\right)\right\}$, donde $X_{e}$ es un proceso estoc\'astico regenerativo y estrictamente estacionario, con distribuci\'on marginal distribuida como $X_{\infty}$
\end{itemize}
\end{Note}


%__________________________________________________________________________________________
\subsection{Procesos Regenerativos Estacionarios - Stidham \cite{Stidham}}
%__________________________________________________________________________________________


Un proceso estoc\'astico a tiempo continuo $\left\{V\left(t\right),t\geq0\right\}$ es un proceso regenerativo si existe una sucesi\'on de variables aleatorias independientes e id\'enticamente distribuidas $\left\{X_{1},X_{2},\ldots\right\}$, sucesi\'on de renovaci\'on, tal que para cualquier conjunto de Borel $A$, 

\begin{eqnarray*}
\prob\left\{V\left(t\right)\in A|X_{1}+X_{2}+\cdots+X_{R\left(t\right)}=s,\left\{V\left(\tau\right),\tau<s\right\}\right\}=\prob\left\{V\left(t-s\right)\in A|X_{1}>t-s\right\},
\end{eqnarray*}
para todo $0\leq s\leq t$, donde $R\left(t\right)=\max\left\{X_{1}+X_{2}+\cdots+X_{j}\leq t\right\}=$n\'umero de renovaciones ({\emph{puntos de regeneraci\'on}}) que ocurren en $\left[0,t\right]$. El intervalo $\left[0,X_{1}\right)$ es llamado {\emph{primer ciclo de regeneraci\'on}} de $\left\{V\left(t \right),t\geq0\right\}$, $\left[X_{1},X_{1}+X_{2}\right)$ el {\emph{segundo ciclo de regeneraci\'on}}, y as\'i sucesivamente.

Sea $X=X_{1}$ y sea $F$ la funci\'on de distrbuci\'on de $X$


\begin{Def}
Se define el proceso estacionario, $\left\{V^{*}\left(t\right),t\geq0\right\}$, para $\left\{V\left(t\right),t\geq0\right\}$ por

\begin{eqnarray*}
\prob\left\{V\left(t\right)\in A\right\}=\frac{1}{\esp\left[X\right]}\int_{0}^{\infty}\prob\left\{V\left(t+x\right)\in A|X>x\right\}\left(1-F\left(x\right)\right)dx,
\end{eqnarray*} 
para todo $t\geq0$ y todo conjunto de Borel $A$.
\end{Def}

\begin{Def}
Una distribuci\'on se dice que es {\emph{aritm\'etica}} si todos sus puntos de incremento son m\'ultiplos de la forma $0,\lambda, 2\lambda,\ldots$ para alguna $\lambda>0$ entera.
\end{Def}


\begin{Def}
Una modificaci\'on medible de un proceso $\left\{V\left(t\right),t\geq0\right\}$, es una versi\'on de este, $\left\{V\left(t,w\right)\right\}$ conjuntamente medible para $t\geq0$ y para $w\in S$, $S$ espacio de estados para $\left\{V\left(t\right),t\geq0\right\}$.
\end{Def}

\begin{Teo}
Sea $\left\{V\left(t\right),t\geq\right\}$ un proceso regenerativo no negativo con modificaci\'on medible. Sea $\esp\left[X\right]<\infty$. Entonces el proceso estacionario dado por la ecuaci\'on anterior est\'a bien definido y tiene funci\'on de distribuci\'on independiente de $t$, adem\'as
\begin{itemize}
\item[i)] \begin{eqnarray*}
\esp\left[V^{*}\left(0\right)\right]&=&\frac{\esp\left[\int_{0}^{X}V\left(s\right)ds\right]}{\esp\left[X\right]}\end{eqnarray*}
\item[ii)] Si $\esp\left[V^{*}\left(0\right)\right]<\infty$, equivalentemente, si $\esp\left[\int_{0}^{X}V\left(s\right)ds\right]<\infty$,entonces
\begin{eqnarray*}
\frac{\int_{0}^{t}V\left(s\right)ds}{t}\rightarrow\frac{\esp\left[\int_{0}^{X}V\left(s\right)ds\right]}{\esp\left[X\right]}
\end{eqnarray*}
con probabilidad 1 y en media, cuando $t\rightarrow\infty$.
\end{itemize}
\end{Teo}

\begin{Coro}
Sea $\left\{V\left(t\right),t\geq0\right\}$ un proceso regenerativo no negativo, con modificaci\'on medible. Si $\esp <\infty$, $F$ es no-aritm\'etica, y para todo $x\geq0$, $P\left\{V\left(t\right)\leq x,C>x\right\}$ es de variaci\'on acotada como funci\'on de $t$ en cada intervalo finito $\left[0,\tau\right]$, entonces $V\left(t\right)$ converge en distribuci\'on  cuando $t\rightarrow\infty$ y $$\esp V=\frac{\esp \int_{0}^{X}V\left(s\right)ds}{\esp X}$$
Donde $V$ tiene la distribuci\'on l\'imite de $V\left(t\right)$ cuando $t\rightarrow\infty$.

\end{Coro}

Para el caso discreto se tienen resultados similares.



%______________________________________________________________________
\subsection{Procesos de Renovaci\'on}
%______________________________________________________________________

\begin{Def}%\label{Def.Tn}
Sean $0\leq T_{1}\leq T_{2}\leq \ldots$ son tiempos aleatorios infinitos en los cuales ocurren ciertos eventos. El n\'umero de tiempos $T_{n}$ en el intervalo $\left[0,t\right)$ es

\begin{eqnarray}
N\left(t\right)=\sum_{n=1}^{\infty}\indora\left(T_{n}\leq t\right),
\end{eqnarray}
para $t\geq0$.
\end{Def}

Si se consideran los puntos $T_{n}$ como elementos de $\rea_{+}$, y $N\left(t\right)$ es el n\'umero de puntos en $\rea$. El proceso denotado por $\left\{N\left(t\right):t\geq0\right\}$, denotado por $N\left(t\right)$, es un proceso puntual en $\rea_{+}$. Los $T_{n}$ son los tiempos de ocurrencia, el proceso puntual $N\left(t\right)$ es simple si su n\'umero de ocurrencias son distintas: $0<T_{1}<T_{2}<\ldots$ casi seguramente.

\begin{Def}
Un proceso puntual $N\left(t\right)$ es un proceso de renovaci\'on si los tiempos de interocurrencia $\xi_{n}=T_{n}-T_{n-1}$, para $n\geq1$, son independientes e identicamente distribuidos con distribuci\'on $F$, donde $F\left(0\right)=0$ y $T_{0}=0$. Los $T_{n}$ son llamados tiempos de renovaci\'on, referente a la independencia o renovaci\'on de la informaci\'on estoc\'astica en estos tiempos. Los $\xi_{n}$ son los tiempos de inter-renovaci\'on, y $N\left(t\right)$ es el n\'umero de renovaciones en el intervalo $\left[0,t\right)$
\end{Def}


\begin{Note}
Para definir un proceso de renovaci\'on para cualquier contexto, solamente hay que especificar una distribuci\'on $F$, con $F\left(0\right)=0$, para los tiempos de inter-renovaci\'on. La funci\'on $F$ en turno degune las otra variables aleatorias. De manera formal, existe un espacio de probabilidad y una sucesi\'on de variables aleatorias $\xi_{1},\xi_{2},\ldots$ definidas en este con distribuci\'on $F$. Entonces las otras cantidades son $T_{n}=\sum_{k=1}^{n}\xi_{k}$ y $N\left(t\right)=\sum_{n=1}^{\infty}\indora\left(T_{n}\leq t\right)$, donde $T_{n}\rightarrow\infty$ casi seguramente por la Ley Fuerte de los Grandes Números.
\end{Note}

%___________________________________________________________________________________________
%
\subsection{Teorema Principal de Renovaci\'on}
%___________________________________________________________________________________________
%

\begin{Note} Una funci\'on $h:\rea_{+}\rightarrow\rea$ es Directamente Riemann Integrable en los siguientes casos:
\begin{itemize}
\item[a)] $h\left(t\right)\geq0$ es decreciente y Riemann Integrable.
\item[b)] $h$ es continua excepto posiblemente en un conjunto de Lebesgue de medida 0, y $|h\left(t\right)|\leq b\left(t\right)$, donde $b$ es DRI.
\end{itemize}
\end{Note}

\begin{Teo}[Teorema Principal de Renovaci\'on]
Si $F$ es no aritm\'etica y $h\left(t\right)$ es Directamente Riemann Integrable (DRI), entonces

\begin{eqnarray*}
lim_{t\rightarrow\infty}U\star h=\frac{1}{\mu}\int_{\rea_{+}}h\left(s\right)ds.
\end{eqnarray*}
\end{Teo}

\begin{Prop}
Cualquier funci\'on $H\left(t\right)$ acotada en intervalos finitos y que es 0 para $t<0$ puede expresarse como
\begin{eqnarray*}
H\left(t\right)=U\star h\left(t\right)\textrm{,  donde }h\left(t\right)=H\left(t\right)-F\star H\left(t\right)
\end{eqnarray*}
\end{Prop}

\begin{Def}
Un proceso estoc\'astico $X\left(t\right)$ es crudamente regenerativo en un tiempo aleatorio positivo $T$ si
\begin{eqnarray*}
\esp\left[X\left(T+t\right)|T\right]=\esp\left[X\left(t\right)\right]\textrm{, para }t\geq0,\end{eqnarray*}
y con las esperanzas anteriores finitas.
\end{Def}

\begin{Prop}
Sup\'ongase que $X\left(t\right)$ es un proceso crudamente regenerativo en $T$, que tiene distribuci\'on $F$. Si $\esp\left[X\left(t\right)\right]$ es acotado en intervalos finitos, entonces
\begin{eqnarray*}
\esp\left[X\left(t\right)\right]=U\star h\left(t\right)\textrm{,  donde }h\left(t\right)=\esp\left[X\left(t\right)\indora\left(T>t\right)\right].
\end{eqnarray*}
\end{Prop}

\begin{Teo}[Regeneraci\'on Cruda]
Sup\'ongase que $X\left(t\right)$ es un proceso con valores positivo crudamente regenerativo en $T$, y def\'inase $M=\sup\left\{|X\left(t\right)|:t\leq T\right\}$. Si $T$ es no aritm\'etico y $M$ y $MT$ tienen media finita, entonces
\begin{eqnarray*}
lim_{t\rightarrow\infty}\esp\left[X\left(t\right)\right]=\frac{1}{\mu}\int_{\rea_{+}}h\left(s\right)ds,
\end{eqnarray*}
donde $h\left(t\right)=\esp\left[X\left(t\right)\indora\left(T>t\right)\right]$.
\end{Teo}

%___________________________________________________________________________________________
%
\subsection{Propiedades de los Procesos de Renovaci\'on}
%___________________________________________________________________________________________
%

Los tiempos $T_{n}$ est\'an relacionados con los conteos de $N\left(t\right)$ por

\begin{eqnarray*}
\left\{N\left(t\right)\geq n\right\}&=&\left\{T_{n}\leq t\right\}\\
T_{N\left(t\right)}\leq &t&<T_{N\left(t\right)+1},
\end{eqnarray*}

adem\'as $N\left(T_{n}\right)=n$, y 

\begin{eqnarray*}
N\left(t\right)=\max\left\{n:T_{n}\leq t\right\}=\min\left\{n:T_{n+1}>t\right\}
\end{eqnarray*}

Por propiedades de la convoluci\'on se sabe que

\begin{eqnarray*}
P\left\{T_{n}\leq t\right\}=F^{n\star}\left(t\right)
\end{eqnarray*}
que es la $n$-\'esima convoluci\'on de $F$. Entonces 

\begin{eqnarray*}
\left\{N\left(t\right)\geq n\right\}&=&\left\{T_{n}\leq t\right\}\\
P\left\{N\left(t\right)\leq n\right\}&=&1-F^{\left(n+1\right)\star}\left(t\right)
\end{eqnarray*}

Adem\'as usando el hecho de que $\esp\left[N\left(t\right)\right]=\sum_{n=1}^{\infty}P\left\{N\left(t\right)\geq n\right\}$
se tiene que

\begin{eqnarray*}
\esp\left[N\left(t\right)\right]=\sum_{n=1}^{\infty}F^{n\star}\left(t\right)
\end{eqnarray*}

\begin{Prop}
Para cada $t\geq0$, la funci\'on generadora de momentos $\esp\left[e^{\alpha N\left(t\right)}\right]$ existe para alguna $\alpha$ en una vecindad del 0, y de aqu\'i que $\esp\left[N\left(t\right)^{m}\right]<\infty$, para $m\geq1$.
\end{Prop}


\begin{Note}
Si el primer tiempo de renovaci\'on $\xi_{1}$ no tiene la misma distribuci\'on que el resto de las $\xi_{n}$, para $n\geq2$, a $N\left(t\right)$ se le llama Proceso de Renovaci\'on retardado, donde si $\xi$ tiene distribuci\'on $G$, entonces el tiempo $T_{n}$ de la $n$-\'esima renovaci\'on tiene distribuci\'on $G\star F^{\left(n-1\right)\star}\left(t\right)$
\end{Note}


\begin{Teo}
Para una constante $\mu\leq\infty$ ( o variable aleatoria), las siguientes expresiones son equivalentes:

\begin{eqnarray}
lim_{n\rightarrow\infty}n^{-1}T_{n}&=&\mu,\textrm{ c.s.}\\
lim_{t\rightarrow\infty}t^{-1}N\left(t\right)&=&1/\mu,\textrm{ c.s.}
\end{eqnarray}
\end{Teo}


Es decir, $T_{n}$ satisface la Ley Fuerte de los Grandes N\'umeros s\'i y s\'olo s\'i $N\left/t\right)$ la cumple.


\begin{Coro}[Ley Fuerte de los Grandes N\'umeros para Procesos de Renovaci\'on]
Si $N\left(t\right)$ es un proceso de renovaci\'on cuyos tiempos de inter-renovaci\'on tienen media $\mu\leq\infty$, entonces
\begin{eqnarray}
t^{-1}N\left(t\right)\rightarrow 1/\mu,\textrm{ c.s. cuando }t\rightarrow\infty.
\end{eqnarray}

\end{Coro}


Considerar el proceso estoc\'astico de valores reales $\left\{Z\left(t\right):t\geq0\right\}$ en el mismo espacio de probabilidad que $N\left(t\right)$

\begin{Def}
Para el proceso $\left\{Z\left(t\right):t\geq0\right\}$ se define la fluctuaci\'on m\'axima de $Z\left(t\right)$ en el intervalo $\left(T_{n-1},T_{n}\right]$:
\begin{eqnarray*}
M_{n}=\sup_{T_{n-1}<t\leq T_{n}}|Z\left(t\right)-Z\left(T_{n-1}\right)|
\end{eqnarray*}
\end{Def}

\begin{Teo}
Sup\'ongase que $n^{-1}T_{n}\rightarrow\mu$ c.s. cuando $n\rightarrow\infty$, donde $\mu\leq\infty$ es una constante o variable aleatoria. Sea $a$ una constante o variable aleatoria que puede ser infinita cuando $\mu$ es finita, y considere las expresiones l\'imite:
\begin{eqnarray}
lim_{n\rightarrow\infty}n^{-1}Z\left(T_{n}\right)&=&a,\textrm{ c.s.}\\
lim_{t\rightarrow\infty}t^{-1}Z\left(t\right)&=&a/\mu,\textrm{ c.s.}
\end{eqnarray}
La segunda expresi\'on implica la primera. Conversamente, la primera implica la segunda si el proceso $Z\left(t\right)$ es creciente, o si $lim_{n\rightarrow\infty}n^{-1}M_{n}=0$ c.s.
\end{Teo}

\begin{Coro}
Si $N\left(t\right)$ es un proceso de renovaci\'on, y $\left(Z\left(T_{n}\right)-Z\left(T_{n-1}\right),M_{n}\right)$, para $n\geq1$, son variables aleatorias independientes e id\'enticamente distribuidas con media finita, entonces,
\begin{eqnarray}
lim_{t\rightarrow\infty}t^{-1}Z\left(t\right)\rightarrow\frac{\esp\left[Z\left(T_{1}\right)-Z\left(T_{0}\right)\right]}{\esp\left[T_{1}\right]},\textrm{ c.s. cuando  }t\rightarrow\infty.
\end{eqnarray}
\end{Coro}



%___________________________________________________________________________________________
%
%\subsection{Propiedades de los Procesos de Renovaci\'on}
%___________________________________________________________________________________________
%

Los tiempos $T_{n}$ est\'an relacionados con los conteos de $N\left(t\right)$ por

\begin{eqnarray*}
\left\{N\left(t\right)\geq n\right\}&=&\left\{T_{n}\leq t\right\}\\
T_{N\left(t\right)}\leq &t&<T_{N\left(t\right)+1},
\end{eqnarray*}

adem\'as $N\left(T_{n}\right)=n$, y 

\begin{eqnarray*}
N\left(t\right)=\max\left\{n:T_{n}\leq t\right\}=\min\left\{n:T_{n+1}>t\right\}
\end{eqnarray*}

Por propiedades de la convoluci\'on se sabe que

\begin{eqnarray*}
P\left\{T_{n}\leq t\right\}=F^{n\star}\left(t\right)
\end{eqnarray*}
que es la $n$-\'esima convoluci\'on de $F$. Entonces 

\begin{eqnarray*}
\left\{N\left(t\right)\geq n\right\}&=&\left\{T_{n}\leq t\right\}\\
P\left\{N\left(t\right)\leq n\right\}&=&1-F^{\left(n+1\right)\star}\left(t\right)
\end{eqnarray*}

Adem\'as usando el hecho de que $\esp\left[N\left(t\right)\right]=\sum_{n=1}^{\infty}P\left\{N\left(t\right)\geq n\right\}$
se tiene que

\begin{eqnarray*}
\esp\left[N\left(t\right)\right]=\sum_{n=1}^{\infty}F^{n\star}\left(t\right)
\end{eqnarray*}

\begin{Prop}
Para cada $t\geq0$, la funci\'on generadora de momentos $\esp\left[e^{\alpha N\left(t\right)}\right]$ existe para alguna $\alpha$ en una vecindad del 0, y de aqu\'i que $\esp\left[N\left(t\right)^{m}\right]<\infty$, para $m\geq1$.
\end{Prop}


\begin{Note}
Si el primer tiempo de renovaci\'on $\xi_{1}$ no tiene la misma distribuci\'on que el resto de las $\xi_{n}$, para $n\geq2$, a $N\left(t\right)$ se le llama Proceso de Renovaci\'on retardado, donde si $\xi$ tiene distribuci\'on $G$, entonces el tiempo $T_{n}$ de la $n$-\'esima renovaci\'on tiene distribuci\'on $G\star F^{\left(n-1\right)\star}\left(t\right)$
\end{Note}


\begin{Teo}
Para una constante $\mu\leq\infty$ ( o variable aleatoria), las siguientes expresiones son equivalentes:

\begin{eqnarray}
lim_{n\rightarrow\infty}n^{-1}T_{n}&=&\mu,\textrm{ c.s.}\\
lim_{t\rightarrow\infty}t^{-1}N\left(t\right)&=&1/\mu,\textrm{ c.s.}
\end{eqnarray}
\end{Teo}


Es decir, $T_{n}$ satisface la Ley Fuerte de los Grandes N\'umeros s\'i y s\'olo s\'i $N\left/t\right)$ la cumple.


\begin{Coro}[Ley Fuerte de los Grandes N\'umeros para Procesos de Renovaci\'on]
Si $N\left(t\right)$ es un proceso de renovaci\'on cuyos tiempos de inter-renovaci\'on tienen media $\mu\leq\infty$, entonces
\begin{eqnarray}
t^{-1}N\left(t\right)\rightarrow 1/\mu,\textrm{ c.s. cuando }t\rightarrow\infty.
\end{eqnarray}

\end{Coro}


Considerar el proceso estoc\'astico de valores reales $\left\{Z\left(t\right):t\geq0\right\}$ en el mismo espacio de probabilidad que $N\left(t\right)$

\begin{Def}
Para el proceso $\left\{Z\left(t\right):t\geq0\right\}$ se define la fluctuaci\'on m\'axima de $Z\left(t\right)$ en el intervalo $\left(T_{n-1},T_{n}\right]$:
\begin{eqnarray*}
M_{n}=\sup_{T_{n-1}<t\leq T_{n}}|Z\left(t\right)-Z\left(T_{n-1}\right)|
\end{eqnarray*}
\end{Def}

\begin{Teo}
Sup\'ongase que $n^{-1}T_{n}\rightarrow\mu$ c.s. cuando $n\rightarrow\infty$, donde $\mu\leq\infty$ es una constante o variable aleatoria. Sea $a$ una constante o variable aleatoria que puede ser infinita cuando $\mu$ es finita, y considere las expresiones l\'imite:
\begin{eqnarray}
lim_{n\rightarrow\infty}n^{-1}Z\left(T_{n}\right)&=&a,\textrm{ c.s.}\\
lim_{t\rightarrow\infty}t^{-1}Z\left(t\right)&=&a/\mu,\textrm{ c.s.}
\end{eqnarray}
La segunda expresi\'on implica la primera. Conversamente, la primera implica la segunda si el proceso $Z\left(t\right)$ es creciente, o si $lim_{n\rightarrow\infty}n^{-1}M_{n}=0$ c.s.
\end{Teo}

\begin{Coro}
Si $N\left(t\right)$ es un proceso de renovaci\'on, y $\left(Z\left(T_{n}\right)-Z\left(T_{n-1}\right),M_{n}\right)$, para $n\geq1$, son variables aleatorias independientes e id\'enticamente distribuidas con media finita, entonces,
\begin{eqnarray}
lim_{t\rightarrow\infty}t^{-1}Z\left(t\right)\rightarrow\frac{\esp\left[Z\left(T_{1}\right)-Z\left(T_{0}\right)\right]}{\esp\left[T_{1}\right]},\textrm{ c.s. cuando  }t\rightarrow\infty.
\end{eqnarray}
\end{Coro}


%___________________________________________________________________________________________
%
%\subsection{Propiedades de los Procesos de Renovaci\'on}
%___________________________________________________________________________________________
%

Los tiempos $T_{n}$ est\'an relacionados con los conteos de $N\left(t\right)$ por

\begin{eqnarray*}
\left\{N\left(t\right)\geq n\right\}&=&\left\{T_{n}\leq t\right\}\\
T_{N\left(t\right)}\leq &t&<T_{N\left(t\right)+1},
\end{eqnarray*}

adem\'as $N\left(T_{n}\right)=n$, y 

\begin{eqnarray*}
N\left(t\right)=\max\left\{n:T_{n}\leq t\right\}=\min\left\{n:T_{n+1}>t\right\}
\end{eqnarray*}

Por propiedades de la convoluci\'on se sabe que

\begin{eqnarray*}
P\left\{T_{n}\leq t\right\}=F^{n\star}\left(t\right)
\end{eqnarray*}
que es la $n$-\'esima convoluci\'on de $F$. Entonces 

\begin{eqnarray*}
\left\{N\left(t\right)\geq n\right\}&=&\left\{T_{n}\leq t\right\}\\
P\left\{N\left(t\right)\leq n\right\}&=&1-F^{\left(n+1\right)\star}\left(t\right)
\end{eqnarray*}

Adem\'as usando el hecho de que $\esp\left[N\left(t\right)\right]=\sum_{n=1}^{\infty}P\left\{N\left(t\right)\geq n\right\}$
se tiene que

\begin{eqnarray*}
\esp\left[N\left(t\right)\right]=\sum_{n=1}^{\infty}F^{n\star}\left(t\right)
\end{eqnarray*}

\begin{Prop}
Para cada $t\geq0$, la funci\'on generadora de momentos $\esp\left[e^{\alpha N\left(t\right)}\right]$ existe para alguna $\alpha$ en una vecindad del 0, y de aqu\'i que $\esp\left[N\left(t\right)^{m}\right]<\infty$, para $m\geq1$.
\end{Prop}


\begin{Note}
Si el primer tiempo de renovaci\'on $\xi_{1}$ no tiene la misma distribuci\'on que el resto de las $\xi_{n}$, para $n\geq2$, a $N\left(t\right)$ se le llama Proceso de Renovaci\'on retardado, donde si $\xi$ tiene distribuci\'on $G$, entonces el tiempo $T_{n}$ de la $n$-\'esima renovaci\'on tiene distribuci\'on $G\star F^{\left(n-1\right)\star}\left(t\right)$
\end{Note}


\begin{Teo}
Para una constante $\mu\leq\infty$ ( o variable aleatoria), las siguientes expresiones son equivalentes:

\begin{eqnarray}
lim_{n\rightarrow\infty}n^{-1}T_{n}&=&\mu,\textrm{ c.s.}\\
lim_{t\rightarrow\infty}t^{-1}N\left(t\right)&=&1/\mu,\textrm{ c.s.}
\end{eqnarray}
\end{Teo}


Es decir, $T_{n}$ satisface la Ley Fuerte de los Grandes N\'umeros s\'i y s\'olo s\'i $N\left/t\right)$ la cumple.


\begin{Coro}[Ley Fuerte de los Grandes N\'umeros para Procesos de Renovaci\'on]
Si $N\left(t\right)$ es un proceso de renovaci\'on cuyos tiempos de inter-renovaci\'on tienen media $\mu\leq\infty$, entonces
\begin{eqnarray}
t^{-1}N\left(t\right)\rightarrow 1/\mu,\textrm{ c.s. cuando }t\rightarrow\infty.
\end{eqnarray}

\end{Coro}


Considerar el proceso estoc\'astico de valores reales $\left\{Z\left(t\right):t\geq0\right\}$ en el mismo espacio de probabilidad que $N\left(t\right)$

\begin{Def}
Para el proceso $\left\{Z\left(t\right):t\geq0\right\}$ se define la fluctuaci\'on m\'axima de $Z\left(t\right)$ en el intervalo $\left(T_{n-1},T_{n}\right]$:
\begin{eqnarray*}
M_{n}=\sup_{T_{n-1}<t\leq T_{n}}|Z\left(t\right)-Z\left(T_{n-1}\right)|
\end{eqnarray*}
\end{Def}

\begin{Teo}
Sup\'ongase que $n^{-1}T_{n}\rightarrow\mu$ c.s. cuando $n\rightarrow\infty$, donde $\mu\leq\infty$ es una constante o variable aleatoria. Sea $a$ una constante o variable aleatoria que puede ser infinita cuando $\mu$ es finita, y considere las expresiones l\'imite:
\begin{eqnarray}
lim_{n\rightarrow\infty}n^{-1}Z\left(T_{n}\right)&=&a,\textrm{ c.s.}\\
lim_{t\rightarrow\infty}t^{-1}Z\left(t\right)&=&a/\mu,\textrm{ c.s.}
\end{eqnarray}
La segunda expresi\'on implica la primera. Conversamente, la primera implica la segunda si el proceso $Z\left(t\right)$ es creciente, o si $lim_{n\rightarrow\infty}n^{-1}M_{n}=0$ c.s.
\end{Teo}

\begin{Coro}
Si $N\left(t\right)$ es un proceso de renovaci\'on, y $\left(Z\left(T_{n}\right)-Z\left(T_{n-1}\right),M_{n}\right)$, para $n\geq1$, son variables aleatorias independientes e id\'enticamente distribuidas con media finita, entonces,
\begin{eqnarray}
lim_{t\rightarrow\infty}t^{-1}Z\left(t\right)\rightarrow\frac{\esp\left[Z\left(T_{1}\right)-Z\left(T_{0}\right)\right]}{\esp\left[T_{1}\right]},\textrm{ c.s. cuando  }t\rightarrow\infty.
\end{eqnarray}
\end{Coro}

%___________________________________________________________________________________________
%
%\subsection{Propiedades de los Procesos de Renovaci\'on}
%___________________________________________________________________________________________
%

Los tiempos $T_{n}$ est\'an relacionados con los conteos de $N\left(t\right)$ por

\begin{eqnarray*}
\left\{N\left(t\right)\geq n\right\}&=&\left\{T_{n}\leq t\right\}\\
T_{N\left(t\right)}\leq &t&<T_{N\left(t\right)+1},
\end{eqnarray*}

adem\'as $N\left(T_{n}\right)=n$, y 

\begin{eqnarray*}
N\left(t\right)=\max\left\{n:T_{n}\leq t\right\}=\min\left\{n:T_{n+1}>t\right\}
\end{eqnarray*}

Por propiedades de la convoluci\'on se sabe que

\begin{eqnarray*}
P\left\{T_{n}\leq t\right\}=F^{n\star}\left(t\right)
\end{eqnarray*}
que es la $n$-\'esima convoluci\'on de $F$. Entonces 

\begin{eqnarray*}
\left\{N\left(t\right)\geq n\right\}&=&\left\{T_{n}\leq t\right\}\\
P\left\{N\left(t\right)\leq n\right\}&=&1-F^{\left(n+1\right)\star}\left(t\right)
\end{eqnarray*}

Adem\'as usando el hecho de que $\esp\left[N\left(t\right)\right]=\sum_{n=1}^{\infty}P\left\{N\left(t\right)\geq n\right\}$
se tiene que

\begin{eqnarray*}
\esp\left[N\left(t\right)\right]=\sum_{n=1}^{\infty}F^{n\star}\left(t\right)
\end{eqnarray*}

\begin{Prop}
Para cada $t\geq0$, la funci\'on generadora de momentos $\esp\left[e^{\alpha N\left(t\right)}\right]$ existe para alguna $\alpha$ en una vecindad del 0, y de aqu\'i que $\esp\left[N\left(t\right)^{m}\right]<\infty$, para $m\geq1$.
\end{Prop}


\begin{Note}
Si el primer tiempo de renovaci\'on $\xi_{1}$ no tiene la misma distribuci\'on que el resto de las $\xi_{n}$, para $n\geq2$, a $N\left(t\right)$ se le llama Proceso de Renovaci\'on retardado, donde si $\xi$ tiene distribuci\'on $G$, entonces el tiempo $T_{n}$ de la $n$-\'esima renovaci\'on tiene distribuci\'on $G\star F^{\left(n-1\right)\star}\left(t\right)$
\end{Note}


\begin{Teo}
Para una constante $\mu\leq\infty$ ( o variable aleatoria), las siguientes expresiones son equivalentes:

\begin{eqnarray}
lim_{n\rightarrow\infty}n^{-1}T_{n}&=&\mu,\textrm{ c.s.}\\
lim_{t\rightarrow\infty}t^{-1}N\left(t\right)&=&1/\mu,\textrm{ c.s.}
\end{eqnarray}
\end{Teo}


Es decir, $T_{n}$ satisface la Ley Fuerte de los Grandes N\'umeros s\'i y s\'olo s\'i $N\left/t\right)$ la cumple.


\begin{Coro}[Ley Fuerte de los Grandes N\'umeros para Procesos de Renovaci\'on]
Si $N\left(t\right)$ es un proceso de renovaci\'on cuyos tiempos de inter-renovaci\'on tienen media $\mu\leq\infty$, entonces
\begin{eqnarray}
t^{-1}N\left(t\right)\rightarrow 1/\mu,\textrm{ c.s. cuando }t\rightarrow\infty.
\end{eqnarray}

\end{Coro}


Considerar el proceso estoc\'astico de valores reales $\left\{Z\left(t\right):t\geq0\right\}$ en el mismo espacio de probabilidad que $N\left(t\right)$

\begin{Def}
Para el proceso $\left\{Z\left(t\right):t\geq0\right\}$ se define la fluctuaci\'on m\'axima de $Z\left(t\right)$ en el intervalo $\left(T_{n-1},T_{n}\right]$:
\begin{eqnarray*}
M_{n}=\sup_{T_{n-1}<t\leq T_{n}}|Z\left(t\right)-Z\left(T_{n-1}\right)|
\end{eqnarray*}
\end{Def}

\begin{Teo}
Sup\'ongase que $n^{-1}T_{n}\rightarrow\mu$ c.s. cuando $n\rightarrow\infty$, donde $\mu\leq\infty$ es una constante o variable aleatoria. Sea $a$ una constante o variable aleatoria que puede ser infinita cuando $\mu$ es finita, y considere las expresiones l\'imite:
\begin{eqnarray}
lim_{n\rightarrow\infty}n^{-1}Z\left(T_{n}\right)&=&a,\textrm{ c.s.}\\
lim_{t\rightarrow\infty}t^{-1}Z\left(t\right)&=&a/\mu,\textrm{ c.s.}
\end{eqnarray}
La segunda expresi\'on implica la primera. Conversamente, la primera implica la segunda si el proceso $Z\left(t\right)$ es creciente, o si $lim_{n\rightarrow\infty}n^{-1}M_{n}=0$ c.s.
\end{Teo}

\begin{Coro}
Si $N\left(t\right)$ es un proceso de renovaci\'on, y $\left(Z\left(T_{n}\right)-Z\left(T_{n-1}\right),M_{n}\right)$, para $n\geq1$, son variables aleatorias independientes e id\'enticamente distribuidas con media finita, entonces,
\begin{eqnarray}
lim_{t\rightarrow\infty}t^{-1}Z\left(t\right)\rightarrow\frac{\esp\left[Z\left(T_{1}\right)-Z\left(T_{0}\right)\right]}{\esp\left[T_{1}\right]},\textrm{ c.s. cuando  }t\rightarrow\infty.
\end{eqnarray}
\end{Coro}
%___________________________________________________________________________________________
%
%\subsection{Propiedades de los Procesos de Renovaci\'on}
%___________________________________________________________________________________________
%

Los tiempos $T_{n}$ est\'an relacionados con los conteos de $N\left(t\right)$ por

\begin{eqnarray*}
\left\{N\left(t\right)\geq n\right\}&=&\left\{T_{n}\leq t\right\}\\
T_{N\left(t\right)}\leq &t&<T_{N\left(t\right)+1},
\end{eqnarray*}

adem\'as $N\left(T_{n}\right)=n$, y 

\begin{eqnarray*}
N\left(t\right)=\max\left\{n:T_{n}\leq t\right\}=\min\left\{n:T_{n+1}>t\right\}
\end{eqnarray*}

Por propiedades de la convoluci\'on se sabe que

\begin{eqnarray*}
P\left\{T_{n}\leq t\right\}=F^{n\star}\left(t\right)
\end{eqnarray*}
que es la $n$-\'esima convoluci\'on de $F$. Entonces 

\begin{eqnarray*}
\left\{N\left(t\right)\geq n\right\}&=&\left\{T_{n}\leq t\right\}\\
P\left\{N\left(t\right)\leq n\right\}&=&1-F^{\left(n+1\right)\star}\left(t\right)
\end{eqnarray*}

Adem\'as usando el hecho de que $\esp\left[N\left(t\right)\right]=\sum_{n=1}^{\infty}P\left\{N\left(t\right)\geq n\right\}$
se tiene que

\begin{eqnarray*}
\esp\left[N\left(t\right)\right]=\sum_{n=1}^{\infty}F^{n\star}\left(t\right)
\end{eqnarray*}

\begin{Prop}
Para cada $t\geq0$, la funci\'on generadora de momentos $\esp\left[e^{\alpha N\left(t\right)}\right]$ existe para alguna $\alpha$ en una vecindad del 0, y de aqu\'i que $\esp\left[N\left(t\right)^{m}\right]<\infty$, para $m\geq1$.
\end{Prop}


\begin{Note}
Si el primer tiempo de renovaci\'on $\xi_{1}$ no tiene la misma distribuci\'on que el resto de las $\xi_{n}$, para $n\geq2$, a $N\left(t\right)$ se le llama Proceso de Renovaci\'on retardado, donde si $\xi$ tiene distribuci\'on $G$, entonces el tiempo $T_{n}$ de la $n$-\'esima renovaci\'on tiene distribuci\'on $G\star F^{\left(n-1\right)\star}\left(t\right)$
\end{Note}


\begin{Teo}
Para una constante $\mu\leq\infty$ ( o variable aleatoria), las siguientes expresiones son equivalentes:

\begin{eqnarray}
lim_{n\rightarrow\infty}n^{-1}T_{n}&=&\mu,\textrm{ c.s.}\\
lim_{t\rightarrow\infty}t^{-1}N\left(t\right)&=&1/\mu,\textrm{ c.s.}
\end{eqnarray}
\end{Teo}


Es decir, $T_{n}$ satisface la Ley Fuerte de los Grandes N\'umeros s\'i y s\'olo s\'i $N\left/t\right)$ la cumple.


\begin{Coro}[Ley Fuerte de los Grandes N\'umeros para Procesos de Renovaci\'on]
Si $N\left(t\right)$ es un proceso de renovaci\'on cuyos tiempos de inter-renovaci\'on tienen media $\mu\leq\infty$, entonces
\begin{eqnarray}
t^{-1}N\left(t\right)\rightarrow 1/\mu,\textrm{ c.s. cuando }t\rightarrow\infty.
\end{eqnarray}

\end{Coro}


Considerar el proceso estoc\'astico de valores reales $\left\{Z\left(t\right):t\geq0\right\}$ en el mismo espacio de probabilidad que $N\left(t\right)$

\begin{Def}
Para el proceso $\left\{Z\left(t\right):t\geq0\right\}$ se define la fluctuaci\'on m\'axima de $Z\left(t\right)$ en el intervalo $\left(T_{n-1},T_{n}\right]$:
\begin{eqnarray*}
M_{n}=\sup_{T_{n-1}<t\leq T_{n}}|Z\left(t\right)-Z\left(T_{n-1}\right)|
\end{eqnarray*}
\end{Def}

\begin{Teo}
Sup\'ongase que $n^{-1}T_{n}\rightarrow\mu$ c.s. cuando $n\rightarrow\infty$, donde $\mu\leq\infty$ es una constante o variable aleatoria. Sea $a$ una constante o variable aleatoria que puede ser infinita cuando $\mu$ es finita, y considere las expresiones l\'imite:
\begin{eqnarray}
lim_{n\rightarrow\infty}n^{-1}Z\left(T_{n}\right)&=&a,\textrm{ c.s.}\\
lim_{t\rightarrow\infty}t^{-1}Z\left(t\right)&=&a/\mu,\textrm{ c.s.}
\end{eqnarray}
La segunda expresi\'on implica la primera. Conversamente, la primera implica la segunda si el proceso $Z\left(t\right)$ es creciente, o si $lim_{n\rightarrow\infty}n^{-1}M_{n}=0$ c.s.
\end{Teo}

\begin{Coro}
Si $N\left(t\right)$ es un proceso de renovaci\'on, y $\left(Z\left(T_{n}\right)-Z\left(T_{n-1}\right),M_{n}\right)$, para $n\geq1$, son variables aleatorias independientes e id\'enticamente distribuidas con media finita, entonces,
\begin{eqnarray}
lim_{t\rightarrow\infty}t^{-1}Z\left(t\right)\rightarrow\frac{\esp\left[Z\left(T_{1}\right)-Z\left(T_{0}\right)\right]}{\esp\left[T_{1}\right]},\textrm{ c.s. cuando  }t\rightarrow\infty.
\end{eqnarray}
\end{Coro}


%___________________________________________________________________________________________
%
\subsection{Funci\'on de Renovaci\'on}
%___________________________________________________________________________________________
%


\begin{Def}
Sea $h\left(t\right)$ funci\'on de valores reales en $\rea$ acotada en intervalos finitos e igual a cero para $t<0$ La ecuaci\'on de renovaci\'on para $h\left(t\right)$ y la distribuci\'on $F$ es

\begin{eqnarray}%\label{Ec.Renovacion}
H\left(t\right)=h\left(t\right)+\int_{\left[0,t\right]}H\left(t-s\right)dF\left(s\right)\textrm{,    }t\geq0,
\end{eqnarray}
donde $H\left(t\right)$ es una funci\'on de valores reales. Esto es $H=h+F\star H$. Decimos que $H\left(t\right)$ es soluci\'on de esta ecuaci\'on si satisface la ecuaci\'on, y es acotada en intervalos finitos e iguales a cero para $t<0$.
\end{Def}

\begin{Prop}
La funci\'on $U\star h\left(t\right)$ es la \'unica soluci\'on de la ecuaci\'on de renovaci\'on (\ref{Ec.Renovacion}).
\end{Prop}

\begin{Teo}[Teorema Renovaci\'on Elemental]
\begin{eqnarray*}
t^{-1}U\left(t\right)\rightarrow 1/\mu\textrm{,    cuando }t\rightarrow\infty.
\end{eqnarray*}
\end{Teo}

%___________________________________________________________________________________________
%
%\subsection{Funci\'on de Renovaci\'on}
%___________________________________________________________________________________________
%


Sup\'ongase que $N\left(t\right)$ es un proceso de renovaci\'on con distribuci\'on $F$ con media finita $\mu$.

\begin{Def}
La funci\'on de renovaci\'on asociada con la distribuci\'on $F$, del proceso $N\left(t\right)$, es
\begin{eqnarray*}
U\left(t\right)=\sum_{n=1}^{\infty}F^{n\star}\left(t\right),\textrm{   }t\geq0,
\end{eqnarray*}
donde $F^{0\star}\left(t\right)=\indora\left(t\geq0\right)$.
\end{Def}


\begin{Prop}
Sup\'ongase que la distribuci\'on de inter-renovaci\'on $F$ tiene densidad $f$. Entonces $U\left(t\right)$ tambi\'en tiene densidad, para $t>0$, y es $U^{'}\left(t\right)=\sum_{n=0}^{\infty}f^{n\star}\left(t\right)$. Adem\'as
\begin{eqnarray*}
\prob\left\{N\left(t\right)>N\left(t-\right)\right\}=0\textrm{,   }t\geq0.
\end{eqnarray*}
\end{Prop}

\begin{Def}
La Transformada de Laplace-Stieljes de $F$ est\'a dada por

\begin{eqnarray*}
\hat{F}\left(\alpha\right)=\int_{\rea_{+}}e^{-\alpha t}dF\left(t\right)\textrm{,  }\alpha\geq0.
\end{eqnarray*}
\end{Def}

Entonces

\begin{eqnarray*}
\hat{U}\left(\alpha\right)=\sum_{n=0}^{\infty}\hat{F^{n\star}}\left(\alpha\right)=\sum_{n=0}^{\infty}\hat{F}\left(\alpha\right)^{n}=\frac{1}{1-\hat{F}\left(\alpha\right)}.
\end{eqnarray*}


\begin{Prop}
La Transformada de Laplace $\hat{U}\left(\alpha\right)$ y $\hat{F}\left(\alpha\right)$ determina una a la otra de manera \'unica por la relaci\'on $\hat{U}\left(\alpha\right)=\frac{1}{1-\hat{F}\left(\alpha\right)}$.
\end{Prop}


\begin{Note}
Un proceso de renovaci\'on $N\left(t\right)$ cuyos tiempos de inter-renovaci\'on tienen media finita, es un proceso Poisson con tasa $\lambda$ si y s\'olo s\'i $\esp\left[U\left(t\right)\right]=\lambda t$, para $t\geq0$.
\end{Note}


\begin{Teo}
Sea $N\left(t\right)$ un proceso puntual simple con puntos de localizaci\'on $T_{n}$ tal que $\eta\left(t\right)=\esp\left[N\left(\right)\right]$ es finita para cada $t$. Entonces para cualquier funci\'on $f:\rea_{+}\rightarrow\rea$,
\begin{eqnarray*}
\esp\left[\sum_{n=1}^{N\left(\right)}f\left(T_{n}\right)\right]=\int_{\left(0,t\right]}f\left(s\right)d\eta\left(s\right)\textrm{,  }t\geq0,
\end{eqnarray*}
suponiendo que la integral exista. Adem\'as si $X_{1},X_{2},\ldots$ son variables aleatorias definidas en el mismo espacio de probabilidad que el proceso $N\left(t\right)$ tal que $\esp\left[X_{n}|T_{n}=s\right]=f\left(s\right)$, independiente de $n$. Entonces
\begin{eqnarray*}
\esp\left[\sum_{n=1}^{N\left(t\right)}X_{n}\right]=\int_{\left(0,t\right]}f\left(s\right)d\eta\left(s\right)\textrm{,  }t\geq0,
\end{eqnarray*} 
suponiendo que la integral exista. 
\end{Teo}

\begin{Coro}[Identidad de Wald para Renovaciones]
Para el proceso de renovaci\'on $N\left(t\right)$,
\begin{eqnarray*}
\esp\left[T_{N\left(t\right)+1}\right]=\mu\esp\left[N\left(t\right)+1\right]\textrm{,  }t\geq0,
\end{eqnarray*}  
\end{Coro}

%______________________________________________________________________
%\subsection{Procesos de Renovaci\'on}
%______________________________________________________________________

\begin{Def}%\label{Def.Tn}
Sean $0\leq T_{1}\leq T_{2}\leq \ldots$ son tiempos aleatorios infinitos en los cuales ocurren ciertos eventos. El n\'umero de tiempos $T_{n}$ en el intervalo $\left[0,t\right)$ es

\begin{eqnarray}
N\left(t\right)=\sum_{n=1}^{\infty}\indora\left(T_{n}\leq t\right),
\end{eqnarray}
para $t\geq0$.
\end{Def}

Si se consideran los puntos $T_{n}$ como elementos de $\rea_{+}$, y $N\left(t\right)$ es el n\'umero de puntos en $\rea$. El proceso denotado por $\left\{N\left(t\right):t\geq0\right\}$, denotado por $N\left(t\right)$, es un proceso puntual en $\rea_{+}$. Los $T_{n}$ son los tiempos de ocurrencia, el proceso puntual $N\left(t\right)$ es simple si su n\'umero de ocurrencias son distintas: $0<T_{1}<T_{2}<\ldots$ casi seguramente.

\begin{Def}
Un proceso puntual $N\left(t\right)$ es un proceso de renovaci\'on si los tiempos de interocurrencia $\xi_{n}=T_{n}-T_{n-1}$, para $n\geq1$, son independientes e identicamente distribuidos con distribuci\'on $F$, donde $F\left(0\right)=0$ y $T_{0}=0$. Los $T_{n}$ son llamados tiempos de renovaci\'on, referente a la independencia o renovaci\'on de la informaci\'on estoc\'astica en estos tiempos. Los $\xi_{n}$ son los tiempos de inter-renovaci\'on, y $N\left(t\right)$ es el n\'umero de renovaciones en el intervalo $\left[0,t\right)$
\end{Def}


\begin{Note}
Para definir un proceso de renovaci\'on para cualquier contexto, solamente hay que especificar una distribuci\'on $F$, con $F\left(0\right)=0$, para los tiempos de inter-renovaci\'on. La funci\'on $F$ en turno degune las otra variables aleatorias. De manera formal, existe un espacio de probabilidad y una sucesi\'on de variables aleatorias $\xi_{1},\xi_{2},\ldots$ definidas en este con distribuci\'on $F$. Entonces las otras cantidades son $T_{n}=\sum_{k=1}^{n}\xi_{k}$ y $N\left(t\right)=\sum_{n=1}^{\infty}\indora\left(T_{n}\leq t\right)$, donde $T_{n}\rightarrow\infty$ casi seguramente por la Ley Fuerte de los Grandes Números.
\end{Note}

%___________________________________________________________________________________________
%
\subsection{Renewal and Regenerative Processes: Serfozo\cite{Serfozo}}
%___________________________________________________________________________________________
%
\begin{Def}%\label{Def.Tn}
Sean $0\leq T_{1}\leq T_{2}\leq \ldots$ son tiempos aleatorios infinitos en los cuales ocurren ciertos eventos. El n\'umero de tiempos $T_{n}$ en el intervalo $\left[0,t\right)$ es

\begin{eqnarray}
N\left(t\right)=\sum_{n=1}^{\infty}\indora\left(T_{n}\leq t\right),
\end{eqnarray}
para $t\geq0$.
\end{Def}

Si se consideran los puntos $T_{n}$ como elementos de $\rea_{+}$, y $N\left(t\right)$ es el n\'umero de puntos en $\rea$. El proceso denotado por $\left\{N\left(t\right):t\geq0\right\}$, denotado por $N\left(t\right)$, es un proceso puntual en $\rea_{+}$. Los $T_{n}$ son los tiempos de ocurrencia, el proceso puntual $N\left(t\right)$ es simple si su n\'umero de ocurrencias son distintas: $0<T_{1}<T_{2}<\ldots$ casi seguramente.

\begin{Def}
Un proceso puntual $N\left(t\right)$ es un proceso de renovaci\'on si los tiempos de interocurrencia $\xi_{n}=T_{n}-T_{n-1}$, para $n\geq1$, son independientes e identicamente distribuidos con distribuci\'on $F$, donde $F\left(0\right)=0$ y $T_{0}=0$. Los $T_{n}$ son llamados tiempos de renovaci\'on, referente a la independencia o renovaci\'on de la informaci\'on estoc\'astica en estos tiempos. Los $\xi_{n}$ son los tiempos de inter-renovaci\'on, y $N\left(t\right)$ es el n\'umero de renovaciones en el intervalo $\left[0,t\right)$
\end{Def}


\begin{Note}
Para definir un proceso de renovaci\'on para cualquier contexto, solamente hay que especificar una distribuci\'on $F$, con $F\left(0\right)=0$, para los tiempos de inter-renovaci\'on. La funci\'on $F$ en turno degune las otra variables aleatorias. De manera formal, existe un espacio de probabilidad y una sucesi\'on de variables aleatorias $\xi_{1},\xi_{2},\ldots$ definidas en este con distribuci\'on $F$. Entonces las otras cantidades son $T_{n}=\sum_{k=1}^{n}\xi_{k}$ y $N\left(t\right)=\sum_{n=1}^{\infty}\indora\left(T_{n}\leq t\right)$, donde $T_{n}\rightarrow\infty$ casi seguramente por la Ley Fuerte de los Grandes N\'umeros.
\end{Note}

Los tiempos $T_{n}$ est\'an relacionados con los conteos de $N\left(t\right)$ por

\begin{eqnarray*}
\left\{N\left(t\right)\geq n\right\}&=&\left\{T_{n}\leq t\right\}\\
T_{N\left(t\right)}\leq &t&<T_{N\left(t\right)+1},
\end{eqnarray*}

adem\'as $N\left(T_{n}\right)=n$, y 

\begin{eqnarray*}
N\left(t\right)=\max\left\{n:T_{n}\leq t\right\}=\min\left\{n:T_{n+1}>t\right\}
\end{eqnarray*}

Por propiedades de la convoluci\'on se sabe que

\begin{eqnarray*}
P\left\{T_{n}\leq t\right\}=F^{n\star}\left(t\right)
\end{eqnarray*}
que es la $n$-\'esima convoluci\'on de $F$. Entonces 

\begin{eqnarray*}
\left\{N\left(t\right)\geq n\right\}&=&\left\{T_{n}\leq t\right\}\\
P\left\{N\left(t\right)\leq n\right\}&=&1-F^{\left(n+1\right)\star}\left(t\right)
\end{eqnarray*}

Adem\'as usando el hecho de que $\esp\left[N\left(t\right)\right]=\sum_{n=1}^{\infty}P\left\{N\left(t\right)\geq n\right\}$
se tiene que

\begin{eqnarray*}
\esp\left[N\left(t\right)\right]=\sum_{n=1}^{\infty}F^{n\star}\left(t\right)
\end{eqnarray*}

\begin{Prop}
Para cada $t\geq0$, la funci\'on generadora de momentos $\esp\left[e^{\alpha N\left(t\right)}\right]$ existe para alguna $\alpha$ en una vecindad del 0, y de aqu\'i que $\esp\left[N\left(t\right)^{m}\right]<\infty$, para $m\geq1$.
\end{Prop}

\begin{Ejem}[\textbf{Proceso Poisson}]

Suponga que se tienen tiempos de inter-renovaci\'on \textit{i.i.d.} del proceso de renovaci\'on $N\left(t\right)$ tienen distribuci\'on exponencial $F\left(t\right)=q-e^{-\lambda t}$ con tasa $\lambda$. Entonces $N\left(t\right)$ es un proceso Poisson con tasa $\lambda$.

\end{Ejem}


\begin{Note}
Si el primer tiempo de renovaci\'on $\xi_{1}$ no tiene la misma distribuci\'on que el resto de las $\xi_{n}$, para $n\geq2$, a $N\left(t\right)$ se le llama Proceso de Renovaci\'on retardado, donde si $\xi$ tiene distribuci\'on $G$, entonces el tiempo $T_{n}$ de la $n$-\'esima renovaci\'on tiene distribuci\'on $G\star F^{\left(n-1\right)\star}\left(t\right)$
\end{Note}


\begin{Teo}
Para una constante $\mu\leq\infty$ ( o variable aleatoria), las siguientes expresiones son equivalentes:

\begin{eqnarray}
lim_{n\rightarrow\infty}n^{-1}T_{n}&=&\mu,\textrm{ c.s.}\\
lim_{t\rightarrow\infty}t^{-1}N\left(t\right)&=&1/\mu,\textrm{ c.s.}
\end{eqnarray}
\end{Teo}


Es decir, $T_{n}$ satisface la Ley Fuerte de los Grandes N\'umeros s\'i y s\'olo s\'i $N\left/t\right)$ la cumple.


\begin{Coro}[Ley Fuerte de los Grandes N\'umeros para Procesos de Renovaci\'on]
Si $N\left(t\right)$ es un proceso de renovaci\'on cuyos tiempos de inter-renovaci\'on tienen media $\mu\leq\infty$, entonces
\begin{eqnarray}
t^{-1}N\left(t\right)\rightarrow 1/\mu,\textrm{ c.s. cuando }t\rightarrow\infty.
\end{eqnarray}

\end{Coro}


Considerar el proceso estoc\'astico de valores reales $\left\{Z\left(t\right):t\geq0\right\}$ en el mismo espacio de probabilidad que $N\left(t\right)$

\begin{Def}
Para el proceso $\left\{Z\left(t\right):t\geq0\right\}$ se define la fluctuaci\'on m\'axima de $Z\left(t\right)$ en el intervalo $\left(T_{n-1},T_{n}\right]$:
\begin{eqnarray*}
M_{n}=\sup_{T_{n-1}<t\leq T_{n}}|Z\left(t\right)-Z\left(T_{n-1}\right)|
\end{eqnarray*}
\end{Def}

\begin{Teo}
Sup\'ongase que $n^{-1}T_{n}\rightarrow\mu$ c.s. cuando $n\rightarrow\infty$, donde $\mu\leq\infty$ es una constante o variable aleatoria. Sea $a$ una constante o variable aleatoria que puede ser infinita cuando $\mu$ es finita, y considere las expresiones l\'imite:
\begin{eqnarray}
lim_{n\rightarrow\infty}n^{-1}Z\left(T_{n}\right)&=&a,\textrm{ c.s.}\\
lim_{t\rightarrow\infty}t^{-1}Z\left(t\right)&=&a/\mu,\textrm{ c.s.}
\end{eqnarray}
La segunda expresi\'on implica la primera. Conversamente, la primera implica la segunda si el proceso $Z\left(t\right)$ es creciente, o si $lim_{n\rightarrow\infty}n^{-1}M_{n}=0$ c.s.
\end{Teo}

\begin{Coro}
Si $N\left(t\right)$ es un proceso de renovaci\'on, y $\left(Z\left(T_{n}\right)-Z\left(T_{n-1}\right),M_{n}\right)$, para $n\geq1$, son variables aleatorias independientes e id\'enticamente distribuidas con media finita, entonces,
\begin{eqnarray}
lim_{t\rightarrow\infty}t^{-1}Z\left(t\right)\rightarrow\frac{\esp\left[Z\left(T_{1}\right)-Z\left(T_{0}\right)\right]}{\esp\left[T_{1}\right]},\textrm{ c.s. cuando  }t\rightarrow\infty.
\end{eqnarray}
\end{Coro}


Sup\'ongase que $N\left(t\right)$ es un proceso de renovaci\'on con distribuci\'on $F$ con media finita $\mu$.

\begin{Def}
La funci\'on de renovaci\'on asociada con la distribuci\'on $F$, del proceso $N\left(t\right)$, es
\begin{eqnarray*}
U\left(t\right)=\sum_{n=1}^{\infty}F^{n\star}\left(t\right),\textrm{   }t\geq0,
\end{eqnarray*}
donde $F^{0\star}\left(t\right)=\indora\left(t\geq0\right)$.
\end{Def}


\begin{Prop}
Sup\'ongase que la distribuci\'on de inter-renovaci\'on $F$ tiene densidad $f$. Entonces $U\left(t\right)$ tambi\'en tiene densidad, para $t>0$, y es $U^{'}\left(t\right)=\sum_{n=0}^{\infty}f^{n\star}\left(t\right)$. Adem\'as
\begin{eqnarray*}
\prob\left\{N\left(t\right)>N\left(t-\right)\right\}=0\textrm{,   }t\geq0.
\end{eqnarray*}
\end{Prop}

\begin{Def}
La Transformada de Laplace-Stieljes de $F$ est\'a dada por

\begin{eqnarray*}
\hat{F}\left(\alpha\right)=\int_{\rea_{+}}e^{-\alpha t}dF\left(t\right)\textrm{,  }\alpha\geq0.
\end{eqnarray*}
\end{Def}

Entonces

\begin{eqnarray*}
\hat{U}\left(\alpha\right)=\sum_{n=0}^{\infty}\hat{F^{n\star}}\left(\alpha\right)=\sum_{n=0}^{\infty}\hat{F}\left(\alpha\right)^{n}=\frac{1}{1-\hat{F}\left(\alpha\right)}.
\end{eqnarray*}


\begin{Prop}
La Transformada de Laplace $\hat{U}\left(\alpha\right)$ y $\hat{F}\left(\alpha\right)$ determina una a la otra de manera \'unica por la relaci\'on $\hat{U}\left(\alpha\right)=\frac{1}{1-\hat{F}\left(\alpha\right)}$.
\end{Prop}


\begin{Note}
Un proceso de renovaci\'on $N\left(t\right)$ cuyos tiempos de inter-renovaci\'on tienen media finita, es un proceso Poisson con tasa $\lambda$ si y s\'olo s\'i $\esp\left[U\left(t\right)\right]=\lambda t$, para $t\geq0$.
\end{Note}


\begin{Teo}
Sea $N\left(t\right)$ un proceso puntual simple con puntos de localizaci\'on $T_{n}$ tal que $\eta\left(t\right)=\esp\left[N\left(\right)\right]$ es finita para cada $t$. Entonces para cualquier funci\'on $f:\rea_{+}\rightarrow\rea$,
\begin{eqnarray*}
\esp\left[\sum_{n=1}^{N\left(\right)}f\left(T_{n}\right)\right]=\int_{\left(0,t\right]}f\left(s\right)d\eta\left(s\right)\textrm{,  }t\geq0,
\end{eqnarray*}
suponiendo que la integral exista. Adem\'as si $X_{1},X_{2},\ldots$ son variables aleatorias definidas en el mismo espacio de probabilidad que el proceso $N\left(t\right)$ tal que $\esp\left[X_{n}|T_{n}=s\right]=f\left(s\right)$, independiente de $n$. Entonces
\begin{eqnarray*}
\esp\left[\sum_{n=1}^{N\left(t\right)}X_{n}\right]=\int_{\left(0,t\right]}f\left(s\right)d\eta\left(s\right)\textrm{,  }t\geq0,
\end{eqnarray*} 
suponiendo que la integral exista. 
\end{Teo}

\begin{Coro}[Identidad de Wald para Renovaciones]
Para el proceso de renovaci\'on $N\left(t\right)$,
\begin{eqnarray*}
\esp\left[T_{N\left(t\right)+1}\right]=\mu\esp\left[N\left(t\right)+1\right]\textrm{,  }t\geq0,
\end{eqnarray*}  
\end{Coro}


\begin{Def}
Sea $h\left(t\right)$ funci\'on de valores reales en $\rea$ acotada en intervalos finitos e igual a cero para $t<0$ La ecuaci\'on de renovaci\'on para $h\left(t\right)$ y la distribuci\'on $F$ es

\begin{eqnarray}%\label{Ec.Renovacion}
H\left(t\right)=h\left(t\right)+\int_{\left[0,t\right]}H\left(t-s\right)dF\left(s\right)\textrm{,    }t\geq0,
\end{eqnarray}
donde $H\left(t\right)$ es una funci\'on de valores reales. Esto es $H=h+F\star H$. Decimos que $H\left(t\right)$ es soluci\'on de esta ecuaci\'on si satisface la ecuaci\'on, y es acotada en intervalos finitos e iguales a cero para $t<0$.
\end{Def}

\begin{Prop}
La funci\'on $U\star h\left(t\right)$ es la \'unica soluci\'on de la ecuaci\'on de renovaci\'on (\ref{Ec.Renovacion}).
\end{Prop}

\begin{Teo}[Teorema Renovaci\'on Elemental]
\begin{eqnarray*}
t^{-1}U\left(t\right)\rightarrow 1/\mu\textrm{,    cuando }t\rightarrow\infty.
\end{eqnarray*}
\end{Teo}



Sup\'ongase que $N\left(t\right)$ es un proceso de renovaci\'on con distribuci\'on $F$ con media finita $\mu$.

\begin{Def}
La funci\'on de renovaci\'on asociada con la distribuci\'on $F$, del proceso $N\left(t\right)$, es
\begin{eqnarray*}
U\left(t\right)=\sum_{n=1}^{\infty}F^{n\star}\left(t\right),\textrm{   }t\geq0,
\end{eqnarray*}
donde $F^{0\star}\left(t\right)=\indora\left(t\geq0\right)$.
\end{Def}


\begin{Prop}
Sup\'ongase que la distribuci\'on de inter-renovaci\'on $F$ tiene densidad $f$. Entonces $U\left(t\right)$ tambi\'en tiene densidad, para $t>0$, y es $U^{'}\left(t\right)=\sum_{n=0}^{\infty}f^{n\star}\left(t\right)$. Adem\'as
\begin{eqnarray*}
\prob\left\{N\left(t\right)>N\left(t-\right)\right\}=0\textrm{,   }t\geq0.
\end{eqnarray*}
\end{Prop}

\begin{Def}
La Transformada de Laplace-Stieljes de $F$ est\'a dada por

\begin{eqnarray*}
\hat{F}\left(\alpha\right)=\int_{\rea_{+}}e^{-\alpha t}dF\left(t\right)\textrm{,  }\alpha\geq0.
\end{eqnarray*}
\end{Def}

Entonces

\begin{eqnarray*}
\hat{U}\left(\alpha\right)=\sum_{n=0}^{\infty}\hat{F^{n\star}}\left(\alpha\right)=\sum_{n=0}^{\infty}\hat{F}\left(\alpha\right)^{n}=\frac{1}{1-\hat{F}\left(\alpha\right)}.
\end{eqnarray*}


\begin{Prop}
La Transformada de Laplace $\hat{U}\left(\alpha\right)$ y $\hat{F}\left(\alpha\right)$ determina una a la otra de manera \'unica por la relaci\'on $\hat{U}\left(\alpha\right)=\frac{1}{1-\hat{F}\left(\alpha\right)}$.
\end{Prop}


\begin{Note}
Un proceso de renovaci\'on $N\left(t\right)$ cuyos tiempos de inter-renovaci\'on tienen media finita, es un proceso Poisson con tasa $\lambda$ si y s\'olo s\'i $\esp\left[U\left(t\right)\right]=\lambda t$, para $t\geq0$.
\end{Note}


\begin{Teo}
Sea $N\left(t\right)$ un proceso puntual simple con puntos de localizaci\'on $T_{n}$ tal que $\eta\left(t\right)=\esp\left[N\left(\right)\right]$ es finita para cada $t$. Entonces para cualquier funci\'on $f:\rea_{+}\rightarrow\rea$,
\begin{eqnarray*}
\esp\left[\sum_{n=1}^{N\left(\right)}f\left(T_{n}\right)\right]=\int_{\left(0,t\right]}f\left(s\right)d\eta\left(s\right)\textrm{,  }t\geq0,
\end{eqnarray*}
suponiendo que la integral exista. Adem\'as si $X_{1},X_{2},\ldots$ son variables aleatorias definidas en el mismo espacio de probabilidad que el proceso $N\left(t\right)$ tal que $\esp\left[X_{n}|T_{n}=s\right]=f\left(s\right)$, independiente de $n$. Entonces
\begin{eqnarray*}
\esp\left[\sum_{n=1}^{N\left(t\right)}X_{n}\right]=\int_{\left(0,t\right]}f\left(s\right)d\eta\left(s\right)\textrm{,  }t\geq0,
\end{eqnarray*} 
suponiendo que la integral exista. 
\end{Teo}

\begin{Coro}[Identidad de Wald para Renovaciones]
Para el proceso de renovaci\'on $N\left(t\right)$,
\begin{eqnarray*}
\esp\left[T_{N\left(t\right)+1}\right]=\mu\esp\left[N\left(t\right)+1\right]\textrm{,  }t\geq0,
\end{eqnarray*}  
\end{Coro}


\begin{Def}
Sea $h\left(t\right)$ funci\'on de valores reales en $\rea$ acotada en intervalos finitos e igual a cero para $t<0$ La ecuaci\'on de renovaci\'on para $h\left(t\right)$ y la distribuci\'on $F$ es

\begin{eqnarray}%\label{Ec.Renovacion}
H\left(t\right)=h\left(t\right)+\int_{\left[0,t\right]}H\left(t-s\right)dF\left(s\right)\textrm{,    }t\geq0,
\end{eqnarray}
donde $H\left(t\right)$ es una funci\'on de valores reales. Esto es $H=h+F\star H$. Decimos que $H\left(t\right)$ es soluci\'on de esta ecuaci\'on si satisface la ecuaci\'on, y es acotada en intervalos finitos e iguales a cero para $t<0$.
\end{Def}

\begin{Prop}
La funci\'on $U\star h\left(t\right)$ es la \'unica soluci\'on de la ecuaci\'on de renovaci\'on (\ref{Ec.Renovacion}).
\end{Prop}

\begin{Teo}[Teorema Renovaci\'on Elemental]
\begin{eqnarray*}
t^{-1}U\left(t\right)\rightarrow 1/\mu\textrm{,    cuando }t\rightarrow\infty.
\end{eqnarray*}
\end{Teo}


\begin{Note} Una funci\'on $h:\rea_{+}\rightarrow\rea$ es Directamente Riemann Integrable en los siguientes casos:
\begin{itemize}
\item[a)] $h\left(t\right)\geq0$ es decreciente y Riemann Integrable.
\item[b)] $h$ es continua excepto posiblemente en un conjunto de Lebesgue de medida 0, y $|h\left(t\right)|\leq b\left(t\right)$, donde $b$ es DRI.
\end{itemize}
\end{Note}

\begin{Teo}[Teorema Principal de Renovaci\'on]
Si $F$ es no aritm\'etica y $h\left(t\right)$ es Directamente Riemann Integrable (DRI), entonces

\begin{eqnarray*}
lim_{t\rightarrow\infty}U\star h=\frac{1}{\mu}\int_{\rea_{+}}h\left(s\right)ds.
\end{eqnarray*}
\end{Teo}

\begin{Prop}
Cualquier funci\'on $H\left(t\right)$ acotada en intervalos finitos y que es 0 para $t<0$ puede expresarse como
\begin{eqnarray*}
H\left(t\right)=U\star h\left(t\right)\textrm{,  donde }h\left(t\right)=H\left(t\right)-F\star H\left(t\right)
\end{eqnarray*}
\end{Prop}

\begin{Def}
Un proceso estoc\'astico $X\left(t\right)$ es crudamente regenerativo en un tiempo aleatorio positivo $T$ si
\begin{eqnarray*}
\esp\left[X\left(T+t\right)|T\right]=\esp\left[X\left(t\right)\right]\textrm{, para }t\geq0,\end{eqnarray*}
y con las esperanzas anteriores finitas.
\end{Def}

\begin{Prop}
Sup\'ongase que $X\left(t\right)$ es un proceso crudamente regenerativo en $T$, que tiene distribuci\'on $F$. Si $\esp\left[X\left(t\right)\right]$ es acotado en intervalos finitos, entonces
\begin{eqnarray*}
\esp\left[X\left(t\right)\right]=U\star h\left(t\right)\textrm{,  donde }h\left(t\right)=\esp\left[X\left(t\right)\indora\left(T>t\right)\right].
\end{eqnarray*}
\end{Prop}

\begin{Teo}[Regeneraci\'on Cruda]
Sup\'ongase que $X\left(t\right)$ es un proceso con valores positivo crudamente regenerativo en $T$, y def\'inase $M=\sup\left\{|X\left(t\right)|:t\leq T\right\}$. Si $T$ es no aritm\'etico y $M$ y $MT$ tienen media finita, entonces
\begin{eqnarray*}
lim_{t\rightarrow\infty}\esp\left[X\left(t\right)\right]=\frac{1}{\mu}\int_{\rea_{+}}h\left(s\right)ds,
\end{eqnarray*}
donde $h\left(t\right)=\esp\left[X\left(t\right)\indora\left(T>t\right)\right]$.
\end{Teo}


\begin{Note} Una funci\'on $h:\rea_{+}\rightarrow\rea$ es Directamente Riemann Integrable en los siguientes casos:
\begin{itemize}
\item[a)] $h\left(t\right)\geq0$ es decreciente y Riemann Integrable.
\item[b)] $h$ es continua excepto posiblemente en un conjunto de Lebesgue de medida 0, y $|h\left(t\right)|\leq b\left(t\right)$, donde $b$ es DRI.
\end{itemize}
\end{Note}

\begin{Teo}[Teorema Principal de Renovaci\'on]
Si $F$ es no aritm\'etica y $h\left(t\right)$ es Directamente Riemann Integrable (DRI), entonces

\begin{eqnarray*}
lim_{t\rightarrow\infty}U\star h=\frac{1}{\mu}\int_{\rea_{+}}h\left(s\right)ds.
\end{eqnarray*}
\end{Teo}

\begin{Prop}
Cualquier funci\'on $H\left(t\right)$ acotada en intervalos finitos y que es 0 para $t<0$ puede expresarse como
\begin{eqnarray*}
H\left(t\right)=U\star h\left(t\right)\textrm{,  donde }h\left(t\right)=H\left(t\right)-F\star H\left(t\right)
\end{eqnarray*}
\end{Prop}

\begin{Def}
Un proceso estoc\'astico $X\left(t\right)$ es crudamente regenerativo en un tiempo aleatorio positivo $T$ si
\begin{eqnarray*}
\esp\left[X\left(T+t\right)|T\right]=\esp\left[X\left(t\right)\right]\textrm{, para }t\geq0,\end{eqnarray*}
y con las esperanzas anteriores finitas.
\end{Def}

\begin{Prop}
Sup\'ongase que $X\left(t\right)$ es un proceso crudamente regenerativo en $T$, que tiene distribuci\'on $F$. Si $\esp\left[X\left(t\right)\right]$ es acotado en intervalos finitos, entonces
\begin{eqnarray*}
\esp\left[X\left(t\right)\right]=U\star h\left(t\right)\textrm{,  donde }h\left(t\right)=\esp\left[X\left(t\right)\indora\left(T>t\right)\right].
\end{eqnarray*}
\end{Prop}

\begin{Teo}[Regeneraci\'on Cruda]
Sup\'ongase que $X\left(t\right)$ es un proceso con valores positivo crudamente regenerativo en $T$, y def\'inase $M=\sup\left\{|X\left(t\right)|:t\leq T\right\}$. Si $T$ es no aritm\'etico y $M$ y $MT$ tienen media finita, entonces
\begin{eqnarray*}
lim_{t\rightarrow\infty}\esp\left[X\left(t\right)\right]=\frac{1}{\mu}\int_{\rea_{+}}h\left(s\right)ds,
\end{eqnarray*}
donde $h\left(t\right)=\esp\left[X\left(t\right)\indora\left(T>t\right)\right]$.
\end{Teo}

\begin{Def}
Para el proceso $\left\{\left(N\left(t\right),X\left(t\right)\right):t\geq0\right\}$, sus trayectoria muestrales en el intervalo de tiempo $\left[T_{n-1},T_{n}\right)$ est\'an descritas por
\begin{eqnarray*}
\zeta_{n}=\left(\xi_{n},\left\{X\left(T_{n-1}+t\right):0\leq t<\xi_{n}\right\}\right)
\end{eqnarray*}
Este $\zeta_{n}$ es el $n$-\'esimo segmento del proceso. El proceso es regenerativo sobre los tiempos $T_{n}$ si sus segmentos $\zeta_{n}$ son independientes e id\'enticamennte distribuidos.
\end{Def}


\begin{Note}
Si $\tilde{X}\left(t\right)$ con espacio de estados $\tilde{S}$ es regenerativo sobre $T_{n}$, entonces $X\left(t\right)=f\left(\tilde{X}\left(t\right)\right)$ tambi\'en es regenerativo sobre $T_{n}$, para cualquier funci\'on $f:\tilde{S}\rightarrow S$.
\end{Note}

\begin{Note}
Los procesos regenerativos son crudamente regenerativos, pero no al rev\'es.
\end{Note}


\begin{Note}
Un proceso estoc\'astico a tiempo continuo o discreto es regenerativo si existe un proceso de renovaci\'on  tal que los segmentos del proceso entre tiempos de renovaci\'on sucesivos son i.i.d., es decir, para $\left\{X\left(t\right):t\geq0\right\}$ proceso estoc\'astico a tiempo continuo con espacio de estados $S$, espacio m\'etrico.
\end{Note}

Para $\left\{X\left(t\right):t\geq0\right\}$ Proceso Estoc\'astico a tiempo continuo con estado de espacios $S$, que es un espacio m\'etrico, con trayectorias continuas por la derecha y con l\'imites por la izquierda c.s. Sea $N\left(t\right)$ un proceso de renovaci\'on en $\rea_{+}$ definido en el mismo espacio de probabilidad que $X\left(t\right)$, con tiempos de renovaci\'on $T$ y tiempos de inter-renovaci\'on $\xi_{n}=T_{n}-T_{n-1}$, con misma distribuci\'on $F$ de media finita $\mu$.



\begin{Def}
Para el proceso $\left\{\left(N\left(t\right),X\left(t\right)\right):t\geq0\right\}$, sus trayectoria muestrales en el intervalo de tiempo $\left[T_{n-1},T_{n}\right)$ est\'an descritas por
\begin{eqnarray*}
\zeta_{n}=\left(\xi_{n},\left\{X\left(T_{n-1}+t\right):0\leq t<\xi_{n}\right\}\right)
\end{eqnarray*}
Este $\zeta_{n}$ es el $n$-\'esimo segmento del proceso. El proceso es regenerativo sobre los tiempos $T_{n}$ si sus segmentos $\zeta_{n}$ son independientes e id\'enticamennte distribuidos.
\end{Def}

\begin{Note}
Un proceso regenerativo con media de la longitud de ciclo finita es llamado positivo recurrente.
\end{Note}

\begin{Teo}[Procesos Regenerativos]
Suponga que el proceso
\end{Teo}


\begin{Def}[Renewal Process Trinity]
Para un proceso de renovaci\'on $N\left(t\right)$, los siguientes procesos proveen de informaci\'on sobre los tiempos de renovaci\'on.
\begin{itemize}
\item $A\left(t\right)=t-T_{N\left(t\right)}$, el tiempo de recurrencia hacia atr\'as al tiempo $t$, que es el tiempo desde la \'ultima renovaci\'on para $t$.

\item $B\left(t\right)=T_{N\left(t\right)+1}-t$, el tiempo de recurrencia hacia adelante al tiempo $t$, residual del tiempo de renovaci\'on, que es el tiempo para la pr\'oxima renovaci\'on despu\'es de $t$.

\item $L\left(t\right)=\xi_{N\left(t\right)+1}=A\left(t\right)+B\left(t\right)$, la longitud del intervalo de renovaci\'on que contiene a $t$.
\end{itemize}
\end{Def}

\begin{Note}
El proceso tridimensional $\left(A\left(t\right),B\left(t\right),L\left(t\right)\right)$ es regenerativo sobre $T_{n}$, y por ende cada proceso lo es. Cada proceso $A\left(t\right)$ y $B\left(t\right)$ son procesos de MArkov a tiempo continuo con trayectorias continuas por partes en el espacio de estados $\rea_{+}$. Una expresi\'on conveniente para su distribuci\'on conjunta es, para $0\leq x<t,y\geq0$
\begin{equation}\label{NoRenovacion}
P\left\{A\left(t\right)>x,B\left(t\right)>y\right\}=
P\left\{N\left(t+y\right)-N\left((t-x)\right)=0\right\}
\end{equation}
\end{Note}

\begin{Ejem}[Tiempos de recurrencia Poisson]
Si $N\left(t\right)$ es un proceso Poisson con tasa $\lambda$, entonces de la expresi\'on (\ref{NoRenovacion}) se tiene que

\begin{eqnarray*}
\begin{array}{lc}
P\left\{A\left(t\right)>x,B\left(t\right)>y\right\}=e^{-\lambda\left(x+y\right)},&0\leq x<t,y\geq0,
\end{array}
\end{eqnarray*}
que es la probabilidad Poisson de no renovaciones en un intervalo de longitud $x+y$.

\end{Ejem}

%\begin{Note}
Una cadena de Markov erg\'odica tiene la propiedad de ser estacionaria si la distribuci\'on de su estado al tiempo $0$ es su distribuci\'on estacionaria.
%\end{Note}


\begin{Def}
Un proceso estoc\'astico a tiempo continuo $\left\{X\left(t\right):t\geq0\right\}$ en un espacio general es estacionario si sus distribuciones finito dimensionales son invariantes bajo cualquier  traslado: para cada $0\leq s_{1}<s_{2}<\cdots<s_{k}$ y $t\geq0$,
\begin{eqnarray*}
\left(X\left(s_{1}+t\right),\ldots,X\left(s_{k}+t\right)\right)=_{d}\left(X\left(s_{1}\right),\ldots,X\left(s_{k}\right)\right).
\end{eqnarray*}
\end{Def}

\begin{Note}
Un proceso de Markov es estacionario si $X\left(t\right)=_{d}X\left(0\right)$, $t\geq0$.
\end{Note}

Considerese el proceso $N\left(t\right)=\sum_{n}\indora\left(\tau_{n}\leq t\right)$ en $\rea_{+}$, con puntos $0<\tau_{1}<\tau_{2}<\cdots$.

\begin{Prop}
Si $N$ es un proceso puntual estacionario y $\esp\left[N\left(1\right)\right]<\infty$, entonces $\esp\left[N\left(t\right)\right]=t\esp\left[N\left(1\right)\right]$, $t\geq0$

\end{Prop}

\begin{Teo}
Los siguientes enunciados son equivalentes
\begin{itemize}
\item[i)] El proceso retardado de renovaci\'on $N$ es estacionario.

\item[ii)] EL proceso de tiempos de recurrencia hacia adelante $B\left(t\right)$ es estacionario.


\item[iii)] $\esp\left[N\left(t\right)\right]=t/\mu$,


\item[iv)] $G\left(t\right)=F_{e}\left(t\right)=\frac{1}{\mu}\int_{0}^{t}\left[1-F\left(s\right)\right]ds$
\end{itemize}
Cuando estos enunciados son ciertos, $P\left\{B\left(t\right)\leq x\right\}=F_{e}\left(x\right)$, para $t,x\geq0$.

\end{Teo}

\begin{Note}
Una consecuencia del teorema anterior es que el Proceso Poisson es el \'unico proceso sin retardo que es estacionario.
\end{Note}

\begin{Coro}
El proceso de renovaci\'on $N\left(t\right)$ sin retardo, y cuyos tiempos de inter renonaci\'on tienen media finita, es estacionario si y s\'olo si es un proceso Poisson.

\end{Coro}


%________________________________________________________________________
\subsection{Procesos Regenerativos}
%________________________________________________________________________



\begin{Note}
Si $\tilde{X}\left(t\right)$ con espacio de estados $\tilde{S}$ es regenerativo sobre $T_{n}$, entonces $X\left(t\right)=f\left(\tilde{X}\left(t\right)\right)$ tambi\'en es regenerativo sobre $T_{n}$, para cualquier funci\'on $f:\tilde{S}\rightarrow S$.
\end{Note}

\begin{Note}
Los procesos regenerativos son crudamente regenerativos, pero no al rev\'es.
\end{Note}
%\subsection*{Procesos Regenerativos: Sigman\cite{Sigman1}}
\begin{Def}[Definici\'on Cl\'asica]
Un proceso estoc\'astico $X=\left\{X\left(t\right):t\geq0\right\}$ es llamado regenerativo is existe una variable aleatoria $R_{1}>0$ tal que
\begin{itemize}
\item[i)] $\left\{X\left(t+R_{1}\right):t\geq0\right\}$ es independiente de $\left\{\left\{X\left(t\right):t<R_{1}\right\},\right\}$
\item[ii)] $\left\{X\left(t+R_{1}\right):t\geq0\right\}$ es estoc\'asticamente equivalente a $\left\{X\left(t\right):t>0\right\}$
\end{itemize}

Llamamos a $R_{1}$ tiempo de regeneraci\'on, y decimos que $X$ se regenera en este punto.
\end{Def}

$\left\{X\left(t+R_{1}\right)\right\}$ es regenerativo con tiempo de regeneraci\'on $R_{2}$, independiente de $R_{1}$ pero con la misma distribuci\'on que $R_{1}$. Procediendo de esta manera se obtiene una secuencia de variables aleatorias independientes e id\'enticamente distribuidas $\left\{R_{n}\right\}$ llamados longitudes de ciclo. Si definimos a $Z_{k}\equiv R_{1}+R_{2}+\cdots+R_{k}$, se tiene un proceso de renovaci\'on llamado proceso de renovaci\'on encajado para $X$.




\begin{Def}
Para $x$ fijo y para cada $t\geq0$, sea $I_{x}\left(t\right)=1$ si $X\left(t\right)\leq x$,  $I_{x}\left(t\right)=0$ en caso contrario, y def\'inanse los tiempos promedio
\begin{eqnarray*}
\overline{X}&=&lim_{t\rightarrow\infty}\frac{1}{t}\int_{0}^{\infty}X\left(u\right)du\\
\prob\left(X_{\infty}\leq x\right)&=&lim_{t\rightarrow\infty}\frac{1}{t}\int_{0}^{\infty}I_{x}\left(u\right)du,
\end{eqnarray*}
cuando estos l\'imites existan.
\end{Def}

Como consecuencia del teorema de Renovaci\'on-Recompensa, se tiene que el primer l\'imite  existe y es igual a la constante
\begin{eqnarray*}
\overline{X}&=&\frac{\esp\left[\int_{0}^{R_{1}}X\left(t\right)dt\right]}{\esp\left[R_{1}\right]},
\end{eqnarray*}
suponiendo que ambas esperanzas son finitas.

\begin{Note}
\begin{itemize}
\item[a)] Si el proceso regenerativo $X$ es positivo recurrente y tiene trayectorias muestrales no negativas, entonces la ecuaci\'on anterior es v\'alida.
\item[b)] Si $X$ es positivo recurrente regenerativo, podemos construir una \'unica versi\'on estacionaria de este proceso, $X_{e}=\left\{X_{e}\left(t\right)\right\}$, donde $X_{e}$ es un proceso estoc\'astico regenerativo y estrictamente estacionario, con distribuci\'on marginal distribuida como $X_{\infty}$
\end{itemize}
\end{Note}

%________________________________________________________________________
%\subsection{Procesos Regenerativos}
%________________________________________________________________________

Para $\left\{X\left(t\right):t\geq0\right\}$ Proceso Estoc\'astico a tiempo continuo con estado de espacios $S$, que es un espacio m\'etrico, con trayectorias continuas por la derecha y con l\'imites por la izquierda c.s. Sea $N\left(t\right)$ un proceso de renovaci\'on en $\rea_{+}$ definido en el mismo espacio de probabilidad que $X\left(t\right)$, con tiempos de renovaci\'on $T$ y tiempos de inter-renovaci\'on $\xi_{n}=T_{n}-T_{n-1}$, con misma distribuci\'on $F$ de media finita $\mu$.



\begin{Def}
Para el proceso $\left\{\left(N\left(t\right),X\left(t\right)\right):t\geq0\right\}$, sus trayectoria muestrales en el intervalo de tiempo $\left[T_{n-1},T_{n}\right)$ est\'an descritas por
\begin{eqnarray*}
\zeta_{n}=\left(\xi_{n},\left\{X\left(T_{n-1}+t\right):0\leq t<\xi_{n}\right\}\right)
\end{eqnarray*}
Este $\zeta_{n}$ es el $n$-\'esimo segmento del proceso. El proceso es regenerativo sobre los tiempos $T_{n}$ si sus segmentos $\zeta_{n}$ son independientes e id\'enticamennte distribuidos.
\end{Def}


\begin{Note}
Si $\tilde{X}\left(t\right)$ con espacio de estados $\tilde{S}$ es regenerativo sobre $T_{n}$, entonces $X\left(t\right)=f\left(\tilde{X}\left(t\right)\right)$ tambi\'en es regenerativo sobre $T_{n}$, para cualquier funci\'on $f:\tilde{S}\rightarrow S$.
\end{Note}

\begin{Note}
Los procesos regenerativos son crudamente regenerativos, pero no al rev\'es.
\end{Note}

\begin{Def}[Definici\'on Cl\'asica]
Un proceso estoc\'astico $X=\left\{X\left(t\right):t\geq0\right\}$ es llamado regenerativo is existe una variable aleatoria $R_{1}>0$ tal que
\begin{itemize}
\item[i)] $\left\{X\left(t+R_{1}\right):t\geq0\right\}$ es independiente de $\left\{\left\{X\left(t\right):t<R_{1}\right\},\right\}$
\item[ii)] $\left\{X\left(t+R_{1}\right):t\geq0\right\}$ es estoc\'asticamente equivalente a $\left\{X\left(t\right):t>0\right\}$
\end{itemize}

Llamamos a $R_{1}$ tiempo de regeneraci\'on, y decimos que $X$ se regenera en este punto.
\end{Def}

$\left\{X\left(t+R_{1}\right)\right\}$ es regenerativo con tiempo de regeneraci\'on $R_{2}$, independiente de $R_{1}$ pero con la misma distribuci\'on que $R_{1}$. Procediendo de esta manera se obtiene una secuencia de variables aleatorias independientes e id\'enticamente distribuidas $\left\{R_{n}\right\}$ llamados longitudes de ciclo. Si definimos a $Z_{k}\equiv R_{1}+R_{2}+\cdots+R_{k}$, se tiene un proceso de renovaci\'on llamado proceso de renovaci\'on encajado para $X$.

\begin{Note}
Un proceso regenerativo con media de la longitud de ciclo finita es llamado positivo recurrente.
\end{Note}


\begin{Def}
Para $x$ fijo y para cada $t\geq0$, sea $I_{x}\left(t\right)=1$ si $X\left(t\right)\leq x$,  $I_{x}\left(t\right)=0$ en caso contrario, y def\'inanse los tiempos promedio
\begin{eqnarray*}
\overline{X}&=&lim_{t\rightarrow\infty}\frac{1}{t}\int_{0}^{\infty}X\left(u\right)du\\
\prob\left(X_{\infty}\leq x\right)&=&lim_{t\rightarrow\infty}\frac{1}{t}\int_{0}^{\infty}I_{x}\left(u\right)du,
\end{eqnarray*}
cuando estos l\'imites existan.
\end{Def}

Como consecuencia del teorema de Renovaci\'on-Recompensa, se tiene que el primer l\'imite  existe y es igual a la constante
\begin{eqnarray*}
\overline{X}&=&\frac{\esp\left[\int_{0}^{R_{1}}X\left(t\right)dt\right]}{\esp\left[R_{1}\right]},
\end{eqnarray*}
suponiendo que ambas esperanzas son finitas.

\begin{Note}
\begin{itemize}
\item[a)] Si el proceso regenerativo $X$ es positivo recurrente y tiene trayectorias muestrales no negativas, entonces la ecuaci\'on anterior es v\'alida.
\item[b)] Si $X$ es positivo recurrente regenerativo, podemos construir una \'unica versi\'on estacionaria de este proceso, $X_{e}=\left\{X_{e}\left(t\right)\right\}$, donde $X_{e}$ es un proceso estoc\'astico regenerativo y estrictamente estacionario, con distribuci\'on marginal distribuida como $X_{\infty}$
\end{itemize}
\end{Note}

%__________________________________________________________________________________________
%\subsection{Procesos Regenerativos Estacionarios - Stidham \cite{Stidham}}
%__________________________________________________________________________________________


Un proceso estoc\'astico a tiempo continuo $\left\{V\left(t\right),t\geq0\right\}$ es un proceso regenerativo si existe una sucesi\'on de variables aleatorias independientes e id\'enticamente distribuidas $\left\{X_{1},X_{2},\ldots\right\}$, sucesi\'on de renovaci\'on, tal que para cualquier conjunto de Borel $A$, 

\begin{eqnarray*}
\prob\left\{V\left(t\right)\in A|X_{1}+X_{2}+\cdots+X_{R\left(t\right)}=s,\left\{V\left(\tau\right),\tau<s\right\}\right\}=\prob\left\{V\left(t-s\right)\in A|X_{1}>t-s\right\},
\end{eqnarray*}
para todo $0\leq s\leq t$, donde $R\left(t\right)=\max\left\{X_{1}+X_{2}+\cdots+X_{j}\leq t\right\}=$n\'umero de renovaciones ({\emph{puntos de regeneraci\'on}}) que ocurren en $\left[0,t\right]$. El intervalo $\left[0,X_{1}\right)$ es llamado {\emph{primer ciclo de regeneraci\'on}} de $\left\{V\left(t \right),t\geq0\right\}$, $\left[X_{1},X_{1}+X_{2}\right)$ el {\emph{segundo ciclo de regeneraci\'on}}, y as\'i sucesivamente.

Sea $X=X_{1}$ y sea $F$ la funci\'on de distrbuci\'on de $X$


\begin{Def}
Se define el proceso estacionario, $\left\{V^{*}\left(t\right),t\geq0\right\}$, para $\left\{V\left(t\right),t\geq0\right\}$ por

\begin{eqnarray*}
\prob\left\{V\left(t\right)\in A\right\}=\frac{1}{\esp\left[X\right]}\int_{0}^{\infty}\prob\left\{V\left(t+x\right)\in A|X>x\right\}\left(1-F\left(x\right)\right)dx,
\end{eqnarray*} 
para todo $t\geq0$ y todo conjunto de Borel $A$.
\end{Def}

\begin{Def}
Una distribuci\'on se dice que es {\emph{aritm\'etica}} si todos sus puntos de incremento son m\'ultiplos de la forma $0,\lambda, 2\lambda,\ldots$ para alguna $\lambda>0$ entera.
\end{Def}


\begin{Def}
Una modificaci\'on medible de un proceso $\left\{V\left(t\right),t\geq0\right\}$, es una versi\'on de este, $\left\{V\left(t,w\right)\right\}$ conjuntamente medible para $t\geq0$ y para $w\in S$, $S$ espacio de estados para $\left\{V\left(t\right),t\geq0\right\}$.
\end{Def}

\begin{Teo}
Sea $\left\{V\left(t\right),t\geq\right\}$ un proceso regenerativo no negativo con modificaci\'on medible. Sea $\esp\left[X\right]<\infty$. Entonces el proceso estacionario dado por la ecuaci\'on anterior est\'a bien definido y tiene funci\'on de distribuci\'on independiente de $t$, adem\'as
\begin{itemize}
\item[i)] \begin{eqnarray*}
\esp\left[V^{*}\left(0\right)\right]&=&\frac{\esp\left[\int_{0}^{X}V\left(s\right)ds\right]}{\esp\left[X\right]}\end{eqnarray*}
\item[ii)] Si $\esp\left[V^{*}\left(0\right)\right]<\infty$, equivalentemente, si $\esp\left[\int_{0}^{X}V\left(s\right)ds\right]<\infty$,entonces
\begin{eqnarray*}
\frac{\int_{0}^{t}V\left(s\right)ds}{t}\rightarrow\frac{\esp\left[\int_{0}^{X}V\left(s\right)ds\right]}{\esp\left[X\right]}
\end{eqnarray*}
con probabilidad 1 y en media, cuando $t\rightarrow\infty$.
\end{itemize}
\end{Teo}

%__________________________________________________________________________________________
%\subsection{Procesos Regenerativos Estacionarios - Stidham \cite{Stidham}}
%__________________________________________________________________________________________


Un proceso estoc\'astico a tiempo continuo $\left\{V\left(t\right),t\geq0\right\}$ es un proceso regenerativo si existe una sucesi\'on de variables aleatorias independientes e id\'enticamente distribuidas $\left\{X_{1},X_{2},\ldots\right\}$, sucesi\'on de renovaci\'on, tal que para cualquier conjunto de Borel $A$, 

\begin{eqnarray*}
\prob\left\{V\left(t\right)\in A|X_{1}+X_{2}+\cdots+X_{R\left(t\right)}=s,\left\{V\left(\tau\right),\tau<s\right\}\right\}=\prob\left\{V\left(t-s\right)\in A|X_{1}>t-s\right\},
\end{eqnarray*}
para todo $0\leq s\leq t$, donde $R\left(t\right)=\max\left\{X_{1}+X_{2}+\cdots+X_{j}\leq t\right\}=$n\'umero de renovaciones ({\emph{puntos de regeneraci\'on}}) que ocurren en $\left[0,t\right]$. El intervalo $\left[0,X_{1}\right)$ es llamado {\emph{primer ciclo de regeneraci\'on}} de $\left\{V\left(t \right),t\geq0\right\}$, $\left[X_{1},X_{1}+X_{2}\right)$ el {\emph{segundo ciclo de regeneraci\'on}}, y as\'i sucesivamente.

Sea $X=X_{1}$ y sea $F$ la funci\'on de distrbuci\'on de $X$


\begin{Def}
Se define el proceso estacionario, $\left\{V^{*}\left(t\right),t\geq0\right\}$, para $\left\{V\left(t\right),t\geq0\right\}$ por

\begin{eqnarray*}
\prob\left\{V\left(t\right)\in A\right\}=\frac{1}{\esp\left[X\right]}\int_{0}^{\infty}\prob\left\{V\left(t+x\right)\in A|X>x\right\}\left(1-F\left(x\right)\right)dx,
\end{eqnarray*} 
para todo $t\geq0$ y todo conjunto de Borel $A$.
\end{Def}

\begin{Def}
Una distribuci\'on se dice que es {\emph{aritm\'etica}} si todos sus puntos de incremento son m\'ultiplos de la forma $0,\lambda, 2\lambda,\ldots$ para alguna $\lambda>0$ entera.
\end{Def}


\begin{Def}
Una modificaci\'on medible de un proceso $\left\{V\left(t\right),t\geq0\right\}$, es una versi\'on de este, $\left\{V\left(t,w\right)\right\}$ conjuntamente medible para $t\geq0$ y para $w\in S$, $S$ espacio de estados para $\left\{V\left(t\right),t\geq0\right\}$.
\end{Def}

\begin{Teo}
Sea $\left\{V\left(t\right),t\geq\right\}$ un proceso regenerativo no negativo con modificaci\'on medible. Sea $\esp\left[X\right]<\infty$. Entonces el proceso estacionario dado por la ecuaci\'on anterior est\'a bien definido y tiene funci\'on de distribuci\'on independiente de $t$, adem\'as
\begin{itemize}
\item[i)] \begin{eqnarray*}
\esp\left[V^{*}\left(0\right)\right]&=&\frac{\esp\left[\int_{0}^{X}V\left(s\right)ds\right]}{\esp\left[X\right]}\end{eqnarray*}
\item[ii)] Si $\esp\left[V^{*}\left(0\right)\right]<\infty$, equivalentemente, si $\esp\left[\int_{0}^{X}V\left(s\right)ds\right]<\infty$,entonces
\begin{eqnarray*}
\frac{\int_{0}^{t}V\left(s\right)ds}{t}\rightarrow\frac{\esp\left[\int_{0}^{X}V\left(s\right)ds\right]}{\esp\left[X\right]}
\end{eqnarray*}
con probabilidad 1 y en media, cuando $t\rightarrow\infty$.
\end{itemize}
\end{Teo}

Para $\left\{X\left(t\right):t\geq0\right\}$ Proceso Estoc\'astico a tiempo continuo con estado de espacios $S$, que es un espacio m\'etrico, con trayectorias continuas por la derecha y con l\'imites por la izquierda c.s. Sea $N\left(t\right)$ un proceso de renovaci\'on en $\rea_{+}$ definido en el mismo espacio de probabilidad que $X\left(t\right)$, con tiempos de renovaci\'on $T$ y tiempos de inter-renovaci\'on $\xi_{n}=T_{n}-T_{n-1}$, con misma distribuci\'on $F$ de media finita $\mu$.
%_____________________________________________________
\subsection{Puntos de Renovaci\'on}
%_____________________________________________________

Para cada cola $Q_{i}$ se tienen los procesos de arribo a la cola, para estas, los tiempos de arribo est\'an dados por $$\left\{T_{1}^{i},T_{2}^{i},\ldots,T_{k}^{i},\ldots\right\},$$ entonces, consideremos solamente los primeros tiempos de arribo a cada una de las colas, es decir, $$\left\{T_{1}^{1},T_{1}^{2},T_{1}^{3},T_{1}^{4}\right\},$$ se sabe que cada uno de estos tiempos se distribuye de manera exponencial con par\'ametro $1/mu_{i}$. Adem\'as se sabe que para $$T^{*}=\min\left\{T_{1}^{1},T_{1}^{2},T_{1}^{3},T_{1}^{4}\right\},$$ $T^{*}$ se distribuye de manera exponencial con par\'ametro $$\mu^{*}=\sum_{i=1}^{4}\mu_{i}.$$ Ahora, dado que 
\begin{center}
\begin{tabular}{lcl}
$\tilde{r}=r_{1}+r_{2}$ & y &$\hat{r}=r_{3}+r_{4}.$
\end{tabular}
\end{center}


Supongamos que $$\tilde{r},\hat{r}<\mu^{*},$$ entonces si tomamos $$r^{*}=\min\left\{\tilde{r},\hat{r}\right\},$$ se tiene que para  $$t^{*}\in\left(0,r^{*}\right)$$ se cumple que 
\begin{center}
\begin{tabular}{lcl}
$\tau_{1}\left(1\right)=0$ & y por tanto & $\overline{\tau}_{1}=0,$
\end{tabular}
\end{center}
entonces para la segunda cola en este primer ciclo se cumple que $$\tau_{2}=\overline{\tau}_{1}+r_{1}=r_{1}<\mu^{*},$$ y por tanto se tiene que  $$\overline{\tau}_{2}=\tau_{2}.$$ Por lo tanto, nuevamente para la primer cola en el segundo ciclo $$\tau_{1}\left(2\right)=\tau_{2}\left(1\right)+r_{2}=\tilde{r}<\mu^{*}.$$ An\'alogamente para el segundo sistema se tiene que ambas colas est\'an vac\'ias, es decir, existe un valor $t^{*}$ tal que en el intervalo $\left(0,t^{*}\right)$ no ha llegado ning\'un usuario, es decir, $$L_{i}\left(t^{*}\right)=0$$ para $i=1,2,3,4$.





%________________________________________________________________________
\subsection{Procesos Regenerativos}
%________________________________________________________________________

Para $\left\{X\left(t\right):t\geq0\right\}$ Proceso Estoc\'astico a tiempo continuo con estado de espacios $S$, que es un espacio m\'etrico, con trayectorias continuas por la derecha y con l\'imites por la izquierda c.s. Sea $N\left(t\right)$ un proceso de renovaci\'on en $\rea_{+}$ definido en el mismo espacio de probabilidad que $X\left(t\right)$, con tiempos de renovaci\'on $T$ y tiempos de inter-renovaci\'on $\xi_{n}=T_{n}-T_{n-1}$, con misma distribuci\'on $F$ de media finita $\mu$.



\begin{Def}
Para el proceso $\left\{\left(N\left(t\right),X\left(t\right)\right):t\geq0\right\}$, sus trayectoria muestrales en el intervalo de tiempo $\left[T_{n-1},T_{n}\right)$ est\'an descritas por
\begin{eqnarray*}
\zeta_{n}=\left(\xi_{n},\left\{X\left(T_{n-1}+t\right):0\leq t<\xi_{n}\right\}\right)
\end{eqnarray*}
Este $\zeta_{n}$ es el $n$-\'esimo segmento del proceso. El proceso es regenerativo sobre los tiempos $T_{n}$ si sus segmentos $\zeta_{n}$ son independientes e id\'enticamennte distribuidos.
\end{Def}


\begin{Obs}
Si $\tilde{X}\left(t\right)$ con espacio de estados $\tilde{S}$ es regenerativo sobre $T_{n}$, entonces $X\left(t\right)=f\left(\tilde{X}\left(t\right)\right)$ tambi\'en es regenerativo sobre $T_{n}$, para cualquier funci\'on $f:\tilde{S}\rightarrow S$.
\end{Obs}

\begin{Obs}
Los procesos regenerativos son crudamente regenerativos, pero no al rev\'es.
\end{Obs}

\begin{Def}[Definici\'on Cl\'asica]
Un proceso estoc\'astico $X=\left\{X\left(t\right):t\geq0\right\}$ es llamado regenerativo is existe una variable aleatoria $R_{1}>0$ tal que
\begin{itemize}
\item[i)] $\left\{X\left(t+R_{1}\right):t\geq0\right\}$ es independiente de $\left\{\left\{X\left(t\right):t<R_{1}\right\},\right\}$
\item[ii)] $\left\{X\left(t+R_{1}\right):t\geq0\right\}$ es estoc\'asticamente equivalente a $\left\{X\left(t\right):t>0\right\}$
\end{itemize}

Llamamos a $R_{1}$ tiempo de regeneraci\'on, y decimos que $X$ se regenera en este punto.
\end{Def}

$\left\{X\left(t+R_{1}\right)\right\}$ es regenerativo con tiempo de regeneraci\'on $R_{2}$, independiente de $R_{1}$ pero con la misma distribuci\'on que $R_{1}$. Procediendo de esta manera se obtiene una secuencia de variables aleatorias independientes e id\'enticamente distribuidas $\left\{R_{n}\right\}$ llamados longitudes de ciclo. Si definimos a $Z_{k}\equiv R_{1}+R_{2}+\cdots+R_{k}$, se tiene un proceso de renovaci\'on llamado proceso de renovaci\'on encajado para $X$.

\begin{Note}
Un proceso regenerativo con media de la longitud de ciclo finita es llamado positivo recurrente.
\end{Note}


\begin{Def}
Para $x$ fijo y para cada $t\geq0$, sea $I_{x}\left(t\right)=1$ si $X\left(t\right)\leq x$,  $I_{x}\left(t\right)=0$ en caso contrario, y def\'inanse los tiempos promedio
\begin{eqnarray*}
\overline{X}&=&lim_{t\rightarrow\infty}\frac{1}{t}\int_{0}^{\infty}X\left(u\right)du\\
\prob\left(X_{\infty}\leq x\right)&=&lim_{t\rightarrow\infty}\frac{1}{t}\int_{0}^{\infty}I_{x}\left(u\right)du,
\end{eqnarray*}
cuando estos l\'imites existan.
\end{Def}

Como consecuencia del teorema de Renovaci\'on-Recompensa, se tiene que el primer l\'imite  existe y es igual a la constante
\begin{eqnarray*}
\overline{X}&=&\frac{\esp\left[\int_{0}^{R_{1}}X\left(t\right)dt\right]}{\esp\left[R_{1}\right]},
\end{eqnarray*}
suponiendo que ambas esperanzas son finitas.

\begin{Note}
\begin{itemize}
\item[a)] Si el proceso regenerativo $X$ es positivo recurrente y tiene trayectorias muestrales no negativas, entonces la ecuaci\'on anterior es v\'alida.
\item[b)] Si $X$ es positivo recurrente regenerativo, podemos construir una \'unica versi\'on estacionaria de este proceso, $X_{e}=\left\{X_{e}\left(t\right)\right\}$, donde $X_{e}$ es un proceso estoc\'astico regenerativo y estrictamente estacionario, con distribuci\'on marginal distribuida como $X_{\infty}$
\end{itemize}
\end{Note}

\subsection{Renewal and Regenerative Processes: Serfozo\cite{Serfozo}}
\begin{Def}\label{Def.Tn}
Sean $0\leq T_{1}\leq T_{2}\leq \ldots$ son tiempos aleatorios infinitos en los cuales ocurren ciertos eventos. El n\'umero de tiempos $T_{n}$ en el intervalo $\left[0,t\right)$ es

\begin{eqnarray}
N\left(t\right)=\sum_{n=1}^{\infty}\indora\left(T_{n}\leq t\right),
\end{eqnarray}
para $t\geq0$.
\end{Def}

Si se consideran los puntos $T_{n}$ como elementos de $\rea_{+}$, y $N\left(t\right)$ es el n\'umero de puntos en $\rea$. El proceso denotado por $\left\{N\left(t\right):t\geq0\right\}$, denotado por $N\left(t\right)$, es un proceso puntual en $\rea_{+}$. Los $T_{n}$ son los tiempos de ocurrencia, el proceso puntual $N\left(t\right)$ es simple si su n\'umero de ocurrencias son distintas: $0<T_{1}<T_{2}<\ldots$ casi seguramente.

\begin{Def}
Un proceso puntual $N\left(t\right)$ es un proceso de renovaci\'on si los tiempos de interocurrencia $\xi_{n}=T_{n}-T_{n-1}$, para $n\geq1$, son independientes e identicamente distribuidos con distribuci\'on $F$, donde $F\left(0\right)=0$ y $T_{0}=0$. Los $T_{n}$ son llamados tiempos de renovaci\'on, referente a la independencia o renovaci\'on de la informaci\'on estoc\'astica en estos tiempos. Los $\xi_{n}$ son los tiempos de inter-renovaci\'on, y $N\left(t\right)$ es el n\'umero de renovaciones en el intervalo $\left[0,t\right)$
\end{Def}


\begin{Note}
Para definir un proceso de renovaci\'on para cualquier contexto, solamente hay que especificar una distribuci\'on $F$, con $F\left(0\right)=0$, para los tiempos de inter-renovaci\'on. La funci\'on $F$ en turno degune las otra variables aleatorias. De manera formal, existe un espacio de probabilidad y una sucesi\'on de variables aleatorias $\xi_{1},\xi_{2},\ldots$ definidas en este con distribuci\'on $F$. Entonces las otras cantidades son $T_{n}=\sum_{k=1}^{n}\xi_{k}$ y $N\left(t\right)=\sum_{n=1}^{\infty}\indora\left(T_{n}\leq t\right)$, donde $T_{n}\rightarrow\infty$ casi seguramente por la Ley Fuerte de los Grandes N\'umeros.
\end{Note}


Los tiempos $T_{n}$ est\'an relacionados con los conteos de $N\left(t\right)$ por

\begin{eqnarray*}
\left\{N\left(t\right)\geq n\right\}&=&\left\{T_{n}\leq t\right\}\\
T_{N\left(t\right)}\leq &t&<T_{N\left(t\right)+1},
\end{eqnarray*}

adem\'as $N\left(T_{n}\right)=n$, y 

\begin{eqnarray*}
N\left(t\right)=\max\left\{n:T_{n}\leq t\right\}=\min\left\{n:T_{n+1}>t\right\}
\end{eqnarray*}

Por propiedades de la convoluci\'on se sabe que

\begin{eqnarray*}
P\left\{T_{n}\leq t\right\}=F^{n\star}\left(t\right)
\end{eqnarray*}
que es la $n$-\'esima convoluci\'on de $F$. Entonces 

\begin{eqnarray*}
\left\{N\left(t\right)\geq n\right\}&=&\left\{T_{n}\leq t\right\}\\
P\left\{N\left(t\right)\leq n\right\}&=&1-F^{\left(n+1\right)\star}\left(t\right)
\end{eqnarray*}

Adem\'as usando el hecho de que $\esp\left[N\left(t\right)\right]=\sum_{n=1}^{\infty}P\left\{N\left(t\right)\geq n\right\}$
se tiene que

\begin{eqnarray*}
\esp\left[N\left(t\right)\right]=\sum_{n=1}^{\infty}F^{n\star}\left(t\right)
\end{eqnarray*}

\begin{Prop}
Para cada $t\geq0$, la funci\'on generadora de momentos $\esp\left[e^{\alpha N\left(t\right)}\right]$ existe para alguna $\alpha$ en una vecindad del 0, y de aqu\'i que $\esp\left[N\left(t\right)^{m}\right]<\infty$, para $m\geq1$.
\end{Prop}


\begin{Note}
Si el primer tiempo de renovaci\'on $\xi_{1}$ no tiene la misma distribuci\'on que el resto de las $\xi_{n}$, para $n\geq2$, a $N\left(t\right)$ se le llama Proceso de Renovaci\'on retardado, donde si $\xi$ tiene distribuci\'on $G$, entonces el tiempo $T_{n}$ de la $n$-\'esima renovaci\'on tiene distribuci\'on $G\star F^{\left(n-1\right)\star}\left(t\right)$
\end{Note}


\begin{Teo}
Para una constante $\mu\leq\infty$ ( o variable aleatoria), las siguientes expresiones son equivalentes:

\begin{eqnarray}
lim_{n\rightarrow\infty}n^{-1}T_{n}&=&\mu,\textrm{ c.s.}\\
lim_{t\rightarrow\infty}t^{-1}N\left(t\right)&=&1/\mu,\textrm{ c.s.}
\end{eqnarray}
\end{Teo}


Es decir, $T_{n}$ satisface la Ley Fuerte de los Grandes N\'umeros s\'i y s\'olo s\'i $N\left/t\right)$ la cumple.


\begin{Coro}[Ley Fuerte de los Grandes N\'umeros para Procesos de Renovaci\'on]
Si $N\left(t\right)$ es un proceso de renovaci\'on cuyos tiempos de inter-renovaci\'on tienen media $\mu\leq\infty$, entonces
\begin{eqnarray}
t^{-1}N\left(t\right)\rightarrow 1/\mu,\textrm{ c.s. cuando }t\rightarrow\infty.
\end{eqnarray}

\end{Coro}


Considerar el proceso estoc\'astico de valores reales $\left\{Z\left(t\right):t\geq0\right\}$ en el mismo espacio de probabilidad que $N\left(t\right)$

\begin{Def}
Para el proceso $\left\{Z\left(t\right):t\geq0\right\}$ se define la fluctuaci\'on m\'axima de $Z\left(t\right)$ en el intervalo $\left(T_{n-1},T_{n}\right]$:
\begin{eqnarray*}
M_{n}=\sup_{T_{n-1}<t\leq T_{n}}|Z\left(t\right)-Z\left(T_{n-1}\right)|
\end{eqnarray*}
\end{Def}

\begin{Teo}
Sup\'ongase que $n^{-1}T_{n}\rightarrow\mu$ c.s. cuando $n\rightarrow\infty$, donde $\mu\leq\infty$ es una constante o variable aleatoria. Sea $a$ una constante o variable aleatoria que puede ser infinita cuando $\mu$ es finita, y considere las expresiones l\'imite:
\begin{eqnarray}
lim_{n\rightarrow\infty}n^{-1}Z\left(T_{n}\right)&=&a,\textrm{ c.s.}\\
lim_{t\rightarrow\infty}t^{-1}Z\left(t\right)&=&a/\mu,\textrm{ c.s.}
\end{eqnarray}
La segunda expresi\'on implica la primera. Conversamente, la primera implica la segunda si el proceso $Z\left(t\right)$ es creciente, o si $lim_{n\rightarrow\infty}n^{-1}M_{n}=0$ c.s.
\end{Teo}

\begin{Coro}
Si $N\left(t\right)$ es un proceso de renovaci\'on, y $\left(Z\left(T_{n}\right)-Z\left(T_{n-1}\right),M_{n}\right)$, para $n\geq1$, son variables aleatorias independientes e id\'enticamente distribuidas con media finita, entonces,
\begin{eqnarray}
lim_{t\rightarrow\infty}t^{-1}Z\left(t\right)\rightarrow\frac{\esp\left[Z\left(T_{1}\right)-Z\left(T_{0}\right)\right]}{\esp\left[T_{1}\right]},\textrm{ c.s. cuando  }t\rightarrow\infty.
\end{eqnarray}
\end{Coro}


Sup\'ongase que $N\left(t\right)$ es un proceso de renovaci\'on con distribuci\'on $F$ con media finita $\mu$.

\begin{Def}
La funci\'on de renovaci\'on asociada con la distribuci\'on $F$, del proceso $N\left(t\right)$, es
\begin{eqnarray*}
U\left(t\right)=\sum_{n=1}^{\infty}F^{n\star}\left(t\right),\textrm{   }t\geq0,
\end{eqnarray*}
donde $F^{0\star}\left(t\right)=\indora\left(t\geq0\right)$.
\end{Def}


\begin{Prop}
Sup\'ongase que la distribuci\'on de inter-renovaci\'on $F$ tiene densidad $f$. Entonces $U\left(t\right)$ tambi\'en tiene densidad, para $t>0$, y es $U^{'}\left(t\right)=\sum_{n=0}^{\infty}f^{n\star}\left(t\right)$. Adem\'as
\begin{eqnarray*}
\prob\left\{N\left(t\right)>N\left(t-\right)\right\}=0\textrm{,   }t\geq0.
\end{eqnarray*}
\end{Prop}

\begin{Def}
La Transformada de Laplace-Stieljes de $F$ est\'a dada por

\begin{eqnarray*}
\hat{F}\left(\alpha\right)=\int_{\rea_{+}}e^{-\alpha t}dF\left(t\right)\textrm{,  }\alpha\geq0.
\end{eqnarray*}
\end{Def}

Entonces

\begin{eqnarray*}
\hat{U}\left(\alpha\right)=\sum_{n=0}^{\infty}\hat{F^{n\star}}\left(\alpha\right)=\sum_{n=0}^{\infty}\hat{F}\left(\alpha\right)^{n}=\frac{1}{1-\hat{F}\left(\alpha\right)}.
\end{eqnarray*}


\begin{Prop}
La Transformada de Laplace $\hat{U}\left(\alpha\right)$ y $\hat{F}\left(\alpha\right)$ determina una a la otra de manera \'unica por la relaci\'on $\hat{U}\left(\alpha\right)=\frac{1}{1-\hat{F}\left(\alpha\right)}$.
\end{Prop}


\begin{Note}
Un proceso de renovaci\'on $N\left(t\right)$ cuyos tiempos de inter-renovaci\'on tienen media finita, es un proceso Poisson con tasa $\lambda$ si y s\'olo s\'i $\esp\left[U\left(t\right)\right]=\lambda t$, para $t\geq0$.
\end{Note}


\begin{Teo}
Sea $N\left(t\right)$ un proceso puntual simple con puntos de localizaci\'on $T_{n}$ tal que $\eta\left(t\right)=\esp\left[N\left(\right)\right]$ es finita para cada $t$. Entonces para cualquier funci\'on $f:\rea_{+}\rightarrow\rea$,
\begin{eqnarray*}
\esp\left[\sum_{n=1}^{N\left(\right)}f\left(T_{n}\right)\right]=\int_{\left(0,t\right]}f\left(s\right)d\eta\left(s\right)\textrm{,  }t\geq0,
\end{eqnarray*}
suponiendo que la integral exista. Adem\'as si $X_{1},X_{2},\ldots$ son variables aleatorias definidas en el mismo espacio de probabilidad que el proceso $N\left(t\right)$ tal que $\esp\left[X_{n}|T_{n}=s\right]=f\left(s\right)$, independiente de $n$. Entonces
\begin{eqnarray*}
\esp\left[\sum_{n=1}^{N\left(t\right)}X_{n}\right]=\int_{\left(0,t\right]}f\left(s\right)d\eta\left(s\right)\textrm{,  }t\geq0,
\end{eqnarray*} 
suponiendo que la integral exista. 
\end{Teo}

\begin{Coro}[Identidad de Wald para Renovaciones]
Para el proceso de renovaci\'on $N\left(t\right)$,
\begin{eqnarray*}
\esp\left[T_{N\left(t\right)+1}\right]=\mu\esp\left[N\left(t\right)+1\right]\textrm{,  }t\geq0,
\end{eqnarray*}  
\end{Coro}


\begin{Def}
Sea $h\left(t\right)$ funci\'on de valores reales en $\rea$ acotada en intervalos finitos e igual a cero para $t<0$ La ecuaci\'on de renovaci\'on para $h\left(t\right)$ y la distribuci\'on $F$ es

\begin{eqnarray}\label{Ec.Renovacion}
H\left(t\right)=h\left(t\right)+\int_{\left[0,t\right]}H\left(t-s\right)dF\left(s\right)\textrm{,    }t\geq0,
\end{eqnarray}
donde $H\left(t\right)$ es una funci\'on de valores reales. Esto es $H=h+F\star H$. Decimos que $H\left(t\right)$ es soluci\'on de esta ecuaci\'on si satisface la ecuaci\'on, y es acotada en intervalos finitos e iguales a cero para $t<0$.
\end{Def}

\begin{Prop}
La funci\'on $U\star h\left(t\right)$ es la \'unica soluci\'on de la ecuaci\'on de renovaci\'on (\ref{Ec.Renovacion}).
\end{Prop}

\begin{Teo}[Teorema Renovaci\'on Elemental]
\begin{eqnarray*}
t^{-1}U\left(t\right)\rightarrow 1/\mu\textrm{,    cuando }t\rightarrow\infty.
\end{eqnarray*}
\end{Teo}



Sup\'ongase que $N\left(t\right)$ es un proceso de renovaci\'on con distribuci\'on $F$ con media finita $\mu$.

\begin{Def}
La funci\'on de renovaci\'on asociada con la distribuci\'on $F$, del proceso $N\left(t\right)$, es
\begin{eqnarray*}
U\left(t\right)=\sum_{n=1}^{\infty}F^{n\star}\left(t\right),\textrm{   }t\geq0,
\end{eqnarray*}
donde $F^{0\star}\left(t\right)=\indora\left(t\geq0\right)$.
\end{Def}


\begin{Prop}
Sup\'ongase que la distribuci\'on de inter-renovaci\'on $F$ tiene densidad $f$. Entonces $U\left(t\right)$ tambi\'en tiene densidad, para $t>0$, y es $U^{'}\left(t\right)=\sum_{n=0}^{\infty}f^{n\star}\left(t\right)$. Adem\'as
\begin{eqnarray*}
\prob\left\{N\left(t\right)>N\left(t-\right)\right\}=0\textrm{,   }t\geq0.
\end{eqnarray*}
\end{Prop}

\begin{Def}
La Transformada de Laplace-Stieljes de $F$ est\'a dada por

\begin{eqnarray*}
\hat{F}\left(\alpha\right)=\int_{\rea_{+}}e^{-\alpha t}dF\left(t\right)\textrm{,  }\alpha\geq0.
\end{eqnarray*}
\end{Def}

Entonces

\begin{eqnarray*}
\hat{U}\left(\alpha\right)=\sum_{n=0}^{\infty}\hat{F^{n\star}}\left(\alpha\right)=\sum_{n=0}^{\infty}\hat{F}\left(\alpha\right)^{n}=\frac{1}{1-\hat{F}\left(\alpha\right)}.
\end{eqnarray*}


\begin{Prop}
La Transformada de Laplace $\hat{U}\left(\alpha\right)$ y $\hat{F}\left(\alpha\right)$ determina una a la otra de manera \'unica por la relaci\'on $\hat{U}\left(\alpha\right)=\frac{1}{1-\hat{F}\left(\alpha\right)}$.
\end{Prop}


\begin{Note}
Un proceso de renovaci\'on $N\left(t\right)$ cuyos tiempos de inter-renovaci\'on tienen media finita, es un proceso Poisson con tasa $\lambda$ si y s\'olo s\'i $\esp\left[U\left(t\right)\right]=\lambda t$, para $t\geq0$.
\end{Note}


\begin{Teo}
Sea $N\left(t\right)$ un proceso puntual simple con puntos de localizaci\'on $T_{n}$ tal que $\eta\left(t\right)=\esp\left[N\left(\right)\right]$ es finita para cada $t$. Entonces para cualquier funci\'on $f:\rea_{+}\rightarrow\rea$,
\begin{eqnarray*}
\esp\left[\sum_{n=1}^{N\left(\right)}f\left(T_{n}\right)\right]=\int_{\left(0,t\right]}f\left(s\right)d\eta\left(s\right)\textrm{,  }t\geq0,
\end{eqnarray*}
suponiendo que la integral exista. Adem\'as si $X_{1},X_{2},\ldots$ son variables aleatorias definidas en el mismo espacio de probabilidad que el proceso $N\left(t\right)$ tal que $\esp\left[X_{n}|T_{n}=s\right]=f\left(s\right)$, independiente de $n$. Entonces
\begin{eqnarray*}
\esp\left[\sum_{n=1}^{N\left(t\right)}X_{n}\right]=\int_{\left(0,t\right]}f\left(s\right)d\eta\left(s\right)\textrm{,  }t\geq0,
\end{eqnarray*} 
suponiendo que la integral exista. 
\end{Teo}

\begin{Coro}[Identidad de Wald para Renovaciones]
Para el proceso de renovaci\'on $N\left(t\right)$,
\begin{eqnarray*}
\esp\left[T_{N\left(t\right)+1}\right]=\mu\esp\left[N\left(t\right)+1\right]\textrm{,  }t\geq0,
\end{eqnarray*}  
\end{Coro}


\begin{Def}
Sea $h\left(t\right)$ funci\'on de valores reales en $\rea$ acotada en intervalos finitos e igual a cero para $t<0$ La ecuaci\'on de renovaci\'on para $h\left(t\right)$ y la distribuci\'on $F$ es

\begin{eqnarray}\label{Ec.Renovacion}
H\left(t\right)=h\left(t\right)+\int_{\left[0,t\right]}H\left(t-s\right)dF\left(s\right)\textrm{,    }t\geq0,
\end{eqnarray}
donde $H\left(t\right)$ es una funci\'on de valores reales. Esto es $H=h+F\star H$. Decimos que $H\left(t\right)$ es soluci\'on de esta ecuaci\'on si satisface la ecuaci\'on, y es acotada en intervalos finitos e iguales a cero para $t<0$.
\end{Def}

\begin{Prop}
La funci\'on $U\star h\left(t\right)$ es la \'unica soluci\'on de la ecuaci\'on de renovaci\'on (\ref{Ec.Renovacion}).
\end{Prop}

\begin{Teo}[Teorema Renovaci\'on Elemental]
\begin{eqnarray*}
t^{-1}U\left(t\right)\rightarrow 1/\mu\textrm{,    cuando }t\rightarrow\infty.
\end{eqnarray*}
\end{Teo}


\begin{Note} Una funci\'on $h:\rea_{+}\rightarrow\rea$ es Directamente Riemann Integrable en los siguientes casos:
\begin{itemize}
\item[a)] $h\left(t\right)\geq0$ es decreciente y Riemann Integrable.
\item[b)] $h$ es continua excepto posiblemente en un conjunto de Lebesgue de medida 0, y $|h\left(t\right)|\leq b\left(t\right)$, donde $b$ es DRI.
\end{itemize}
\end{Note}

\begin{Teo}[Teorema Principal de Renovaci\'on]
Si $F$ es no aritm\'etica y $h\left(t\right)$ es Directamente Riemann Integrable (DRI), entonces

\begin{eqnarray*}
lim_{t\rightarrow\infty}U\star h=\frac{1}{\mu}\int_{\rea_{+}}h\left(s\right)ds.
\end{eqnarray*}
\end{Teo}

\begin{Prop}
Cualquier funci\'on $H\left(t\right)$ acotada en intervalos finitos y que es 0 para $t<0$ puede expresarse como
\begin{eqnarray*}
H\left(t\right)=U\star h\left(t\right)\textrm{,  donde }h\left(t\right)=H\left(t\right)-F\star H\left(t\right)
\end{eqnarray*}
\end{Prop}

\begin{Def}
Un proceso estoc\'astico $X\left(t\right)$ es crudamente regenerativo en un tiempo aleatorio positivo $T$ si
\begin{eqnarray*}
\esp\left[X\left(T+t\right)|T\right]=\esp\left[X\left(t\right)\right]\textrm{, para }t\geq0,\end{eqnarray*}
y con las esperanzas anteriores finitas.
\end{Def}

\begin{Prop}
Sup\'ongase que $X\left(t\right)$ es un proceso crudamente regenerativo en $T$, que tiene distribuci\'on $F$. Si $\esp\left[X\left(t\right)\right]$ es acotado en intervalos finitos, entonces
\begin{eqnarray*}
\esp\left[X\left(t\right)\right]=U\star h\left(t\right)\textrm{,  donde }h\left(t\right)=\esp\left[X\left(t\right)\indora\left(T>t\right)\right].
\end{eqnarray*}
\end{Prop}

\begin{Teo}[Regeneraci\'on Cruda]
Sup\'ongase que $X\left(t\right)$ es un proceso con valores positivo crudamente regenerativo en $T$, y def\'inase $M=\sup\left\{|X\left(t\right)|:t\leq T\right\}$. Si $T$ es no aritm\'etico y $M$ y $MT$ tienen media finita, entonces
\begin{eqnarray*}
lim_{t\rightarrow\infty}\esp\left[X\left(t\right)\right]=\frac{1}{\mu}\int_{\rea_{+}}h\left(s\right)ds,
\end{eqnarray*}
donde $h\left(t\right)=\esp\left[X\left(t\right)\indora\left(T>t\right)\right]$.
\end{Teo}


\begin{Note} Una funci\'on $h:\rea_{+}\rightarrow\rea$ es Directamente Riemann Integrable en los siguientes casos:
\begin{itemize}
\item[a)] $h\left(t\right)\geq0$ es decreciente y Riemann Integrable.
\item[b)] $h$ es continua excepto posiblemente en un conjunto de Lebesgue de medida 0, y $|h\left(t\right)|\leq b\left(t\right)$, donde $b$ es DRI.
\end{itemize}
\end{Note}

\begin{Teo}[Teorema Principal de Renovaci\'on]
Si $F$ es no aritm\'etica y $h\left(t\right)$ es Directamente Riemann Integrable (DRI), entonces

\begin{eqnarray*}
lim_{t\rightarrow\infty}U\star h=\frac{1}{\mu}\int_{\rea_{+}}h\left(s\right)ds.
\end{eqnarray*}
\end{Teo}

\begin{Prop}
Cualquier funci\'on $H\left(t\right)$ acotada en intervalos finitos y que es 0 para $t<0$ puede expresarse como
\begin{eqnarray*}
H\left(t\right)=U\star h\left(t\right)\textrm{,  donde }h\left(t\right)=H\left(t\right)-F\star H\left(t\right)
\end{eqnarray*}
\end{Prop}

\begin{Def}
Un proceso estoc\'astico $X\left(t\right)$ es crudamente regenerativo en un tiempo aleatorio positivo $T$ si
\begin{eqnarray*}
\esp\left[X\left(T+t\right)|T\right]=\esp\left[X\left(t\right)\right]\textrm{, para }t\geq0,\end{eqnarray*}
y con las esperanzas anteriores finitas.
\end{Def}

\begin{Prop}
Sup\'ongase que $X\left(t\right)$ es un proceso crudamente regenerativo en $T$, que tiene distribuci\'on $F$. Si $\esp\left[X\left(t\right)\right]$ es acotado en intervalos finitos, entonces
\begin{eqnarray*}
\esp\left[X\left(t\right)\right]=U\star h\left(t\right)\textrm{,  donde }h\left(t\right)=\esp\left[X\left(t\right)\indora\left(T>t\right)\right].
\end{eqnarray*}
\end{Prop}

\begin{Teo}[Regeneraci\'on Cruda]
Sup\'ongase que $X\left(t\right)$ es un proceso con valores positivo crudamente regenerativo en $T$, y def\'inase $M=\sup\left\{|X\left(t\right)|:t\leq T\right\}$. Si $T$ es no aritm\'etico y $M$ y $MT$ tienen media finita, entonces
\begin{eqnarray*}
lim_{t\rightarrow\infty}\esp\left[X\left(t\right)\right]=\frac{1}{\mu}\int_{\rea_{+}}h\left(s\right)ds,
\end{eqnarray*}
donde $h\left(t\right)=\esp\left[X\left(t\right)\indora\left(T>t\right)\right]$.
\end{Teo}

%________________________________________________________________________
\subsection{Procesos Regenerativos}
%________________________________________________________________________

Para $\left\{X\left(t\right):t\geq0\right\}$ Proceso Estoc\'astico a tiempo continuo con estado de espacios $S$, que es un espacio m\'etrico, con trayectorias continuas por la derecha y con l\'imites por la izquierda c.s. Sea $N\left(t\right)$ un proceso de renovaci\'on en $\rea_{+}$ definido en el mismo espacio de probabilidad que $X\left(t\right)$, con tiempos de renovaci\'on $T$ y tiempos de inter-renovaci\'on $\xi_{n}=T_{n}-T_{n-1}$, con misma distribuci\'on $F$ de media finita $\mu$.



\begin{Def}
Para el proceso $\left\{\left(N\left(t\right),X\left(t\right)\right):t\geq0\right\}$, sus trayectoria muestrales en el intervalo de tiempo $\left[T_{n-1},T_{n}\right)$ est\'an descritas por
\begin{eqnarray*}
\zeta_{n}=\left(\xi_{n},\left\{X\left(T_{n-1}+t\right):0\leq t<\xi_{n}\right\}\right)
\end{eqnarray*}
Este $\zeta_{n}$ es el $n$-\'esimo segmento del proceso. El proceso es regenerativo sobre los tiempos $T_{n}$ si sus segmentos $\zeta_{n}$ son independientes e id\'enticamennte distribuidos.
\end{Def}


\begin{Obs}
Si $\tilde{X}\left(t\right)$ con espacio de estados $\tilde{S}$ es regenerativo sobre $T_{n}$, entonces $X\left(t\right)=f\left(\tilde{X}\left(t\right)\right)$ tambi\'en es regenerativo sobre $T_{n}$, para cualquier funci\'on $f:\tilde{S}\rightarrow S$.
\end{Obs}

\begin{Obs}
Los procesos regenerativos son crudamente regenerativos, pero no al rev\'es.
\end{Obs}

\begin{Def}[Definici\'on Cl\'asica]
Un proceso estoc\'astico $X=\left\{X\left(t\right):t\geq0\right\}$ es llamado regenerativo is existe una variable aleatoria $R_{1}>0$ tal que
\begin{itemize}
\item[i)] $\left\{X\left(t+R_{1}\right):t\geq0\right\}$ es independiente de $\left\{\left\{X\left(t\right):t<R_{1}\right\},\right\}$
\item[ii)] $\left\{X\left(t+R_{1}\right):t\geq0\right\}$ es estoc\'asticamente equivalente a $\left\{X\left(t\right):t>0\right\}$
\end{itemize}

Llamamos a $R_{1}$ tiempo de regeneraci\'on, y decimos que $X$ se regenera en este punto.
\end{Def}

$\left\{X\left(t+R_{1}\right)\right\}$ es regenerativo con tiempo de regeneraci\'on $R_{2}$, independiente de $R_{1}$ pero con la misma distribuci\'on que $R_{1}$. Procediendo de esta manera se obtiene una secuencia de variables aleatorias independientes e id\'enticamente distribuidas $\left\{R_{n}\right\}$ llamados longitudes de ciclo. Si definimos a $Z_{k}\equiv R_{1}+R_{2}+\cdots+R_{k}$, se tiene un proceso de renovaci\'on llamado proceso de renovaci\'on encajado para $X$.

\begin{Note}
Un proceso regenerativo con media de la longitud de ciclo finita es llamado positivo recurrente.
\end{Note}


\begin{Def}
Para $x$ fijo y para cada $t\geq0$, sea $I_{x}\left(t\right)=1$ si $X\left(t\right)\leq x$,  $I_{x}\left(t\right)=0$ en caso contrario, y def\'inanse los tiempos promedio
\begin{eqnarray*}
\overline{X}&=&lim_{t\rightarrow\infty}\frac{1}{t}\int_{0}^{\infty}X\left(u\right)du\\
\prob\left(X_{\infty}\leq x\right)&=&lim_{t\rightarrow\infty}\frac{1}{t}\int_{0}^{\infty}I_{x}\left(u\right)du,
\end{eqnarray*}
cuando estos l\'imites existan.
\end{Def}

Como consecuencia del teorema de Renovaci\'on-Recompensa, se tiene que el primer l\'imite  existe y es igual a la constante
\begin{eqnarray*}
\overline{X}&=&\frac{\esp\left[\int_{0}^{R_{1}}X\left(t\right)dt\right]}{\esp\left[R_{1}\right]},
\end{eqnarray*}
suponiendo que ambas esperanzas son finitas.

\begin{Note}
\begin{itemize}
\item[a)] Si el proceso regenerativo $X$ es positivo recurrente y tiene trayectorias muestrales no negativas, entonces la ecuaci\'on anterior es v\'alida.
\item[b)] Si $X$ es positivo recurrente regenerativo, podemos construir una \'unica versi\'on estacionaria de este proceso, $X_{e}=\left\{X_{e}\left(t\right)\right\}$, donde $X_{e}$ es un proceso estoc\'astico regenerativo y estrictamente estacionario, con distribuci\'on marginal distribuida como $X_{\infty}$
\end{itemize}
\end{Note}

%________________________________________________________________________
\subsection{Procesos Regenerativos}
%________________________________________________________________________

Para $\left\{X\left(t\right):t\geq0\right\}$ Proceso Estoc\'astico a tiempo continuo con estado de espacios $S$, que es un espacio m\'etrico, con trayectorias continuas por la derecha y con l\'imites por la izquierda c.s. Sea $N\left(t\right)$ un proceso de renovaci\'on en $\rea_{+}$ definido en el mismo espacio de probabilidad que $X\left(t\right)$, con tiempos de renovaci\'on $T$ y tiempos de inter-renovaci\'on $\xi_{n}=T_{n}-T_{n-1}$, con misma distribuci\'on $F$ de media finita $\mu$.



\begin{Def}
Para el proceso $\left\{\left(N\left(t\right),X\left(t\right)\right):t\geq0\right\}$, sus trayectoria muestrales en el intervalo de tiempo $\left[T_{n-1},T_{n}\right)$ est\'an descritas por
\begin{eqnarray*}
\zeta_{n}=\left(\xi_{n},\left\{X\left(T_{n-1}+t\right):0\leq t<\xi_{n}\right\}\right)
\end{eqnarray*}
Este $\zeta_{n}$ es el $n$-\'esimo segmento del proceso. El proceso es regenerativo sobre los tiempos $T_{n}$ si sus segmentos $\zeta_{n}$ son independientes e id\'enticamennte distribuidos.
\end{Def}


\begin{Obs}
Si $\tilde{X}\left(t\right)$ con espacio de estados $\tilde{S}$ es regenerativo sobre $T_{n}$, entonces $X\left(t\right)=f\left(\tilde{X}\left(t\right)\right)$ tambi\'en es regenerativo sobre $T_{n}$, para cualquier funci\'on $f:\tilde{S}\rightarrow S$.
\end{Obs}

\begin{Obs}
Los procesos regenerativos son crudamente regenerativos, pero no al rev\'es.
\end{Obs}

\begin{Def}[Definici\'on Cl\'asica]
Un proceso estoc\'astico $X=\left\{X\left(t\right):t\geq0\right\}$ es llamado regenerativo is existe una variable aleatoria $R_{1}>0$ tal que
\begin{itemize}
\item[i)] $\left\{X\left(t+R_{1}\right):t\geq0\right\}$ es independiente de $\left\{\left\{X\left(t\right):t<R_{1}\right\},\right\}$
\item[ii)] $\left\{X\left(t+R_{1}\right):t\geq0\right\}$ es estoc\'asticamente equivalente a $\left\{X\left(t\right):t>0\right\}$
\end{itemize}

Llamamos a $R_{1}$ tiempo de regeneraci\'on, y decimos que $X$ se regenera en este punto.
\end{Def}

$\left\{X\left(t+R_{1}\right)\right\}$ es regenerativo con tiempo de regeneraci\'on $R_{2}$, independiente de $R_{1}$ pero con la misma distribuci\'on que $R_{1}$. Procediendo de esta manera se obtiene una secuencia de variables aleatorias independientes e id\'enticamente distribuidas $\left\{R_{n}\right\}$ llamados longitudes de ciclo. Si definimos a $Z_{k}\equiv R_{1}+R_{2}+\cdots+R_{k}$, se tiene un proceso de renovaci\'on llamado proceso de renovaci\'on encajado para $X$.

\begin{Note}
Un proceso regenerativo con media de la longitud de ciclo finita es llamado positivo recurrente.
\end{Note}


\begin{Def}
Para $x$ fijo y para cada $t\geq0$, sea $I_{x}\left(t\right)=1$ si $X\left(t\right)\leq x$,  $I_{x}\left(t\right)=0$ en caso contrario, y def\'inanse los tiempos promedio
\begin{eqnarray*}
\overline{X}&=&lim_{t\rightarrow\infty}\frac{1}{t}\int_{0}^{\infty}X\left(u\right)du\\
\prob\left(X_{\infty}\leq x\right)&=&lim_{t\rightarrow\infty}\frac{1}{t}\int_{0}^{\infty}I_{x}\left(u\right)du,
\end{eqnarray*}
cuando estos l\'imites existan.
\end{Def}

Como consecuencia del teorema de Renovaci\'on-Recompensa, se tiene que el primer l\'imite  existe y es igual a la constante
\begin{eqnarray*}
\overline{X}&=&\frac{\esp\left[\int_{0}^{R_{1}}X\left(t\right)dt\right]}{\esp\left[R_{1}\right]},
\end{eqnarray*}
suponiendo que ambas esperanzas son finitas.

\begin{Note}
\begin{itemize}
\item[a)] Si el proceso regenerativo $X$ es positivo recurrente y tiene trayectorias muestrales no negativas, entonces la ecuaci\'on anterior es v\'alida.
\item[b)] Si $X$ es positivo recurrente regenerativo, podemos construir una \'unica versi\'on estacionaria de este proceso, $X_{e}=\left\{X_{e}\left(t\right)\right\}$, donde $X_{e}$ es un proceso estoc\'astico regenerativo y estrictamente estacionario, con distribuci\'on marginal distribuida como $X_{\infty}$
\end{itemize}
\end{Note}
%__________________________________________________________________________________________
\subsection{Procesos Regenerativos Estacionarios - Stidham \cite{Stidham}}
%__________________________________________________________________________________________


Un proceso estoc\'astico a tiempo continuo $\left\{V\left(t\right),t\geq0\right\}$ es un proceso regenerativo si existe una sucesi\'on de variables aleatorias independientes e id\'enticamente distribuidas $\left\{X_{1},X_{2},\ldots\right\}$, sucesi\'on de renovaci\'on, tal que para cualquier conjunto de Borel $A$, 

\begin{eqnarray*}
\prob\left\{V\left(t\right)\in A|X_{1}+X_{2}+\cdots+X_{R\left(t\right)}=s,\left\{V\left(\tau\right),\tau<s\right\}\right\}=\prob\left\{V\left(t-s\right)\in A|X_{1}>t-s\right\},
\end{eqnarray*}
para todo $0\leq s\leq t$, donde $R\left(t\right)=\max\left\{X_{1}+X_{2}+\cdots+X_{j}\leq t\right\}=$n\'umero de renovaciones ({\emph{puntos de regeneraci\'on}}) que ocurren en $\left[0,t\right]$. El intervalo $\left[0,X_{1}\right)$ es llamado {\emph{primer ciclo de regeneraci\'on}} de $\left\{V\left(t \right),t\geq0\right\}$, $\left[X_{1},X_{1}+X_{2}\right)$ el {\emph{segundo ciclo de regeneraci\'on}}, y as\'i sucesivamente.

Sea $X=X_{1}$ y sea $F$ la funci\'on de distrbuci\'on de $X$


\begin{Def}
Se define el proceso estacionario, $\left\{V^{*}\left(t\right),t\geq0\right\}$, para $\left\{V\left(t\right),t\geq0\right\}$ por

\begin{eqnarray*}
\prob\left\{V\left(t\right)\in A\right\}=\frac{1}{\esp\left[X\right]}\int_{0}^{\infty}\prob\left\{V\left(t+x\right)\in A|X>x\right\}\left(1-F\left(x\right)\right)dx,
\end{eqnarray*} 
para todo $t\geq0$ y todo conjunto de Borel $A$.
\end{Def}

\begin{Def}
Una distribuci\'on se dice que es {\emph{aritm\'etica}} si todos sus puntos de incremento son m\'ultiplos de la forma $0,\lambda, 2\lambda,\ldots$ para alguna $\lambda>0$ entera.
\end{Def}


\begin{Def}
Una modificaci\'on medible de un proceso $\left\{V\left(t\right),t\geq0\right\}$, es una versi\'on de este, $\left\{V\left(t,w\right)\right\}$ conjuntamente medible para $t\geq0$ y para $w\in S$, $S$ espacio de estados para $\left\{V\left(t\right),t\geq0\right\}$.
\end{Def}

\begin{Teo}
Sea $\left\{V\left(t\right),t\geq\right\}$ un proceso regenerativo no negativo con modificaci\'on medible. Sea $\esp\left[X\right]<\infty$. Entonces el proceso estacionario dado por la ecuaci\'on anterior est\'a bien definido y tiene funci\'on de distribuci\'on independiente de $t$, adem\'as
\begin{itemize}
\item[i)] \begin{eqnarray*}
\esp\left[V^{*}\left(0\right)\right]&=&\frac{\esp\left[\int_{0}^{X}V\left(s\right)ds\right]}{\esp\left[X\right]}\end{eqnarray*}
\item[ii)] Si $\esp\left[V^{*}\left(0\right)\right]<\infty$, equivalentemente, si $\esp\left[\int_{0}^{X}V\left(s\right)ds\right]<\infty$,entonces
\begin{eqnarray*}
\frac{\int_{0}^{t}V\left(s\right)ds}{t}\rightarrow\frac{\esp\left[\int_{0}^{X}V\left(s\right)ds\right]}{\esp\left[X\right]}
\end{eqnarray*}
con probabilidad 1 y en media, cuando $t\rightarrow\infty$.
\end{itemize}
\end{Teo}


%__________________________________________________________________________________________
\subsection{Procesos Regenerativos Estacionarios - Stidham \cite{Stidham}}
%__________________________________________________________________________________________


Un proceso estoc\'astico a tiempo continuo $\left\{V\left(t\right),t\geq0\right\}$ es un proceso regenerativo si existe una sucesi\'on de variables aleatorias independientes e id\'enticamente distribuidas $\left\{X_{1},X_{2},\ldots\right\}$, sucesi\'on de renovaci\'on, tal que para cualquier conjunto de Borel $A$, 

\begin{eqnarray*}
\prob\left\{V\left(t\right)\in A|X_{1}+X_{2}+\cdots+X_{R\left(t\right)}=s,\left\{V\left(\tau\right),\tau<s\right\}\right\}=\prob\left\{V\left(t-s\right)\in A|X_{1}>t-s\right\},
\end{eqnarray*}
para todo $0\leq s\leq t$, donde $R\left(t\right)=\max\left\{X_{1}+X_{2}+\cdots+X_{j}\leq t\right\}=$n\'umero de renovaciones ({\emph{puntos de regeneraci\'on}}) que ocurren en $\left[0,t\right]$. El intervalo $\left[0,X_{1}\right)$ es llamado {\emph{primer ciclo de regeneraci\'on}} de $\left\{V\left(t \right),t\geq0\right\}$, $\left[X_{1},X_{1}+X_{2}\right)$ el {\emph{segundo ciclo de regeneraci\'on}}, y as\'i sucesivamente.

Sea $X=X_{1}$ y sea $F$ la funci\'on de distrbuci\'on de $X$


\begin{Def}
Se define el proceso estacionario, $\left\{V^{*}\left(t\right),t\geq0\right\}$, para $\left\{V\left(t\right),t\geq0\right\}$ por

\begin{eqnarray*}
\prob\left\{V\left(t\right)\in A\right\}=\frac{1}{\esp\left[X\right]}\int_{0}^{\infty}\prob\left\{V\left(t+x\right)\in A|X>x\right\}\left(1-F\left(x\right)\right)dx,
\end{eqnarray*} 
para todo $t\geq0$ y todo conjunto de Borel $A$.
\end{Def}

\begin{Def}
Una distribuci\'on se dice que es {\emph{aritm\'etica}} si todos sus puntos de incremento son m\'ultiplos de la forma $0,\lambda, 2\lambda,\ldots$ para alguna $\lambda>0$ entera.
\end{Def}


\begin{Def}
Una modificaci\'on medible de un proceso $\left\{V\left(t\right),t\geq0\right\}$, es una versi\'on de este, $\left\{V\left(t,w\right)\right\}$ conjuntamente medible para $t\geq0$ y para $w\in S$, $S$ espacio de estados para $\left\{V\left(t\right),t\geq0\right\}$.
\end{Def}

\begin{Teo}
Sea $\left\{V\left(t\right),t\geq\right\}$ un proceso regenerativo no negativo con modificaci\'on medible. Sea $\esp\left[X\right]<\infty$. Entonces el proceso estacionario dado por la ecuaci\'on anterior est\'a bien definido y tiene funci\'on de distribuci\'on independiente de $t$, adem\'as
\begin{itemize}
\item[i)] \begin{eqnarray*}
\esp\left[V^{*}\left(0\right)\right]&=&\frac{\esp\left[\int_{0}^{X}V\left(s\right)ds\right]}{\esp\left[X\right]}\end{eqnarray*}
\item[ii)] Si $\esp\left[V^{*}\left(0\right)\right]<\infty$, equivalentemente, si $\esp\left[\int_{0}^{X}V\left(s\right)ds\right]<\infty$,entonces
\begin{eqnarray*}
\frac{\int_{0}^{t}V\left(s\right)ds}{t}\rightarrow\frac{\esp\left[\int_{0}^{X}V\left(s\right)ds\right]}{\esp\left[X\right]}
\end{eqnarray*}
con probabilidad 1 y en media, cuando $t\rightarrow\infty$.
\end{itemize}
\end{Teo}
%___________________________________________________________________________________________
%
\subsection{Propiedades de los Procesos de Renovaci\'on}
%___________________________________________________________________________________________
%

Los tiempos $T_{n}$ est\'an relacionados con los conteos de $N\left(t\right)$ por

\begin{eqnarray*}
\left\{N\left(t\right)\geq n\right\}&=&\left\{T_{n}\leq t\right\}\\
T_{N\left(t\right)}\leq &t&<T_{N\left(t\right)+1},
\end{eqnarray*}

adem\'as $N\left(T_{n}\right)=n$, y 

\begin{eqnarray*}
N\left(t\right)=\max\left\{n:T_{n}\leq t\right\}=\min\left\{n:T_{n+1}>t\right\}
\end{eqnarray*}

Por propiedades de la convoluci\'on se sabe que

\begin{eqnarray*}
P\left\{T_{n}\leq t\right\}=F^{n\star}\left(t\right)
\end{eqnarray*}
que es la $n$-\'esima convoluci\'on de $F$. Entonces 

\begin{eqnarray*}
\left\{N\left(t\right)\geq n\right\}&=&\left\{T_{n}\leq t\right\}\\
P\left\{N\left(t\right)\leq n\right\}&=&1-F^{\left(n+1\right)\star}\left(t\right)
\end{eqnarray*}

Adem\'as usando el hecho de que $\esp\left[N\left(t\right)\right]=\sum_{n=1}^{\infty}P\left\{N\left(t\right)\geq n\right\}$
se tiene que

\begin{eqnarray*}
\esp\left[N\left(t\right)\right]=\sum_{n=1}^{\infty}F^{n\star}\left(t\right)
\end{eqnarray*}

\begin{Prop}
Para cada $t\geq0$, la funci\'on generadora de momentos $\esp\left[e^{\alpha N\left(t\right)}\right]$ existe para alguna $\alpha$ en una vecindad del 0, y de aqu\'i que $\esp\left[N\left(t\right)^{m}\right]<\infty$, para $m\geq1$.
\end{Prop}


\begin{Note}
Si el primer tiempo de renovaci\'on $\xi_{1}$ no tiene la misma distribuci\'on que el resto de las $\xi_{n}$, para $n\geq2$, a $N\left(t\right)$ se le llama Proceso de Renovaci\'on retardado, donde si $\xi$ tiene distribuci\'on $G$, entonces el tiempo $T_{n}$ de la $n$-\'esima renovaci\'on tiene distribuci\'on $G\star F^{\left(n-1\right)\star}\left(t\right)$
\end{Note}


\begin{Teo}
Para una constante $\mu\leq\infty$ ( o variable aleatoria), las siguientes expresiones son equivalentes:

\begin{eqnarray}
lim_{n\rightarrow\infty}n^{-1}T_{n}&=&\mu,\textrm{ c.s.}\\
lim_{t\rightarrow\infty}t^{-1}N\left(t\right)&=&1/\mu,\textrm{ c.s.}
\end{eqnarray}
\end{Teo}


Es decir, $T_{n}$ satisface la Ley Fuerte de los Grandes N\'umeros s\'i y s\'olo s\'i $N\left/t\right)$ la cumple.


\begin{Coro}[Ley Fuerte de los Grandes N\'umeros para Procesos de Renovaci\'on]
Si $N\left(t\right)$ es un proceso de renovaci\'on cuyos tiempos de inter-renovaci\'on tienen media $\mu\leq\infty$, entonces
\begin{eqnarray}
t^{-1}N\left(t\right)\rightarrow 1/\mu,\textrm{ c.s. cuando }t\rightarrow\infty.
\end{eqnarray}

\end{Coro}


Considerar el proceso estoc\'astico de valores reales $\left\{Z\left(t\right):t\geq0\right\}$ en el mismo espacio de probabilidad que $N\left(t\right)$

\begin{Def}
Para el proceso $\left\{Z\left(t\right):t\geq0\right\}$ se define la fluctuaci\'on m\'axima de $Z\left(t\right)$ en el intervalo $\left(T_{n-1},T_{n}\right]$:
\begin{eqnarray*}
M_{n}=\sup_{T_{n-1}<t\leq T_{n}}|Z\left(t\right)-Z\left(T_{n-1}\right)|
\end{eqnarray*}
\end{Def}

\begin{Teo}
Sup\'ongase que $n^{-1}T_{n}\rightarrow\mu$ c.s. cuando $n\rightarrow\infty$, donde $\mu\leq\infty$ es una constante o variable aleatoria. Sea $a$ una constante o variable aleatoria que puede ser infinita cuando $\mu$ es finita, y considere las expresiones l\'imite:
\begin{eqnarray}
lim_{n\rightarrow\infty}n^{-1}Z\left(T_{n}\right)&=&a,\textrm{ c.s.}\\
lim_{t\rightarrow\infty}t^{-1}Z\left(t\right)&=&a/\mu,\textrm{ c.s.}
\end{eqnarray}
La segunda expresi\'on implica la primera. Conversamente, la primera implica la segunda si el proceso $Z\left(t\right)$ es creciente, o si $lim_{n\rightarrow\infty}n^{-1}M_{n}=0$ c.s.
\end{Teo}

\begin{Coro}
Si $N\left(t\right)$ es un proceso de renovaci\'on, y $\left(Z\left(T_{n}\right)-Z\left(T_{n-1}\right),M_{n}\right)$, para $n\geq1$, son variables aleatorias independientes e id\'enticamente distribuidas con media finita, entonces,
\begin{eqnarray}
lim_{t\rightarrow\infty}t^{-1}Z\left(t\right)\rightarrow\frac{\esp\left[Z\left(T_{1}\right)-Z\left(T_{0}\right)\right]}{\esp\left[T_{1}\right]},\textrm{ c.s. cuando  }t\rightarrow\infty.
\end{eqnarray}
\end{Coro}


%___________________________________________________________________________________________
%
\subsection{Propiedades de los Procesos de Renovaci\'on}
%___________________________________________________________________________________________
%

Los tiempos $T_{n}$ est\'an relacionados con los conteos de $N\left(t\right)$ por

\begin{eqnarray*}
\left\{N\left(t\right)\geq n\right\}&=&\left\{T_{n}\leq t\right\}\\
T_{N\left(t\right)}\leq &t&<T_{N\left(t\right)+1},
\end{eqnarray*}

adem\'as $N\left(T_{n}\right)=n$, y 

\begin{eqnarray*}
N\left(t\right)=\max\left\{n:T_{n}\leq t\right\}=\min\left\{n:T_{n+1}>t\right\}
\end{eqnarray*}

Por propiedades de la convoluci\'on se sabe que

\begin{eqnarray*}
P\left\{T_{n}\leq t\right\}=F^{n\star}\left(t\right)
\end{eqnarray*}
que es la $n$-\'esima convoluci\'on de $F$. Entonces 

\begin{eqnarray*}
\left\{N\left(t\right)\geq n\right\}&=&\left\{T_{n}\leq t\right\}\\
P\left\{N\left(t\right)\leq n\right\}&=&1-F^{\left(n+1\right)\star}\left(t\right)
\end{eqnarray*}

Adem\'as usando el hecho de que $\esp\left[N\left(t\right)\right]=\sum_{n=1}^{\infty}P\left\{N\left(t\right)\geq n\right\}$
se tiene que

\begin{eqnarray*}
\esp\left[N\left(t\right)\right]=\sum_{n=1}^{\infty}F^{n\star}\left(t\right)
\end{eqnarray*}

\begin{Prop}
Para cada $t\geq0$, la funci\'on generadora de momentos $\esp\left[e^{\alpha N\left(t\right)}\right]$ existe para alguna $\alpha$ en una vecindad del 0, y de aqu\'i que $\esp\left[N\left(t\right)^{m}\right]<\infty$, para $m\geq1$.
\end{Prop}


\begin{Note}
Si el primer tiempo de renovaci\'on $\xi_{1}$ no tiene la misma distribuci\'on que el resto de las $\xi_{n}$, para $n\geq2$, a $N\left(t\right)$ se le llama Proceso de Renovaci\'on retardado, donde si $\xi$ tiene distribuci\'on $G$, entonces el tiempo $T_{n}$ de la $n$-\'esima renovaci\'on tiene distribuci\'on $G\star F^{\left(n-1\right)\star}\left(t\right)$
\end{Note}


\begin{Teo}
Para una constante $\mu\leq\infty$ ( o variable aleatoria), las siguientes expresiones son equivalentes:

\begin{eqnarray}
lim_{n\rightarrow\infty}n^{-1}T_{n}&=&\mu,\textrm{ c.s.}\\
lim_{t\rightarrow\infty}t^{-1}N\left(t\right)&=&1/\mu,\textrm{ c.s.}
\end{eqnarray}
\end{Teo}


Es decir, $T_{n}$ satisface la Ley Fuerte de los Grandes N\'umeros s\'i y s\'olo s\'i $N\left/t\right)$ la cumple.


\begin{Coro}[Ley Fuerte de los Grandes N\'umeros para Procesos de Renovaci\'on]
Si $N\left(t\right)$ es un proceso de renovaci\'on cuyos tiempos de inter-renovaci\'on tienen media $\mu\leq\infty$, entonces
\begin{eqnarray}
t^{-1}N\left(t\right)\rightarrow 1/\mu,\textrm{ c.s. cuando }t\rightarrow\infty.
\end{eqnarray}

\end{Coro}


Considerar el proceso estoc\'astico de valores reales $\left\{Z\left(t\right):t\geq0\right\}$ en el mismo espacio de probabilidad que $N\left(t\right)$

\begin{Def}
Para el proceso $\left\{Z\left(t\right):t\geq0\right\}$ se define la fluctuaci\'on m\'axima de $Z\left(t\right)$ en el intervalo $\left(T_{n-1},T_{n}\right]$:
\begin{eqnarray*}
M_{n}=\sup_{T_{n-1}<t\leq T_{n}}|Z\left(t\right)-Z\left(T_{n-1}\right)|
\end{eqnarray*}
\end{Def}

\begin{Teo}
Sup\'ongase que $n^{-1}T_{n}\rightarrow\mu$ c.s. cuando $n\rightarrow\infty$, donde $\mu\leq\infty$ es una constante o variable aleatoria. Sea $a$ una constante o variable aleatoria que puede ser infinita cuando $\mu$ es finita, y considere las expresiones l\'imite:
\begin{eqnarray}
lim_{n\rightarrow\infty}n^{-1}Z\left(T_{n}\right)&=&a,\textrm{ c.s.}\\
lim_{t\rightarrow\infty}t^{-1}Z\left(t\right)&=&a/\mu,\textrm{ c.s.}
\end{eqnarray}
La segunda expresi\'on implica la primera. Conversamente, la primera implica la segunda si el proceso $Z\left(t\right)$ es creciente, o si $lim_{n\rightarrow\infty}n^{-1}M_{n}=0$ c.s.
\end{Teo}

\begin{Coro}
Si $N\left(t\right)$ es un proceso de renovaci\'on, y $\left(Z\left(T_{n}\right)-Z\left(T_{n-1}\right),M_{n}\right)$, para $n\geq1$, son variables aleatorias independientes e id\'enticamente distribuidas con media finita, entonces,
\begin{eqnarray}
lim_{t\rightarrow\infty}t^{-1}Z\left(t\right)\rightarrow\frac{\esp\left[Z\left(T_{1}\right)-Z\left(T_{0}\right)\right]}{\esp\left[T_{1}\right]},\textrm{ c.s. cuando  }t\rightarrow\infty.
\end{eqnarray}
\end{Coro}

%___________________________________________________________________________________________
%
\subsection{Propiedades de los Procesos de Renovaci\'on}
%___________________________________________________________________________________________
%

Los tiempos $T_{n}$ est\'an relacionados con los conteos de $N\left(t\right)$ por

\begin{eqnarray*}
\left\{N\left(t\right)\geq n\right\}&=&\left\{T_{n}\leq t\right\}\\
T_{N\left(t\right)}\leq &t&<T_{N\left(t\right)+1},
\end{eqnarray*}

adem\'as $N\left(T_{n}\right)=n$, y 

\begin{eqnarray*}
N\left(t\right)=\max\left\{n:T_{n}\leq t\right\}=\min\left\{n:T_{n+1}>t\right\}
\end{eqnarray*}

Por propiedades de la convoluci\'on se sabe que

\begin{eqnarray*}
P\left\{T_{n}\leq t\right\}=F^{n\star}\left(t\right)
\end{eqnarray*}
que es la $n$-\'esima convoluci\'on de $F$. Entonces 

\begin{eqnarray*}
\left\{N\left(t\right)\geq n\right\}&=&\left\{T_{n}\leq t\right\}\\
P\left\{N\left(t\right)\leq n\right\}&=&1-F^{\left(n+1\right)\star}\left(t\right)
\end{eqnarray*}

Adem\'as usando el hecho de que $\esp\left[N\left(t\right)\right]=\sum_{n=1}^{\infty}P\left\{N\left(t\right)\geq n\right\}$
se tiene que

\begin{eqnarray*}
\esp\left[N\left(t\right)\right]=\sum_{n=1}^{\infty}F^{n\star}\left(t\right)
\end{eqnarray*}

\begin{Prop}
Para cada $t\geq0$, la funci\'on generadora de momentos $\esp\left[e^{\alpha N\left(t\right)}\right]$ existe para alguna $\alpha$ en una vecindad del 0, y de aqu\'i que $\esp\left[N\left(t\right)^{m}\right]<\infty$, para $m\geq1$.
\end{Prop}


\begin{Note}
Si el primer tiempo de renovaci\'on $\xi_{1}$ no tiene la misma distribuci\'on que el resto de las $\xi_{n}$, para $n\geq2$, a $N\left(t\right)$ se le llama Proceso de Renovaci\'on retardado, donde si $\xi$ tiene distribuci\'on $G$, entonces el tiempo $T_{n}$ de la $n$-\'esima renovaci\'on tiene distribuci\'on $G\star F^{\left(n-1\right)\star}\left(t\right)$
\end{Note}


\begin{Teo}
Para una constante $\mu\leq\infty$ ( o variable aleatoria), las siguientes expresiones son equivalentes:

\begin{eqnarray}
lim_{n\rightarrow\infty}n^{-1}T_{n}&=&\mu,\textrm{ c.s.}\\
lim_{t\rightarrow\infty}t^{-1}N\left(t\right)&=&1/\mu,\textrm{ c.s.}
\end{eqnarray}
\end{Teo}


Es decir, $T_{n}$ satisface la Ley Fuerte de los Grandes N\'umeros s\'i y s\'olo s\'i $N\left/t\right)$ la cumple.


\begin{Coro}[Ley Fuerte de los Grandes N\'umeros para Procesos de Renovaci\'on]
Si $N\left(t\right)$ es un proceso de renovaci\'on cuyos tiempos de inter-renovaci\'on tienen media $\mu\leq\infty$, entonces
\begin{eqnarray}
t^{-1}N\left(t\right)\rightarrow 1/\mu,\textrm{ c.s. cuando }t\rightarrow\infty.
\end{eqnarray}

\end{Coro}


Considerar el proceso estoc\'astico de valores reales $\left\{Z\left(t\right):t\geq0\right\}$ en el mismo espacio de probabilidad que $N\left(t\right)$

\begin{Def}
Para el proceso $\left\{Z\left(t\right):t\geq0\right\}$ se define la fluctuaci\'on m\'axima de $Z\left(t\right)$ en el intervalo $\left(T_{n-1},T_{n}\right]$:
\begin{eqnarray*}
M_{n}=\sup_{T_{n-1}<t\leq T_{n}}|Z\left(t\right)-Z\left(T_{n-1}\right)|
\end{eqnarray*}
\end{Def}

\begin{Teo}
Sup\'ongase que $n^{-1}T_{n}\rightarrow\mu$ c.s. cuando $n\rightarrow\infty$, donde $\mu\leq\infty$ es una constante o variable aleatoria. Sea $a$ una constante o variable aleatoria que puede ser infinita cuando $\mu$ es finita, y considere las expresiones l\'imite:
\begin{eqnarray}
lim_{n\rightarrow\infty}n^{-1}Z\left(T_{n}\right)&=&a,\textrm{ c.s.}\\
lim_{t\rightarrow\infty}t^{-1}Z\left(t\right)&=&a/\mu,\textrm{ c.s.}
\end{eqnarray}
La segunda expresi\'on implica la primera. Conversamente, la primera implica la segunda si el proceso $Z\left(t\right)$ es creciente, o si $lim_{n\rightarrow\infty}n^{-1}M_{n}=0$ c.s.
\end{Teo}

\begin{Coro}
Si $N\left(t\right)$ es un proceso de renovaci\'on, y $\left(Z\left(T_{n}\right)-Z\left(T_{n-1}\right),M_{n}\right)$, para $n\geq1$, son variables aleatorias independientes e id\'enticamente distribuidas con media finita, entonces,
\begin{eqnarray}
lim_{t\rightarrow\infty}t^{-1}Z\left(t\right)\rightarrow\frac{\esp\left[Z\left(T_{1}\right)-Z\left(T_{0}\right)\right]}{\esp\left[T_{1}\right]},\textrm{ c.s. cuando  }t\rightarrow\infty.
\end{eqnarray}
\end{Coro}

%___________________________________________________________________________________________
%
\subsection{Propiedades de los Procesos de Renovaci\'on}
%___________________________________________________________________________________________
%

Los tiempos $T_{n}$ est\'an relacionados con los conteos de $N\left(t\right)$ por

\begin{eqnarray*}
\left\{N\left(t\right)\geq n\right\}&=&\left\{T_{n}\leq t\right\}\\
T_{N\left(t\right)}\leq &t&<T_{N\left(t\right)+1},
\end{eqnarray*}

adem\'as $N\left(T_{n}\right)=n$, y 

\begin{eqnarray*}
N\left(t\right)=\max\left\{n:T_{n}\leq t\right\}=\min\left\{n:T_{n+1}>t\right\}
\end{eqnarray*}

Por propiedades de la convoluci\'on se sabe que

\begin{eqnarray*}
P\left\{T_{n}\leq t\right\}=F^{n\star}\left(t\right)
\end{eqnarray*}
que es la $n$-\'esima convoluci\'on de $F$. Entonces 

\begin{eqnarray*}
\left\{N\left(t\right)\geq n\right\}&=&\left\{T_{n}\leq t\right\}\\
P\left\{N\left(t\right)\leq n\right\}&=&1-F^{\left(n+1\right)\star}\left(t\right)
\end{eqnarray*}

Adem\'as usando el hecho de que $\esp\left[N\left(t\right)\right]=\sum_{n=1}^{\infty}P\left\{N\left(t\right)\geq n\right\}$
se tiene que

\begin{eqnarray*}
\esp\left[N\left(t\right)\right]=\sum_{n=1}^{\infty}F^{n\star}\left(t\right)
\end{eqnarray*}

\begin{Prop}
Para cada $t\geq0$, la funci\'on generadora de momentos $\esp\left[e^{\alpha N\left(t\right)}\right]$ existe para alguna $\alpha$ en una vecindad del 0, y de aqu\'i que $\esp\left[N\left(t\right)^{m}\right]<\infty$, para $m\geq1$.
\end{Prop}


\begin{Note}
Si el primer tiempo de renovaci\'on $\xi_{1}$ no tiene la misma distribuci\'on que el resto de las $\xi_{n}$, para $n\geq2$, a $N\left(t\right)$ se le llama Proceso de Renovaci\'on retardado, donde si $\xi$ tiene distribuci\'on $G$, entonces el tiempo $T_{n}$ de la $n$-\'esima renovaci\'on tiene distribuci\'on $G\star F^{\left(n-1\right)\star}\left(t\right)$
\end{Note}


\begin{Teo}
Para una constante $\mu\leq\infty$ ( o variable aleatoria), las siguientes expresiones son equivalentes:

\begin{eqnarray}
lim_{n\rightarrow\infty}n^{-1}T_{n}&=&\mu,\textrm{ c.s.}\\
lim_{t\rightarrow\infty}t^{-1}N\left(t\right)&=&1/\mu,\textrm{ c.s.}
\end{eqnarray}
\end{Teo}


Es decir, $T_{n}$ satisface la Ley Fuerte de los Grandes N\'umeros s\'i y s\'olo s\'i $N\left/t\right)$ la cumple.


\begin{Coro}[Ley Fuerte de los Grandes N\'umeros para Procesos de Renovaci\'on]
Si $N\left(t\right)$ es un proceso de renovaci\'on cuyos tiempos de inter-renovaci\'on tienen media $\mu\leq\infty$, entonces
\begin{eqnarray}
t^{-1}N\left(t\right)\rightarrow 1/\mu,\textrm{ c.s. cuando }t\rightarrow\infty.
\end{eqnarray}

\end{Coro}


Considerar el proceso estoc\'astico de valores reales $\left\{Z\left(t\right):t\geq0\right\}$ en el mismo espacio de probabilidad que $N\left(t\right)$

\begin{Def}
Para el proceso $\left\{Z\left(t\right):t\geq0\right\}$ se define la fluctuaci\'on m\'axima de $Z\left(t\right)$ en el intervalo $\left(T_{n-1},T_{n}\right]$:
\begin{eqnarray*}
M_{n}=\sup_{T_{n-1}<t\leq T_{n}}|Z\left(t\right)-Z\left(T_{n-1}\right)|
\end{eqnarray*}
\end{Def}

\begin{Teo}
Sup\'ongase que $n^{-1}T_{n}\rightarrow\mu$ c.s. cuando $n\rightarrow\infty$, donde $\mu\leq\infty$ es una constante o variable aleatoria. Sea $a$ una constante o variable aleatoria que puede ser infinita cuando $\mu$ es finita, y considere las expresiones l\'imite:
\begin{eqnarray}
lim_{n\rightarrow\infty}n^{-1}Z\left(T_{n}\right)&=&a,\textrm{ c.s.}\\
lim_{t\rightarrow\infty}t^{-1}Z\left(t\right)&=&a/\mu,\textrm{ c.s.}
\end{eqnarray}
La segunda expresi\'on implica la primera. Conversamente, la primera implica la segunda si el proceso $Z\left(t\right)$ es creciente, o si $lim_{n\rightarrow\infty}n^{-1}M_{n}=0$ c.s.
\end{Teo}

\begin{Coro}
Si $N\left(t\right)$ es un proceso de renovaci\'on, y $\left(Z\left(T_{n}\right)-Z\left(T_{n-1}\right),M_{n}\right)$, para $n\geq1$, son variables aleatorias independientes e id\'enticamente distribuidas con media finita, entonces,
\begin{eqnarray}
lim_{t\rightarrow\infty}t^{-1}Z\left(t\right)\rightarrow\frac{\esp\left[Z\left(T_{1}\right)-Z\left(T_{0}\right)\right]}{\esp\left[T_{1}\right]},\textrm{ c.s. cuando  }t\rightarrow\infty.
\end{eqnarray}
\end{Coro}


%__________________________________________________________________________________________
\subsection{Procesos Regenerativos Estacionarios - Stidham \cite{Stidham}}
%__________________________________________________________________________________________


Un proceso estoc\'astico a tiempo continuo $\left\{V\left(t\right),t\geq0\right\}$ es un proceso regenerativo si existe una sucesi\'on de variables aleatorias independientes e id\'enticamente distribuidas $\left\{X_{1},X_{2},\ldots\right\}$, sucesi\'on de renovaci\'on, tal que para cualquier conjunto de Borel $A$, 

\begin{eqnarray*}
\prob\left\{V\left(t\right)\in A|X_{1}+X_{2}+\cdots+X_{R\left(t\right)}=s,\left\{V\left(\tau\right),\tau<s\right\}\right\}=\prob\left\{V\left(t-s\right)\in A|X_{1}>t-s\right\},
\end{eqnarray*}
para todo $0\leq s\leq t$, donde $R\left(t\right)=\max\left\{X_{1}+X_{2}+\cdots+X_{j}\leq t\right\}=$n\'umero de renovaciones ({\emph{puntos de regeneraci\'on}}) que ocurren en $\left[0,t\right]$. El intervalo $\left[0,X_{1}\right)$ es llamado {\emph{primer ciclo de regeneraci\'on}} de $\left\{V\left(t \right),t\geq0\right\}$, $\left[X_{1},X_{1}+X_{2}\right)$ el {\emph{segundo ciclo de regeneraci\'on}}, y as\'i sucesivamente.

Sea $X=X_{1}$ y sea $F$ la funci\'on de distrbuci\'on de $X$


\begin{Def}
Se define el proceso estacionario, $\left\{V^{*}\left(t\right),t\geq0\right\}$, para $\left\{V\left(t\right),t\geq0\right\}$ por

\begin{eqnarray*}
\prob\left\{V\left(t\right)\in A\right\}=\frac{1}{\esp\left[X\right]}\int_{0}^{\infty}\prob\left\{V\left(t+x\right)\in A|X>x\right\}\left(1-F\left(x\right)\right)dx,
\end{eqnarray*} 
para todo $t\geq0$ y todo conjunto de Borel $A$.
\end{Def}

\begin{Def}
Una distribuci\'on se dice que es {\emph{aritm\'etica}} si todos sus puntos de incremento son m\'ultiplos de la forma $0,\lambda, 2\lambda,\ldots$ para alguna $\lambda>0$ entera.
\end{Def}


\begin{Def}
Una modificaci\'on medible de un proceso $\left\{V\left(t\right),t\geq0\right\}$, es una versi\'on de este, $\left\{V\left(t,w\right)\right\}$ conjuntamente medible para $t\geq0$ y para $w\in S$, $S$ espacio de estados para $\left\{V\left(t\right),t\geq0\right\}$.
\end{Def}

\begin{Teo}
Sea $\left\{V\left(t\right),t\geq\right\}$ un proceso regenerativo no negativo con modificaci\'on medible. Sea $\esp\left[X\right]<\infty$. Entonces el proceso estacionario dado por la ecuaci\'on anterior est\'a bien definido y tiene funci\'on de distribuci\'on independiente de $t$, adem\'as
\begin{itemize}
\item[i)] \begin{eqnarray*}
\esp\left[V^{*}\left(0\right)\right]&=&\frac{\esp\left[\int_{0}^{X}V\left(s\right)ds\right]}{\esp\left[X\right]}\end{eqnarray*}
\item[ii)] Si $\esp\left[V^{*}\left(0\right)\right]<\infty$, equivalentemente, si $\esp\left[\int_{0}^{X}V\left(s\right)ds\right]<\infty$,entonces
\begin{eqnarray*}
\frac{\int_{0}^{t}V\left(s\right)ds}{t}\rightarrow\frac{\esp\left[\int_{0}^{X}V\left(s\right)ds\right]}{\esp\left[X\right]}
\end{eqnarray*}
con probabilidad 1 y en media, cuando $t\rightarrow\infty$.
\end{itemize}
\end{Teo}

%______________________________________________________________________
\subsection{Procesos de Renovaci\'on}
%______________________________________________________________________

\begin{Def}\label{Def.Tn}
Sean $0\leq T_{1}\leq T_{2}\leq \ldots$ son tiempos aleatorios infinitos en los cuales ocurren ciertos eventos. El n\'umero de tiempos $T_{n}$ en el intervalo $\left[0,t\right)$ es

\begin{eqnarray}
N\left(t\right)=\sum_{n=1}^{\infty}\indora\left(T_{n}\leq t\right),
\end{eqnarray}
para $t\geq0$.
\end{Def}

Si se consideran los puntos $T_{n}$ como elementos de $\rea_{+}$, y $N\left(t\right)$ es el n\'umero de puntos en $\rea$. El proceso denotado por $\left\{N\left(t\right):t\geq0\right\}$, denotado por $N\left(t\right)$, es un proceso puntual en $\rea_{+}$. Los $T_{n}$ son los tiempos de ocurrencia, el proceso puntual $N\left(t\right)$ es simple si su n\'umero de ocurrencias son distintas: $0<T_{1}<T_{2}<\ldots$ casi seguramente.

\begin{Def}
Un proceso puntual $N\left(t\right)$ es un proceso de renovaci\'on si los tiempos de interocurrencia $\xi_{n}=T_{n}-T_{n-1}$, para $n\geq1$, son independientes e identicamente distribuidos con distribuci\'on $F$, donde $F\left(0\right)=0$ y $T_{0}=0$. Los $T_{n}$ son llamados tiempos de renovaci\'on, referente a la independencia o renovaci\'on de la informaci\'on estoc\'astica en estos tiempos. Los $\xi_{n}$ son los tiempos de inter-renovaci\'on, y $N\left(t\right)$ es el n\'umero de renovaciones en el intervalo $\left[0,t\right)$
\end{Def}


\begin{Note}
Para definir un proceso de renovaci\'on para cualquier contexto, solamente hay que especificar una distribuci\'on $F$, con $F\left(0\right)=0$, para los tiempos de inter-renovaci\'on. La funci\'on $F$ en turno degune las otra variables aleatorias. De manera formal, existe un espacio de probabilidad y una sucesi\'on de variables aleatorias $\xi_{1},\xi_{2},\ldots$ definidas en este con distribuci\'on $F$. Entonces las otras cantidades son $T_{n}=\sum_{k=1}^{n}\xi_{k}$ y $N\left(t\right)=\sum_{n=1}^{\infty}\indora\left(T_{n}\leq t\right)$, donde $T_{n}\rightarrow\infty$ casi seguramente por la Ley Fuerte de los Grandes Números.
\end{Note}

%___________________________________________________________________________________________
%
\subsection{Teorema Principal de Renovaci\'on}
%___________________________________________________________________________________________
%

\begin{Note} Una funci\'on $h:\rea_{+}\rightarrow\rea$ es Directamente Riemann Integrable en los siguientes casos:
\begin{itemize}
\item[a)] $h\left(t\right)\geq0$ es decreciente y Riemann Integrable.
\item[b)] $h$ es continua excepto posiblemente en un conjunto de Lebesgue de medida 0, y $|h\left(t\right)|\leq b\left(t\right)$, donde $b$ es DRI.
\end{itemize}
\end{Note}

\begin{Teo}[Teorema Principal de Renovaci\'on]
Si $F$ es no aritm\'etica y $h\left(t\right)$ es Directamente Riemann Integrable (DRI), entonces

\begin{eqnarray*}
lim_{t\rightarrow\infty}U\star h=\frac{1}{\mu}\int_{\rea_{+}}h\left(s\right)ds.
\end{eqnarray*}
\end{Teo}

\begin{Prop}
Cualquier funci\'on $H\left(t\right)$ acotada en intervalos finitos y que es 0 para $t<0$ puede expresarse como
\begin{eqnarray*}
H\left(t\right)=U\star h\left(t\right)\textrm{,  donde }h\left(t\right)=H\left(t\right)-F\star H\left(t\right)
\end{eqnarray*}
\end{Prop}

\begin{Def}
Un proceso estoc\'astico $X\left(t\right)$ es crudamente regenerativo en un tiempo aleatorio positivo $T$ si
\begin{eqnarray*}
\esp\left[X\left(T+t\right)|T\right]=\esp\left[X\left(t\right)\right]\textrm{, para }t\geq0,\end{eqnarray*}
y con las esperanzas anteriores finitas.
\end{Def}

\begin{Prop}
Sup\'ongase que $X\left(t\right)$ es un proceso crudamente regenerativo en $T$, que tiene distribuci\'on $F$. Si $\esp\left[X\left(t\right)\right]$ es acotado en intervalos finitos, entonces
\begin{eqnarray*}
\esp\left[X\left(t\right)\right]=U\star h\left(t\right)\textrm{,  donde }h\left(t\right)=\esp\left[X\left(t\right)\indora\left(T>t\right)\right].
\end{eqnarray*}
\end{Prop}

\begin{Teo}[Regeneraci\'on Cruda]
Sup\'ongase que $X\left(t\right)$ es un proceso con valores positivo crudamente regenerativo en $T$, y def\'inase $M=\sup\left\{|X\left(t\right)|:t\leq T\right\}$. Si $T$ es no aritm\'etico y $M$ y $MT$ tienen media finita, entonces
\begin{eqnarray*}
lim_{t\rightarrow\infty}\esp\left[X\left(t\right)\right]=\frac{1}{\mu}\int_{\rea_{+}}h\left(s\right)ds,
\end{eqnarray*}
donde $h\left(t\right)=\esp\left[X\left(t\right)\indora\left(T>t\right)\right]$.
\end{Teo}



%___________________________________________________________________________________________
%
\subsection{Funci\'on de Renovaci\'on}
%___________________________________________________________________________________________
%


\begin{Def}
Sea $h\left(t\right)$ funci\'on de valores reales en $\rea$ acotada en intervalos finitos e igual a cero para $t<0$ La ecuaci\'on de renovaci\'on para $h\left(t\right)$ y la distribuci\'on $F$ es

\begin{eqnarray}\label{Ec.Renovacion}
H\left(t\right)=h\left(t\right)+\int_{\left[0,t\right]}H\left(t-s\right)dF\left(s\right)\textrm{,    }t\geq0,
\end{eqnarray}
donde $H\left(t\right)$ es una funci\'on de valores reales. Esto es $H=h+F\star H$. Decimos que $H\left(t\right)$ es soluci\'on de esta ecuaci\'on si satisface la ecuaci\'on, y es acotada en intervalos finitos e iguales a cero para $t<0$.
\end{Def}

\begin{Prop}
La funci\'on $U\star h\left(t\right)$ es la \'unica soluci\'on de la ecuaci\'on de renovaci\'on (\ref{Ec.Renovacion}).
\end{Prop}

\begin{Teo}[Teorema Renovaci\'on Elemental]
\begin{eqnarray*}
t^{-1}U\left(t\right)\rightarrow 1/\mu\textrm{,    cuando }t\rightarrow\infty.
\end{eqnarray*}
\end{Teo}

%___________________________________________________________________________________________
%
\subsection{Propiedades de los Procesos de Renovaci\'on}
%___________________________________________________________________________________________
%

Los tiempos $T_{n}$ est\'an relacionados con los conteos de $N\left(t\right)$ por

\begin{eqnarray*}
\left\{N\left(t\right)\geq n\right\}&=&\left\{T_{n}\leq t\right\}\\
T_{N\left(t\right)}\leq &t&<T_{N\left(t\right)+1},
\end{eqnarray*}

adem\'as $N\left(T_{n}\right)=n$, y 

\begin{eqnarray*}
N\left(t\right)=\max\left\{n:T_{n}\leq t\right\}=\min\left\{n:T_{n+1}>t\right\}
\end{eqnarray*}

Por propiedades de la convoluci\'on se sabe que

\begin{eqnarray*}
P\left\{T_{n}\leq t\right\}=F^{n\star}\left(t\right)
\end{eqnarray*}
que es la $n$-\'esima convoluci\'on de $F$. Entonces 

\begin{eqnarray*}
\left\{N\left(t\right)\geq n\right\}&=&\left\{T_{n}\leq t\right\}\\
P\left\{N\left(t\right)\leq n\right\}&=&1-F^{\left(n+1\right)\star}\left(t\right)
\end{eqnarray*}

Adem\'as usando el hecho de que $\esp\left[N\left(t\right)\right]=\sum_{n=1}^{\infty}P\left\{N\left(t\right)\geq n\right\}$
se tiene que

\begin{eqnarray*}
\esp\left[N\left(t\right)\right]=\sum_{n=1}^{\infty}F^{n\star}\left(t\right)
\end{eqnarray*}

\begin{Prop}
Para cada $t\geq0$, la funci\'on generadora de momentos $\esp\left[e^{\alpha N\left(t\right)}\right]$ existe para alguna $\alpha$ en una vecindad del 0, y de aqu\'i que $\esp\left[N\left(t\right)^{m}\right]<\infty$, para $m\geq1$.
\end{Prop}


\begin{Note}
Si el primer tiempo de renovaci\'on $\xi_{1}$ no tiene la misma distribuci\'on que el resto de las $\xi_{n}$, para $n\geq2$, a $N\left(t\right)$ se le llama Proceso de Renovaci\'on retardado, donde si $\xi$ tiene distribuci\'on $G$, entonces el tiempo $T_{n}$ de la $n$-\'esima renovaci\'on tiene distribuci\'on $G\star F^{\left(n-1\right)\star}\left(t\right)$
\end{Note}


\begin{Teo}
Para una constante $\mu\leq\infty$ ( o variable aleatoria), las siguientes expresiones son equivalentes:

\begin{eqnarray}
lim_{n\rightarrow\infty}n^{-1}T_{n}&=&\mu,\textrm{ c.s.}\\
lim_{t\rightarrow\infty}t^{-1}N\left(t\right)&=&1/\mu,\textrm{ c.s.}
\end{eqnarray}
\end{Teo}


Es decir, $T_{n}$ satisface la Ley Fuerte de los Grandes N\'umeros s\'i y s\'olo s\'i $N\left/t\right)$ la cumple.


\begin{Coro}[Ley Fuerte de los Grandes N\'umeros para Procesos de Renovaci\'on]
Si $N\left(t\right)$ es un proceso de renovaci\'on cuyos tiempos de inter-renovaci\'on tienen media $\mu\leq\infty$, entonces
\begin{eqnarray}
t^{-1}N\left(t\right)\rightarrow 1/\mu,\textrm{ c.s. cuando }t\rightarrow\infty.
\end{eqnarray}

\end{Coro}


Considerar el proceso estoc\'astico de valores reales $\left\{Z\left(t\right):t\geq0\right\}$ en el mismo espacio de probabilidad que $N\left(t\right)$

\begin{Def}
Para el proceso $\left\{Z\left(t\right):t\geq0\right\}$ se define la fluctuaci\'on m\'axima de $Z\left(t\right)$ en el intervalo $\left(T_{n-1},T_{n}\right]$:
\begin{eqnarray*}
M_{n}=\sup_{T_{n-1}<t\leq T_{n}}|Z\left(t\right)-Z\left(T_{n-1}\right)|
\end{eqnarray*}
\end{Def}

\begin{Teo}
Sup\'ongase que $n^{-1}T_{n}\rightarrow\mu$ c.s. cuando $n\rightarrow\infty$, donde $\mu\leq\infty$ es una constante o variable aleatoria. Sea $a$ una constante o variable aleatoria que puede ser infinita cuando $\mu$ es finita, y considere las expresiones l\'imite:
\begin{eqnarray}
lim_{n\rightarrow\infty}n^{-1}Z\left(T_{n}\right)&=&a,\textrm{ c.s.}\\
lim_{t\rightarrow\infty}t^{-1}Z\left(t\right)&=&a/\mu,\textrm{ c.s.}
\end{eqnarray}
La segunda expresi\'on implica la primera. Conversamente, la primera implica la segunda si el proceso $Z\left(t\right)$ es creciente, o si $lim_{n\rightarrow\infty}n^{-1}M_{n}=0$ c.s.
\end{Teo}

\begin{Coro}
Si $N\left(t\right)$ es un proceso de renovaci\'on, y $\left(Z\left(T_{n}\right)-Z\left(T_{n-1}\right),M_{n}\right)$, para $n\geq1$, son variables aleatorias independientes e id\'enticamente distribuidas con media finita, entonces,
\begin{eqnarray}
lim_{t\rightarrow\infty}t^{-1}Z\left(t\right)\rightarrow\frac{\esp\left[Z\left(T_{1}\right)-Z\left(T_{0}\right)\right]}{\esp\left[T_{1}\right]},\textrm{ c.s. cuando  }t\rightarrow\infty.
\end{eqnarray}
\end{Coro}

%___________________________________________________________________________________________
%
\subsection{Funci\'on de Renovaci\'on}
%___________________________________________________________________________________________
%


Sup\'ongase que $N\left(t\right)$ es un proceso de renovaci\'on con distribuci\'on $F$ con media finita $\mu$.

\begin{Def}
La funci\'on de renovaci\'on asociada con la distribuci\'on $F$, del proceso $N\left(t\right)$, es
\begin{eqnarray*}
U\left(t\right)=\sum_{n=1}^{\infty}F^{n\star}\left(t\right),\textrm{   }t\geq0,
\end{eqnarray*}
donde $F^{0\star}\left(t\right)=\indora\left(t\geq0\right)$.
\end{Def}


\begin{Prop}
Sup\'ongase que la distribuci\'on de inter-renovaci\'on $F$ tiene densidad $f$. Entonces $U\left(t\right)$ tambi\'en tiene densidad, para $t>0$, y es $U^{'}\left(t\right)=\sum_{n=0}^{\infty}f^{n\star}\left(t\right)$. Adem\'as
\begin{eqnarray*}
\prob\left\{N\left(t\right)>N\left(t-\right)\right\}=0\textrm{,   }t\geq0.
\end{eqnarray*}
\end{Prop}

\begin{Def}
La Transformada de Laplace-Stieljes de $F$ est\'a dada por

\begin{eqnarray*}
\hat{F}\left(\alpha\right)=\int_{\rea_{+}}e^{-\alpha t}dF\left(t\right)\textrm{,  }\alpha\geq0.
\end{eqnarray*}
\end{Def}

Entonces

\begin{eqnarray*}
\hat{U}\left(\alpha\right)=\sum_{n=0}^{\infty}\hat{F^{n\star}}\left(\alpha\right)=\sum_{n=0}^{\infty}\hat{F}\left(\alpha\right)^{n}=\frac{1}{1-\hat{F}\left(\alpha\right)}.
\end{eqnarray*}


\begin{Prop}
La Transformada de Laplace $\hat{U}\left(\alpha\right)$ y $\hat{F}\left(\alpha\right)$ determina una a la otra de manera \'unica por la relaci\'on $\hat{U}\left(\alpha\right)=\frac{1}{1-\hat{F}\left(\alpha\right)}$.
\end{Prop}


\begin{Note}
Un proceso de renovaci\'on $N\left(t\right)$ cuyos tiempos de inter-renovaci\'on tienen media finita, es un proceso Poisson con tasa $\lambda$ si y s\'olo s\'i $\esp\left[U\left(t\right)\right]=\lambda t$, para $t\geq0$.
\end{Note}


\begin{Teo}
Sea $N\left(t\right)$ un proceso puntual simple con puntos de localizaci\'on $T_{n}$ tal que $\eta\left(t\right)=\esp\left[N\left(\right)\right]$ es finita para cada $t$. Entonces para cualquier funci\'on $f:\rea_{+}\rightarrow\rea$,
\begin{eqnarray*}
\esp\left[\sum_{n=1}^{N\left(\right)}f\left(T_{n}\right)\right]=\int_{\left(0,t\right]}f\left(s\right)d\eta\left(s\right)\textrm{,  }t\geq0,
\end{eqnarray*}
suponiendo que la integral exista. Adem\'as si $X_{1},X_{2},\ldots$ son variables aleatorias definidas en el mismo espacio de probabilidad que el proceso $N\left(t\right)$ tal que $\esp\left[X_{n}|T_{n}=s\right]=f\left(s\right)$, independiente de $n$. Entonces
\begin{eqnarray*}
\esp\left[\sum_{n=1}^{N\left(t\right)}X_{n}\right]=\int_{\left(0,t\right]}f\left(s\right)d\eta\left(s\right)\textrm{,  }t\geq0,
\end{eqnarray*} 
suponiendo que la integral exista. 
\end{Teo}

\begin{Coro}[Identidad de Wald para Renovaciones]
Para el proceso de renovaci\'on $N\left(t\right)$,
\begin{eqnarray*}
\esp\left[T_{N\left(t\right)+1}\right]=\mu\esp\left[N\left(t\right)+1\right]\textrm{,  }t\geq0,
\end{eqnarray*}  
\end{Coro}

%______________________________________________________________________
\subsection{Procesos de Renovaci\'on}
%______________________________________________________________________

\begin{Def}\label{Def.Tn}
Sean $0\leq T_{1}\leq T_{2}\leq \ldots$ son tiempos aleatorios infinitos en los cuales ocurren ciertos eventos. El n\'umero de tiempos $T_{n}$ en el intervalo $\left[0,t\right)$ es

\begin{eqnarray}
N\left(t\right)=\sum_{n=1}^{\infty}\indora\left(T_{n}\leq t\right),
\end{eqnarray}
para $t\geq0$.
\end{Def}

Si se consideran los puntos $T_{n}$ como elementos de $\rea_{+}$, y $N\left(t\right)$ es el n\'umero de puntos en $\rea$. El proceso denotado por $\left\{N\left(t\right):t\geq0\right\}$, denotado por $N\left(t\right)$, es un proceso puntual en $\rea_{+}$. Los $T_{n}$ son los tiempos de ocurrencia, el proceso puntual $N\left(t\right)$ es simple si su n\'umero de ocurrencias son distintas: $0<T_{1}<T_{2}<\ldots$ casi seguramente.

\begin{Def}
Un proceso puntual $N\left(t\right)$ es un proceso de renovaci\'on si los tiempos de interocurrencia $\xi_{n}=T_{n}-T_{n-1}$, para $n\geq1$, son independientes e identicamente distribuidos con distribuci\'on $F$, donde $F\left(0\right)=0$ y $T_{0}=0$. Los $T_{n}$ son llamados tiempos de renovaci\'on, referente a la independencia o renovaci\'on de la informaci\'on estoc\'astica en estos tiempos. Los $\xi_{n}$ son los tiempos de inter-renovaci\'on, y $N\left(t\right)$ es el n\'umero de renovaciones en el intervalo $\left[0,t\right)$
\end{Def}


\begin{Note}
Para definir un proceso de renovaci\'on para cualquier contexto, solamente hay que especificar una distribuci\'on $F$, con $F\left(0\right)=0$, para los tiempos de inter-renovaci\'on. La funci\'on $F$ en turno degune las otra variables aleatorias. De manera formal, existe un espacio de probabilidad y una sucesi\'on de variables aleatorias $\xi_{1},\xi_{2},\ldots$ definidas en este con distribuci\'on $F$. Entonces las otras cantidades son $T_{n}=\sum_{k=1}^{n}\xi_{k}$ y $N\left(t\right)=\sum_{n=1}^{\infty}\indora\left(T_{n}\leq t\right)$, donde $T_{n}\rightarrow\infty$ casi seguramente por la Ley Fuerte de los Grandes Números.
\end{Note}
%_____________________________________________________
\subsection{Puntos de Renovaci\'on}
%_____________________________________________________

Para cada cola $Q_{i}$ se tienen los procesos de arribo a la cola, para estas, los tiempos de arribo est\'an dados por $$\left\{T_{1}^{i},T_{2}^{i},\ldots,T_{k}^{i},\ldots\right\},$$ entonces, consideremos solamente los primeros tiempos de arribo a cada una de las colas, es decir, $$\left\{T_{1}^{1},T_{1}^{2},T_{1}^{3},T_{1}^{4}\right\},$$ se sabe que cada uno de estos tiempos se distribuye de manera exponencial con par\'ametro $1/mu_{i}$. Adem\'as se sabe que para $$T^{*}=\min\left\{T_{1}^{1},T_{1}^{2},T_{1}^{3},T_{1}^{4}\right\},$$ $T^{*}$ se distribuye de manera exponencial con par\'ametro $$\mu^{*}=\sum_{i=1}^{4}\mu_{i}.$$ Ahora, dado que 
\begin{center}
\begin{tabular}{lcl}
$\tilde{r}=r_{1}+r_{2}$ & y &$\hat{r}=r_{3}+r_{4}.$
\end{tabular}
\end{center}


Supongamos que $$\tilde{r},\hat{r}<\mu^{*},$$ entonces si tomamos $$r^{*}=\min\left\{\tilde{r},\hat{r}\right\},$$ se tiene que para  $$t^{*}\in\left(0,r^{*}\right)$$ se cumple que 
\begin{center}
\begin{tabular}{lcl}
$\tau_{1}\left(1\right)=0$ & y por tanto & $\overline{\tau}_{1}=0,$
\end{tabular}
\end{center}
entonces para la segunda cola en este primer ciclo se cumple que $$\tau_{2}=\overline{\tau}_{1}+r_{1}=r_{1}<\mu^{*},$$ y por tanto se tiene que  $$\overline{\tau}_{2}=\tau_{2}.$$ Por lo tanto, nuevamente para la primer cola en el segundo ciclo $$\tau_{1}\left(2\right)=\tau_{2}\left(1\right)+r_{2}=\tilde{r}<\mu^{*}.$$ An\'alogamente para el segundo sistema se tiene que ambas colas est\'an vac\'ias, es decir, existe un valor $t^{*}$ tal que en el intervalo $\left(0,t^{*}\right)$ no ha llegado ning\'un usuario, es decir, $$L_{i}\left(t^{*}\right)=0$$ para $i=1,2,3,4$.



%________________________________________________________________________
\subsection{Procesos Regenerativos}
%________________________________________________________________________

Para $\left\{X\left(t\right):t\geq0\right\}$ Proceso Estoc\'astico a tiempo continuo con estado de espacios $S$, que es un espacio m\'etrico, con trayectorias continuas por la derecha y con l\'imites por la izquierda c.s. Sea $N\left(t\right)$ un proceso de renovaci\'on en $\rea_{+}$ definido en el mismo espacio de probabilidad que $X\left(t\right)$, con tiempos de renovaci\'on $T$ y tiempos de inter-renovaci\'on $\xi_{n}=T_{n}-T_{n-1}$, con misma distribuci\'on $F$ de media finita $\mu$.



\begin{Def}
Para el proceso $\left\{\left(N\left(t\right),X\left(t\right)\right):t\geq0\right\}$, sus trayectoria muestrales en el intervalo de tiempo $\left[T_{n-1},T_{n}\right)$ est\'an descritas por
\begin{eqnarray*}
\zeta_{n}=\left(\xi_{n},\left\{X\left(T_{n-1}+t\right):0\leq t<\xi_{n}\right\}\right)
\end{eqnarray*}
Este $\zeta_{n}$ es el $n$-\'esimo segmento del proceso. El proceso es regenerativo sobre los tiempos $T_{n}$ si sus segmentos $\zeta_{n}$ son independientes e id\'enticamennte distribuidos.
\end{Def}


\begin{Obs}
Si $\tilde{X}\left(t\right)$ con espacio de estados $\tilde{S}$ es regenerativo sobre $T_{n}$, entonces $X\left(t\right)=f\left(\tilde{X}\left(t\right)\right)$ tambi\'en es regenerativo sobre $T_{n}$, para cualquier funci\'on $f:\tilde{S}\rightarrow S$.
\end{Obs}

\begin{Obs}
Los procesos regenerativos son crudamente regenerativos, pero no al rev\'es.
\end{Obs}

\begin{Def}[Definici\'on Cl\'asica]
Un proceso estoc\'astico $X=\left\{X\left(t\right):t\geq0\right\}$ es llamado regenerativo is existe una variable aleatoria $R_{1}>0$ tal que
\begin{itemize}
\item[i)] $\left\{X\left(t+R_{1}\right):t\geq0\right\}$ es independiente de $\left\{\left\{X\left(t\right):t<R_{1}\right\},\right\}$
\item[ii)] $\left\{X\left(t+R_{1}\right):t\geq0\right\}$ es estoc\'asticamente equivalente a $\left\{X\left(t\right):t>0\right\}$
\end{itemize}

Llamamos a $R_{1}$ tiempo de regeneraci\'on, y decimos que $X$ se regenera en este punto.
\end{Def}

$\left\{X\left(t+R_{1}\right)\right\}$ es regenerativo con tiempo de regeneraci\'on $R_{2}$, independiente de $R_{1}$ pero con la misma distribuci\'on que $R_{1}$. Procediendo de esta manera se obtiene una secuencia de variables aleatorias independientes e id\'enticamente distribuidas $\left\{R_{n}\right\}$ llamados longitudes de ciclo. Si definimos a $Z_{k}\equiv R_{1}+R_{2}+\cdots+R_{k}$, se tiene un proceso de renovaci\'on llamado proceso de renovaci\'on encajado para $X$.

\begin{Note}
Un proceso regenerativo con media de la longitud de ciclo finita es llamado positivo recurrente.
\end{Note}


\begin{Def}
Para $x$ fijo y para cada $t\geq0$, sea $I_{x}\left(t\right)=1$ si $X\left(t\right)\leq x$,  $I_{x}\left(t\right)=0$ en caso contrario, y def\'inanse los tiempos promedio
\begin{eqnarray*}
\overline{X}&=&lim_{t\rightarrow\infty}\frac{1}{t}\int_{0}^{\infty}X\left(u\right)du\\
\prob\left(X_{\infty}\leq x\right)&=&lim_{t\rightarrow\infty}\frac{1}{t}\int_{0}^{\infty}I_{x}\left(u\right)du,
\end{eqnarray*}
cuando estos l\'imites existan.
\end{Def}

Como consecuencia del teorema de Renovaci\'on-Recompensa, se tiene que el primer l\'imite  existe y es igual a la constante
\begin{eqnarray*}
\overline{X}&=&\frac{\esp\left[\int_{0}^{R_{1}}X\left(t\right)dt\right]}{\esp\left[R_{1}\right]},
\end{eqnarray*}
suponiendo que ambas esperanzas son finitas.

\begin{Note}
\begin{itemize}
\item[a)] Si el proceso regenerativo $X$ es positivo recurrente y tiene trayectorias muestrales no negativas, entonces la ecuaci\'on anterior es v\'alida.
\item[b)] Si $X$ es positivo recurrente regenerativo, podemos construir una \'unica versi\'on estacionaria de este proceso, $X_{e}=\left\{X_{e}\left(t\right)\right\}$, donde $X_{e}$ es un proceso estoc\'astico regenerativo y estrictamente estacionario, con distribuci\'on marginal distribuida como $X_{\infty}$
\end{itemize}
\end{Note}

\subsection{Renewal and Regenerative Processes: Serfozo\cite{Serfozo}}
\begin{Def}\label{Def.Tn}
Sean $0\leq T_{1}\leq T_{2}\leq \ldots$ son tiempos aleatorios infinitos en los cuales ocurren ciertos eventos. El n\'umero de tiempos $T_{n}$ en el intervalo $\left[0,t\right)$ es

\begin{eqnarray}
N\left(t\right)=\sum_{n=1}^{\infty}\indora\left(T_{n}\leq t\right),
\end{eqnarray}
para $t\geq0$.
\end{Def}

Si se consideran los puntos $T_{n}$ como elementos de $\rea_{+}$, y $N\left(t\right)$ es el n\'umero de puntos en $\rea$. El proceso denotado por $\left\{N\left(t\right):t\geq0\right\}$, denotado por $N\left(t\right)$, es un proceso puntual en $\rea_{+}$. Los $T_{n}$ son los tiempos de ocurrencia, el proceso puntual $N\left(t\right)$ es simple si su n\'umero de ocurrencias son distintas: $0<T_{1}<T_{2}<\ldots$ casi seguramente.

\begin{Def}
Un proceso puntual $N\left(t\right)$ es un proceso de renovaci\'on si los tiempos de interocurrencia $\xi_{n}=T_{n}-T_{n-1}$, para $n\geq1$, son independientes e identicamente distribuidos con distribuci\'on $F$, donde $F\left(0\right)=0$ y $T_{0}=0$. Los $T_{n}$ son llamados tiempos de renovaci\'on, referente a la independencia o renovaci\'on de la informaci\'on estoc\'astica en estos tiempos. Los $\xi_{n}$ son los tiempos de inter-renovaci\'on, y $N\left(t\right)$ es el n\'umero de renovaciones en el intervalo $\left[0,t\right)$
\end{Def}


\begin{Note}
Para definir un proceso de renovaci\'on para cualquier contexto, solamente hay que especificar una distribuci\'on $F$, con $F\left(0\right)=0$, para los tiempos de inter-renovaci\'on. La funci\'on $F$ en turno degune las otra variables aleatorias. De manera formal, existe un espacio de probabilidad y una sucesi\'on de variables aleatorias $\xi_{1},\xi_{2},\ldots$ definidas en este con distribuci\'on $F$. Entonces las otras cantidades son $T_{n}=\sum_{k=1}^{n}\xi_{k}$ y $N\left(t\right)=\sum_{n=1}^{\infty}\indora\left(T_{n}\leq t\right)$, donde $T_{n}\rightarrow\infty$ casi seguramente por la Ley Fuerte de los Grandes N\'umeros.
\end{Note}







Los tiempos $T_{n}$ est\'an relacionados con los conteos de $N\left(t\right)$ por

\begin{eqnarray*}
\left\{N\left(t\right)\geq n\right\}&=&\left\{T_{n}\leq t\right\}\\
T_{N\left(t\right)}\leq &t&<T_{N\left(t\right)+1},
\end{eqnarray*}

adem\'as $N\left(T_{n}\right)=n$, y 

\begin{eqnarray*}
N\left(t\right)=\max\left\{n:T_{n}\leq t\right\}=\min\left\{n:T_{n+1}>t\right\}
\end{eqnarray*}

Por propiedades de la convoluci\'on se sabe que

\begin{eqnarray*}
P\left\{T_{n}\leq t\right\}=F^{n\star}\left(t\right)
\end{eqnarray*}
que es la $n$-\'esima convoluci\'on de $F$. Entonces 

\begin{eqnarray*}
\left\{N\left(t\right)\geq n\right\}&=&\left\{T_{n}\leq t\right\}\\
P\left\{N\left(t\right)\leq n\right\}&=&1-F^{\left(n+1\right)\star}\left(t\right)
\end{eqnarray*}

Adem\'as usando el hecho de que $\esp\left[N\left(t\right)\right]=\sum_{n=1}^{\infty}P\left\{N\left(t\right)\geq n\right\}$
se tiene que

\begin{eqnarray*}
\esp\left[N\left(t\right)\right]=\sum_{n=1}^{\infty}F^{n\star}\left(t\right)
\end{eqnarray*}

\begin{Prop}
Para cada $t\geq0$, la funci\'on generadora de momentos $\esp\left[e^{\alpha N\left(t\right)}\right]$ existe para alguna $\alpha$ en una vecindad del 0, y de aqu\'i que $\esp\left[N\left(t\right)^{m}\right]<\infty$, para $m\geq1$.
\end{Prop}


\begin{Note}
Si el primer tiempo de renovaci\'on $\xi_{1}$ no tiene la misma distribuci\'on que el resto de las $\xi_{n}$, para $n\geq2$, a $N\left(t\right)$ se le llama Proceso de Renovaci\'on retardado, donde si $\xi$ tiene distribuci\'on $G$, entonces el tiempo $T_{n}$ de la $n$-\'esima renovaci\'on tiene distribuci\'on $G\star F^{\left(n-1\right)\star}\left(t\right)$
\end{Note}


\begin{Teo}
Para una constante $\mu\leq\infty$ ( o variable aleatoria), las siguientes expresiones son equivalentes:

\begin{eqnarray}
lim_{n\rightarrow\infty}n^{-1}T_{n}&=&\mu,\textrm{ c.s.}\\
lim_{t\rightarrow\infty}t^{-1}N\left(t\right)&=&1/\mu,\textrm{ c.s.}
\end{eqnarray}
\end{Teo}


Es decir, $T_{n}$ satisface la Ley Fuerte de los Grandes N\'umeros s\'i y s\'olo s\'i $N\left/t\right)$ la cumple.


\begin{Coro}[Ley Fuerte de los Grandes N\'umeros para Procesos de Renovaci\'on]
Si $N\left(t\right)$ es un proceso de renovaci\'on cuyos tiempos de inter-renovaci\'on tienen media $\mu\leq\infty$, entonces
\begin{eqnarray}
t^{-1}N\left(t\right)\rightarrow 1/\mu,\textrm{ c.s. cuando }t\rightarrow\infty.
\end{eqnarray}

\end{Coro}


Considerar el proceso estoc\'astico de valores reales $\left\{Z\left(t\right):t\geq0\right\}$ en el mismo espacio de probabilidad que $N\left(t\right)$

\begin{Def}
Para el proceso $\left\{Z\left(t\right):t\geq0\right\}$ se define la fluctuaci\'on m\'axima de $Z\left(t\right)$ en el intervalo $\left(T_{n-1},T_{n}\right]$:
\begin{eqnarray*}
M_{n}=\sup_{T_{n-1}<t\leq T_{n}}|Z\left(t\right)-Z\left(T_{n-1}\right)|
\end{eqnarray*}
\end{Def}

\begin{Teo}
Sup\'ongase que $n^{-1}T_{n}\rightarrow\mu$ c.s. cuando $n\rightarrow\infty$, donde $\mu\leq\infty$ es una constante o variable aleatoria. Sea $a$ una constante o variable aleatoria que puede ser infinita cuando $\mu$ es finita, y considere las expresiones l\'imite:
\begin{eqnarray}
lim_{n\rightarrow\infty}n^{-1}Z\left(T_{n}\right)&=&a,\textrm{ c.s.}\\
lim_{t\rightarrow\infty}t^{-1}Z\left(t\right)&=&a/\mu,\textrm{ c.s.}
\end{eqnarray}
La segunda expresi\'on implica la primera. Conversamente, la primera implica la segunda si el proceso $Z\left(t\right)$ es creciente, o si $lim_{n\rightarrow\infty}n^{-1}M_{n}=0$ c.s.
\end{Teo}

\begin{Coro}
Si $N\left(t\right)$ es un proceso de renovaci\'on, y $\left(Z\left(T_{n}\right)-Z\left(T_{n-1}\right),M_{n}\right)$, para $n\geq1$, son variables aleatorias independientes e id\'enticamente distribuidas con media finita, entonces,
\begin{eqnarray}
lim_{t\rightarrow\infty}t^{-1}Z\left(t\right)\rightarrow\frac{\esp\left[Z\left(T_{1}\right)-Z\left(T_{0}\right)\right]}{\esp\left[T_{1}\right]},\textrm{ c.s. cuando  }t\rightarrow\infty.
\end{eqnarray}
\end{Coro}


Sup\'ongase que $N\left(t\right)$ es un proceso de renovaci\'on con distribuci\'on $F$ con media finita $\mu$.

\begin{Def}
La funci\'on de renovaci\'on asociada con la distribuci\'on $F$, del proceso $N\left(t\right)$, es
\begin{eqnarray*}
U\left(t\right)=\sum_{n=1}^{\infty}F^{n\star}\left(t\right),\textrm{   }t\geq0,
\end{eqnarray*}
donde $F^{0\star}\left(t\right)=\indora\left(t\geq0\right)$.
\end{Def}


\begin{Prop}
Sup\'ongase que la distribuci\'on de inter-renovaci\'on $F$ tiene densidad $f$. Entonces $U\left(t\right)$ tambi\'en tiene densidad, para $t>0$, y es $U^{'}\left(t\right)=\sum_{n=0}^{\infty}f^{n\star}\left(t\right)$. Adem\'as
\begin{eqnarray*}
\prob\left\{N\left(t\right)>N\left(t-\right)\right\}=0\textrm{,   }t\geq0.
\end{eqnarray*}
\end{Prop}

\begin{Def}
La Transformada de Laplace-Stieljes de $F$ est\'a dada por

\begin{eqnarray*}
\hat{F}\left(\alpha\right)=\int_{\rea_{+}}e^{-\alpha t}dF\left(t\right)\textrm{,  }\alpha\geq0.
\end{eqnarray*}
\end{Def}

Entonces

\begin{eqnarray*}
\hat{U}\left(\alpha\right)=\sum_{n=0}^{\infty}\hat{F^{n\star}}\left(\alpha\right)=\sum_{n=0}^{\infty}\hat{F}\left(\alpha\right)^{n}=\frac{1}{1-\hat{F}\left(\alpha\right)}.
\end{eqnarray*}


\begin{Prop}
La Transformada de Laplace $\hat{U}\left(\alpha\right)$ y $\hat{F}\left(\alpha\right)$ determina una a la otra de manera \'unica por la relaci\'on $\hat{U}\left(\alpha\right)=\frac{1}{1-\hat{F}\left(\alpha\right)}$.
\end{Prop}


\begin{Note}
Un proceso de renovaci\'on $N\left(t\right)$ cuyos tiempos de inter-renovaci\'on tienen media finita, es un proceso Poisson con tasa $\lambda$ si y s\'olo s\'i $\esp\left[U\left(t\right)\right]=\lambda t$, para $t\geq0$.
\end{Note}


\begin{Teo}
Sea $N\left(t\right)$ un proceso puntual simple con puntos de localizaci\'on $T_{n}$ tal que $\eta\left(t\right)=\esp\left[N\left(\right)\right]$ es finita para cada $t$. Entonces para cualquier funci\'on $f:\rea_{+}\rightarrow\rea$,
\begin{eqnarray*}
\esp\left[\sum_{n=1}^{N\left(\right)}f\left(T_{n}\right)\right]=\int_{\left(0,t\right]}f\left(s\right)d\eta\left(s\right)\textrm{,  }t\geq0,
\end{eqnarray*}
suponiendo que la integral exista. Adem\'as si $X_{1},X_{2},\ldots$ son variables aleatorias definidas en el mismo espacio de probabilidad que el proceso $N\left(t\right)$ tal que $\esp\left[X_{n}|T_{n}=s\right]=f\left(s\right)$, independiente de $n$. Entonces
\begin{eqnarray*}
\esp\left[\sum_{n=1}^{N\left(t\right)}X_{n}\right]=\int_{\left(0,t\right]}f\left(s\right)d\eta\left(s\right)\textrm{,  }t\geq0,
\end{eqnarray*} 
suponiendo que la integral exista. 
\end{Teo}

\begin{Coro}[Identidad de Wald para Renovaciones]
Para el proceso de renovaci\'on $N\left(t\right)$,
\begin{eqnarray*}
\esp\left[T_{N\left(t\right)+1}\right]=\mu\esp\left[N\left(t\right)+1\right]\textrm{,  }t\geq0,
\end{eqnarray*}  
\end{Coro}


\begin{Def}
Sea $h\left(t\right)$ funci\'on de valores reales en $\rea$ acotada en intervalos finitos e igual a cero para $t<0$ La ecuaci\'on de renovaci\'on para $h\left(t\right)$ y la distribuci\'on $F$ es

\begin{eqnarray}\label{Ec.Renovacion}
H\left(t\right)=h\left(t\right)+\int_{\left[0,t\right]}H\left(t-s\right)dF\left(s\right)\textrm{,    }t\geq0,
\end{eqnarray}
donde $H\left(t\right)$ es una funci\'on de valores reales. Esto es $H=h+F\star H$. Decimos que $H\left(t\right)$ es soluci\'on de esta ecuaci\'on si satisface la ecuaci\'on, y es acotada en intervalos finitos e iguales a cero para $t<0$.
\end{Def}

\begin{Prop}
La funci\'on $U\star h\left(t\right)$ es la \'unica soluci\'on de la ecuaci\'on de renovaci\'on (\ref{Ec.Renovacion}).
\end{Prop}

\begin{Teo}[Teorema Renovaci\'on Elemental]
\begin{eqnarray*}
t^{-1}U\left(t\right)\rightarrow 1/\mu\textrm{,    cuando }t\rightarrow\infty.
\end{eqnarray*}
\end{Teo}



Sup\'ongase que $N\left(t\right)$ es un proceso de renovaci\'on con distribuci\'on $F$ con media finita $\mu$.

\begin{Def}
La funci\'on de renovaci\'on asociada con la distribuci\'on $F$, del proceso $N\left(t\right)$, es
\begin{eqnarray*}
U\left(t\right)=\sum_{n=1}^{\infty}F^{n\star}\left(t\right),\textrm{   }t\geq0,
\end{eqnarray*}
donde $F^{0\star}\left(t\right)=\indora\left(t\geq0\right)$.
\end{Def}


\begin{Prop}
Sup\'ongase que la distribuci\'on de inter-renovaci\'on $F$ tiene densidad $f$. Entonces $U\left(t\right)$ tambi\'en tiene densidad, para $t>0$, y es $U^{'}\left(t\right)=\sum_{n=0}^{\infty}f^{n\star}\left(t\right)$. Adem\'as
\begin{eqnarray*}
\prob\left\{N\left(t\right)>N\left(t-\right)\right\}=0\textrm{,   }t\geq0.
\end{eqnarray*}
\end{Prop}

\begin{Def}
La Transformada de Laplace-Stieljes de $F$ est\'a dada por

\begin{eqnarray*}
\hat{F}\left(\alpha\right)=\int_{\rea_{+}}e^{-\alpha t}dF\left(t\right)\textrm{,  }\alpha\geq0.
\end{eqnarray*}
\end{Def}

Entonces

\begin{eqnarray*}
\hat{U}\left(\alpha\right)=\sum_{n=0}^{\infty}\hat{F^{n\star}}\left(\alpha\right)=\sum_{n=0}^{\infty}\hat{F}\left(\alpha\right)^{n}=\frac{1}{1-\hat{F}\left(\alpha\right)}.
\end{eqnarray*}


\begin{Prop}
La Transformada de Laplace $\hat{U}\left(\alpha\right)$ y $\hat{F}\left(\alpha\right)$ determina una a la otra de manera \'unica por la relaci\'on $\hat{U}\left(\alpha\right)=\frac{1}{1-\hat{F}\left(\alpha\right)}$.
\end{Prop}


\begin{Note}
Un proceso de renovaci\'on $N\left(t\right)$ cuyos tiempos de inter-renovaci\'on tienen media finita, es un proceso Poisson con tasa $\lambda$ si y s\'olo s\'i $\esp\left[U\left(t\right)\right]=\lambda t$, para $t\geq0$.
\end{Note}


\begin{Teo}
Sea $N\left(t\right)$ un proceso puntual simple con puntos de localizaci\'on $T_{n}$ tal que $\eta\left(t\right)=\esp\left[N\left(\right)\right]$ es finita para cada $t$. Entonces para cualquier funci\'on $f:\rea_{+}\rightarrow\rea$,
\begin{eqnarray*}
\esp\left[\sum_{n=1}^{N\left(\right)}f\left(T_{n}\right)\right]=\int_{\left(0,t\right]}f\left(s\right)d\eta\left(s\right)\textrm{,  }t\geq0,
\end{eqnarray*}
suponiendo que la integral exista. Adem\'as si $X_{1},X_{2},\ldots$ son variables aleatorias definidas en el mismo espacio de probabilidad que el proceso $N\left(t\right)$ tal que $\esp\left[X_{n}|T_{n}=s\right]=f\left(s\right)$, independiente de $n$. Entonces
\begin{eqnarray*}
\esp\left[\sum_{n=1}^{N\left(t\right)}X_{n}\right]=\int_{\left(0,t\right]}f\left(s\right)d\eta\left(s\right)\textrm{,  }t\geq0,
\end{eqnarray*} 
suponiendo que la integral exista. 
\end{Teo}

\begin{Coro}[Identidad de Wald para Renovaciones]
Para el proceso de renovaci\'on $N\left(t\right)$,
\begin{eqnarray*}
\esp\left[T_{N\left(t\right)+1}\right]=\mu\esp\left[N\left(t\right)+1\right]\textrm{,  }t\geq0,
\end{eqnarray*}  
\end{Coro}


\begin{Def}
Sea $h\left(t\right)$ funci\'on de valores reales en $\rea$ acotada en intervalos finitos e igual a cero para $t<0$ La ecuaci\'on de renovaci\'on para $h\left(t\right)$ y la distribuci\'on $F$ es

\begin{eqnarray}\label{Ec.Renovacion}
H\left(t\right)=h\left(t\right)+\int_{\left[0,t\right]}H\left(t-s\right)dF\left(s\right)\textrm{,    }t\geq0,
\end{eqnarray}
donde $H\left(t\right)$ es una funci\'on de valores reales. Esto es $H=h+F\star H$. Decimos que $H\left(t\right)$ es soluci\'on de esta ecuaci\'on si satisface la ecuaci\'on, y es acotada en intervalos finitos e iguales a cero para $t<0$.
\end{Def}

\begin{Prop}
La funci\'on $U\star h\left(t\right)$ es la \'unica soluci\'on de la ecuaci\'on de renovaci\'on (\ref{Ec.Renovacion}).
\end{Prop}

\begin{Teo}[Teorema Renovaci\'on Elemental]
\begin{eqnarray*}
t^{-1}U\left(t\right)\rightarrow 1/\mu\textrm{,    cuando }t\rightarrow\infty.
\end{eqnarray*}
\end{Teo}


\begin{Note} Una funci\'on $h:\rea_{+}\rightarrow\rea$ es Directamente Riemann Integrable en los siguientes casos:
\begin{itemize}
\item[a)] $h\left(t\right)\geq0$ es decreciente y Riemann Integrable.
\item[b)] $h$ es continua excepto posiblemente en un conjunto de Lebesgue de medida 0, y $|h\left(t\right)|\leq b\left(t\right)$, donde $b$ es DRI.
\end{itemize}
\end{Note}

\begin{Teo}[Teorema Principal de Renovaci\'on]
Si $F$ es no aritm\'etica y $h\left(t\right)$ es Directamente Riemann Integrable (DRI), entonces

\begin{eqnarray*}
lim_{t\rightarrow\infty}U\star h=\frac{1}{\mu}\int_{\rea_{+}}h\left(s\right)ds.
\end{eqnarray*}
\end{Teo}

\begin{Prop}
Cualquier funci\'on $H\left(t\right)$ acotada en intervalos finitos y que es 0 para $t<0$ puede expresarse como
\begin{eqnarray*}
H\left(t\right)=U\star h\left(t\right)\textrm{,  donde }h\left(t\right)=H\left(t\right)-F\star H\left(t\right)
\end{eqnarray*}
\end{Prop}

\begin{Def}
Un proceso estoc\'astico $X\left(t\right)$ es crudamente regenerativo en un tiempo aleatorio positivo $T$ si
\begin{eqnarray*}
\esp\left[X\left(T+t\right)|T\right]=\esp\left[X\left(t\right)\right]\textrm{, para }t\geq0,\end{eqnarray*}
y con las esperanzas anteriores finitas.
\end{Def}

\begin{Prop}
Sup\'ongase que $X\left(t\right)$ es un proceso crudamente regenerativo en $T$, que tiene distribuci\'on $F$. Si $\esp\left[X\left(t\right)\right]$ es acotado en intervalos finitos, entonces
\begin{eqnarray*}
\esp\left[X\left(t\right)\right]=U\star h\left(t\right)\textrm{,  donde }h\left(t\right)=\esp\left[X\left(t\right)\indora\left(T>t\right)\right].
\end{eqnarray*}
\end{Prop}

\begin{Teo}[Regeneraci\'on Cruda]
Sup\'ongase que $X\left(t\right)$ es un proceso con valores positivo crudamente regenerativo en $T$, y def\'inase $M=\sup\left\{|X\left(t\right)|:t\leq T\right\}$. Si $T$ es no aritm\'etico y $M$ y $MT$ tienen media finita, entonces
\begin{eqnarray*}
lim_{t\rightarrow\infty}\esp\left[X\left(t\right)\right]=\frac{1}{\mu}\int_{\rea_{+}}h\left(s\right)ds,
\end{eqnarray*}
donde $h\left(t\right)=\esp\left[X\left(t\right)\indora\left(T>t\right)\right]$.
\end{Teo}


\begin{Note} Una funci\'on $h:\rea_{+}\rightarrow\rea$ es Directamente Riemann Integrable en los siguientes casos:
\begin{itemize}
\item[a)] $h\left(t\right)\geq0$ es decreciente y Riemann Integrable.
\item[b)] $h$ es continua excepto posiblemente en un conjunto de Lebesgue de medida 0, y $|h\left(t\right)|\leq b\left(t\right)$, donde $b$ es DRI.
\end{itemize}
\end{Note}

\begin{Teo}[Teorema Principal de Renovaci\'on]
Si $F$ es no aritm\'etica y $h\left(t\right)$ es Directamente Riemann Integrable (DRI), entonces

\begin{eqnarray*}
lim_{t\rightarrow\infty}U\star h=\frac{1}{\mu}\int_{\rea_{+}}h\left(s\right)ds.
\end{eqnarray*}
\end{Teo}

\begin{Prop}
Cualquier funci\'on $H\left(t\right)$ acotada en intervalos finitos y que es 0 para $t<0$ puede expresarse como
\begin{eqnarray*}
H\left(t\right)=U\star h\left(t\right)\textrm{,  donde }h\left(t\right)=H\left(t\right)-F\star H\left(t\right)
\end{eqnarray*}
\end{Prop}

\begin{Def}
Un proceso estoc\'astico $X\left(t\right)$ es crudamente regenerativo en un tiempo aleatorio positivo $T$ si
\begin{eqnarray*}
\esp\left[X\left(T+t\right)|T\right]=\esp\left[X\left(t\right)\right]\textrm{, para }t\geq0,\end{eqnarray*}
y con las esperanzas anteriores finitas.
\end{Def}

\begin{Prop}
Sup\'ongase que $X\left(t\right)$ es un proceso crudamente regenerativo en $T$, que tiene distribuci\'on $F$. Si $\esp\left[X\left(t\right)\right]$ es acotado en intervalos finitos, entonces
\begin{eqnarray*}
\esp\left[X\left(t\right)\right]=U\star h\left(t\right)\textrm{,  donde }h\left(t\right)=\esp\left[X\left(t\right)\indora\left(T>t\right)\right].
\end{eqnarray*}
\end{Prop}

\begin{Teo}[Regeneraci\'on Cruda]
Sup\'ongase que $X\left(t\right)$ es un proceso con valores positivo crudamente regenerativo en $T$, y def\'inase $M=\sup\left\{|X\left(t\right)|:t\leq T\right\}$. Si $T$ es no aritm\'etico y $M$ y $MT$ tienen media finita, entonces
\begin{eqnarray*}
lim_{t\rightarrow\infty}\esp\left[X\left(t\right)\right]=\frac{1}{\mu}\int_{\rea_{+}}h\left(s\right)ds,
\end{eqnarray*}
donde $h\left(t\right)=\esp\left[X\left(t\right)\indora\left(T>t\right)\right]$.
\end{Teo}

%________________________________________________________________________
\subsection{Procesos Regenerativos}
%________________________________________________________________________

Para $\left\{X\left(t\right):t\geq0\right\}$ Proceso Estoc\'astico a tiempo continuo con estado de espacios $S$, que es un espacio m\'etrico, con trayectorias continuas por la derecha y con l\'imites por la izquierda c.s. Sea $N\left(t\right)$ un proceso de renovaci\'on en $\rea_{+}$ definido en el mismo espacio de probabilidad que $X\left(t\right)$, con tiempos de renovaci\'on $T$ y tiempos de inter-renovaci\'on $\xi_{n}=T_{n}-T_{n-1}$, con misma distribuci\'on $F$ de media finita $\mu$.



\begin{Def}
Para el proceso $\left\{\left(N\left(t\right),X\left(t\right)\right):t\geq0\right\}$, sus trayectoria muestrales en el intervalo de tiempo $\left[T_{n-1},T_{n}\right)$ est\'an descritas por
\begin{eqnarray*}
\zeta_{n}=\left(\xi_{n},\left\{X\left(T_{n-1}+t\right):0\leq t<\xi_{n}\right\}\right)
\end{eqnarray*}
Este $\zeta_{n}$ es el $n$-\'esimo segmento del proceso. El proceso es regenerativo sobre los tiempos $T_{n}$ si sus segmentos $\zeta_{n}$ son independientes e id\'enticamennte distribuidos.
\end{Def}


\begin{Obs}
Si $\tilde{X}\left(t\right)$ con espacio de estados $\tilde{S}$ es regenerativo sobre $T_{n}$, entonces $X\left(t\right)=f\left(\tilde{X}\left(t\right)\right)$ tambi\'en es regenerativo sobre $T_{n}$, para cualquier funci\'on $f:\tilde{S}\rightarrow S$.
\end{Obs}

\begin{Obs}
Los procesos regenerativos son crudamente regenerativos, pero no al rev\'es.
\end{Obs}

\begin{Def}[Definici\'on Cl\'asica]
Un proceso estoc\'astico $X=\left\{X\left(t\right):t\geq0\right\}$ es llamado regenerativo is existe una variable aleatoria $R_{1}>0$ tal que
\begin{itemize}
\item[i)] $\left\{X\left(t+R_{1}\right):t\geq0\right\}$ es independiente de $\left\{\left\{X\left(t\right):t<R_{1}\right\},\right\}$
\item[ii)] $\left\{X\left(t+R_{1}\right):t\geq0\right\}$ es estoc\'asticamente equivalente a $\left\{X\left(t\right):t>0\right\}$
\end{itemize}

Llamamos a $R_{1}$ tiempo de regeneraci\'on, y decimos que $X$ se regenera en este punto.
\end{Def}

$\left\{X\left(t+R_{1}\right)\right\}$ es regenerativo con tiempo de regeneraci\'on $R_{2}$, independiente de $R_{1}$ pero con la misma distribuci\'on que $R_{1}$. Procediendo de esta manera se obtiene una secuencia de variables aleatorias independientes e id\'enticamente distribuidas $\left\{R_{n}\right\}$ llamados longitudes de ciclo. Si definimos a $Z_{k}\equiv R_{1}+R_{2}+\cdots+R_{k}$, se tiene un proceso de renovaci\'on llamado proceso de renovaci\'on encajado para $X$.

\begin{Note}
Un proceso regenerativo con media de la longitud de ciclo finita es llamado positivo recurrente.
\end{Note}


\begin{Def}
Para $x$ fijo y para cada $t\geq0$, sea $I_{x}\left(t\right)=1$ si $X\left(t\right)\leq x$,  $I_{x}\left(t\right)=0$ en caso contrario, y def\'inanse los tiempos promedio
\begin{eqnarray*}
\overline{X}&=&lim_{t\rightarrow\infty}\frac{1}{t}\int_{0}^{\infty}X\left(u\right)du\\
\prob\left(X_{\infty}\leq x\right)&=&lim_{t\rightarrow\infty}\frac{1}{t}\int_{0}^{\infty}I_{x}\left(u\right)du,
\end{eqnarray*}
cuando estos l\'imites existan.
\end{Def}

Como consecuencia del teorema de Renovaci\'on-Recompensa, se tiene que el primer l\'imite  existe y es igual a la constante
\begin{eqnarray*}
\overline{X}&=&\frac{\esp\left[\int_{0}^{R_{1}}X\left(t\right)dt\right]}{\esp\left[R_{1}\right]},
\end{eqnarray*}
suponiendo que ambas esperanzas son finitas.

\begin{Note}
\begin{itemize}
\item[a)] Si el proceso regenerativo $X$ es positivo recurrente y tiene trayectorias muestrales no negativas, entonces la ecuaci\'on anterior es v\'alida.
\item[b)] Si $X$ es positivo recurrente regenerativo, podemos construir una \'unica versi\'on estacionaria de este proceso, $X_{e}=\left\{X_{e}\left(t\right)\right\}$, donde $X_{e}$ es un proceso estoc\'astico regenerativo y estrictamente estacionario, con distribuci\'on marginal distribuida como $X_{\infty}$
\end{itemize}
\end{Note}

%________________________________________________________________________
\subsection{Procesos Regenerativos}
%________________________________________________________________________

Para $\left\{X\left(t\right):t\geq0\right\}$ Proceso Estoc\'astico a tiempo continuo con estado de espacios $S$, que es un espacio m\'etrico, con trayectorias continuas por la derecha y con l\'imites por la izquierda c.s. Sea $N\left(t\right)$ un proceso de renovaci\'on en $\rea_{+}$ definido en el mismo espacio de probabilidad que $X\left(t\right)$, con tiempos de renovaci\'on $T$ y tiempos de inter-renovaci\'on $\xi_{n}=T_{n}-T_{n-1}$, con misma distribuci\'on $F$ de media finita $\mu$.



\begin{Def}
Para el proceso $\left\{\left(N\left(t\right),X\left(t\right)\right):t\geq0\right\}$, sus trayectoria muestrales en el intervalo de tiempo $\left[T_{n-1},T_{n}\right)$ est\'an descritas por
\begin{eqnarray*}
\zeta_{n}=\left(\xi_{n},\left\{X\left(T_{n-1}+t\right):0\leq t<\xi_{n}\right\}\right)
\end{eqnarray*}
Este $\zeta_{n}$ es el $n$-\'esimo segmento del proceso. El proceso es regenerativo sobre los tiempos $T_{n}$ si sus segmentos $\zeta_{n}$ son independientes e id\'enticamennte distribuidos.
\end{Def}


\begin{Obs}
Si $\tilde{X}\left(t\right)$ con espacio de estados $\tilde{S}$ es regenerativo sobre $T_{n}$, entonces $X\left(t\right)=f\left(\tilde{X}\left(t\right)\right)$ tambi\'en es regenerativo sobre $T_{n}$, para cualquier funci\'on $f:\tilde{S}\rightarrow S$.
\end{Obs}

\begin{Obs}
Los procesos regenerativos son crudamente regenerativos, pero no al rev\'es.
\end{Obs}

\begin{Def}[Definici\'on Cl\'asica]
Un proceso estoc\'astico $X=\left\{X\left(t\right):t\geq0\right\}$ es llamado regenerativo is existe una variable aleatoria $R_{1}>0$ tal que
\begin{itemize}
\item[i)] $\left\{X\left(t+R_{1}\right):t\geq0\right\}$ es independiente de $\left\{\left\{X\left(t\right):t<R_{1}\right\},\right\}$
\item[ii)] $\left\{X\left(t+R_{1}\right):t\geq0\right\}$ es estoc\'asticamente equivalente a $\left\{X\left(t\right):t>0\right\}$
\end{itemize}

Llamamos a $R_{1}$ tiempo de regeneraci\'on, y decimos que $X$ se regenera en este punto.
\end{Def}

$\left\{X\left(t+R_{1}\right)\right\}$ es regenerativo con tiempo de regeneraci\'on $R_{2}$, independiente de $R_{1}$ pero con la misma distribuci\'on que $R_{1}$. Procediendo de esta manera se obtiene una secuencia de variables aleatorias independientes e id\'enticamente distribuidas $\left\{R_{n}\right\}$ llamados longitudes de ciclo. Si definimos a $Z_{k}\equiv R_{1}+R_{2}+\cdots+R_{k}$, se tiene un proceso de renovaci\'on llamado proceso de renovaci\'on encajado para $X$.

\begin{Note}
Un proceso regenerativo con media de la longitud de ciclo finita es llamado positivo recurrente.
\end{Note}


\begin{Def}
Para $x$ fijo y para cada $t\geq0$, sea $I_{x}\left(t\right)=1$ si $X\left(t\right)\leq x$,  $I_{x}\left(t\right)=0$ en caso contrario, y def\'inanse los tiempos promedio
\begin{eqnarray*}
\overline{X}&=&lim_{t\rightarrow\infty}\frac{1}{t}\int_{0}^{\infty}X\left(u\right)du\\
\prob\left(X_{\infty}\leq x\right)&=&lim_{t\rightarrow\infty}\frac{1}{t}\int_{0}^{\infty}I_{x}\left(u\right)du,
\end{eqnarray*}
cuando estos l\'imites existan.
\end{Def}

Como consecuencia del teorema de Renovaci\'on-Recompensa, se tiene que el primer l\'imite  existe y es igual a la constante
\begin{eqnarray*}
\overline{X}&=&\frac{\esp\left[\int_{0}^{R_{1}}X\left(t\right)dt\right]}{\esp\left[R_{1}\right]},
\end{eqnarray*}
suponiendo que ambas esperanzas son finitas.

\begin{Note}
\begin{itemize}
\item[a)] Si el proceso regenerativo $X$ es positivo recurrente y tiene trayectorias muestrales no negativas, entonces la ecuaci\'on anterior es v\'alida.
\item[b)] Si $X$ es positivo recurrente regenerativo, podemos construir una \'unica versi\'on estacionaria de este proceso, $X_{e}=\left\{X_{e}\left(t\right)\right\}$, donde $X_{e}$ es un proceso estoc\'astico regenerativo y estrictamente estacionario, con distribuci\'on marginal distribuida como $X_{\infty}$
\end{itemize}
\end{Note}
%__________________________________________________________________________________________
\subsection{Procesos Regenerativos Estacionarios - Stidham \cite{Stidham}}
%__________________________________________________________________________________________


Un proceso estoc\'astico a tiempo continuo $\left\{V\left(t\right),t\geq0\right\}$ es un proceso regenerativo si existe una sucesi\'on de variables aleatorias independientes e id\'enticamente distribuidas $\left\{X_{1},X_{2},\ldots\right\}$, sucesi\'on de renovaci\'on, tal que para cualquier conjunto de Borel $A$, 

\begin{eqnarray*}
\prob\left\{V\left(t\right)\in A|X_{1}+X_{2}+\cdots+X_{R\left(t\right)}=s,\left\{V\left(\tau\right),\tau<s\right\}\right\}=\prob\left\{V\left(t-s\right)\in A|X_{1}>t-s\right\},
\end{eqnarray*}
para todo $0\leq s\leq t$, donde $R\left(t\right)=\max\left\{X_{1}+X_{2}+\cdots+X_{j}\leq t\right\}=$n\'umero de renovaciones ({\emph{puntos de regeneraci\'on}}) que ocurren en $\left[0,t\right]$. El intervalo $\left[0,X_{1}\right)$ es llamado {\emph{primer ciclo de regeneraci\'on}} de $\left\{V\left(t \right),t\geq0\right\}$, $\left[X_{1},X_{1}+X_{2}\right)$ el {\emph{segundo ciclo de regeneraci\'on}}, y as\'i sucesivamente.

Sea $X=X_{1}$ y sea $F$ la funci\'on de distrbuci\'on de $X$


\begin{Def}
Se define el proceso estacionario, $\left\{V^{*}\left(t\right),t\geq0\right\}$, para $\left\{V\left(t\right),t\geq0\right\}$ por

\begin{eqnarray*}
\prob\left\{V\left(t\right)\in A\right\}=\frac{1}{\esp\left[X\right]}\int_{0}^{\infty}\prob\left\{V\left(t+x\right)\in A|X>x\right\}\left(1-F\left(x\right)\right)dx,
\end{eqnarray*} 
para todo $t\geq0$ y todo conjunto de Borel $A$.
\end{Def}

\begin{Def}
Una distribuci\'on se dice que es {\emph{aritm\'etica}} si todos sus puntos de incremento son m\'ultiplos de la forma $0,\lambda, 2\lambda,\ldots$ para alguna $\lambda>0$ entera.
\end{Def}


\begin{Def}
Una modificaci\'on medible de un proceso $\left\{V\left(t\right),t\geq0\right\}$, es una versi\'on de este, $\left\{V\left(t,w\right)\right\}$ conjuntamente medible para $t\geq0$ y para $w\in S$, $S$ espacio de estados para $\left\{V\left(t\right),t\geq0\right\}$.
\end{Def}

\begin{Teo}
Sea $\left\{V\left(t\right),t\geq\right\}$ un proceso regenerativo no negativo con modificaci\'on medible. Sea $\esp\left[X\right]<\infty$. Entonces el proceso estacionario dado por la ecuaci\'on anterior est\'a bien definido y tiene funci\'on de distribuci\'on independiente de $t$, adem\'as
\begin{itemize}
\item[i)] \begin{eqnarray*}
\esp\left[V^{*}\left(0\right)\right]&=&\frac{\esp\left[\int_{0}^{X}V\left(s\right)ds\right]}{\esp\left[X\right]}\end{eqnarray*}
\item[ii)] Si $\esp\left[V^{*}\left(0\right)\right]<\infty$, equivalentemente, si $\esp\left[\int_{0}^{X}V\left(s\right)ds\right]<\infty$,entonces
\begin{eqnarray*}
\frac{\int_{0}^{t}V\left(s\right)ds}{t}\rightarrow\frac{\esp\left[\int_{0}^{X}V\left(s\right)ds\right]}{\esp\left[X\right]}
\end{eqnarray*}
con probabilidad 1 y en media, cuando $t\rightarrow\infty$.
\end{itemize}
\end{Teo}


%__________________________________________________________________________________________
\subsection{Procesos Regenerativos Estacionarios - Stidham \cite{Stidham}}
%__________________________________________________________________________________________


Un proceso estoc\'astico a tiempo continuo $\left\{V\left(t\right),t\geq0\right\}$ es un proceso regenerativo si existe una sucesi\'on de variables aleatorias independientes e id\'enticamente distribuidas $\left\{X_{1},X_{2},\ldots\right\}$, sucesi\'on de renovaci\'on, tal que para cualquier conjunto de Borel $A$, 

\begin{eqnarray*}
\prob\left\{V\left(t\right)\in A|X_{1}+X_{2}+\cdots+X_{R\left(t\right)}=s,\left\{V\left(\tau\right),\tau<s\right\}\right\}=\prob\left\{V\left(t-s\right)\in A|X_{1}>t-s\right\},
\end{eqnarray*}
para todo $0\leq s\leq t$, donde $R\left(t\right)=\max\left\{X_{1}+X_{2}+\cdots+X_{j}\leq t\right\}=$n\'umero de renovaciones ({\emph{puntos de regeneraci\'on}}) que ocurren en $\left[0,t\right]$. El intervalo $\left[0,X_{1}\right)$ es llamado {\emph{primer ciclo de regeneraci\'on}} de $\left\{V\left(t \right),t\geq0\right\}$, $\left[X_{1},X_{1}+X_{2}\right)$ el {\emph{segundo ciclo de regeneraci\'on}}, y as\'i sucesivamente.

Sea $X=X_{1}$ y sea $F$ la funci\'on de distrbuci\'on de $X$


\begin{Def}
Se define el proceso estacionario, $\left\{V^{*}\left(t\right),t\geq0\right\}$, para $\left\{V\left(t\right),t\geq0\right\}$ por

\begin{eqnarray*}
\prob\left\{V\left(t\right)\in A\right\}=\frac{1}{\esp\left[X\right]}\int_{0}^{\infty}\prob\left\{V\left(t+x\right)\in A|X>x\right\}\left(1-F\left(x\right)\right)dx,
\end{eqnarray*} 
para todo $t\geq0$ y todo conjunto de Borel $A$.
\end{Def}

\begin{Def}
Una distribuci\'on se dice que es {\emph{aritm\'etica}} si todos sus puntos de incremento son m\'ultiplos de la forma $0,\lambda, 2\lambda,\ldots$ para alguna $\lambda>0$ entera.
\end{Def}


\begin{Def}
Una modificaci\'on medible de un proceso $\left\{V\left(t\right),t\geq0\right\}$, es una versi\'on de este, $\left\{V\left(t,w\right)\right\}$ conjuntamente medible para $t\geq0$ y para $w\in S$, $S$ espacio de estados para $\left\{V\left(t\right),t\geq0\right\}$.
\end{Def}

\begin{Teo}
Sea $\left\{V\left(t\right),t\geq\right\}$ un proceso regenerativo no negativo con modificaci\'on medible. Sea $\esp\left[X\right]<\infty$. Entonces el proceso estacionario dado por la ecuaci\'on anterior est\'a bien definido y tiene funci\'on de distribuci\'on independiente de $t$, adem\'as
\begin{itemize}
\item[i)] \begin{eqnarray*}
\esp\left[V^{*}\left(0\right)\right]&=&\frac{\esp\left[\int_{0}^{X}V\left(s\right)ds\right]}{\esp\left[X\right]}\end{eqnarray*}
\item[ii)] Si $\esp\left[V^{*}\left(0\right)\right]<\infty$, equivalentemente, si $\esp\left[\int_{0}^{X}V\left(s\right)ds\right]<\infty$,entonces
\begin{eqnarray*}
\frac{\int_{0}^{t}V\left(s\right)ds}{t}\rightarrow\frac{\esp\left[\int_{0}^{X}V\left(s\right)ds\right]}{\esp\left[X\right]}
\end{eqnarray*}
con probabilidad 1 y en media, cuando $t\rightarrow\infty$.
\end{itemize}
\end{Teo}
%
%___________________________________________________________________________________________
%\vspace{5.5cm}
%\chapter{Cadenas de Markov estacionarias}
%\vspace{-1.0cm}
%___________________________________________________________________________________________
%
\subsection{Propiedades de los Procesos de Renovaci\'on}
%___________________________________________________________________________________________
%

Los tiempos $T_{n}$ est\'an relacionados con los conteos de $N\left(t\right)$ por

\begin{eqnarray*}
\left\{N\left(t\right)\geq n\right\}&=&\left\{T_{n}\leq t\right\}\\
T_{N\left(t\right)}\leq &t&<T_{N\left(t\right)+1},
\end{eqnarray*}

adem\'as $N\left(T_{n}\right)=n$, y 

\begin{eqnarray*}
N\left(t\right)=\max\left\{n:T_{n}\leq t\right\}=\min\left\{n:T_{n+1}>t\right\}
\end{eqnarray*}

Por propiedades de la convoluci\'on se sabe que

\begin{eqnarray*}
P\left\{T_{n}\leq t\right\}=F^{n\star}\left(t\right)
\end{eqnarray*}
que es la $n$-\'esima convoluci\'on de $F$. Entonces 

\begin{eqnarray*}
\left\{N\left(t\right)\geq n\right\}&=&\left\{T_{n}\leq t\right\}\\
P\left\{N\left(t\right)\leq n\right\}&=&1-F^{\left(n+1\right)\star}\left(t\right)
\end{eqnarray*}

Adem\'as usando el hecho de que $\esp\left[N\left(t\right)\right]=\sum_{n=1}^{\infty}P\left\{N\left(t\right)\geq n\right\}$
se tiene que

\begin{eqnarray*}
\esp\left[N\left(t\right)\right]=\sum_{n=1}^{\infty}F^{n\star}\left(t\right)
\end{eqnarray*}

\begin{Prop}
Para cada $t\geq0$, la funci\'on generadora de momentos $\esp\left[e^{\alpha N\left(t\right)}\right]$ existe para alguna $\alpha$ en una vecindad del 0, y de aqu\'i que $\esp\left[N\left(t\right)^{m}\right]<\infty$, para $m\geq1$.
\end{Prop}


\begin{Note}
Si el primer tiempo de renovaci\'on $\xi_{1}$ no tiene la misma distribuci\'on que el resto de las $\xi_{n}$, para $n\geq2$, a $N\left(t\right)$ se le llama Proceso de Renovaci\'on retardado, donde si $\xi$ tiene distribuci\'on $G$, entonces el tiempo $T_{n}$ de la $n$-\'esima renovaci\'on tiene distribuci\'on $G\star F^{\left(n-1\right)\star}\left(t\right)$
\end{Note}


\begin{Teo}
Para una constante $\mu\leq\infty$ ( o variable aleatoria), las siguientes expresiones son equivalentes:

\begin{eqnarray}
lim_{n\rightarrow\infty}n^{-1}T_{n}&=&\mu,\textrm{ c.s.}\\
lim_{t\rightarrow\infty}t^{-1}N\left(t\right)&=&1/\mu,\textrm{ c.s.}
\end{eqnarray}
\end{Teo}


Es decir, $T_{n}$ satisface la Ley Fuerte de los Grandes N\'umeros s\'i y s\'olo s\'i $N\left/t\right)$ la cumple.


\begin{Coro}[Ley Fuerte de los Grandes N\'umeros para Procesos de Renovaci\'on]
Si $N\left(t\right)$ es un proceso de renovaci\'on cuyos tiempos de inter-renovaci\'on tienen media $\mu\leq\infty$, entonces
\begin{eqnarray}
t^{-1}N\left(t\right)\rightarrow 1/\mu,\textrm{ c.s. cuando }t\rightarrow\infty.
\end{eqnarray}

\end{Coro}


Considerar el proceso estoc\'astico de valores reales $\left\{Z\left(t\right):t\geq0\right\}$ en el mismo espacio de probabilidad que $N\left(t\right)$

\begin{Def}
Para el proceso $\left\{Z\left(t\right):t\geq0\right\}$ se define la fluctuaci\'on m\'axima de $Z\left(t\right)$ en el intervalo $\left(T_{n-1},T_{n}\right]$:
\begin{eqnarray*}
M_{n}=\sup_{T_{n-1}<t\leq T_{n}}|Z\left(t\right)-Z\left(T_{n-1}\right)|
\end{eqnarray*}
\end{Def}

\begin{Teo}
Sup\'ongase que $n^{-1}T_{n}\rightarrow\mu$ c.s. cuando $n\rightarrow\infty$, donde $\mu\leq\infty$ es una constante o variable aleatoria. Sea $a$ una constante o variable aleatoria que puede ser infinita cuando $\mu$ es finita, y considere las expresiones l\'imite:
\begin{eqnarray}
lim_{n\rightarrow\infty}n^{-1}Z\left(T_{n}\right)&=&a,\textrm{ c.s.}\\
lim_{t\rightarrow\infty}t^{-1}Z\left(t\right)&=&a/\mu,\textrm{ c.s.}
\end{eqnarray}
La segunda expresi\'on implica la primera. Conversamente, la primera implica la segunda si el proceso $Z\left(t\right)$ es creciente, o si $lim_{n\rightarrow\infty}n^{-1}M_{n}=0$ c.s.
\end{Teo}

\begin{Coro}
Si $N\left(t\right)$ es un proceso de renovaci\'on, y $\left(Z\left(T_{n}\right)-Z\left(T_{n-1}\right),M_{n}\right)$, para $n\geq1$, son variables aleatorias independientes e id\'enticamente distribuidas con media finita, entonces,
\begin{eqnarray}
lim_{t\rightarrow\infty}t^{-1}Z\left(t\right)\rightarrow\frac{\esp\left[Z\left(T_{1}\right)-Z\left(T_{0}\right)\right]}{\esp\left[T_{1}\right]},\textrm{ c.s. cuando  }t\rightarrow\infty.
\end{eqnarray}
\end{Coro}


%___________________________________________________________________________________________
%
%\subsection{Propiedades de los Procesos de Renovaci\'on}
%___________________________________________________________________________________________
%

Los tiempos $T_{n}$ est\'an relacionados con los conteos de $N\left(t\right)$ por

\begin{eqnarray*}
\left\{N\left(t\right)\geq n\right\}&=&\left\{T_{n}\leq t\right\}\\
T_{N\left(t\right)}\leq &t&<T_{N\left(t\right)+1},
\end{eqnarray*}

adem\'as $N\left(T_{n}\right)=n$, y 

\begin{eqnarray*}
N\left(t\right)=\max\left\{n:T_{n}\leq t\right\}=\min\left\{n:T_{n+1}>t\right\}
\end{eqnarray*}

Por propiedades de la convoluci\'on se sabe que

\begin{eqnarray*}
P\left\{T_{n}\leq t\right\}=F^{n\star}\left(t\right)
\end{eqnarray*}
que es la $n$-\'esima convoluci\'on de $F$. Entonces 

\begin{eqnarray*}
\left\{N\left(t\right)\geq n\right\}&=&\left\{T_{n}\leq t\right\}\\
P\left\{N\left(t\right)\leq n\right\}&=&1-F^{\left(n+1\right)\star}\left(t\right)
\end{eqnarray*}

Adem\'as usando el hecho de que $\esp\left[N\left(t\right)\right]=\sum_{n=1}^{\infty}P\left\{N\left(t\right)\geq n\right\}$
se tiene que

\begin{eqnarray*}
\esp\left[N\left(t\right)\right]=\sum_{n=1}^{\infty}F^{n\star}\left(t\right)
\end{eqnarray*}

\begin{Prop}
Para cada $t\geq0$, la funci\'on generadora de momentos $\esp\left[e^{\alpha N\left(t\right)}\right]$ existe para alguna $\alpha$ en una vecindad del 0, y de aqu\'i que $\esp\left[N\left(t\right)^{m}\right]<\infty$, para $m\geq1$.
\end{Prop}


\begin{Note}
Si el primer tiempo de renovaci\'on $\xi_{1}$ no tiene la misma distribuci\'on que el resto de las $\xi_{n}$, para $n\geq2$, a $N\left(t\right)$ se le llama Proceso de Renovaci\'on retardado, donde si $\xi$ tiene distribuci\'on $G$, entonces el tiempo $T_{n}$ de la $n$-\'esima renovaci\'on tiene distribuci\'on $G\star F^{\left(n-1\right)\star}\left(t\right)$
\end{Note}


\begin{Teo}
Para una constante $\mu\leq\infty$ ( o variable aleatoria), las siguientes expresiones son equivalentes:

\begin{eqnarray}
lim_{n\rightarrow\infty}n^{-1}T_{n}&=&\mu,\textrm{ c.s.}\\
lim_{t\rightarrow\infty}t^{-1}N\left(t\right)&=&1/\mu,\textrm{ c.s.}
\end{eqnarray}
\end{Teo}


Es decir, $T_{n}$ satisface la Ley Fuerte de los Grandes N\'umeros s\'i y s\'olo s\'i $N\left/t\right)$ la cumple.


\begin{Coro}[Ley Fuerte de los Grandes N\'umeros para Procesos de Renovaci\'on]
Si $N\left(t\right)$ es un proceso de renovaci\'on cuyos tiempos de inter-renovaci\'on tienen media $\mu\leq\infty$, entonces
\begin{eqnarray}
t^{-1}N\left(t\right)\rightarrow 1/\mu,\textrm{ c.s. cuando }t\rightarrow\infty.
\end{eqnarray}

\end{Coro}


Considerar el proceso estoc\'astico de valores reales $\left\{Z\left(t\right):t\geq0\right\}$ en el mismo espacio de probabilidad que $N\left(t\right)$

\begin{Def}
Para el proceso $\left\{Z\left(t\right):t\geq0\right\}$ se define la fluctuaci\'on m\'axima de $Z\left(t\right)$ en el intervalo $\left(T_{n-1},T_{n}\right]$:
\begin{eqnarray*}
M_{n}=\sup_{T_{n-1}<t\leq T_{n}}|Z\left(t\right)-Z\left(T_{n-1}\right)|
\end{eqnarray*}
\end{Def}

\begin{Teo}
Sup\'ongase que $n^{-1}T_{n}\rightarrow\mu$ c.s. cuando $n\rightarrow\infty$, donde $\mu\leq\infty$ es una constante o variable aleatoria. Sea $a$ una constante o variable aleatoria que puede ser infinita cuando $\mu$ es finita, y considere las expresiones l\'imite:
\begin{eqnarray}
lim_{n\rightarrow\infty}n^{-1}Z\left(T_{n}\right)&=&a,\textrm{ c.s.}\\
lim_{t\rightarrow\infty}t^{-1}Z\left(t\right)&=&a/\mu,\textrm{ c.s.}
\end{eqnarray}
La segunda expresi\'on implica la primera. Conversamente, la primera implica la segunda si el proceso $Z\left(t\right)$ es creciente, o si $lim_{n\rightarrow\infty}n^{-1}M_{n}=0$ c.s.
\end{Teo}

\begin{Coro}
Si $N\left(t\right)$ es un proceso de renovaci\'on, y $\left(Z\left(T_{n}\right)-Z\left(T_{n-1}\right),M_{n}\right)$, para $n\geq1$, son variables aleatorias independientes e id\'enticamente distribuidas con media finita, entonces,
\begin{eqnarray}
lim_{t\rightarrow\infty}t^{-1}Z\left(t\right)\rightarrow\frac{\esp\left[Z\left(T_{1}\right)-Z\left(T_{0}\right)\right]}{\esp\left[T_{1}\right]},\textrm{ c.s. cuando  }t\rightarrow\infty.
\end{eqnarray}
\end{Coro}

%___________________________________________________________________________________________
%
%\subsection{Propiedades de los Procesos de Renovaci\'on}
%___________________________________________________________________________________________
%

Los tiempos $T_{n}$ est\'an relacionados con los conteos de $N\left(t\right)$ por

\begin{eqnarray*}
\left\{N\left(t\right)\geq n\right\}&=&\left\{T_{n}\leq t\right\}\\
T_{N\left(t\right)}\leq &t&<T_{N\left(t\right)+1},
\end{eqnarray*}

adem\'as $N\left(T_{n}\right)=n$, y 

\begin{eqnarray*}
N\left(t\right)=\max\left\{n:T_{n}\leq t\right\}=\min\left\{n:T_{n+1}>t\right\}
\end{eqnarray*}

Por propiedades de la convoluci\'on se sabe que

\begin{eqnarray*}
P\left\{T_{n}\leq t\right\}=F^{n\star}\left(t\right)
\end{eqnarray*}
que es la $n$-\'esima convoluci\'on de $F$. Entonces 

\begin{eqnarray*}
\left\{N\left(t\right)\geq n\right\}&=&\left\{T_{n}\leq t\right\}\\
P\left\{N\left(t\right)\leq n\right\}&=&1-F^{\left(n+1\right)\star}\left(t\right)
\end{eqnarray*}

Adem\'as usando el hecho de que $\esp\left[N\left(t\right)\right]=\sum_{n=1}^{\infty}P\left\{N\left(t\right)\geq n\right\}$
se tiene que

\begin{eqnarray*}
\esp\left[N\left(t\right)\right]=\sum_{n=1}^{\infty}F^{n\star}\left(t\right)
\end{eqnarray*}

\begin{Prop}
Para cada $t\geq0$, la funci\'on generadora de momentos $\esp\left[e^{\alpha N\left(t\right)}\right]$ existe para alguna $\alpha$ en una vecindad del 0, y de aqu\'i que $\esp\left[N\left(t\right)^{m}\right]<\infty$, para $m\geq1$.
\end{Prop}


\begin{Note}
Si el primer tiempo de renovaci\'on $\xi_{1}$ no tiene la misma distribuci\'on que el resto de las $\xi_{n}$, para $n\geq2$, a $N\left(t\right)$ se le llama Proceso de Renovaci\'on retardado, donde si $\xi$ tiene distribuci\'on $G$, entonces el tiempo $T_{n}$ de la $n$-\'esima renovaci\'on tiene distribuci\'on $G\star F^{\left(n-1\right)\star}\left(t\right)$
\end{Note}


\begin{Teo}
Para una constante $\mu\leq\infty$ ( o variable aleatoria), las siguientes expresiones son equivalentes:

\begin{eqnarray}
lim_{n\rightarrow\infty}n^{-1}T_{n}&=&\mu,\textrm{ c.s.}\\
lim_{t\rightarrow\infty}t^{-1}N\left(t\right)&=&1/\mu,\textrm{ c.s.}
\end{eqnarray}
\end{Teo}


Es decir, $T_{n}$ satisface la Ley Fuerte de los Grandes N\'umeros s\'i y s\'olo s\'i $N\left/t\right)$ la cumple.


\begin{Coro}[Ley Fuerte de los Grandes N\'umeros para Procesos de Renovaci\'on]
Si $N\left(t\right)$ es un proceso de renovaci\'on cuyos tiempos de inter-renovaci\'on tienen media $\mu\leq\infty$, entonces
\begin{eqnarray}
t^{-1}N\left(t\right)\rightarrow 1/\mu,\textrm{ c.s. cuando }t\rightarrow\infty.
\end{eqnarray}

\end{Coro}


Considerar el proceso estoc\'astico de valores reales $\left\{Z\left(t\right):t\geq0\right\}$ en el mismo espacio de probabilidad que $N\left(t\right)$

\begin{Def}
Para el proceso $\left\{Z\left(t\right):t\geq0\right\}$ se define la fluctuaci\'on m\'axima de $Z\left(t\right)$ en el intervalo $\left(T_{n-1},T_{n}\right]$:
\begin{eqnarray*}
M_{n}=\sup_{T_{n-1}<t\leq T_{n}}|Z\left(t\right)-Z\left(T_{n-1}\right)|
\end{eqnarray*}
\end{Def}

\begin{Teo}
Sup\'ongase que $n^{-1}T_{n}\rightarrow\mu$ c.s. cuando $n\rightarrow\infty$, donde $\mu\leq\infty$ es una constante o variable aleatoria. Sea $a$ una constante o variable aleatoria que puede ser infinita cuando $\mu$ es finita, y considere las expresiones l\'imite:
\begin{eqnarray}
lim_{n\rightarrow\infty}n^{-1}Z\left(T_{n}\right)&=&a,\textrm{ c.s.}\\
lim_{t\rightarrow\infty}t^{-1}Z\left(t\right)&=&a/\mu,\textrm{ c.s.}
\end{eqnarray}
La segunda expresi\'on implica la primera. Conversamente, la primera implica la segunda si el proceso $Z\left(t\right)$ es creciente, o si $lim_{n\rightarrow\infty}n^{-1}M_{n}=0$ c.s.
\end{Teo}

\begin{Coro}
Si $N\left(t\right)$ es un proceso de renovaci\'on, y $\left(Z\left(T_{n}\right)-Z\left(T_{n-1}\right),M_{n}\right)$, para $n\geq1$, son variables aleatorias independientes e id\'enticamente distribuidas con media finita, entonces,
\begin{eqnarray}
lim_{t\rightarrow\infty}t^{-1}Z\left(t\right)\rightarrow\frac{\esp\left[Z\left(T_{1}\right)-Z\left(T_{0}\right)\right]}{\esp\left[T_{1}\right]},\textrm{ c.s. cuando  }t\rightarrow\infty.
\end{eqnarray}
\end{Coro}



%___________________________________________________________________________________________
%
\subsection{Propiedades de los Procesos de Renovaci\'on}
%___________________________________________________________________________________________
%

Los tiempos $T_{n}$ est\'an relacionados con los conteos de $N\left(t\right)$ por

\begin{eqnarray*}
\left\{N\left(t\right)\geq n\right\}&=&\left\{T_{n}\leq t\right\}\\
T_{N\left(t\right)}\leq &t&<T_{N\left(t\right)+1},
\end{eqnarray*}

adem\'as $N\left(T_{n}\right)=n$, y 

\begin{eqnarray*}
N\left(t\right)=\max\left\{n:T_{n}\leq t\right\}=\min\left\{n:T_{n+1}>t\right\}
\end{eqnarray*}

Por propiedades de la convoluci\'on se sabe que

\begin{eqnarray*}
P\left\{T_{n}\leq t\right\}=F^{n\star}\left(t\right)
\end{eqnarray*}
que es la $n$-\'esima convoluci\'on de $F$. Entonces 

\begin{eqnarray*}
\left\{N\left(t\right)\geq n\right\}&=&\left\{T_{n}\leq t\right\}\\
P\left\{N\left(t\right)\leq n\right\}&=&1-F^{\left(n+1\right)\star}\left(t\right)
\end{eqnarray*}

Adem\'as usando el hecho de que $\esp\left[N\left(t\right)\right]=\sum_{n=1}^{\infty}P\left\{N\left(t\right)\geq n\right\}$
se tiene que

\begin{eqnarray*}
\esp\left[N\left(t\right)\right]=\sum_{n=1}^{\infty}F^{n\star}\left(t\right)
\end{eqnarray*}

\begin{Prop}
Para cada $t\geq0$, la funci\'on generadora de momentos $\esp\left[e^{\alpha N\left(t\right)}\right]$ existe para alguna $\alpha$ en una vecindad del 0, y de aqu\'i que $\esp\left[N\left(t\right)^{m}\right]<\infty$, para $m\geq1$.
\end{Prop}


\begin{Note}
Si el primer tiempo de renovaci\'on $\xi_{1}$ no tiene la misma distribuci\'on que el resto de las $\xi_{n}$, para $n\geq2$, a $N\left(t\right)$ se le llama Proceso de Renovaci\'on retardado, donde si $\xi$ tiene distribuci\'on $G$, entonces el tiempo $T_{n}$ de la $n$-\'esima renovaci\'on tiene distribuci\'on $G\star F^{\left(n-1\right)\star}\left(t\right)$
\end{Note}


\begin{Teo}
Para una constante $\mu\leq\infty$ ( o variable aleatoria), las siguientes expresiones son equivalentes:

\begin{eqnarray}
lim_{n\rightarrow\infty}n^{-1}T_{n}&=&\mu,\textrm{ c.s.}\\
lim_{t\rightarrow\infty}t^{-1}N\left(t\right)&=&1/\mu,\textrm{ c.s.}
\end{eqnarray}
\end{Teo}


Es decir, $T_{n}$ satisface la Ley Fuerte de los Grandes N\'umeros s\'i y s\'olo s\'i $N\left/t\right)$ la cumple.


\begin{Coro}[Ley Fuerte de los Grandes N\'umeros para Procesos de Renovaci\'on]
Si $N\left(t\right)$ es un proceso de renovaci\'on cuyos tiempos de inter-renovaci\'on tienen media $\mu\leq\infty$, entonces
\begin{eqnarray}
t^{-1}N\left(t\right)\rightarrow 1/\mu,\textrm{ c.s. cuando }t\rightarrow\infty.
\end{eqnarray}

\end{Coro}


Considerar el proceso estoc\'astico de valores reales $\left\{Z\left(t\right):t\geq0\right\}$ en el mismo espacio de probabilidad que $N\left(t\right)$

\begin{Def}
Para el proceso $\left\{Z\left(t\right):t\geq0\right\}$ se define la fluctuaci\'on m\'axima de $Z\left(t\right)$ en el intervalo $\left(T_{n-1},T_{n}\right]$:
\begin{eqnarray*}
M_{n}=\sup_{T_{n-1}<t\leq T_{n}}|Z\left(t\right)-Z\left(T_{n-1}\right)|
\end{eqnarray*}
\end{Def}

\begin{Teo}
Sup\'ongase que $n^{-1}T_{n}\rightarrow\mu$ c.s. cuando $n\rightarrow\infty$, donde $\mu\leq\infty$ es una constante o variable aleatoria. Sea $a$ una constante o variable aleatoria que puede ser infinita cuando $\mu$ es finita, y considere las expresiones l\'imite:
\begin{eqnarray}
lim_{n\rightarrow\infty}n^{-1}Z\left(T_{n}\right)&=&a,\textrm{ c.s.}\\
lim_{t\rightarrow\infty}t^{-1}Z\left(t\right)&=&a/\mu,\textrm{ c.s.}
\end{eqnarray}
La segunda expresi\'on implica la primera. Conversamente, la primera implica la segunda si el proceso $Z\left(t\right)$ es creciente, o si $lim_{n\rightarrow\infty}n^{-1}M_{n}=0$ c.s.
\end{Teo}

\begin{Coro}
Si $N\left(t\right)$ es un proceso de renovaci\'on, y $\left(Z\left(T_{n}\right)-Z\left(T_{n-1}\right),M_{n}\right)$, para $n\geq1$, son variables aleatorias independientes e id\'enticamente distribuidas con media finita, entonces,
\begin{eqnarray}
lim_{t\rightarrow\infty}t^{-1}Z\left(t\right)\rightarrow\frac{\esp\left[Z\left(T_{1}\right)-Z\left(T_{0}\right)\right]}{\esp\left[T_{1}\right]},\textrm{ c.s. cuando  }t\rightarrow\infty.
\end{eqnarray}
\end{Coro}




%__________________________________________________________________________________________
\subsection{Procesos Regenerativos Estacionarios - Stidham \cite{Stidham}}
%__________________________________________________________________________________________


Un proceso estoc\'astico a tiempo continuo $\left\{V\left(t\right),t\geq0\right\}$ es un proceso regenerativo si existe una sucesi\'on de variables aleatorias independientes e id\'enticamente distribuidas $\left\{X_{1},X_{2},\ldots\right\}$, sucesi\'on de renovaci\'on, tal que para cualquier conjunto de Borel $A$, 

\begin{eqnarray*}
\prob\left\{V\left(t\right)\in A|X_{1}+X_{2}+\cdots+X_{R\left(t\right)}=s,\left\{V\left(\tau\right),\tau<s\right\}\right\}=\prob\left\{V\left(t-s\right)\in A|X_{1}>t-s\right\},
\end{eqnarray*}
para todo $0\leq s\leq t$, donde $R\left(t\right)=\max\left\{X_{1}+X_{2}+\cdots+X_{j}\leq t\right\}=$n\'umero de renovaciones ({\emph{puntos de regeneraci\'on}}) que ocurren en $\left[0,t\right]$. El intervalo $\left[0,X_{1}\right)$ es llamado {\emph{primer ciclo de regeneraci\'on}} de $\left\{V\left(t \right),t\geq0\right\}$, $\left[X_{1},X_{1}+X_{2}\right)$ el {\emph{segundo ciclo de regeneraci\'on}}, y as\'i sucesivamente.

Sea $X=X_{1}$ y sea $F$ la funci\'on de distrbuci\'on de $X$


\begin{Def}
Se define el proceso estacionario, $\left\{V^{*}\left(t\right),t\geq0\right\}$, para $\left\{V\left(t\right),t\geq0\right\}$ por

\begin{eqnarray*}
\prob\left\{V\left(t\right)\in A\right\}=\frac{1}{\esp\left[X\right]}\int_{0}^{\infty}\prob\left\{V\left(t+x\right)\in A|X>x\right\}\left(1-F\left(x\right)\right)dx,
\end{eqnarray*} 
para todo $t\geq0$ y todo conjunto de Borel $A$.
\end{Def}

\begin{Def}
Una distribuci\'on se dice que es {\emph{aritm\'etica}} si todos sus puntos de incremento son m\'ultiplos de la forma $0,\lambda, 2\lambda,\ldots$ para alguna $\lambda>0$ entera.
\end{Def}


\begin{Def}
Una modificaci\'on medible de un proceso $\left\{V\left(t\right),t\geq0\right\}$, es una versi\'on de este, $\left\{V\left(t,w\right)\right\}$ conjuntamente medible para $t\geq0$ y para $w\in S$, $S$ espacio de estados para $\left\{V\left(t\right),t\geq0\right\}$.
\end{Def}

\begin{Teo}
Sea $\left\{V\left(t\right),t\geq\right\}$ un proceso regenerativo no negativo con modificaci\'on medible. Sea $\esp\left[X\right]<\infty$. Entonces el proceso estacionario dado por la ecuaci\'on anterior est\'a bien definido y tiene funci\'on de distribuci\'on independiente de $t$, adem\'as
\begin{itemize}
\item[i)] \begin{eqnarray*}
\esp\left[V^{*}\left(0\right)\right]&=&\frac{\esp\left[\int_{0}^{X}V\left(s\right)ds\right]}{\esp\left[X\right]}\end{eqnarray*}
\item[ii)] Si $\esp\left[V^{*}\left(0\right)\right]<\infty$, equivalentemente, si $\esp\left[\int_{0}^{X}V\left(s\right)ds\right]<\infty$,entonces
\begin{eqnarray*}
\frac{\int_{0}^{t}V\left(s\right)ds}{t}\rightarrow\frac{\esp\left[\int_{0}^{X}V\left(s\right)ds\right]}{\esp\left[X\right]}
\end{eqnarray*}
con probabilidad 1 y en media, cuando $t\rightarrow\infty$.
\end{itemize}
\end{Teo}

%______________________________________________________________________
\subsection{Procesos de Renovaci\'on}
%______________________________________________________________________

\begin{Def}\label{Def.Tn}
Sean $0\leq T_{1}\leq T_{2}\leq \ldots$ son tiempos aleatorios infinitos en los cuales ocurren ciertos eventos. El n\'umero de tiempos $T_{n}$ en el intervalo $\left[0,t\right)$ es

\begin{eqnarray}
N\left(t\right)=\sum_{n=1}^{\infty}\indora\left(T_{n}\leq t\right),
\end{eqnarray}
para $t\geq0$.
\end{Def}

Si se consideran los puntos $T_{n}$ como elementos de $\rea_{+}$, y $N\left(t\right)$ es el n\'umero de puntos en $\rea$. El proceso denotado por $\left\{N\left(t\right):t\geq0\right\}$, denotado por $N\left(t\right)$, es un proceso puntual en $\rea_{+}$. Los $T_{n}$ son los tiempos de ocurrencia, el proceso puntual $N\left(t\right)$ es simple si su n\'umero de ocurrencias son distintas: $0<T_{1}<T_{2}<\ldots$ casi seguramente.

\begin{Def}
Un proceso puntual $N\left(t\right)$ es un proceso de renovaci\'on si los tiempos de interocurrencia $\xi_{n}=T_{n}-T_{n-1}$, para $n\geq1$, son independientes e identicamente distribuidos con distribuci\'on $F$, donde $F\left(0\right)=0$ y $T_{0}=0$. Los $T_{n}$ son llamados tiempos de renovaci\'on, referente a la independencia o renovaci\'on de la informaci\'on estoc\'astica en estos tiempos. Los $\xi_{n}$ son los tiempos de inter-renovaci\'on, y $N\left(t\right)$ es el n\'umero de renovaciones en el intervalo $\left[0,t\right)$
\end{Def}


\begin{Note}
Para definir un proceso de renovaci\'on para cualquier contexto, solamente hay que especificar una distribuci\'on $F$, con $F\left(0\right)=0$, para los tiempos de inter-renovaci\'on. La funci\'on $F$ en turno degune las otra variables aleatorias. De manera formal, existe un espacio de probabilidad y una sucesi\'on de variables aleatorias $\xi_{1},\xi_{2},\ldots$ definidas en este con distribuci\'on $F$. Entonces las otras cantidades son $T_{n}=\sum_{k=1}^{n}\xi_{k}$ y $N\left(t\right)=\sum_{n=1}^{\infty}\indora\left(T_{n}\leq t\right)$, donde $T_{n}\rightarrow\infty$ casi seguramente por la Ley Fuerte de los Grandes Números.
\end{Note}

%___________________________________________________________________________________________
%
\subsection{Teorema Principal de Renovaci\'on}
%___________________________________________________________________________________________
%

\begin{Note} Una funci\'on $h:\rea_{+}\rightarrow\rea$ es Directamente Riemann Integrable en los siguientes casos:
\begin{itemize}
\item[a)] $h\left(t\right)\geq0$ es decreciente y Riemann Integrable.
\item[b)] $h$ es continua excepto posiblemente en un conjunto de Lebesgue de medida 0, y $|h\left(t\right)|\leq b\left(t\right)$, donde $b$ es DRI.
\end{itemize}
\end{Note}

\begin{Teo}[Teorema Principal de Renovaci\'on]
Si $F$ es no aritm\'etica y $h\left(t\right)$ es Directamente Riemann Integrable (DRI), entonces

\begin{eqnarray*}
lim_{t\rightarrow\infty}U\star h=\frac{1}{\mu}\int_{\rea_{+}}h\left(s\right)ds.
\end{eqnarray*}
\end{Teo}

\begin{Prop}
Cualquier funci\'on $H\left(t\right)$ acotada en intervalos finitos y que es 0 para $t<0$ puede expresarse como
\begin{eqnarray*}
H\left(t\right)=U\star h\left(t\right)\textrm{,  donde }h\left(t\right)=H\left(t\right)-F\star H\left(t\right)
\end{eqnarray*}
\end{Prop}

\begin{Def}
Un proceso estoc\'astico $X\left(t\right)$ es crudamente regenerativo en un tiempo aleatorio positivo $T$ si
\begin{eqnarray*}
\esp\left[X\left(T+t\right)|T\right]=\esp\left[X\left(t\right)\right]\textrm{, para }t\geq0,\end{eqnarray*}
y con las esperanzas anteriores finitas.
\end{Def}

\begin{Prop}
Sup\'ongase que $X\left(t\right)$ es un proceso crudamente regenerativo en $T$, que tiene distribuci\'on $F$. Si $\esp\left[X\left(t\right)\right]$ es acotado en intervalos finitos, entonces
\begin{eqnarray*}
\esp\left[X\left(t\right)\right]=U\star h\left(t\right)\textrm{,  donde }h\left(t\right)=\esp\left[X\left(t\right)\indora\left(T>t\right)\right].
\end{eqnarray*}
\end{Prop}

\begin{Teo}[Regeneraci\'on Cruda]
Sup\'ongase que $X\left(t\right)$ es un proceso con valores positivo crudamente regenerativo en $T$, y def\'inase $M=\sup\left\{|X\left(t\right)|:t\leq T\right\}$. Si $T$ es no aritm\'etico y $M$ y $MT$ tienen media finita, entonces
\begin{eqnarray*}
lim_{t\rightarrow\infty}\esp\left[X\left(t\right)\right]=\frac{1}{\mu}\int_{\rea_{+}}h\left(s\right)ds,
\end{eqnarray*}
donde $h\left(t\right)=\esp\left[X\left(t\right)\indora\left(T>t\right)\right]$.
\end{Teo}



%___________________________________________________________________________________________
%
\subsection{Funci\'on de Renovaci\'on}
%___________________________________________________________________________________________
%


\begin{Def}
Sea $h\left(t\right)$ funci\'on de valores reales en $\rea$ acotada en intervalos finitos e igual a cero para $t<0$ La ecuaci\'on de renovaci\'on para $h\left(t\right)$ y la distribuci\'on $F$ es

\begin{eqnarray}\label{Ec.Renovacion}
H\left(t\right)=h\left(t\right)+\int_{\left[0,t\right]}H\left(t-s\right)dF\left(s\right)\textrm{,    }t\geq0,
\end{eqnarray}
donde $H\left(t\right)$ es una funci\'on de valores reales. Esto es $H=h+F\star H$. Decimos que $H\left(t\right)$ es soluci\'on de esta ecuaci\'on si satisface la ecuaci\'on, y es acotada en intervalos finitos e iguales a cero para $t<0$.
\end{Def}

\begin{Prop}
La funci\'on $U\star h\left(t\right)$ es la \'unica soluci\'on de la ecuaci\'on de renovaci\'on (\ref{Ec.Renovacion}).
\end{Prop}

\begin{Teo}[Teorema Renovaci\'on Elemental]
\begin{eqnarray*}
t^{-1}U\left(t\right)\rightarrow 1/\mu\textrm{,    cuando }t\rightarrow\infty.
\end{eqnarray*}
\end{Teo}

%___________________________________________________________________________________________
%
\subsection{Propiedades de los Procesos de Renovaci\'on}
%___________________________________________________________________________________________
%

Los tiempos $T_{n}$ est\'an relacionados con los conteos de $N\left(t\right)$ por

\begin{eqnarray*}
\left\{N\left(t\right)\geq n\right\}&=&\left\{T_{n}\leq t\right\}\\
T_{N\left(t\right)}\leq &t&<T_{N\left(t\right)+1},
\end{eqnarray*}

adem\'as $N\left(T_{n}\right)=n$, y 

\begin{eqnarray*}
N\left(t\right)=\max\left\{n:T_{n}\leq t\right\}=\min\left\{n:T_{n+1}>t\right\}
\end{eqnarray*}

Por propiedades de la convoluci\'on se sabe que

\begin{eqnarray*}
P\left\{T_{n}\leq t\right\}=F^{n\star}\left(t\right)
\end{eqnarray*}
que es la $n$-\'esima convoluci\'on de $F$. Entonces 

\begin{eqnarray*}
\left\{N\left(t\right)\geq n\right\}&=&\left\{T_{n}\leq t\right\}\\
P\left\{N\left(t\right)\leq n\right\}&=&1-F^{\left(n+1\right)\star}\left(t\right)
\end{eqnarray*}

Adem\'as usando el hecho de que $\esp\left[N\left(t\right)\right]=\sum_{n=1}^{\infty}P\left\{N\left(t\right)\geq n\right\}$
se tiene que

\begin{eqnarray*}
\esp\left[N\left(t\right)\right]=\sum_{n=1}^{\infty}F^{n\star}\left(t\right)
\end{eqnarray*}

\begin{Prop}
Para cada $t\geq0$, la funci\'on generadora de momentos $\esp\left[e^{\alpha N\left(t\right)}\right]$ existe para alguna $\alpha$ en una vecindad del 0, y de aqu\'i que $\esp\left[N\left(t\right)^{m}\right]<\infty$, para $m\geq1$.
\end{Prop}


\begin{Note}
Si el primer tiempo de renovaci\'on $\xi_{1}$ no tiene la misma distribuci\'on que el resto de las $\xi_{n}$, para $n\geq2$, a $N\left(t\right)$ se le llama Proceso de Renovaci\'on retardado, donde si $\xi$ tiene distribuci\'on $G$, entonces el tiempo $T_{n}$ de la $n$-\'esima renovaci\'on tiene distribuci\'on $G\star F^{\left(n-1\right)\star}\left(t\right)$
\end{Note}


\begin{Teo}
Para una constante $\mu\leq\infty$ ( o variable aleatoria), las siguientes expresiones son equivalentes:

\begin{eqnarray}
lim_{n\rightarrow\infty}n^{-1}T_{n}&=&\mu,\textrm{ c.s.}\\
lim_{t\rightarrow\infty}t^{-1}N\left(t\right)&=&1/\mu,\textrm{ c.s.}
\end{eqnarray}
\end{Teo}


Es decir, $T_{n}$ satisface la Ley Fuerte de los Grandes N\'umeros s\'i y s\'olo s\'i $N\left/t\right)$ la cumple.


\begin{Coro}[Ley Fuerte de los Grandes N\'umeros para Procesos de Renovaci\'on]
Si $N\left(t\right)$ es un proceso de renovaci\'on cuyos tiempos de inter-renovaci\'on tienen media $\mu\leq\infty$, entonces
\begin{eqnarray}
t^{-1}N\left(t\right)\rightarrow 1/\mu,\textrm{ c.s. cuando }t\rightarrow\infty.
\end{eqnarray}

\end{Coro}


Considerar el proceso estoc\'astico de valores reales $\left\{Z\left(t\right):t\geq0\right\}$ en el mismo espacio de probabilidad que $N\left(t\right)$

\begin{Def}
Para el proceso $\left\{Z\left(t\right):t\geq0\right\}$ se define la fluctuaci\'on m\'axima de $Z\left(t\right)$ en el intervalo $\left(T_{n-1},T_{n}\right]$:
\begin{eqnarray*}
M_{n}=\sup_{T_{n-1}<t\leq T_{n}}|Z\left(t\right)-Z\left(T_{n-1}\right)|
\end{eqnarray*}
\end{Def}

\begin{Teo}
Sup\'ongase que $n^{-1}T_{n}\rightarrow\mu$ c.s. cuando $n\rightarrow\infty$, donde $\mu\leq\infty$ es una constante o variable aleatoria. Sea $a$ una constante o variable aleatoria que puede ser infinita cuando $\mu$ es finita, y considere las expresiones l\'imite:
\begin{eqnarray}
lim_{n\rightarrow\infty}n^{-1}Z\left(T_{n}\right)&=&a,\textrm{ c.s.}\\
lim_{t\rightarrow\infty}t^{-1}Z\left(t\right)&=&a/\mu,\textrm{ c.s.}
\end{eqnarray}
La segunda expresi\'on implica la primera. Conversamente, la primera implica la segunda si el proceso $Z\left(t\right)$ es creciente, o si $lim_{n\rightarrow\infty}n^{-1}M_{n}=0$ c.s.
\end{Teo}

\begin{Coro}
Si $N\left(t\right)$ es un proceso de renovaci\'on, y $\left(Z\left(T_{n}\right)-Z\left(T_{n-1}\right),M_{n}\right)$, para $n\geq1$, son variables aleatorias independientes e id\'enticamente distribuidas con media finita, entonces,
\begin{eqnarray}
lim_{t\rightarrow\infty}t^{-1}Z\left(t\right)\rightarrow\frac{\esp\left[Z\left(T_{1}\right)-Z\left(T_{0}\right)\right]}{\esp\left[T_{1}\right]},\textrm{ c.s. cuando  }t\rightarrow\infty.
\end{eqnarray}
\end{Coro}

%___________________________________________________________________________________________
%
\subsection{Funci\'on de Renovaci\'on}
%___________________________________________________________________________________________
%


Sup\'ongase que $N\left(t\right)$ es un proceso de renovaci\'on con distribuci\'on $F$ con media finita $\mu$.

\begin{Def}
La funci\'on de renovaci\'on asociada con la distribuci\'on $F$, del proceso $N\left(t\right)$, es
\begin{eqnarray*}
U\left(t\right)=\sum_{n=1}^{\infty}F^{n\star}\left(t\right),\textrm{   }t\geq0,
\end{eqnarray*}
donde $F^{0\star}\left(t\right)=\indora\left(t\geq0\right)$.
\end{Def}


\begin{Prop}
Sup\'ongase que la distribuci\'on de inter-renovaci\'on $F$ tiene densidad $f$. Entonces $U\left(t\right)$ tambi\'en tiene densidad, para $t>0$, y es $U^{'}\left(t\right)=\sum_{n=0}^{\infty}f^{n\star}\left(t\right)$. Adem\'as
\begin{eqnarray*}
\prob\left\{N\left(t\right)>N\left(t-\right)\right\}=0\textrm{,   }t\geq0.
\end{eqnarray*}
\end{Prop}

\begin{Def}
La Transformada de Laplace-Stieljes de $F$ est\'a dada por

\begin{eqnarray*}
\hat{F}\left(\alpha\right)=\int_{\rea_{+}}e^{-\alpha t}dF\left(t\right)\textrm{,  }\alpha\geq0.
\end{eqnarray*}
\end{Def}

Entonces

\begin{eqnarray*}
\hat{U}\left(\alpha\right)=\sum_{n=0}^{\infty}\hat{F^{n\star}}\left(\alpha\right)=\sum_{n=0}^{\infty}\hat{F}\left(\alpha\right)^{n}=\frac{1}{1-\hat{F}\left(\alpha\right)}.
\end{eqnarray*}


\begin{Prop}
La Transformada de Laplace $\hat{U}\left(\alpha\right)$ y $\hat{F}\left(\alpha\right)$ determina una a la otra de manera \'unica por la relaci\'on $\hat{U}\left(\alpha\right)=\frac{1}{1-\hat{F}\left(\alpha\right)}$.
\end{Prop}


\begin{Note}
Un proceso de renovaci\'on $N\left(t\right)$ cuyos tiempos de inter-renovaci\'on tienen media finita, es un proceso Poisson con tasa $\lambda$ si y s\'olo s\'i $\esp\left[U\left(t\right)\right]=\lambda t$, para $t\geq0$.
\end{Note}


\begin{Teo}
Sea $N\left(t\right)$ un proceso puntual simple con puntos de localizaci\'on $T_{n}$ tal que $\eta\left(t\right)=\esp\left[N\left(\right)\right]$ es finita para cada $t$. Entonces para cualquier funci\'on $f:\rea_{+}\rightarrow\rea$,
\begin{eqnarray*}
\esp\left[\sum_{n=1}^{N\left(\right)}f\left(T_{n}\right)\right]=\int_{\left(0,t\right]}f\left(s\right)d\eta\left(s\right)\textrm{,  }t\geq0,
\end{eqnarray*}
suponiendo que la integral exista. Adem\'as si $X_{1},X_{2},\ldots$ son variables aleatorias definidas en el mismo espacio de probabilidad que el proceso $N\left(t\right)$ tal que $\esp\left[X_{n}|T_{n}=s\right]=f\left(s\right)$, independiente de $n$. Entonces
\begin{eqnarray*}
\esp\left[\sum_{n=1}^{N\left(t\right)}X_{n}\right]=\int_{\left(0,t\right]}f\left(s\right)d\eta\left(s\right)\textrm{,  }t\geq0,
\end{eqnarray*} 
suponiendo que la integral exista. 
\end{Teo}

\begin{Coro}[Identidad de Wald para Renovaciones]
Para el proceso de renovaci\'on $N\left(t\right)$,
\begin{eqnarray*}
\esp\left[T_{N\left(t\right)+1}\right]=\mu\esp\left[N\left(t\right)+1\right]\textrm{,  }t\geq0,
\end{eqnarray*}  
\end{Coro}

%______________________________________________________________________
\subsection{Procesos de Renovaci\'on}
%______________________________________________________________________

\begin{Def}\label{Def.Tn}
Sean $0\leq T_{1}\leq T_{2}\leq \ldots$ son tiempos aleatorios infinitos en los cuales ocurren ciertos eventos. El n\'umero de tiempos $T_{n}$ en el intervalo $\left[0,t\right)$ es

\begin{eqnarray}
N\left(t\right)=\sum_{n=1}^{\infty}\indora\left(T_{n}\leq t\right),
\end{eqnarray}
para $t\geq0$.
\end{Def}

Si se consideran los puntos $T_{n}$ como elementos de $\rea_{+}$, y $N\left(t\right)$ es el n\'umero de puntos en $\rea$. El proceso denotado por $\left\{N\left(t\right):t\geq0\right\}$, denotado por $N\left(t\right)$, es un proceso puntual en $\rea_{+}$. Los $T_{n}$ son los tiempos de ocurrencia, el proceso puntual $N\left(t\right)$ es simple si su n\'umero de ocurrencias son distintas: $0<T_{1}<T_{2}<\ldots$ casi seguramente.

\begin{Def}
Un proceso puntual $N\left(t\right)$ es un proceso de renovaci\'on si los tiempos de interocurrencia $\xi_{n}=T_{n}-T_{n-1}$, para $n\geq1$, son independientes e identicamente distribuidos con distribuci\'on $F$, donde $F\left(0\right)=0$ y $T_{0}=0$. Los $T_{n}$ son llamados tiempos de renovaci\'on, referente a la independencia o renovaci\'on de la informaci\'on estoc\'astica en estos tiempos. Los $\xi_{n}$ son los tiempos de inter-renovaci\'on, y $N\left(t\right)$ es el n\'umero de renovaciones en el intervalo $\left[0,t\right)$
\end{Def}


\begin{Note}
Para definir un proceso de renovaci\'on para cualquier contexto, solamente hay que especificar una distribuci\'on $F$, con $F\left(0\right)=0$, para los tiempos de inter-renovaci\'on. La funci\'on $F$ en turno degune las otra variables aleatorias. De manera formal, existe un espacio de probabilidad y una sucesi\'on de variables aleatorias $\xi_{1},\xi_{2},\ldots$ definidas en este con distribuci\'on $F$. Entonces las otras cantidades son $T_{n}=\sum_{k=1}^{n}\xi_{k}$ y $N\left(t\right)=\sum_{n=1}^{\infty}\indora\left(T_{n}\leq t\right)$, donde $T_{n}\rightarrow\infty$ casi seguramente por la Ley Fuerte de los Grandes Números.
\end{Note}
%_____________________________________________________
\subsection{Puntos de Renovaci\'on}
%_____________________________________________________

Para cada cola $Q_{i}$ se tienen los procesos de arribo a la cola, para estas, los tiempos de arribo est\'an dados por $$\left\{T_{1}^{i},T_{2}^{i},\ldots,T_{k}^{i},\ldots\right\},$$ entonces, consideremos solamente los primeros tiempos de arribo a cada una de las colas, es decir, $$\left\{T_{1}^{1},T_{1}^{2},T_{1}^{3},T_{1}^{4}\right\},$$ se sabe que cada uno de estos tiempos se distribuye de manera exponencial con par\'ametro $1/mu_{i}$. Adem\'as se sabe que para $$T^{*}=\min\left\{T_{1}^{1},T_{1}^{2},T_{1}^{3},T_{1}^{4}\right\},$$ $T^{*}$ se distribuye de manera exponencial con par\'ametro $$\mu^{*}=\sum_{i=1}^{4}\mu_{i}.$$ Ahora, dado que 
\begin{center}
\begin{tabular}{lcl}
$\tilde{r}=r_{1}+r_{2}$ & y &$\hat{r}=r_{3}+r_{4}.$
\end{tabular}
\end{center}


Supongamos que $$\tilde{r},\hat{r}<\mu^{*},$$ entonces si tomamos $$r^{*}=\min\left\{\tilde{r},\hat{r}\right\},$$ se tiene que para  $$t^{*}\in\left(0,r^{*}\right)$$ se cumple que 
\begin{center}
\begin{tabular}{lcl}
$\tau_{1}\left(1\right)=0$ & y por tanto & $\overline{\tau}_{1}=0,$
\end{tabular}
\end{center}
entonces para la segunda cola en este primer ciclo se cumple que $$\tau_{2}=\overline{\tau}_{1}+r_{1}=r_{1}<\mu^{*},$$ y por tanto se tiene que  $$\overline{\tau}_{2}=\tau_{2}.$$ Por lo tanto, nuevamente para la primer cola en el segundo ciclo $$\tau_{1}\left(2\right)=\tau_{2}\left(1\right)+r_{2}=\tilde{r}<\mu^{*}.$$ An\'alogamente para el segundo sistema se tiene que ambas colas est\'an vac\'ias, es decir, existe un valor $t^{*}$ tal que en el intervalo $\left(0,t^{*}\right)$ no ha llegado ning\'un usuario, es decir, $$L_{i}\left(t^{*}\right)=0$$ para $i=1,2,3,4$.




Sup\'ongase que $N\left(t\right)$ es un proceso de renovaci\'on con distribuci\'on $F$ con media finita $\mu$.

\begin{Def}
La funci\'on de renovaci\'on asociada con la distribuci\'on $F$, del proceso $N\left(t\right)$, es
\begin{eqnarray*}
U\left(t\right)=\sum_{n=1}^{\infty}F^{n\star}\left(t\right),\textrm{   }t\geq0,
\end{eqnarray*}
donde $F^{0\star}\left(t\right)=\indora\left(t\geq0\right)$.
\end{Def}


\begin{Prop}
Sup\'ongase que la distribuci\'on de inter-renovaci\'on $F$ tiene densidad $f$. Entonces $U\left(t\right)$ tambi\'en tiene densidad, para $t>0$, y es $U^{'}\left(t\right)=\sum_{n=0}^{\infty}f^{n\star}\left(t\right)$. Adem\'as
\begin{eqnarray*}
\prob\left\{N\left(t\right)>N\left(t-\right)\right\}=0\textrm{,   }t\geq0.
\end{eqnarray*}
\end{Prop}

\begin{Def}
La Transformada de Laplace-Stieljes de $F$ est\'a dada por

\begin{eqnarray*}
\hat{F}\left(\alpha\right)=\int_{\rea_{+}}e^{-\alpha t}dF\left(t\right)\textrm{,  }\alpha\geq0.
\end{eqnarray*}
\end{Def}

Entonces

\begin{eqnarray*}
\hat{U}\left(\alpha\right)=\sum_{n=0}^{\infty}\hat{F^{n\star}}\left(\alpha\right)=\sum_{n=0}^{\infty}\hat{F}\left(\alpha\right)^{n}=\frac{1}{1-\hat{F}\left(\alpha\right)}.
\end{eqnarray*}


\begin{Prop}
La Transformada de Laplace $\hat{U}\left(\alpha\right)$ y $\hat{F}\left(\alpha\right)$ determina una a la otra de manera \'unica por la relaci\'on $\hat{U}\left(\alpha\right)=\frac{1}{1-\hat{F}\left(\alpha\right)}$.
\end{Prop}


\begin{Note}
Un proceso de renovaci\'on $N\left(t\right)$ cuyos tiempos de inter-renovaci\'on tienen media finita, es un proceso Poisson con tasa $\lambda$ si y s\'olo s\'i $\esp\left[U\left(t\right)\right]=\lambda t$, para $t\geq0$.
\end{Note}


\begin{Teo}
Sea $N\left(t\right)$ un proceso puntual simple con puntos de localizaci\'on $T_{n}$ tal que $\eta\left(t\right)=\esp\left[N\left(\right)\right]$ es finita para cada $t$. Entonces para cualquier funci\'on $f:\rea_{+}\rightarrow\rea$,
\begin{eqnarray*}
\esp\left[\sum_{n=1}^{N\left(\right)}f\left(T_{n}\right)\right]=\int_{\left(0,t\right]}f\left(s\right)d\eta\left(s\right)\textrm{,  }t\geq0,
\end{eqnarray*}
suponiendo que la integral exista. Adem\'as si $X_{1},X_{2},\ldots$ son variables aleatorias definidas en el mismo espacio de probabilidad que el proceso $N\left(t\right)$ tal que $\esp\left[X_{n}|T_{n}=s\right]=f\left(s\right)$, independiente de $n$. Entonces
\begin{eqnarray*}
\esp\left[\sum_{n=1}^{N\left(t\right)}X_{n}\right]=\int_{\left(0,t\right]}f\left(s\right)d\eta\left(s\right)\textrm{,  }t\geq0,
\end{eqnarray*} 
suponiendo que la integral exista. 
\end{Teo}

\begin{Coro}[Identidad de Wald para Renovaciones]
Para el proceso de renovaci\'on $N\left(t\right)$,
\begin{eqnarray*}
\esp\left[T_{N\left(t\right)+1}\right]=\mu\esp\left[N\left(t\right)+1\right]\textrm{,  }t\geq0,
\end{eqnarray*}  
\end{Coro}

%___________________________________________________________________________________________
%
\subsection{Funci\'on de Renovaci\'on}
%___________________________________________________________________________________________
%


Sup\'ongase que $N\left(t\right)$ es un proceso de renovaci\'on con distribuci\'on $F$ con media finita $\mu$.

\begin{Def}
La funci\'on de renovaci\'on asociada con la distribuci\'on $F$, del proceso $N\left(t\right)$, es
\begin{eqnarray*}
U\left(t\right)=\sum_{n=1}^{\infty}F^{n\star}\left(t\right),\textrm{   }t\geq0,
\end{eqnarray*}
donde $F^{0\star}\left(t\right)=\indora\left(t\geq0\right)$.
\end{Def}


\begin{Prop}
Sup\'ongase que la distribuci\'on de inter-renovaci\'on $F$ tiene densidad $f$. Entonces $U\left(t\right)$ tambi\'en tiene densidad, para $t>0$, y es $U^{'}\left(t\right)=\sum_{n=0}^{\infty}f^{n\star}\left(t\right)$. Adem\'as
\begin{eqnarray*}
\prob\left\{N\left(t\right)>N\left(t-\right)\right\}=0\textrm{,   }t\geq0.
\end{eqnarray*}
\end{Prop}

\begin{Def}
La Transformada de Laplace-Stieljes de $F$ est\'a dada por

\begin{eqnarray*}
\hat{F}\left(\alpha\right)=\int_{\rea_{+}}e^{-\alpha t}dF\left(t\right)\textrm{,  }\alpha\geq0.
\end{eqnarray*}
\end{Def}

Entonces

\begin{eqnarray*}
\hat{U}\left(\alpha\right)=\sum_{n=0}^{\infty}\hat{F^{n\star}}\left(\alpha\right)=\sum_{n=0}^{\infty}\hat{F}\left(\alpha\right)^{n}=\frac{1}{1-\hat{F}\left(\alpha\right)}.
\end{eqnarray*}


\begin{Prop}
La Transformada de Laplace $\hat{U}\left(\alpha\right)$ y $\hat{F}\left(\alpha\right)$ determina una a la otra de manera \'unica por la relaci\'on $\hat{U}\left(\alpha\right)=\frac{1}{1-\hat{F}\left(\alpha\right)}$.
\end{Prop}


\begin{Note}
Un proceso de renovaci\'on $N\left(t\right)$ cuyos tiempos de inter-renovaci\'on tienen media finita, es un proceso Poisson con tasa $\lambda$ si y s\'olo s\'i $\esp\left[U\left(t\right)\right]=\lambda t$, para $t\geq0$.
\end{Note}


\begin{Teo}
Sea $N\left(t\right)$ un proceso puntual simple con puntos de localizaci\'on $T_{n}$ tal que $\eta\left(t\right)=\esp\left[N\left(\right)\right]$ es finita para cada $t$. Entonces para cualquier funci\'on $f:\rea_{+}\rightarrow\rea$,
\begin{eqnarray*}
\esp\left[\sum_{n=1}^{N\left(\right)}f\left(T_{n}\right)\right]=\int_{\left(0,t\right]}f\left(s\right)d\eta\left(s\right)\textrm{,  }t\geq0,
\end{eqnarray*}
suponiendo que la integral exista. Adem\'as si $X_{1},X_{2},\ldots$ son variables aleatorias definidas en el mismo espacio de probabilidad que el proceso $N\left(t\right)$ tal que $\esp\left[X_{n}|T_{n}=s\right]=f\left(s\right)$, independiente de $n$. Entonces
\begin{eqnarray*}
\esp\left[\sum_{n=1}^{N\left(t\right)}X_{n}\right]=\int_{\left(0,t\right]}f\left(s\right)d\eta\left(s\right)\textrm{,  }t\geq0,
\end{eqnarray*}
suponiendo que la integral exista.
\end{Teo}

\begin{Coro}[Identidad de Wald para Renovaciones]
Para el proceso de renovaci\'on $N\left(t\right)$,
\begin{eqnarray*}
\esp\left[T_{N\left(t\right)+1}\right]=\mu\esp\left[N\left(t\right)+1\right]\textrm{,  }t\geq0,
\end{eqnarray*}
\end{Coro}
%___________________________________________________________________________________________
%
\subsection{Funci\'on de Renovaci\'on}
%___________________________________________________________________________________________
%


\begin{Def}
Sea $h\left(t\right)$ funci\'on de valores reales en $\rea$ acotada en intervalos finitos e igual a cero para $t<0$ La ecuaci\'on de renovaci\'on para $h\left(t\right)$ y la distribuci\'on $F$ es

\begin{eqnarray}\label{Ec.Renovacion}
H\left(t\right)=h\left(t\right)+\int_{\left[0,t\right]}H\left(t-s\right)dF\left(s\right)\textrm{,    }t\geq0,
\end{eqnarray}
donde $H\left(t\right)$ es una funci\'on de valores reales. Esto es $H=h+F\star H$. Decimos que $H\left(t\right)$ es soluci\'on de esta ecuaci\'on si satisface la ecuaci\'on, y es acotada en intervalos finitos e iguales a cero para $t<0$.
\end{Def}

\begin{Prop}
La funci\'on $U\star h\left(t\right)$ es la \'unica soluci\'on de la ecuaci\'on de renovaci\'on (\ref{Ec.Renovacion}).
\end{Prop}

\begin{Teo}[Teorema Renovaci\'on Elemental]
\begin{eqnarray*}
t^{-1}U\left(t\right)\rightarrow 1/\mu\textrm{,    cuando }t\rightarrow\infty.
\end{eqnarray*}
\end{Teo}


%___________________________________________________________________________________________
%
\subsection{Funci\'on de Renovaci\'on}
%___________________________________________________________________________________________
%


\begin{Def}
Sea $h\left(t\right)$ funci\'on de valores reales en $\rea$ acotada en intervalos finitos e igual a cero para $t<0$ La ecuaci\'on de renovaci\'on para $h\left(t\right)$ y la distribuci\'on $F$ es

\begin{eqnarray}\label{Ec.Renovacion}
H\left(t\right)=h\left(t\right)+\int_{\left[0,t\right]}H\left(t-s\right)dF\left(s\right)\textrm{,    }t\geq0,
\end{eqnarray}
donde $H\left(t\right)$ es una funci\'on de valores reales. Esto es $H=h+F\star H$. Decimos que $H\left(t\right)$ es soluci\'on de esta ecuaci\'on si satisface la ecuaci\'on, y es acotada en intervalos finitos e iguales a cero para $t<0$.
\end{Def}

\begin{Prop}
La funci\'on $U\star h\left(t\right)$ es la \'unica soluci\'on de la ecuaci\'on de renovaci\'on (\ref{Ec.Renovacion}).
\end{Prop}

\begin{Teo}[Teorema Renovaci\'on Elemental]
\begin{eqnarray*}
t^{-1}U\left(t\right)\rightarrow 1/\mu\textrm{,    cuando }t\rightarrow\infty.
\end{eqnarray*}
\end{Teo}


%___________________________________________________________________________________________
%
\subsection{Funci\'on de Renovaci\'on}
%___________________________________________________________________________________________
%


\begin{Def}
Sea $h\left(t\right)$ funci\'on de valores reales en $\rea$ acotada en intervalos finitos e igual a cero para $t<0$ La ecuaci\'on de renovaci\'on para $h\left(t\right)$ y la distribuci\'on $F$ es

\begin{eqnarray}%\label{Ec.Renovacion}
H\left(t\right)=h\left(t\right)+\int_{\left[0,t\right]}H\left(t-s\right)dF\left(s\right)\textrm{,    }t\geq0,
\end{eqnarray}
donde $H\left(t\right)$ es una funci\'on de valores reales. Esto es $H=h+F\star H$. Decimos que $H\left(t\right)$ es soluci\'on de esta ecuaci\'on si satisface la ecuaci\'on, y es acotada en intervalos finitos e iguales a cero para $t<0$.
\end{Def}

\begin{Prop}
La funci\'on $U\star h\left(t\right)$ es la \'unica soluci\'on de la ecuaci\'on de renovaci\'on (\ref{Ec.Renovacion}).
\end{Prop}

\begin{Teo}[Teorema Renovaci\'on Elemental]
\begin{eqnarray*}
t^{-1}U\left(t\right)\rightarrow 1/\mu\textrm{,    cuando }t\rightarrow\infty.
\end{eqnarray*}
\end{Teo}

%___________________________________________________________________________________________
%
\subsection{Funci\'on de Renovaci\'on}
%___________________________________________________________________________________________
%


Sup\'ongase que $N\left(t\right)$ es un proceso de renovaci\'on con distribuci\'on $F$ con media finita $\mu$.

\begin{Def}
La funci\'on de renovaci\'on asociada con la distribuci\'on $F$, del proceso $N\left(t\right)$, es
\begin{eqnarray*}
U\left(t\right)=\sum_{n=1}^{\infty}F^{n\star}\left(t\right),\textrm{   }t\geq0,
\end{eqnarray*}
donde $F^{0\star}\left(t\right)=\indora\left(t\geq0\right)$.
\end{Def}


\begin{Prop}
Sup\'ongase que la distribuci\'on de inter-renovaci\'on $F$ tiene densidad $f$. Entonces $U\left(t\right)$ tambi\'en tiene densidad, para $t>0$, y es $U^{'}\left(t\right)=\sum_{n=0}^{\infty}f^{n\star}\left(t\right)$. Adem\'as
\begin{eqnarray*}
\prob\left\{N\left(t\right)>N\left(t-\right)\right\}=0\textrm{,   }t\geq0.
\end{eqnarray*}
\end{Prop}

\begin{Def}
La Transformada de Laplace-Stieljes de $F$ est\'a dada por

\begin{eqnarray*}
\hat{F}\left(\alpha\right)=\int_{\rea_{+}}e^{-\alpha t}dF\left(t\right)\textrm{,  }\alpha\geq0.
\end{eqnarray*}
\end{Def}

Entonces

\begin{eqnarray*}
\hat{U}\left(\alpha\right)=\sum_{n=0}^{\infty}\hat{F^{n\star}}\left(\alpha\right)=\sum_{n=0}^{\infty}\hat{F}\left(\alpha\right)^{n}=\frac{1}{1-\hat{F}\left(\alpha\right)}.
\end{eqnarray*}


\begin{Prop}
La Transformada de Laplace $\hat{U}\left(\alpha\right)$ y $\hat{F}\left(\alpha\right)$ determina una a la otra de manera \'unica por la relaci\'on $\hat{U}\left(\alpha\right)=\frac{1}{1-\hat{F}\left(\alpha\right)}$.
\end{Prop}


\begin{Note}
Un proceso de renovaci\'on $N\left(t\right)$ cuyos tiempos de inter-renovaci\'on tienen media finita, es un proceso Poisson con tasa $\lambda$ si y s\'olo s\'i $\esp\left[U\left(t\right)\right]=\lambda t$, para $t\geq0$.
\end{Note}


\begin{Teo}
Sea $N\left(t\right)$ un proceso puntual simple con puntos de localizaci\'on $T_{n}$ tal que $\eta\left(t\right)=\esp\left[N\left(\right)\right]$ es finita para cada $t$. Entonces para cualquier funci\'on $f:\rea_{+}\rightarrow\rea$,
\begin{eqnarray*}
\esp\left[\sum_{n=1}^{N\left(\right)}f\left(T_{n}\right)\right]=\int_{\left(0,t\right]}f\left(s\right)d\eta\left(s\right)\textrm{,  }t\geq0,
\end{eqnarray*}
suponiendo que la integral exista. Adem\'as si $X_{1},X_{2},\ldots$ son variables aleatorias definidas en el mismo espacio de probabilidad que el proceso $N\left(t\right)$ tal que $\esp\left[X_{n}|T_{n}=s\right]=f\left(s\right)$, independiente de $n$. Entonces
\begin{eqnarray*}
\esp\left[\sum_{n=1}^{N\left(t\right)}X_{n}\right]=\int_{\left(0,t\right]}f\left(s\right)d\eta\left(s\right)\textrm{,  }t\geq0,
\end{eqnarray*} 
suponiendo que la integral exista. 
\end{Teo}

\begin{Coro}[Identidad de Wald para Renovaciones]
Para el proceso de renovaci\'on $N\left(t\right)$,
\begin{eqnarray*}
\esp\left[T_{N\left(t\right)+1}\right]=\mu\esp\left[N\left(t\right)+1\right]\textrm{,  }t\geq0,
\end{eqnarray*}  
\end{Coro}

%___________________________________________________________________________________________
%
\subsection{Funci\'on de Renovaci\'on}
%___________________________________________________________________________________________
%


\begin{Def}
Sea $h\left(t\right)$ funci\'on de valores reales en $\rea$ acotada en intervalos finitos e igual a cero para $t<0$ La ecuaci\'on de renovaci\'on para $h\left(t\right)$ y la distribuci\'on $F$ es

\begin{eqnarray}%\label{Ec.Renovacion}
H\left(t\right)=h\left(t\right)+\int_{\left[0,t\right]}H\left(t-s\right)dF\left(s\right)\textrm{,    }t\geq0,
\end{eqnarray}
donde $H\left(t\right)$ es una funci\'on de valores reales. Esto es $H=h+F\star H$. Decimos que $H\left(t\right)$ es soluci\'on de esta ecuaci\'on si satisface la ecuaci\'on, y es acotada en intervalos finitos e iguales a cero para $t<0$.
\end{Def}

\begin{Prop}
La funci\'on $U\star h\left(t\right)$ es la \'unica soluci\'on de la ecuaci\'on de renovaci\'on (\ref{Ec.Renovacion}).
\end{Prop}

\begin{Teo}[Teorema Renovaci\'on Elemental]
\begin{eqnarray*}
t^{-1}U\left(t\right)\rightarrow 1/\mu\textrm{,    cuando }t\rightarrow\infty.
\end{eqnarray*}
\end{Teo}

%___________________________________________________________________________________________
%
\subsection{Funci\'on de Renovaci\'on}
%___________________________________________________________________________________________
%


Sup\'ongase que $N\left(t\right)$ es un proceso de renovaci\'on con distribuci\'on $F$ con media finita $\mu$.

\begin{Def}
La funci\'on de renovaci\'on asociada con la distribuci\'on $F$, del proceso $N\left(t\right)$, es
\begin{eqnarray*}
U\left(t\right)=\sum_{n=1}^{\infty}F^{n\star}\left(t\right),\textrm{   }t\geq0,
\end{eqnarray*}
donde $F^{0\star}\left(t\right)=\indora\left(t\geq0\right)$.
\end{Def}


\begin{Prop}
Sup\'ongase que la distribuci\'on de inter-renovaci\'on $F$ tiene densidad $f$. Entonces $U\left(t\right)$ tambi\'en tiene densidad, para $t>0$, y es $U^{'}\left(t\right)=\sum_{n=0}^{\infty}f^{n\star}\left(t\right)$. Adem\'as
\begin{eqnarray*}
\prob\left\{N\left(t\right)>N\left(t-\right)\right\}=0\textrm{,   }t\geq0.
\end{eqnarray*}
\end{Prop}

\begin{Def}
La Transformada de Laplace-Stieljes de $F$ est\'a dada por

\begin{eqnarray*}
\hat{F}\left(\alpha\right)=\int_{\rea_{+}}e^{-\alpha t}dF\left(t\right)\textrm{,  }\alpha\geq0.
\end{eqnarray*}
\end{Def}

Entonces

\begin{eqnarray*}
\hat{U}\left(\alpha\right)=\sum_{n=0}^{\infty}\hat{F^{n\star}}\left(\alpha\right)=\sum_{n=0}^{\infty}\hat{F}\left(\alpha\right)^{n}=\frac{1}{1-\hat{F}\left(\alpha\right)}.
\end{eqnarray*}


\begin{Prop}
La Transformada de Laplace $\hat{U}\left(\alpha\right)$ y $\hat{F}\left(\alpha\right)$ determina una a la otra de manera \'unica por la relaci\'on $\hat{U}\left(\alpha\right)=\frac{1}{1-\hat{F}\left(\alpha\right)}$.
\end{Prop}


\begin{Note}
Un proceso de renovaci\'on $N\left(t\right)$ cuyos tiempos de inter-renovaci\'on tienen media finita, es un proceso Poisson con tasa $\lambda$ si y s\'olo s\'i $\esp\left[U\left(t\right)\right]=\lambda t$, para $t\geq0$.
\end{Note}


\begin{Teo}
Sea $N\left(t\right)$ un proceso puntual simple con puntos de localizaci\'on $T_{n}$ tal que $\eta\left(t\right)=\esp\left[N\left(\right)\right]$ es finita para cada $t$. Entonces para cualquier funci\'on $f:\rea_{+}\rightarrow\rea$,
\begin{eqnarray*}
\esp\left[\sum_{n=1}^{N\left(\right)}f\left(T_{n}\right)\right]=\int_{\left(0,t\right]}f\left(s\right)d\eta\left(s\right)\textrm{,  }t\geq0,
\end{eqnarray*}
suponiendo que la integral exista. Adem\'as si $X_{1},X_{2},\ldots$ son variables aleatorias definidas en el mismo espacio de probabilidad que el proceso $N\left(t\right)$ tal que $\esp\left[X_{n}|T_{n}=s\right]=f\left(s\right)$, independiente de $n$. Entonces
\begin{eqnarray*}
\esp\left[\sum_{n=1}^{N\left(t\right)}X_{n}\right]=\int_{\left(0,t\right]}f\left(s\right)d\eta\left(s\right)\textrm{,  }t\geq0,
\end{eqnarray*} 
suponiendo que la integral exista. 
\end{Teo}

\begin{Coro}[Identidad de Wald para Renovaciones]
Para el proceso de renovaci\'on $N\left(t\right)$,
\begin{eqnarray*}
\esp\left[T_{N\left(t\right)+1}\right]=\mu\esp\left[N\left(t\right)+1\right]\textrm{,  }t\geq0,
\end{eqnarray*}  
\end{Coro}


%________________________________________________________________________
\subsection{Procesos Regenerativos}
%________________________________________________________________________

Para $\left\{X\left(t\right):t\geq0\right\}$ Proceso Estoc\'astico a tiempo continuo con estado de espacios $S$, que es un espacio m\'etrico, con trayectorias continuas por la derecha y con l\'imites por la izquierda c.s. Sea $N\left(t\right)$ un proceso de renovaci\'on en $\rea_{+}$ definido en el mismo espacio de probabilidad que $X\left(t\right)$, con tiempos de renovaci\'on $T$ y tiempos de inter-renovaci\'on $\xi_{n}=T_{n}-T_{n-1}$, con misma distribuci\'on $F$ de media finita $\mu$.



\begin{Def}
Para el proceso $\left\{\left(N\left(t\right),X\left(t\right)\right):t\geq0\right\}$, sus trayectoria muestrales en el intervalo de tiempo $\left[T_{n-1},T_{n}\right)$ est\'an descritas por
\begin{eqnarray*}
\zeta_{n}=\left(\xi_{n},\left\{X\left(T_{n-1}+t\right):0\leq t<\xi_{n}\right\}\right)
\end{eqnarray*}
Este $\zeta_{n}$ es el $n$-\'esimo segmento del proceso. El proceso es regenerativo sobre los tiempos $T_{n}$ si sus segmentos $\zeta_{n}$ son independientes e id\'enticamennte distribuidos.
\end{Def}


\begin{Obs}
Si $\tilde{X}\left(t\right)$ con espacio de estados $\tilde{S}$ es regenerativo sobre $T_{n}$, entonces $X\left(t\right)=f\left(\tilde{X}\left(t\right)\right)$ tambi\'en es regenerativo sobre $T_{n}$, para cualquier funci\'on $f:\tilde{S}\rightarrow S$.
\end{Obs}

\begin{Obs}
Los procesos regenerativos son crudamente regenerativos, pero no al rev\'es.
\end{Obs}

\begin{Def}[Definici\'on Cl\'asica]
Un proceso estoc\'astico $X=\left\{X\left(t\right):t\geq0\right\}$ es llamado regenerativo is existe una variable aleatoria $R_{1}>0$ tal que
\begin{itemize}
\item[i)] $\left\{X\left(t+R_{1}\right):t\geq0\right\}$ es independiente de $\left\{\left\{X\left(t\right):t<R_{1}\right\},\right\}$
\item[ii)] $\left\{X\left(t+R_{1}\right):t\geq0\right\}$ es estoc\'asticamente equivalente a $\left\{X\left(t\right):t>0\right\}$
\end{itemize}

Llamamos a $R_{1}$ tiempo de regeneraci\'on, y decimos que $X$ se regenera en este punto.
\end{Def}

$\left\{X\left(t+R_{1}\right)\right\}$ es regenerativo con tiempo de regeneraci\'on $R_{2}$, independiente de $R_{1}$ pero con la misma distribuci\'on que $R_{1}$. Procediendo de esta manera se obtiene una secuencia de variables aleatorias independientes e id\'enticamente distribuidas $\left\{R_{n}\right\}$ llamados longitudes de ciclo. Si definimos a $Z_{k}\equiv R_{1}+R_{2}+\cdots+R_{k}$, se tiene un proceso de renovaci\'on llamado proceso de renovaci\'on encajado para $X$.

\begin{Note}
Un proceso regenerativo con media de la longitud de ciclo finita es llamado positivo recurrente.
\end{Note}


\begin{Def}
Para $x$ fijo y para cada $t\geq0$, sea $I_{x}\left(t\right)=1$ si $X\left(t\right)\leq x$,  $I_{x}\left(t\right)=0$ en caso contrario, y def\'inanse los tiempos promedio
\begin{eqnarray*}
\overline{X}&=&lim_{t\rightarrow\infty}\frac{1}{t}\int_{0}^{\infty}X\left(u\right)du\\
\prob\left(X_{\infty}\leq x\right)&=&lim_{t\rightarrow\infty}\frac{1}{t}\int_{0}^{\infty}I_{x}\left(u\right)du,
\end{eqnarray*}
cuando estos l\'imites existan.
\end{Def}

Como consecuencia del teorema de Renovaci\'on-Recompensa, se tiene que el primer l\'imite  existe y es igual a la constante
\begin{eqnarray*}
\overline{X}&=&\frac{\esp\left[\int_{0}^{R_{1}}X\left(t\right)dt\right]}{\esp\left[R_{1}\right]},
\end{eqnarray*}
suponiendo que ambas esperanzas son finitas.

\begin{Note}
\begin{itemize}
\item[a)] Si el proceso regenerativo $X$ es positivo recurrente y tiene trayectorias muestrales no negativas, entonces la ecuaci\'on anterior es v\'alida.
\item[b)] Si $X$ es positivo recurrente regenerativo, podemos construir una \'unica versi\'on estacionaria de este proceso, $X_{e}=\left\{X_{e}\left(t\right)\right\}$, donde $X_{e}$ es un proceso estoc\'astico regenerativo y estrictamente estacionario, con distribuci\'on marginal distribuida como $X_{\infty}$
\end{itemize}
\end{Note}



%________________________________________________________________________
%\subsection{Procesos Regenerativos Sigman, Thorisson y Wolff \cite{Sigman1}}
%________________________________________________________________________


\begin{Def}[Definici\'on Cl\'asica]
Un proceso estoc\'astico $X=\left\{X\left(t\right):t\geq0\right\}$ es llamado regenerativo is existe una variable aleatoria $R_{1}>0$ tal que
\begin{itemize}
\item[i)] $\left\{X\left(t+R_{1}\right):t\geq0\right\}$ es independiente de $\left\{\left\{X\left(t\right):t<R_{1}\right\},\right\}$
\item[ii)] $\left\{X\left(t+R_{1}\right):t\geq0\right\}$ es estoc\'asticamente equivalente a $\left\{X\left(t\right):t>0\right\}$
\end{itemize}

Llamamos a $R_{1}$ tiempo de regeneraci\'on, y decimos que $X$ se regenera en este punto.
\end{Def}

$\left\{X\left(t+R_{1}\right)\right\}$ es regenerativo con tiempo de regeneraci\'on $R_{2}$, independiente de $R_{1}$ pero con la misma distribuci\'on que $R_{1}$. Procediendo de esta manera se obtiene una secuencia de variables aleatorias independientes e id\'enticamente distribuidas $\left\{R_{n}\right\}$ llamados longitudes de ciclo. Si definimos a $Z_{k}\equiv R_{1}+R_{2}+\cdots+R_{k}$, se tiene un proceso de renovaci\'on llamado proceso de renovaci\'on encajado para $X$.


\begin{Note}
La existencia de un primer tiempo de regeneraci\'on, $R_{1}$, implica la existencia de una sucesi\'on completa de estos tiempos $R_{1},R_{2}\ldots,$ que satisfacen la propiedad deseada \cite{Sigman2}.
\end{Note}


\begin{Note} Para la cola $GI/GI/1$ los usuarios arriban con tiempos $t_{n}$ y son atendidos con tiempos de servicio $S_{n}$, los tiempos de arribo forman un proceso de renovaci\'on  con tiempos entre arribos independientes e identicamente distribuidos (\texttt{i.i.d.})$T_{n}=t_{n}-t_{n-1}$, adem\'as los tiempos de servicio son \texttt{i.i.d.} e independientes de los procesos de arribo. Por \textit{estable} se entiende que $\esp S_{n}<\esp T_{n}<\infty$.
\end{Note}
 


\begin{Def}
Para $x$ fijo y para cada $t\geq0$, sea $I_{x}\left(t\right)=1$ si $X\left(t\right)\leq x$,  $I_{x}\left(t\right)=0$ en caso contrario, y def\'inanse los tiempos promedio
\begin{eqnarray*}
\overline{X}&=&lim_{t\rightarrow\infty}\frac{1}{t}\int_{0}^{\infty}X\left(u\right)du\\
\prob\left(X_{\infty}\leq x\right)&=&lim_{t\rightarrow\infty}\frac{1}{t}\int_{0}^{\infty}I_{x}\left(u\right)du,
\end{eqnarray*}
cuando estos l\'imites existan.
\end{Def}

Como consecuencia del teorema de Renovaci\'on-Recompensa, se tiene que el primer l\'imite  existe y es igual a la constante
\begin{eqnarray*}
\overline{X}&=&\frac{\esp\left[\int_{0}^{R_{1}}X\left(t\right)dt\right]}{\esp\left[R_{1}\right]},
\end{eqnarray*}
suponiendo que ambas esperanzas son finitas.
 
\begin{Note}
Funciones de procesos regenerativos son regenerativas, es decir, si $X\left(t\right)$ es regenerativo y se define el proceso $Y\left(t\right)$ por $Y\left(t\right)=f\left(X\left(t\right)\right)$ para alguna funci\'on Borel medible $f\left(\cdot\right)$. Adem\'as $Y$ es regenerativo con los mismos tiempos de renovaci\'on que $X$. 

En general, los tiempos de renovaci\'on, $Z_{k}$ de un proceso regenerativo no requieren ser tiempos de paro con respecto a la evoluci\'on de $X\left(t\right)$.
\end{Note} 

\begin{Note}
Una funci\'on de un proceso de Markov, usualmente no ser\'a un proceso de Markov, sin embargo ser\'a regenerativo si el proceso de Markov lo es.
\end{Note}

 
\begin{Note}
Un proceso regenerativo con media de la longitud de ciclo finita es llamado positivo recurrente.
\end{Note}


\begin{Note}
\begin{itemize}
\item[a)] Si el proceso regenerativo $X$ es positivo recurrente y tiene trayectorias muestrales no negativas, entonces la ecuaci\'on anterior es v\'alida.
\item[b)] Si $X$ es positivo recurrente regenerativo, podemos construir una \'unica versi\'on estacionaria de este proceso, $X_{e}=\left\{X_{e}\left(t\right)\right\}$, donde $X_{e}$ es un proceso estoc\'astico regenerativo y estrictamente estacionario, con distribuci\'on marginal distribuida como $X_{\infty}$
\end{itemize}
\end{Note}


%__________________________________________________________________________________________
%\subsection{Procesos Regenerativos Estacionarios - Stidham \cite{Stidham}}
%__________________________________________________________________________________________


Un proceso estoc\'astico a tiempo continuo $\left\{V\left(t\right),t\geq0\right\}$ es un proceso regenerativo si existe una sucesi\'on de variables aleatorias independientes e id\'enticamente distribuidas $\left\{X_{1},X_{2},\ldots\right\}$, sucesi\'on de renovaci\'on, tal que para cualquier conjunto de Borel $A$, 

\begin{eqnarray*}
\prob\left\{V\left(t\right)\in A|X_{1}+X_{2}+\cdots+X_{R\left(t\right)}=s,\left\{V\left(\tau\right),\tau<s\right\}\right\}=\prob\left\{V\left(t-s\right)\in A|X_{1}>t-s\right\},
\end{eqnarray*}
para todo $0\leq s\leq t$, donde $R\left(t\right)=\max\left\{X_{1}+X_{2}+\cdots+X_{j}\leq t\right\}=$n\'umero de renovaciones ({\emph{puntos de regeneraci\'on}}) que ocurren en $\left[0,t\right]$. El intervalo $\left[0,X_{1}\right)$ es llamado {\emph{primer ciclo de regeneraci\'on}} de $\left\{V\left(t \right),t\geq0\right\}$, $\left[X_{1},X_{1}+X_{2}\right)$ el {\emph{segundo ciclo de regeneraci\'on}}, y as\'i sucesivamente.

Sea $X=X_{1}$ y sea $F$ la funci\'on de distrbuci\'on de $X$


\begin{Def}
Se define el proceso estacionario, $\left\{V^{*}\left(t\right),t\geq0\right\}$, para $\left\{V\left(t\right),t\geq0\right\}$ por

\begin{eqnarray*}
\prob\left\{V\left(t\right)\in A\right\}=\frac{1}{\esp\left[X\right]}\int_{0}^{\infty}\prob\left\{V\left(t+x\right)\in A|X>x\right\}\left(1-F\left(x\right)\right)dx,
\end{eqnarray*} 
para todo $t\geq0$ y todo conjunto de Borel $A$.
\end{Def}

\begin{Def}
Una distribuci\'on se dice que es {\emph{aritm\'etica}} si todos sus puntos de incremento son m\'ultiplos de la forma $0,\lambda, 2\lambda,\ldots$ para alguna $\lambda>0$ entera.
\end{Def}


\begin{Def}
Una modificaci\'on medible de un proceso $\left\{V\left(t\right),t\geq0\right\}$, es una versi\'on de este, $\left\{V\left(t,w\right)\right\}$ conjuntamente medible para $t\geq0$ y para $w\in S$, $S$ espacio de estados para $\left\{V\left(t\right),t\geq0\right\}$.
\end{Def}

\begin{Teo}
Sea $\left\{V\left(t\right),t\geq\right\}$ un proceso regenerativo no negativo con modificaci\'on medible. Sea $\esp\left[X\right]<\infty$. Entonces el proceso estacionario dado por la ecuaci\'on anterior est\'a bien definido y tiene funci\'on de distribuci\'on independiente de $t$, adem\'as
\begin{itemize}
\item[i)] \begin{eqnarray*}
\esp\left[V^{*}\left(0\right)\right]&=&\frac{\esp\left[\int_{0}^{X}V\left(s\right)ds\right]}{\esp\left[X\right]}\end{eqnarray*}
\item[ii)] Si $\esp\left[V^{*}\left(0\right)\right]<\infty$, equivalentemente, si $\esp\left[\int_{0}^{X}V\left(s\right)ds\right]<\infty$,entonces
\begin{eqnarray*}
\frac{\int_{0}^{t}V\left(s\right)ds}{t}\rightarrow\frac{\esp\left[\int_{0}^{X}V\left(s\right)ds\right]}{\esp\left[X\right]}
\end{eqnarray*}
con probabilidad 1 y en media, cuando $t\rightarrow\infty$.
\end{itemize}
\end{Teo}

\begin{Coro}
Sea $\left\{V\left(t\right),t\geq0\right\}$ un proceso regenerativo no negativo, con modificaci\'on medible. Si $\esp <\infty$, $F$ es no-aritm\'etica, y para todo $x\geq0$, $P\left\{V\left(t\right)\leq x,C>x\right\}$ es de variaci\'on acotada como funci\'on de $t$ en cada intervalo finito $\left[0,\tau\right]$, entonces $V\left(t\right)$ converge en distribuci\'on  cuando $t\rightarrow\infty$ y $$\esp V=\frac{\esp \int_{0}^{X}V\left(s\right)ds}{\esp X}$$
Donde $V$ tiene la distribuci\'on l\'imite de $V\left(t\right)$ cuando $t\rightarrow\infty$.

\end{Coro}

Para el caso discreto se tienen resultados similares.



%______________________________________________________________________
%\subsection{Procesos de Renovaci\'on}
%______________________________________________________________________

\begin{Def}%\label{Def.Tn}
Sean $0\leq T_{1}\leq T_{2}\leq \ldots$ son tiempos aleatorios infinitos en los cuales ocurren ciertos eventos. El n\'umero de tiempos $T_{n}$ en el intervalo $\left[0,t\right)$ es

\begin{eqnarray}
N\left(t\right)=\sum_{n=1}^{\infty}\indora\left(T_{n}\leq t\right),
\end{eqnarray}
para $t\geq0$.
\end{Def}

Si se consideran los puntos $T_{n}$ como elementos de $\rea_{+}$, y $N\left(t\right)$ es el n\'umero de puntos en $\rea$. El proceso denotado por $\left\{N\left(t\right):t\geq0\right\}$, denotado por $N\left(t\right)$, es un proceso puntual en $\rea_{+}$. Los $T_{n}$ son los tiempos de ocurrencia, el proceso puntual $N\left(t\right)$ es simple si su n\'umero de ocurrencias son distintas: $0<T_{1}<T_{2}<\ldots$ casi seguramente.

\begin{Def}
Un proceso puntual $N\left(t\right)$ es un proceso de renovaci\'on si los tiempos de interocurrencia $\xi_{n}=T_{n}-T_{n-1}$, para $n\geq1$, son independientes e identicamente distribuidos con distribuci\'on $F$, donde $F\left(0\right)=0$ y $T_{0}=0$. Los $T_{n}$ son llamados tiempos de renovaci\'on, referente a la independencia o renovaci\'on de la informaci\'on estoc\'astica en estos tiempos. Los $\xi_{n}$ son los tiempos de inter-renovaci\'on, y $N\left(t\right)$ es el n\'umero de renovaciones en el intervalo $\left[0,t\right)$
\end{Def}


\begin{Note}
Para definir un proceso de renovaci\'on para cualquier contexto, solamente hay que especificar una distribuci\'on $F$, con $F\left(0\right)=0$, para los tiempos de inter-renovaci\'on. La funci\'on $F$ en turno degune las otra variables aleatorias. De manera formal, existe un espacio de probabilidad y una sucesi\'on de variables aleatorias $\xi_{1},\xi_{2},\ldots$ definidas en este con distribuci\'on $F$. Entonces las otras cantidades son $T_{n}=\sum_{k=1}^{n}\xi_{k}$ y $N\left(t\right)=\sum_{n=1}^{\infty}\indora\left(T_{n}\leq t\right)$, donde $T_{n}\rightarrow\infty$ casi seguramente por la Ley Fuerte de los Grandes Números.
\end{Note}

%___________________________________________________________________________________________
%
%\subsection{Teorema Principal de Renovaci\'on}
%___________________________________________________________________________________________
%

\begin{Note} Una funci\'on $h:\rea_{+}\rightarrow\rea$ es Directamente Riemann Integrable en los siguientes casos:
\begin{itemize}
\item[a)] $h\left(t\right)\geq0$ es decreciente y Riemann Integrable.
\item[b)] $h$ es continua excepto posiblemente en un conjunto de Lebesgue de medida 0, y $|h\left(t\right)|\leq b\left(t\right)$, donde $b$ es DRI.
\end{itemize}
\end{Note}

\begin{Teo}[Teorema Principal de Renovaci\'on]
Si $F$ es no aritm\'etica y $h\left(t\right)$ es Directamente Riemann Integrable (DRI), entonces

\begin{eqnarray*}
lim_{t\rightarrow\infty}U\star h=\frac{1}{\mu}\int_{\rea_{+}}h\left(s\right)ds.
\end{eqnarray*}
\end{Teo}

\begin{Prop}
Cualquier funci\'on $H\left(t\right)$ acotada en intervalos finitos y que es 0 para $t<0$ puede expresarse como
\begin{eqnarray*}
H\left(t\right)=U\star h\left(t\right)\textrm{,  donde }h\left(t\right)=H\left(t\right)-F\star H\left(t\right)
\end{eqnarray*}
\end{Prop}

\begin{Def}
Un proceso estoc\'astico $X\left(t\right)$ es crudamente regenerativo en un tiempo aleatorio positivo $T$ si
\begin{eqnarray*}
\esp\left[X\left(T+t\right)|T\right]=\esp\left[X\left(t\right)\right]\textrm{, para }t\geq0,\end{eqnarray*}
y con las esperanzas anteriores finitas.
\end{Def}

\begin{Prop}
Sup\'ongase que $X\left(t\right)$ es un proceso crudamente regenerativo en $T$, que tiene distribuci\'on $F$. Si $\esp\left[X\left(t\right)\right]$ es acotado en intervalos finitos, entonces
\begin{eqnarray*}
\esp\left[X\left(t\right)\right]=U\star h\left(t\right)\textrm{,  donde }h\left(t\right)=\esp\left[X\left(t\right)\indora\left(T>t\right)\right].
\end{eqnarray*}
\end{Prop}

\begin{Teo}[Regeneraci\'on Cruda]
Sup\'ongase que $X\left(t\right)$ es un proceso con valores positivo crudamente regenerativo en $T$, y def\'inase $M=\sup\left\{|X\left(t\right)|:t\leq T\right\}$. Si $T$ es no aritm\'etico y $M$ y $MT$ tienen media finita, entonces
\begin{eqnarray*}
lim_{t\rightarrow\infty}\esp\left[X\left(t\right)\right]=\frac{1}{\mu}\int_{\rea_{+}}h\left(s\right)ds,
\end{eqnarray*}
donde $h\left(t\right)=\esp\left[X\left(t\right)\indora\left(T>t\right)\right]$.
\end{Teo}

%___________________________________________________________________________________________
%
%\subsection{Propiedades de los Procesos de Renovaci\'on}
%___________________________________________________________________________________________
%

Los tiempos $T_{n}$ est\'an relacionados con los conteos de $N\left(t\right)$ por

\begin{eqnarray*}
\left\{N\left(t\right)\geq n\right\}&=&\left\{T_{n}\leq t\right\}\\
T_{N\left(t\right)}\leq &t&<T_{N\left(t\right)+1},
\end{eqnarray*}

adem\'as $N\left(T_{n}\right)=n$, y 

\begin{eqnarray*}
N\left(t\right)=\max\left\{n:T_{n}\leq t\right\}=\min\left\{n:T_{n+1}>t\right\}
\end{eqnarray*}

Por propiedades de la convoluci\'on se sabe que

\begin{eqnarray*}
P\left\{T_{n}\leq t\right\}=F^{n\star}\left(t\right)
\end{eqnarray*}
que es la $n$-\'esima convoluci\'on de $F$. Entonces 

\begin{eqnarray*}
\left\{N\left(t\right)\geq n\right\}&=&\left\{T_{n}\leq t\right\}\\
P\left\{N\left(t\right)\leq n\right\}&=&1-F^{\left(n+1\right)\star}\left(t\right)
\end{eqnarray*}

Adem\'as usando el hecho de que $\esp\left[N\left(t\right)\right]=\sum_{n=1}^{\infty}P\left\{N\left(t\right)\geq n\right\}$
se tiene que

\begin{eqnarray*}
\esp\left[N\left(t\right)\right]=\sum_{n=1}^{\infty}F^{n\star}\left(t\right)
\end{eqnarray*}

\begin{Prop}
Para cada $t\geq0$, la funci\'on generadora de momentos $\esp\left[e^{\alpha N\left(t\right)}\right]$ existe para alguna $\alpha$ en una vecindad del 0, y de aqu\'i que $\esp\left[N\left(t\right)^{m}\right]<\infty$, para $m\geq1$.
\end{Prop}


\begin{Note}
Si el primer tiempo de renovaci\'on $\xi_{1}$ no tiene la misma distribuci\'on que el resto de las $\xi_{n}$, para $n\geq2$, a $N\left(t\right)$ se le llama Proceso de Renovaci\'on retardado, donde si $\xi$ tiene distribuci\'on $G$, entonces el tiempo $T_{n}$ de la $n$-\'esima renovaci\'on tiene distribuci\'on $G\star F^{\left(n-1\right)\star}\left(t\right)$
\end{Note}


\begin{Teo}
Para una constante $\mu\leq\infty$ ( o variable aleatoria), las siguientes expresiones son equivalentes:

\begin{eqnarray}
lim_{n\rightarrow\infty}n^{-1}T_{n}&=&\mu,\textrm{ c.s.}\\
lim_{t\rightarrow\infty}t^{-1}N\left(t\right)&=&1/\mu,\textrm{ c.s.}
\end{eqnarray}
\end{Teo}


Es decir, $T_{n}$ satisface la Ley Fuerte de los Grandes N\'umeros s\'i y s\'olo s\'i $N\left/t\right)$ la cumple.


\begin{Coro}[Ley Fuerte de los Grandes N\'umeros para Procesos de Renovaci\'on]
Si $N\left(t\right)$ es un proceso de renovaci\'on cuyos tiempos de inter-renovaci\'on tienen media $\mu\leq\infty$, entonces
\begin{eqnarray}
t^{-1}N\left(t\right)\rightarrow 1/\mu,\textrm{ c.s. cuando }t\rightarrow\infty.
\end{eqnarray}

\end{Coro}


Considerar el proceso estoc\'astico de valores reales $\left\{Z\left(t\right):t\geq0\right\}$ en el mismo espacio de probabilidad que $N\left(t\right)$

\begin{Def}
Para el proceso $\left\{Z\left(t\right):t\geq0\right\}$ se define la fluctuaci\'on m\'axima de $Z\left(t\right)$ en el intervalo $\left(T_{n-1},T_{n}\right]$:
\begin{eqnarray*}
M_{n}=\sup_{T_{n-1}<t\leq T_{n}}|Z\left(t\right)-Z\left(T_{n-1}\right)|
\end{eqnarray*}
\end{Def}

\begin{Teo}
Sup\'ongase que $n^{-1}T_{n}\rightarrow\mu$ c.s. cuando $n\rightarrow\infty$, donde $\mu\leq\infty$ es una constante o variable aleatoria. Sea $a$ una constante o variable aleatoria que puede ser infinita cuando $\mu$ es finita, y considere las expresiones l\'imite:
\begin{eqnarray}
lim_{n\rightarrow\infty}n^{-1}Z\left(T_{n}\right)&=&a,\textrm{ c.s.}\\
lim_{t\rightarrow\infty}t^{-1}Z\left(t\right)&=&a/\mu,\textrm{ c.s.}
\end{eqnarray}
La segunda expresi\'on implica la primera. Conversamente, la primera implica la segunda si el proceso $Z\left(t\right)$ es creciente, o si $lim_{n\rightarrow\infty}n^{-1}M_{n}=0$ c.s.
\end{Teo}

\begin{Coro}
Si $N\left(t\right)$ es un proceso de renovaci\'on, y $\left(Z\left(T_{n}\right)-Z\left(T_{n-1}\right),M_{n}\right)$, para $n\geq1$, son variables aleatorias independientes e id\'enticamente distribuidas con media finita, entonces,
\begin{eqnarray}
lim_{t\rightarrow\infty}t^{-1}Z\left(t\right)\rightarrow\frac{\esp\left[Z\left(T_{1}\right)-Z\left(T_{0}\right)\right]}{\esp\left[T_{1}\right]},\textrm{ c.s. cuando  }t\rightarrow\infty.
\end{eqnarray}
\end{Coro}



%___________________________________________________________________________________________
%
%\subsection{Propiedades de los Procesos de Renovaci\'on}
%___________________________________________________________________________________________
%

Los tiempos $T_{n}$ est\'an relacionados con los conteos de $N\left(t\right)$ por

\begin{eqnarray*}
\left\{N\left(t\right)\geq n\right\}&=&\left\{T_{n}\leq t\right\}\\
T_{N\left(t\right)}\leq &t&<T_{N\left(t\right)+1},
\end{eqnarray*}

adem\'as $N\left(T_{n}\right)=n$, y 

\begin{eqnarray*}
N\left(t\right)=\max\left\{n:T_{n}\leq t\right\}=\min\left\{n:T_{n+1}>t\right\}
\end{eqnarray*}

Por propiedades de la convoluci\'on se sabe que

\begin{eqnarray*}
P\left\{T_{n}\leq t\right\}=F^{n\star}\left(t\right)
\end{eqnarray*}
que es la $n$-\'esima convoluci\'on de $F$. Entonces 

\begin{eqnarray*}
\left\{N\left(t\right)\geq n\right\}&=&\left\{T_{n}\leq t\right\}\\
P\left\{N\left(t\right)\leq n\right\}&=&1-F^{\left(n+1\right)\star}\left(t\right)
\end{eqnarray*}

Adem\'as usando el hecho de que $\esp\left[N\left(t\right)\right]=\sum_{n=1}^{\infty}P\left\{N\left(t\right)\geq n\right\}$
se tiene que

\begin{eqnarray*}
\esp\left[N\left(t\right)\right]=\sum_{n=1}^{\infty}F^{n\star}\left(t\right)
\end{eqnarray*}

\begin{Prop}
Para cada $t\geq0$, la funci\'on generadora de momentos $\esp\left[e^{\alpha N\left(t\right)}\right]$ existe para alguna $\alpha$ en una vecindad del 0, y de aqu\'i que $\esp\left[N\left(t\right)^{m}\right]<\infty$, para $m\geq1$.
\end{Prop}


\begin{Note}
Si el primer tiempo de renovaci\'on $\xi_{1}$ no tiene la misma distribuci\'on que el resto de las $\xi_{n}$, para $n\geq2$, a $N\left(t\right)$ se le llama Proceso de Renovaci\'on retardado, donde si $\xi$ tiene distribuci\'on $G$, entonces el tiempo $T_{n}$ de la $n$-\'esima renovaci\'on tiene distribuci\'on $G\star F^{\left(n-1\right)\star}\left(t\right)$
\end{Note}


\begin{Teo}
Para una constante $\mu\leq\infty$ ( o variable aleatoria), las siguientes expresiones son equivalentes:

\begin{eqnarray}
lim_{n\rightarrow\infty}n^{-1}T_{n}&=&\mu,\textrm{ c.s.}\\
lim_{t\rightarrow\infty}t^{-1}N\left(t\right)&=&1/\mu,\textrm{ c.s.}
\end{eqnarray}
\end{Teo}


Es decir, $T_{n}$ satisface la Ley Fuerte de los Grandes N\'umeros s\'i y s\'olo s\'i $N\left/t\right)$ la cumple.


\begin{Coro}[Ley Fuerte de los Grandes N\'umeros para Procesos de Renovaci\'on]
Si $N\left(t\right)$ es un proceso de renovaci\'on cuyos tiempos de inter-renovaci\'on tienen media $\mu\leq\infty$, entonces
\begin{eqnarray}
t^{-1}N\left(t\right)\rightarrow 1/\mu,\textrm{ c.s. cuando }t\rightarrow\infty.
\end{eqnarray}

\end{Coro}


Considerar el proceso estoc\'astico de valores reales $\left\{Z\left(t\right):t\geq0\right\}$ en el mismo espacio de probabilidad que $N\left(t\right)$

\begin{Def}
Para el proceso $\left\{Z\left(t\right):t\geq0\right\}$ se define la fluctuaci\'on m\'axima de $Z\left(t\right)$ en el intervalo $\left(T_{n-1},T_{n}\right]$:
\begin{eqnarray*}
M_{n}=\sup_{T_{n-1}<t\leq T_{n}}|Z\left(t\right)-Z\left(T_{n-1}\right)|
\end{eqnarray*}
\end{Def}

\begin{Teo}
Sup\'ongase que $n^{-1}T_{n}\rightarrow\mu$ c.s. cuando $n\rightarrow\infty$, donde $\mu\leq\infty$ es una constante o variable aleatoria. Sea $a$ una constante o variable aleatoria que puede ser infinita cuando $\mu$ es finita, y considere las expresiones l\'imite:
\begin{eqnarray}
lim_{n\rightarrow\infty}n^{-1}Z\left(T_{n}\right)&=&a,\textrm{ c.s.}\\
lim_{t\rightarrow\infty}t^{-1}Z\left(t\right)&=&a/\mu,\textrm{ c.s.}
\end{eqnarray}
La segunda expresi\'on implica la primera. Conversamente, la primera implica la segunda si el proceso $Z\left(t\right)$ es creciente, o si $lim_{n\rightarrow\infty}n^{-1}M_{n}=0$ c.s.
\end{Teo}

\begin{Coro}
Si $N\left(t\right)$ es un proceso de renovaci\'on, y $\left(Z\left(T_{n}\right)-Z\left(T_{n-1}\right),M_{n}\right)$, para $n\geq1$, son variables aleatorias independientes e id\'enticamente distribuidas con media finita, entonces,
\begin{eqnarray}
lim_{t\rightarrow\infty}t^{-1}Z\left(t\right)\rightarrow\frac{\esp\left[Z\left(T_{1}\right)-Z\left(T_{0}\right)\right]}{\esp\left[T_{1}\right]},\textrm{ c.s. cuando  }t\rightarrow\infty.
\end{eqnarray}
\end{Coro}


%___________________________________________________________________________________________
%
%\subsection{Propiedades de los Procesos de Renovaci\'on}
%___________________________________________________________________________________________
%

Los tiempos $T_{n}$ est\'an relacionados con los conteos de $N\left(t\right)$ por

\begin{eqnarray*}
\left\{N\left(t\right)\geq n\right\}&=&\left\{T_{n}\leq t\right\}\\
T_{N\left(t\right)}\leq &t&<T_{N\left(t\right)+1},
\end{eqnarray*}

adem\'as $N\left(T_{n}\right)=n$, y 

\begin{eqnarray*}
N\left(t\right)=\max\left\{n:T_{n}\leq t\right\}=\min\left\{n:T_{n+1}>t\right\}
\end{eqnarray*}

Por propiedades de la convoluci\'on se sabe que

\begin{eqnarray*}
P\left\{T_{n}\leq t\right\}=F^{n\star}\left(t\right)
\end{eqnarray*}
que es la $n$-\'esima convoluci\'on de $F$. Entonces 

\begin{eqnarray*}
\left\{N\left(t\right)\geq n\right\}&=&\left\{T_{n}\leq t\right\}\\
P\left\{N\left(t\right)\leq n\right\}&=&1-F^{\left(n+1\right)\star}\left(t\right)
\end{eqnarray*}

Adem\'as usando el hecho de que $\esp\left[N\left(t\right)\right]=\sum_{n=1}^{\infty}P\left\{N\left(t\right)\geq n\right\}$
se tiene que

\begin{eqnarray*}
\esp\left[N\left(t\right)\right]=\sum_{n=1}^{\infty}F^{n\star}\left(t\right)
\end{eqnarray*}

\begin{Prop}
Para cada $t\geq0$, la funci\'on generadora de momentos $\esp\left[e^{\alpha N\left(t\right)}\right]$ existe para alguna $\alpha$ en una vecindad del 0, y de aqu\'i que $\esp\left[N\left(t\right)^{m}\right]<\infty$, para $m\geq1$.
\end{Prop}


\begin{Note}
Si el primer tiempo de renovaci\'on $\xi_{1}$ no tiene la misma distribuci\'on que el resto de las $\xi_{n}$, para $n\geq2$, a $N\left(t\right)$ se le llama Proceso de Renovaci\'on retardado, donde si $\xi$ tiene distribuci\'on $G$, entonces el tiempo $T_{n}$ de la $n$-\'esima renovaci\'on tiene distribuci\'on $G\star F^{\left(n-1\right)\star}\left(t\right)$
\end{Note}


\begin{Teo}
Para una constante $\mu\leq\infty$ ( o variable aleatoria), las siguientes expresiones son equivalentes:

\begin{eqnarray}
lim_{n\rightarrow\infty}n^{-1}T_{n}&=&\mu,\textrm{ c.s.}\\
lim_{t\rightarrow\infty}t^{-1}N\left(t\right)&=&1/\mu,\textrm{ c.s.}
\end{eqnarray}
\end{Teo}


Es decir, $T_{n}$ satisface la Ley Fuerte de los Grandes N\'umeros s\'i y s\'olo s\'i $N\left/t\right)$ la cumple.


\begin{Coro}[Ley Fuerte de los Grandes N\'umeros para Procesos de Renovaci\'on]
Si $N\left(t\right)$ es un proceso de renovaci\'on cuyos tiempos de inter-renovaci\'on tienen media $\mu\leq\infty$, entonces
\begin{eqnarray}
t^{-1}N\left(t\right)\rightarrow 1/\mu,\textrm{ c.s. cuando }t\rightarrow\infty.
\end{eqnarray}

\end{Coro}


Considerar el proceso estoc\'astico de valores reales $\left\{Z\left(t\right):t\geq0\right\}$ en el mismo espacio de probabilidad que $N\left(t\right)$

\begin{Def}
Para el proceso $\left\{Z\left(t\right):t\geq0\right\}$ se define la fluctuaci\'on m\'axima de $Z\left(t\right)$ en el intervalo $\left(T_{n-1},T_{n}\right]$:
\begin{eqnarray*}
M_{n}=\sup_{T_{n-1}<t\leq T_{n}}|Z\left(t\right)-Z\left(T_{n-1}\right)|
\end{eqnarray*}
\end{Def}

\begin{Teo}
Sup\'ongase que $n^{-1}T_{n}\rightarrow\mu$ c.s. cuando $n\rightarrow\infty$, donde $\mu\leq\infty$ es una constante o variable aleatoria. Sea $a$ una constante o variable aleatoria que puede ser infinita cuando $\mu$ es finita, y considere las expresiones l\'imite:
\begin{eqnarray}
lim_{n\rightarrow\infty}n^{-1}Z\left(T_{n}\right)&=&a,\textrm{ c.s.}\\
lim_{t\rightarrow\infty}t^{-1}Z\left(t\right)&=&a/\mu,\textrm{ c.s.}
\end{eqnarray}
La segunda expresi\'on implica la primera. Conversamente, la primera implica la segunda si el proceso $Z\left(t\right)$ es creciente, o si $lim_{n\rightarrow\infty}n^{-1}M_{n}=0$ c.s.
\end{Teo}

\begin{Coro}
Si $N\left(t\right)$ es un proceso de renovaci\'on, y $\left(Z\left(T_{n}\right)-Z\left(T_{n-1}\right),M_{n}\right)$, para $n\geq1$, son variables aleatorias independientes e id\'enticamente distribuidas con media finita, entonces,
\begin{eqnarray}
lim_{t\rightarrow\infty}t^{-1}Z\left(t\right)\rightarrow\frac{\esp\left[Z\left(T_{1}\right)-Z\left(T_{0}\right)\right]}{\esp\left[T_{1}\right]},\textrm{ c.s. cuando  }t\rightarrow\infty.
\end{eqnarray}
\end{Coro}

%___________________________________________________________________________________________
%
%\subsection{Propiedades de los Procesos de Renovaci\'on}
%___________________________________________________________________________________________
%

Los tiempos $T_{n}$ est\'an relacionados con los conteos de $N\left(t\right)$ por

\begin{eqnarray*}
\left\{N\left(t\right)\geq n\right\}&=&\left\{T_{n}\leq t\right\}\\
T_{N\left(t\right)}\leq &t&<T_{N\left(t\right)+1},
\end{eqnarray*}

adem\'as $N\left(T_{n}\right)=n$, y 

\begin{eqnarray*}
N\left(t\right)=\max\left\{n:T_{n}\leq t\right\}=\min\left\{n:T_{n+1}>t\right\}
\end{eqnarray*}

Por propiedades de la convoluci\'on se sabe que

\begin{eqnarray*}
P\left\{T_{n}\leq t\right\}=F^{n\star}\left(t\right)
\end{eqnarray*}
que es la $n$-\'esima convoluci\'on de $F$. Entonces 

\begin{eqnarray*}
\left\{N\left(t\right)\geq n\right\}&=&\left\{T_{n}\leq t\right\}\\
P\left\{N\left(t\right)\leq n\right\}&=&1-F^{\left(n+1\right)\star}\left(t\right)
\end{eqnarray*}

Adem\'as usando el hecho de que $\esp\left[N\left(t\right)\right]=\sum_{n=1}^{\infty}P\left\{N\left(t\right)\geq n\right\}$
se tiene que

\begin{eqnarray*}
\esp\left[N\left(t\right)\right]=\sum_{n=1}^{\infty}F^{n\star}\left(t\right)
\end{eqnarray*}

\begin{Prop}
Para cada $t\geq0$, la funci\'on generadora de momentos $\esp\left[e^{\alpha N\left(t\right)}\right]$ existe para alguna $\alpha$ en una vecindad del 0, y de aqu\'i que $\esp\left[N\left(t\right)^{m}\right]<\infty$, para $m\geq1$.
\end{Prop}


\begin{Note}
Si el primer tiempo de renovaci\'on $\xi_{1}$ no tiene la misma distribuci\'on que el resto de las $\xi_{n}$, para $n\geq2$, a $N\left(t\right)$ se le llama Proceso de Renovaci\'on retardado, donde si $\xi$ tiene distribuci\'on $G$, entonces el tiempo $T_{n}$ de la $n$-\'esima renovaci\'on tiene distribuci\'on $G\star F^{\left(n-1\right)\star}\left(t\right)$
\end{Note}


\begin{Teo}
Para una constante $\mu\leq\infty$ ( o variable aleatoria), las siguientes expresiones son equivalentes:

\begin{eqnarray}
lim_{n\rightarrow\infty}n^{-1}T_{n}&=&\mu,\textrm{ c.s.}\\
lim_{t\rightarrow\infty}t^{-1}N\left(t\right)&=&1/\mu,\textrm{ c.s.}
\end{eqnarray}
\end{Teo}


Es decir, $T_{n}$ satisface la Ley Fuerte de los Grandes N\'umeros s\'i y s\'olo s\'i $N\left/t\right)$ la cumple.


\begin{Coro}[Ley Fuerte de los Grandes N\'umeros para Procesos de Renovaci\'on]
Si $N\left(t\right)$ es un proceso de renovaci\'on cuyos tiempos de inter-renovaci\'on tienen media $\mu\leq\infty$, entonces
\begin{eqnarray}
t^{-1}N\left(t\right)\rightarrow 1/\mu,\textrm{ c.s. cuando }t\rightarrow\infty.
\end{eqnarray}

\end{Coro}


Considerar el proceso estoc\'astico de valores reales $\left\{Z\left(t\right):t\geq0\right\}$ en el mismo espacio de probabilidad que $N\left(t\right)$

\begin{Def}
Para el proceso $\left\{Z\left(t\right):t\geq0\right\}$ se define la fluctuaci\'on m\'axima de $Z\left(t\right)$ en el intervalo $\left(T_{n-1},T_{n}\right]$:
\begin{eqnarray*}
M_{n}=\sup_{T_{n-1}<t\leq T_{n}}|Z\left(t\right)-Z\left(T_{n-1}\right)|
\end{eqnarray*}
\end{Def}

\begin{Teo}
Sup\'ongase que $n^{-1}T_{n}\rightarrow\mu$ c.s. cuando $n\rightarrow\infty$, donde $\mu\leq\infty$ es una constante o variable aleatoria. Sea $a$ una constante o variable aleatoria que puede ser infinita cuando $\mu$ es finita, y considere las expresiones l\'imite:
\begin{eqnarray}
lim_{n\rightarrow\infty}n^{-1}Z\left(T_{n}\right)&=&a,\textrm{ c.s.}\\
lim_{t\rightarrow\infty}t^{-1}Z\left(t\right)&=&a/\mu,\textrm{ c.s.}
\end{eqnarray}
La segunda expresi\'on implica la primera. Conversamente, la primera implica la segunda si el proceso $Z\left(t\right)$ es creciente, o si $lim_{n\rightarrow\infty}n^{-1}M_{n}=0$ c.s.
\end{Teo}

\begin{Coro}
Si $N\left(t\right)$ es un proceso de renovaci\'on, y $\left(Z\left(T_{n}\right)-Z\left(T_{n-1}\right),M_{n}\right)$, para $n\geq1$, son variables aleatorias independientes e id\'enticamente distribuidas con media finita, entonces,
\begin{eqnarray}
lim_{t\rightarrow\infty}t^{-1}Z\left(t\right)\rightarrow\frac{\esp\left[Z\left(T_{1}\right)-Z\left(T_{0}\right)\right]}{\esp\left[T_{1}\right]},\textrm{ c.s. cuando  }t\rightarrow\infty.
\end{eqnarray}
\end{Coro}
%___________________________________________________________________________________________
%
%\subsection{Propiedades de los Procesos de Renovaci\'on}
%___________________________________________________________________________________________
%

Los tiempos $T_{n}$ est\'an relacionados con los conteos de $N\left(t\right)$ por

\begin{eqnarray*}
\left\{N\left(t\right)\geq n\right\}&=&\left\{T_{n}\leq t\right\}\\
T_{N\left(t\right)}\leq &t&<T_{N\left(t\right)+1},
\end{eqnarray*}

adem\'as $N\left(T_{n}\right)=n$, y 

\begin{eqnarray*}
N\left(t\right)=\max\left\{n:T_{n}\leq t\right\}=\min\left\{n:T_{n+1}>t\right\}
\end{eqnarray*}

Por propiedades de la convoluci\'on se sabe que

\begin{eqnarray*}
P\left\{T_{n}\leq t\right\}=F^{n\star}\left(t\right)
\end{eqnarray*}
que es la $n$-\'esima convoluci\'on de $F$. Entonces 

\begin{eqnarray*}
\left\{N\left(t\right)\geq n\right\}&=&\left\{T_{n}\leq t\right\}\\
P\left\{N\left(t\right)\leq n\right\}&=&1-F^{\left(n+1\right)\star}\left(t\right)
\end{eqnarray*}

Adem\'as usando el hecho de que $\esp\left[N\left(t\right)\right]=\sum_{n=1}^{\infty}P\left\{N\left(t\right)\geq n\right\}$
se tiene que

\begin{eqnarray*}
\esp\left[N\left(t\right)\right]=\sum_{n=1}^{\infty}F^{n\star}\left(t\right)
\end{eqnarray*}

\begin{Prop}
Para cada $t\geq0$, la funci\'on generadora de momentos $\esp\left[e^{\alpha N\left(t\right)}\right]$ existe para alguna $\alpha$ en una vecindad del 0, y de aqu\'i que $\esp\left[N\left(t\right)^{m}\right]<\infty$, para $m\geq1$.
\end{Prop}


\begin{Note}
Si el primer tiempo de renovaci\'on $\xi_{1}$ no tiene la misma distribuci\'on que el resto de las $\xi_{n}$, para $n\geq2$, a $N\left(t\right)$ se le llama Proceso de Renovaci\'on retardado, donde si $\xi$ tiene distribuci\'on $G$, entonces el tiempo $T_{n}$ de la $n$-\'esima renovaci\'on tiene distribuci\'on $G\star F^{\left(n-1\right)\star}\left(t\right)$
\end{Note}


\begin{Teo}
Para una constante $\mu\leq\infty$ ( o variable aleatoria), las siguientes expresiones son equivalentes:

\begin{eqnarray}
lim_{n\rightarrow\infty}n^{-1}T_{n}&=&\mu,\textrm{ c.s.}\\
lim_{t\rightarrow\infty}t^{-1}N\left(t\right)&=&1/\mu,\textrm{ c.s.}
\end{eqnarray}
\end{Teo}


Es decir, $T_{n}$ satisface la Ley Fuerte de los Grandes N\'umeros s\'i y s\'olo s\'i $N\left/t\right)$ la cumple.


\begin{Coro}[Ley Fuerte de los Grandes N\'umeros para Procesos de Renovaci\'on]
Si $N\left(t\right)$ es un proceso de renovaci\'on cuyos tiempos de inter-renovaci\'on tienen media $\mu\leq\infty$, entonces
\begin{eqnarray}
t^{-1}N\left(t\right)\rightarrow 1/\mu,\textrm{ c.s. cuando }t\rightarrow\infty.
\end{eqnarray}

\end{Coro}


Considerar el proceso estoc\'astico de valores reales $\left\{Z\left(t\right):t\geq0\right\}$ en el mismo espacio de probabilidad que $N\left(t\right)$

\begin{Def}
Para el proceso $\left\{Z\left(t\right):t\geq0\right\}$ se define la fluctuaci\'on m\'axima de $Z\left(t\right)$ en el intervalo $\left(T_{n-1},T_{n}\right]$:
\begin{eqnarray*}
M_{n}=\sup_{T_{n-1}<t\leq T_{n}}|Z\left(t\right)-Z\left(T_{n-1}\right)|
\end{eqnarray*}
\end{Def}

\begin{Teo}
Sup\'ongase que $n^{-1}T_{n}\rightarrow\mu$ c.s. cuando $n\rightarrow\infty$, donde $\mu\leq\infty$ es una constante o variable aleatoria. Sea $a$ una constante o variable aleatoria que puede ser infinita cuando $\mu$ es finita, y considere las expresiones l\'imite:
\begin{eqnarray}
lim_{n\rightarrow\infty}n^{-1}Z\left(T_{n}\right)&=&a,\textrm{ c.s.}\\
lim_{t\rightarrow\infty}t^{-1}Z\left(t\right)&=&a/\mu,\textrm{ c.s.}
\end{eqnarray}
La segunda expresi\'on implica la primera. Conversamente, la primera implica la segunda si el proceso $Z\left(t\right)$ es creciente, o si $lim_{n\rightarrow\infty}n^{-1}M_{n}=0$ c.s.
\end{Teo}

\begin{Coro}
Si $N\left(t\right)$ es un proceso de renovaci\'on, y $\left(Z\left(T_{n}\right)-Z\left(T_{n-1}\right),M_{n}\right)$, para $n\geq1$, son variables aleatorias independientes e id\'enticamente distribuidas con media finita, entonces,
\begin{eqnarray}
lim_{t\rightarrow\infty}t^{-1}Z\left(t\right)\rightarrow\frac{\esp\left[Z\left(T_{1}\right)-Z\left(T_{0}\right)\right]}{\esp\left[T_{1}\right]},\textrm{ c.s. cuando  }t\rightarrow\infty.
\end{eqnarray}
\end{Coro}


%___________________________________________________________________________________________
%
%\subsection{Funci\'on de Renovaci\'on}
%___________________________________________________________________________________________
%


\begin{Def}
Sea $h\left(t\right)$ funci\'on de valores reales en $\rea$ acotada en intervalos finitos e igual a cero para $t<0$ La ecuaci\'on de renovaci\'on para $h\left(t\right)$ y la distribuci\'on $F$ es

\begin{eqnarray}%\label{Ec.Renovacion}
H\left(t\right)=h\left(t\right)+\int_{\left[0,t\right]}H\left(t-s\right)dF\left(s\right)\textrm{,    }t\geq0,
\end{eqnarray}
donde $H\left(t\right)$ es una funci\'on de valores reales. Esto es $H=h+F\star H$. Decimos que $H\left(t\right)$ es soluci\'on de esta ecuaci\'on si satisface la ecuaci\'on, y es acotada en intervalos finitos e iguales a cero para $t<0$.
\end{Def}

\begin{Prop}
La funci\'on $U\star h\left(t\right)$ es la \'unica soluci\'on de la ecuaci\'on de renovaci\'on (\ref{Ec.Renovacion}).
\end{Prop}

\begin{Teo}[Teorema Renovaci\'on Elemental]
\begin{eqnarray*}
t^{-1}U\left(t\right)\rightarrow 1/\mu\textrm{,    cuando }t\rightarrow\infty.
\end{eqnarray*}
\end{Teo}

%___________________________________________________________________________________________
%
%\subsection{Funci\'on de Renovaci\'on}
%___________________________________________________________________________________________
%


Sup\'ongase que $N\left(t\right)$ es un proceso de renovaci\'on con distribuci\'on $F$ con media finita $\mu$.

\begin{Def}
La funci\'on de renovaci\'on asociada con la distribuci\'on $F$, del proceso $N\left(t\right)$, es
\begin{eqnarray*}
U\left(t\right)=\sum_{n=1}^{\infty}F^{n\star}\left(t\right),\textrm{   }t\geq0,
\end{eqnarray*}
donde $F^{0\star}\left(t\right)=\indora\left(t\geq0\right)$.
\end{Def}


\begin{Prop}
Sup\'ongase que la distribuci\'on de inter-renovaci\'on $F$ tiene densidad $f$. Entonces $U\left(t\right)$ tambi\'en tiene densidad, para $t>0$, y es $U^{'}\left(t\right)=\sum_{n=0}^{\infty}f^{n\star}\left(t\right)$. Adem\'as
\begin{eqnarray*}
\prob\left\{N\left(t\right)>N\left(t-\right)\right\}=0\textrm{,   }t\geq0.
\end{eqnarray*}
\end{Prop}

\begin{Def}
La Transformada de Laplace-Stieljes de $F$ est\'a dada por

\begin{eqnarray*}
\hat{F}\left(\alpha\right)=\int_{\rea_{+}}e^{-\alpha t}dF\left(t\right)\textrm{,  }\alpha\geq0.
\end{eqnarray*}
\end{Def}

Entonces

\begin{eqnarray*}
\hat{U}\left(\alpha\right)=\sum_{n=0}^{\infty}\hat{F^{n\star}}\left(\alpha\right)=\sum_{n=0}^{\infty}\hat{F}\left(\alpha\right)^{n}=\frac{1}{1-\hat{F}\left(\alpha\right)}.
\end{eqnarray*}


\begin{Prop}
La Transformada de Laplace $\hat{U}\left(\alpha\right)$ y $\hat{F}\left(\alpha\right)$ determina una a la otra de manera \'unica por la relaci\'on $\hat{U}\left(\alpha\right)=\frac{1}{1-\hat{F}\left(\alpha\right)}$.
\end{Prop}


\begin{Note}
Un proceso de renovaci\'on $N\left(t\right)$ cuyos tiempos de inter-renovaci\'on tienen media finita, es un proceso Poisson con tasa $\lambda$ si y s\'olo s\'i $\esp\left[U\left(t\right)\right]=\lambda t$, para $t\geq0$.
\end{Note}


\begin{Teo}
Sea $N\left(t\right)$ un proceso puntual simple con puntos de localizaci\'on $T_{n}$ tal que $\eta\left(t\right)=\esp\left[N\left(\right)\right]$ es finita para cada $t$. Entonces para cualquier funci\'on $f:\rea_{+}\rightarrow\rea$,
\begin{eqnarray*}
\esp\left[\sum_{n=1}^{N\left(\right)}f\left(T_{n}\right)\right]=\int_{\left(0,t\right]}f\left(s\right)d\eta\left(s\right)\textrm{,  }t\geq0,
\end{eqnarray*}
suponiendo que la integral exista. Adem\'as si $X_{1},X_{2},\ldots$ son variables aleatorias definidas en el mismo espacio de probabilidad que el proceso $N\left(t\right)$ tal que $\esp\left[X_{n}|T_{n}=s\right]=f\left(s\right)$, independiente de $n$. Entonces
\begin{eqnarray*}
\esp\left[\sum_{n=1}^{N\left(t\right)}X_{n}\right]=\int_{\left(0,t\right]}f\left(s\right)d\eta\left(s\right)\textrm{,  }t\geq0,
\end{eqnarray*} 
suponiendo que la integral exista. 
\end{Teo}

\begin{Coro}[Identidad de Wald para Renovaciones]
Para el proceso de renovaci\'on $N\left(t\right)$,
\begin{eqnarray*}
\esp\left[T_{N\left(t\right)+1}\right]=\mu\esp\left[N\left(t\right)+1\right]\textrm{,  }t\geq0,
\end{eqnarray*}  
\end{Coro}

%______________________________________________________________________
%\subsection{Procesos de Renovaci\'on}
%______________________________________________________________________

\begin{Def}%\label{Def.Tn}
Sean $0\leq T_{1}\leq T_{2}\leq \ldots$ son tiempos aleatorios infinitos en los cuales ocurren ciertos eventos. El n\'umero de tiempos $T_{n}$ en el intervalo $\left[0,t\right)$ es

\begin{eqnarray}
N\left(t\right)=\sum_{n=1}^{\infty}\indora\left(T_{n}\leq t\right),
\end{eqnarray}
para $t\geq0$.
\end{Def}

Si se consideran los puntos $T_{n}$ como elementos de $\rea_{+}$, y $N\left(t\right)$ es el n\'umero de puntos en $\rea$. El proceso denotado por $\left\{N\left(t\right):t\geq0\right\}$, denotado por $N\left(t\right)$, es un proceso puntual en $\rea_{+}$. Los $T_{n}$ son los tiempos de ocurrencia, el proceso puntual $N\left(t\right)$ es simple si su n\'umero de ocurrencias son distintas: $0<T_{1}<T_{2}<\ldots$ casi seguramente.

\begin{Def}
Un proceso puntual $N\left(t\right)$ es un proceso de renovaci\'on si los tiempos de interocurrencia $\xi_{n}=T_{n}-T_{n-1}$, para $n\geq1$, son independientes e identicamente distribuidos con distribuci\'on $F$, donde $F\left(0\right)=0$ y $T_{0}=0$. Los $T_{n}$ son llamados tiempos de renovaci\'on, referente a la independencia o renovaci\'on de la informaci\'on estoc\'astica en estos tiempos. Los $\xi_{n}$ son los tiempos de inter-renovaci\'on, y $N\left(t\right)$ es el n\'umero de renovaciones en el intervalo $\left[0,t\right)$
\end{Def}


\begin{Note}
Para definir un proceso de renovaci\'on para cualquier contexto, solamente hay que especificar una distribuci\'on $F$, con $F\left(0\right)=0$, para los tiempos de inter-renovaci\'on. La funci\'on $F$ en turno degune las otra variables aleatorias. De manera formal, existe un espacio de probabilidad y una sucesi\'on de variables aleatorias $\xi_{1},\xi_{2},\ldots$ definidas en este con distribuci\'on $F$. Entonces las otras cantidades son $T_{n}=\sum_{k=1}^{n}\xi_{k}$ y $N\left(t\right)=\sum_{n=1}^{\infty}\indora\left(T_{n}\leq t\right)$, donde $T_{n}\rightarrow\infty$ casi seguramente por la Ley Fuerte de los Grandes Números.
\end{Note}

%___________________________________________________________________________________________
%
%\subsection{Renewal and Regenerative Processes: Serfozo\cite{Serfozo}}
%___________________________________________________________________________________________
%
\begin{Def}%\label{Def.Tn}
Sean $0\leq T_{1}\leq T_{2}\leq \ldots$ son tiempos aleatorios infinitos en los cuales ocurren ciertos eventos. El n\'umero de tiempos $T_{n}$ en el intervalo $\left[0,t\right)$ es

\begin{eqnarray}
N\left(t\right)=\sum_{n=1}^{\infty}\indora\left(T_{n}\leq t\right),
\end{eqnarray}
para $t\geq0$.
\end{Def}

Si se consideran los puntos $T_{n}$ como elementos de $\rea_{+}$, y $N\left(t\right)$ es el n\'umero de puntos en $\rea$. El proceso denotado por $\left\{N\left(t\right):t\geq0\right\}$, denotado por $N\left(t\right)$, es un proceso puntual en $\rea_{+}$. Los $T_{n}$ son los tiempos de ocurrencia, el proceso puntual $N\left(t\right)$ es simple si su n\'umero de ocurrencias son distintas: $0<T_{1}<T_{2}<\ldots$ casi seguramente.

\begin{Def}
Un proceso puntual $N\left(t\right)$ es un proceso de renovaci\'on si los tiempos de interocurrencia $\xi_{n}=T_{n}-T_{n-1}$, para $n\geq1$, son independientes e identicamente distribuidos con distribuci\'on $F$, donde $F\left(0\right)=0$ y $T_{0}=0$. Los $T_{n}$ son llamados tiempos de renovaci\'on, referente a la independencia o renovaci\'on de la informaci\'on estoc\'astica en estos tiempos. Los $\xi_{n}$ son los tiempos de inter-renovaci\'on, y $N\left(t\right)$ es el n\'umero de renovaciones en el intervalo $\left[0,t\right)$
\end{Def}


\begin{Note}
Para definir un proceso de renovaci\'on para cualquier contexto, solamente hay que especificar una distribuci\'on $F$, con $F\left(0\right)=0$, para los tiempos de inter-renovaci\'on. La funci\'on $F$ en turno degune las otra variables aleatorias. De manera formal, existe un espacio de probabilidad y una sucesi\'on de variables aleatorias $\xi_{1},\xi_{2},\ldots$ definidas en este con distribuci\'on $F$. Entonces las otras cantidades son $T_{n}=\sum_{k=1}^{n}\xi_{k}$ y $N\left(t\right)=\sum_{n=1}^{\infty}\indora\left(T_{n}\leq t\right)$, donde $T_{n}\rightarrow\infty$ casi seguramente por la Ley Fuerte de los Grandes N\'umeros.
\end{Note}







Los tiempos $T_{n}$ est\'an relacionados con los conteos de $N\left(t\right)$ por

\begin{eqnarray*}
\left\{N\left(t\right)\geq n\right\}&=&\left\{T_{n}\leq t\right\}\\
T_{N\left(t\right)}\leq &t&<T_{N\left(t\right)+1},
\end{eqnarray*}

adem\'as $N\left(T_{n}\right)=n$, y 

\begin{eqnarray*}
N\left(t\right)=\max\left\{n:T_{n}\leq t\right\}=\min\left\{n:T_{n+1}>t\right\}
\end{eqnarray*}

Por propiedades de la convoluci\'on se sabe que

\begin{eqnarray*}
P\left\{T_{n}\leq t\right\}=F^{n\star}\left(t\right)
\end{eqnarray*}
que es la $n$-\'esima convoluci\'on de $F$. Entonces 

\begin{eqnarray*}
\left\{N\left(t\right)\geq n\right\}&=&\left\{T_{n}\leq t\right\}\\
P\left\{N\left(t\right)\leq n\right\}&=&1-F^{\left(n+1\right)\star}\left(t\right)
\end{eqnarray*}

Adem\'as usando el hecho de que $\esp\left[N\left(t\right)\right]=\sum_{n=1}^{\infty}P\left\{N\left(t\right)\geq n\right\}$
se tiene que

\begin{eqnarray*}
\esp\left[N\left(t\right)\right]=\sum_{n=1}^{\infty}F^{n\star}\left(t\right)
\end{eqnarray*}

\begin{Prop}
Para cada $t\geq0$, la funci\'on generadora de momentos $\esp\left[e^{\alpha N\left(t\right)}\right]$ existe para alguna $\alpha$ en una vecindad del 0, y de aqu\'i que $\esp\left[N\left(t\right)^{m}\right]<\infty$, para $m\geq1$.
\end{Prop}

\begin{Ejem}[\textbf{Proceso Poisson}]

Suponga que se tienen tiempos de inter-renovaci\'on \textit{i.i.d.} del proceso de renovaci\'on $N\left(t\right)$ tienen distribuci\'on exponencial $F\left(t\right)=q-e^{-\lambda t}$ con tasa $\lambda$. Entonces $N\left(t\right)$ es un proceso Poisson con tasa $\lambda$.

\end{Ejem}


\begin{Note}
Si el primer tiempo de renovaci\'on $\xi_{1}$ no tiene la misma distribuci\'on que el resto de las $\xi_{n}$, para $n\geq2$, a $N\left(t\right)$ se le llama Proceso de Renovaci\'on retardado, donde si $\xi$ tiene distribuci\'on $G$, entonces el tiempo $T_{n}$ de la $n$-\'esima renovaci\'on tiene distribuci\'on $G\star F^{\left(n-1\right)\star}\left(t\right)$
\end{Note}


\begin{Teo}
Para una constante $\mu\leq\infty$ ( o variable aleatoria), las siguientes expresiones son equivalentes:

\begin{eqnarray}
lim_{n\rightarrow\infty}n^{-1}T_{n}&=&\mu,\textrm{ c.s.}\\
lim_{t\rightarrow\infty}t^{-1}N\left(t\right)&=&1/\mu,\textrm{ c.s.}
\end{eqnarray}
\end{Teo}


Es decir, $T_{n}$ satisface la Ley Fuerte de los Grandes N\'umeros s\'i y s\'olo s\'i $N\left/t\right)$ la cumple.


\begin{Coro}[Ley Fuerte de los Grandes N\'umeros para Procesos de Renovaci\'on]
Si $N\left(t\right)$ es un proceso de renovaci\'on cuyos tiempos de inter-renovaci\'on tienen media $\mu\leq\infty$, entonces
\begin{eqnarray}
t^{-1}N\left(t\right)\rightarrow 1/\mu,\textrm{ c.s. cuando }t\rightarrow\infty.
\end{eqnarray}

\end{Coro}


Considerar el proceso estoc\'astico de valores reales $\left\{Z\left(t\right):t\geq0\right\}$ en el mismo espacio de probabilidad que $N\left(t\right)$

\begin{Def}
Para el proceso $\left\{Z\left(t\right):t\geq0\right\}$ se define la fluctuaci\'on m\'axima de $Z\left(t\right)$ en el intervalo $\left(T_{n-1},T_{n}\right]$:
\begin{eqnarray*}
M_{n}=\sup_{T_{n-1}<t\leq T_{n}}|Z\left(t\right)-Z\left(T_{n-1}\right)|
\end{eqnarray*}
\end{Def}

\begin{Teo}
Sup\'ongase que $n^{-1}T_{n}\rightarrow\mu$ c.s. cuando $n\rightarrow\infty$, donde $\mu\leq\infty$ es una constante o variable aleatoria. Sea $a$ una constante o variable aleatoria que puede ser infinita cuando $\mu$ es finita, y considere las expresiones l\'imite:
\begin{eqnarray}
lim_{n\rightarrow\infty}n^{-1}Z\left(T_{n}\right)&=&a,\textrm{ c.s.}\\
lim_{t\rightarrow\infty}t^{-1}Z\left(t\right)&=&a/\mu,\textrm{ c.s.}
\end{eqnarray}
La segunda expresi\'on implica la primera. Conversamente, la primera implica la segunda si el proceso $Z\left(t\right)$ es creciente, o si $lim_{n\rightarrow\infty}n^{-1}M_{n}=0$ c.s.
\end{Teo}

\begin{Coro}
Si $N\left(t\right)$ es un proceso de renovaci\'on, y $\left(Z\left(T_{n}\right)-Z\left(T_{n-1}\right),M_{n}\right)$, para $n\geq1$, son variables aleatorias independientes e id\'enticamente distribuidas con media finita, entonces,
\begin{eqnarray}
lim_{t\rightarrow\infty}t^{-1}Z\left(t\right)\rightarrow\frac{\esp\left[Z\left(T_{1}\right)-Z\left(T_{0}\right)\right]}{\esp\left[T_{1}\right]},\textrm{ c.s. cuando  }t\rightarrow\infty.
\end{eqnarray}
\end{Coro}


Sup\'ongase que $N\left(t\right)$ es un proceso de renovaci\'on con distribuci\'on $F$ con media finita $\mu$.

\begin{Def}
La funci\'on de renovaci\'on asociada con la distribuci\'on $F$, del proceso $N\left(t\right)$, es
\begin{eqnarray*}
U\left(t\right)=\sum_{n=1}^{\infty}F^{n\star}\left(t\right),\textrm{   }t\geq0,
\end{eqnarray*}
donde $F^{0\star}\left(t\right)=\indora\left(t\geq0\right)$.
\end{Def}


\begin{Prop}
Sup\'ongase que la distribuci\'on de inter-renovaci\'on $F$ tiene densidad $f$. Entonces $U\left(t\right)$ tambi\'en tiene densidad, para $t>0$, y es $U^{'}\left(t\right)=\sum_{n=0}^{\infty}f^{n\star}\left(t\right)$. Adem\'as
\begin{eqnarray*}
\prob\left\{N\left(t\right)>N\left(t-\right)\right\}=0\textrm{,   }t\geq0.
\end{eqnarray*}
\end{Prop}

\begin{Def}
La Transformada de Laplace-Stieljes de $F$ est\'a dada por

\begin{eqnarray*}
\hat{F}\left(\alpha\right)=\int_{\rea_{+}}e^{-\alpha t}dF\left(t\right)\textrm{,  }\alpha\geq0.
\end{eqnarray*}
\end{Def}

Entonces

\begin{eqnarray*}
\hat{U}\left(\alpha\right)=\sum_{n=0}^{\infty}\hat{F^{n\star}}\left(\alpha\right)=\sum_{n=0}^{\infty}\hat{F}\left(\alpha\right)^{n}=\frac{1}{1-\hat{F}\left(\alpha\right)}.
\end{eqnarray*}


\begin{Prop}
La Transformada de Laplace $\hat{U}\left(\alpha\right)$ y $\hat{F}\left(\alpha\right)$ determina una a la otra de manera \'unica por la relaci\'on $\hat{U}\left(\alpha\right)=\frac{1}{1-\hat{F}\left(\alpha\right)}$.
\end{Prop}


\begin{Note}
Un proceso de renovaci\'on $N\left(t\right)$ cuyos tiempos de inter-renovaci\'on tienen media finita, es un proceso Poisson con tasa $\lambda$ si y s\'olo s\'i $\esp\left[U\left(t\right)\right]=\lambda t$, para $t\geq0$.
\end{Note}


\begin{Teo}
Sea $N\left(t\right)$ un proceso puntual simple con puntos de localizaci\'on $T_{n}$ tal que $\eta\left(t\right)=\esp\left[N\left(\right)\right]$ es finita para cada $t$. Entonces para cualquier funci\'on $f:\rea_{+}\rightarrow\rea$,
\begin{eqnarray*}
\esp\left[\sum_{n=1}^{N\left(\right)}f\left(T_{n}\right)\right]=\int_{\left(0,t\right]}f\left(s\right)d\eta\left(s\right)\textrm{,  }t\geq0,
\end{eqnarray*}
suponiendo que la integral exista. Adem\'as si $X_{1},X_{2},\ldots$ son variables aleatorias definidas en el mismo espacio de probabilidad que el proceso $N\left(t\right)$ tal que $\esp\left[X_{n}|T_{n}=s\right]=f\left(s\right)$, independiente de $n$. Entonces
\begin{eqnarray*}
\esp\left[\sum_{n=1}^{N\left(t\right)}X_{n}\right]=\int_{\left(0,t\right]}f\left(s\right)d\eta\left(s\right)\textrm{,  }t\geq0,
\end{eqnarray*} 
suponiendo que la integral exista. 
\end{Teo}

\begin{Coro}[Identidad de Wald para Renovaciones]
Para el proceso de renovaci\'on $N\left(t\right)$,
\begin{eqnarray*}
\esp\left[T_{N\left(t\right)+1}\right]=\mu\esp\left[N\left(t\right)+1\right]\textrm{,  }t\geq0,
\end{eqnarray*}  
\end{Coro}


\begin{Def}
Sea $h\left(t\right)$ funci\'on de valores reales en $\rea$ acotada en intervalos finitos e igual a cero para $t<0$ La ecuaci\'on de renovaci\'on para $h\left(t\right)$ y la distribuci\'on $F$ es

\begin{eqnarray}%\label{Ec.Renovacion}
H\left(t\right)=h\left(t\right)+\int_{\left[0,t\right]}H\left(t-s\right)dF\left(s\right)\textrm{,    }t\geq0,
\end{eqnarray}
donde $H\left(t\right)$ es una funci\'on de valores reales. Esto es $H=h+F\star H$. Decimos que $H\left(t\right)$ es soluci\'on de esta ecuaci\'on si satisface la ecuaci\'on, y es acotada en intervalos finitos e iguales a cero para $t<0$.
\end{Def}

\begin{Prop}
La funci\'on $U\star h\left(t\right)$ es la \'unica soluci\'on de la ecuaci\'on de renovaci\'on (\ref{Ec.Renovacion}).
\end{Prop}

\begin{Teo}[Teorema Renovaci\'on Elemental]
\begin{eqnarray*}
t^{-1}U\left(t\right)\rightarrow 1/\mu\textrm{,    cuando }t\rightarrow\infty.
\end{eqnarray*}
\end{Teo}



Sup\'ongase que $N\left(t\right)$ es un proceso de renovaci\'on con distribuci\'on $F$ con media finita $\mu$.

\begin{Def}
La funci\'on de renovaci\'on asociada con la distribuci\'on $F$, del proceso $N\left(t\right)$, es
\begin{eqnarray*}
U\left(t\right)=\sum_{n=1}^{\infty}F^{n\star}\left(t\right),\textrm{   }t\geq0,
\end{eqnarray*}
donde $F^{0\star}\left(t\right)=\indora\left(t\geq0\right)$.
\end{Def}


\begin{Prop}
Sup\'ongase que la distribuci\'on de inter-renovaci\'on $F$ tiene densidad $f$. Entonces $U\left(t\right)$ tambi\'en tiene densidad, para $t>0$, y es $U^{'}\left(t\right)=\sum_{n=0}^{\infty}f^{n\star}\left(t\right)$. Adem\'as
\begin{eqnarray*}
\prob\left\{N\left(t\right)>N\left(t-\right)\right\}=0\textrm{,   }t\geq0.
\end{eqnarray*}
\end{Prop}

\begin{Def}
La Transformada de Laplace-Stieljes de $F$ est\'a dada por

\begin{eqnarray*}
\hat{F}\left(\alpha\right)=\int_{\rea_{+}}e^{-\alpha t}dF\left(t\right)\textrm{,  }\alpha\geq0.
\end{eqnarray*}
\end{Def}

Entonces

\begin{eqnarray*}
\hat{U}\left(\alpha\right)=\sum_{n=0}^{\infty}\hat{F^{n\star}}\left(\alpha\right)=\sum_{n=0}^{\infty}\hat{F}\left(\alpha\right)^{n}=\frac{1}{1-\hat{F}\left(\alpha\right)}.
\end{eqnarray*}


\begin{Prop}
La Transformada de Laplace $\hat{U}\left(\alpha\right)$ y $\hat{F}\left(\alpha\right)$ determina una a la otra de manera \'unica por la relaci\'on $\hat{U}\left(\alpha\right)=\frac{1}{1-\hat{F}\left(\alpha\right)}$.
\end{Prop}


\begin{Note}
Un proceso de renovaci\'on $N\left(t\right)$ cuyos tiempos de inter-renovaci\'on tienen media finita, es un proceso Poisson con tasa $\lambda$ si y s\'olo s\'i $\esp\left[U\left(t\right)\right]=\lambda t$, para $t\geq0$.
\end{Note}


\begin{Teo}
Sea $N\left(t\right)$ un proceso puntual simple con puntos de localizaci\'on $T_{n}$ tal que $\eta\left(t\right)=\esp\left[N\left(\right)\right]$ es finita para cada $t$. Entonces para cualquier funci\'on $f:\rea_{+}\rightarrow\rea$,
\begin{eqnarray*}
\esp\left[\sum_{n=1}^{N\left(\right)}f\left(T_{n}\right)\right]=\int_{\left(0,t\right]}f\left(s\right)d\eta\left(s\right)\textrm{,  }t\geq0,
\end{eqnarray*}
suponiendo que la integral exista. Adem\'as si $X_{1},X_{2},\ldots$ son variables aleatorias definidas en el mismo espacio de probabilidad que el proceso $N\left(t\right)$ tal que $\esp\left[X_{n}|T_{n}=s\right]=f\left(s\right)$, independiente de $n$. Entonces
\begin{eqnarray*}
\esp\left[\sum_{n=1}^{N\left(t\right)}X_{n}\right]=\int_{\left(0,t\right]}f\left(s\right)d\eta\left(s\right)\textrm{,  }t\geq0,
\end{eqnarray*} 
suponiendo que la integral exista. 
\end{Teo}

\begin{Coro}[Identidad de Wald para Renovaciones]
Para el proceso de renovaci\'on $N\left(t\right)$,
\begin{eqnarray*}
\esp\left[T_{N\left(t\right)+1}\right]=\mu\esp\left[N\left(t\right)+1\right]\textrm{,  }t\geq0,
\end{eqnarray*}  
\end{Coro}


\begin{Def}
Sea $h\left(t\right)$ funci\'on de valores reales en $\rea$ acotada en intervalos finitos e igual a cero para $t<0$ La ecuaci\'on de renovaci\'on para $h\left(t\right)$ y la distribuci\'on $F$ es

\begin{eqnarray}%\label{Ec.Renovacion}
H\left(t\right)=h\left(t\right)+\int_{\left[0,t\right]}H\left(t-s\right)dF\left(s\right)\textrm{,    }t\geq0,
\end{eqnarray}
donde $H\left(t\right)$ es una funci\'on de valores reales. Esto es $H=h+F\star H$. Decimos que $H\left(t\right)$ es soluci\'on de esta ecuaci\'on si satisface la ecuaci\'on, y es acotada en intervalos finitos e iguales a cero para $t<0$.
\end{Def}

\begin{Prop}
La funci\'on $U\star h\left(t\right)$ es la \'unica soluci\'on de la ecuaci\'on de renovaci\'on (\ref{Ec.Renovacion}).
\end{Prop}

\begin{Teo}[Teorema Renovaci\'on Elemental]
\begin{eqnarray*}
t^{-1}U\left(t\right)\rightarrow 1/\mu\textrm{,    cuando }t\rightarrow\infty.
\end{eqnarray*}
\end{Teo}


\begin{Note} Una funci\'on $h:\rea_{+}\rightarrow\rea$ es Directamente Riemann Integrable en los siguientes casos:
\begin{itemize}
\item[a)] $h\left(t\right)\geq0$ es decreciente y Riemann Integrable.
\item[b)] $h$ es continua excepto posiblemente en un conjunto de Lebesgue de medida 0, y $|h\left(t\right)|\leq b\left(t\right)$, donde $b$ es DRI.
\end{itemize}
\end{Note}

\begin{Teo}[Teorema Principal de Renovaci\'on]
Si $F$ es no aritm\'etica y $h\left(t\right)$ es Directamente Riemann Integrable (DRI), entonces

\begin{eqnarray*}
lim_{t\rightarrow\infty}U\star h=\frac{1}{\mu}\int_{\rea_{+}}h\left(s\right)ds.
\end{eqnarray*}
\end{Teo}

\begin{Prop}
Cualquier funci\'on $H\left(t\right)$ acotada en intervalos finitos y que es 0 para $t<0$ puede expresarse como
\begin{eqnarray*}
H\left(t\right)=U\star h\left(t\right)\textrm{,  donde }h\left(t\right)=H\left(t\right)-F\star H\left(t\right)
\end{eqnarray*}
\end{Prop}

\begin{Def}
Un proceso estoc\'astico $X\left(t\right)$ es crudamente regenerativo en un tiempo aleatorio positivo $T$ si
\begin{eqnarray*}
\esp\left[X\left(T+t\right)|T\right]=\esp\left[X\left(t\right)\right]\textrm{, para }t\geq0,\end{eqnarray*}
y con las esperanzas anteriores finitas.
\end{Def}

\begin{Prop}
Sup\'ongase que $X\left(t\right)$ es un proceso crudamente regenerativo en $T$, que tiene distribuci\'on $F$. Si $\esp\left[X\left(t\right)\right]$ es acotado en intervalos finitos, entonces
\begin{eqnarray*}
\esp\left[X\left(t\right)\right]=U\star h\left(t\right)\textrm{,  donde }h\left(t\right)=\esp\left[X\left(t\right)\indora\left(T>t\right)\right].
\end{eqnarray*}
\end{Prop}

\begin{Teo}[Regeneraci\'on Cruda]
Sup\'ongase que $X\left(t\right)$ es un proceso con valores positivo crudamente regenerativo en $T$, y def\'inase $M=\sup\left\{|X\left(t\right)|:t\leq T\right\}$. Si $T$ es no aritm\'etico y $M$ y $MT$ tienen media finita, entonces
\begin{eqnarray*}
lim_{t\rightarrow\infty}\esp\left[X\left(t\right)\right]=\frac{1}{\mu}\int_{\rea_{+}}h\left(s\right)ds,
\end{eqnarray*}
donde $h\left(t\right)=\esp\left[X\left(t\right)\indora\left(T>t\right)\right]$.
\end{Teo}


\begin{Note} Una funci\'on $h:\rea_{+}\rightarrow\rea$ es Directamente Riemann Integrable en los siguientes casos:
\begin{itemize}
\item[a)] $h\left(t\right)\geq0$ es decreciente y Riemann Integrable.
\item[b)] $h$ es continua excepto posiblemente en un conjunto de Lebesgue de medida 0, y $|h\left(t\right)|\leq b\left(t\right)$, donde $b$ es DRI.
\end{itemize}
\end{Note}

\begin{Teo}[Teorema Principal de Renovaci\'on]
Si $F$ es no aritm\'etica y $h\left(t\right)$ es Directamente Riemann Integrable (DRI), entonces

\begin{eqnarray*}
lim_{t\rightarrow\infty}U\star h=\frac{1}{\mu}\int_{\rea_{+}}h\left(s\right)ds.
\end{eqnarray*}
\end{Teo}

\begin{Prop}
Cualquier funci\'on $H\left(t\right)$ acotada en intervalos finitos y que es 0 para $t<0$ puede expresarse como
\begin{eqnarray*}
H\left(t\right)=U\star h\left(t\right)\textrm{,  donde }h\left(t\right)=H\left(t\right)-F\star H\left(t\right)
\end{eqnarray*}
\end{Prop}

\begin{Def}
Un proceso estoc\'astico $X\left(t\right)$ es crudamente regenerativo en un tiempo aleatorio positivo $T$ si
\begin{eqnarray*}
\esp\left[X\left(T+t\right)|T\right]=\esp\left[X\left(t\right)\right]\textrm{, para }t\geq0,\end{eqnarray*}
y con las esperanzas anteriores finitas.
\end{Def}

\begin{Prop}
Sup\'ongase que $X\left(t\right)$ es un proceso crudamente regenerativo en $T$, que tiene distribuci\'on $F$. Si $\esp\left[X\left(t\right)\right]$ es acotado en intervalos finitos, entonces
\begin{eqnarray*}
\esp\left[X\left(t\right)\right]=U\star h\left(t\right)\textrm{,  donde }h\left(t\right)=\esp\left[X\left(t\right)\indora\left(T>t\right)\right].
\end{eqnarray*}
\end{Prop}

\begin{Teo}[Regeneraci\'on Cruda]
Sup\'ongase que $X\left(t\right)$ es un proceso con valores positivo crudamente regenerativo en $T$, y def\'inase $M=\sup\left\{|X\left(t\right)|:t\leq T\right\}$. Si $T$ es no aritm\'etico y $M$ y $MT$ tienen media finita, entonces
\begin{eqnarray*}
lim_{t\rightarrow\infty}\esp\left[X\left(t\right)\right]=\frac{1}{\mu}\int_{\rea_{+}}h\left(s\right)ds,
\end{eqnarray*}
donde $h\left(t\right)=\esp\left[X\left(t\right)\indora\left(T>t\right)\right]$.
\end{Teo}

\begin{Def}
Para el proceso $\left\{\left(N\left(t\right),X\left(t\right)\right):t\geq0\right\}$, sus trayectoria muestrales en el intervalo de tiempo $\left[T_{n-1},T_{n}\right)$ est\'an descritas por
\begin{eqnarray*}
\zeta_{n}=\left(\xi_{n},\left\{X\left(T_{n-1}+t\right):0\leq t<\xi_{n}\right\}\right)
\end{eqnarray*}
Este $\zeta_{n}$ es el $n$-\'esimo segmento del proceso. El proceso es regenerativo sobre los tiempos $T_{n}$ si sus segmentos $\zeta_{n}$ son independientes e id\'enticamennte distribuidos.
\end{Def}


\begin{Note}
Si $\tilde{X}\left(t\right)$ con espacio de estados $\tilde{S}$ es regenerativo sobre $T_{n}$, entonces $X\left(t\right)=f\left(\tilde{X}\left(t\right)\right)$ tambi\'en es regenerativo sobre $T_{n}$, para cualquier funci\'on $f:\tilde{S}\rightarrow S$.
\end{Note}

\begin{Note}
Los procesos regenerativos son crudamente regenerativos, pero no al rev\'es.
\end{Note}


\begin{Note}
Un proceso estoc\'astico a tiempo continuo o discreto es regenerativo si existe un proceso de renovaci\'on  tal que los segmentos del proceso entre tiempos de renovaci\'on sucesivos son i.i.d., es decir, para $\left\{X\left(t\right):t\geq0\right\}$ proceso estoc\'astico a tiempo continuo con espacio de estados $S$, espacio m\'etrico.
\end{Note}

Para $\left\{X\left(t\right):t\geq0\right\}$ Proceso Estoc\'astico a tiempo continuo con estado de espacios $S$, que es un espacio m\'etrico, con trayectorias continuas por la derecha y con l\'imites por la izquierda c.s. Sea $N\left(t\right)$ un proceso de renovaci\'on en $\rea_{+}$ definido en el mismo espacio de probabilidad que $X\left(t\right)$, con tiempos de renovaci\'on $T$ y tiempos de inter-renovaci\'on $\xi_{n}=T_{n}-T_{n-1}$, con misma distribuci\'on $F$ de media finita $\mu$.



\begin{Def}
Para el proceso $\left\{\left(N\left(t\right),X\left(t\right)\right):t\geq0\right\}$, sus trayectoria muestrales en el intervalo de tiempo $\left[T_{n-1},T_{n}\right)$ est\'an descritas por
\begin{eqnarray*}
\zeta_{n}=\left(\xi_{n},\left\{X\left(T_{n-1}+t\right):0\leq t<\xi_{n}\right\}\right)
\end{eqnarray*}
Este $\zeta_{n}$ es el $n$-\'esimo segmento del proceso. El proceso es regenerativo sobre los tiempos $T_{n}$ si sus segmentos $\zeta_{n}$ son independientes e id\'enticamennte distribuidos.
\end{Def}

\begin{Note}
Un proceso regenerativo con media de la longitud de ciclo finita es llamado positivo recurrente.
\end{Note}

\begin{Teo}[Procesos Regenerativos]
Suponga que el proceso
\end{Teo}


\begin{Def}[Renewal Process Trinity]
Para un proceso de renovaci\'on $N\left(t\right)$, los siguientes procesos proveen de informaci\'on sobre los tiempos de renovaci\'on.
\begin{itemize}
\item $A\left(t\right)=t-T_{N\left(t\right)}$, el tiempo de recurrencia hacia atr\'as al tiempo $t$, que es el tiempo desde la \'ultima renovaci\'on para $t$.

\item $B\left(t\right)=T_{N\left(t\right)+1}-t$, el tiempo de recurrencia hacia adelante al tiempo $t$, residual del tiempo de renovaci\'on, que es el tiempo para la pr\'oxima renovaci\'on despu\'es de $t$.

\item $L\left(t\right)=\xi_{N\left(t\right)+1}=A\left(t\right)+B\left(t\right)$, la longitud del intervalo de renovaci\'on que contiene a $t$.
\end{itemize}
\end{Def}

\begin{Note}
El proceso tridimensional $\left(A\left(t\right),B\left(t\right),L\left(t\right)\right)$ es regenerativo sobre $T_{n}$, y por ende cada proceso lo es. Cada proceso $A\left(t\right)$ y $B\left(t\right)$ son procesos de MArkov a tiempo continuo con trayectorias continuas por partes en el espacio de estados $\rea_{+}$. Una expresi\'on conveniente para su distribuci\'on conjunta es, para $0\leq x<t,y\geq0$
\begin{equation}\label{NoRenovacion}
P\left\{A\left(t\right)>x,B\left(t\right)>y\right\}=
P\left\{N\left(t+y\right)-N\left((t-x)\right)=0\right\}
\end{equation}
\end{Note}

\begin{Ejem}[Tiempos de recurrencia Poisson]
Si $N\left(t\right)$ es un proceso Poisson con tasa $\lambda$, entonces de la expresi\'on (\ref{NoRenovacion}) se tiene que

\begin{eqnarray*}
\begin{array}{lc}
P\left\{A\left(t\right)>x,B\left(t\right)>y\right\}=e^{-\lambda\left(x+y\right)},&0\leq x<t,y\geq0,
\end{array}
\end{eqnarray*}
que es la probabilidad Poisson de no renovaciones en un intervalo de longitud $x+y$.

\end{Ejem}

\begin{Note}
Una cadena de Markov erg\'odica tiene la propiedad de ser estacionaria si la distribuci\'on de su estado al tiempo $0$ es su distribuci\'on estacionaria.
\end{Note}


\begin{Def}
Un proceso estoc\'astico a tiempo continuo $\left\{X\left(t\right):t\geq0\right\}$ en un espacio general es estacionario si sus distribuciones finito dimensionales son invariantes bajo cualquier  traslado: para cada $0\leq s_{1}<s_{2}<\cdots<s_{k}$ y $t\geq0$,
\begin{eqnarray*}
\left(X\left(s_{1}+t\right),\ldots,X\left(s_{k}+t\right)\right)=_{d}\left(X\left(s_{1}\right),\ldots,X\left(s_{k}\right)\right).
\end{eqnarray*}
\end{Def}

\begin{Note}
Un proceso de Markov es estacionario si $X\left(t\right)=_{d}X\left(0\right)$, $t\geq0$.
\end{Note}

Considerese el proceso $N\left(t\right)=\sum_{n}\indora\left(\tau_{n}\leq t\right)$ en $\rea_{+}$, con puntos $0<\tau_{1}<\tau_{2}<\cdots$.

\begin{Prop}
Si $N$ es un proceso puntual estacionario y $\esp\left[N\left(1\right)\right]<\infty$, entonces $\esp\left[N\left(t\right)\right]=t\esp\left[N\left(1\right)\right]$, $t\geq0$

\end{Prop}

\begin{Teo}
Los siguientes enunciados son equivalentes
\begin{itemize}
\item[i)] El proceso retardado de renovaci\'on $N$ es estacionario.

\item[ii)] EL proceso de tiempos de recurrencia hacia adelante $B\left(t\right)$ es estacionario.


\item[iii)] $\esp\left[N\left(t\right)\right]=t/\mu$,


\item[iv)] $G\left(t\right)=F_{e}\left(t\right)=\frac{1}{\mu}\int_{0}^{t}\left[1-F\left(s\right)\right]ds$
\end{itemize}
Cuando estos enunciados son ciertos, $P\left\{B\left(t\right)\leq x\right\}=F_{e}\left(x\right)$, para $t,x\geq0$.

\end{Teo}

\begin{Note}
Una consecuencia del teorema anterior es que el Proceso Poisson es el \'unico proceso sin retardo que es estacionario.
\end{Note}

\begin{Coro}
El proceso de renovaci\'on $N\left(t\right)$ sin retardo, y cuyos tiempos de inter renonaci\'on tienen media finita, es estacionario si y s\'olo si es un proceso Poisson.

\end{Coro}


%________________________________________________________________________
%\subsection{Procesos Regenerativos}
%________________________________________________________________________



\begin{Note}
Si $\tilde{X}\left(t\right)$ con espacio de estados $\tilde{S}$ es regenerativo sobre $T_{n}$, entonces $X\left(t\right)=f\left(\tilde{X}\left(t\right)\right)$ tambi\'en es regenerativo sobre $T_{n}$, para cualquier funci\'on $f:\tilde{S}\rightarrow S$.
\end{Note}

\begin{Note}
Los procesos regenerativos son crudamente regenerativos, pero no al rev\'es.
\end{Note}
%\subsection*{Procesos Regenerativos: Sigman\cite{Sigman1}}
\begin{Def}[Definici\'on Cl\'asica]
Un proceso estoc\'astico $X=\left\{X\left(t\right):t\geq0\right\}$ es llamado regenerativo is existe una variable aleatoria $R_{1}>0$ tal que
\begin{itemize}
\item[i)] $\left\{X\left(t+R_{1}\right):t\geq0\right\}$ es independiente de $\left\{\left\{X\left(t\right):t<R_{1}\right\},\right\}$
\item[ii)] $\left\{X\left(t+R_{1}\right):t\geq0\right\}$ es estoc\'asticamente equivalente a $\left\{X\left(t\right):t>0\right\}$
\end{itemize}

Llamamos a $R_{1}$ tiempo de regeneraci\'on, y decimos que $X$ se regenera en este punto.
\end{Def}

$\left\{X\left(t+R_{1}\right)\right\}$ es regenerativo con tiempo de regeneraci\'on $R_{2}$, independiente de $R_{1}$ pero con la misma distribuci\'on que $R_{1}$. Procediendo de esta manera se obtiene una secuencia de variables aleatorias independientes e id\'enticamente distribuidas $\left\{R_{n}\right\}$ llamados longitudes de ciclo. Si definimos a $Z_{k}\equiv R_{1}+R_{2}+\cdots+R_{k}$, se tiene un proceso de renovaci\'on llamado proceso de renovaci\'on encajado para $X$.




\begin{Def}
Para $x$ fijo y para cada $t\geq0$, sea $I_{x}\left(t\right)=1$ si $X\left(t\right)\leq x$,  $I_{x}\left(t\right)=0$ en caso contrario, y def\'inanse los tiempos promedio
\begin{eqnarray*}
\overline{X}&=&lim_{t\rightarrow\infty}\frac{1}{t}\int_{0}^{\infty}X\left(u\right)du\\
\prob\left(X_{\infty}\leq x\right)&=&lim_{t\rightarrow\infty}\frac{1}{t}\int_{0}^{\infty}I_{x}\left(u\right)du,
\end{eqnarray*}
cuando estos l\'imites existan.
\end{Def}

Como consecuencia del teorema de Renovaci\'on-Recompensa, se tiene que el primer l\'imite  existe y es igual a la constante
\begin{eqnarray*}
\overline{X}&=&\frac{\esp\left[\int_{0}^{R_{1}}X\left(t\right)dt\right]}{\esp\left[R_{1}\right]},
\end{eqnarray*}
suponiendo que ambas esperanzas son finitas.

\begin{Note}
\begin{itemize}
\item[a)] Si el proceso regenerativo $X$ es positivo recurrente y tiene trayectorias muestrales no negativas, entonces la ecuaci\'on anterior es v\'alida.
\item[b)] Si $X$ es positivo recurrente regenerativo, podemos construir una \'unica versi\'on estacionaria de este proceso, $X_{e}=\left\{X_{e}\left(t\right)\right\}$, donde $X_{e}$ es un proceso estoc\'astico regenerativo y estrictamente estacionario, con distribuci\'on marginal distribuida como $X_{\infty}$
\end{itemize}
\end{Note}

%________________________________________________________________________
%\subsection{Procesos Regenerativos}
%________________________________________________________________________

Para $\left\{X\left(t\right):t\geq0\right\}$ Proceso Estoc\'astico a tiempo continuo con estado de espacios $S$, que es un espacio m\'etrico, con trayectorias continuas por la derecha y con l\'imites por la izquierda c.s. Sea $N\left(t\right)$ un proceso de renovaci\'on en $\rea_{+}$ definido en el mismo espacio de probabilidad que $X\left(t\right)$, con tiempos de renovaci\'on $T$ y tiempos de inter-renovaci\'on $\xi_{n}=T_{n}-T_{n-1}$, con misma distribuci\'on $F$ de media finita $\mu$.



\begin{Def}
Para el proceso $\left\{\left(N\left(t\right),X\left(t\right)\right):t\geq0\right\}$, sus trayectoria muestrales en el intervalo de tiempo $\left[T_{n-1},T_{n}\right)$ est\'an descritas por
\begin{eqnarray*}
\zeta_{n}=\left(\xi_{n},\left\{X\left(T_{n-1}+t\right):0\leq t<\xi_{n}\right\}\right)
\end{eqnarray*}
Este $\zeta_{n}$ es el $n$-\'esimo segmento del proceso. El proceso es regenerativo sobre los tiempos $T_{n}$ si sus segmentos $\zeta_{n}$ son independientes e id\'enticamennte distribuidos.
\end{Def}


\begin{Note}
Si $\tilde{X}\left(t\right)$ con espacio de estados $\tilde{S}$ es regenerativo sobre $T_{n}$, entonces $X\left(t\right)=f\left(\tilde{X}\left(t\right)\right)$ tambi\'en es regenerativo sobre $T_{n}$, para cualquier funci\'on $f:\tilde{S}\rightarrow S$.
\end{Note}

\begin{Note}
Los procesos regenerativos son crudamente regenerativos, pero no al rev\'es.
\end{Note}

\begin{Def}[Definici\'on Cl\'asica]
Un proceso estoc\'astico $X=\left\{X\left(t\right):t\geq0\right\}$ es llamado regenerativo is existe una variable aleatoria $R_{1}>0$ tal que
\begin{itemize}
\item[i)] $\left\{X\left(t+R_{1}\right):t\geq0\right\}$ es independiente de $\left\{\left\{X\left(t\right):t<R_{1}\right\},\right\}$
\item[ii)] $\left\{X\left(t+R_{1}\right):t\geq0\right\}$ es estoc\'asticamente equivalente a $\left\{X\left(t\right):t>0\right\}$
\end{itemize}

Llamamos a $R_{1}$ tiempo de regeneraci\'on, y decimos que $X$ se regenera en este punto.
\end{Def}

$\left\{X\left(t+R_{1}\right)\right\}$ es regenerativo con tiempo de regeneraci\'on $R_{2}$, independiente de $R_{1}$ pero con la misma distribuci\'on que $R_{1}$. Procediendo de esta manera se obtiene una secuencia de variables aleatorias independientes e id\'enticamente distribuidas $\left\{R_{n}\right\}$ llamados longitudes de ciclo. Si definimos a $Z_{k}\equiv R_{1}+R_{2}+\cdots+R_{k}$, se tiene un proceso de renovaci\'on llamado proceso de renovaci\'on encajado para $X$.

\begin{Note}
Un proceso regenerativo con media de la longitud de ciclo finita es llamado positivo recurrente.
\end{Note}


\begin{Def}
Para $x$ fijo y para cada $t\geq0$, sea $I_{x}\left(t\right)=1$ si $X\left(t\right)\leq x$,  $I_{x}\left(t\right)=0$ en caso contrario, y def\'inanse los tiempos promedio
\begin{eqnarray*}
\overline{X}&=&lim_{t\rightarrow\infty}\frac{1}{t}\int_{0}^{\infty}X\left(u\right)du\\
\prob\left(X_{\infty}\leq x\right)&=&lim_{t\rightarrow\infty}\frac{1}{t}\int_{0}^{\infty}I_{x}\left(u\right)du,
\end{eqnarray*}
cuando estos l\'imites existan.
\end{Def}

Como consecuencia del teorema de Renovaci\'on-Recompensa, se tiene que el primer l\'imite  existe y es igual a la constante
\begin{eqnarray*}
\overline{X}&=&\frac{\esp\left[\int_{0}^{R_{1}}X\left(t\right)dt\right]}{\esp\left[R_{1}\right]},
\end{eqnarray*}
suponiendo que ambas esperanzas son finitas.

\begin{Note}
\begin{itemize}
\item[a)] Si el proceso regenerativo $X$ es positivo recurrente y tiene trayectorias muestrales no negativas, entonces la ecuaci\'on anterior es v\'alida.
\item[b)] Si $X$ es positivo recurrente regenerativo, podemos construir una \'unica versi\'on estacionaria de este proceso, $X_{e}=\left\{X_{e}\left(t\right)\right\}$, donde $X_{e}$ es un proceso estoc\'astico regenerativo y estrictamente estacionario, con distribuci\'on marginal distribuida como $X_{\infty}$
\end{itemize}
\end{Note}

%__________________________________________________________________________________________
%\subsection{Procesos Regenerativos Estacionarios - Stidham \cite{Stidham}}
%__________________________________________________________________________________________


Un proceso estoc\'astico a tiempo continuo $\left\{V\left(t\right),t\geq0\right\}$ es un proceso regenerativo si existe una sucesi\'on de variables aleatorias independientes e id\'enticamente distribuidas $\left\{X_{1},X_{2},\ldots\right\}$, sucesi\'on de renovaci\'on, tal que para cualquier conjunto de Borel $A$, 

\begin{eqnarray*}
\prob\left\{V\left(t\right)\in A|X_{1}+X_{2}+\cdots+X_{R\left(t\right)}=s,\left\{V\left(\tau\right),\tau<s\right\}\right\}=\prob\left\{V\left(t-s\right)\in A|X_{1}>t-s\right\},
\end{eqnarray*}
para todo $0\leq s\leq t$, donde $R\left(t\right)=\max\left\{X_{1}+X_{2}+\cdots+X_{j}\leq t\right\}=$n\'umero de renovaciones ({\emph{puntos de regeneraci\'on}}) que ocurren en $\left[0,t\right]$. El intervalo $\left[0,X_{1}\right)$ es llamado {\emph{primer ciclo de regeneraci\'on}} de $\left\{V\left(t \right),t\geq0\right\}$, $\left[X_{1},X_{1}+X_{2}\right)$ el {\emph{segundo ciclo de regeneraci\'on}}, y as\'i sucesivamente.

Sea $X=X_{1}$ y sea $F$ la funci\'on de distrbuci\'on de $X$


\begin{Def}
Se define el proceso estacionario, $\left\{V^{*}\left(t\right),t\geq0\right\}$, para $\left\{V\left(t\right),t\geq0\right\}$ por

\begin{eqnarray*}
\prob\left\{V\left(t\right)\in A\right\}=\frac{1}{\esp\left[X\right]}\int_{0}^{\infty}\prob\left\{V\left(t+x\right)\in A|X>x\right\}\left(1-F\left(x\right)\right)dx,
\end{eqnarray*} 
para todo $t\geq0$ y todo conjunto de Borel $A$.
\end{Def}

\begin{Def}
Una distribuci\'on se dice que es {\emph{aritm\'etica}} si todos sus puntos de incremento son m\'ultiplos de la forma $0,\lambda, 2\lambda,\ldots$ para alguna $\lambda>0$ entera.
\end{Def}


\begin{Def}
Una modificaci\'on medible de un proceso $\left\{V\left(t\right),t\geq0\right\}$, es una versi\'on de este, $\left\{V\left(t,w\right)\right\}$ conjuntamente medible para $t\geq0$ y para $w\in S$, $S$ espacio de estados para $\left\{V\left(t\right),t\geq0\right\}$.
\end{Def}

\begin{Teo}
Sea $\left\{V\left(t\right),t\geq\right\}$ un proceso regenerativo no negativo con modificaci\'on medible. Sea $\esp\left[X\right]<\infty$. Entonces el proceso estacionario dado por la ecuaci\'on anterior est\'a bien definido y tiene funci\'on de distribuci\'on independiente de $t$, adem\'as
\begin{itemize}
\item[i)] \begin{eqnarray*}
\esp\left[V^{*}\left(0\right)\right]&=&\frac{\esp\left[\int_{0}^{X}V\left(s\right)ds\right]}{\esp\left[X\right]}\end{eqnarray*}
\item[ii)] Si $\esp\left[V^{*}\left(0\right)\right]<\infty$, equivalentemente, si $\esp\left[\int_{0}^{X}V\left(s\right)ds\right]<\infty$,entonces
\begin{eqnarray*}
\frac{\int_{0}^{t}V\left(s\right)ds}{t}\rightarrow\frac{\esp\left[\int_{0}^{X}V\left(s\right)ds\right]}{\esp\left[X\right]}
\end{eqnarray*}
con probabilidad 1 y en media, cuando $t\rightarrow\infty$.
\end{itemize}
\end{Teo}
%
%___________________________________________________________________________________________
%\vspace{5.5cm}
%\chapter{Cadenas de Markov estacionarias}
%\vspace{-1.0cm}


%__________________________________________________________________________________________
%\subsection{Procesos Regenerativos Estacionarios - Stidham \cite{Stidham}}
%__________________________________________________________________________________________


Un proceso estoc\'astico a tiempo continuo $\left\{V\left(t\right),t\geq0\right\}$ es un proceso regenerativo si existe una sucesi\'on de variables aleatorias independientes e id\'enticamente distribuidas $\left\{X_{1},X_{2},\ldots\right\}$, sucesi\'on de renovaci\'on, tal que para cualquier conjunto de Borel $A$, 

\begin{eqnarray*}
\prob\left\{V\left(t\right)\in A|X_{1}+X_{2}+\cdots+X_{R\left(t\right)}=s,\left\{V\left(\tau\right),\tau<s\right\}\right\}=\prob\left\{V\left(t-s\right)\in A|X_{1}>t-s\right\},
\end{eqnarray*}
para todo $0\leq s\leq t$, donde $R\left(t\right)=\max\left\{X_{1}+X_{2}+\cdots+X_{j}\leq t\right\}=$n\'umero de renovaciones ({\emph{puntos de regeneraci\'on}}) que ocurren en $\left[0,t\right]$. El intervalo $\left[0,X_{1}\right)$ es llamado {\emph{primer ciclo de regeneraci\'on}} de $\left\{V\left(t \right),t\geq0\right\}$, $\left[X_{1},X_{1}+X_{2}\right)$ el {\emph{segundo ciclo de regeneraci\'on}}, y as\'i sucesivamente.

Sea $X=X_{1}$ y sea $F$ la funci\'on de distrbuci\'on de $X$


\begin{Def}
Se define el proceso estacionario, $\left\{V^{*}\left(t\right),t\geq0\right\}$, para $\left\{V\left(t\right),t\geq0\right\}$ por

\begin{eqnarray*}
\prob\left\{V\left(t\right)\in A\right\}=\frac{1}{\esp\left[X\right]}\int_{0}^{\infty}\prob\left\{V\left(t+x\right)\in A|X>x\right\}\left(1-F\left(x\right)\right)dx,
\end{eqnarray*} 
para todo $t\geq0$ y todo conjunto de Borel $A$.
\end{Def}

\begin{Def}
Una distribuci\'on se dice que es {\emph{aritm\'etica}} si todos sus puntos de incremento son m\'ultiplos de la forma $0,\lambda, 2\lambda,\ldots$ para alguna $\lambda>0$ entera.
\end{Def}


\begin{Def}
Una modificaci\'on medible de un proceso $\left\{V\left(t\right),t\geq0\right\}$, es una versi\'on de este, $\left\{V\left(t,w\right)\right\}$ conjuntamente medible para $t\geq0$ y para $w\in S$, $S$ espacio de estados para $\left\{V\left(t\right),t\geq0\right\}$.
\end{Def}

\begin{Teo}
Sea $\left\{V\left(t\right),t\geq\right\}$ un proceso regenerativo no negativo con modificaci\'on medible. Sea $\esp\left[X\right]<\infty$. Entonces el proceso estacionario dado por la ecuaci\'on anterior est\'a bien definido y tiene funci\'on de distribuci\'on independiente de $t$, adem\'as
\begin{itemize}
\item[i)] \begin{eqnarray*}
\esp\left[V^{*}\left(0\right)\right]&=&\frac{\esp\left[\int_{0}^{X}V\left(s\right)ds\right]}{\esp\left[X\right]}\end{eqnarray*}
\item[ii)] Si $\esp\left[V^{*}\left(0\right)\right]<\infty$, equivalentemente, si $\esp\left[\int_{0}^{X}V\left(s\right)ds\right]<\infty$,entonces
\begin{eqnarray*}
\frac{\int_{0}^{t}V\left(s\right)ds}{t}\rightarrow\frac{\esp\left[\int_{0}^{X}V\left(s\right)ds\right]}{\esp\left[X\right]}
\end{eqnarray*}
con probabilidad 1 y en media, cuando $t\rightarrow\infty$.
\end{itemize}
\end{Teo}

Para $\left\{X\left(t\right):t\geq0\right\}$ Proceso Estoc\'astico a tiempo continuo con estado de espacios $S$, que es un espacio m\'etrico, con trayectorias continuas por la derecha y con l\'imites por la izquierda c.s. Sea $N\left(t\right)$ un proceso de renovaci\'on en $\rea_{+}$ definido en el mismo espacio de probabilidad que $X\left(t\right)$, con tiempos de renovaci\'on $T$ y tiempos de inter-renovaci\'on $\xi_{n}=T_{n}-T_{n-1}$, con misma distribuci\'on $F$ de media finita $\mu$.


%______________________________________________________________________
%\subsection{Ejemplos, Notas importantes}


Sean $T_{1},T_{2},\ldots$ los puntos donde las longitudes de las colas de la red de sistemas de visitas c\'iclicas son cero simult\'aneamente, cuando la cola $Q_{j}$ es visitada por el servidor para dar servicio, es decir, $L_{1}\left(T_{i}\right)=0,L_{2}\left(T_{i}\right)=0,\hat{L}_{1}\left(T_{i}\right)=0$ y $\hat{L}_{2}\left(T_{i}\right)=0$, a estos puntos se les denominar\'a puntos regenerativos. Sea la funci\'on generadora de momentos para $L_{i}$, el n\'umero de usuarios en la cola $Q_{i}\left(z\right)$ en cualquier momento, est\'a dada por el tiempo promedio de $z^{L_{i}\left(t\right)}$ sobre el ciclo regenerativo definido anteriormente:

\begin{eqnarray*}
Q_{i}\left(z\right)&=&\esp\left[z^{L_{i}\left(t\right)}\right]=\frac{\esp\left[\sum_{m=1}^{M_{i}}\sum_{t=\tau_{i}\left(m\right)}^{\tau_{i}\left(m+1\right)-1}z^{L_{i}\left(t\right)}\right]}{\esp\left[\sum_{m=1}^{M_{i}}\tau_{i}\left(m+1\right)-\tau_{i}\left(m\right)\right]}
\end{eqnarray*}

$M_{i}$ es un tiempo de paro en el proceso regenerativo con $\esp\left[M_{i}\right]<\infty$\footnote{En Stidham\cite{Stidham} y Heyman  se muestra que una condici\'on suficiente para que el proceso regenerativo 
estacionario sea un procesoo estacionario es que el valor esperado del tiempo del ciclo regenerativo sea finito, es decir: $\esp\left[\sum_{m=1}^{M_{i}}C_{i}^{(m)}\right]<\infty$, como cada $C_{i}^{(m)}$ contiene intervalos de r\'eplica positivos, se tiene que $\esp\left[M_{i}\right]<\infty$, adem\'as, como $M_{i}>0$, se tiene que la condici\'on anterior es equivalente a tener que $\esp\left[C_{i}\right]<\infty$,
por lo tanto una condici\'on suficiente para la existencia del proceso regenerativo est\'a dada por $\sum_{k=1}^{N}\mu_{k}<1.$}, se sigue del lema de Wald que:


\begin{eqnarray*}
\esp\left[\sum_{m=1}^{M_{i}}\sum_{t=\tau_{i}\left(m\right)}^{\tau_{i}\left(m+1\right)-1}z^{L_{i}\left(t\right)}\right]&=&\esp\left[M_{i}\right]\esp\left[\sum_{t=\tau_{i}\left(m\right)}^{\tau_{i}\left(m+1\right)-1}z^{L_{i}\left(t\right)}\right]\\
\esp\left[\sum_{m=1}^{M_{i}}\tau_{i}\left(m+1\right)-\tau_{i}\left(m\right)\right]&=&\esp\left[M_{i}\right]\esp\left[\tau_{i}\left(m+1\right)-\tau_{i}\left(m\right)\right]
\end{eqnarray*}

por tanto se tiene que


\begin{eqnarray*}
Q_{i}\left(z\right)&=&\frac{\esp\left[\sum_{t=\tau_{i}\left(m\right)}^{\tau_{i}\left(m+1\right)-1}z^{L_{i}\left(t\right)}\right]}{\esp\left[\tau_{i}\left(m+1\right)-\tau_{i}\left(m\right)\right]}
\end{eqnarray*}

observar que el denominador es simplemente la duraci\'on promedio del tiempo del ciclo.


Haciendo las siguientes sustituciones en la ecuaci\'on (\ref{Corolario2}): $n\rightarrow t-\tau_{i}\left(m\right)$, $T \rightarrow \overline{\tau}_{i}\left(m\right)-\tau_{i}\left(m\right)$, $L_{n}\rightarrow L_{i}\left(t\right)$ y $F\left(z\right)=\esp\left[z^{L_{0}}\right]\rightarrow F_{i}\left(z\right)=\esp\left[z^{L_{i}\tau_{i}\left(m\right)}\right]$, se puede ver que

\begin{eqnarray}\label{Eq.Arribos.Primera}
\esp\left[\sum_{n=0}^{T-1}z^{L_{n}}\right]=
\esp\left[\sum_{t=\tau_{i}\left(m\right)}^{\overline{\tau}_{i}\left(m\right)-1}z^{L_{i}\left(t\right)}\right]
=z\frac{F_{i}\left(z\right)-1}{z-P_{i}\left(z\right)}
\end{eqnarray}

Por otra parte durante el tiempo de intervisita para la cola $i$, $L_{i}\left(t\right)$ solamente se incrementa de manera que el incremento por intervalo de tiempo est\'a dado por la funci\'on generadora de probabilidades de $P_{i}\left(z\right)$, por tanto la suma sobre el tiempo de intervisita puede evaluarse como:

\begin{eqnarray*}
\esp\left[\sum_{t=\tau_{i}\left(m\right)}^{\tau_{i}\left(m+1\right)-1}z^{L_{i}\left(t\right)}\right]&=&\esp\left[\sum_{t=\tau_{i}\left(m\right)}^{\tau_{i}\left(m+1\right)-1}\left\{P_{i}\left(z\right)\right\}^{t-\overline{\tau}_{i}\left(m\right)}\right]=\frac{1-\esp\left[\left\{P_{i}\left(z\right)\right\}^{\tau_{i}\left(m+1\right)-\overline{\tau}_{i}\left(m\right)}\right]}{1-P_{i}\left(z\right)}\\
&=&\frac{1-I_{i}\left[P_{i}\left(z\right)\right]}{1-P_{i}\left(z\right)}
\end{eqnarray*}
por tanto

\begin{eqnarray*}
\esp\left[\sum_{t=\tau_{i}\left(m\right)}^{\tau_{i}\left(m+1\right)-1}z^{L_{i}\left(t\right)}\right]&=&
\frac{1-F_{i}\left(z\right)}{1-P_{i}\left(z\right)}
\end{eqnarray*}

Por lo tanto

\begin{eqnarray*}
Q_{i}\left(z\right)&=&\frac{\esp\left[\sum_{t=\tau_{i}\left(m\right)}^{\tau_{i}
\left(m+1\right)-1}z^{L_{i}\left(t\right)}\right]}{\esp\left[\tau_{i}\left(m+1\right)-\tau_{i}\left(m\right)\right]}\\
&=&\frac{1}{\esp\left[\tau_{i}\left(m+1\right)-\tau_{i}\left(m\right)\right]}
\left\{
\esp\left[\sum_{t=\tau_{i}\left(m\right)}^{\overline{\tau}_{i}\left(m\right)-1}
z^{L_{i}\left(t\right)}\right]
+\esp\left[\sum_{t=\overline{\tau}_{i}\left(m\right)}^{\tau_{i}\left(m+1\right)-1}
z^{L_{i}\left(t\right)}\right]\right\}\\
&=&\frac{1}{\esp\left[\tau_{i}\left(m+1\right)-\tau_{i}\left(m\right)\right]}
\left\{
z\frac{F_{i}\left(z\right)-1}{z-P_{i}\left(z\right)}+\frac{1-F_{i}\left(z\right)}
{1-P_{i}\left(z\right)}
\right\}
\end{eqnarray*}

es decir

\begin{equation}
Q_{i}\left(z\right)=\frac{1}{\esp\left[C_{i}\right]}\cdot\frac{1-F_{i}\left(z\right)}{P_{i}\left(z\right)-z}\cdot\frac{\left(1-z\right)P_{i}\left(z\right)}{1-P_{i}\left(z\right)}
\end{equation}

\begin{Teo}
Dada una Red de Sistemas de Visitas C\'iclicas (RSVC), conformada por dos Sistemas de Visitas C\'iclicas (SVC), donde cada uno de ellos consta de dos colas tipo $M/M/1$. Los dos sistemas est\'an comunicados entre s\'i por medio de la transferencia de usuarios entre las colas $Q_{1}\leftrightarrow Q_{3}$ y $Q_{2}\leftrightarrow Q_{4}$. Se definen los eventos para los procesos de arribos al tiempo $t$, $A_{j}\left(t\right)=\left\{0 \textrm{ arribos en }Q_{j}\textrm{ al tiempo }t\right\}$ para alg\'un tiempo $t\geq0$ y $Q_{j}$ la cola $j$-\'esima en la RSVC, para $j=1,2,3,4$.  Existe un intervalo $I\neq\emptyset$ tal que para $T^{*}\in I$, tal que $\prob\left\{A_{1}\left(T^{*}\right),A_{2}\left(Tt^{*}\right),
A_{3}\left(T^{*}\right),A_{4}\left(T^{*}\right)|T^{*}\in I\right\}>0$.
\end{Teo}

\begin{proof}
Sin p\'erdida de generalidad podemos considerar como base del an\'alisis a la cola $Q_{1}$ del primer sistema que conforma la RSVC.

Sea $n>0$, ciclo en el primer sistema en el que se sabe que $L_{j}\left(\overline{\tau}_{1}\left(n\right)\right)=0$, pues la pol\'itica de servicio con que atienden los servidores es la exhaustiva. Como es sabido, para trasladarse a la siguiente cola, el servidor incurre en un tiempo de traslado $r_{1}\left(n\right)>0$, entonces tenemos que $\tau_{2}\left(n\right)=\overline{\tau}_{1}\left(n\right)+r_{1}\left(n\right)$.


Definamos el intervalo $I_{1}\equiv\left[\overline{\tau}_{1}\left(n\right),\tau_{2}\left(n\right)\right]$ de longitud $\xi_{1}=r_{1}\left(n\right)$. Dado que los tiempos entre arribo son exponenciales con tasa $\tilde{\mu}_{1}=\mu_{1}+\hat{\mu}_{1}$ ($\mu_{1}$ son los arribos a $Q_{1}$ por primera vez al sistema, mientras que $\hat{\mu}_{1}$ son los arribos de traslado procedentes de $Q_{3}$) se tiene que la probabilidad del evento $A_{1}\left(t\right)$ est\'a dada por 

\begin{equation}
\prob\left\{A_{1}\left(t\right)|t\in I_{1}\left(n\right)\right\}=e^{-\tilde{\mu}_{1}\xi_{1}\left(n\right)}.
\end{equation} 

Por otra parte, para la cola $Q_{2}$, el tiempo $\overline{\tau}_{2}\left(n-1\right)$ es tal que $L_{2}\left(\overline{\tau}_{2}\left(n-1\right)\right)=0$, es decir, es el tiempo en que la cola queda totalmente vac\'ia en el ciclo anterior a $n$. Entonces tenemos un sgundo intervalo $I_{2}\equiv\left[\overline{\tau}_{2}\left(n-1\right),\tau_{2}\left(n\right)\right]$. Por lo tanto la probabilidad del evento $A_{2}\left(t\right)$ tiene probabilidad dada por

\begin{equation}
\prob\left\{A_{2}\left(t\right)|t\in I_{2}\left(n\right)\right\}=e^{-\tilde{\mu}_{2}\xi_{2}\left(n\right)},
\end{equation} 

donde $\xi_{2}\left(n\right)=\tau_{2}\left(n\right)-\overline{\tau}_{2}\left(n-1\right)$.



Entonces, se tiene que

\begin{eqnarray*}
\prob\left\{A_{1}\left(t\right),A_{2}\left(t\right)|t\in I_{1}\left(n\right)\right\}&=&
\prob\left\{A_{1}\left(t\right)|t\in I_{1}\left(n\right)\right\}
\prob\left\{A_{2}\left(t\right)|t\in I_{1}\left(n\right)\right\}\\
&\geq&
\prob\left\{A_{1}\left(t\right)|t\in I_{1}\left(n\right)\right\}
\prob\left\{A_{2}\left(t\right)|t\in I_{2}\left(n\right)\right\}\\
&=&e^{-\tilde{\mu}_{1}\xi_{1}\left(n\right)}e^{-\tilde{\mu}_{2}\xi_{2}\left(n\right)}
=e^{-\left[\tilde{\mu}_{1}\xi_{1}\left(n\right)+\tilde{\mu}_{2}\xi_{2}\left(n\right)\right]}.
\end{eqnarray*}


es decir, 

\begin{equation}
\prob\left\{A_{1}\left(t\right),A_{2}\left(t\right)|t\in I_{1}\left(n\right)\right\}
=e^{-\left[\tilde{\mu}_{1}\xi_{1}\left(n\right)+\tilde{\mu}_{2}\xi_{2}
\left(n\right)\right]}>0.
\end{equation}

En lo que respecta a la relaci\'on entre los dos SVC que conforman la RSVC, se afirma que existe $m>0$ tal que $\overline{\tau}_{3}\left(m\right)<\tau_{2}\left(n\right)<\tau_{4}\left(m\right)$.

Para $Q_{3}$ sea $I_{3}=\left[\overline{\tau}_{3}\left(m\right),\tau_{4}\left(m\right)\right]$ con longitud  $\xi_{3}\left(m\right)=r_{3}\left(m\right)$, entonces 

\begin{equation}
\prob\left\{A_{3}\left(t\right)|t\in I_{3}\left(n\right)\right\}=e^{-\tilde{\mu}_{3}\xi_{3}\left(n\right)}.
\end{equation} 

An\'alogamente que como se hizo para $Q_{2}$, tenemos que para $Q_{4}$ se tiene el intervalo $I_{4}=\left[\overline{\tau}_{4}\left(m-1\right),\tau_{4}\left(m\right)\right]$ con longitud $\xi_{4}\left(m\right)=\tau_{4}\left(m\right)-\overline{\tau}_{4}\left(m-1\right)$, entonces


\begin{equation}
\prob\left\{A_{4}\left(t\right)|t\in I_{4}\left(m\right)\right\}=e^{-\tilde{\mu}_{4}\xi_{4}\left(n\right)}.
\end{equation} 

Al igual que para el primer sistema, dado que $I_{3}\left(m\right)\subset I_{4}\left(m\right)$, se tiene que

\begin{eqnarray*}
\xi_{3}\left(m\right)\leq\xi_{4}\left(m\right)&\Leftrightarrow& -\xi_{3}\left(m\right)\geq-\xi_{4}\left(m\right)
\\
-\tilde{\mu}_{4}\xi_{3}\left(m\right)\geq-\tilde{\mu}_{4}\xi_{4}\left(m\right)&\Leftrightarrow&
e^{-\tilde{\mu}_{4}\xi_{3}\left(m\right)}\geq e^{-\tilde{\mu}_{4}\xi_{4}\left(m\right)}\\
\prob\left\{A_{4}\left(t\right)|t\in I_{3}\left(m\right)\right\}&\geq&
\prob\left\{A_{4}\left(t\right)|t\in I_{4}\left(m\right)\right\}
\end{eqnarray*}

Entonces, dado que los eventos $A_{3}$ y $A_{4}$ son independientes, se tiene que

\begin{eqnarray*}
\prob\left\{A_{3}\left(t\right),A_{4}\left(t\right)|t\in I_{3}\left(m\right)\right\}&=&
\prob\left\{A_{3}\left(t\right)|t\in I_{3}\left(m\right)\right\}
\prob\left\{A_{4}\left(t\right)|t\in I_{3}\left(m\right)\right\}\\
&\geq&
\prob\left\{A_{3}\left(t\right)|t\in I_{3}\left(n\right)\right\}
\prob\left\{A_{4}\left(t\right)|t\in I_{4}\left(n\right)\right\}\\
&=&e^{-\tilde{\mu}_{3}\xi_{3}\left(m\right)}e^{-\tilde{\mu}_{4}\xi_{4}
\left(m\right)}
=e^{-\left[\tilde{\mu}_{3}\xi_{3}\left(m\right)+\tilde{\mu}_{4}\xi_{4}
\left(m\right)\right]}.
\end{eqnarray*}


es decir, 

\begin{equation}
\prob\left\{A_{3}\left(t\right),A_{4}\left(t\right)|t\in I_{3}\left(m\right)\right\}
=e^{-\left[\tilde{\mu}_{3}\xi_{3}\left(m\right)+\tilde{\mu}_{4}\xi_{4}
\left(m\right)\right]}>0.
\end{equation}

Por construcci\'on se tiene que $I\left(n,m\right)\equiv I_{1}\left(n\right)\cap I_{3}\left(m\right)\neq\emptyset$,entonces en particular se tienen las contenciones $I\left(n,m\right)\subseteq I_{1}\left(n\right)$ y $I\left(n,m\right)\subseteq I_{3}\left(m\right)$, por lo tanto si definimos $\xi_{n,m}\equiv\ell\left(I\left(n,m\right)\right)$ tenemos que

\begin{eqnarray*}
\xi_{n,m}\leq\xi_{1}\left(n\right)\textrm{ y }\xi_{n,m}\leq\xi_{3}\left(m\right)\textrm{ entonces }
-\xi_{n,m}\geq-\xi_{1}\left(n\right)\textrm{ y }-\xi_{n,m}\leq-\xi_{3}\left(m\right)\\
\end{eqnarray*}
por lo tanto tenemos las desigualdades 



\begin{eqnarray*}
\begin{array}{ll}
-\tilde{\mu}_{1}\xi_{n,m}\geq-\tilde{\mu}_{1}\xi_{1}\left(n\right),&
-\tilde{\mu}_{2}\xi_{n,m}\geq-\tilde{\mu}_{2}\xi_{1}\left(n\right)
\geq-\tilde{\mu}_{2}\xi_{2}\left(n\right),\\
-\tilde{\mu}_{3}\xi_{n,m}\geq-\tilde{\mu}_{3}\xi_{3}\left(m\right),&
-\tilde{\mu}_{4}\xi_{n,m}\geq-\tilde{\mu}_{4}\xi_{3}\left(m\right)
\geq-\tilde{\mu}_{4}\xi_{4}\left(m\right).
\end{array}
\end{eqnarray*}

Sea $T^{*}\in I_{n,m}$, entonces dado que en particular $T^{*}\in I_{1}\left(n\right)$ se cumple con probabilidad positiva que no hay arribos a las colas $Q_{1}$ y $Q_{2}$, en consecuencia, tampoco hay usuarios de transferencia para $Q_{3}$ y $Q_{4}$, es decir, $\tilde{\mu}_{1}=\mu_{1}$, $\tilde{\mu}_{2}=\mu_{2}$, $\tilde{\mu}_{3}=\mu_{3}$, $\tilde{\mu}_{4}=\mu_{4}$, es decir, los eventos $Q_{1}$ y $Q_{3}$ son condicionalmente independientes en el intervalo $I_{n,m}$; lo mismo ocurre para las colas $Q_{2}$ y $Q_{4}$, por lo tanto tenemos que


\begin{eqnarray}
\begin{array}{l}
\prob\left\{A_{1}\left(T^{*}\right),A_{2}\left(T^{*}\right),
A_{3}\left(T^{*}\right),A_{4}\left(T^{*}\right)|T^{*}\in I_{n,m}\right\}
=\prod_{j=1}^{4}\prob\left\{A_{j}\left(T^{*}\right)|T^{*}\in I_{n,m}\right\}\\
\geq\prob\left\{A_{1}\left(T^{*}\right)|T^{*}\in I_{1}\left(n\right)\right\}
\prob\left\{A_{2}\left(T^{*}\right)|T^{*}\in I_{2}\left(n\right)\right\}
\prob\left\{A_{3}\left(T^{*}\right)|T^{*}\in I_{3}\left(m\right)\right\}
\prob\left\{A_{4}\left(T^{*}\right)|T^{*}\in I_{4}\left(m\right)\right\}\\
=e^{-\mu_{1}\xi_{1}\left(n\right)}
e^{-\mu_{2}\xi_{2}\left(n\right)}
e^{-\mu_{3}\xi_{3}\left(m\right)}
e^{-\mu_{4}\xi_{4}\left(m\right)}
=e^{-\left[\tilde{\mu}_{1}\xi_{1}\left(n\right)
+\tilde{\mu}_{2}\xi_{2}\left(n\right)
+\tilde{\mu}_{3}\xi_{3}\left(m\right)
+\tilde{\mu}_{4}\xi_{4}
\left(m\right)\right]}>0.
\end{array}
\end{eqnarray}
\end{proof}


Estos resultados aparecen en Daley (1968) \cite{Daley68} para $\left\{T_{n}\right\}$ intervalos de inter-arribo, $\left\{D_{n}\right\}$ intervalos de inter-salida y $\left\{S_{n}\right\}$ tiempos de servicio.

\begin{itemize}
\item Si el proceso $\left\{T_{n}\right\}$ es Poisson, el proceso $\left\{D_{n}\right\}$ es no correlacionado si y s\'olo si es un proceso Poisso, lo cual ocurre si y s\'olo si $\left\{S_{n}\right\}$ son exponenciales negativas.

\item Si $\left\{S_{n}\right\}$ son exponenciales negativas, $\left\{D_{n}\right\}$ es un proceso de renovaci\'on  si y s\'olo si es un proceso Poisson, lo cual ocurre si y s\'olo si $\left\{T_{n}\right\}$ es un proceso Poisson.

\item $\esp\left(D_{n}\right)=\esp\left(T_{n}\right)$.

\item Para un sistema de visitas $GI/M/1$ se tiene el siguiente teorema:

\begin{Teo}
En un sistema estacionario $GI/M/1$ los intervalos de interpartida tienen
\begin{eqnarray*}
\esp\left(e^{-\theta D_{n}}\right)&=&\mu\left(\mu+\theta\right)^{-1}\left[\delta\theta
-\mu\left(1-\delta\right)\alpha\left(\theta\right)\right]
\left[\theta-\mu\left(1-\delta\right)^{-1}\right]\\
\alpha\left(\theta\right)&=&\esp\left[e^{-\theta T_{0}}\right]\\
var\left(D_{n}\right)&=&var\left(T_{0}\right)-\left(\tau^{-1}-\delta^{-1}\right)
2\delta\left(\esp\left(S_{0}\right)\right)^{2}\left(1-\delta\right)^{-1}.
\end{eqnarray*}
\end{Teo}



\begin{Teo}
El proceso de salida de un sistema de colas estacionario $GI/M/1$ es un proceso de renovaci\'on si y s\'olo si el proceso de entrada es un proceso Poisson, en cuyo caso el proceso de salida es un proceso Poisson.
\end{Teo}


\begin{Teo}
Los intervalos de interpartida $\left\{D_{n}\right\}$ de un sistema $M/G/1$ estacionario son no correlacionados si y s\'olo si la distribuci\'on de los tiempos de servicio es exponencial negativa, es decir, el sistema es de tipo  $M/M/1$.

\end{Teo}



\end{itemize}


%\section{Resultados para Procesos de Salida}

En Sigman, Thorison y Wolff \cite{Sigman2} prueban que para la existencia de un una sucesi\'on infinita no decreciente de tiempos de regeneraci\'on $\tau_{1}\leq\tau_{2}\leq\cdots$ en los cuales el proceso se regenera, basta un tiempo de regeneraci\'on $R_{1}$, donde $R_{j}=\tau_{j}-\tau_{j-1}$. Para tal efecto se requiere la existencia de un espacio de probabilidad $\left(\Omega,\mathcal{F},\prob\right)$, y proceso estoc\'astico $\textit{X}=\left\{X\left(t\right):t\geq0\right\}$ con espacio de estados $\left(S,\mathcal{R}\right)$, con $\mathcal{R}$ $\sigma$-\'algebra.

\begin{Prop}
Si existe una variable aleatoria no negativa $R_{1}$ tal que $\theta_{R\footnotesize{1}}X=_{D}X$, entonces $\left(\Omega,\mathcal{F},\prob\right)$ puede extenderse para soportar una sucesi\'on estacionaria de variables aleatorias $R=\left\{R_{k}:k\geq1\right\}$, tal que para $k\geq1$,
\begin{eqnarray*}
\theta_{k}\left(X,R\right)=_{D}\left(X,R\right).
\end{eqnarray*}

Adem\'as, para $k\geq1$, $\theta_{k}R$ es condicionalmente independiente de $\left(X,R_{1},\ldots,R_{k}\right)$, dado $\theta_{\tau k}X$.

\end{Prop}


\begin{itemize}
\item Doob en 1953 demostr\'o que el estado estacionario de un proceso de partida en un sistema de espera $M/G/\infty$, es Poisson con la misma tasa que el proceso de arribos.

\item Burke en 1968, fue el primero en demostrar que el estado estacionario de un proceso de salida de una cola $M/M/s$ es un proceso Poisson.

\item Disney en 1973 obtuvo el siguiente resultado:

\begin{Teo}
Para el sistema de espera $M/G/1/L$ con disciplina FIFO, el proceso $\textbf{I}$ es un proceso de renovaci\'on si y s\'olo si el proceso denominado longitud de la cola es estacionario y se cumple cualquiera de los siguientes casos:

\begin{itemize}
\item[a)] Los tiempos de servicio son identicamente cero;
\item[b)] $L=0$, para cualquier proceso de servicio $S$;
\item[c)] $L=1$ y $G=D$;
\item[d)] $L=\infty$ y $G=M$.
\end{itemize}
En estos casos, respectivamente, las distribuciones de interpartida $P\left\{T_{n+1}-T_{n}\leq t\right\}$ son


\begin{itemize}
\item[a)] $1-e^{-\lambda t}$, $t\geq0$;
\item[b)] $1-e^{-\lambda t}*F\left(t\right)$, $t\geq0$;
\item[c)] $1-e^{-\lambda t}*\indora_{d}\left(t\right)$, $t\geq0$;
\item[d)] $1-e^{-\lambda t}*F\left(t\right)$, $t\geq0$.
\end{itemize}
\end{Teo}


\item Finch (1959) mostr\'o que para los sistemas $M/G/1/L$, con $1\leq L\leq \infty$ con distribuciones de servicio dos veces diferenciable, solamente el sistema $M/M/1/\infty$ tiene proceso de salida de renovaci\'on estacionario.

\item King (1971) demostro que un sistema de colas estacionario $M/G/1/1$ tiene sus tiempos de interpartida sucesivas $D_{n}$ y $D_{n+1}$ son independientes, si y s\'olo si, $G=D$, en cuyo caso le proceso de salida es de renovaci\'on.

\item Disney (1973) demostr\'o que el \'unico sistema estacionario $M/G/1/L$, que tiene proceso de salida de renovaci\'on  son los sistemas $M/M/1$ y $M/D/1/1$.



\item El siguiente resultado es de Disney y Koning (1985)
\begin{Teo}
En un sistema de espera $M/G/s$, el estado estacionario del proceso de salida es un proceso Poisson para cualquier distribuci\'on de los tiempos de servicio si el sistema tiene cualquiera de las siguientes cuatro propiedades.

\begin{itemize}
\item[a)] $s=\infty$
\item[b)] La disciplina de servicio es de procesador compartido.
\item[c)] La disciplina de servicio es LCFS y preemptive resume, esto se cumple para $L<\infty$
\item[d)] $G=M$.
\end{itemize}

\end{Teo}

\item El siguiente resultado es de Alamatsaz (1983)

\begin{Teo}
En cualquier sistema de colas $GI/G/1/L$ con $1\leq L<\infty$ y distribuci\'on de interarribos $A$ y distribuci\'on de los tiempos de servicio $B$, tal que $A\left(0\right)=0$, $A\left(t\right)\left(1-B\left(t\right)\right)>0$ para alguna $t>0$ y $B\left(t\right)$ para toda $t>0$, es imposible que el proceso de salida estacionario sea de renovaci\'on.
\end{Teo}

\end{itemize}

Estos resultados aparecen en Daley (1968) \cite{Daley68} para $\left\{T_{n}\right\}$ intervalos de inter-arribo, $\left\{D_{n}\right\}$ intervalos de inter-salida y $\left\{S_{n}\right\}$ tiempos de servicio.

\begin{itemize}
\item Si el proceso $\left\{T_{n}\right\}$ es Poisson, el proceso $\left\{D_{n}\right\}$ es no correlacionado si y s\'olo si es un proceso Poisso, lo cual ocurre si y s\'olo si $\left\{S_{n}\right\}$ son exponenciales negativas.

\item Si $\left\{S_{n}\right\}$ son exponenciales negativas, $\left\{D_{n}\right\}$ es un proceso de renovaci\'on  si y s\'olo si es un proceso Poisson, lo cual ocurre si y s\'olo si $\left\{T_{n}\right\}$ es un proceso Poisson.

\item $\esp\left(D_{n}\right)=\esp\left(T_{n}\right)$.

\item Para un sistema de visitas $GI/M/1$ se tiene el siguiente teorema:

\begin{Teo}
En un sistema estacionario $GI/M/1$ los intervalos de interpartida tienen
\begin{eqnarray*}
\esp\left(e^{-\theta D_{n}}\right)&=&\mu\left(\mu+\theta\right)^{-1}\left[\delta\theta
-\mu\left(1-\delta\right)\alpha\left(\theta\right)\right]
\left[\theta-\mu\left(1-\delta\right)^{-1}\right]\\
\alpha\left(\theta\right)&=&\esp\left[e^{-\theta T_{0}}\right]\\
var\left(D_{n}\right)&=&var\left(T_{0}\right)-\left(\tau^{-1}-\delta^{-1}\right)
2\delta\left(\esp\left(S_{0}\right)\right)^{2}\left(1-\delta\right)^{-1}.
\end{eqnarray*}
\end{Teo}



\begin{Teo}
El proceso de salida de un sistema de colas estacionario $GI/M/1$ es un proceso de renovaci\'on si y s\'olo si el proceso de entrada es un proceso Poisson, en cuyo caso el proceso de salida es un proceso Poisson.
\end{Teo}


\begin{Teo}
Los intervalos de interpartida $\left\{D_{n}\right\}$ de un sistema $M/G/1$ estacionario son no correlacionados si y s\'olo si la distribuci\'on de los tiempos de servicio es exponencial negativa, es decir, el sistema es de tipo  $M/M/1$.

\end{Teo}



\end{itemize}
%\newpage
%________________________________________________________________________
%\section{Redes de Sistemas de Visitas C\'iclicas}
%________________________________________________________________________

Sean $Q_{1},Q_{2},Q_{3}$ y $Q_{4}$ en una Red de Sistemas de Visitas C\'iclicas (RSVC). Supongamos que cada una de las colas es del tipo $M/M/1$ con tasa de arribo $\mu_{i}$ y que la transferencia de usuarios entre los dos sistemas ocurre entre $Q_{1}\leftrightarrow Q_{3}$ y $Q_{2}\leftrightarrow Q_{4}$ con respectiva tasa de arribo igual a la tasa de salida $\hat{\mu}_{i}=\mu_{i}$, esto se sabe por lo desarrollado en la secci\'on anterior.  

Consideremos, sin p\'erdida de generalidad como base del an\'alisis, la cola $Q_{1}$ adem\'as supongamos al servidor lo comenzamos a observar una vez que termina de atender a la misma para desplazarse y llegar a $Q_{2}$, es decir al tiempo $\tau_{2}$.

Sea $n\in\nat$, $n>0$, ciclo del servidor en que regresa a $Q_{1}$ para dar servicio y atender conforme a la pol\'itica exhaustiva, entonces se tiene que $\overline{\tau}_{1}\left(n\right)$ es el tiempo del servidor en el sistema 1 en que termina de dar servicio a todos los usuarios presentes en la cola, por lo tanto se cumple que $L_{1}\left(\overline{\tau}_{1}\left(n\right)\right)=0$, entonces el servidor para llegar a $Q_{2}$ incurre en un tiempo de traslado $r_{1}$ y por tanto se cumple que $\tau_{2}\left(n\right)=\overline{\tau}_{1}\left(n\right)+r_{1}$. Dado que los tiempos entre arribos son exponenciales se cumple que 

\begin{eqnarray*}
\prob\left\{0 \textrm{ arribos en }Q_{1}\textrm{ en el intervalo }\left[\overline{\tau}_{1}\left(n\right),\overline{\tau}_{1}\left(n\right)+r_{1}\right]\right\}=e^{-\tilde{\mu}_{1}r_{1}},\\
\prob\left\{0 \textrm{ arribos en }Q_{2}\textrm{ en el intervalo }\left[\overline{\tau}_{1}\left(n\right),\overline{\tau}_{1}\left(n\right)+r_{1}\right]\right\}=e^{-\tilde{\mu}_{2}r_{1}}.
\end{eqnarray*}

El evento que nos interesa consiste en que no haya arribos desde que el servidor abandon\'o $Q_{2}$ y regresa nuevamente para dar servicio, es decir en el intervalo de tiempo $\left[\overline{\tau}_{2}\left(n-1\right),\tau_{2}\left(n\right)\right]$. Entonces, si hacemos


\begin{eqnarray*}
\varphi_{1}\left(n\right)&\equiv&\overline{\tau}_{1}\left(n\right)+r_{1}=\overline{\tau}_{2}\left(n-1\right)+r_{1}+r_{2}+\overline{\tau}_{1}\left(n\right)-\tau_{1}\left(n\right)\\
&=&\overline{\tau}_{2}\left(n-1\right)+\overline{\tau}_{1}\left(n\right)-\tau_{1}\left(n\right)+r,,
\end{eqnarray*}

y longitud del intervalo

\begin{eqnarray*}
\xi&\equiv&\overline{\tau}_{1}\left(n\right)+r_{1}-\overline{\tau}_{2}\left(n-1\right)
=\overline{\tau}_{2}\left(n-1\right)+\overline{\tau}_{1}\left(n\right)-\tau_{1}\left(n\right)+r-\overline{\tau}_{2}\left(n-1\right)\\
&=&\overline{\tau}_{1}\left(n\right)-\tau_{1}\left(n\right)+r.
\end{eqnarray*}


Entonces, determinemos la probabilidad del evento no arribos a $Q_{2}$ en $\left[\overline{\tau}_{2}\left(n-1\right),\varphi_{1}\left(n\right)\right]$:

\begin{eqnarray}
\prob\left\{0 \textrm{ arribos en }Q_{2}\textrm{ en el intervalo }\left[\overline{\tau}_{2}\left(n-1\right),\varphi_{1}\left(n\right)\right]\right\}
=e^{-\tilde{\mu}_{2}\xi}.
\end{eqnarray}

De manera an\'aloga, tenemos que la probabilidad de no arribos a $Q_{1}$ en $\left[\overline{\tau}_{2}\left(n-1\right),\varphi_{1}\left(n\right)\right]$ esta dada por

\begin{eqnarray}
\prob\left\{0 \textrm{ arribos en }Q_{1}\textrm{ en el intervalo }\left[\overline{\tau}_{2}\left(n-1\right),\varphi_{1}\left(n\right)\right]\right\}
=e^{-\tilde{\mu}_{1}\xi},
\end{eqnarray}

\begin{eqnarray}
\prob\left\{0 \textrm{ arribos en }Q_{2}\textrm{ en el intervalo }\left[\overline{\tau}_{2}\left(n-1\right),\varphi_{1}\left(n\right)\right]\right\}
=e^{-\tilde{\mu}_{2}\xi}.
\end{eqnarray}

Por tanto 

\begin{eqnarray}
\begin{array}{l}
\prob\left\{0 \textrm{ arribos en }Q_{1}\textrm{ y }Q_{2}\textrm{ en el intervalo }\left[\overline{\tau}_{2}\left(n-1\right),\varphi_{1}\left(n\right)\right]\right\}\\
=\prob\left\{0 \textrm{ arribos en }Q_{1}\textrm{ en el intervalo }\left[\overline{\tau}_{2}\left(n-1\right),\varphi_{1}\left(n\right)\right]\right\}\\
\times
\prob\left\{0 \textrm{ arribos en }Q_{2}\textrm{ en el intervalo }\left[\overline{\tau}_{2}\left(n-1\right),\varphi_{1}\left(n\right)\right]\right\}=e^{-\tilde{\mu}_{1}\xi}e^{-\tilde{\mu}_{2}\xi}
=e^{-\tilde{\mu}\xi}.
\end{array}
\end{eqnarray}

Para el segundo sistema, consideremos nuevamente $\overline{\tau}_{1}\left(n\right)+r_{1}$, sin p\'erdida de generalidad podemos suponer que existe $m>0$ tal que $\overline{\tau}_{3}\left(m\right)<\overline{\tau}_{1}+r_{1}<\tau_{4}\left(m\right)$, entonces

\begin{eqnarray}
\prob\left\{0 \textrm{ arribos en }Q_{3}\textrm{ en el intervalo }\left[\overline{\tau}_{3}\left(m\right),\overline{\tau}_{1}\left(n\right)+r_{1}\right]\right\}
=e^{-\tilde{\mu}_{3}\xi_{3}},
\end{eqnarray}
donde 
\begin{eqnarray}
\xi_{3}=\overline{\tau}_{1}\left(n\right)+r_{1}-\overline{\tau}_{3}\left(m\right)=
\overline{\tau}_{1}\left(n\right)-\overline{\tau}_{3}\left(m\right)+r_{1},
\end{eqnarray}

mientras que para $Q_{4}$ al igual que con $Q_{2}$ escribiremos $\tau_{4}\left(m\right)$ en t\'erminos de $\overline{\tau}_{4}\left(m-1\right)$:

$\varphi_{2}\equiv\tau_{4}\left(m\right)=\overline{\tau}_{4}\left(m-1\right)+r_{4}+\overline{\tau}_{3}\left(m\right)
-\tau_{3}\left(m\right)+r_{3}=\overline{\tau}_{4}\left(m-1\right)+\overline{\tau}_{3}\left(m\right)
-\tau_{3}\left(m\right)+\hat{r}$, adem\'as,

$\xi_{2}\equiv\varphi_{2}\left(m\right)-\overline{\tau}_{1}-r_{1}=\overline{\tau}_{4}\left(m-1\right)+\overline{\tau}_{3}\left(m\right)
-\tau_{3}\left(m\right)-\overline{\tau}_{1}\left(n\right)+\hat{r}-r_{1}$. 

Entonces


\begin{eqnarray}
\prob\left\{0 \textrm{ arribos en }Q_{4}\textrm{ en el intervalo }\left[\overline{\tau}_{1}\left(n\right)+r_{1},\varphi_{2}\left(m\right)\right]\right\}
=e^{-\tilde{\mu}_{4}\xi_{2}},
\end{eqnarray}

mientras que para $Q_{3}$ se tiene que 

\begin{eqnarray}
\prob\left\{0 \textrm{ arribos en }Q_{3}\textrm{ en el intervalo }\left[\overline{\tau}_{1}\left(n\right)+r_{1},\varphi_{2}\left(m\right)\right]\right\}
=e^{-\tilde{\mu}_{3}\xi_{2}}
\end{eqnarray}

Por tanto

\begin{eqnarray}
\prob\left\{0 \textrm{ arribos en }Q_{3}\wedge Q_{4}\textrm{ en el intervalo }\left[\overline{\tau}_{1}\left(n\right)+r_{1},\varphi_{2}\left(m\right)\right]\right\}
=e^{-\hat{\mu}\xi_{2}}
\end{eqnarray}
donde $\hat{\mu}=\tilde{\mu}_{3}+\tilde{\mu}_{4}$.

Ahora, definamos los intervalos $\mathcal{I}_{1}=\left[\overline{\tau}_{1}\left(n\right)+r_{1},\varphi_{1}\left(n\right)\right]$  y $\mathcal{I}_{2}=\left[\overline{\tau}_{1}\left(n\right)+r_{1},\varphi_{2}\left(m\right)\right]$, entonces, sea $\mathcal{I}=\mathcal{I}_{1}\cap\mathcal{I}_{2}$ el intervalo donde cada una de las colas se encuentran vac\'ias, es decir, si tomamos $T^{*}\in\mathcal{I}$, entonces  $L_{1}\left(T^{*}\right)=L_{2}\left(T^{*}\right)=L_{3}\left(T^{*}\right)=L_{4}\left(T^{*}\right)=0$.

Ahora, dado que por construcci\'on $\mathcal{I}\neq\emptyset$ y que para $T^{*}\in\mathcal{I}$ en ninguna de las colas han llegado usuarios, se tiene que no hay transferencia entre las colas, por lo tanto, el sistema 1 y el sistema 2 son condicionalmente independientes en $\mathcal{I}$, es decir

\begin{eqnarray}
\prob\left\{L_{1}\left(T^{*}\right),L_{2}\left(T^{*}\right),L_{3}\left(T^{*}\right),L_{4}\left(T^{*}\right)|T^{*}\in\mathcal{I}\right\}=\prod_{j=1}^{4}\prob\left\{L_{j}\left(T^{*}\right)\right\},
\end{eqnarray}

para $T^{*}\in\mathcal{I}$. 

%\newpage























%________________________________________________________________________
%\section{Procesos Regenerativos}
%________________________________________________________________________

%________________________________________________________________________
%\subsection*{Procesos Regenerativos Sigman, Thorisson y Wolff \cite{Sigman1}}
%________________________________________________________________________


\begin{Def}[Definici\'on Cl\'asica]
Un proceso estoc\'astico $X=\left\{X\left(t\right):t\geq0\right\}$ es llamado regenerativo is existe una variable aleatoria $R_{1}>0$ tal que
\begin{itemize}
\item[i)] $\left\{X\left(t+R_{1}\right):t\geq0\right\}$ es independiente de $\left\{\left\{X\left(t\right):t<R_{1}\right\},\right\}$
\item[ii)] $\left\{X\left(t+R_{1}\right):t\geq0\right\}$ es estoc\'asticamente equivalente a $\left\{X\left(t\right):t>0\right\}$
\end{itemize}

Llamamos a $R_{1}$ tiempo de regeneraci\'on, y decimos que $X$ se regenera en este punto.
\end{Def}

$\left\{X\left(t+R_{1}\right)\right\}$ es regenerativo con tiempo de regeneraci\'on $R_{2}$, independiente de $R_{1}$ pero con la misma distribuci\'on que $R_{1}$. Procediendo de esta manera se obtiene una secuencia de variables aleatorias independientes e id\'enticamente distribuidas $\left\{R_{n}\right\}$ llamados longitudes de ciclo. Si definimos a $Z_{k}\equiv R_{1}+R_{2}+\cdots+R_{k}$, se tiene un proceso de renovaci\'on llamado proceso de renovaci\'on encajado para $X$.


\begin{Note}
La existencia de un primer tiempo de regeneraci\'on, $R_{1}$, implica la existencia de una sucesi\'on completa de estos tiempos $R_{1},R_{2}\ldots,$ que satisfacen la propiedad deseada \cite{Sigman2}.
\end{Note}


\begin{Note} Para la cola $GI/GI/1$ los usuarios arriban con tiempos $t_{n}$ y son atendidos con tiempos de servicio $S_{n}$, los tiempos de arribo forman un proceso de renovaci\'on  con tiempos entre arribos independientes e identicamente distribuidos (\texttt{i.i.d.})$T_{n}=t_{n}-t_{n-1}$, adem\'as los tiempos de servicio son \texttt{i.i.d.} e independientes de los procesos de arribo. Por \textit{estable} se entiende que $\esp S_{n}<\esp T_{n}<\infty$.
\end{Note}
 


\begin{Def}
Para $x$ fijo y para cada $t\geq0$, sea $I_{x}\left(t\right)=1$ si $X\left(t\right)\leq x$,  $I_{x}\left(t\right)=0$ en caso contrario, y def\'inanse los tiempos promedio
\begin{eqnarray*}
\overline{X}&=&lim_{t\rightarrow\infty}\frac{1}{t}\int_{0}^{\infty}X\left(u\right)du\\
\prob\left(X_{\infty}\leq x\right)&=&lim_{t\rightarrow\infty}\frac{1}{t}\int_{0}^{\infty}I_{x}\left(u\right)du,
\end{eqnarray*}
cuando estos l\'imites existan.
\end{Def}

Como consecuencia del teorema de Renovaci\'on-Recompensa, se tiene que el primer l\'imite  existe y es igual a la constante
\begin{eqnarray*}
\overline{X}&=&\frac{\esp\left[\int_{0}^{R_{1}}X\left(t\right)dt\right]}{\esp\left[R_{1}\right]},
\end{eqnarray*}
suponiendo que ambas esperanzas son finitas.
 
\begin{Note}
Funciones de procesos regenerativos son regenerativas, es decir, si $X\left(t\right)$ es regenerativo y se define el proceso $Y\left(t\right)$ por $Y\left(t\right)=f\left(X\left(t\right)\right)$ para alguna funci\'on Borel medible $f\left(\cdot\right)$. Adem\'as $Y$ es regenerativo con los mismos tiempos de renovaci\'on que $X$. 

En general, los tiempos de renovaci\'on, $Z_{k}$ de un proceso regenerativo no requieren ser tiempos de paro con respecto a la evoluci\'on de $X\left(t\right)$.
\end{Note} 

\begin{Note}
Una funci\'on de un proceso de Markov, usualmente no ser\'a un proceso de Markov, sin embargo ser\'a regenerativo si el proceso de Markov lo es.
\end{Note}

 
\begin{Note}
Un proceso regenerativo con media de la longitud de ciclo finita es llamado positivo recurrente.
\end{Note}


\begin{Note}
\begin{itemize}
\item[a)] Si el proceso regenerativo $X$ es positivo recurrente y tiene trayectorias muestrales no negativas, entonces la ecuaci\'on anterior es v\'alida.
\item[b)] Si $X$ es positivo recurrente regenerativo, podemos construir una \'unica versi\'on estacionaria de este proceso, $X_{e}=\left\{X_{e}\left(t\right)\right\}$, donde $X_{e}$ es un proceso estoc\'astico regenerativo y estrictamente estacionario, con distribuci\'on marginal distribuida como $X_{\infty}$
\end{itemize}
\end{Note}


%__________________________________________________________________________________________
%\subsection*{Procesos Regenerativos Estacionarios - Stidham \cite{Stidham}}
%__________________________________________________________________________________________


Un proceso estoc\'astico a tiempo continuo $\left\{V\left(t\right),t\geq0\right\}$ es un proceso regenerativo si existe una sucesi\'on de variables aleatorias independientes e id\'enticamente distribuidas $\left\{X_{1},X_{2},\ldots\right\}$, sucesi\'on de renovaci\'on, tal que para cualquier conjunto de Borel $A$, 

\begin{eqnarray*}
\prob\left\{V\left(t\right)\in A|X_{1}+X_{2}+\cdots+X_{R\left(t\right)}=s,\left\{V\left(\tau\right),\tau<s\right\}\right\}=\prob\left\{V\left(t-s\right)\in A|X_{1}>t-s\right\},
\end{eqnarray*}
para todo $0\leq s\leq t$, donde $R\left(t\right)=\max\left\{X_{1}+X_{2}+\cdots+X_{j}\leq t\right\}=$n\'umero de renovaciones ({\emph{puntos de regeneraci\'on}}) que ocurren en $\left[0,t\right]$. El intervalo $\left[0,X_{1}\right)$ es llamado {\emph{primer ciclo de regeneraci\'on}} de $\left\{V\left(t \right),t\geq0\right\}$, $\left[X_{1},X_{1}+X_{2}\right)$ el {\emph{segundo ciclo de regeneraci\'on}}, y as\'i sucesivamente.

Sea $X=X_{1}$ y sea $F$ la funci\'on de distrbuci\'on de $X$


\begin{Def}
Se define el proceso estacionario, $\left\{V^{*}\left(t\right),t\geq0\right\}$, para $\left\{V\left(t\right),t\geq0\right\}$ por

\begin{eqnarray*}
\prob\left\{V\left(t\right)\in A\right\}=\frac{1}{\esp\left[X\right]}\int_{0}^{\infty}\prob\left\{V\left(t+x\right)\in A|X>x\right\}\left(1-F\left(x\right)\right)dx,
\end{eqnarray*} 
para todo $t\geq0$ y todo conjunto de Borel $A$.
\end{Def}

\begin{Def}
Una distribuci\'on se dice que es {\emph{aritm\'etica}} si todos sus puntos de incremento son m\'ultiplos de la forma $0,\lambda, 2\lambda,\ldots$ para alguna $\lambda>0$ entera.
\end{Def}


\begin{Def}
Una modificaci\'on medible de un proceso $\left\{V\left(t\right),t\geq0\right\}$, es una versi\'on de este, $\left\{V\left(t,w\right)\right\}$ conjuntamente medible para $t\geq0$ y para $w\in S$, $S$ espacio de estados para $\left\{V\left(t\right),t\geq0\right\}$.
\end{Def}

\begin{Teo}
Sea $\left\{V\left(t\right),t\geq\right\}$ un proceso regenerativo no negativo con modificaci\'on medible. Sea $\esp\left[X\right]<\infty$. Entonces el proceso estacionario dado por la ecuaci\'on anterior est\'a bien definido y tiene funci\'on de distribuci\'on independiente de $t$, adem\'as
\begin{itemize}
\item[i)] \begin{eqnarray*}
\esp\left[V^{*}\left(0\right)\right]&=&\frac{\esp\left[\int_{0}^{X}V\left(s\right)ds\right]}{\esp\left[X\right]}\end{eqnarray*}
\item[ii)] Si $\esp\left[V^{*}\left(0\right)\right]<\infty$, equivalentemente, si $\esp\left[\int_{0}^{X}V\left(s\right)ds\right]<\infty$,entonces
\begin{eqnarray*}
\frac{\int_{0}^{t}V\left(s\right)ds}{t}\rightarrow\frac{\esp\left[\int_{0}^{X}V\left(s\right)ds\right]}{\esp\left[X\right]}
\end{eqnarray*}
con probabilidad 1 y en media, cuando $t\rightarrow\infty$.
\end{itemize}
\end{Teo}

\begin{Coro}
Sea $\left\{V\left(t\right),t\geq0\right\}$ un proceso regenerativo no negativo, con modificaci\'on medible. Si $\esp <\infty$, $F$ es no-aritm\'etica, y para todo $x\geq0$, $P\left\{V\left(t\right)\leq x,C>x\right\}$ es de variaci\'on acotada como funci\'on de $t$ en cada intervalo finito $\left[0,\tau\right]$, entonces $V\left(t\right)$ converge en distribuci\'on  cuando $t\rightarrow\infty$ y $$\esp V=\frac{\esp \int_{0}^{X}V\left(s\right)ds}{\esp X}$$
Donde $V$ tiene la distribuci\'on l\'imite de $V\left(t\right)$ cuando $t\rightarrow\infty$.

\end{Coro}

Para el caso discreto se tienen resultados similares.



%______________________________________________________________________
%\section{Procesos de Renovaci\'on}
%______________________________________________________________________

\begin{Def}\label{Def.Tn}
Sean $0\leq T_{1}\leq T_{2}\leq \ldots$ son tiempos aleatorios infinitos en los cuales ocurren ciertos eventos. El n\'umero de tiempos $T_{n}$ en el intervalo $\left[0,t\right)$ es

\begin{eqnarray}
N\left(t\right)=\sum_{n=1}^{\infty}\indora\left(T_{n}\leq t\right),
\end{eqnarray}
para $t\geq0$.
\end{Def}

Si se consideran los puntos $T_{n}$ como elementos de $\rea_{+}$, y $N\left(t\right)$ es el n\'umero de puntos en $\rea$. El proceso denotado por $\left\{N\left(t\right):t\geq0\right\}$, denotado por $N\left(t\right)$, es un proceso puntual en $\rea_{+}$. Los $T_{n}$ son los tiempos de ocurrencia, el proceso puntual $N\left(t\right)$ es simple si su n\'umero de ocurrencias son distintas: $0<T_{1}<T_{2}<\ldots$ casi seguramente.

\begin{Def}
Un proceso puntual $N\left(t\right)$ es un proceso de renovaci\'on si los tiempos de interocurrencia $\xi_{n}=T_{n}-T_{n-1}$, para $n\geq1$, son independientes e identicamente distribuidos con distribuci\'on $F$, donde $F\left(0\right)=0$ y $T_{0}=0$. Los $T_{n}$ son llamados tiempos de renovaci\'on, referente a la independencia o renovaci\'on de la informaci\'on estoc\'astica en estos tiempos. Los $\xi_{n}$ son los tiempos de inter-renovaci\'on, y $N\left(t\right)$ es el n\'umero de renovaciones en el intervalo $\left[0,t\right)$
\end{Def}


\begin{Note}
Para definir un proceso de renovaci\'on para cualquier contexto, solamente hay que especificar una distribuci\'on $F$, con $F\left(0\right)=0$, para los tiempos de inter-renovaci\'on. La funci\'on $F$ en turno degune las otra variables aleatorias. De manera formal, existe un espacio de probabilidad y una sucesi\'on de variables aleatorias $\xi_{1},\xi_{2},\ldots$ definidas en este con distribuci\'on $F$. Entonces las otras cantidades son $T_{n}=\sum_{k=1}^{n}\xi_{k}$ y $N\left(t\right)=\sum_{n=1}^{\infty}\indora\left(T_{n}\leq t\right)$, donde $T_{n}\rightarrow\infty$ casi seguramente por la Ley Fuerte de los Grandes Números.
\end{Note}

%___________________________________________________________________________________________
%
%\subsection*{Teorema Principal de Renovaci\'on}
%___________________________________________________________________________________________
%

\begin{Note} Una funci\'on $h:\rea_{+}\rightarrow\rea$ es Directamente Riemann Integrable en los siguientes casos:
\begin{itemize}
\item[a)] $h\left(t\right)\geq0$ es decreciente y Riemann Integrable.
\item[b)] $h$ es continua excepto posiblemente en un conjunto de Lebesgue de medida 0, y $|h\left(t\right)|\leq b\left(t\right)$, donde $b$ es DRI.
\end{itemize}
\end{Note}

\begin{Teo}[Teorema Principal de Renovaci\'on]
Si $F$ es no aritm\'etica y $h\left(t\right)$ es Directamente Riemann Integrable (DRI), entonces

\begin{eqnarray*}
lim_{t\rightarrow\infty}U\star h=\frac{1}{\mu}\int_{\rea_{+}}h\left(s\right)ds.
\end{eqnarray*}
\end{Teo}

\begin{Prop}
Cualquier funci\'on $H\left(t\right)$ acotada en intervalos finitos y que es 0 para $t<0$ puede expresarse como
\begin{eqnarray*}
H\left(t\right)=U\star h\left(t\right)\textrm{,  donde }h\left(t\right)=H\left(t\right)-F\star H\left(t\right)
\end{eqnarray*}
\end{Prop}

\begin{Def}
Un proceso estoc\'astico $X\left(t\right)$ es crudamente regenerativo en un tiempo aleatorio positivo $T$ si
\begin{eqnarray*}
\esp\left[X\left(T+t\right)|T\right]=\esp\left[X\left(t\right)\right]\textrm{, para }t\geq0,\end{eqnarray*}
y con las esperanzas anteriores finitas.
\end{Def}

\begin{Prop}
Sup\'ongase que $X\left(t\right)$ es un proceso crudamente regenerativo en $T$, que tiene distribuci\'on $F$. Si $\esp\left[X\left(t\right)\right]$ es acotado en intervalos finitos, entonces
\begin{eqnarray*}
\esp\left[X\left(t\right)\right]=U\star h\left(t\right)\textrm{,  donde }h\left(t\right)=\esp\left[X\left(t\right)\indora\left(T>t\right)\right].
\end{eqnarray*}
\end{Prop}

\begin{Teo}[Regeneraci\'on Cruda]
Sup\'ongase que $X\left(t\right)$ es un proceso con valores positivo crudamente regenerativo en $T$, y def\'inase $M=\sup\left\{|X\left(t\right)|:t\leq T\right\}$. Si $T$ es no aritm\'etico y $M$ y $MT$ tienen media finita, entonces
\begin{eqnarray*}
lim_{t\rightarrow\infty}\esp\left[X\left(t\right)\right]=\frac{1}{\mu}\int_{\rea_{+}}h\left(s\right)ds,
\end{eqnarray*}
donde $h\left(t\right)=\esp\left[X\left(t\right)\indora\left(T>t\right)\right]$.
\end{Teo}

%___________________________________________________________________________________________
%
%\subsection*{Propiedades de los Procesos de Renovaci\'on}
%___________________________________________________________________________________________
%

Los tiempos $T_{n}$ est\'an relacionados con los conteos de $N\left(t\right)$ por

\begin{eqnarray*}
\left\{N\left(t\right)\geq n\right\}&=&\left\{T_{n}\leq t\right\}\\
T_{N\left(t\right)}\leq &t&<T_{N\left(t\right)+1},
\end{eqnarray*}

adem\'as $N\left(T_{n}\right)=n$, y 

\begin{eqnarray*}
N\left(t\right)=\max\left\{n:T_{n}\leq t\right\}=\min\left\{n:T_{n+1}>t\right\}
\end{eqnarray*}

Por propiedades de la convoluci\'on se sabe que

\begin{eqnarray*}
P\left\{T_{n}\leq t\right\}=F^{n\star}\left(t\right)
\end{eqnarray*}
que es la $n$-\'esima convoluci\'on de $F$. Entonces 

\begin{eqnarray*}
\left\{N\left(t\right)\geq n\right\}&=&\left\{T_{n}\leq t\right\}\\
P\left\{N\left(t\right)\leq n\right\}&=&1-F^{\left(n+1\right)\star}\left(t\right)
\end{eqnarray*}

Adem\'as usando el hecho de que $\esp\left[N\left(t\right)\right]=\sum_{n=1}^{\infty}P\left\{N\left(t\right)\geq n\right\}$
se tiene que

\begin{eqnarray*}
\esp\left[N\left(t\right)\right]=\sum_{n=1}^{\infty}F^{n\star}\left(t\right)
\end{eqnarray*}

\begin{Prop}
Para cada $t\geq0$, la funci\'on generadora de momentos $\esp\left[e^{\alpha N\left(t\right)}\right]$ existe para alguna $\alpha$ en una vecindad del 0, y de aqu\'i que $\esp\left[N\left(t\right)^{m}\right]<\infty$, para $m\geq1$.
\end{Prop}


\begin{Note}
Si el primer tiempo de renovaci\'on $\xi_{1}$ no tiene la misma distribuci\'on que el resto de las $\xi_{n}$, para $n\geq2$, a $N\left(t\right)$ se le llama Proceso de Renovaci\'on retardado, donde si $\xi$ tiene distribuci\'on $G$, entonces el tiempo $T_{n}$ de la $n$-\'esima renovaci\'on tiene distribuci\'on $G\star F^{\left(n-1\right)\star}\left(t\right)$
\end{Note}


\begin{Teo}
Para una constante $\mu\leq\infty$ ( o variable aleatoria), las siguientes expresiones son equivalentes:

\begin{eqnarray}
lim_{n\rightarrow\infty}n^{-1}T_{n}&=&\mu,\textrm{ c.s.}\\
lim_{t\rightarrow\infty}t^{-1}N\left(t\right)&=&1/\mu,\textrm{ c.s.}
\end{eqnarray}
\end{Teo}


Es decir, $T_{n}$ satisface la Ley Fuerte de los Grandes N\'umeros s\'i y s\'olo s\'i $N\left/t\right)$ la cumple.


\begin{Coro}[Ley Fuerte de los Grandes N\'umeros para Procesos de Renovaci\'on]
Si $N\left(t\right)$ es un proceso de renovaci\'on cuyos tiempos de inter-renovaci\'on tienen media $\mu\leq\infty$, entonces
\begin{eqnarray}
t^{-1}N\left(t\right)\rightarrow 1/\mu,\textrm{ c.s. cuando }t\rightarrow\infty.
\end{eqnarray}

\end{Coro}


Considerar el proceso estoc\'astico de valores reales $\left\{Z\left(t\right):t\geq0\right\}$ en el mismo espacio de probabilidad que $N\left(t\right)$

\begin{Def}
Para el proceso $\left\{Z\left(t\right):t\geq0\right\}$ se define la fluctuaci\'on m\'axima de $Z\left(t\right)$ en el intervalo $\left(T_{n-1},T_{n}\right]$:
\begin{eqnarray*}
M_{n}=\sup_{T_{n-1}<t\leq T_{n}}|Z\left(t\right)-Z\left(T_{n-1}\right)|
\end{eqnarray*}
\end{Def}

\begin{Teo}
Sup\'ongase que $n^{-1}T_{n}\rightarrow\mu$ c.s. cuando $n\rightarrow\infty$, donde $\mu\leq\infty$ es una constante o variable aleatoria. Sea $a$ una constante o variable aleatoria que puede ser infinita cuando $\mu$ es finita, y considere las expresiones l\'imite:
\begin{eqnarray}
lim_{n\rightarrow\infty}n^{-1}Z\left(T_{n}\right)&=&a,\textrm{ c.s.}\\
lim_{t\rightarrow\infty}t^{-1}Z\left(t\right)&=&a/\mu,\textrm{ c.s.}
\end{eqnarray}
La segunda expresi\'on implica la primera. Conversamente, la primera implica la segunda si el proceso $Z\left(t\right)$ es creciente, o si $lim_{n\rightarrow\infty}n^{-1}M_{n}=0$ c.s.
\end{Teo}

\begin{Coro}
Si $N\left(t\right)$ es un proceso de renovaci\'on, y $\left(Z\left(T_{n}\right)-Z\left(T_{n-1}\right),M_{n}\right)$, para $n\geq1$, son variables aleatorias independientes e id\'enticamente distribuidas con media finita, entonces,
\begin{eqnarray}
lim_{t\rightarrow\infty}t^{-1}Z\left(t\right)\rightarrow\frac{\esp\left[Z\left(T_{1}\right)-Z\left(T_{0}\right)\right]}{\esp\left[T_{1}\right]},\textrm{ c.s. cuando  }t\rightarrow\infty.
\end{eqnarray}
\end{Coro}



%___________________________________________________________________________________________
%
%\subsection{Propiedades de los Procesos de Renovaci\'on}
%___________________________________________________________________________________________
%

Los tiempos $T_{n}$ est\'an relacionados con los conteos de $N\left(t\right)$ por

\begin{eqnarray*}
\left\{N\left(t\right)\geq n\right\}&=&\left\{T_{n}\leq t\right\}\\
T_{N\left(t\right)}\leq &t&<T_{N\left(t\right)+1},
\end{eqnarray*}

adem\'as $N\left(T_{n}\right)=n$, y 

\begin{eqnarray*}
N\left(t\right)=\max\left\{n:T_{n}\leq t\right\}=\min\left\{n:T_{n+1}>t\right\}
\end{eqnarray*}

Por propiedades de la convoluci\'on se sabe que

\begin{eqnarray*}
P\left\{T_{n}\leq t\right\}=F^{n\star}\left(t\right)
\end{eqnarray*}
que es la $n$-\'esima convoluci\'on de $F$. Entonces 

\begin{eqnarray*}
\left\{N\left(t\right)\geq n\right\}&=&\left\{T_{n}\leq t\right\}\\
P\left\{N\left(t\right)\leq n\right\}&=&1-F^{\left(n+1\right)\star}\left(t\right)
\end{eqnarray*}

Adem\'as usando el hecho de que $\esp\left[N\left(t\right)\right]=\sum_{n=1}^{\infty}P\left\{N\left(t\right)\geq n\right\}$
se tiene que

\begin{eqnarray*}
\esp\left[N\left(t\right)\right]=\sum_{n=1}^{\infty}F^{n\star}\left(t\right)
\end{eqnarray*}

\begin{Prop}
Para cada $t\geq0$, la funci\'on generadora de momentos $\esp\left[e^{\alpha N\left(t\right)}\right]$ existe para alguna $\alpha$ en una vecindad del 0, y de aqu\'i que $\esp\left[N\left(t\right)^{m}\right]<\infty$, para $m\geq1$.
\end{Prop}


\begin{Note}
Si el primer tiempo de renovaci\'on $\xi_{1}$ no tiene la misma distribuci\'on que el resto de las $\xi_{n}$, para $n\geq2$, a $N\left(t\right)$ se le llama Proceso de Renovaci\'on retardado, donde si $\xi$ tiene distribuci\'on $G$, entonces el tiempo $T_{n}$ de la $n$-\'esima renovaci\'on tiene distribuci\'on $G\star F^{\left(n-1\right)\star}\left(t\right)$
\end{Note}


\begin{Teo}
Para una constante $\mu\leq\infty$ ( o variable aleatoria), las siguientes expresiones son equivalentes:

\begin{eqnarray}
lim_{n\rightarrow\infty}n^{-1}T_{n}&=&\mu,\textrm{ c.s.}\\
lim_{t\rightarrow\infty}t^{-1}N\left(t\right)&=&1/\mu,\textrm{ c.s.}
\end{eqnarray}
\end{Teo}


Es decir, $T_{n}$ satisface la Ley Fuerte de los Grandes N\'umeros s\'i y s\'olo s\'i $N\left/t\right)$ la cumple.


\begin{Coro}[Ley Fuerte de los Grandes N\'umeros para Procesos de Renovaci\'on]
Si $N\left(t\right)$ es un proceso de renovaci\'on cuyos tiempos de inter-renovaci\'on tienen media $\mu\leq\infty$, entonces
\begin{eqnarray}
t^{-1}N\left(t\right)\rightarrow 1/\mu,\textrm{ c.s. cuando }t\rightarrow\infty.
\end{eqnarray}

\end{Coro}


Considerar el proceso estoc\'astico de valores reales $\left\{Z\left(t\right):t\geq0\right\}$ en el mismo espacio de probabilidad que $N\left(t\right)$

\begin{Def}
Para el proceso $\left\{Z\left(t\right):t\geq0\right\}$ se define la fluctuaci\'on m\'axima de $Z\left(t\right)$ en el intervalo $\left(T_{n-1},T_{n}\right]$:
\begin{eqnarray*}
M_{n}=\sup_{T_{n-1}<t\leq T_{n}}|Z\left(t\right)-Z\left(T_{n-1}\right)|
\end{eqnarray*}
\end{Def}

\begin{Teo}
Sup\'ongase que $n^{-1}T_{n}\rightarrow\mu$ c.s. cuando $n\rightarrow\infty$, donde $\mu\leq\infty$ es una constante o variable aleatoria. Sea $a$ una constante o variable aleatoria que puede ser infinita cuando $\mu$ es finita, y considere las expresiones l\'imite:
\begin{eqnarray}
lim_{n\rightarrow\infty}n^{-1}Z\left(T_{n}\right)&=&a,\textrm{ c.s.}\\
lim_{t\rightarrow\infty}t^{-1}Z\left(t\right)&=&a/\mu,\textrm{ c.s.}
\end{eqnarray}
La segunda expresi\'on implica la primera. Conversamente, la primera implica la segunda si el proceso $Z\left(t\right)$ es creciente, o si $lim_{n\rightarrow\infty}n^{-1}M_{n}=0$ c.s.
\end{Teo}

\begin{Coro}
Si $N\left(t\right)$ es un proceso de renovaci\'on, y $\left(Z\left(T_{n}\right)-Z\left(T_{n-1}\right),M_{n}\right)$, para $n\geq1$, son variables aleatorias independientes e id\'enticamente distribuidas con media finita, entonces,
\begin{eqnarray}
lim_{t\rightarrow\infty}t^{-1}Z\left(t\right)\rightarrow\frac{\esp\left[Z\left(T_{1}\right)-Z\left(T_{0}\right)\right]}{\esp\left[T_{1}\right]},\textrm{ c.s. cuando  }t\rightarrow\infty.
\end{eqnarray}
\end{Coro}


%___________________________________________________________________________________________
%
%\subsection{Propiedades de los Procesos de Renovaci\'on}
%___________________________________________________________________________________________
%

Los tiempos $T_{n}$ est\'an relacionados con los conteos de $N\left(t\right)$ por

\begin{eqnarray*}
\left\{N\left(t\right)\geq n\right\}&=&\left\{T_{n}\leq t\right\}\\
T_{N\left(t\right)}\leq &t&<T_{N\left(t\right)+1},
\end{eqnarray*}

adem\'as $N\left(T_{n}\right)=n$, y 

\begin{eqnarray*}
N\left(t\right)=\max\left\{n:T_{n}\leq t\right\}=\min\left\{n:T_{n+1}>t\right\}
\end{eqnarray*}

Por propiedades de la convoluci\'on se sabe que

\begin{eqnarray*}
P\left\{T_{n}\leq t\right\}=F^{n\star}\left(t\right)
\end{eqnarray*}
que es la $n$-\'esima convoluci\'on de $F$. Entonces 

\begin{eqnarray*}
\left\{N\left(t\right)\geq n\right\}&=&\left\{T_{n}\leq t\right\}\\
P\left\{N\left(t\right)\leq n\right\}&=&1-F^{\left(n+1\right)\star}\left(t\right)
\end{eqnarray*}

Adem\'as usando el hecho de que $\esp\left[N\left(t\right)\right]=\sum_{n=1}^{\infty}P\left\{N\left(t\right)\geq n\right\}$
se tiene que

\begin{eqnarray*}
\esp\left[N\left(t\right)\right]=\sum_{n=1}^{\infty}F^{n\star}\left(t\right)
\end{eqnarray*}

\begin{Prop}
Para cada $t\geq0$, la funci\'on generadora de momentos $\esp\left[e^{\alpha N\left(t\right)}\right]$ existe para alguna $\alpha$ en una vecindad del 0, y de aqu\'i que $\esp\left[N\left(t\right)^{m}\right]<\infty$, para $m\geq1$.
\end{Prop}


\begin{Note}
Si el primer tiempo de renovaci\'on $\xi_{1}$ no tiene la misma distribuci\'on que el resto de las $\xi_{n}$, para $n\geq2$, a $N\left(t\right)$ se le llama Proceso de Renovaci\'on retardado, donde si $\xi$ tiene distribuci\'on $G$, entonces el tiempo $T_{n}$ de la $n$-\'esima renovaci\'on tiene distribuci\'on $G\star F^{\left(n-1\right)\star}\left(t\right)$
\end{Note}


\begin{Teo}
Para una constante $\mu\leq\infty$ ( o variable aleatoria), las siguientes expresiones son equivalentes:

\begin{eqnarray}
lim_{n\rightarrow\infty}n^{-1}T_{n}&=&\mu,\textrm{ c.s.}\\
lim_{t\rightarrow\infty}t^{-1}N\left(t\right)&=&1/\mu,\textrm{ c.s.}
\end{eqnarray}
\end{Teo}


Es decir, $T_{n}$ satisface la Ley Fuerte de los Grandes N\'umeros s\'i y s\'olo s\'i $N\left/t\right)$ la cumple.


\begin{Coro}[Ley Fuerte de los Grandes N\'umeros para Procesos de Renovaci\'on]
Si $N\left(t\right)$ es un proceso de renovaci\'on cuyos tiempos de inter-renovaci\'on tienen media $\mu\leq\infty$, entonces
\begin{eqnarray}
t^{-1}N\left(t\right)\rightarrow 1/\mu,\textrm{ c.s. cuando }t\rightarrow\infty.
\end{eqnarray}

\end{Coro}


Considerar el proceso estoc\'astico de valores reales $\left\{Z\left(t\right):t\geq0\right\}$ en el mismo espacio de probabilidad que $N\left(t\right)$

\begin{Def}
Para el proceso $\left\{Z\left(t\right):t\geq0\right\}$ se define la fluctuaci\'on m\'axima de $Z\left(t\right)$ en el intervalo $\left(T_{n-1},T_{n}\right]$:
\begin{eqnarray*}
M_{n}=\sup_{T_{n-1}<t\leq T_{n}}|Z\left(t\right)-Z\left(T_{n-1}\right)|
\end{eqnarray*}
\end{Def}

\begin{Teo}
Sup\'ongase que $n^{-1}T_{n}\rightarrow\mu$ c.s. cuando $n\rightarrow\infty$, donde $\mu\leq\infty$ es una constante o variable aleatoria. Sea $a$ una constante o variable aleatoria que puede ser infinita cuando $\mu$ es finita, y considere las expresiones l\'imite:
\begin{eqnarray}
lim_{n\rightarrow\infty}n^{-1}Z\left(T_{n}\right)&=&a,\textrm{ c.s.}\\
lim_{t\rightarrow\infty}t^{-1}Z\left(t\right)&=&a/\mu,\textrm{ c.s.}
\end{eqnarray}
La segunda expresi\'on implica la primera. Conversamente, la primera implica la segunda si el proceso $Z\left(t\right)$ es creciente, o si $lim_{n\rightarrow\infty}n^{-1}M_{n}=0$ c.s.
\end{Teo}

\begin{Coro}
Si $N\left(t\right)$ es un proceso de renovaci\'on, y $\left(Z\left(T_{n}\right)-Z\left(T_{n-1}\right),M_{n}\right)$, para $n\geq1$, son variables aleatorias independientes e id\'enticamente distribuidas con media finita, entonces,
\begin{eqnarray}
lim_{t\rightarrow\infty}t^{-1}Z\left(t\right)\rightarrow\frac{\esp\left[Z\left(T_{1}\right)-Z\left(T_{0}\right)\right]}{\esp\left[T_{1}\right]},\textrm{ c.s. cuando  }t\rightarrow\infty.
\end{eqnarray}
\end{Coro}

%___________________________________________________________________________________________
%
%\subsection{Propiedades de los Procesos de Renovaci\'on}
%___________________________________________________________________________________________
%

Los tiempos $T_{n}$ est\'an relacionados con los conteos de $N\left(t\right)$ por

\begin{eqnarray*}
\left\{N\left(t\right)\geq n\right\}&=&\left\{T_{n}\leq t\right\}\\
T_{N\left(t\right)}\leq &t&<T_{N\left(t\right)+1},
\end{eqnarray*}

adem\'as $N\left(T_{n}\right)=n$, y 

\begin{eqnarray*}
N\left(t\right)=\max\left\{n:T_{n}\leq t\right\}=\min\left\{n:T_{n+1}>t\right\}
\end{eqnarray*}

Por propiedades de la convoluci\'on se sabe que

\begin{eqnarray*}
P\left\{T_{n}\leq t\right\}=F^{n\star}\left(t\right)
\end{eqnarray*}
que es la $n$-\'esima convoluci\'on de $F$. Entonces 

\begin{eqnarray*}
\left\{N\left(t\right)\geq n\right\}&=&\left\{T_{n}\leq t\right\}\\
P\left\{N\left(t\right)\leq n\right\}&=&1-F^{\left(n+1\right)\star}\left(t\right)
\end{eqnarray*}

Adem\'as usando el hecho de que $\esp\left[N\left(t\right)\right]=\sum_{n=1}^{\infty}P\left\{N\left(t\right)\geq n\right\}$
se tiene que

\begin{eqnarray*}
\esp\left[N\left(t\right)\right]=\sum_{n=1}^{\infty}F^{n\star}\left(t\right)
\end{eqnarray*}

\begin{Prop}
Para cada $t\geq0$, la funci\'on generadora de momentos $\esp\left[e^{\alpha N\left(t\right)}\right]$ existe para alguna $\alpha$ en una vecindad del 0, y de aqu\'i que $\esp\left[N\left(t\right)^{m}\right]<\infty$, para $m\geq1$.
\end{Prop}


\begin{Note}
Si el primer tiempo de renovaci\'on $\xi_{1}$ no tiene la misma distribuci\'on que el resto de las $\xi_{n}$, para $n\geq2$, a $N\left(t\right)$ se le llama Proceso de Renovaci\'on retardado, donde si $\xi$ tiene distribuci\'on $G$, entonces el tiempo $T_{n}$ de la $n$-\'esima renovaci\'on tiene distribuci\'on $G\star F^{\left(n-1\right)\star}\left(t\right)$
\end{Note}


\begin{Teo}
Para una constante $\mu\leq\infty$ ( o variable aleatoria), las siguientes expresiones son equivalentes:

\begin{eqnarray}
lim_{n\rightarrow\infty}n^{-1}T_{n}&=&\mu,\textrm{ c.s.}\\
lim_{t\rightarrow\infty}t^{-1}N\left(t\right)&=&1/\mu,\textrm{ c.s.}
\end{eqnarray}
\end{Teo}


Es decir, $T_{n}$ satisface la Ley Fuerte de los Grandes N\'umeros s\'i y s\'olo s\'i $N\left/t\right)$ la cumple.


\begin{Coro}[Ley Fuerte de los Grandes N\'umeros para Procesos de Renovaci\'on]
Si $N\left(t\right)$ es un proceso de renovaci\'on cuyos tiempos de inter-renovaci\'on tienen media $\mu\leq\infty$, entonces
\begin{eqnarray}
t^{-1}N\left(t\right)\rightarrow 1/\mu,\textrm{ c.s. cuando }t\rightarrow\infty.
\end{eqnarray}

\end{Coro}


Considerar el proceso estoc\'astico de valores reales $\left\{Z\left(t\right):t\geq0\right\}$ en el mismo espacio de probabilidad que $N\left(t\right)$

\begin{Def}
Para el proceso $\left\{Z\left(t\right):t\geq0\right\}$ se define la fluctuaci\'on m\'axima de $Z\left(t\right)$ en el intervalo $\left(T_{n-1},T_{n}\right]$:
\begin{eqnarray*}
M_{n}=\sup_{T_{n-1}<t\leq T_{n}}|Z\left(t\right)-Z\left(T_{n-1}\right)|
\end{eqnarray*}
\end{Def}

\begin{Teo}
Sup\'ongase que $n^{-1}T_{n}\rightarrow\mu$ c.s. cuando $n\rightarrow\infty$, donde $\mu\leq\infty$ es una constante o variable aleatoria. Sea $a$ una constante o variable aleatoria que puede ser infinita cuando $\mu$ es finita, y considere las expresiones l\'imite:
\begin{eqnarray}
lim_{n\rightarrow\infty}n^{-1}Z\left(T_{n}\right)&=&a,\textrm{ c.s.}\\
lim_{t\rightarrow\infty}t^{-1}Z\left(t\right)&=&a/\mu,\textrm{ c.s.}
\end{eqnarray}
La segunda expresi\'on implica la primera. Conversamente, la primera implica la segunda si el proceso $Z\left(t\right)$ es creciente, o si $lim_{n\rightarrow\infty}n^{-1}M_{n}=0$ c.s.
\end{Teo}

\begin{Coro}
Si $N\left(t\right)$ es un proceso de renovaci\'on, y $\left(Z\left(T_{n}\right)-Z\left(T_{n-1}\right),M_{n}\right)$, para $n\geq1$, son variables aleatorias independientes e id\'enticamente distribuidas con media finita, entonces,
\begin{eqnarray}
lim_{t\rightarrow\infty}t^{-1}Z\left(t\right)\rightarrow\frac{\esp\left[Z\left(T_{1}\right)-Z\left(T_{0}\right)\right]}{\esp\left[T_{1}\right]},\textrm{ c.s. cuando  }t\rightarrow\infty.
\end{eqnarray}
\end{Coro}
%___________________________________________________________________________________________
%
%\subsection{Propiedades de los Procesos de Renovaci\'on}
%___________________________________________________________________________________________
%

Los tiempos $T_{n}$ est\'an relacionados con los conteos de $N\left(t\right)$ por

\begin{eqnarray*}
\left\{N\left(t\right)\geq n\right\}&=&\left\{T_{n}\leq t\right\}\\
T_{N\left(t\right)}\leq &t&<T_{N\left(t\right)+1},
\end{eqnarray*}

adem\'as $N\left(T_{n}\right)=n$, y 

\begin{eqnarray*}
N\left(t\right)=\max\left\{n:T_{n}\leq t\right\}=\min\left\{n:T_{n+1}>t\right\}
\end{eqnarray*}

Por propiedades de la convoluci\'on se sabe que

\begin{eqnarray*}
P\left\{T_{n}\leq t\right\}=F^{n\star}\left(t\right)
\end{eqnarray*}
que es la $n$-\'esima convoluci\'on de $F$. Entonces 

\begin{eqnarray*}
\left\{N\left(t\right)\geq n\right\}&=&\left\{T_{n}\leq t\right\}\\
P\left\{N\left(t\right)\leq n\right\}&=&1-F^{\left(n+1\right)\star}\left(t\right)
\end{eqnarray*}

Adem\'as usando el hecho de que $\esp\left[N\left(t\right)\right]=\sum_{n=1}^{\infty}P\left\{N\left(t\right)\geq n\right\}$
se tiene que

\begin{eqnarray*}
\esp\left[N\left(t\right)\right]=\sum_{n=1}^{\infty}F^{n\star}\left(t\right)
\end{eqnarray*}

\begin{Prop}
Para cada $t\geq0$, la funci\'on generadora de momentos $\esp\left[e^{\alpha N\left(t\right)}\right]$ existe para alguna $\alpha$ en una vecindad del 0, y de aqu\'i que $\esp\left[N\left(t\right)^{m}\right]<\infty$, para $m\geq1$.
\end{Prop}


\begin{Note}
Si el primer tiempo de renovaci\'on $\xi_{1}$ no tiene la misma distribuci\'on que el resto de las $\xi_{n}$, para $n\geq2$, a $N\left(t\right)$ se le llama Proceso de Renovaci\'on retardado, donde si $\xi$ tiene distribuci\'on $G$, entonces el tiempo $T_{n}$ de la $n$-\'esima renovaci\'on tiene distribuci\'on $G\star F^{\left(n-1\right)\star}\left(t\right)$
\end{Note}


\begin{Teo}
Para una constante $\mu\leq\infty$ ( o variable aleatoria), las siguientes expresiones son equivalentes:

\begin{eqnarray}
lim_{n\rightarrow\infty}n^{-1}T_{n}&=&\mu,\textrm{ c.s.}\\
lim_{t\rightarrow\infty}t^{-1}N\left(t\right)&=&1/\mu,\textrm{ c.s.}
\end{eqnarray}
\end{Teo}


Es decir, $T_{n}$ satisface la Ley Fuerte de los Grandes N\'umeros s\'i y s\'olo s\'i $N\left/t\right)$ la cumple.


\begin{Coro}[Ley Fuerte de los Grandes N\'umeros para Procesos de Renovaci\'on]
Si $N\left(t\right)$ es un proceso de renovaci\'on cuyos tiempos de inter-renovaci\'on tienen media $\mu\leq\infty$, entonces
\begin{eqnarray}
t^{-1}N\left(t\right)\rightarrow 1/\mu,\textrm{ c.s. cuando }t\rightarrow\infty.
\end{eqnarray}

\end{Coro}


Considerar el proceso estoc\'astico de valores reales $\left\{Z\left(t\right):t\geq0\right\}$ en el mismo espacio de probabilidad que $N\left(t\right)$

\begin{Def}
Para el proceso $\left\{Z\left(t\right):t\geq0\right\}$ se define la fluctuaci\'on m\'axima de $Z\left(t\right)$ en el intervalo $\left(T_{n-1},T_{n}\right]$:
\begin{eqnarray*}
M_{n}=\sup_{T_{n-1}<t\leq T_{n}}|Z\left(t\right)-Z\left(T_{n-1}\right)|
\end{eqnarray*}
\end{Def}

\begin{Teo}
Sup\'ongase que $n^{-1}T_{n}\rightarrow\mu$ c.s. cuando $n\rightarrow\infty$, donde $\mu\leq\infty$ es una constante o variable aleatoria. Sea $a$ una constante o variable aleatoria que puede ser infinita cuando $\mu$ es finita, y considere las expresiones l\'imite:
\begin{eqnarray}
lim_{n\rightarrow\infty}n^{-1}Z\left(T_{n}\right)&=&a,\textrm{ c.s.}\\
lim_{t\rightarrow\infty}t^{-1}Z\left(t\right)&=&a/\mu,\textrm{ c.s.}
\end{eqnarray}
La segunda expresi\'on implica la primera. Conversamente, la primera implica la segunda si el proceso $Z\left(t\right)$ es creciente, o si $lim_{n\rightarrow\infty}n^{-1}M_{n}=0$ c.s.
\end{Teo}

\begin{Coro}
Si $N\left(t\right)$ es un proceso de renovaci\'on, y $\left(Z\left(T_{n}\right)-Z\left(T_{n-1}\right),M_{n}\right)$, para $n\geq1$, son variables aleatorias independientes e id\'enticamente distribuidas con media finita, entonces,
\begin{eqnarray}
lim_{t\rightarrow\infty}t^{-1}Z\left(t\right)\rightarrow\frac{\esp\left[Z\left(T_{1}\right)-Z\left(T_{0}\right)\right]}{\esp\left[T_{1}\right]},\textrm{ c.s. cuando  }t\rightarrow\infty.
\end{eqnarray}
\end{Coro}


%___________________________________________________________________________________________
%
%\subsection*{Funci\'on de Renovaci\'on}
%___________________________________________________________________________________________
%


\begin{Def}
Sea $h\left(t\right)$ funci\'on de valores reales en $\rea$ acotada en intervalos finitos e igual a cero para $t<0$ La ecuaci\'on de renovaci\'on para $h\left(t\right)$ y la distribuci\'on $F$ es

\begin{eqnarray}\label{Ec.Renovacion}
H\left(t\right)=h\left(t\right)+\int_{\left[0,t\right]}H\left(t-s\right)dF\left(s\right)\textrm{,    }t\geq0,
\end{eqnarray}
donde $H\left(t\right)$ es una funci\'on de valores reales. Esto es $H=h+F\star H$. Decimos que $H\left(t\right)$ es soluci\'on de esta ecuaci\'on si satisface la ecuaci\'on, y es acotada en intervalos finitos e iguales a cero para $t<0$.
\end{Def}

\begin{Prop}
La funci\'on $U\star h\left(t\right)$ es la \'unica soluci\'on de la ecuaci\'on de renovaci\'on (\ref{Ec.Renovacion}).
\end{Prop}

\begin{Teo}[Teorema Renovaci\'on Elemental]
\begin{eqnarray*}
t^{-1}U\left(t\right)\rightarrow 1/\mu\textrm{,    cuando }t\rightarrow\infty.
\end{eqnarray*}
\end{Teo}

%___________________________________________________________________________________________
%
%\subsection{Funci\'on de Renovaci\'on}
%___________________________________________________________________________________________
%


Sup\'ongase que $N\left(t\right)$ es un proceso de renovaci\'on con distribuci\'on $F$ con media finita $\mu$.

\begin{Def}
La funci\'on de renovaci\'on asociada con la distribuci\'on $F$, del proceso $N\left(t\right)$, es
\begin{eqnarray*}
U\left(t\right)=\sum_{n=1}^{\infty}F^{n\star}\left(t\right),\textrm{   }t\geq0,
\end{eqnarray*}
donde $F^{0\star}\left(t\right)=\indora\left(t\geq0\right)$.
\end{Def}


\begin{Prop}
Sup\'ongase que la distribuci\'on de inter-renovaci\'on $F$ tiene densidad $f$. Entonces $U\left(t\right)$ tambi\'en tiene densidad, para $t>0$, y es $U^{'}\left(t\right)=\sum_{n=0}^{\infty}f^{n\star}\left(t\right)$. Adem\'as
\begin{eqnarray*}
\prob\left\{N\left(t\right)>N\left(t-\right)\right\}=0\textrm{,   }t\geq0.
\end{eqnarray*}
\end{Prop}

\begin{Def}
La Transformada de Laplace-Stieljes de $F$ est\'a dada por

\begin{eqnarray*}
\hat{F}\left(\alpha\right)=\int_{\rea_{+}}e^{-\alpha t}dF\left(t\right)\textrm{,  }\alpha\geq0.
\end{eqnarray*}
\end{Def}

Entonces

\begin{eqnarray*}
\hat{U}\left(\alpha\right)=\sum_{n=0}^{\infty}\hat{F^{n\star}}\left(\alpha\right)=\sum_{n=0}^{\infty}\hat{F}\left(\alpha\right)^{n}=\frac{1}{1-\hat{F}\left(\alpha\right)}.
\end{eqnarray*}


\begin{Prop}
La Transformada de Laplace $\hat{U}\left(\alpha\right)$ y $\hat{F}\left(\alpha\right)$ determina una a la otra de manera \'unica por la relaci\'on $\hat{U}\left(\alpha\right)=\frac{1}{1-\hat{F}\left(\alpha\right)}$.
\end{Prop}


\begin{Note}
Un proceso de renovaci\'on $N\left(t\right)$ cuyos tiempos de inter-renovaci\'on tienen media finita, es un proceso Poisson con tasa $\lambda$ si y s\'olo s\'i $\esp\left[U\left(t\right)\right]=\lambda t$, para $t\geq0$.
\end{Note}


\begin{Teo}
Sea $N\left(t\right)$ un proceso puntual simple con puntos de localizaci\'on $T_{n}$ tal que $\eta\left(t\right)=\esp\left[N\left(\right)\right]$ es finita para cada $t$. Entonces para cualquier funci\'on $f:\rea_{+}\rightarrow\rea$,
\begin{eqnarray*}
\esp\left[\sum_{n=1}^{N\left(\right)}f\left(T_{n}\right)\right]=\int_{\left(0,t\right]}f\left(s\right)d\eta\left(s\right)\textrm{,  }t\geq0,
\end{eqnarray*}
suponiendo que la integral exista. Adem\'as si $X_{1},X_{2},\ldots$ son variables aleatorias definidas en el mismo espacio de probabilidad que el proceso $N\left(t\right)$ tal que $\esp\left[X_{n}|T_{n}=s\right]=f\left(s\right)$, independiente de $n$. Entonces
\begin{eqnarray*}
\esp\left[\sum_{n=1}^{N\left(t\right)}X_{n}\right]=\int_{\left(0,t\right]}f\left(s\right)d\eta\left(s\right)\textrm{,  }t\geq0,
\end{eqnarray*} 
suponiendo que la integral exista. 
\end{Teo}

\begin{Coro}[Identidad de Wald para Renovaciones]
Para el proceso de renovaci\'on $N\left(t\right)$,
\begin{eqnarray*}
\esp\left[T_{N\left(t\right)+1}\right]=\mu\esp\left[N\left(t\right)+1\right]\textrm{,  }t\geq0,
\end{eqnarray*}  
\end{Coro}

%______________________________________________________________________
%\subsection{Procesos de Renovaci\'on}
%______________________________________________________________________

\begin{Def}\label{Def.Tn}
Sean $0\leq T_{1}\leq T_{2}\leq \ldots$ son tiempos aleatorios infinitos en los cuales ocurren ciertos eventos. El n\'umero de tiempos $T_{n}$ en el intervalo $\left[0,t\right)$ es

\begin{eqnarray}
N\left(t\right)=\sum_{n=1}^{\infty}\indora\left(T_{n}\leq t\right),
\end{eqnarray}
para $t\geq0$.
\end{Def}

Si se consideran los puntos $T_{n}$ como elementos de $\rea_{+}$, y $N\left(t\right)$ es el n\'umero de puntos en $\rea$. El proceso denotado por $\left\{N\left(t\right):t\geq0\right\}$, denotado por $N\left(t\right)$, es un proceso puntual en $\rea_{+}$. Los $T_{n}$ son los tiempos de ocurrencia, el proceso puntual $N\left(t\right)$ es simple si su n\'umero de ocurrencias son distintas: $0<T_{1}<T_{2}<\ldots$ casi seguramente.

\begin{Def}
Un proceso puntual $N\left(t\right)$ es un proceso de renovaci\'on si los tiempos de interocurrencia $\xi_{n}=T_{n}-T_{n-1}$, para $n\geq1$, son independientes e identicamente distribuidos con distribuci\'on $F$, donde $F\left(0\right)=0$ y $T_{0}=0$. Los $T_{n}$ son llamados tiempos de renovaci\'on, referente a la independencia o renovaci\'on de la informaci\'on estoc\'astica en estos tiempos. Los $\xi_{n}$ son los tiempos de inter-renovaci\'on, y $N\left(t\right)$ es el n\'umero de renovaciones en el intervalo $\left[0,t\right)$
\end{Def}


\begin{Note}
Para definir un proceso de renovaci\'on para cualquier contexto, solamente hay que especificar una distribuci\'on $F$, con $F\left(0\right)=0$, para los tiempos de inter-renovaci\'on. La funci\'on $F$ en turno degune las otra variables aleatorias. De manera formal, existe un espacio de probabilidad y una sucesi\'on de variables aleatorias $\xi_{1},\xi_{2},\ldots$ definidas en este con distribuci\'on $F$. Entonces las otras cantidades son $T_{n}=\sum_{k=1}^{n}\xi_{k}$ y $N\left(t\right)=\sum_{n=1}^{\infty}\indora\left(T_{n}\leq t\right)$, donde $T_{n}\rightarrow\infty$ casi seguramente por la Ley Fuerte de los Grandes Números.
\end{Note}

\begin{Def}\label{Def.Tn}
Sean $0\leq T_{1}\leq T_{2}\leq \ldots$ son tiempos aleatorios infinitos en los cuales ocurren ciertos eventos. El n\'umero de tiempos $T_{n}$ en el intervalo $\left[0,t\right)$ es

\begin{eqnarray}
N\left(t\right)=\sum_{n=1}^{\infty}\indora\left(T_{n}\leq t\right),
\end{eqnarray}
para $t\geq0$.
\end{Def}

Si se consideran los puntos $T_{n}$ como elementos de $\rea_{+}$, y $N\left(t\right)$ es el n\'umero de puntos en $\rea$. El proceso denotado por $\left\{N\left(t\right):t\geq0\right\}$, denotado por $N\left(t\right)$, es un proceso puntual en $\rea_{+}$. Los $T_{n}$ son los tiempos de ocurrencia, el proceso puntual $N\left(t\right)$ es simple si su n\'umero de ocurrencias son distintas: $0<T_{1}<T_{2}<\ldots$ casi seguramente.

\begin{Def}
Un proceso puntual $N\left(t\right)$ es un proceso de renovaci\'on si los tiempos de interocurrencia $\xi_{n}=T_{n}-T_{n-1}$, para $n\geq1$, son independientes e identicamente distribuidos con distribuci\'on $F$, donde $F\left(0\right)=0$ y $T_{0}=0$. Los $T_{n}$ son llamados tiempos de renovaci\'on, referente a la independencia o renovaci\'on de la informaci\'on estoc\'astica en estos tiempos. Los $\xi_{n}$ son los tiempos de inter-renovaci\'on, y $N\left(t\right)$ es el n\'umero de renovaciones en el intervalo $\left[0,t\right)$
\end{Def}


\begin{Note}
Para definir un proceso de renovaci\'on para cualquier contexto, solamente hay que especificar una distribuci\'on $F$, con $F\left(0\right)=0$, para los tiempos de inter-renovaci\'on. La funci\'on $F$ en turno degune las otra variables aleatorias. De manera formal, existe un espacio de probabilidad y una sucesi\'on de variables aleatorias $\xi_{1},\xi_{2},\ldots$ definidas en este con distribuci\'on $F$. Entonces las otras cantidades son $T_{n}=\sum_{k=1}^{n}\xi_{k}$ y $N\left(t\right)=\sum_{n=1}^{\infty}\indora\left(T_{n}\leq t\right)$, donde $T_{n}\rightarrow\infty$ casi seguramente por la Ley Fuerte de los Grandes N\'umeros.
\end{Note}







Los tiempos $T_{n}$ est\'an relacionados con los conteos de $N\left(t\right)$ por

\begin{eqnarray*}
\left\{N\left(t\right)\geq n\right\}&=&\left\{T_{n}\leq t\right\}\\
T_{N\left(t\right)}\leq &t&<T_{N\left(t\right)+1},
\end{eqnarray*}

adem\'as $N\left(T_{n}\right)=n$, y 

\begin{eqnarray*}
N\left(t\right)=\max\left\{n:T_{n}\leq t\right\}=\min\left\{n:T_{n+1}>t\right\}
\end{eqnarray*}

Por propiedades de la convoluci\'on se sabe que

\begin{eqnarray*}
P\left\{T_{n}\leq t\right\}=F^{n\star}\left(t\right)
\end{eqnarray*}
que es la $n$-\'esima convoluci\'on de $F$. Entonces 

\begin{eqnarray*}
\left\{N\left(t\right)\geq n\right\}&=&\left\{T_{n}\leq t\right\}\\
P\left\{N\left(t\right)\leq n\right\}&=&1-F^{\left(n+1\right)\star}\left(t\right)
\end{eqnarray*}

Adem\'as usando el hecho de que $\esp\left[N\left(t\right)\right]=\sum_{n=1}^{\infty}P\left\{N\left(t\right)\geq n\right\}$
se tiene que

\begin{eqnarray*}
\esp\left[N\left(t\right)\right]=\sum_{n=1}^{\infty}F^{n\star}\left(t\right)
\end{eqnarray*}

\begin{Prop}
Para cada $t\geq0$, la funci\'on generadora de momentos $\esp\left[e^{\alpha N\left(t\right)}\right]$ existe para alguna $\alpha$ en una vecindad del 0, y de aqu\'i que $\esp\left[N\left(t\right)^{m}\right]<\infty$, para $m\geq1$.
\end{Prop}

\begin{Ejem}[\textbf{Proceso Poisson}]

Suponga que se tienen tiempos de inter-renovaci\'on \textit{i.i.d.} del proceso de renovaci\'on $N\left(t\right)$ tienen distribuci\'on exponencial $F\left(t\right)=q-e^{-\lambda t}$ con tasa $\lambda$. Entonces $N\left(t\right)$ es un proceso Poisson con tasa $\lambda$.

\end{Ejem}


\begin{Note}
Si el primer tiempo de renovaci\'on $\xi_{1}$ no tiene la misma distribuci\'on que el resto de las $\xi_{n}$, para $n\geq2$, a $N\left(t\right)$ se le llama Proceso de Renovaci\'on retardado, donde si $\xi$ tiene distribuci\'on $G$, entonces el tiempo $T_{n}$ de la $n$-\'esima renovaci\'on tiene distribuci\'on $G\star F^{\left(n-1\right)\star}\left(t\right)$
\end{Note}


\begin{Teo}
Para una constante $\mu\leq\infty$ ( o variable aleatoria), las siguientes expresiones son equivalentes:

\begin{eqnarray}
lim_{n\rightarrow\infty}n^{-1}T_{n}&=&\mu,\textrm{ c.s.}\\
lim_{t\rightarrow\infty}t^{-1}N\left(t\right)&=&1/\mu,\textrm{ c.s.}
\end{eqnarray}
\end{Teo}


Es decir, $T_{n}$ satisface la Ley Fuerte de los Grandes N\'umeros s\'i y s\'olo s\'i $N\left/t\right)$ la cumple.


\begin{Coro}[Ley Fuerte de los Grandes N\'umeros para Procesos de Renovaci\'on]
Si $N\left(t\right)$ es un proceso de renovaci\'on cuyos tiempos de inter-renovaci\'on tienen media $\mu\leq\infty$, entonces
\begin{eqnarray}
t^{-1}N\left(t\right)\rightarrow 1/\mu,\textrm{ c.s. cuando }t\rightarrow\infty.
\end{eqnarray}

\end{Coro}


Considerar el proceso estoc\'astico de valores reales $\left\{Z\left(t\right):t\geq0\right\}$ en el mismo espacio de probabilidad que $N\left(t\right)$

\begin{Def}
Para el proceso $\left\{Z\left(t\right):t\geq0\right\}$ se define la fluctuaci\'on m\'axima de $Z\left(t\right)$ en el intervalo $\left(T_{n-1},T_{n}\right]$:
\begin{eqnarray*}
M_{n}=\sup_{T_{n-1}<t\leq T_{n}}|Z\left(t\right)-Z\left(T_{n-1}\right)|
\end{eqnarray*}
\end{Def}

\begin{Teo}
Sup\'ongase que $n^{-1}T_{n}\rightarrow\mu$ c.s. cuando $n\rightarrow\infty$, donde $\mu\leq\infty$ es una constante o variable aleatoria. Sea $a$ una constante o variable aleatoria que puede ser infinita cuando $\mu$ es finita, y considere las expresiones l\'imite:
\begin{eqnarray}
lim_{n\rightarrow\infty}n^{-1}Z\left(T_{n}\right)&=&a,\textrm{ c.s.}\\
lim_{t\rightarrow\infty}t^{-1}Z\left(t\right)&=&a/\mu,\textrm{ c.s.}
\end{eqnarray}
La segunda expresi\'on implica la primera. Conversamente, la primera implica la segunda si el proceso $Z\left(t\right)$ es creciente, o si $lim_{n\rightarrow\infty}n^{-1}M_{n}=0$ c.s.
\end{Teo}

\begin{Coro}
Si $N\left(t\right)$ es un proceso de renovaci\'on, y $\left(Z\left(T_{n}\right)-Z\left(T_{n-1}\right),M_{n}\right)$, para $n\geq1$, son variables aleatorias independientes e id\'enticamente distribuidas con media finita, entonces,
\begin{eqnarray}
lim_{t\rightarrow\infty}t^{-1}Z\left(t\right)\rightarrow\frac{\esp\left[Z\left(T_{1}\right)-Z\left(T_{0}\right)\right]}{\esp\left[T_{1}\right]},\textrm{ c.s. cuando  }t\rightarrow\infty.
\end{eqnarray}
\end{Coro}


Sup\'ongase que $N\left(t\right)$ es un proceso de renovaci\'on con distribuci\'on $F$ con media finita $\mu$.

\begin{Def}
La funci\'on de renovaci\'on asociada con la distribuci\'on $F$, del proceso $N\left(t\right)$, es
\begin{eqnarray*}
U\left(t\right)=\sum_{n=1}^{\infty}F^{n\star}\left(t\right),\textrm{   }t\geq0,
\end{eqnarray*}
donde $F^{0\star}\left(t\right)=\indora\left(t\geq0\right)$.
\end{Def}


\begin{Prop}
Sup\'ongase que la distribuci\'on de inter-renovaci\'on $F$ tiene densidad $f$. Entonces $U\left(t\right)$ tambi\'en tiene densidad, para $t>0$, y es $U^{'}\left(t\right)=\sum_{n=0}^{\infty}f^{n\star}\left(t\right)$. Adem\'as
\begin{eqnarray*}
\prob\left\{N\left(t\right)>N\left(t-\right)\right\}=0\textrm{,   }t\geq0.
\end{eqnarray*}
\end{Prop}

\begin{Def}
La Transformada de Laplace-Stieljes de $F$ est\'a dada por

\begin{eqnarray*}
\hat{F}\left(\alpha\right)=\int_{\rea_{+}}e^{-\alpha t}dF\left(t\right)\textrm{,  }\alpha\geq0.
\end{eqnarray*}
\end{Def}

Entonces

\begin{eqnarray*}
\hat{U}\left(\alpha\right)=\sum_{n=0}^{\infty}\hat{F^{n\star}}\left(\alpha\right)=\sum_{n=0}^{\infty}\hat{F}\left(\alpha\right)^{n}=\frac{1}{1-\hat{F}\left(\alpha\right)}.
\end{eqnarray*}


\begin{Prop}
La Transformada de Laplace $\hat{U}\left(\alpha\right)$ y $\hat{F}\left(\alpha\right)$ determina una a la otra de manera \'unica por la relaci\'on $\hat{U}\left(\alpha\right)=\frac{1}{1-\hat{F}\left(\alpha\right)}$.
\end{Prop}


\begin{Note}
Un proceso de renovaci\'on $N\left(t\right)$ cuyos tiempos de inter-renovaci\'on tienen media finita, es un proceso Poisson con tasa $\lambda$ si y s\'olo s\'i $\esp\left[U\left(t\right)\right]=\lambda t$, para $t\geq0$.
\end{Note}


\begin{Teo}
Sea $N\left(t\right)$ un proceso puntual simple con puntos de localizaci\'on $T_{n}$ tal que $\eta\left(t\right)=\esp\left[N\left(\right)\right]$ es finita para cada $t$. Entonces para cualquier funci\'on $f:\rea_{+}\rightarrow\rea$,
\begin{eqnarray*}
\esp\left[\sum_{n=1}^{N\left(\right)}f\left(T_{n}\right)\right]=\int_{\left(0,t\right]}f\left(s\right)d\eta\left(s\right)\textrm{,  }t\geq0,
\end{eqnarray*}
suponiendo que la integral exista. Adem\'as si $X_{1},X_{2},\ldots$ son variables aleatorias definidas en el mismo espacio de probabilidad que el proceso $N\left(t\right)$ tal que $\esp\left[X_{n}|T_{n}=s\right]=f\left(s\right)$, independiente de $n$. Entonces
\begin{eqnarray*}
\esp\left[\sum_{n=1}^{N\left(t\right)}X_{n}\right]=\int_{\left(0,t\right]}f\left(s\right)d\eta\left(s\right)\textrm{,  }t\geq0,
\end{eqnarray*} 
suponiendo que la integral exista. 
\end{Teo}

\begin{Coro}[Identidad de Wald para Renovaciones]
Para el proceso de renovaci\'on $N\left(t\right)$,
\begin{eqnarray*}
\esp\left[T_{N\left(t\right)+1}\right]=\mu\esp\left[N\left(t\right)+1\right]\textrm{,  }t\geq0,
\end{eqnarray*}  
\end{Coro}


\begin{Def}
Sea $h\left(t\right)$ funci\'on de valores reales en $\rea$ acotada en intervalos finitos e igual a cero para $t<0$ La ecuaci\'on de renovaci\'on para $h\left(t\right)$ y la distribuci\'on $F$ es

\begin{eqnarray}\label{Ec.Renovacion}
H\left(t\right)=h\left(t\right)+\int_{\left[0,t\right]}H\left(t-s\right)dF\left(s\right)\textrm{,    }t\geq0,
\end{eqnarray}
donde $H\left(t\right)$ es una funci\'on de valores reales. Esto es $H=h+F\star H$. Decimos que $H\left(t\right)$ es soluci\'on de esta ecuaci\'on si satisface la ecuaci\'on, y es acotada en intervalos finitos e iguales a cero para $t<0$.
\end{Def}

\begin{Prop}
La funci\'on $U\star h\left(t\right)$ es la \'unica soluci\'on de la ecuaci\'on de renovaci\'on (\ref{Ec.Renovacion}).
\end{Prop}

\begin{Teo}[Teorema Renovaci\'on Elemental]
\begin{eqnarray*}
t^{-1}U\left(t\right)\rightarrow 1/\mu\textrm{,    cuando }t\rightarrow\infty.
\end{eqnarray*}
\end{Teo}



Sup\'ongase que $N\left(t\right)$ es un proceso de renovaci\'on con distribuci\'on $F$ con media finita $\mu$.

\begin{Def}
La funci\'on de renovaci\'on asociada con la distribuci\'on $F$, del proceso $N\left(t\right)$, es
\begin{eqnarray*}
U\left(t\right)=\sum_{n=1}^{\infty}F^{n\star}\left(t\right),\textrm{   }t\geq0,
\end{eqnarray*}
donde $F^{0\star}\left(t\right)=\indora\left(t\geq0\right)$.
\end{Def}


\begin{Prop}
Sup\'ongase que la distribuci\'on de inter-renovaci\'on $F$ tiene densidad $f$. Entonces $U\left(t\right)$ tambi\'en tiene densidad, para $t>0$, y es $U^{'}\left(t\right)=\sum_{n=0}^{\infty}f^{n\star}\left(t\right)$. Adem\'as
\begin{eqnarray*}
\prob\left\{N\left(t\right)>N\left(t-\right)\right\}=0\textrm{,   }t\geq0.
\end{eqnarray*}
\end{Prop}

\begin{Def}
La Transformada de Laplace-Stieljes de $F$ est\'a dada por

\begin{eqnarray*}
\hat{F}\left(\alpha\right)=\int_{\rea_{+}}e^{-\alpha t}dF\left(t\right)\textrm{,  }\alpha\geq0.
\end{eqnarray*}
\end{Def}

Entonces

\begin{eqnarray*}
\hat{U}\left(\alpha\right)=\sum_{n=0}^{\infty}\hat{F^{n\star}}\left(\alpha\right)=\sum_{n=0}^{\infty}\hat{F}\left(\alpha\right)^{n}=\frac{1}{1-\hat{F}\left(\alpha\right)}.
\end{eqnarray*}


\begin{Prop}
La Transformada de Laplace $\hat{U}\left(\alpha\right)$ y $\hat{F}\left(\alpha\right)$ determina una a la otra de manera \'unica por la relaci\'on $\hat{U}\left(\alpha\right)=\frac{1}{1-\hat{F}\left(\alpha\right)}$.
\end{Prop}


\begin{Note}
Un proceso de renovaci\'on $N\left(t\right)$ cuyos tiempos de inter-renovaci\'on tienen media finita, es un proceso Poisson con tasa $\lambda$ si y s\'olo s\'i $\esp\left[U\left(t\right)\right]=\lambda t$, para $t\geq0$.
\end{Note}


\begin{Teo}
Sea $N\left(t\right)$ un proceso puntual simple con puntos de localizaci\'on $T_{n}$ tal que $\eta\left(t\right)=\esp\left[N\left(\right)\right]$ es finita para cada $t$. Entonces para cualquier funci\'on $f:\rea_{+}\rightarrow\rea$,
\begin{eqnarray*}
\esp\left[\sum_{n=1}^{N\left(\right)}f\left(T_{n}\right)\right]=\int_{\left(0,t\right]}f\left(s\right)d\eta\left(s\right)\textrm{,  }t\geq0,
\end{eqnarray*}
suponiendo que la integral exista. Adem\'as si $X_{1},X_{2},\ldots$ son variables aleatorias definidas en el mismo espacio de probabilidad que el proceso $N\left(t\right)$ tal que $\esp\left[X_{n}|T_{n}=s\right]=f\left(s\right)$, independiente de $n$. Entonces
\begin{eqnarray*}
\esp\left[\sum_{n=1}^{N\left(t\right)}X_{n}\right]=\int_{\left(0,t\right]}f\left(s\right)d\eta\left(s\right)\textrm{,  }t\geq0,
\end{eqnarray*} 
suponiendo que la integral exista. 
\end{Teo}

\begin{Coro}[Identidad de Wald para Renovaciones]
Para el proceso de renovaci\'on $N\left(t\right)$,
\begin{eqnarray*}
\esp\left[T_{N\left(t\right)+1}\right]=\mu\esp\left[N\left(t\right)+1\right]\textrm{,  }t\geq0,
\end{eqnarray*}  
\end{Coro}


\begin{Def}
Sea $h\left(t\right)$ funci\'on de valores reales en $\rea$ acotada en intervalos finitos e igual a cero para $t<0$ La ecuaci\'on de renovaci\'on para $h\left(t\right)$ y la distribuci\'on $F$ es

\begin{eqnarray}\label{Ec.Renovacion}
H\left(t\right)=h\left(t\right)+\int_{\left[0,t\right]}H\left(t-s\right)dF\left(s\right)\textrm{,    }t\geq0,
\end{eqnarray}
donde $H\left(t\right)$ es una funci\'on de valores reales. Esto es $H=h+F\star H$. Decimos que $H\left(t\right)$ es soluci\'on de esta ecuaci\'on si satisface la ecuaci\'on, y es acotada en intervalos finitos e iguales a cero para $t<0$.
\end{Def}

\begin{Prop}
La funci\'on $U\star h\left(t\right)$ es la \'unica soluci\'on de la ecuaci\'on de renovaci\'on (\ref{Ec.Renovacion}).
\end{Prop}

\begin{Teo}[Teorema Renovaci\'on Elemental]
\begin{eqnarray*}
t^{-1}U\left(t\right)\rightarrow 1/\mu\textrm{,    cuando }t\rightarrow\infty.
\end{eqnarray*}
\end{Teo}


\begin{Note} Una funci\'on $h:\rea_{+}\rightarrow\rea$ es Directamente Riemann Integrable en los siguientes casos:
\begin{itemize}
\item[a)] $h\left(t\right)\geq0$ es decreciente y Riemann Integrable.
\item[b)] $h$ es continua excepto posiblemente en un conjunto de Lebesgue de medida 0, y $|h\left(t\right)|\leq b\left(t\right)$, donde $b$ es DRI.
\end{itemize}
\end{Note}

\begin{Teo}[Teorema Principal de Renovaci\'on]
Si $F$ es no aritm\'etica y $h\left(t\right)$ es Directamente Riemann Integrable (DRI), entonces

\begin{eqnarray*}
lim_{t\rightarrow\infty}U\star h=\frac{1}{\mu}\int_{\rea_{+}}h\left(s\right)ds.
\end{eqnarray*}
\end{Teo}

\begin{Prop}
Cualquier funci\'on $H\left(t\right)$ acotada en intervalos finitos y que es 0 para $t<0$ puede expresarse como
\begin{eqnarray*}
H\left(t\right)=U\star h\left(t\right)\textrm{,  donde }h\left(t\right)=H\left(t\right)-F\star H\left(t\right)
\end{eqnarray*}
\end{Prop}

\begin{Def}
Un proceso estoc\'astico $X\left(t\right)$ es crudamente regenerativo en un tiempo aleatorio positivo $T$ si
\begin{eqnarray*}
\esp\left[X\left(T+t\right)|T\right]=\esp\left[X\left(t\right)\right]\textrm{, para }t\geq0,\end{eqnarray*}
y con las esperanzas anteriores finitas.
\end{Def}

\begin{Prop}
Sup\'ongase que $X\left(t\right)$ es un proceso crudamente regenerativo en $T$, que tiene distribuci\'on $F$. Si $\esp\left[X\left(t\right)\right]$ es acotado en intervalos finitos, entonces
\begin{eqnarray*}
\esp\left[X\left(t\right)\right]=U\star h\left(t\right)\textrm{,  donde }h\left(t\right)=\esp\left[X\left(t\right)\indora\left(T>t\right)\right].
\end{eqnarray*}
\end{Prop}

\begin{Teo}[Regeneraci\'on Cruda]
Sup\'ongase que $X\left(t\right)$ es un proceso con valores positivo crudamente regenerativo en $T$, y def\'inase $M=\sup\left\{|X\left(t\right)|:t\leq T\right\}$. Si $T$ es no aritm\'etico y $M$ y $MT$ tienen media finita, entonces
\begin{eqnarray*}
lim_{t\rightarrow\infty}\esp\left[X\left(t\right)\right]=\frac{1}{\mu}\int_{\rea_{+}}h\left(s\right)ds,
\end{eqnarray*}
donde $h\left(t\right)=\esp\left[X\left(t\right)\indora\left(T>t\right)\right]$.
\end{Teo}


\begin{Note} Una funci\'on $h:\rea_{+}\rightarrow\rea$ es Directamente Riemann Integrable en los siguientes casos:
\begin{itemize}
\item[a)] $h\left(t\right)\geq0$ es decreciente y Riemann Integrable.
\item[b)] $h$ es continua excepto posiblemente en un conjunto de Lebesgue de medida 0, y $|h\left(t\right)|\leq b\left(t\right)$, donde $b$ es DRI.
\end{itemize}
\end{Note}

\begin{Teo}[Teorema Principal de Renovaci\'on]
Si $F$ es no aritm\'etica y $h\left(t\right)$ es Directamente Riemann Integrable (DRI), entonces

\begin{eqnarray*}
lim_{t\rightarrow\infty}U\star h=\frac{1}{\mu}\int_{\rea_{+}}h\left(s\right)ds.
\end{eqnarray*}
\end{Teo}

\begin{Prop}
Cualquier funci\'on $H\left(t\right)$ acotada en intervalos finitos y que es 0 para $t<0$ puede expresarse como
\begin{eqnarray*}
H\left(t\right)=U\star h\left(t\right)\textrm{,  donde }h\left(t\right)=H\left(t\right)-F\star H\left(t\right)
\end{eqnarray*}
\end{Prop}

\begin{Def}
Un proceso estoc\'astico $X\left(t\right)$ es crudamente regenerativo en un tiempo aleatorio positivo $T$ si
\begin{eqnarray*}
\esp\left[X\left(T+t\right)|T\right]=\esp\left[X\left(t\right)\right]\textrm{, para }t\geq0,\end{eqnarray*}
y con las esperanzas anteriores finitas.
\end{Def}

\begin{Prop}
Sup\'ongase que $X\left(t\right)$ es un proceso crudamente regenerativo en $T$, que tiene distribuci\'on $F$. Si $\esp\left[X\left(t\right)\right]$ es acotado en intervalos finitos, entonces
\begin{eqnarray*}
\esp\left[X\left(t\right)\right]=U\star h\left(t\right)\textrm{,  donde }h\left(t\right)=\esp\left[X\left(t\right)\indora\left(T>t\right)\right].
\end{eqnarray*}
\end{Prop}

\begin{Teo}[Regeneraci\'on Cruda]
Sup\'ongase que $X\left(t\right)$ es un proceso con valores positivo crudamente regenerativo en $T$, y def\'inase $M=\sup\left\{|X\left(t\right)|:t\leq T\right\}$. Si $T$ es no aritm\'etico y $M$ y $MT$ tienen media finita, entonces
\begin{eqnarray*}
lim_{t\rightarrow\infty}\esp\left[X\left(t\right)\right]=\frac{1}{\mu}\int_{\rea_{+}}h\left(s\right)ds,
\end{eqnarray*}
donde $h\left(t\right)=\esp\left[X\left(t\right)\indora\left(T>t\right)\right]$.
\end{Teo}

\begin{Def}
Para el proceso $\left\{\left(N\left(t\right),X\left(t\right)\right):t\geq0\right\}$, sus trayectoria muestrales en el intervalo de tiempo $\left[T_{n-1},T_{n}\right)$ est\'an descritas por
\begin{eqnarray*}
\zeta_{n}=\left(\xi_{n},\left\{X\left(T_{n-1}+t\right):0\leq t<\xi_{n}\right\}\right)
\end{eqnarray*}
Este $\zeta_{n}$ es el $n$-\'esimo segmento del proceso. El proceso es regenerativo sobre los tiempos $T_{n}$ si sus segmentos $\zeta_{n}$ son independientes e id\'enticamennte distribuidos.
\end{Def}


\begin{Note}
Si $\tilde{X}\left(t\right)$ con espacio de estados $\tilde{S}$ es regenerativo sobre $T_{n}$, entonces $X\left(t\right)=f\left(\tilde{X}\left(t\right)\right)$ tambi\'en es regenerativo sobre $T_{n}$, para cualquier funci\'on $f:\tilde{S}\rightarrow S$.
\end{Note}

\begin{Note}
Los procesos regenerativos son crudamente regenerativos, pero no al rev\'es.
\end{Note}


\begin{Note}
Un proceso estoc\'astico a tiempo continuo o discreto es regenerativo si existe un proceso de renovaci\'on  tal que los segmentos del proceso entre tiempos de renovaci\'on sucesivos son i.i.d., es decir, para $\left\{X\left(t\right):t\geq0\right\}$ proceso estoc\'astico a tiempo continuo con espacio de estados $S$, espacio m\'etrico.
\end{Note}

Para $\left\{X\left(t\right):t\geq0\right\}$ Proceso Estoc\'astico a tiempo continuo con estado de espacios $S$, que es un espacio m\'etrico, con trayectorias continuas por la derecha y con l\'imites por la izquierda c.s. Sea $N\left(t\right)$ un proceso de renovaci\'on en $\rea_{+}$ definido en el mismo espacio de probabilidad que $X\left(t\right)$, con tiempos de renovaci\'on $T$ y tiempos de inter-renovaci\'on $\xi_{n}=T_{n}-T_{n-1}$, con misma distribuci\'on $F$ de media finita $\mu$.



\begin{Def}
Para el proceso $\left\{\left(N\left(t\right),X\left(t\right)\right):t\geq0\right\}$, sus trayectoria muestrales en el intervalo de tiempo $\left[T_{n-1},T_{n}\right)$ est\'an descritas por
\begin{eqnarray*}
\zeta_{n}=\left(\xi_{n},\left\{X\left(T_{n-1}+t\right):0\leq t<\xi_{n}\right\}\right)
\end{eqnarray*}
Este $\zeta_{n}$ es el $n$-\'esimo segmento del proceso. El proceso es regenerativo sobre los tiempos $T_{n}$ si sus segmentos $\zeta_{n}$ son independientes e id\'enticamennte distribuidos.
\end{Def}

\begin{Note}
Un proceso regenerativo con media de la longitud de ciclo finita es llamado positivo recurrente.
\end{Note}

\begin{Teo}[Procesos Regenerativos]
Suponga que el proceso
\end{Teo}


\begin{Def}[Renewal Process Trinity]
Para un proceso de renovaci\'on $N\left(t\right)$, los siguientes procesos proveen de informaci\'on sobre los tiempos de renovaci\'on.
\begin{itemize}
\item $A\left(t\right)=t-T_{N\left(t\right)}$, el tiempo de recurrencia hacia atr\'as al tiempo $t$, que es el tiempo desde la \'ultima renovaci\'on para $t$.

\item $B\left(t\right)=T_{N\left(t\right)+1}-t$, el tiempo de recurrencia hacia adelante al tiempo $t$, residual del tiempo de renovaci\'on, que es el tiempo para la pr\'oxima renovaci\'on despu\'es de $t$.

\item $L\left(t\right)=\xi_{N\left(t\right)+1}=A\left(t\right)+B\left(t\right)$, la longitud del intervalo de renovaci\'on que contiene a $t$.
\end{itemize}
\end{Def}

\begin{Note}
El proceso tridimensional $\left(A\left(t\right),B\left(t\right),L\left(t\right)\right)$ es regenerativo sobre $T_{n}$, y por ende cada proceso lo es. Cada proceso $A\left(t\right)$ y $B\left(t\right)$ son procesos de MArkov a tiempo continuo con trayectorias continuas por partes en el espacio de estados $\rea_{+}$. Una expresi\'on conveniente para su distribuci\'on conjunta es, para $0\leq x<t,y\geq0$
\begin{equation}\label{NoRenovacion}
P\left\{A\left(t\right)>x,B\left(t\right)>y\right\}=
P\left\{N\left(t+y\right)-N\left((t-x)\right)=0\right\}
\end{equation}
\end{Note}

\begin{Ejem}[Tiempos de recurrencia Poisson]
Si $N\left(t\right)$ es un proceso Poisson con tasa $\lambda$, entonces de la expresi\'on (\ref{NoRenovacion}) se tiene que

\begin{eqnarray*}
\begin{array}{lc}
P\left\{A\left(t\right)>x,B\left(t\right)>y\right\}=e^{-\lambda\left(x+y\right)},&0\leq x<t,y\geq0,
\end{array}
\end{eqnarray*}
que es la probabilidad Poisson de no renovaciones en un intervalo de longitud $x+y$.

\end{Ejem}

\begin{Note}
Una cadena de Markov erg\'odica tiene la propiedad de ser estacionaria si la distribuci\'on de su estado al tiempo $0$ es su distribuci\'on estacionaria.
\end{Note}


\begin{Def}
Un proceso estoc\'astico a tiempo continuo $\left\{X\left(t\right):t\geq0\right\}$ en un espacio general es estacionario si sus distribuciones finito dimensionales son invariantes bajo cualquier  traslado: para cada $0\leq s_{1}<s_{2}<\cdots<s_{k}$ y $t\geq0$,
\begin{eqnarray*}
\left(X\left(s_{1}+t\right),\ldots,X\left(s_{k}+t\right)\right)=_{d}\left(X\left(s_{1}\right),\ldots,X\left(s_{k}\right)\right).
\end{eqnarray*}
\end{Def}

\begin{Note}
Un proceso de Markov es estacionario si $X\left(t\right)=_{d}X\left(0\right)$, $t\geq0$.
\end{Note}

Considerese el proceso $N\left(t\right)=\sum_{n}\indora\left(\tau_{n}\leq t\right)$ en $\rea_{+}$, con puntos $0<\tau_{1}<\tau_{2}<\cdots$.

\begin{Prop}
Si $N$ es un proceso puntual estacionario y $\esp\left[N\left(1\right)\right]<\infty$, entonces $\esp\left[N\left(t\right)\right]=t\esp\left[N\left(1\right)\right]$, $t\geq0$

\end{Prop}

\begin{Teo}
Los siguientes enunciados son equivalentes
\begin{itemize}
\item[i)] El proceso retardado de renovaci\'on $N$ es estacionario.

\item[ii)] EL proceso de tiempos de recurrencia hacia adelante $B\left(t\right)$ es estacionario.


\item[iii)] $\esp\left[N\left(t\right)\right]=t/\mu$,


\item[iv)] $G\left(t\right)=F_{e}\left(t\right)=\frac{1}{\mu}\int_{0}^{t}\left[1-F\left(s\right)\right]ds$
\end{itemize}
Cuando estos enunciados son ciertos, $P\left\{B\left(t\right)\leq x\right\}=F_{e}\left(x\right)$, para $t,x\geq0$.

\end{Teo}

\begin{Note}
Una consecuencia del teorema anterior es que el Proceso Poisson es el \'unico proceso sin retardo que es estacionario.
\end{Note}

\begin{Coro}
El proceso de renovaci\'on $N\left(t\right)$ sin retardo, y cuyos tiempos de inter renonaci\'on tienen media finita, es estacionario si y s\'olo si es un proceso Poisson.

\end{Coro}

%______________________________________________________________________

%\section{Ejemplos, Notas importantes}
%______________________________________________________________________
%\section*{Ap\'endice A}
%__________________________________________________________________

%________________________________________________________________________
%\subsection*{Procesos Regenerativos}
%________________________________________________________________________



\begin{Note}
Si $\tilde{X}\left(t\right)$ con espacio de estados $\tilde{S}$ es regenerativo sobre $T_{n}$, entonces $X\left(t\right)=f\left(\tilde{X}\left(t\right)\right)$ tambi\'en es regenerativo sobre $T_{n}$, para cualquier funci\'on $f:\tilde{S}\rightarrow S$.
\end{Note}

\begin{Note}
Los procesos regenerativos son crudamente regenerativos, pero no al rev\'es.
\end{Note}
%\subsection*{Procesos Regenerativos: Sigman\cite{Sigman1}}
\begin{Def}[Definici\'on Cl\'asica]
Un proceso estoc\'astico $X=\left\{X\left(t\right):t\geq0\right\}$ es llamado regenerativo is existe una variable aleatoria $R_{1}>0$ tal que
\begin{itemize}
\item[i)] $\left\{X\left(t+R_{1}\right):t\geq0\right\}$ es independiente de $\left\{\left\{X\left(t\right):t<R_{1}\right\},\right\}$
\item[ii)] $\left\{X\left(t+R_{1}\right):t\geq0\right\}$ es estoc\'asticamente equivalente a $\left\{X\left(t\right):t>0\right\}$
\end{itemize}

Llamamos a $R_{1}$ tiempo de regeneraci\'on, y decimos que $X$ se regenera en este punto.
\end{Def}

$\left\{X\left(t+R_{1}\right)\right\}$ es regenerativo con tiempo de regeneraci\'on $R_{2}$, independiente de $R_{1}$ pero con la misma distribuci\'on que $R_{1}$. Procediendo de esta manera se obtiene una secuencia de variables aleatorias independientes e id\'enticamente distribuidas $\left\{R_{n}\right\}$ llamados longitudes de ciclo. Si definimos a $Z_{k}\equiv R_{1}+R_{2}+\cdots+R_{k}$, se tiene un proceso de renovaci\'on llamado proceso de renovaci\'on encajado para $X$.




\begin{Def}
Para $x$ fijo y para cada $t\geq0$, sea $I_{x}\left(t\right)=1$ si $X\left(t\right)\leq x$,  $I_{x}\left(t\right)=0$ en caso contrario, y def\'inanse los tiempos promedio
\begin{eqnarray*}
\overline{X}&=&lim_{t\rightarrow\infty}\frac{1}{t}\int_{0}^{\infty}X\left(u\right)du\\
\prob\left(X_{\infty}\leq x\right)&=&lim_{t\rightarrow\infty}\frac{1}{t}\int_{0}^{\infty}I_{x}\left(u\right)du,
\end{eqnarray*}
cuando estos l\'imites existan.
\end{Def}

Como consecuencia del teorema de Renovaci\'on-Recompensa, se tiene que el primer l\'imite  existe y es igual a la constante
\begin{eqnarray*}
\overline{X}&=&\frac{\esp\left[\int_{0}^{R_{1}}X\left(t\right)dt\right]}{\esp\left[R_{1}\right]},
\end{eqnarray*}
suponiendo que ambas esperanzas son finitas.

\begin{Note}
\begin{itemize}
\item[a)] Si el proceso regenerativo $X$ es positivo recurrente y tiene trayectorias muestrales no negativas, entonces la ecuaci\'on anterior es v\'alida.
\item[b)] Si $X$ es positivo recurrente regenerativo, podemos construir una \'unica versi\'on estacionaria de este proceso, $X_{e}=\left\{X_{e}\left(t\right)\right\}$, donde $X_{e}$ es un proceso estoc\'astico regenerativo y estrictamente estacionario, con distribuci\'on marginal distribuida como $X_{\infty}$
\end{itemize}
\end{Note}

Para $\left\{X\left(t\right):t\geq0\right\}$ Proceso Estoc\'astico a tiempo continuo con estado de espacios $S$, que es un espacio m\'etrico, con trayectorias continuas por la derecha y con l\'imites por la izquierda c.s. Sea $N\left(t\right)$ un proceso de renovaci\'on en $\rea_{+}$ definido en el mismo espacio de probabilidad que $X\left(t\right)$, con tiempos de renovaci\'on $T$ y tiempos de inter-renovaci\'on $\xi_{n}=T_{n}-T_{n-1}$, con misma distribuci\'on $F$ de media finita $\mu$.


\begin{Def}
Para el proceso $\left\{\left(N\left(t\right),X\left(t\right)\right):t\geq0\right\}$, sus trayectoria muestrales en el intervalo de tiempo $\left[T_{n-1},T_{n}\right)$ est\'an descritas por
\begin{eqnarray*}
\zeta_{n}=\left(\xi_{n},\left\{X\left(T_{n-1}+t\right):0\leq t<\xi_{n}\right\}\right)
\end{eqnarray*}
Este $\zeta_{n}$ es el $n$-\'esimo segmento del proceso. El proceso es regenerativo sobre los tiempos $T_{n}$ si sus segmentos $\zeta_{n}$ son independientes e id\'enticamennte distribuidos.
\end{Def}


\begin{Note}
Si $\tilde{X}\left(t\right)$ con espacio de estados $\tilde{S}$ es regenerativo sobre $T_{n}$, entonces $X\left(t\right)=f\left(\tilde{X}\left(t\right)\right)$ tambi\'en es regenerativo sobre $T_{n}$, para cualquier funci\'on $f:\tilde{S}\rightarrow S$.
\end{Note}

\begin{Note}
Los procesos regenerativos son crudamente regenerativos, pero no al rev\'es.
\end{Note}

\begin{Def}[Definici\'on Cl\'asica]
Un proceso estoc\'astico $X=\left\{X\left(t\right):t\geq0\right\}$ es llamado regenerativo is existe una variable aleatoria $R_{1}>0$ tal que
\begin{itemize}
\item[i)] $\left\{X\left(t+R_{1}\right):t\geq0\right\}$ es independiente de $\left\{\left\{X\left(t\right):t<R_{1}\right\},\right\}$
\item[ii)] $\left\{X\left(t+R_{1}\right):t\geq0\right\}$ es estoc\'asticamente equivalente a $\left\{X\left(t\right):t>0\right\}$
\end{itemize}

Llamamos a $R_{1}$ tiempo de regeneraci\'on, y decimos que $X$ se regenera en este punto.
\end{Def}

$\left\{X\left(t+R_{1}\right)\right\}$ es regenerativo con tiempo de regeneraci\'on $R_{2}$, independiente de $R_{1}$ pero con la misma distribuci\'on que $R_{1}$. Procediendo de esta manera se obtiene una secuencia de variables aleatorias independientes e id\'enticamente distribuidas $\left\{R_{n}\right\}$ llamados longitudes de ciclo. Si definimos a $Z_{k}\equiv R_{1}+R_{2}+\cdots+R_{k}$, se tiene un proceso de renovaci\'on llamado proceso de renovaci\'on encajado para $X$.

\begin{Note}
Un proceso regenerativo con media de la longitud de ciclo finita es llamado positivo recurrente.
\end{Note}


\begin{Def}
Para $x$ fijo y para cada $t\geq0$, sea $I_{x}\left(t\right)=1$ si $X\left(t\right)\leq x$,  $I_{x}\left(t\right)=0$ en caso contrario, y def\'inanse los tiempos promedio
\begin{eqnarray*}
\overline{X}&=&lim_{t\rightarrow\infty}\frac{1}{t}\int_{0}^{\infty}X\left(u\right)du\\
\prob\left(X_{\infty}\leq x\right)&=&lim_{t\rightarrow\infty}\frac{1}{t}\int_{0}^{\infty}I_{x}\left(u\right)du,
\end{eqnarray*}
cuando estos l\'imites existan.
\end{Def}

Como consecuencia del teorema de Renovaci\'on-Recompensa, se tiene que el primer l\'imite  existe y es igual a la constante
\begin{eqnarray*}
\overline{X}&=&\frac{\esp\left[\int_{0}^{R_{1}}X\left(t\right)dt\right]}{\esp\left[R_{1}\right]},
\end{eqnarray*}
suponiendo que ambas esperanzas son finitas.

\begin{Note}
\begin{itemize}
\item[a)] Si el proceso regenerativo $X$ es positivo recurrente y tiene trayectorias muestrales no negativas, entonces la ecuaci\'on anterior es v\'alida.
\item[b)] Si $X$ es positivo recurrente regenerativo, podemos construir una \'unica versi\'on estacionaria de este proceso, $X_{e}=\left\{X_{e}\left(t\right)\right\}$, donde $X_{e}$ es un proceso estoc\'astico regenerativo y estrictamente estacionario, con distribuci\'on marginal distribuida como $X_{\infty}$
\end{itemize}
\end{Note}

%__________________________________________________________________________________________
%\subsection{Procesos Regenerativos Estacionarios - Stidham \cite{Stidham}}
%__________________________________________________________________________________________


Un proceso estoc\'astico a tiempo continuo $\left\{V\left(t\right),t\geq0\right\}$ es un proceso regenerativo si existe una sucesi\'on de variables aleatorias independientes e id\'enticamente distribuidas $\left\{X_{1},X_{2},\ldots\right\}$, sucesi\'on de renovaci\'on, tal que para cualquier conjunto de Borel $A$, 

\begin{eqnarray*}
\prob\left\{V\left(t\right)\in A|X_{1}+X_{2}+\cdots+X_{R\left(t\right)}=s,\left\{V\left(\tau\right),\tau<s\right\}\right\}=\prob\left\{V\left(t-s\right)\in A|X_{1}>t-s\right\},
\end{eqnarray*}
para todo $0\leq s\leq t$, donde $R\left(t\right)=\max\left\{X_{1}+X_{2}+\cdots+X_{j}\leq t\right\}=$n\'umero de renovaciones ({\emph{puntos de regeneraci\'on}}) que ocurren en $\left[0,t\right]$. El intervalo $\left[0,X_{1}\right)$ es llamado {\emph{primer ciclo de regeneraci\'on}} de $\left\{V\left(t \right),t\geq0\right\}$, $\left[X_{1},X_{1}+X_{2}\right)$ el {\emph{segundo ciclo de regeneraci\'on}}, y as\'i sucesivamente.

Sea $X=X_{1}$ y sea $F$ la funci\'on de distrbuci\'on de $X$


\begin{Def}
Se define el proceso estacionario, $\left\{V^{*}\left(t\right),t\geq0\right\}$, para $\left\{V\left(t\right),t\geq0\right\}$ por

\begin{eqnarray*}
\prob\left\{V\left(t\right)\in A\right\}=\frac{1}{\esp\left[X\right]}\int_{0}^{\infty}\prob\left\{V\left(t+x\right)\in A|X>x\right\}\left(1-F\left(x\right)\right)dx,
\end{eqnarray*} 
para todo $t\geq0$ y todo conjunto de Borel $A$.
\end{Def}

\begin{Def}
Una distribuci\'on se dice que es {\emph{aritm\'etica}} si todos sus puntos de incremento son m\'ultiplos de la forma $0,\lambda, 2\lambda,\ldots$ para alguna $\lambda>0$ entera.
\end{Def}


\begin{Def}
Una modificaci\'on medible de un proceso $\left\{V\left(t\right),t\geq0\right\}$, es una versi\'on de este, $\left\{V\left(t,w\right)\right\}$ conjuntamente medible para $t\geq0$ y para $w\in S$, $S$ espacio de estados para $\left\{V\left(t\right),t\geq0\right\}$.
\end{Def}

\begin{Teo}
Sea $\left\{V\left(t\right),t\geq\right\}$ un proceso regenerativo no negativo con modificaci\'on medible. Sea $\esp\left[X\right]<\infty$. Entonces el proceso estacionario dado por la ecuaci\'on anterior est\'a bien definido y tiene funci\'on de distribuci\'on independiente de $t$, adem\'as
\begin{itemize}
\item[i)] \begin{eqnarray*}
\esp\left[V^{*}\left(0\right)\right]&=&\frac{\esp\left[\int_{0}^{X}V\left(s\right)ds\right]}{\esp\left[X\right]}\end{eqnarray*}
\item[ii)] Si $\esp\left[V^{*}\left(0\right)\right]<\infty$, equivalentemente, si $\esp\left[\int_{0}^{X}V\left(s\right)ds\right]<\infty$,entonces
\begin{eqnarray*}
\frac{\int_{0}^{t}V\left(s\right)ds}{t}\rightarrow\frac{\esp\left[\int_{0}^{X}V\left(s\right)ds\right]}{\esp\left[X\right]}
\end{eqnarray*}
con probabilidad 1 y en media, cuando $t\rightarrow\infty$.
\end{itemize}
\end{Teo}
%
%___________________________________________________________________________________________
%\vspace{5.5cm}
%\chapter{Cadenas de Markov estacionarias}
%\vspace{-1.0cm}


%__________________________________________________________________________________________
%\subsection{Procesos Regenerativos Estacionarios - Stidham \cite{Stidham}}
%__________________________________________________________________________________________


Un proceso estoc\'astico a tiempo continuo $\left\{V\left(t\right),t\geq0\right\}$ es un proceso regenerativo si existe una sucesi\'on de variables aleatorias independientes e id\'enticamente distribuidas $\left\{X_{1},X_{2},\ldots\right\}$, sucesi\'on de renovaci\'on, tal que para cualquier conjunto de Borel $A$, 

\begin{eqnarray*}
\prob\left\{V\left(t\right)\in A|X_{1}+X_{2}+\cdots+X_{R\left(t\right)}=s,\left\{V\left(\tau\right),\tau<s\right\}\right\}=\prob\left\{V\left(t-s\right)\in A|X_{1}>t-s\right\},
\end{eqnarray*}
para todo $0\leq s\leq t$, donde $R\left(t\right)=\max\left\{X_{1}+X_{2}+\cdots+X_{j}\leq t\right\}=$n\'umero de renovaciones ({\emph{puntos de regeneraci\'on}}) que ocurren en $\left[0,t\right]$. El intervalo $\left[0,X_{1}\right)$ es llamado {\emph{primer ciclo de regeneraci\'on}} de $\left\{V\left(t \right),t\geq0\right\}$, $\left[X_{1},X_{1}+X_{2}\right)$ el {\emph{segundo ciclo de regeneraci\'on}}, y as\'i sucesivamente.

Sea $X=X_{1}$ y sea $F$ la funci\'on de distrbuci\'on de $X$


\begin{Def}
Se define el proceso estacionario, $\left\{V^{*}\left(t\right),t\geq0\right\}$, para $\left\{V\left(t\right),t\geq0\right\}$ por

\begin{eqnarray*}
\prob\left\{V\left(t\right)\in A\right\}=\frac{1}{\esp\left[X\right]}\int_{0}^{\infty}\prob\left\{V\left(t+x\right)\in A|X>x\right\}\left(1-F\left(x\right)\right)dx,
\end{eqnarray*} 
para todo $t\geq0$ y todo conjunto de Borel $A$.
\end{Def}

\begin{Def}
Una distribuci\'on se dice que es {\emph{aritm\'etica}} si todos sus puntos de incremento son m\'ultiplos de la forma $0,\lambda, 2\lambda,\ldots$ para alguna $\lambda>0$ entera.
\end{Def}


\begin{Def}
Una modificaci\'on medible de un proceso $\left\{V\left(t\right),t\geq0\right\}$, es una versi\'on de este, $\left\{V\left(t,w\right)\right\}$ conjuntamente medible para $t\geq0$ y para $w\in S$, $S$ espacio de estados para $\left\{V\left(t\right),t\geq0\right\}$.
\end{Def}

\begin{Teo}
Sea $\left\{V\left(t\right),t\geq\right\}$ un proceso regenerativo no negativo con modificaci\'on medible. Sea $\esp\left[X\right]<\infty$. Entonces el proceso estacionario dado por la ecuaci\'on anterior est\'a bien definido y tiene funci\'on de distribuci\'on independiente de $t$, adem\'as
\begin{itemize}
\item[i)] \begin{eqnarray*}
\esp\left[V^{*}\left(0\right)\right]&=&\frac{\esp\left[\int_{0}^{X}V\left(s\right)ds\right]}{\esp\left[X\right]}\end{eqnarray*}
\item[ii)] Si $\esp\left[V^{*}\left(0\right)\right]<\infty$, equivalentemente, si $\esp\left[\int_{0}^{X}V\left(s\right)ds\right]<\infty$,entonces
\begin{eqnarray*}
\frac{\int_{0}^{t}V\left(s\right)ds}{t}\rightarrow\frac{\esp\left[\int_{0}^{X}V\left(s\right)ds\right]}{\esp\left[X\right]}
\end{eqnarray*}
con probabilidad 1 y en media, cuando $t\rightarrow\infty$.
\end{itemize}
\end{Teo}

Sea la funci\'on generadora de momentos para $L_{i}$, el n\'umero de usuarios en la cola $Q_{i}\left(z\right)$ en cualquier momento, est\'a dada por el tiempo promedio de $z^{L_{i}\left(t\right)}$ sobre el ciclo regenerativo definido anteriormente. Entonces 



Es decir, es posible determinar las longitudes de las colas a cualquier tiempo $t$. Entonces, determinando el primer momento es posible ver que


\begin{Def}
El tiempo de Ciclo $C_{i}$ es el periodo de tiempo que comienza cuando la cola $i$ es visitada por primera vez en un ciclo, y termina cuando es visitado nuevamente en el pr\'oximo ciclo. La duraci\'on del mismo est\'a dada por $\tau_{i}\left(m+1\right)-\tau_{i}\left(m\right)$, o equivalentemente $\overline{\tau}_{i}\left(m+1\right)-\overline{\tau}_{i}\left(m\right)$ bajo condiciones de estabilidad.
\end{Def}


\begin{Def}
El tiempo de intervisita $I_{i}$ es el periodo de tiempo que comienza cuando se ha completado el servicio en un ciclo y termina cuando es visitada nuevamente en el pr\'oximo ciclo. Su  duraci\'on del mismo est\'a dada por $\tau_{i}\left(m+1\right)-\overline{\tau}_{i}\left(m\right)$.
\end{Def}

La duraci\'on del tiempo de intervisita es $\tau_{i}\left(m+1\right)-\overline{\tau}\left(m\right)$. Dado que el n\'umero de usuarios presentes en $Q_{i}$ al tiempo $t=\tau_{i}\left(m+1\right)$ es igual al n\'umero de arribos durante el intervalo de tiempo $\left[\overline{\tau}\left(m\right),\tau_{i}\left(m+1\right)\right]$ se tiene que


\begin{eqnarray*}
\esp\left[z_{i}^{L_{i}\left(\tau_{i}\left(m+1\right)\right)}\right]=\esp\left[\left\{P_{i}\left(z_{i}\right)\right\}^{\tau_{i}\left(m+1\right)-\overline{\tau}\left(m\right)}\right]
\end{eqnarray*}

entonces, si $I_{i}\left(z\right)=\esp\left[z^{\tau_{i}\left(m+1\right)-\overline{\tau}\left(m\right)}\right]$
se tiene que $F_{i}\left(z\right)=I_{i}\left[P_{i}\left(z\right)\right]$
para $i=1,2$.

Conforme a la definici\'on dada al principio del cap\'itulo, definici\'on (\ref{Def.Tn}), sean $T_{1},T_{2},\ldots$ los puntos donde las longitudes de las colas de la red de sistemas de visitas c\'iclicas son cero simult\'aneamente, cuando la cola $Q_{j}$ es visitada por el servidor para dar servicio, es decir, $L_{1}\left(T_{i}\right)=0,L_{2}\left(T_{i}\right)=0,\hat{L}_{1}\left(T_{i}\right)=0$ y $\hat{L}_{2}\left(T_{i}\right)=0$, a estos puntos se les denominar\'a puntos regenerativos. Entonces, 

\begin{Def}
Al intervalo de tiempo entre dos puntos regenerativos se le llamar\'a ciclo regenerativo.
\end{Def}

\begin{Def}
Para $T_{i}$ se define, $M_{i}$, el n\'umero de ciclos de visita a la cola $Q_{l}$, durante el ciclo regenerativo, es decir, $M_{i}$ es un proceso de renovaci\'on.
\end{Def}

\begin{Def}
Para cada uno de los $M_{i}$'s, se definen a su vez la duraci\'on de cada uno de estos ciclos de visita en el ciclo regenerativo, $C_{i}^{(m)}$, para $m=1,2,\ldots,M_{i}$, que a su vez, tambi\'en es n proceso de renovaci\'on.
\end{Def}

\footnote{In Stidham and  Heyman \cite{Stidham} shows that is sufficient for the regenerative process to be stationary that the mean regenerative cycle time is finite: $\esp\left[\sum_{m=1}^{M_{i}}C_{i}^{(m)}\right]<\infty$, 


 como cada $C_{i}^{(m)}$ contiene intervalos de r\'eplica positivos, se tiene que $\esp\left[M_{i}\right]<\infty$, adem\'as, como $M_{i}>0$, se tiene que la condici\'on anterior es equivalente a tener que $\esp\left[C_{i}\right]<\infty$,
por lo tanto una condici\'on suficiente para la existencia del proceso regenerativo est\'a dada por $\sum_{k=1}^{N}\mu_{k}<1.$}

Para $\left\{X\left(t\right):t\geq0\right\}$ Proceso Estoc\'astico a tiempo continuo con estado de espacios $S$, que es un espacio m\'etrico, con trayectorias continuas por la derecha y con l\'imites por la izquierda c.s. Sea $N\left(t\right)$ un proceso de renovaci\'on en $\rea_{+}$ definido en el mismo espacio de probabilidad que $X\left(t\right)$, con tiempos de renovaci\'on $T$ y tiempos de inter-renovaci\'on $\xi_{n}=T_{n}-T_{n-1}$, con misma distribuci\'on $F$ de media finita $\mu$.

\begin{Def}
Un elemento aleatorio en un espacio medible $\left(E,\mathcal{E}\right)$ en un espacio de probabilidad $\left(\Omega,\mathcal{F},\prob\right)$ a $\left(E,\mathcal{E}\right)$, es decir,
para $A\in \mathcal{E}$,  se tiene que $\left\{Y\in A\right\}\in\mathcal{F}$, donde $\left\{Y\in A\right\}:=\left\{w\in\Omega:Y\left(w\right)\in A\right\}=:Y^{-1}A$.
\end{Def}

\begin{Note}
Tambi\'en se dice que $Y$ est\'a soportado por el espacio de probabilidad $\left(\Omega,\mathcal{F},\prob\right)$ y que $Y$ es un mapeo medible de $\Omega$ en $E$, es decir, es $\mathcal{F}/\mathcal{E}$ medible.
\end{Note}

\begin{Def}
Para cada $i\in \mathbb{I}$ sea $P_{i}$ una medida de probabilidad en un espacio medible $\left(E_{i},\mathcal{E}_{i}\right)$. Se define el espacio producto
$\otimes_{i\in\mathbb{I}}\left(E_{i},\mathcal{E}_{i}\right):=\left(\prod_{i\in\mathbb{I}}E_{i},\otimes_{i\in\mathbb{I}}\mathcal{E}_{i}\right)$, donde $\prod_{i\in\mathbb{I}}E_{i}$ es el producto cartesiano de los $E_{i}$'s, y $\otimes_{i\in\mathbb{I}}\mathcal{E}_{i}$ es la $\sigma$-\'algebra producto, es decir, es la $\sigma$-\'algebra m\'as peque\~na en $\prod_{i\in\mathbb{I}}E_{i}$ que hace al $i$-\'esimo mapeo proyecci\'on en $E_{i}$ medible para toda $i\in\mathbb{I}$ es la $\sigma$-\'algebra inducida por los mapeos proyecci\'on. $$\otimes_{i\in\mathbb{I}}\mathcal{E}_{i}:=\sigma\left\{\left\{y:y_{i}\in A\right\}:i\in\mathbb{I}\textrm{ y }A\in\mathcal{E}_{i}\right\}.$$
\end{Def}

\begin{Def}
Un espacio de probabilidad $\left(\tilde{\Omega},\tilde{\mathcal{F}},\tilde{\prob}\right)$ es una extensi\'on de otro espacio de probabilidad $\left(\Omega,\mathcal{F},\prob\right)$ si $\left(\tilde{\Omega},\tilde{\mathcal{F}},\tilde{\prob}\right)$ soporta un elemento aleatorio $\xi\in\left(\Omega,\mathcal{F}\right)$ que tienen a $\prob$ como distribuci\'on.
\end{Def}

\begin{Teo}
Sea $\mathbb{I}$ un conjunto de \'indices arbitrario. Para cada $i\in\mathbb{I}$ sea $P_{i}$ una medida de probabilidad en un espacio medible $\left(E_{i},\mathcal{E}_{i}\right)$. Entonces existe una \'unica medida de probabilidad $\otimes_{i\in\mathbb{I}}P_{i}$ en $\otimes_{i\in\mathbb{I}}\left(E_{i},\mathcal{E}_{i}\right)$ tal que 

\begin{eqnarray*}
\otimes_{i\in\mathbb{I}}P_{i}\left(y\in\prod_{i\in\mathbb{I}}E_{i}:y_{i}\in A_{i_{1}},\ldots,y_{n}\in A_{i_{n}}\right)=P_{i_{1}}\left(A_{i_{n}}\right)\cdots P_{i_{n}}\left(A_{i_{n}}\right)
\end{eqnarray*}
para todos los enteros $n>0$, toda $i_{1},\ldots,i_{n}\in\mathbb{I}$ y todo $A_{i_{1}}\in\mathcal{E}_{i_{1}},\ldots,A_{i_{n}}\in\mathcal{E}_{i_{n}}$
\end{Teo}

La medida $\otimes_{i\in\mathbb{I}}P_{i}$ es llamada la medida producto y $\otimes_{i\in\mathbb{I}}\left(E_{i},\mathcal{E}_{i},P_{i}\right):=\left(\prod_{i\in\mathbb{I}},E_{i},\otimes_{i\in\mathbb{I}}\mathcal{E}_{i},\otimes_{i\in\mathbb{I}}P_{i}\right)$, es llamado espacio de probabilidad producto.


\begin{Def}
Un espacio medible $\left(E,\mathcal{E}\right)$ es \textit{Polaco} si existe una m\'etrica en $E$ tal que $E$ es completo, es decir cada sucesi\'on de Cauchy converge a un l\'imite en $E$, y \textit{separable}, $E$ tienen un subconjunto denso numerable, y tal que $\mathcal{E}$ es generado por conjuntos abiertos.
\end{Def}


\begin{Def}
Dos espacios medibles $\left(E,\mathcal{E}\right)$ y $\left(G,\mathcal{G}\right)$ son Borel equivalentes \textit{isomorfos} si existe una biyecci\'on $f:E\rightarrow G$ tal que $f$ es $\mathcal{E}/\mathcal{G}$ medible y su inversa $f^{-1}$ es $\mathcal{G}/\mathcal{E}$ medible. La biyecci\'on es una equivalencia de Borel.
\end{Def}

\begin{Def}
Un espacio medible  $\left(E,\mathcal{E}\right)$ es un \textit{espacio est\'andar} si es Borel equivalente a $\left(G,\mathcal{G}\right)$, donde $G$ es un subconjunto de Borel de $\left[0,1\right]$ y $\mathcal{G}$ son los subconjuntos de Borel de $G$.
\end{Def}

\begin{Note}
Cualquier espacio Polaco es un espacio est\'andar.
\end{Note}


\begin{Def}
Un proceso estoc\'astico con conjunto de \'indices $\mathbb{I}$ y espacio de estados $\left(E,\mathcal{E}\right)$ es una familia $Z=\left(\mathbb{Z}_{s}\right)_{s\in\mathbb{I}}$ donde $\mathbb{Z}_{s}$ son elementos aleatorios definidos en un espacio de probabilidad com\'un $\left(\Omega,\mathcal{F},\prob\right)$ y todos toman valores en $\left(E,\mathcal{E}\right)$.
\end{Def}

\begin{Def}
Un proceso estoc\'astico \textit{one-sided contiuous time} (\textbf{PEOSCT}) es un proceso estoc\'astico con conjunto de \'indices $\mathbb{I}=\left[0,\infty\right)$.
\end{Def}


Sea $\left(E^{\mathbb{I}},\mathcal{E}^{\mathbb{I}}\right)$ denota el espacio producto $\left(E^{\mathbb{I}},\mathcal{E}^{\mathbb{I}}\right):=\otimes_{s\in\mathbb{I}}\left(E,\mathcal{E}\right)$. Vamos a considerar $\mathbb{Z}$ como un mapeo aleatorio, es decir, como un elemento aleatorio en $\left(E^{\mathbb{I}},\mathcal{E}^{\mathbb{I}}\right)$ definido por $Z\left(w\right)=\left(Z_{s}\left(w\right)\right)_{s\in\mathbb{I}}$ y $w\in\Omega$.

\begin{Note}
La distribuci\'on de un proceso estoc\'astico $Z$ es la distribuci\'on de $Z$ como un elemento aleatorio en $\left(E^{\mathbb{I}},\mathcal{E}^{\mathbb{I}}\right)$. La distribuci\'on de $Z$ esta determinada de manera \'unica por las distribuciones finito dimensionales.
\end{Note}

\begin{Note}
En particular cuando $Z$ toma valores reales, es decir, $\left(E,\mathcal{E}\right)=\left(\mathbb{R},\mathcal{B}\right)$ las distribuciones finito dimensionales est\'an determinadas por las funciones de distribuci\'on finito dimensionales

\begin{eqnarray}
\prob\left(Z_{t_{1}}\leq x_{1},\ldots,Z_{t_{n}}\leq x_{n}\right),x_{1},\ldots,x_{n}\in\mathbb{R},t_{1},\ldots,t_{n}\in\mathbb{I},n\geq1.
\end{eqnarray}
\end{Note}

\begin{Note}
Para espacios polacos $\left(E,\mathcal{E}\right)$ el Teorema de Consistencia de Kolmogorov asegura que dada una colecci\'on de distribuciones finito dimensionales consistentes, siempre existe un proceso estoc\'astico que posee tales distribuciones finito dimensionales.
\end{Note}


\begin{Def}
Las trayectorias de $Z$ son las realizaciones $Z\left(w\right)$ para $w\in\Omega$ del mapeo aleatorio $Z$.
\end{Def}

\begin{Note}
Algunas restricciones se imponen sobre las trayectorias, por ejemplo que sean continuas por la derecha, o continuas por la derecha con l\'imites por la izquierda, o de manera m\'as general, se pedir\'a que caigan en alg\'un subconjunto $H$ de $E^{\mathbb{I}}$. En este caso es natural considerar a $Z$ como un elemento aleatorio que no est\'a en $\left(E^{\mathbb{I}},\mathcal{E}^{\mathbb{I}}\right)$ sino en $\left(H,\mathcal{H}\right)$, donde $\mathcal{H}$ es la $\sigma$-\'algebra generada por los mapeos proyecci\'on que toman a $z\in H$ a $z_{t}\in E$ para $t\in\mathbb{I}$. A $\mathcal{H}$ se le conoce como la traza de $H$ en $E^{\mathbb{I}}$, es decir,
\begin{eqnarray}
\mathcal{H}:=E^{\mathbb{I}}\cap H:=\left\{A\cap H:A\in E^{\mathbb{I}}\right\}.
\end{eqnarray}
\end{Note}


\begin{Note}
$Z$ tiene trayectorias con valores en $H$ y cada $Z_{t}$ es un mapeo medible de $\left(\Omega,\mathcal{F}\right)$ a $\left(H,\mathcal{H}\right)$. Cuando se considera un espacio de trayectorias en particular $H$, al espacio $\left(H,\mathcal{H}\right)$ se le llama el espacio de trayectorias de $Z$.
\end{Note}

\begin{Note}
La distribuci\'on del proceso estoc\'astico $Z$ con espacio de trayectorias $\left(H,\mathcal{H}\right)$ es la distribuci\'on de $Z$ como  un elemento aleatorio en $\left(H,\mathcal{H}\right)$. La distribuci\'on, nuevemente, est\'a determinada de manera \'unica por las distribuciones finito dimensionales.
\end{Note}


\begin{Def}
Sea $Z$ un PEOSCT  con espacio de estados $\left(E,\mathcal{E}\right)$ y sea $T$ un tiempo aleatorio en $\left[0,\infty\right)$. Por $Z_{T}$ se entiende el mapeo con valores en $E$ definido en $\Omega$ en la manera obvia:
\begin{eqnarray*}
Z_{T}\left(w\right):=Z_{T\left(w\right)}\left(w\right). w\in\Omega.
\end{eqnarray*}
\end{Def}

\begin{Def}
Un PEOSCT $Z$ es conjuntamente medible (\textbf{CM}) si el mapeo que toma $\left(w,t\right)\in\Omega\times\left[0,\infty\right)$ a $Z_{t}\left(w\right)\in E$ es $\mathcal{F}\otimes\mathcal{B}\left[0,\infty\right)/\mathcal{E}$ medible.
\end{Def}

\begin{Note}
Un PEOSCT-CM implica que el proceso es medible, dado que $Z_{T}$ es una composici\'on  de dos mapeos continuos: el primero que toma $w$ en $\left(w,T\left(w\right)\right)$ es $\mathcal{F}/\mathcal{F}\otimes\mathcal{B}\left[0,\infty\right)$ medible, mientras que el segundo toma $\left(w,T\left(w\right)\right)$ en $Z_{T\left(w\right)}\left(w\right)$ es $\mathcal{F}\otimes\mathcal{B}\left[0,\infty\right)/\mathcal{E}$ medible.
\end{Note}


\begin{Def}
Un PEOSCT con espacio de estados $\left(H,\mathcal{H}\right)$ es can\'onicamente conjuntamente medible (\textbf{CCM}) si el mapeo $\left(z,t\right)\in H\times\left[0,\infty\right)$ en $Z_{t}\in E$ es $\mathcal{H}\otimes\mathcal{B}\left[0,\infty\right)/\mathcal{E}$ medible.
\end{Def}

\begin{Note}
Un PEOSCTCCM implica que el proceso es CM, dado que un PECCM $Z$ es un mapeo de $\Omega\times\left[0,\infty\right)$ a $E$, es la composici\'on de dos mapeos medibles: el primero, toma $\left(w,t\right)$ en $\left(Z\left(w\right),t\right)$ es $\mathcal{F}\otimes\mathcal{B}\left[0,\infty\right)/\mathcal{H}\otimes\mathcal{B}\left[0,\infty\right)$ medible, y el segundo que toma $\left(Z\left(w\right),t\right)$  en $Z_{t}\left(w\right)$ es $\mathcal{H}\otimes\mathcal{B}\left[0,\infty\right)/\mathcal{E}$ medible. Por tanto CCM es una condici\'on m\'as fuerte que CM.
\end{Note}

\begin{Def}
Un conjunto de trayectorias $H$ de un PEOSCT $Z$, es internamente shift-invariante (\textbf{ISI}) si 
\begin{eqnarray*}
\left\{\left(z_{t+s}\right)_{s\in\left[0,\infty\right)}:z\in H\right\}=H\textrm{, }t\in\left[0,\infty\right).
\end{eqnarray*}
\end{Def}


\begin{Def}
Dado un PEOSCTISI, se define el mapeo-shift $\theta_{t}$, $t\in\left[0,\infty\right)$, de $H$ a $H$ por 
\begin{eqnarray*}
\theta_{t}z=\left(z_{t+s}\right)_{s\in\left[0,\infty\right)}\textrm{, }z\in H.
\end{eqnarray*}
\end{Def}

\begin{Def}
Se dice que un proceso $Z$ es shift-medible (\textbf{SM}) si $Z$ tiene un conjunto de trayectorias $H$ que es ISI y adem\'as el mapeo que toma $\left(z,t\right)\in H\times\left[0,\infty\right)$ en $\theta_{t}z\in H$ es $\mathcal{H}\otimes\mathcal{B}\left[0,\infty\right)/\mathcal{H}$ medible.
\end{Def}

\begin{Note}
Un proceso estoc\'astico con conjunto de trayectorias $H$ ISI es shift-medible si y s\'olo si es CCM
\end{Note}

\begin{Note}
\begin{itemize}
\item Dado el espacio polaco $\left(E,\mathcal{E}\right)$ se tiene el  conjunto de trayectorias $D_{E}\left[0,\infty\right)$ que es ISI, entonces cumpe con ser CCM.

\item Si $G$ es abierto, podemos cubrirlo por bolas abiertas cuay cerradura este contenida en $G$, y como $G$ es segundo numerable como subespacio de $E$, lo podemos cubrir por una cantidad numerable de bolas abiertas.

\end{itemize}
\end{Note}


\begin{Note}
Los procesos estoc\'asticos $Z$ a tiempo discreto con espacio de estados polaco, tambi\'en tiene un espacio de trayectorias polaco y por tanto tiene distribuciones condicionales regulares.
\end{Note}

\begin{Teo}
El producto numerable de espacios polacos es polaco.
\end{Teo}


\begin{Def}
Sea $\left(\Omega,\mathcal{F},\prob\right)$ espacio de probabilidad que soporta al proceso $Z=\left(Z_{s}\right)_{s\in\left[0,\infty\right)}$ y $S=\left(S_{k}\right)_{0}^{\infty}$ donde $Z$ es un PEOSCTM con espacio de estados $\left(E,\mathcal{E}\right)$  y espacio de trayectorias $\left(H,\mathcal{H}\right)$  y adem\'as $S$ es una sucesi\'on de tiempos aleatorios one-sided que satisfacen la condici\'on $0\leq S_{0}<S_{1}<\cdots\rightarrow\infty$. Considerando $S$ como un mapeo medible de $\left(\Omega,\mathcal{F}\right)$ al espacio sucesi\'on $\left(L,\mathcal{L}\right)$, donde 
\begin{eqnarray*}
L=\left\{\left(s_{k}\right)_{0}^{\infty}\in\left[0,\infty\right)^{\left\{0,1,\ldots\right\}}:s_{0}<s_{1}<\cdots\rightarrow\infty\right\},
\end{eqnarray*}
donde $\mathcal{L}$ son los subconjuntos de Borel de $L$, es decir, $\mathcal{L}=L\cap\mathcal{B}^{\left\{0,1,\ldots\right\}}$.

As\'i el par $\left(Z,S\right)$ es un mapeo medible de  $\left(\Omega,\mathcal{F}\right)$ en $\left(H\times L,\mathcal{H}\otimes\mathcal{L}\right)$. El par $\mathcal{H}\otimes\mathcal{L}^{+}$ denotar\'a la clase de todas las funciones medibles de $\left(H\times L,\mathcal{H}\otimes\mathcal{L}\right)$ en $\left(\left[0,\infty\right),\mathcal{B}\left[0,\infty\right)\right)$.
\end{Def}


\begin{Def}
Sea $\theta_{t}$ el mapeo-shift conjunto de $H\times L$ en $H\times L$ dado por
\begin{eqnarray*}
\theta_{t}\left(z,\left(s_{k}\right)_{0}^{\infty}\right)=\theta_{t}\left(z,\left(s_{n_{t-}+k}-t\right)_{0}^{\infty}\right)
\end{eqnarray*}
donde 
$n_{t-}=inf\left\{n\geq1:s_{n}\geq t\right\}$.
\end{Def}

\begin{Note}
Con la finalidad de poder realizar los shift's sin complicaciones de medibilidad, se supondr\'a que $Z$ es shit-medible, es decir, el conjunto de trayectorias $H$ es invariante bajo shifts del tiempo y el mapeo que toma $\left(z,t\right)\in H\times\left[0,\infty\right)$ en $z_{t}\in E$ es $\mathcal{H}\otimes\mathcal{B}\left[0,\infty\right)/\mathcal{E}$ medible.
\end{Note}

\begin{Def}
Dado un proceso \textbf{PEOSSM} (Proceso Estoc\'astico One Side Shift Medible) $Z$, se dice regenerativo cl\'asico con tiempos de regeneraci\'on $S$ si 

\begin{eqnarray*}
\theta_{S_{n}}\left(Z,S\right)=\left(Z^{0},S^{0}\right),n\geq0
\end{eqnarray*}
y adem\'as $\theta_{S_{n}}\left(Z,S\right)$ es independiente de $\left(\left(Z_{s}\right)s\in\left[0,S_{n}\right),S_{0},\ldots,S_{n}\right)$
Si lo anterior se cumple, al par $\left(Z,S\right)$ se le llama regenerativo cl\'asico.
\end{Def}

\begin{Note}
Si el par $\left(Z,S\right)$ es regenerativo cl\'asico, entonces las longitudes de los ciclos $X_{1},X_{2},\ldots,$ son i.i.d. e independientes de la longitud del retraso $S_{0}$, es decir, $S$ es un proceso de renovaci\'on. Las longitudes de los ciclos tambi\'en son llamados tiempos de inter-regeneraci\'on y tiempos de ocurrencia.

\end{Note}

\begin{Teo}
Sup\'ongase que el par $\left(Z,S\right)$ es regenerativo cl\'asico con $\esp\left[X_{1}\right]<\infty$. Entonces $\left(Z^{*},S^{*}\right)$ en el teorema 2.1 es una versi\'on estacionaria de $\left(Z,S\right)$. Adem\'as, si $X_{1}$ es lattice con span $d$, entonces $\left(Z^{**},S^{**}\right)$ en el teorema 2.2 es una versi\'on periodicamente estacionaria de $\left(Z,S\right)$ con periodo $d$.

\end{Teo}

\begin{Def}
Una variable aleatoria $X_{1}$ es \textit{spread out} si existe una $n\geq1$ y una  funci\'on $f\in\mathcal{B}^{+}$ tal que $\int_{\rea}f\left(x\right)dx>0$ con $X_{2},X_{3},\ldots,X_{n}$ copias i.i.d  de $X_{1}$, $$\prob\left(X_{1}+\cdots+X_{n}\in B\right)\geq\int_{B}f\left(x\right)dx$$ para $B\in\mathcal{B}$.

\end{Def}



\begin{Def}
Dado un proceso estoc\'astico $Z$ se le llama \textit{wide-sense regenerative} (\textbf{WSR}) con tiempos de regeneraci\'on $S$ si $\theta_{S_{n}}\left(Z,S\right)=\left(Z^{0},S^{0}\right)$ para $n\geq0$ en distribuci\'on y $\theta_{S_{n}}\left(Z,S\right)$ es independiente de $\left(S_{0},S_{1},\ldots,S_{n}\right)$ para $n\geq0$.
Se dice que el par $\left(Z,S\right)$ es WSR si lo anterior se cumple.
\end{Def}


\begin{Note}
\begin{itemize}
\item El proceso de trayectorias $\left(\theta_{s}Z\right)_{s\in\left[0,\infty\right)}$ es WSR con tiempos de regeneraci\'on $S$ pero no es regenerativo cl\'asico.

\item Si $Z$ es cualquier proceso estacionario y $S$ es un proceso de renovaci\'on que es independiente de $Z$, entonces $\left(Z,S\right)$ es WSR pero en general no es regenerativo cl\'asico

\end{itemize}

\end{Note}


\begin{Note}
Para cualquier proceso estoc\'astico $Z$, el proceso de trayectorias $\left(\theta_{s}Z\right)_{s\in\left[0,\infty\right)}$ es siempre un proceso de Markov.
\end{Note}



\begin{Teo}
Supongase que el par $\left(Z,S\right)$ es WSR con $\esp\left[X_{1}\right]<\infty$. Entonces $\left(Z^{*},S^{*}\right)$ en el teorema 2.1 es una versi\'on estacionaria de 
$\left(Z,S\right)$.
\end{Teo}


\begin{Teo}
Supongase que $\left(Z,S\right)$ es cycle-stationary con $\esp\left[X_{1}\right]<\infty$. Sea $U$ distribuida uniformemente en $\left[0,1\right)$ e independiente de $\left(Z^{0},S^{0}\right)$ y sea $\prob^{*}$ la medida de probabilidad en $\left(\Omega,\prob\right)$ definida por $$d\prob^{*}=\frac{X_{1}}{\esp\left[X_{1}\right]}d\prob$$. Sea $\left(Z^{*},S^{*}\right)$ con distribuci\'on $\prob^{*}\left(\theta_{UX_{1}}\left(Z^{0},S^{0}\right)\in\cdot\right)$. Entonces $\left(Z^{}*,S^{*}\right)$ es estacionario,
\begin{eqnarray*}
\esp\left[f\left(Z^{*},S^{*}\right)\right]=\esp\left[\int_{0}^{X_{1}}f\left(\theta_{s}\left(Z^{0},S^{0}\right)\right)ds\right]/\esp\left[X_{1}\right]
\end{eqnarray*}
$f\in\mathcal{H}\otimes\mathcal{L}^{+}$, and $S_{0}^{*}$ es continuo con funci\'on distribuci\'on $G_{\infty}$ definida por $$G_{\infty}\left(x\right):=\frac{\esp\left[X_{1}\right]\wedge x}{\esp\left[X_{1}\right]}$$ para $x\geq0$ y densidad $\prob\left[X_{1}>x\right]/\esp\left[X_{1}\right]$, con $x\geq0$.

\end{Teo}


\begin{Teo}
Sea $Z$ un Proceso Estoc\'astico un lado shift-medible \textit{one-sided shift-measurable stochastic process}, (PEOSSM),
y $S_{0}$ y $S_{1}$ tiempos aleatorios tales que $0\leq S_{0}<S_{1}$ y
\begin{equation}
\theta_{S_{1}}Z=\theta_{S_{0}}Z\textrm{ en distribuci\'on}.
\end{equation}

Entonces el espacio de probabilidad subyacente $\left(\Omega,\mathcal{F},\prob\right)$ puede extenderse para soportar una sucesi\'on de tiempos aleatorios $S$ tales que

\begin{eqnarray}
\theta_{S_{n}}\left(Z,S\right)=\left(Z^{0},S^{0}\right),n\geq0,\textrm{ en distribuci\'on},\\
\left(Z,S_{0},S_{1}\right)\textrm{ depende de }\left(X_{2},X_{3},\ldots\right)\textrm{ solamente a traves de }\theta_{S_{1}}Z.
\end{eqnarray}
\end{Teo}





%_________________________________________________________________________
%
%\subsection{Una vez que se tiene estabilidad}
%_________________________________________________________________________
%

Also the intervisit time $I_{i}$ is defined as the period beginning at the time of its service completion in a cycle and ending at the time when it is polled in the next cycle; its duration is given by $\tau_{i}\left(m+1\right)-\overline{\tau}_{i}\left(m\right)$.

So we the following are still true 

\begin{eqnarray}
\begin{array}{ll}
\esp\left[L_{i}\right]=\mu_{i}\esp\left[I_{i}\right], &
\esp\left[C_{i}\right]=\frac{f_{i}\left(i\right)}{\mu_{i}\left(1-\mu_{i}\right)},\\
\esp\left[S_{i}\right]=\mu_{i}\esp\left[C_{i}\right],&
\esp\left[I_{i}\right]=\left(1-\mu_{i}\right)\esp\left[C_{i}\right],\\
Var\left[L_{i}\right]= \mu_{i}^{2}Var\left[I_{i}\right]+\sigma^{2}\esp\left[I_{i}\right],& 
Var\left[C_{i}\right]=\frac{Var\left[L_{i}^{*}\right]}{\mu_{i}^{2}\left(1-\mu_{i}\right)^{2}},\\
Var\left[S_{i}\right]= \frac{Var\left[L_{i}^{*}\right]}{\left(1-\mu_{i}\right)^{2}}+\frac{\sigma^{2}\esp\left[L_{i}^{*}\right]}{\left(1-\mu_{i}\right)^{3}},&
Var\left[I_{i}\right]= \frac{Var\left[L_{i}^{*}\right]}{\mu_{i}^{2}}-\frac{\sigma_{i}^{2}}{\mu_{i}^{2}}f_{i}\left(i\right).
\end{array}
\end{eqnarray}
\begin{Def}
El tiempo de Ciclo $C_{i}$ es el periodo de tiempo que comienza cuando la cola $i$ es visitada por primera vez en un ciclo, y termina cuando es visitado nuevamente en el pr\'oximo ciclo. La duraci\'on del mismo est\'a dada por $\tau_{i}\left(m+1\right)-\tau_{i}\left(m\right)$, o equivalentemente $\overline{\tau}_{i}\left(m+1\right)-\overline{\tau}_{i}\left(m\right)$ bajo condiciones de estabilidad.
\end{Def}


\begin{Def}
El tiempo de intervisita $I_{i}$ es el periodo de tiempo que comienza cuando se ha completado el servicio en un ciclo y termina cuando es visitada nuevamente en el pr\'oximo ciclo. Su  duraci\'on del mismo est\'a dada por $\tau_{i}\left(m+1\right)-\overline{\tau}_{i}\left(m\right)$.
\end{Def}

La duraci\'on del tiempo de intervisita es $\tau_{i}\left(m+1\right)-\overline{\tau}\left(m\right)$. Dado que el n\'umero de usuarios presentes en $Q_{i}$ al tiempo $t=\tau_{i}\left(m+1\right)$ es igual al n\'umero de arribos durante el intervalo de tiempo $\left[\overline{\tau}\left(m\right),\tau_{i}\left(m+1\right)\right]$ se tiene que


\begin{eqnarray*}
\esp\left[z_{i}^{L_{i}\left(\tau_{i}\left(m+1\right)\right)}\right]=\esp\left[\left\{P_{i}\left(z_{i}\right)\right\}^{\tau_{i}\left(m+1\right)-\overline{\tau}\left(m\right)}\right]
\end{eqnarray*}

entonces, si $I_{i}\left(z\right)=\esp\left[z^{\tau_{i}\left(m+1\right)-\overline{\tau}\left(m\right)}\right]$
se tiene que $F_{i}\left(z\right)=I_{i}\left[P_{i}\left(z\right)\right]$
para $i=1,2$.

Conforme a la definici\'on dada al principio del cap\'itulo, definici\'on (\ref{Def.Tn}), sean $T_{1},T_{2},\ldots$ los puntos donde las longitudes de las colas de la red de sistemas de visitas c\'iclicas son cero simult\'aneamente, cuando la cola $Q_{j}$ es visitada por el servidor para dar servicio, es decir, $L_{1}\left(T_{i}\right)=0,L_{2}\left(T_{i}\right)=0,\hat{L}_{1}\left(T_{i}\right)=0$ y $\hat{L}_{2}\left(T_{i}\right)=0$, a estos puntos se les denominar\'a puntos regenerativos. Entonces, 

\begin{Def}
Al intervalo de tiempo entre dos puntos regenerativos se le llamar\'a ciclo regenerativo.
\end{Def}

\begin{Def}
Para $T_{i}$ se define, $M_{i}$, el n\'umero de ciclos de visita a la cola $Q_{l}$, durante el ciclo regenerativo, es decir, $M_{i}$ es un proceso de renovaci\'on.
\end{Def}

\begin{Def}
Para cada uno de los $M_{i}$'s, se definen a su vez la duraci\'on de cada uno de estos ciclos de visita en el ciclo regenerativo, $C_{i}^{(m)}$, para $m=1,2,\ldots,M_{i}$, que a su vez, tambi\'en es n proceso de renovaci\'on.
\end{Def}


Sea la funci\'on generadora de momentos para $L_{i}$, el n\'umero de usuarios en la cola $Q_{i}\left(z\right)$ en cualquier momento, est\'a dada por el tiempo promedio de $z^{L_{i}\left(t\right)}$ sobre el ciclo regenerativo definido anteriormente:

\begin{eqnarray*}
Q_{i}\left(z\right)&=&\esp\left[z^{L_{i}\left(t\right)}\right]=\frac{\esp\left[\sum_{m=1}^{M_{i}}\sum_{t=\tau_{i}\left(m\right)}^{\tau_{i}\left(m+1\right)-1}z^{L_{i}\left(t\right)}\right]}{\esp\left[\sum_{m=1}^{M_{i}}\tau_{i}\left(m+1\right)-\tau_{i}\left(m\right)\right]}
\end{eqnarray*}

$M_{i}$ es un tiempo de paro en el proceso regenerativo con $\esp\left[M_{i}\right]<\infty$\footnote{En Stidham\cite{Stidham} y Heyman  se muestra que una condici\'on suficiente para que el proceso regenerativo 
estacionario sea un procesoo estacionario es que el valor esperado del tiempo del ciclo regenerativo sea finito, es decir: $\esp\left[\sum_{m=1}^{M_{i}}C_{i}^{(m)}\right]<\infty$, como cada $C_{i}^{(m)}$ contiene intervalos de r\'eplica positivos, se tiene que $\esp\left[M_{i}\right]<\infty$, adem\'as, como $M_{i}>0$, se tiene que la condici\'on anterior es equivalente a tener que $\esp\left[C_{i}\right]<\infty$,
por lo tanto una condici\'on suficiente para la existencia del proceso regenerativo est\'a dada por $\sum_{k=1}^{N}\mu_{k}<1.$}, se sigue del lema de Wald que:


\begin{eqnarray*}
\esp\left[\sum_{m=1}^{M_{i}}\sum_{t=\tau_{i}\left(m\right)}^{\tau_{i}\left(m+1\right)-1}z^{L_{i}\left(t\right)}\right]&=&\esp\left[M_{i}\right]\esp\left[\sum_{t=\tau_{i}\left(m\right)}^{\tau_{i}\left(m+1\right)-1}z^{L_{i}\left(t\right)}\right]\\
\esp\left[\sum_{m=1}^{M_{i}}\tau_{i}\left(m+1\right)-\tau_{i}\left(m\right)\right]&=&\esp\left[M_{i}\right]\esp\left[\tau_{i}\left(m+1\right)-\tau_{i}\left(m\right)\right]
\end{eqnarray*}

por tanto se tiene que


\begin{eqnarray*}
Q_{i}\left(z\right)&=&\frac{\esp\left[\sum_{t=\tau_{i}\left(m\right)}^{\tau_{i}\left(m+1\right)-1}z^{L_{i}\left(t\right)}\right]}{\esp\left[\tau_{i}\left(m+1\right)-\tau_{i}\left(m\right)\right]}
\end{eqnarray*}

observar que el denominador es simplemente la duraci\'on promedio del tiempo del ciclo.


Haciendo las siguientes sustituciones en la ecuaci\'on (\ref{Corolario2}): $n\rightarrow t-\tau_{i}\left(m\right)$, $T \rightarrow \overline{\tau}_{i}\left(m\right)-\tau_{i}\left(m\right)$, $L_{n}\rightarrow L_{i}\left(t\right)$ y $F\left(z\right)=\esp\left[z^{L_{0}}\right]\rightarrow F_{i}\left(z\right)=\esp\left[z^{L_{i}\tau_{i}\left(m\right)}\right]$, se puede ver que

\begin{eqnarray}\label{Eq.Arribos.Primera}
\esp\left[\sum_{n=0}^{T-1}z^{L_{n}}\right]=
\esp\left[\sum_{t=\tau_{i}\left(m\right)}^{\overline{\tau}_{i}\left(m\right)-1}z^{L_{i}\left(t\right)}\right]
=z\frac{F_{i}\left(z\right)-1}{z-P_{i}\left(z\right)}
\end{eqnarray}

Por otra parte durante el tiempo de intervisita para la cola $i$, $L_{i}\left(t\right)$ solamente se incrementa de manera que el incremento por intervalo de tiempo est\'a dado por la funci\'on generadora de probabilidades de $P_{i}\left(z\right)$, por tanto la suma sobre el tiempo de intervisita puede evaluarse como:

\begin{eqnarray*}
\esp\left[\sum_{t=\tau_{i}\left(m\right)}^{\tau_{i}\left(m+1\right)-1}z^{L_{i}\left(t\right)}\right]&=&\esp\left[\sum_{t=\tau_{i}\left(m\right)}^{\tau_{i}\left(m+1\right)-1}\left\{P_{i}\left(z\right)\right\}^{t-\overline{\tau}_{i}\left(m\right)}\right]=\frac{1-\esp\left[\left\{P_{i}\left(z\right)\right\}^{\tau_{i}\left(m+1\right)-\overline{\tau}_{i}\left(m\right)}\right]}{1-P_{i}\left(z\right)}\\
&=&\frac{1-I_{i}\left[P_{i}\left(z\right)\right]}{1-P_{i}\left(z\right)}
\end{eqnarray*}
por tanto

\begin{eqnarray*}
\esp\left[\sum_{t=\tau_{i}\left(m\right)}^{\tau_{i}\left(m+1\right)-1}z^{L_{i}\left(t\right)}\right]&=&
\frac{1-F_{i}\left(z\right)}{1-P_{i}\left(z\right)}
\end{eqnarray*}

Por lo tanto

\begin{eqnarray*}
Q_{i}\left(z\right)&=&\frac{\esp\left[\sum_{t=\tau_{i}\left(m\right)}^{\tau_{i}\left(m+1\right)-1}z^{L_{i}\left(t\right)}\right]}{\esp\left[\tau_{i}\left(m+1\right)-\tau_{i}\left(m\right)\right]}
=\frac{1}{\esp\left[\tau_{i}\left(m+1\right)-\tau_{i}\left(m\right)\right]}
\esp\left[\sum_{t=\tau_{i}\left(m\right)}^{\tau_{i}\left(m+1\right)-1}z^{L_{i}\left(t\right)}\right]\\
&=&\frac{1}{\esp\left[\tau_{i}\left(m+1\right)-\tau_{i}\left(m\right)\right]}
\esp\left[\sum_{t=\tau_{i}\left(m\right)}^{\overline{\tau}_{i}\left(m\right)-1}z^{L_{i}\left(t\right)}
+\sum_{t=\overline{\tau}_{i}\left(m\right)}^{\tau_{i}\left(m+1\right)-1}z^{L_{i}\left(t\right)}\right]\\
&=&\frac{1}{\esp\left[\tau_{i}\left(m+1\right)-\tau_{i}\left(m\right)\right]}\left\{
\esp\left[\sum_{t=\tau_{i}\left(m\right)}^{\overline{\tau}_{i}\left(m\right)-1}z^{L_{i}\left(t\right)}\right]
+\esp\left[\sum_{t=\overline{\tau}_{i}\left(m\right)}^{\tau_{i}\left(m+1\right)-1}z^{L_{i}\left(t\right)}\right]\right\}\\
&=&\frac{1}{\esp\left[\tau_{i}\left(m+1\right)-\tau_{i}\left(m\right)\right]}\left\{
z\frac{F_{i}\left(z\right)-1}{z-P_{i}\left(z\right)}+\frac{1-F_{i}\left(z\right)}{1-P_{i}\left(z\right)}
\right\}\\
&=&\frac{1}{\esp\left[C_{i}\right]}\cdot\frac{1-F_{i}\left(z\right)}{P_{i}\left(z\right)-z}\cdot\frac{\left(1-z\right)P_{i}\left(z\right)}{1-P_{i}\left(z\right)}
\end{eqnarray*}

es decir

\begin{equation}
Q_{i}\left(z\right)=\frac{1}{\esp\left[C_{i}\right]}\cdot\frac{1-F_{i}\left(z\right)}{P_{i}\left(z\right)-z}\cdot\frac{\left(1-z\right)P_{i}\left(z\right)}{1-P_{i}\left(z\right)}
\end{equation}


Si hacemos:

\begin{eqnarray}
S\left(z\right)&=&1-F\left(z\right)\\
T\left(z\right)&=&z-P\left(z\right)\\
U\left(z\right)&=&1-P\left(z\right)
\end{eqnarray}
entonces 

\begin{eqnarray}
\esp\left[C_{i}\right]Q\left(z\right)=\frac{\left(z-1\right)S\left(z\right)P\left(z\right)}{T\left(z\right)U\left(z\right)}
\end{eqnarray}

A saber, si $a_{k}=P\left\{L\left(t\right)=k\right\}$
\begin{eqnarray*}
S\left(z\right)=1-F\left(z\right)=1-\sum_{k=0}^{+\infty}a_{k}z^{k}
\end{eqnarray*}
entonces

%\begin{eqnarray}
%\begin{array}{ll}
%S^{'}\left(z\right)=-\sum_{k=1}^{+\infty}ka_{k}z^{k-1},& %S^{(1)}\left(1\right)=-\sum_{k=1}^{+\infty}ka_{k}=-\esp\left[L\left(t\right)\right],\\
%S^{''}\left(z\right)=-\sum_{k=2}^{+\infty}k(k-1)a_{k}z^{k-2},& S^{(2)}\left(1\right)=-\sum_{k=2}^{+\infty}k(k-1)a_{k}=\esp\left[L\left(L-1\right)\right],\\
%S^{'''}\left(z\right)=-\sum_{k=3}^{+\infty}k(k-1)(k-2)a_{k}z^{k-3},&
%S^{(3)}\left(1\right)=-\sum_{k=3}^{+\infty}k(k-1)(k-2)a_{k}\\
%&=-\esp\left[L\left(L-1\right)\left(L-2\right)\right]\\
%&=-\esp\left[L^{3}\right]+3-\esp\left[L^{2}\right]-2-\esp\left[L\right];
%\end{array}
%\end{eqnarray}

$S^{'}\left(z\right)=-\sum_{k=1}^{+\infty}ka_{k}z^{k-1}$, por tanto $S^{(1)}\left(1\right)=-\sum_{k=1}^{+\infty}ka_{k}=-\esp\left[L\left(t\right)\right]$,
luego $S^{''}\left(z\right)=-\sum_{k=2}^{+\infty}k(k-1)a_{k}z^{k-2}$ y $S^{(2)}\left(1\right)=-\sum_{k=2}^{+\infty}k(k-1)a_{k}=\esp\left[L\left(L-1\right)\right]$;
de la misma manera $S^{'''}\left(z\right)=-\sum_{k=3}^{+\infty}k(k-1)(k-2)a_{k}z^{k-3}$ y $S^{(3)}\left(1\right)=-\sum_{k=3}^{+\infty}k(k-1)(k-2)a_{k}=-\esp\left[L\left(L-1\right)\left(L-2\right)\right]
=-\esp\left[L^{3}\right]+3-\esp\left[L^{2}\right]-2-\esp\left[L\right]$. 

Es decir

\begin{eqnarray*}
S^{(1)}\left(1\right)&=&-\esp\left[L\left(t\right)\right],\\ S^{(2)}\left(1\right)&=&-\esp\left[L\left(L-1\right)\right]
=-\esp\left[L^{2}\right]+\esp\left[L\right],\\
S^{(3)}\left(1\right)&=&-\esp\left[L\left(L-1\right)\left(L-2\right)\right]
=-\esp\left[L^{3}\right]+3\esp\left[L^{2}\right]-2\esp\left[L\right].
\end{eqnarray*}


Expandiendo alrededor de $z=1$

\begin{eqnarray*}
S\left(z\right)&=&S\left(1\right)+\frac{S^{'}\left(1\right)}{1!}\left(z-1\right)+\frac{S^{''}\left(1\right)}{2!}\left(z-1\right)^{2}+\frac{S^{'''}\left(1\right)}{3!}\left(z-1\right)^{3}+\ldots+\\
&=&\left(z-1\right)\left\{S^{'}\left(1\right)+\frac{S^{''}\left(1\right)}{2!}\left(z-1\right)+\frac{S^{'''}\left(1\right)}{3!}\left(z-1\right)^{2}+\ldots+\right\}\\
&=&\left(z-1\right)R_{1}\left(z\right)
\end{eqnarray*}
con $R_{1}\left(z\right)\neq0$, pues

\begin{eqnarray}\label{Eq.R1}
R_{1}\left(z\right)=-\esp\left[L\right]
\end{eqnarray}
entonces

\begin{eqnarray}
R_{1}\left(z\right)&=&S^{'}\left(1\right)+\frac{S^{''}\left(1\right)}{2!}\left(z-1\right)+\frac{S^{'''}\left(1\right)}{3!}\left(z-1\right)^{2}+\frac{S^{iv}\left(1\right)}{4!}\left(z-1\right)^{3}+\ldots+
\end{eqnarray}
Calculando las derivadas y evaluando en $z=1$

\begin{eqnarray}
R_{1}\left(1\right)&=&S^{(1)}\left(1\right)=-\esp\left[L\right]\\
R_{1}^{(1)}\left(1\right)&=&\frac{1}{2}S^{(2)}\left(1\right)=-\frac{1}{2}\esp\left[L^{2}\right]+\frac{1}{2}\esp\left[L\right]\\
R_{1}^{(2)}\left(1\right)&=&\frac{2}{3!}S^{(3)}\left(1\right)
=-\frac{1}{3}\esp\left[L^{3}\right]+\esp\left[L^{2}\right]-\frac{2}{3}\esp\left[L\right]
\end{eqnarray}

De manera an\'aloga se puede ver que para $T\left(z\right)=z-P\left(z\right)$ se puede encontrar una expanci\'on alrededor de $z=1$

Expandiendo alrededor de $z=1$

\begin{eqnarray*}
T\left(z\right)&=&T\left(1\right)+\frac{T^{'}\left(1\right)}{1!}\left(z-1\right)+\frac{T^{''}\left(1\right)}{2!}\left(z-1\right)^{2}+\frac{T^{'''}\left(1\right)}{3!}\left(z-1\right)^{3}+\ldots+\\
&=&\left(z-1\right)\left\{T^{'}\left(1\right)+\frac{T^{''}\left(1\right)}{2!}\left(z-1\right)+\frac{T^{'''}\left(1\right)}{3!}\left(z-1\right)^{2}+\ldots+\right\}\\
&=&\left(z-1\right)R_{2}\left(z\right)
\end{eqnarray*}

donde 
\begin{eqnarray*}
T^{(1)}\left(1\right)&=&-\esp\left[X\left(t\right)\right]=-\mu,\\ T^{(2)}\left(1\right)&=&-\esp\left[X\left(X-1\right)\right]
=-\esp\left[X^{2}\right]+\esp\left[X\right]=-\esp\left[X^{2}\right]+\mu,\\
T^{(3)}\left(1\right)&=&-\esp\left[X\left(X-1\right)\left(X-2\right)\right]
=-\esp\left[X^{3}\right]+3\esp\left[X^{2}\right]-2\esp\left[X\right]\\
&=&-\esp\left[X^{3}\right]+3\esp\left[X^{2}\right]-2\mu.
\end{eqnarray*}

Por lo tanto $R_{2}\left(1\right)\neq0$, pues

\begin{eqnarray}\label{Eq.R2}
R_{2}\left(1\right)=1-\esp\left[X\right]=1-\mu
\end{eqnarray}
entonces

\begin{eqnarray}
R_{2}\left(z\right)&=&T^{'}\left(1\right)+\frac{T^{''}\left(1\right)}{2!}\left(z-1\right)+\frac{T^{'''}\left(1\right)}{3!}\left(z-1\right)^{2}+\frac{T^{(iv)}\left(1\right)}{4!}\left(z-1\right)^{3}+\ldots+
\end{eqnarray}
Calculando las derivadas y evaluando en $z=1$

\begin{eqnarray}
R_{2}\left(1\right)&=&T^{(1)}\left(1\right)=1-\mu\\
R_{2}^{(1)}\left(1\right)&=&\frac{1}{2}T^{(2)}\left(1\right)=-\frac{1}{2}\esp\left[X^{2}\right]+\frac{1}{2}\mu\\
R_{2}^{(2)}\left(1\right)&=&\frac{2}{3!}T^{(3)}\left(1\right)
=-\frac{1}{3}\esp\left[X^{3}\right]+\esp\left[X^{2}\right]-\frac{2}{3}\mu
\end{eqnarray}

Finalmente para de manera an\'aloga se puede ver que para $U\left(z\right)=1-P\left(z\right)$ se puede encontrar una expanci\'on alrededor de $z=1$

\begin{eqnarray*}
U\left(z\right)&=&U\left(1\right)+\frac{U^{'}\left(1\right)}{1!}\left(z-1\right)+\frac{U^{''}\left(1\right)}{2!}\left(z-1\right)^{2}+\frac{U^{'''}\left(1\right)}{3!}\left(z-1\right)^{3}+\ldots+\\
&=&\left(z-1\right)\left\{U^{'}\left(1\right)+\frac{U^{''}\left(1\right)}{2!}\left(z-1\right)+\frac{U^{'''}\left(1\right)}{3!}\left(z-1\right)^{2}+\ldots+\right\}\\
&=&\left(z-1\right)R_{3}\left(z\right)
\end{eqnarray*}

donde 
\begin{eqnarray*}
U^{(1)}\left(1\right)&=&-\esp\left[X\left(t\right)\right]=-\mu,\\ U^{(2)}\left(1\right)&=&-\esp\left[X\left(X-1\right)\right]
=-\esp\left[X^{2}\right]+\esp\left[X\right]=-\esp\left[X^{2}\right]+\mu,\\
U^{(3)}\left(1\right)&=&-\esp\left[X\left(X-1\right)\left(X-2\right)\right]
=-\esp\left[X^{3}\right]+3\esp\left[X^{2}\right]-2\esp\left[X\right]\\
&=&-\esp\left[X^{3}\right]+3\esp\left[X^{2}\right]-2\mu.
\end{eqnarray*}

Por lo tanto $R_{3}\left(1\right)\neq0$, pues

\begin{eqnarray}\label{Eq.R2}
R_{3}\left(1\right)=-\esp\left[X\right]=-\mu
\end{eqnarray}
entonces

\begin{eqnarray}
R_{3}\left(z\right)&=&U^{'}\left(1\right)+\frac{U^{''}\left(1\right)}{2!}\left(z-1\right)+\frac{U^{'''}\left(1\right)}{3!}\left(z-1\right)^{2}+\frac{U^{(iv)}\left(1\right)}{4!}\left(z-1\right)^{3}+\ldots+
\end{eqnarray}

Calculando las derivadas y evaluando en $z=1$

\begin{eqnarray}
R_{3}\left(1\right)&=&U^{(1)}\left(1\right)=-\mu\\
R_{3}^{(1)}\left(1\right)&=&\frac{1}{2}U^{(2)}\left(1\right)=-\frac{1}{2}\esp\left[X^{2}\right]+\frac{1}{2}\mu\\
R_{3}^{(2)}\left(1\right)&=&\frac{2}{3!}U^{(3)}\left(1\right)
=-\frac{1}{3}\esp\left[X^{3}\right]+\esp\left[X^{2}\right]-\frac{2}{3}\mu
\end{eqnarray}

Por lo tanto

\begin{eqnarray}
\esp\left[C_{i}\right]Q\left(z\right)&=&\frac{\left(z-1\right)\left(z-1\right)R_{1}\left(z\right)P\left(z\right)}{\left(z-1\right)R_{2}\left(z\right)\left(z-1\right)R_{3}\left(z\right)}
=\frac{R_{1}\left(z\right)P\left(z\right)}{R_{2}\left(z\right)R_{3}\left(z\right)}\equiv\frac{R_{1}P}{R_{2}R_{3}}
\end{eqnarray}

Entonces

\begin{eqnarray}\label{Eq.Primer.Derivada.Q}
\left[\frac{R_{1}\left(z\right)P\left(z\right)}{R_{2}\left(z\right)R_{3}\left(z\right)}\right]^{'}&=&\frac{PR_{2}R_{3}R_{1}^{'}
+R_{1}R_{2}R_{3}P^{'}-R_{3}R_{1}PR_{2}-R_{2}R_{1}PR_{3}^{'}}{\left(R_{2}R_{3}\right)^{2}}
\end{eqnarray}
Evaluando en $z=1$
\begin{eqnarray*}
&=&\frac{R_{2}(1)R_{3}(1)R_{1}^{(1)}(1)+R_{1}(1)R_{2}(1)R_{3}(1)P^{'}(1)-R_{3}(1)R_{1}(1)R_{2}(1)^{(1)}-R_{2}(1)R_{1}(1)R_{3}^{'}(1)}{\left(R_{2}(1)R_{3}(1)\right)^{2}}\\
&=&\frac{1}{\left(1-\mu\right)^{2}\mu^{2}}\left\{\left(-\frac{1}{2}\esp L^{2}+\frac{1}{2}\esp L\right)\left(1-\mu\right)\left(-\mu\right)+\left(-\esp L\right)\left(1-\mu\right)\left(-\mu\right)\mu\right.\\
&&\left.-\left(-\frac{1}{2}\esp X^{2}+\frac{1}{2}\mu\right)\left(-\mu\right)\left(-\esp L\right)-\left(1-\mu\right)\left(-\esp L\right)\left(-\frac{1}{2}\esp X^{2}+\frac{1}{2}\mu\right)\right\}\\
&=&\frac{1}{\left(1-\mu\right)^{2}\mu^{2}}\left\{\left(-\frac{1}{2}\esp L^{2}+\frac{1}{2}\esp L\right)\left(\mu^{2}-\mu\right)
+\left(\mu^{2}-\mu^{3}\right)\esp L\right.\\
&&\left.-\mu\esp L\left(-\frac{1}{2}\esp X^{2}+\frac{1}{2}\mu\right)
+\left(\esp L-\mu\esp L\right)\left(-\frac{1}{2}\esp X^{2}+\frac{1}{2}\mu\right)\right\}\\
&=&\frac{1}{\left(1-\mu\right)^{2}\mu^{2}}\left\{-\frac{1}{2}\mu^{2}\esp L^{2}
+\frac{1}{2}\mu\esp L^{2}
+\frac{1}{2}\mu^{2}\esp L
-\mu^{3}\esp L
+\mu\esp L\esp X^{2}
-\frac{1}{2}\esp L\esp X^{2}\right\}\\
&=&\frac{1}{\left(1-\mu\right)^{2}\mu^{2}}\left\{
\frac{1}{2}\mu\esp L^{2}\left(1-\mu\right)
+\esp L\left(\frac{1}{2}-\mu\right)\left(\mu^{2}-\esp X^{2}\right)\right\}\\
&=&\frac{1}{2\mu\left(1-\mu\right)}\esp L^{2}-\frac{\frac{1}{2}-\mu}{\left(1-\mu\right)^{2}\mu^{2}}\sigma^{2}\esp L
\end{eqnarray*}

por lo tanto (para Takagi)

\begin{eqnarray*}
Q^{(1)}=\frac{1}{\esp C}\left\{\frac{1}{2\mu\left(1-\mu\right)}\esp L^{2}-\frac{\frac{1}{2}-\mu}{\left(1-\mu\right)^{2}\mu^{2}}\sigma^{2}\esp L\right\}
\end{eqnarray*}
donde 

\begin{eqnarray*}
\esp C = \frac{\esp L}{\mu\left(1-\mu\right)}
\end{eqnarray*}
entonces

\begin{eqnarray*}
Q^{(1)}&=&\frac{1}{2}\frac{\esp L^{2}}{\esp L}-\frac{\frac{1}{2}-\mu}{\left(1-\mu\right)\mu}\sigma^{2}
=\frac{\esp L^{2}}{2\esp L}-\frac{\sigma^{2}}{2}\left\{\frac{2\mu-1}{\left(1-\mu\right)\mu}\right\}\\
&=&\frac{\esp L^{2}}{2\esp L}+\frac{\sigma^{2}}{2}\left\{\frac{1}{1-\mu}+\frac{1}{\mu}\right\}
\end{eqnarray*}

Mientras que para nosotros

\begin{eqnarray*}
Q^{(1)}=\frac{1}{\mu\left(1-\mu\right)}\frac{\esp L^{2}}{2\esp C}
-\sigma^{2}\frac{\esp L}{2\esp C}\cdot\frac{1-2\mu}{\left(1-\mu\right)^{2}\mu^{2}}
\end{eqnarray*}

Retomando la ecuaci\'on (\ref{Eq.Primer.Derivada.Q})

\begin{eqnarray*}
\left[\frac{R_{1}\left(z\right)P\left(z\right)}{R_{2}\left(z\right)R_{3}\left(z\right)}\right]^{'}&=&\frac{PR_{2}R_{3}R_{1}^{'}
+R_{1}R_{2}R_{3}P^{'}-R_{3}R_{1}PR_{2}-R_{2}R_{1}PR_{3}^{'}}{\left(R_{2}R_{3}\right)^{2}}
=\frac{F\left(z\right)}{G\left(z\right)}
\end{eqnarray*}

donde 

\begin{eqnarray*}
F\left(z\right)&=&PR_{2}R_{3}R_{1}^{'}
+R_{1}R_{2}R_{3}P^{'}-R_{3}R_{1}PR_{2}^{'}-R_{2}R_{1}PR_{3}^{'}\\
G\left(z\right)&=&R_{2}^{2}R_{3}^{2}\\
G^{2}\left(z\right)&=&R_{2}^{4}R_{3}^{4}=\left(1-\mu\right)^{4}\mu^{4}
\end{eqnarray*}
y por tanto

\begin{eqnarray*}
G^{'}\left(z\right)&=&2R_{2}R_{3}\left[R_{2}^{'}R_{3}+R_{2}R_{3}^{'}\right]\\
G^{'}\left(1\right)&=&-2\left(1-\mu\right)\mu\left[\left(-\frac{1}{2}\esp\left[X^{2}\right]+\frac{1}{2}\mu\right)\left(-\mu\right)+\left(1-\mu\right)\left(-\frac{1}{2}\esp\left[X^{2}\right]+\frac{1}{2}\mu\right)\right]
\end{eqnarray*}


\begin{eqnarray*}
F^{'}\left(z\right)&=&\left[\left(R_{2}R_{3}\right)R_{1}^{''}
-\left(R_{1}R_{3}\right)R_{2}^{''}
-\left(R_{1}R_{2}\right)R_{3}^{''}
-2\left(R_{2}^{'}R_{3}^{'}\right)R_{1}\right]P
+2\left(R_{2}R_{3}\right)R_{1}^{'}P^{'}
+\left(R_{1}R_{2}R_{3}\right)P^{''}
\end{eqnarray*}

Por lo tanto, encontremos $F^{'}\left(z\right)G\left(z\right)+F\left(z\right)G^{'}\left(z\right)$:

\begin{eqnarray*}
&&F^{'}\left(z\right)G\left(z\right)+F\left(z\right)G^{'}\left(z\right)=
\left\{\left[\left(R_{2}R_{3}\right)R_{1}^{''}
-\left(R_{1}R_{3}\right)R_{2}^{''}
-\left(R_{1}R_{2}\right)R_{3}^{''}
-2\left(R_{2}^{'}R_{3}^{'}\right)R_{1}\right]P\right.\\
&&\left.+2\left(R_{2}R_{3}\right)R_{1}^{'}P^{'}
+\left(R_{1}R_{2}R_{3}\right)P^{''}\right\}R_{2}^{2}R_{3}^{2}
-\left\{\left[PR_{2}R_{3}R_{1}^{'}+R_{1}R_{2}R_{3}P^{'}
-R_{3}R_{1}PR_{2}^{'}\right.\right.\\
&&\left.\left.
-R_{2}R_{1}PR_{3}^{'}\right]\left[2R_{2}R_{3}\left(R_{2}^{'}R_{3}+R_{2}R_{3}^{'}\right)\right]\right\}
\end{eqnarray*}
Evaluando en $z=1$

\begin{eqnarray*}
&=&\left(1+R_{3}\right)^{3}R_{3}^{3}R_{1}^{''}-\left(1+R_{3}\right)^{2}R_{1}R_{3}^{3}R_{3}^{''}
-\left(1+R_{3}\right)^{3}R_{3}^{2}R_{1}R_{3}^{''}-2\left(1+R_{3}\right)^{2}R_{3}^{2}
\left(R_{3}^{'}\right)^{2}\\
&+&2\left(1+R_{3}\right)^{3}R_{3}^{3}R_{1}^{'}P^{'}
+\left(1+R_{3}\right)^{3}R_{3}^{3}R_{1}P^{''}
-2\left(1+R_{3}\right)^{2}R_{3}^{2}\left(1+2R_{3}\right)R_{3}^{'}R_{1}^{'}\\
&-&2\left(1+R_{3}\right)^{2}R_{3}^{2}R_{1}R_{3}^{'}\left(1+2R_{3}\right)P^{'}
+2\left(1+R_{3}\right)\left(1+2R_{3}\right)R_{3}^{3}R_{1}\left(R_{3}^{'}\right)^{2}\\
&+&2\left(1+R_{3}\right)^{2}\left(1+2R_{3}\right)R_{1}R_{3}R_{3}^{'}\\
&=&-\left(1-\mu\right)^{3}\mu^{3}R_{1}^{''}-\left(1-\mu\right)^{2}\mu^{2}R_{1}\left(1-2\mu\right)R_{3}^{''}
-\left(1-\mu\right)^{3}\mu^{3}R_{1}P^{''}\\
&+&2\left(1-\mu\right)\mu^{2}\left[\left(1-2\mu\right)R_{1}-\left(1-\mu\right)\right]\left(R_{3}^{'}\right)^{2}
-2\left(1-\mu\right)^{2}\mu R_{1}\left(1-2\mu\right)R_{3}^{'}\\
&-&2\left(1-\mu\right)^{3}\mu^{4}R_{1}^{'}-2\mu\left(1-\mu\right)\left(1-2\mu\right)R_{3}^{'}R_{1}^{'}
-2\mu^{3}\left(1-\mu\right)^{2}\left(1-2\mu\right)R_{1}R_{1}^{'}
\end{eqnarray*}

por tanto

\begin{eqnarray*}
\left[\frac{F\left(z\right)}{G\left(z\right)}\right]^{'}&=&\frac{1}{\mu^{3}\left(1-\mu\right)^{3}}\left\{
-\left(1-\mu\right)^{2}\mu^{2}R_{1}^{''}-\mu\left(1-\mu\right)\left(1-2\mu\right)R_{1}R_{3}^{''}
-\mu^{2}\left(1-\mu\right)^{2}R_{1}P^{''}\right.\\
&&\left.+2\mu\left[\left(1-2\mu\right)R_{1}-\left(1-\mu\right)\right]\left(R_{3}^{'}\right)^{2}
-2\left(1-\mu\right)\left(1-2\mu\right)R_{1}R_{3}^{'}-2\mu^{3}\left(1-\mu\right)^{2}R_{1}^{'}\right.\\
&&\left.-2\left(1-2\mu\right)R_{3}^{'}R_{1}^{'}-2\mu^{2}\left(1-\mu\right)\left(1-2\mu\right)R_{1}R_{1}^{'}\right\}
\end{eqnarray*}

recordemos que


\begin{eqnarray*}
R_{1}&=&-\esp L\\
R_{3}&=& -\mu\\
R_{1}^{'}&=&-\frac{1}{2}\esp L^{2}+\frac{1}{2}\esp L\\
R_{3}^{'}&=&-\frac{1}{2}\esp X^{2}+\frac{1}{2}\mu\\
R_{1}^{''}&=&-\frac{1}{3}\esp L^{3}+\esp L^{2}-\frac{2}{3}\esp L\\
R_{3}^{''}&=&-\frac{1}{3}\esp X^{3}+\esp X^{2}-\frac{2}{3}\mu\\
R_{1}R_{3}^{'}&=&\frac{1}{2}\esp X^{2}\esp L-\frac{1}{2}\esp X\esp L\\
R_{1}R_{1}^{'}&=&\frac{1}{2}\esp L^{2}\esp L+\frac{1}{2}\esp^{2}L\\
R_{3}^{'}R_{1}^{'}&=&\frac{1}{4}\esp X^{2}\esp L^{2}-\frac{1}{4}\esp X^{2}\esp L-\frac{1}{4}\esp L^{2}\esp X+\frac{1}{4}\esp X\esp L\\
R_{1}R_{3}^{''}&=&\frac{1}{6}\esp X^{3}\esp L^{2}-\frac{1}{6}\esp X^{3}\esp L-\frac{1}{2}\esp L^{2}\esp X^{2}+\frac{1}{2}\esp X^{2}\esp L+\frac{1}{3}\esp X\esp L^{2}-\frac{1}{3}\esp X\esp L\\
R_{1}P^{''}&=&-\esp X^{2}\esp L\\
\left(R_{3}^{'}\right)^{2}&=&\frac{1}{4}\esp^{2}X^{2}-\frac{1}{2}\esp X^{2}\esp X+\frac{1}{4}\esp^{2} X
\end{eqnarray*}




\begin{Def}
Let $L_{i}^{*}$ be the number of users at queue $Q_{i}$ when it is polled, then
\begin{eqnarray}
\begin{array}{cc}
\esp\left[L_{i}^{*}\right]=f_{i}\left(i\right), &
Var\left[L_{i}^{*}\right]=f_{i}\left(i,i\right)+\esp\left[L_{i}^{*}\right]-\esp\left[L_{i}^{*}\right]^{2}.
\end{array}
\end{eqnarray}
\end{Def}

\begin{Def}
The cycle time $C_{i}$ for the queue $Q_{i}$ is the period beginning at the time when it is polled in a cycle and ending at the time when it is polled in the next cycle; it's duration is given by $\tau_{i}\left(m+1\right)-\tau_{i}\left(m\right)$, equivalently $\overline{\tau}_{i}\left(m+1\right)-\overline{\tau}_{i}\left(m\right)$ under steady state assumption.
\end{Def}

\begin{Def}
The intervisit time $I_{i}$ is defined as the period beginning at the time of its service completion in a cycle and ending at the time when it is polled in the next cycle; its duration is given by $\tau_{i}\left(m+1\right)-\overline{\tau}_{i}\left(m\right)$.
\end{Def}

The intervisit time duration $\tau_{i}\left(m+1\right)-\overline{\tau}\left(m\right)$ given the number of users found at queue $Q_{i}$ at time $t=\tau_{i}\left(m+1\right)$ is equal to the number of arrivals during the preceding intervisit time $\left[\overline{\tau}\left(m\right),\tau_{i}\left(m+1\right)\right]$. 

So we have



\begin{eqnarray*}
\esp\left[z_{i}^{L_{i}\left(\tau_{i}\left(m+1\right)\right)}\right]=\esp\left[\left\{P_{i}\left(z_{i}\right)\right\}^{\tau_{i}\left(m+1\right)-\overline{\tau}\left(m\right)}\right]
\end{eqnarray*}

if $I_{i}\left(z\right)=\esp\left[z^{\tau_{i}\left(m+1\right)-\overline{\tau}\left(m\right)}\right]$
we have $F_{i}\left(z\right)=I_{i}\left[P_{i}\left(z\right)\right]$
for $i=1,2$. Futhermore can be proved that

\begin{eqnarray}
\begin{array}{ll}
\esp\left[L_{i}\right]=\mu_{i}\esp\left[I_{i}\right], &
\esp\left[C_{i}\right]=\frac{f_{i}\left(i\right)}{\mu_{i}\left(1-\mu_{i}\right)},\\
\esp\left[S_{i}\right]=\mu_{i}\esp\left[C_{i}\right],&
\esp\left[I_{i}\right]=\left(1-\mu_{i}\right)\esp\left[C_{i}\right],\\
Var\left[L_{i}\right]= \mu_{i}^{2}Var\left[I_{i}\right]+\sigma^{2}\esp\left[I_{i}\right],& 
Var\left[C_{i}\right]=\frac{Var\left[L_{i}^{*}\right]}{\mu_{i}^{2}\left(1-\mu_{i}\right)^{2}},\\
Var\left[S_{i}\right]= \frac{Var\left[L_{i}^{*}\right]}{\left(1-\mu_{i}\right)^{2}}+\frac{\sigma^{2}\esp\left[L_{i}^{*}\right]}{\left(1-\mu_{i}\right)^{3}},&
Var\left[I_{i}\right]= \frac{Var\left[L_{i}^{*}\right]}{\mu_{i}^{2}}-\frac{\sigma_{i}^{2}}{\mu_{i}^{2}}f_{i}\left(i\right).
\end{array}
\end{eqnarray}

Let consider the points when the process $\left[L_{1}\left(1\right),L_{2}\left(1\right),L_{3}\left(1\right),L_{4}\left(1\right)
\right]$ becomes zero at the same time, this points, $T_{1},T_{2},\ldots$ will be denoted as regeneration points, then we have that

\begin{Def}
the interval between two such succesive regeneration points will be called regenerative cycle.
\end{Def}

\begin{Def}
Para $T_{i}$ se define, $M_{i}$, el n\'umero de ciclos de visita a la cola $Q_{l}$, durante el ciclo regenerativo, es decir, $M_{i}$ es un proceso de renovaci\'on.
\end{Def}

\begin{Def}
Para cada uno de los $M_{i}$'s, se definen a su vez la duraci\'on de cada uno de estos ciclos de visita en el ciclo regenerativo, $C_{i}^{(m)}$, para $m=1,2,\ldots,M_{i}$, que a su vez, tambi\'en es n proceso de renovaci\'on.
\end{Def}



Sea la funci\'on generadora de momentos para $L_{i}$, el n\'umero de usuarios en la cola $Q_{i}\left(z\right)$ en cualquier momento, est\'a dada por el tiempo promedio de $z^{L_{i}\left(t\right)}$ sobre el ciclo regenerativo definido anteriormente. Entonces 

\begin{equation}\label{Eq.Longitud.Tiempo.t}
Q_{i}\left(z\right)=\frac{1}{\esp\left[C_{i}\right]}\cdot\frac{1-F_{i}\left(z\right)}{P_{i}\left(z\right)-z}\cdot\frac{\left(1-z\right)P_{i}\left(z\right)}{1-P_{i}\left(z\right)}.
\end{equation}

Es decir, es posible determinar las longitudes de las colas a cualquier tiempo $t$. Entonces, determinando el primer momento es posible ver que


$M_{i}$ is an stopping time for the regenerative process with $\esp\left[M_{i}\right]<\infty$, from Wald's lemma follows that:


\begin{eqnarray*}
\esp\left[\sum_{m=1}^{M_{i}}\sum_{t=\tau_{i}\left(m\right)}^{\tau_{i}\left(m+1\right)-1}z^{L_{i}\left(t\right)}\right]&=&\esp\left[M_{i}\right]\esp\left[\sum_{t=\tau_{i}\left(m\right)}^{\tau_{i}\left(m+1\right)-1}z^{L_{i}\left(t\right)}\right]\\
\esp\left[\sum_{m=1}^{M_{i}}\tau_{i}\left(m+1\right)-\tau_{i}\left(m\right)\right]&=&\esp\left[M_{i}\right]\esp\left[\tau_{i}\left(m+1\right)-\tau_{i}\left(m\right)\right]
\end{eqnarray*}
therefore 

\begin{eqnarray*}
Q_{i}\left(z\right)&=&\frac{\esp\left[\sum_{t=\tau_{i}\left(m\right)}^{\tau_{i}\left(m+1\right)-1}z^{L_{i}\left(t\right)}\right]}{\esp\left[\tau_{i}\left(m+1\right)-\tau_{i}\left(m\right)\right]}
\end{eqnarray*}

Doing the following substitutions en (\ref{Corolario2}): $n\rightarrow t-\tau_{i}\left(m\right)$, $T \rightarrow \overline{\tau}_{i}\left(m\right)-\tau_{i}\left(m\right)$, $L_{n}\rightarrow L_{i}\left(t\right)$ and $F\left(z\right)=\esp\left[z^{L_{0}}\right]\rightarrow F_{i}\left(z\right)=\esp\left[z^{L_{i}\tau_{i}\left(m\right)}\right]$, 
we obtain

\begin{eqnarray}\label{Eq.Arribos.Primera}
\esp\left[\sum_{n=0}^{T-1}z^{L_{n}}\right]=
\esp\left[\sum_{t=\tau_{i}\left(m\right)}^{\overline{\tau}_{i}\left(m\right)-1}z^{L_{i}\left(t\right)}\right]
=z\frac{F_{i}\left(z\right)-1}{z-P_{i}\left(z\right)}
\end{eqnarray}



Por otra parte durante el tiempo de intervisita para la cola $i$, $L_{i}\left(t\right)$ solamente se incrementa de manera que el incremento por intervalo de tiempo est\'a dado por la funci\'on generadora de probabilidades de $P_{i}\left(z\right)$, por tanto la suma sobre el tiempo de intervisita puede evaluarse como:

\begin{eqnarray*}
\esp\left[\sum_{t=\tau_{i}\left(m\right)}^{\tau_{i}\left(m+1\right)-1}z^{L_{i}\left(t\right)}\right]&=&\esp\left[\sum_{t=\tau_{i}\left(m\right)}^{\tau_{i}\left(m+1\right)-1}\left\{P_{i}\left(z\right)\right\}^{t-\overline{\tau}_{i}\left(m\right)}\right]=\frac{1-\esp\left[\left\{P_{i}\left(z\right)\right\}^{\tau_{i}\left(m+1\right)-\overline{\tau}_{i}\left(m\right)}\right]}{1-P_{i}\left(z\right)}\\
&=&\frac{1-I_{i}\left[P_{i}\left(z\right)\right]}{1-P_{i}\left(z\right)}
\end{eqnarray*}
por tanto

\begin{eqnarray*}
\esp\left[\sum_{t=\tau_{i}\left(m\right)}^{\tau_{i}\left(m+1\right)-1}z^{L_{i}\left(t\right)}\right]&=&
\frac{1-F_{i}\left(z\right)}{1-P_{i}\left(z\right)}
\end{eqnarray*}

Por lo tanto

\begin{eqnarray*}
Q_{i}\left(z\right)&=&\frac{\esp\left[\sum_{t=\tau_{i}\left(m\right)}^{\tau_{i}\left(m+1\right)-1}z^{L_{i}\left(t\right)}\right]}{\esp\left[\tau_{i}\left(m+1\right)-\tau_{i}\left(m\right)\right]}
=\frac{1}{\esp\left[\tau_{i}\left(m+1\right)-\tau_{i}\left(m\right)\right]}
\esp\left[\sum_{t=\tau_{i}\left(m\right)}^{\tau_{i}\left(m+1\right)-1}z^{L_{i}\left(t\right)}\right]\\
&=&\frac{1}{\esp\left[\tau_{i}\left(m+1\right)-\tau_{i}\left(m\right)\right]}
\esp\left[\sum_{t=\tau_{i}\left(m\right)}^{\overline{\tau}_{i}\left(m\right)-1}z^{L_{i}\left(t\right)}
+\sum_{t=\overline{\tau}_{i}\left(m\right)}^{\tau_{i}\left(m+1\right)-1}z^{L_{i}\left(t\right)}\right]\\
&=&\frac{1}{\esp\left[\tau_{i}\left(m+1\right)-\tau_{i}\left(m\right)\right]}\left\{
\esp\left[\sum_{t=\tau_{i}\left(m\right)}^{\overline{\tau}_{i}\left(m\right)-1}z^{L_{i}\left(t\right)}\right]
+\esp\left[\sum_{t=\overline{\tau}_{i}\left(m\right)}^{\tau_{i}\left(m+1\right)-1}z^{L_{i}\left(t\right)}\right]\right\}\\
&=&\frac{1}{\esp\left[\tau_{i}\left(m+1\right)-\tau_{i}\left(m\right)\right]}\left\{
z\frac{F_{i}\left(z\right)-1}{z-P_{i}\left(z\right)}+\frac{1-F_{i}\left(z\right)}{1-P_{i}\left(z\right)}
\right\}\\
&=&\frac{1}{\esp\left[C_{i}\right]}\cdot\frac{1-F_{i}\left(z\right)}{P_{i}\left(z\right)-z}\cdot\frac{\left(1-z\right)P_{i}\left(z\right)}{1-P_{i}\left(z\right)}
\end{eqnarray*}

es decir

\begin{eqnarray}
\begin{array}{ll}
S^{'}\left(z\right)=-\sum_{k=1}^{+\infty}ka_{k}z^{k-1},& S^{(1)}\left(1\right)=-\sum_{k=1}^{+\infty}ka_{k}=-\esp\left[L\left(t\right)\right],\\
S^{''}\left(z\right)=-\sum_{k=2}^{+\infty}k(k-1)a_{k}z^{k-2},& S^{(2)}\left(1\right)=-\sum_{k=2}^{+\infty}k(k-1)a_{k}=\esp\left[L\left(L-1\right)\right],\\
S^{'''}\left(z\right)=-\sum_{k=3}^{+\infty}k(k-1)(k-2)a_{k}z^{k-3},&
S^{(3)}\left(1\right)=-\sum_{k=3}^{+\infty}k(k-1)(k-2)a_{k}\\
&=-\esp\left[L\left(L-1\right)\left(L-2\right)\right]\\
&=-\esp\left[L^{3}\right]+3-\esp\left[L^{2}\right]-2-\esp\left[L\right];
\end{array}
\end{eqnarray}








%________________________________________________________________________
%\subsection{Procesos Regenerativos Sigman, Thorisson y Wolff \cite{Sigman1}}
%________________________________________________________________________


\begin{Def}[Definici\'on Cl\'asica]
Un proceso estoc\'astico $X=\left\{X\left(t\right):t\geq0\right\}$ es llamado regenerativo is existe una variable aleatoria $R_{1}>0$ tal que
\begin{itemize}
\item[i)] $\left\{X\left(t+R_{1}\right):t\geq0\right\}$ es independiente de $\left\{\left\{X\left(t\right):t<R_{1}\right\},\right\}$
\item[ii)] $\left\{X\left(t+R_{1}\right):t\geq0\right\}$ es estoc\'asticamente equivalente a $\left\{X\left(t\right):t>0\right\}$
\end{itemize}

Llamamos a $R_{1}$ tiempo de regeneraci\'on, y decimos que $X$ se regenera en este punto.
\end{Def}

$\left\{X\left(t+R_{1}\right)\right\}$ es regenerativo con tiempo de regeneraci\'on $R_{2}$, independiente de $R_{1}$ pero con la misma distribuci\'on que $R_{1}$. Procediendo de esta manera se obtiene una secuencia de variables aleatorias independientes e id\'enticamente distribuidas $\left\{R_{n}\right\}$ llamados longitudes de ciclo. Si definimos a $Z_{k}\equiv R_{1}+R_{2}+\cdots+R_{k}$, se tiene un proceso de renovaci\'on llamado proceso de renovaci\'on encajado para $X$.


\begin{Note}
La existencia de un primer tiempo de regeneraci\'on, $R_{1}$, implica la existencia de una sucesi\'on completa de estos tiempos $R_{1},R_{2}\ldots,$ que satisfacen la propiedad deseada \cite{Sigman2}.
\end{Note}


\begin{Note} Para la cola $GI/GI/1$ los usuarios arriban con tiempos $t_{n}$ y son atendidos con tiempos de servicio $S_{n}$, los tiempos de arribo forman un proceso de renovaci\'on  con tiempos entre arribos independientes e identicamente distribuidos (\texttt{i.i.d.})$T_{n}=t_{n}-t_{n-1}$, adem\'as los tiempos de servicio son \texttt{i.i.d.} e independientes de los procesos de arribo. Por \textit{estable} se entiende que $\esp S_{n}<\esp T_{n}<\infty$.
\end{Note}
 


\begin{Def}
Para $x$ fijo y para cada $t\geq0$, sea $I_{x}\left(t\right)=1$ si $X\left(t\right)\leq x$,  $I_{x}\left(t\right)=0$ en caso contrario, y def\'inanse los tiempos promedio
\begin{eqnarray*}
\overline{X}&=&lim_{t\rightarrow\infty}\frac{1}{t}\int_{0}^{\infty}X\left(u\right)du\\
\prob\left(X_{\infty}\leq x\right)&=&lim_{t\rightarrow\infty}\frac{1}{t}\int_{0}^{\infty}I_{x}\left(u\right)du,
\end{eqnarray*}
cuando estos l\'imites existan.
\end{Def}

Como consecuencia del teorema de Renovaci\'on-Recompensa, se tiene que el primer l\'imite  existe y es igual a la constante
\begin{eqnarray*}
\overline{X}&=&\frac{\esp\left[\int_{0}^{R_{1}}X\left(t\right)dt\right]}{\esp\left[R_{1}\right]},
\end{eqnarray*}
suponiendo que ambas esperanzas son finitas.
 
\begin{Note}
Funciones de procesos regenerativos son regenerativas, es decir, si $X\left(t\right)$ es regenerativo y se define el proceso $Y\left(t\right)$ por $Y\left(t\right)=f\left(X\left(t\right)\right)$ para alguna funci\'on Borel medible $f\left(\cdot\right)$. Adem\'as $Y$ es regenerativo con los mismos tiempos de renovaci\'on que $X$. 

En general, los tiempos de renovaci\'on, $Z_{k}$ de un proceso regenerativo no requieren ser tiempos de paro con respecto a la evoluci\'on de $X\left(t\right)$.
\end{Note} 

\begin{Note}
Una funci\'on de un proceso de Markov, usualmente no ser\'a un proceso de Markov, sin embargo ser\'a regenerativo si el proceso de Markov lo es.
\end{Note}

 
\begin{Note}
Un proceso regenerativo con media de la longitud de ciclo finita es llamado positivo recurrente.
\end{Note}


\begin{Note}
\begin{itemize}
\item[a)] Si el proceso regenerativo $X$ es positivo recurrente y tiene trayectorias muestrales no negativas, entonces la ecuaci\'on anterior es v\'alida.
\item[b)] Si $X$ es positivo recurrente regenerativo, podemos construir una \'unica versi\'on estacionaria de este proceso, $X_{e}=\left\{X_{e}\left(t\right)\right\}$, donde $X_{e}$ es un proceso estoc\'astico regenerativo y estrictamente estacionario, con distribuci\'on marginal distribuida como $X_{\infty}$
\end{itemize}
\end{Note}


%__________________________________________________________________________________________
%\subsection{Procesos Regenerativos Estacionarios - Stidham \cite{Stidham}}
%__________________________________________________________________________________________


Un proceso estoc\'astico a tiempo continuo $\left\{V\left(t\right),t\geq0\right\}$ es un proceso regenerativo si existe una sucesi\'on de variables aleatorias independientes e id\'enticamente distribuidas $\left\{X_{1},X_{2},\ldots\right\}$, sucesi\'on de renovaci\'on, tal que para cualquier conjunto de Borel $A$, 

\begin{eqnarray*}
\prob\left\{V\left(t\right)\in A|X_{1}+X_{2}+\cdots+X_{R\left(t\right)}=s,\left\{V\left(\tau\right),\tau<s\right\}\right\}=\prob\left\{V\left(t-s\right)\in A|X_{1}>t-s\right\},
\end{eqnarray*}
para todo $0\leq s\leq t$, donde $R\left(t\right)=\max\left\{X_{1}+X_{2}+\cdots+X_{j}\leq t\right\}=$n\'umero de renovaciones ({\emph{puntos de regeneraci\'on}}) que ocurren en $\left[0,t\right]$. El intervalo $\left[0,X_{1}\right)$ es llamado {\emph{primer ciclo de regeneraci\'on}} de $\left\{V\left(t \right),t\geq0\right\}$, $\left[X_{1},X_{1}+X_{2}\right)$ el {\emph{segundo ciclo de regeneraci\'on}}, y as\'i sucesivamente.

Sea $X=X_{1}$ y sea $F$ la funci\'on de distrbuci\'on de $X$


\begin{Def}
Se define el proceso estacionario, $\left\{V^{*}\left(t\right),t\geq0\right\}$, para $\left\{V\left(t\right),t\geq0\right\}$ por

\begin{eqnarray*}
\prob\left\{V\left(t\right)\in A\right\}=\frac{1}{\esp\left[X\right]}\int_{0}^{\infty}\prob\left\{V\left(t+x\right)\in A|X>x\right\}\left(1-F\left(x\right)\right)dx,
\end{eqnarray*} 
para todo $t\geq0$ y todo conjunto de Borel $A$.
\end{Def}

\begin{Def}
Una distribuci\'on se dice que es {\emph{aritm\'etica}} si todos sus puntos de incremento son m\'ultiplos de la forma $0,\lambda, 2\lambda,\ldots$ para alguna $\lambda>0$ entera.
\end{Def}


\begin{Def}
Una modificaci\'on medible de un proceso $\left\{V\left(t\right),t\geq0\right\}$, es una versi\'on de este, $\left\{V\left(t,w\right)\right\}$ conjuntamente medible para $t\geq0$ y para $w\in S$, $S$ espacio de estados para $\left\{V\left(t\right),t\geq0\right\}$.
\end{Def}

\begin{Teo}
Sea $\left\{V\left(t\right),t\geq\right\}$ un proceso regenerativo no negativo con modificaci\'on medible. Sea $\esp\left[X\right]<\infty$. Entonces el proceso estacionario dado por la ecuaci\'on anterior est\'a bien definido y tiene funci\'on de distribuci\'on independiente de $t$, adem\'as
\begin{itemize}
\item[i)] \begin{eqnarray*}
\esp\left[V^{*}\left(0\right)\right]&=&\frac{\esp\left[\int_{0}^{X}V\left(s\right)ds\right]}{\esp\left[X\right]}\end{eqnarray*}
\item[ii)] Si $\esp\left[V^{*}\left(0\right)\right]<\infty$, equivalentemente, si $\esp\left[\int_{0}^{X}V\left(s\right)ds\right]<\infty$,entonces
\begin{eqnarray*}
\frac{\int_{0}^{t}V\left(s\right)ds}{t}\rightarrow\frac{\esp\left[\int_{0}^{X}V\left(s\right)ds\right]}{\esp\left[X\right]}
\end{eqnarray*}
con probabilidad 1 y en media, cuando $t\rightarrow\infty$.
\end{itemize}
\end{Teo}

\begin{Coro}
Sea $\left\{V\left(t\right),t\geq0\right\}$ un proceso regenerativo no negativo, con modificaci\'on medible. Si $\esp <\infty$, $F$ es no-aritm\'etica, y para todo $x\geq0$, $P\left\{V\left(t\right)\leq x,C>x\right\}$ es de variaci\'on acotada como funci\'on de $t$ en cada intervalo finito $\left[0,\tau\right]$, entonces $V\left(t\right)$ converge en distribuci\'on  cuando $t\rightarrow\infty$ y $$\esp V=\frac{\esp \int_{0}^{X}V\left(s\right)ds}{\esp X}$$
Donde $V$ tiene la distribuci\'on l\'imite de $V\left(t\right)$ cuando $t\rightarrow\infty$.

\end{Coro}

Para el caso discreto se tienen resultados similares.



%______________________________________________________________________
%\subsection{Procesos de Renovaci\'on}
%______________________________________________________________________

\begin{Def}%\label{Def.Tn}
Sean $0\leq T_{1}\leq T_{2}\leq \ldots$ son tiempos aleatorios infinitos en los cuales ocurren ciertos eventos. El n\'umero de tiempos $T_{n}$ en el intervalo $\left[0,t\right)$ es

\begin{eqnarray}
N\left(t\right)=\sum_{n=1}^{\infty}\indora\left(T_{n}\leq t\right),
\end{eqnarray}
para $t\geq0$.
\end{Def}

Si se consideran los puntos $T_{n}$ como elementos de $\rea_{+}$, y $N\left(t\right)$ es el n\'umero de puntos en $\rea$. El proceso denotado por $\left\{N\left(t\right):t\geq0\right\}$, denotado por $N\left(t\right)$, es un proceso puntual en $\rea_{+}$. Los $T_{n}$ son los tiempos de ocurrencia, el proceso puntual $N\left(t\right)$ es simple si su n\'umero de ocurrencias son distintas: $0<T_{1}<T_{2}<\ldots$ casi seguramente.

\begin{Def}
Un proceso puntual $N\left(t\right)$ es un proceso de renovaci\'on si los tiempos de interocurrencia $\xi_{n}=T_{n}-T_{n-1}$, para $n\geq1$, son independientes e identicamente distribuidos con distribuci\'on $F$, donde $F\left(0\right)=0$ y $T_{0}=0$. Los $T_{n}$ son llamados tiempos de renovaci\'on, referente a la independencia o renovaci\'on de la informaci\'on estoc\'astica en estos tiempos. Los $\xi_{n}$ son los tiempos de inter-renovaci\'on, y $N\left(t\right)$ es el n\'umero de renovaciones en el intervalo $\left[0,t\right)$
\end{Def}


\begin{Note}
Para definir un proceso de renovaci\'on para cualquier contexto, solamente hay que especificar una distribuci\'on $F$, con $F\left(0\right)=0$, para los tiempos de inter-renovaci\'on. La funci\'on $F$ en turno degune las otra variables aleatorias. De manera formal, existe un espacio de probabilidad y una sucesi\'on de variables aleatorias $\xi_{1},\xi_{2},\ldots$ definidas en este con distribuci\'on $F$. Entonces las otras cantidades son $T_{n}=\sum_{k=1}^{n}\xi_{k}$ y $N\left(t\right)=\sum_{n=1}^{\infty}\indora\left(T_{n}\leq t\right)$, donde $T_{n}\rightarrow\infty$ casi seguramente por la Ley Fuerte de los Grandes Números.
\end{Note}

%___________________________________________________________________________________________
%
%\subsection{Teorema Principal de Renovaci\'on}
%___________________________________________________________________________________________
%

\begin{Note} Una funci\'on $h:\rea_{+}\rightarrow\rea$ es Directamente Riemann Integrable en los siguientes casos:
\begin{itemize}
\item[a)] $h\left(t\right)\geq0$ es decreciente y Riemann Integrable.
\item[b)] $h$ es continua excepto posiblemente en un conjunto de Lebesgue de medida 0, y $|h\left(t\right)|\leq b\left(t\right)$, donde $b$ es DRI.
\end{itemize}
\end{Note}

\begin{Teo}[Teorema Principal de Renovaci\'on]
Si $F$ es no aritm\'etica y $h\left(t\right)$ es Directamente Riemann Integrable (DRI), entonces

\begin{eqnarray*}
lim_{t\rightarrow\infty}U\star h=\frac{1}{\mu}\int_{\rea_{+}}h\left(s\right)ds.
\end{eqnarray*}
\end{Teo}

\begin{Prop}
Cualquier funci\'on $H\left(t\right)$ acotada en intervalos finitos y que es 0 para $t<0$ puede expresarse como
\begin{eqnarray*}
H\left(t\right)=U\star h\left(t\right)\textrm{,  donde }h\left(t\right)=H\left(t\right)-F\star H\left(t\right)
\end{eqnarray*}
\end{Prop}

\begin{Def}
Un proceso estoc\'astico $X\left(t\right)$ es crudamente regenerativo en un tiempo aleatorio positivo $T$ si
\begin{eqnarray*}
\esp\left[X\left(T+t\right)|T\right]=\esp\left[X\left(t\right)\right]\textrm{, para }t\geq0,\end{eqnarray*}
y con las esperanzas anteriores finitas.
\end{Def}

\begin{Prop}
Sup\'ongase que $X\left(t\right)$ es un proceso crudamente regenerativo en $T$, que tiene distribuci\'on $F$. Si $\esp\left[X\left(t\right)\right]$ es acotado en intervalos finitos, entonces
\begin{eqnarray*}
\esp\left[X\left(t\right)\right]=U\star h\left(t\right)\textrm{,  donde }h\left(t\right)=\esp\left[X\left(t\right)\indora\left(T>t\right)\right].
\end{eqnarray*}
\end{Prop}

\begin{Teo}[Regeneraci\'on Cruda]
Sup\'ongase que $X\left(t\right)$ es un proceso con valores positivo crudamente regenerativo en $T$, y def\'inase $M=\sup\left\{|X\left(t\right)|:t\leq T\right\}$. Si $T$ es no aritm\'etico y $M$ y $MT$ tienen media finita, entonces
\begin{eqnarray*}
lim_{t\rightarrow\infty}\esp\left[X\left(t\right)\right]=\frac{1}{\mu}\int_{\rea_{+}}h\left(s\right)ds,
\end{eqnarray*}
donde $h\left(t\right)=\esp\left[X\left(t\right)\indora\left(T>t\right)\right]$.
\end{Teo}

%___________________________________________________________________________________________
%
%\subsection{Propiedades de los Procesos de Renovaci\'on}
%___________________________________________________________________________________________
%

Los tiempos $T_{n}$ est\'an relacionados con los conteos de $N\left(t\right)$ por

\begin{eqnarray*}
\left\{N\left(t\right)\geq n\right\}&=&\left\{T_{n}\leq t\right\}\\
T_{N\left(t\right)}\leq &t&<T_{N\left(t\right)+1},
\end{eqnarray*}

adem\'as $N\left(T_{n}\right)=n$, y 

\begin{eqnarray*}
N\left(t\right)=\max\left\{n:T_{n}\leq t\right\}=\min\left\{n:T_{n+1}>t\right\}
\end{eqnarray*}

Por propiedades de la convoluci\'on se sabe que

\begin{eqnarray*}
P\left\{T_{n}\leq t\right\}=F^{n\star}\left(t\right)
\end{eqnarray*}
que es la $n$-\'esima convoluci\'on de $F$. Entonces 

\begin{eqnarray*}
\left\{N\left(t\right)\geq n\right\}&=&\left\{T_{n}\leq t\right\}\\
P\left\{N\left(t\right)\leq n\right\}&=&1-F^{\left(n+1\right)\star}\left(t\right)
\end{eqnarray*}

Adem\'as usando el hecho de que $\esp\left[N\left(t\right)\right]=\sum_{n=1}^{\infty}P\left\{N\left(t\right)\geq n\right\}$
se tiene que

\begin{eqnarray*}
\esp\left[N\left(t\right)\right]=\sum_{n=1}^{\infty}F^{n\star}\left(t\right)
\end{eqnarray*}

\begin{Prop}
Para cada $t\geq0$, la funci\'on generadora de momentos $\esp\left[e^{\alpha N\left(t\right)}\right]$ existe para alguna $\alpha$ en una vecindad del 0, y de aqu\'i que $\esp\left[N\left(t\right)^{m}\right]<\infty$, para $m\geq1$.
\end{Prop}


\begin{Note}
Si el primer tiempo de renovaci\'on $\xi_{1}$ no tiene la misma distribuci\'on que el resto de las $\xi_{n}$, para $n\geq2$, a $N\left(t\right)$ se le llama Proceso de Renovaci\'on retardado, donde si $\xi$ tiene distribuci\'on $G$, entonces el tiempo $T_{n}$ de la $n$-\'esima renovaci\'on tiene distribuci\'on $G\star F^{\left(n-1\right)\star}\left(t\right)$
\end{Note}


\begin{Teo}
Para una constante $\mu\leq\infty$ ( o variable aleatoria), las siguientes expresiones son equivalentes:

\begin{eqnarray}
lim_{n\rightarrow\infty}n^{-1}T_{n}&=&\mu,\textrm{ c.s.}\\
lim_{t\rightarrow\infty}t^{-1}N\left(t\right)&=&1/\mu,\textrm{ c.s.}
\end{eqnarray}
\end{Teo}


Es decir, $T_{n}$ satisface la Ley Fuerte de los Grandes N\'umeros s\'i y s\'olo s\'i $N\left/t\right)$ la cumple.


\begin{Coro}[Ley Fuerte de los Grandes N\'umeros para Procesos de Renovaci\'on]
Si $N\left(t\right)$ es un proceso de renovaci\'on cuyos tiempos de inter-renovaci\'on tienen media $\mu\leq\infty$, entonces
\begin{eqnarray}
t^{-1}N\left(t\right)\rightarrow 1/\mu,\textrm{ c.s. cuando }t\rightarrow\infty.
\end{eqnarray}

\end{Coro}


Considerar el proceso estoc\'astico de valores reales $\left\{Z\left(t\right):t\geq0\right\}$ en el mismo espacio de probabilidad que $N\left(t\right)$

\begin{Def}
Para el proceso $\left\{Z\left(t\right):t\geq0\right\}$ se define la fluctuaci\'on m\'axima de $Z\left(t\right)$ en el intervalo $\left(T_{n-1},T_{n}\right]$:
\begin{eqnarray*}
M_{n}=\sup_{T_{n-1}<t\leq T_{n}}|Z\left(t\right)-Z\left(T_{n-1}\right)|
\end{eqnarray*}
\end{Def}

\begin{Teo}
Sup\'ongase que $n^{-1}T_{n}\rightarrow\mu$ c.s. cuando $n\rightarrow\infty$, donde $\mu\leq\infty$ es una constante o variable aleatoria. Sea $a$ una constante o variable aleatoria que puede ser infinita cuando $\mu$ es finita, y considere las expresiones l\'imite:
\begin{eqnarray}
lim_{n\rightarrow\infty}n^{-1}Z\left(T_{n}\right)&=&a,\textrm{ c.s.}\\
lim_{t\rightarrow\infty}t^{-1}Z\left(t\right)&=&a/\mu,\textrm{ c.s.}
\end{eqnarray}
La segunda expresi\'on implica la primera. Conversamente, la primera implica la segunda si el proceso $Z\left(t\right)$ es creciente, o si $lim_{n\rightarrow\infty}n^{-1}M_{n}=0$ c.s.
\end{Teo}

\begin{Coro}
Si $N\left(t\right)$ es un proceso de renovaci\'on, y $\left(Z\left(T_{n}\right)-Z\left(T_{n-1}\right),M_{n}\right)$, para $n\geq1$, son variables aleatorias independientes e id\'enticamente distribuidas con media finita, entonces,
\begin{eqnarray}
lim_{t\rightarrow\infty}t^{-1}Z\left(t\right)\rightarrow\frac{\esp\left[Z\left(T_{1}\right)-Z\left(T_{0}\right)\right]}{\esp\left[T_{1}\right]},\textrm{ c.s. cuando  }t\rightarrow\infty.
\end{eqnarray}
\end{Coro}



%___________________________________________________________________________________________
%
%\subsection{Propiedades de los Procesos de Renovaci\'on}
%___________________________________________________________________________________________
%

Los tiempos $T_{n}$ est\'an relacionados con los conteos de $N\left(t\right)$ por

\begin{eqnarray*}
\left\{N\left(t\right)\geq n\right\}&=&\left\{T_{n}\leq t\right\}\\
T_{N\left(t\right)}\leq &t&<T_{N\left(t\right)+1},
\end{eqnarray*}

adem\'as $N\left(T_{n}\right)=n$, y 

\begin{eqnarray*}
N\left(t\right)=\max\left\{n:T_{n}\leq t\right\}=\min\left\{n:T_{n+1}>t\right\}
\end{eqnarray*}

Por propiedades de la convoluci\'on se sabe que

\begin{eqnarray*}
P\left\{T_{n}\leq t\right\}=F^{n\star}\left(t\right)
\end{eqnarray*}
que es la $n$-\'esima convoluci\'on de $F$. Entonces 

\begin{eqnarray*}
\left\{N\left(t\right)\geq n\right\}&=&\left\{T_{n}\leq t\right\}\\
P\left\{N\left(t\right)\leq n\right\}&=&1-F^{\left(n+1\right)\star}\left(t\right)
\end{eqnarray*}

Adem\'as usando el hecho de que $\esp\left[N\left(t\right)\right]=\sum_{n=1}^{\infty}P\left\{N\left(t\right)\geq n\right\}$
se tiene que

\begin{eqnarray*}
\esp\left[N\left(t\right)\right]=\sum_{n=1}^{\infty}F^{n\star}\left(t\right)
\end{eqnarray*}

\begin{Prop}
Para cada $t\geq0$, la funci\'on generadora de momentos $\esp\left[e^{\alpha N\left(t\right)}\right]$ existe para alguna $\alpha$ en una vecindad del 0, y de aqu\'i que $\esp\left[N\left(t\right)^{m}\right]<\infty$, para $m\geq1$.
\end{Prop}


\begin{Note}
Si el primer tiempo de renovaci\'on $\xi_{1}$ no tiene la misma distribuci\'on que el resto de las $\xi_{n}$, para $n\geq2$, a $N\left(t\right)$ se le llama Proceso de Renovaci\'on retardado, donde si $\xi$ tiene distribuci\'on $G$, entonces el tiempo $T_{n}$ de la $n$-\'esima renovaci\'on tiene distribuci\'on $G\star F^{\left(n-1\right)\star}\left(t\right)$
\end{Note}


\begin{Teo}
Para una constante $\mu\leq\infty$ ( o variable aleatoria), las siguientes expresiones son equivalentes:

\begin{eqnarray}
lim_{n\rightarrow\infty}n^{-1}T_{n}&=&\mu,\textrm{ c.s.}\\
lim_{t\rightarrow\infty}t^{-1}N\left(t\right)&=&1/\mu,\textrm{ c.s.}
\end{eqnarray}
\end{Teo}


Es decir, $T_{n}$ satisface la Ley Fuerte de los Grandes N\'umeros s\'i y s\'olo s\'i $N\left/t\right)$ la cumple.


\begin{Coro}[Ley Fuerte de los Grandes N\'umeros para Procesos de Renovaci\'on]
Si $N\left(t\right)$ es un proceso de renovaci\'on cuyos tiempos de inter-renovaci\'on tienen media $\mu\leq\infty$, entonces
\begin{eqnarray}
t^{-1}N\left(t\right)\rightarrow 1/\mu,\textrm{ c.s. cuando }t\rightarrow\infty.
\end{eqnarray}

\end{Coro}


Considerar el proceso estoc\'astico de valores reales $\left\{Z\left(t\right):t\geq0\right\}$ en el mismo espacio de probabilidad que $N\left(t\right)$

\begin{Def}
Para el proceso $\left\{Z\left(t\right):t\geq0\right\}$ se define la fluctuaci\'on m\'axima de $Z\left(t\right)$ en el intervalo $\left(T_{n-1},T_{n}\right]$:
\begin{eqnarray*}
M_{n}=\sup_{T_{n-1}<t\leq T_{n}}|Z\left(t\right)-Z\left(T_{n-1}\right)|
\end{eqnarray*}
\end{Def}

\begin{Teo}
Sup\'ongase que $n^{-1}T_{n}\rightarrow\mu$ c.s. cuando $n\rightarrow\infty$, donde $\mu\leq\infty$ es una constante o variable aleatoria. Sea $a$ una constante o variable aleatoria que puede ser infinita cuando $\mu$ es finita, y considere las expresiones l\'imite:
\begin{eqnarray}
lim_{n\rightarrow\infty}n^{-1}Z\left(T_{n}\right)&=&a,\textrm{ c.s.}\\
lim_{t\rightarrow\infty}t^{-1}Z\left(t\right)&=&a/\mu,\textrm{ c.s.}
\end{eqnarray}
La segunda expresi\'on implica la primera. Conversamente, la primera implica la segunda si el proceso $Z\left(t\right)$ es creciente, o si $lim_{n\rightarrow\infty}n^{-1}M_{n}=0$ c.s.
\end{Teo}

\begin{Coro}
Si $N\left(t\right)$ es un proceso de renovaci\'on, y $\left(Z\left(T_{n}\right)-Z\left(T_{n-1}\right),M_{n}\right)$, para $n\geq1$, son variables aleatorias independientes e id\'enticamente distribuidas con media finita, entonces,
\begin{eqnarray}
lim_{t\rightarrow\infty}t^{-1}Z\left(t\right)\rightarrow\frac{\esp\left[Z\left(T_{1}\right)-Z\left(T_{0}\right)\right]}{\esp\left[T_{1}\right]},\textrm{ c.s. cuando  }t\rightarrow\infty.
\end{eqnarray}
\end{Coro}


%___________________________________________________________________________________________
%
%\subsection{Propiedades de los Procesos de Renovaci\'on}
%___________________________________________________________________________________________
%

Los tiempos $T_{n}$ est\'an relacionados con los conteos de $N\left(t\right)$ por

\begin{eqnarray*}
\left\{N\left(t\right)\geq n\right\}&=&\left\{T_{n}\leq t\right\}\\
T_{N\left(t\right)}\leq &t&<T_{N\left(t\right)+1},
\end{eqnarray*}

adem\'as $N\left(T_{n}\right)=n$, y 

\begin{eqnarray*}
N\left(t\right)=\max\left\{n:T_{n}\leq t\right\}=\min\left\{n:T_{n+1}>t\right\}
\end{eqnarray*}

Por propiedades de la convoluci\'on se sabe que

\begin{eqnarray*}
P\left\{T_{n}\leq t\right\}=F^{n\star}\left(t\right)
\end{eqnarray*}
que es la $n$-\'esima convoluci\'on de $F$. Entonces 

\begin{eqnarray*}
\left\{N\left(t\right)\geq n\right\}&=&\left\{T_{n}\leq t\right\}\\
P\left\{N\left(t\right)\leq n\right\}&=&1-F^{\left(n+1\right)\star}\left(t\right)
\end{eqnarray*}

Adem\'as usando el hecho de que $\esp\left[N\left(t\right)\right]=\sum_{n=1}^{\infty}P\left\{N\left(t\right)\geq n\right\}$
se tiene que

\begin{eqnarray*}
\esp\left[N\left(t\right)\right]=\sum_{n=1}^{\infty}F^{n\star}\left(t\right)
\end{eqnarray*}

\begin{Prop}
Para cada $t\geq0$, la funci\'on generadora de momentos $\esp\left[e^{\alpha N\left(t\right)}\right]$ existe para alguna $\alpha$ en una vecindad del 0, y de aqu\'i que $\esp\left[N\left(t\right)^{m}\right]<\infty$, para $m\geq1$.
\end{Prop}


\begin{Note}
Si el primer tiempo de renovaci\'on $\xi_{1}$ no tiene la misma distribuci\'on que el resto de las $\xi_{n}$, para $n\geq2$, a $N\left(t\right)$ se le llama Proceso de Renovaci\'on retardado, donde si $\xi$ tiene distribuci\'on $G$, entonces el tiempo $T_{n}$ de la $n$-\'esima renovaci\'on tiene distribuci\'on $G\star F^{\left(n-1\right)\star}\left(t\right)$
\end{Note}


\begin{Teo}
Para una constante $\mu\leq\infty$ ( o variable aleatoria), las siguientes expresiones son equivalentes:

\begin{eqnarray}
lim_{n\rightarrow\infty}n^{-1}T_{n}&=&\mu,\textrm{ c.s.}\\
lim_{t\rightarrow\infty}t^{-1}N\left(t\right)&=&1/\mu,\textrm{ c.s.}
\end{eqnarray}
\end{Teo}


Es decir, $T_{n}$ satisface la Ley Fuerte de los Grandes N\'umeros s\'i y s\'olo s\'i $N\left/t\right)$ la cumple.


\begin{Coro}[Ley Fuerte de los Grandes N\'umeros para Procesos de Renovaci\'on]
Si $N\left(t\right)$ es un proceso de renovaci\'on cuyos tiempos de inter-renovaci\'on tienen media $\mu\leq\infty$, entonces
\begin{eqnarray}
t^{-1}N\left(t\right)\rightarrow 1/\mu,\textrm{ c.s. cuando }t\rightarrow\infty.
\end{eqnarray}

\end{Coro}


Considerar el proceso estoc\'astico de valores reales $\left\{Z\left(t\right):t\geq0\right\}$ en el mismo espacio de probabilidad que $N\left(t\right)$

\begin{Def}
Para el proceso $\left\{Z\left(t\right):t\geq0\right\}$ se define la fluctuaci\'on m\'axima de $Z\left(t\right)$ en el intervalo $\left(T_{n-1},T_{n}\right]$:
\begin{eqnarray*}
M_{n}=\sup_{T_{n-1}<t\leq T_{n}}|Z\left(t\right)-Z\left(T_{n-1}\right)|
\end{eqnarray*}
\end{Def}

\begin{Teo}
Sup\'ongase que $n^{-1}T_{n}\rightarrow\mu$ c.s. cuando $n\rightarrow\infty$, donde $\mu\leq\infty$ es una constante o variable aleatoria. Sea $a$ una constante o variable aleatoria que puede ser infinita cuando $\mu$ es finita, y considere las expresiones l\'imite:
\begin{eqnarray}
lim_{n\rightarrow\infty}n^{-1}Z\left(T_{n}\right)&=&a,\textrm{ c.s.}\\
lim_{t\rightarrow\infty}t^{-1}Z\left(t\right)&=&a/\mu,\textrm{ c.s.}
\end{eqnarray}
La segunda expresi\'on implica la primera. Conversamente, la primera implica la segunda si el proceso $Z\left(t\right)$ es creciente, o si $lim_{n\rightarrow\infty}n^{-1}M_{n}=0$ c.s.
\end{Teo}

\begin{Coro}
Si $N\left(t\right)$ es un proceso de renovaci\'on, y $\left(Z\left(T_{n}\right)-Z\left(T_{n-1}\right),M_{n}\right)$, para $n\geq1$, son variables aleatorias independientes e id\'enticamente distribuidas con media finita, entonces,
\begin{eqnarray}
lim_{t\rightarrow\infty}t^{-1}Z\left(t\right)\rightarrow\frac{\esp\left[Z\left(T_{1}\right)-Z\left(T_{0}\right)\right]}{\esp\left[T_{1}\right]},\textrm{ c.s. cuando  }t\rightarrow\infty.
\end{eqnarray}
\end{Coro}

%___________________________________________________________________________________________
%
%\subsection{Propiedades de los Procesos de Renovaci\'on}
%___________________________________________________________________________________________
%

Los tiempos $T_{n}$ est\'an relacionados con los conteos de $N\left(t\right)$ por

\begin{eqnarray*}
\left\{N\left(t\right)\geq n\right\}&=&\left\{T_{n}\leq t\right\}\\
T_{N\left(t\right)}\leq &t&<T_{N\left(t\right)+1},
\end{eqnarray*}

adem\'as $N\left(T_{n}\right)=n$, y 

\begin{eqnarray*}
N\left(t\right)=\max\left\{n:T_{n}\leq t\right\}=\min\left\{n:T_{n+1}>t\right\}
\end{eqnarray*}

Por propiedades de la convoluci\'on se sabe que

\begin{eqnarray*}
P\left\{T_{n}\leq t\right\}=F^{n\star}\left(t\right)
\end{eqnarray*}
que es la $n$-\'esima convoluci\'on de $F$. Entonces 

\begin{eqnarray*}
\left\{N\left(t\right)\geq n\right\}&=&\left\{T_{n}\leq t\right\}\\
P\left\{N\left(t\right)\leq n\right\}&=&1-F^{\left(n+1\right)\star}\left(t\right)
\end{eqnarray*}

Adem\'as usando el hecho de que $\esp\left[N\left(t\right)\right]=\sum_{n=1}^{\infty}P\left\{N\left(t\right)\geq n\right\}$
se tiene que

\begin{eqnarray*}
\esp\left[N\left(t\right)\right]=\sum_{n=1}^{\infty}F^{n\star}\left(t\right)
\end{eqnarray*}

\begin{Prop}
Para cada $t\geq0$, la funci\'on generadora de momentos $\esp\left[e^{\alpha N\left(t\right)}\right]$ existe para alguna $\alpha$ en una vecindad del 0, y de aqu\'i que $\esp\left[N\left(t\right)^{m}\right]<\infty$, para $m\geq1$.
\end{Prop}


\begin{Note}
Si el primer tiempo de renovaci\'on $\xi_{1}$ no tiene la misma distribuci\'on que el resto de las $\xi_{n}$, para $n\geq2$, a $N\left(t\right)$ se le llama Proceso de Renovaci\'on retardado, donde si $\xi$ tiene distribuci\'on $G$, entonces el tiempo $T_{n}$ de la $n$-\'esima renovaci\'on tiene distribuci\'on $G\star F^{\left(n-1\right)\star}\left(t\right)$
\end{Note}


\begin{Teo}
Para una constante $\mu\leq\infty$ ( o variable aleatoria), las siguientes expresiones son equivalentes:

\begin{eqnarray}
lim_{n\rightarrow\infty}n^{-1}T_{n}&=&\mu,\textrm{ c.s.}\\
lim_{t\rightarrow\infty}t^{-1}N\left(t\right)&=&1/\mu,\textrm{ c.s.}
\end{eqnarray}
\end{Teo}


Es decir, $T_{n}$ satisface la Ley Fuerte de los Grandes N\'umeros s\'i y s\'olo s\'i $N\left/t\right)$ la cumple.


\begin{Coro}[Ley Fuerte de los Grandes N\'umeros para Procesos de Renovaci\'on]
Si $N\left(t\right)$ es un proceso de renovaci\'on cuyos tiempos de inter-renovaci\'on tienen media $\mu\leq\infty$, entonces
\begin{eqnarray}
t^{-1}N\left(t\right)\rightarrow 1/\mu,\textrm{ c.s. cuando }t\rightarrow\infty.
\end{eqnarray}

\end{Coro}


Considerar el proceso estoc\'astico de valores reales $\left\{Z\left(t\right):t\geq0\right\}$ en el mismo espacio de probabilidad que $N\left(t\right)$

\begin{Def}
Para el proceso $\left\{Z\left(t\right):t\geq0\right\}$ se define la fluctuaci\'on m\'axima de $Z\left(t\right)$ en el intervalo $\left(T_{n-1},T_{n}\right]$:
\begin{eqnarray*}
M_{n}=\sup_{T_{n-1}<t\leq T_{n}}|Z\left(t\right)-Z\left(T_{n-1}\right)|
\end{eqnarray*}
\end{Def}

\begin{Teo}
Sup\'ongase que $n^{-1}T_{n}\rightarrow\mu$ c.s. cuando $n\rightarrow\infty$, donde $\mu\leq\infty$ es una constante o variable aleatoria. Sea $a$ una constante o variable aleatoria que puede ser infinita cuando $\mu$ es finita, y considere las expresiones l\'imite:
\begin{eqnarray}
lim_{n\rightarrow\infty}n^{-1}Z\left(T_{n}\right)&=&a,\textrm{ c.s.}\\
lim_{t\rightarrow\infty}t^{-1}Z\left(t\right)&=&a/\mu,\textrm{ c.s.}
\end{eqnarray}
La segunda expresi\'on implica la primera. Conversamente, la primera implica la segunda si el proceso $Z\left(t\right)$ es creciente, o si $lim_{n\rightarrow\infty}n^{-1}M_{n}=0$ c.s.
\end{Teo}

\begin{Coro}
Si $N\left(t\right)$ es un proceso de renovaci\'on, y $\left(Z\left(T_{n}\right)-Z\left(T_{n-1}\right),M_{n}\right)$, para $n\geq1$, son variables aleatorias independientes e id\'enticamente distribuidas con media finita, entonces,
\begin{eqnarray}
lim_{t\rightarrow\infty}t^{-1}Z\left(t\right)\rightarrow\frac{\esp\left[Z\left(T_{1}\right)-Z\left(T_{0}\right)\right]}{\esp\left[T_{1}\right]},\textrm{ c.s. cuando  }t\rightarrow\infty.
\end{eqnarray}
\end{Coro}
%___________________________________________________________________________________________
%
%\subsection{Propiedades de los Procesos de Renovaci\'on}
%___________________________________________________________________________________________
%

Los tiempos $T_{n}$ est\'an relacionados con los conteos de $N\left(t\right)$ por

\begin{eqnarray*}
\left\{N\left(t\right)\geq n\right\}&=&\left\{T_{n}\leq t\right\}\\
T_{N\left(t\right)}\leq &t&<T_{N\left(t\right)+1},
\end{eqnarray*}

adem\'as $N\left(T_{n}\right)=n$, y 

\begin{eqnarray*}
N\left(t\right)=\max\left\{n:T_{n}\leq t\right\}=\min\left\{n:T_{n+1}>t\right\}
\end{eqnarray*}

Por propiedades de la convoluci\'on se sabe que

\begin{eqnarray*}
P\left\{T_{n}\leq t\right\}=F^{n\star}\left(t\right)
\end{eqnarray*}
que es la $n$-\'esima convoluci\'on de $F$. Entonces 

\begin{eqnarray*}
\left\{N\left(t\right)\geq n\right\}&=&\left\{T_{n}\leq t\right\}\\
P\left\{N\left(t\right)\leq n\right\}&=&1-F^{\left(n+1\right)\star}\left(t\right)
\end{eqnarray*}

Adem\'as usando el hecho de que $\esp\left[N\left(t\right)\right]=\sum_{n=1}^{\infty}P\left\{N\left(t\right)\geq n\right\}$
se tiene que

\begin{eqnarray*}
\esp\left[N\left(t\right)\right]=\sum_{n=1}^{\infty}F^{n\star}\left(t\right)
\end{eqnarray*}

\begin{Prop}
Para cada $t\geq0$, la funci\'on generadora de momentos $\esp\left[e^{\alpha N\left(t\right)}\right]$ existe para alguna $\alpha$ en una vecindad del 0, y de aqu\'i que $\esp\left[N\left(t\right)^{m}\right]<\infty$, para $m\geq1$.
\end{Prop}


\begin{Note}
Si el primer tiempo de renovaci\'on $\xi_{1}$ no tiene la misma distribuci\'on que el resto de las $\xi_{n}$, para $n\geq2$, a $N\left(t\right)$ se le llama Proceso de Renovaci\'on retardado, donde si $\xi$ tiene distribuci\'on $G$, entonces el tiempo $T_{n}$ de la $n$-\'esima renovaci\'on tiene distribuci\'on $G\star F^{\left(n-1\right)\star}\left(t\right)$
\end{Note}


\begin{Teo}
Para una constante $\mu\leq\infty$ ( o variable aleatoria), las siguientes expresiones son equivalentes:

\begin{eqnarray}
lim_{n\rightarrow\infty}n^{-1}T_{n}&=&\mu,\textrm{ c.s.}\\
lim_{t\rightarrow\infty}t^{-1}N\left(t\right)&=&1/\mu,\textrm{ c.s.}
\end{eqnarray}
\end{Teo}


Es decir, $T_{n}$ satisface la Ley Fuerte de los Grandes N\'umeros s\'i y s\'olo s\'i $N\left/t\right)$ la cumple.


\begin{Coro}[Ley Fuerte de los Grandes N\'umeros para Procesos de Renovaci\'on]
Si $N\left(t\right)$ es un proceso de renovaci\'on cuyos tiempos de inter-renovaci\'on tienen media $\mu\leq\infty$, entonces
\begin{eqnarray}
t^{-1}N\left(t\right)\rightarrow 1/\mu,\textrm{ c.s. cuando }t\rightarrow\infty.
\end{eqnarray}

\end{Coro}


Considerar el proceso estoc\'astico de valores reales $\left\{Z\left(t\right):t\geq0\right\}$ en el mismo espacio de probabilidad que $N\left(t\right)$

\begin{Def}
Para el proceso $\left\{Z\left(t\right):t\geq0\right\}$ se define la fluctuaci\'on m\'axima de $Z\left(t\right)$ en el intervalo $\left(T_{n-1},T_{n}\right]$:
\begin{eqnarray*}
M_{n}=\sup_{T_{n-1}<t\leq T_{n}}|Z\left(t\right)-Z\left(T_{n-1}\right)|
\end{eqnarray*}
\end{Def}

\begin{Teo}
Sup\'ongase que $n^{-1}T_{n}\rightarrow\mu$ c.s. cuando $n\rightarrow\infty$, donde $\mu\leq\infty$ es una constante o variable aleatoria. Sea $a$ una constante o variable aleatoria que puede ser infinita cuando $\mu$ es finita, y considere las expresiones l\'imite:
\begin{eqnarray}
lim_{n\rightarrow\infty}n^{-1}Z\left(T_{n}\right)&=&a,\textrm{ c.s.}\\
lim_{t\rightarrow\infty}t^{-1}Z\left(t\right)&=&a/\mu,\textrm{ c.s.}
\end{eqnarray}
La segunda expresi\'on implica la primera. Conversamente, la primera implica la segunda si el proceso $Z\left(t\right)$ es creciente, o si $lim_{n\rightarrow\infty}n^{-1}M_{n}=0$ c.s.
\end{Teo}

\begin{Coro}
Si $N\left(t\right)$ es un proceso de renovaci\'on, y $\left(Z\left(T_{n}\right)-Z\left(T_{n-1}\right),M_{n}\right)$, para $n\geq1$, son variables aleatorias independientes e id\'enticamente distribuidas con media finita, entonces,
\begin{eqnarray}
lim_{t\rightarrow\infty}t^{-1}Z\left(t\right)\rightarrow\frac{\esp\left[Z\left(T_{1}\right)-Z\left(T_{0}\right)\right]}{\esp\left[T_{1}\right]},\textrm{ c.s. cuando  }t\rightarrow\infty.
\end{eqnarray}
\end{Coro}


%___________________________________________________________________________________________
%
%\subsection{Funci\'on de Renovaci\'on}
%___________________________________________________________________________________________
%


\begin{Def}
Sea $h\left(t\right)$ funci\'on de valores reales en $\rea$ acotada en intervalos finitos e igual a cero para $t<0$ La ecuaci\'on de renovaci\'on para $h\left(t\right)$ y la distribuci\'on $F$ es

\begin{eqnarray}%\label{Ec.Renovacion}
H\left(t\right)=h\left(t\right)+\int_{\left[0,t\right]}H\left(t-s\right)dF\left(s\right)\textrm{,    }t\geq0,
\end{eqnarray}
donde $H\left(t\right)$ es una funci\'on de valores reales. Esto es $H=h+F\star H$. Decimos que $H\left(t\right)$ es soluci\'on de esta ecuaci\'on si satisface la ecuaci\'on, y es acotada en intervalos finitos e iguales a cero para $t<0$.
\end{Def}

\begin{Prop}
La funci\'on $U\star h\left(t\right)$ es la \'unica soluci\'on de la ecuaci\'on de renovaci\'on (\ref{Ec.Renovacion}).
\end{Prop}

\begin{Teo}[Teorema Renovaci\'on Elemental]
\begin{eqnarray*}
t^{-1}U\left(t\right)\rightarrow 1/\mu\textrm{,    cuando }t\rightarrow\infty.
\end{eqnarray*}
\end{Teo}

%___________________________________________________________________________________________
%
%\subsection{Funci\'on de Renovaci\'on}
%___________________________________________________________________________________________
%


Sup\'ongase que $N\left(t\right)$ es un proceso de renovaci\'on con distribuci\'on $F$ con media finita $\mu$.

\begin{Def}
La funci\'on de renovaci\'on asociada con la distribuci\'on $F$, del proceso $N\left(t\right)$, es
\begin{eqnarray*}
U\left(t\right)=\sum_{n=1}^{\infty}F^{n\star}\left(t\right),\textrm{   }t\geq0,
\end{eqnarray*}
donde $F^{0\star}\left(t\right)=\indora\left(t\geq0\right)$.
\end{Def}


\begin{Prop}
Sup\'ongase que la distribuci\'on de inter-renovaci\'on $F$ tiene densidad $f$. Entonces $U\left(t\right)$ tambi\'en tiene densidad, para $t>0$, y es $U^{'}\left(t\right)=\sum_{n=0}^{\infty}f^{n\star}\left(t\right)$. Adem\'as
\begin{eqnarray*}
\prob\left\{N\left(t\right)>N\left(t-\right)\right\}=0\textrm{,   }t\geq0.
\end{eqnarray*}
\end{Prop}

\begin{Def}
La Transformada de Laplace-Stieljes de $F$ est\'a dada por

\begin{eqnarray*}
\hat{F}\left(\alpha\right)=\int_{\rea_{+}}e^{-\alpha t}dF\left(t\right)\textrm{,  }\alpha\geq0.
\end{eqnarray*}
\end{Def}

Entonces

\begin{eqnarray*}
\hat{U}\left(\alpha\right)=\sum_{n=0}^{\infty}\hat{F^{n\star}}\left(\alpha\right)=\sum_{n=0}^{\infty}\hat{F}\left(\alpha\right)^{n}=\frac{1}{1-\hat{F}\left(\alpha\right)}.
\end{eqnarray*}


\begin{Prop}
La Transformada de Laplace $\hat{U}\left(\alpha\right)$ y $\hat{F}\left(\alpha\right)$ determina una a la otra de manera \'unica por la relaci\'on $\hat{U}\left(\alpha\right)=\frac{1}{1-\hat{F}\left(\alpha\right)}$.
\end{Prop}


\begin{Note}
Un proceso de renovaci\'on $N\left(t\right)$ cuyos tiempos de inter-renovaci\'on tienen media finita, es un proceso Poisson con tasa $\lambda$ si y s\'olo s\'i $\esp\left[U\left(t\right)\right]=\lambda t$, para $t\geq0$.
\end{Note}


\begin{Teo}
Sea $N\left(t\right)$ un proceso puntual simple con puntos de localizaci\'on $T_{n}$ tal que $\eta\left(t\right)=\esp\left[N\left(\right)\right]$ es finita para cada $t$. Entonces para cualquier funci\'on $f:\rea_{+}\rightarrow\rea$,
\begin{eqnarray*}
\esp\left[\sum_{n=1}^{N\left(\right)}f\left(T_{n}\right)\right]=\int_{\left(0,t\right]}f\left(s\right)d\eta\left(s\right)\textrm{,  }t\geq0,
\end{eqnarray*}
suponiendo que la integral exista. Adem\'as si $X_{1},X_{2},\ldots$ son variables aleatorias definidas en el mismo espacio de probabilidad que el proceso $N\left(t\right)$ tal que $\esp\left[X_{n}|T_{n}=s\right]=f\left(s\right)$, independiente de $n$. Entonces
\begin{eqnarray*}
\esp\left[\sum_{n=1}^{N\left(t\right)}X_{n}\right]=\int_{\left(0,t\right]}f\left(s\right)d\eta\left(s\right)\textrm{,  }t\geq0,
\end{eqnarray*} 
suponiendo que la integral exista. 
\end{Teo}

\begin{Coro}[Identidad de Wald para Renovaciones]
Para el proceso de renovaci\'on $N\left(t\right)$,
\begin{eqnarray*}
\esp\left[T_{N\left(t\right)+1}\right]=\mu\esp\left[N\left(t\right)+1\right]\textrm{,  }t\geq0,
\end{eqnarray*}  
\end{Coro}

%______________________________________________________________________
%\subsection{Procesos de Renovaci\'on}
%______________________________________________________________________

\begin{Def}%\label{Def.Tn}
Sean $0\leq T_{1}\leq T_{2}\leq \ldots$ son tiempos aleatorios infinitos en los cuales ocurren ciertos eventos. El n\'umero de tiempos $T_{n}$ en el intervalo $\left[0,t\right)$ es

\begin{eqnarray}
N\left(t\right)=\sum_{n=1}^{\infty}\indora\left(T_{n}\leq t\right),
\end{eqnarray}
para $t\geq0$.
\end{Def}

Si se consideran los puntos $T_{n}$ como elementos de $\rea_{+}$, y $N\left(t\right)$ es el n\'umero de puntos en $\rea$. El proceso denotado por $\left\{N\left(t\right):t\geq0\right\}$, denotado por $N\left(t\right)$, es un proceso puntual en $\rea_{+}$. Los $T_{n}$ son los tiempos de ocurrencia, el proceso puntual $N\left(t\right)$ es simple si su n\'umero de ocurrencias son distintas: $0<T_{1}<T_{2}<\ldots$ casi seguramente.

\begin{Def}
Un proceso puntual $N\left(t\right)$ es un proceso de renovaci\'on si los tiempos de interocurrencia $\xi_{n}=T_{n}-T_{n-1}$, para $n\geq1$, son independientes e identicamente distribuidos con distribuci\'on $F$, donde $F\left(0\right)=0$ y $T_{0}=0$. Los $T_{n}$ son llamados tiempos de renovaci\'on, referente a la independencia o renovaci\'on de la informaci\'on estoc\'astica en estos tiempos. Los $\xi_{n}$ son los tiempos de inter-renovaci\'on, y $N\left(t\right)$ es el n\'umero de renovaciones en el intervalo $\left[0,t\right)$
\end{Def}


\begin{Note}
Para definir un proceso de renovaci\'on para cualquier contexto, solamente hay que especificar una distribuci\'on $F$, con $F\left(0\right)=0$, para los tiempos de inter-renovaci\'on. La funci\'on $F$ en turno degune las otra variables aleatorias. De manera formal, existe un espacio de probabilidad y una sucesi\'on de variables aleatorias $\xi_{1},\xi_{2},\ldots$ definidas en este con distribuci\'on $F$. Entonces las otras cantidades son $T_{n}=\sum_{k=1}^{n}\xi_{k}$ y $N\left(t\right)=\sum_{n=1}^{\infty}\indora\left(T_{n}\leq t\right)$, donde $T_{n}\rightarrow\infty$ casi seguramente por la Ley Fuerte de los Grandes Números.
\end{Note}

%___________________________________________________________________________________________
%
%\subsection{Renewal and Regenerative Processes: Serfozo\cite{Serfozo}}
%___________________________________________________________________________________________
%
\begin{Def}%\label{Def.Tn}
Sean $0\leq T_{1}\leq T_{2}\leq \ldots$ son tiempos aleatorios infinitos en los cuales ocurren ciertos eventos. El n\'umero de tiempos $T_{n}$ en el intervalo $\left[0,t\right)$ es

\begin{eqnarray}
N\left(t\right)=\sum_{n=1}^{\infty}\indora\left(T_{n}\leq t\right),
\end{eqnarray}
para $t\geq0$.
\end{Def}

Si se consideran los puntos $T_{n}$ como elementos de $\rea_{+}$, y $N\left(t\right)$ es el n\'umero de puntos en $\rea$. El proceso denotado por $\left\{N\left(t\right):t\geq0\right\}$, denotado por $N\left(t\right)$, es un proceso puntual en $\rea_{+}$. Los $T_{n}$ son los tiempos de ocurrencia, el proceso puntual $N\left(t\right)$ es simple si su n\'umero de ocurrencias son distintas: $0<T_{1}<T_{2}<\ldots$ casi seguramente.

\begin{Def}
Un proceso puntual $N\left(t\right)$ es un proceso de renovaci\'on si los tiempos de interocurrencia $\xi_{n}=T_{n}-T_{n-1}$, para $n\geq1$, son independientes e identicamente distribuidos con distribuci\'on $F$, donde $F\left(0\right)=0$ y $T_{0}=0$. Los $T_{n}$ son llamados tiempos de renovaci\'on, referente a la independencia o renovaci\'on de la informaci\'on estoc\'astica en estos tiempos. Los $\xi_{n}$ son los tiempos de inter-renovaci\'on, y $N\left(t\right)$ es el n\'umero de renovaciones en el intervalo $\left[0,t\right)$
\end{Def}


\begin{Note}
Para definir un proceso de renovaci\'on para cualquier contexto, solamente hay que especificar una distribuci\'on $F$, con $F\left(0\right)=0$, para los tiempos de inter-renovaci\'on. La funci\'on $F$ en turno degune las otra variables aleatorias. De manera formal, existe un espacio de probabilidad y una sucesi\'on de variables aleatorias $\xi_{1},\xi_{2},\ldots$ definidas en este con distribuci\'on $F$. Entonces las otras cantidades son $T_{n}=\sum_{k=1}^{n}\xi_{k}$ y $N\left(t\right)=\sum_{n=1}^{\infty}\indora\left(T_{n}\leq t\right)$, donde $T_{n}\rightarrow\infty$ casi seguramente por la Ley Fuerte de los Grandes N\'umeros.
\end{Note}







Los tiempos $T_{n}$ est\'an relacionados con los conteos de $N\left(t\right)$ por

\begin{eqnarray*}
\left\{N\left(t\right)\geq n\right\}&=&\left\{T_{n}\leq t\right\}\\
T_{N\left(t\right)}\leq &t&<T_{N\left(t\right)+1},
\end{eqnarray*}

adem\'as $N\left(T_{n}\right)=n$, y 

\begin{eqnarray*}
N\left(t\right)=\max\left\{n:T_{n}\leq t\right\}=\min\left\{n:T_{n+1}>t\right\}
\end{eqnarray*}

Por propiedades de la convoluci\'on se sabe que

\begin{eqnarray*}
P\left\{T_{n}\leq t\right\}=F^{n\star}\left(t\right)
\end{eqnarray*}
que es la $n$-\'esima convoluci\'on de $F$. Entonces 

\begin{eqnarray*}
\left\{N\left(t\right)\geq n\right\}&=&\left\{T_{n}\leq t\right\}\\
P\left\{N\left(t\right)\leq n\right\}&=&1-F^{\left(n+1\right)\star}\left(t\right)
\end{eqnarray*}

Adem\'as usando el hecho de que $\esp\left[N\left(t\right)\right]=\sum_{n=1}^{\infty}P\left\{N\left(t\right)\geq n\right\}$
se tiene que

\begin{eqnarray*}
\esp\left[N\left(t\right)\right]=\sum_{n=1}^{\infty}F^{n\star}\left(t\right)
\end{eqnarray*}

\begin{Prop}
Para cada $t\geq0$, la funci\'on generadora de momentos $\esp\left[e^{\alpha N\left(t\right)}\right]$ existe para alguna $\alpha$ en una vecindad del 0, y de aqu\'i que $\esp\left[N\left(t\right)^{m}\right]<\infty$, para $m\geq1$.
\end{Prop}

\begin{Ejem}[\textbf{Proceso Poisson}]

Suponga que se tienen tiempos de inter-renovaci\'on \textit{i.i.d.} del proceso de renovaci\'on $N\left(t\right)$ tienen distribuci\'on exponencial $F\left(t\right)=q-e^{-\lambda t}$ con tasa $\lambda$. Entonces $N\left(t\right)$ es un proceso Poisson con tasa $\lambda$.

\end{Ejem}


\begin{Note}
Si el primer tiempo de renovaci\'on $\xi_{1}$ no tiene la misma distribuci\'on que el resto de las $\xi_{n}$, para $n\geq2$, a $N\left(t\right)$ se le llama Proceso de Renovaci\'on retardado, donde si $\xi$ tiene distribuci\'on $G$, entonces el tiempo $T_{n}$ de la $n$-\'esima renovaci\'on tiene distribuci\'on $G\star F^{\left(n-1\right)\star}\left(t\right)$
\end{Note}


\begin{Teo}
Para una constante $\mu\leq\infty$ ( o variable aleatoria), las siguientes expresiones son equivalentes:

\begin{eqnarray}
lim_{n\rightarrow\infty}n^{-1}T_{n}&=&\mu,\textrm{ c.s.}\\
lim_{t\rightarrow\infty}t^{-1}N\left(t\right)&=&1/\mu,\textrm{ c.s.}
\end{eqnarray}
\end{Teo}


Es decir, $T_{n}$ satisface la Ley Fuerte de los Grandes N\'umeros s\'i y s\'olo s\'i $N\left/t\right)$ la cumple.


\begin{Coro}[Ley Fuerte de los Grandes N\'umeros para Procesos de Renovaci\'on]
Si $N\left(t\right)$ es un proceso de renovaci\'on cuyos tiempos de inter-renovaci\'on tienen media $\mu\leq\infty$, entonces
\begin{eqnarray}
t^{-1}N\left(t\right)\rightarrow 1/\mu,\textrm{ c.s. cuando }t\rightarrow\infty.
\end{eqnarray}

\end{Coro}


Considerar el proceso estoc\'astico de valores reales $\left\{Z\left(t\right):t\geq0\right\}$ en el mismo espacio de probabilidad que $N\left(t\right)$

\begin{Def}
Para el proceso $\left\{Z\left(t\right):t\geq0\right\}$ se define la fluctuaci\'on m\'axima de $Z\left(t\right)$ en el intervalo $\left(T_{n-1},T_{n}\right]$:
\begin{eqnarray*}
M_{n}=\sup_{T_{n-1}<t\leq T_{n}}|Z\left(t\right)-Z\left(T_{n-1}\right)|
\end{eqnarray*}
\end{Def}

\begin{Teo}
Sup\'ongase que $n^{-1}T_{n}\rightarrow\mu$ c.s. cuando $n\rightarrow\infty$, donde $\mu\leq\infty$ es una constante o variable aleatoria. Sea $a$ una constante o variable aleatoria que puede ser infinita cuando $\mu$ es finita, y considere las expresiones l\'imite:
\begin{eqnarray}
lim_{n\rightarrow\infty}n^{-1}Z\left(T_{n}\right)&=&a,\textrm{ c.s.}\\
lim_{t\rightarrow\infty}t^{-1}Z\left(t\right)&=&a/\mu,\textrm{ c.s.}
\end{eqnarray}
La segunda expresi\'on implica la primera. Conversamente, la primera implica la segunda si el proceso $Z\left(t\right)$ es creciente, o si $lim_{n\rightarrow\infty}n^{-1}M_{n}=0$ c.s.
\end{Teo}

\begin{Coro}
Si $N\left(t\right)$ es un proceso de renovaci\'on, y $\left(Z\left(T_{n}\right)-Z\left(T_{n-1}\right),M_{n}\right)$, para $n\geq1$, son variables aleatorias independientes e id\'enticamente distribuidas con media finita, entonces,
\begin{eqnarray}
lim_{t\rightarrow\infty}t^{-1}Z\left(t\right)\rightarrow\frac{\esp\left[Z\left(T_{1}\right)-Z\left(T_{0}\right)\right]}{\esp\left[T_{1}\right]},\textrm{ c.s. cuando  }t\rightarrow\infty.
\end{eqnarray}
\end{Coro}


Sup\'ongase que $N\left(t\right)$ es un proceso de renovaci\'on con distribuci\'on $F$ con media finita $\mu$.

\begin{Def}
La funci\'on de renovaci\'on asociada con la distribuci\'on $F$, del proceso $N\left(t\right)$, es
\begin{eqnarray*}
U\left(t\right)=\sum_{n=1}^{\infty}F^{n\star}\left(t\right),\textrm{   }t\geq0,
\end{eqnarray*}
donde $F^{0\star}\left(t\right)=\indora\left(t\geq0\right)$.
\end{Def}


\begin{Prop}
Sup\'ongase que la distribuci\'on de inter-renovaci\'on $F$ tiene densidad $f$. Entonces $U\left(t\right)$ tambi\'en tiene densidad, para $t>0$, y es $U^{'}\left(t\right)=\sum_{n=0}^{\infty}f^{n\star}\left(t\right)$. Adem\'as
\begin{eqnarray*}
\prob\left\{N\left(t\right)>N\left(t-\right)\right\}=0\textrm{,   }t\geq0.
\end{eqnarray*}
\end{Prop}

\begin{Def}
La Transformada de Laplace-Stieljes de $F$ est\'a dada por

\begin{eqnarray*}
\hat{F}\left(\alpha\right)=\int_{\rea_{+}}e^{-\alpha t}dF\left(t\right)\textrm{,  }\alpha\geq0.
\end{eqnarray*}
\end{Def}

Entonces

\begin{eqnarray*}
\hat{U}\left(\alpha\right)=\sum_{n=0}^{\infty}\hat{F^{n\star}}\left(\alpha\right)=\sum_{n=0}^{\infty}\hat{F}\left(\alpha\right)^{n}=\frac{1}{1-\hat{F}\left(\alpha\right)}.
\end{eqnarray*}


\begin{Prop}
La Transformada de Laplace $\hat{U}\left(\alpha\right)$ y $\hat{F}\left(\alpha\right)$ determina una a la otra de manera \'unica por la relaci\'on $\hat{U}\left(\alpha\right)=\frac{1}{1-\hat{F}\left(\alpha\right)}$.
\end{Prop}


\begin{Note}
Un proceso de renovaci\'on $N\left(t\right)$ cuyos tiempos de inter-renovaci\'on tienen media finita, es un proceso Poisson con tasa $\lambda$ si y s\'olo s\'i $\esp\left[U\left(t\right)\right]=\lambda t$, para $t\geq0$.
\end{Note}


\begin{Teo}
Sea $N\left(t\right)$ un proceso puntual simple con puntos de localizaci\'on $T_{n}$ tal que $\eta\left(t\right)=\esp\left[N\left(\right)\right]$ es finita para cada $t$. Entonces para cualquier funci\'on $f:\rea_{+}\rightarrow\rea$,
\begin{eqnarray*}
\esp\left[\sum_{n=1}^{N\left(\right)}f\left(T_{n}\right)\right]=\int_{\left(0,t\right]}f\left(s\right)d\eta\left(s\right)\textrm{,  }t\geq0,
\end{eqnarray*}
suponiendo que la integral exista. Adem\'as si $X_{1},X_{2},\ldots$ son variables aleatorias definidas en el mismo espacio de probabilidad que el proceso $N\left(t\right)$ tal que $\esp\left[X_{n}|T_{n}=s\right]=f\left(s\right)$, independiente de $n$. Entonces
\begin{eqnarray*}
\esp\left[\sum_{n=1}^{N\left(t\right)}X_{n}\right]=\int_{\left(0,t\right]}f\left(s\right)d\eta\left(s\right)\textrm{,  }t\geq0,
\end{eqnarray*} 
suponiendo que la integral exista. 
\end{Teo}

\begin{Coro}[Identidad de Wald para Renovaciones]
Para el proceso de renovaci\'on $N\left(t\right)$,
\begin{eqnarray*}
\esp\left[T_{N\left(t\right)+1}\right]=\mu\esp\left[N\left(t\right)+1\right]\textrm{,  }t\geq0,
\end{eqnarray*}  
\end{Coro}


\begin{Def}
Sea $h\left(t\right)$ funci\'on de valores reales en $\rea$ acotada en intervalos finitos e igual a cero para $t<0$ La ecuaci\'on de renovaci\'on para $h\left(t\right)$ y la distribuci\'on $F$ es

\begin{eqnarray}%\label{Ec.Renovacion}
H\left(t\right)=h\left(t\right)+\int_{\left[0,t\right]}H\left(t-s\right)dF\left(s\right)\textrm{,    }t\geq0,
\end{eqnarray}
donde $H\left(t\right)$ es una funci\'on de valores reales. Esto es $H=h+F\star H$. Decimos que $H\left(t\right)$ es soluci\'on de esta ecuaci\'on si satisface la ecuaci\'on, y es acotada en intervalos finitos e iguales a cero para $t<0$.
\end{Def}

\begin{Prop}
La funci\'on $U\star h\left(t\right)$ es la \'unica soluci\'on de la ecuaci\'on de renovaci\'on (\ref{Ec.Renovacion}).
\end{Prop}

\begin{Teo}[Teorema Renovaci\'on Elemental]
\begin{eqnarray*}
t^{-1}U\left(t\right)\rightarrow 1/\mu\textrm{,    cuando }t\rightarrow\infty.
\end{eqnarray*}
\end{Teo}



Sup\'ongase que $N\left(t\right)$ es un proceso de renovaci\'on con distribuci\'on $F$ con media finita $\mu$.

\begin{Def}
La funci\'on de renovaci\'on asociada con la distribuci\'on $F$, del proceso $N\left(t\right)$, es
\begin{eqnarray*}
U\left(t\right)=\sum_{n=1}^{\infty}F^{n\star}\left(t\right),\textrm{   }t\geq0,
\end{eqnarray*}
donde $F^{0\star}\left(t\right)=\indora\left(t\geq0\right)$.
\end{Def}


\begin{Prop}
Sup\'ongase que la distribuci\'on de inter-renovaci\'on $F$ tiene densidad $f$. Entonces $U\left(t\right)$ tambi\'en tiene densidad, para $t>0$, y es $U^{'}\left(t\right)=\sum_{n=0}^{\infty}f^{n\star}\left(t\right)$. Adem\'as
\begin{eqnarray*}
\prob\left\{N\left(t\right)>N\left(t-\right)\right\}=0\textrm{,   }t\geq0.
\end{eqnarray*}
\end{Prop}

\begin{Def}
La Transformada de Laplace-Stieljes de $F$ est\'a dada por

\begin{eqnarray*}
\hat{F}\left(\alpha\right)=\int_{\rea_{+}}e^{-\alpha t}dF\left(t\right)\textrm{,  }\alpha\geq0.
\end{eqnarray*}
\end{Def}

Entonces

\begin{eqnarray*}
\hat{U}\left(\alpha\right)=\sum_{n=0}^{\infty}\hat{F^{n\star}}\left(\alpha\right)=\sum_{n=0}^{\infty}\hat{F}\left(\alpha\right)^{n}=\frac{1}{1-\hat{F}\left(\alpha\right)}.
\end{eqnarray*}


\begin{Prop}
La Transformada de Laplace $\hat{U}\left(\alpha\right)$ y $\hat{F}\left(\alpha\right)$ determina una a la otra de manera \'unica por la relaci\'on $\hat{U}\left(\alpha\right)=\frac{1}{1-\hat{F}\left(\alpha\right)}$.
\end{Prop}


\begin{Note}
Un proceso de renovaci\'on $N\left(t\right)$ cuyos tiempos de inter-renovaci\'on tienen media finita, es un proceso Poisson con tasa $\lambda$ si y s\'olo s\'i $\esp\left[U\left(t\right)\right]=\lambda t$, para $t\geq0$.
\end{Note}


\begin{Teo}
Sea $N\left(t\right)$ un proceso puntual simple con puntos de localizaci\'on $T_{n}$ tal que $\eta\left(t\right)=\esp\left[N\left(\right)\right]$ es finita para cada $t$. Entonces para cualquier funci\'on $f:\rea_{+}\rightarrow\rea$,
\begin{eqnarray*}
\esp\left[\sum_{n=1}^{N\left(\right)}f\left(T_{n}\right)\right]=\int_{\left(0,t\right]}f\left(s\right)d\eta\left(s\right)\textrm{,  }t\geq0,
\end{eqnarray*}
suponiendo que la integral exista. Adem\'as si $X_{1},X_{2},\ldots$ son variables aleatorias definidas en el mismo espacio de probabilidad que el proceso $N\left(t\right)$ tal que $\esp\left[X_{n}|T_{n}=s\right]=f\left(s\right)$, independiente de $n$. Entonces
\begin{eqnarray*}
\esp\left[\sum_{n=1}^{N\left(t\right)}X_{n}\right]=\int_{\left(0,t\right]}f\left(s\right)d\eta\left(s\right)\textrm{,  }t\geq0,
\end{eqnarray*} 
suponiendo que la integral exista. 
\end{Teo}

\begin{Coro}[Identidad de Wald para Renovaciones]
Para el proceso de renovaci\'on $N\left(t\right)$,
\begin{eqnarray*}
\esp\left[T_{N\left(t\right)+1}\right]=\mu\esp\left[N\left(t\right)+1\right]\textrm{,  }t\geq0,
\end{eqnarray*}  
\end{Coro}


\begin{Def}
Sea $h\left(t\right)$ funci\'on de valores reales en $\rea$ acotada en intervalos finitos e igual a cero para $t<0$ La ecuaci\'on de renovaci\'on para $h\left(t\right)$ y la distribuci\'on $F$ es

\begin{eqnarray}%\label{Ec.Renovacion}
H\left(t\right)=h\left(t\right)+\int_{\left[0,t\right]}H\left(t-s\right)dF\left(s\right)\textrm{,    }t\geq0,
\end{eqnarray}
donde $H\left(t\right)$ es una funci\'on de valores reales. Esto es $H=h+F\star H$. Decimos que $H\left(t\right)$ es soluci\'on de esta ecuaci\'on si satisface la ecuaci\'on, y es acotada en intervalos finitos e iguales a cero para $t<0$.
\end{Def}

\begin{Prop}
La funci\'on $U\star h\left(t\right)$ es la \'unica soluci\'on de la ecuaci\'on de renovaci\'on (\ref{Ec.Renovacion}).
\end{Prop}

\begin{Teo}[Teorema Renovaci\'on Elemental]
\begin{eqnarray*}
t^{-1}U\left(t\right)\rightarrow 1/\mu\textrm{,    cuando }t\rightarrow\infty.
\end{eqnarray*}
\end{Teo}


\begin{Note} Una funci\'on $h:\rea_{+}\rightarrow\rea$ es Directamente Riemann Integrable en los siguientes casos:
\begin{itemize}
\item[a)] $h\left(t\right)\geq0$ es decreciente y Riemann Integrable.
\item[b)] $h$ es continua excepto posiblemente en un conjunto de Lebesgue de medida 0, y $|h\left(t\right)|\leq b\left(t\right)$, donde $b$ es DRI.
\end{itemize}
\end{Note}

\begin{Teo}[Teorema Principal de Renovaci\'on]
Si $F$ es no aritm\'etica y $h\left(t\right)$ es Directamente Riemann Integrable (DRI), entonces

\begin{eqnarray*}
lim_{t\rightarrow\infty}U\star h=\frac{1}{\mu}\int_{\rea_{+}}h\left(s\right)ds.
\end{eqnarray*}
\end{Teo}

\begin{Prop}
Cualquier funci\'on $H\left(t\right)$ acotada en intervalos finitos y que es 0 para $t<0$ puede expresarse como
\begin{eqnarray*}
H\left(t\right)=U\star h\left(t\right)\textrm{,  donde }h\left(t\right)=H\left(t\right)-F\star H\left(t\right)
\end{eqnarray*}
\end{Prop}

\begin{Def}
Un proceso estoc\'astico $X\left(t\right)$ es crudamente regenerativo en un tiempo aleatorio positivo $T$ si
\begin{eqnarray*}
\esp\left[X\left(T+t\right)|T\right]=\esp\left[X\left(t\right)\right]\textrm{, para }t\geq0,\end{eqnarray*}
y con las esperanzas anteriores finitas.
\end{Def}

\begin{Prop}
Sup\'ongase que $X\left(t\right)$ es un proceso crudamente regenerativo en $T$, que tiene distribuci\'on $F$. Si $\esp\left[X\left(t\right)\right]$ es acotado en intervalos finitos, entonces
\begin{eqnarray*}
\esp\left[X\left(t\right)\right]=U\star h\left(t\right)\textrm{,  donde }h\left(t\right)=\esp\left[X\left(t\right)\indora\left(T>t\right)\right].
\end{eqnarray*}
\end{Prop}

\begin{Teo}[Regeneraci\'on Cruda]
Sup\'ongase que $X\left(t\right)$ es un proceso con valores positivo crudamente regenerativo en $T$, y def\'inase $M=\sup\left\{|X\left(t\right)|:t\leq T\right\}$. Si $T$ es no aritm\'etico y $M$ y $MT$ tienen media finita, entonces
\begin{eqnarray*}
lim_{t\rightarrow\infty}\esp\left[X\left(t\right)\right]=\frac{1}{\mu}\int_{\rea_{+}}h\left(s\right)ds,
\end{eqnarray*}
donde $h\left(t\right)=\esp\left[X\left(t\right)\indora\left(T>t\right)\right]$.
\end{Teo}


\begin{Note} Una funci\'on $h:\rea_{+}\rightarrow\rea$ es Directamente Riemann Integrable en los siguientes casos:
\begin{itemize}
\item[a)] $h\left(t\right)\geq0$ es decreciente y Riemann Integrable.
\item[b)] $h$ es continua excepto posiblemente en un conjunto de Lebesgue de medida 0, y $|h\left(t\right)|\leq b\left(t\right)$, donde $b$ es DRI.
\end{itemize}
\end{Note}

\begin{Teo}[Teorema Principal de Renovaci\'on]
Si $F$ es no aritm\'etica y $h\left(t\right)$ es Directamente Riemann Integrable (DRI), entonces

\begin{eqnarray*}
lim_{t\rightarrow\infty}U\star h=\frac{1}{\mu}\int_{\rea_{+}}h\left(s\right)ds.
\end{eqnarray*}
\end{Teo}

\begin{Prop}
Cualquier funci\'on $H\left(t\right)$ acotada en intervalos finitos y que es 0 para $t<0$ puede expresarse como
\begin{eqnarray*}
H\left(t\right)=U\star h\left(t\right)\textrm{,  donde }h\left(t\right)=H\left(t\right)-F\star H\left(t\right)
\end{eqnarray*}
\end{Prop}

\begin{Def}
Un proceso estoc\'astico $X\left(t\right)$ es crudamente regenerativo en un tiempo aleatorio positivo $T$ si
\begin{eqnarray*}
\esp\left[X\left(T+t\right)|T\right]=\esp\left[X\left(t\right)\right]\textrm{, para }t\geq0,\end{eqnarray*}
y con las esperanzas anteriores finitas.
\end{Def}

\begin{Prop}
Sup\'ongase que $X\left(t\right)$ es un proceso crudamente regenerativo en $T$, que tiene distribuci\'on $F$. Si $\esp\left[X\left(t\right)\right]$ es acotado en intervalos finitos, entonces
\begin{eqnarray*}
\esp\left[X\left(t\right)\right]=U\star h\left(t\right)\textrm{,  donde }h\left(t\right)=\esp\left[X\left(t\right)\indora\left(T>t\right)\right].
\end{eqnarray*}
\end{Prop}

\begin{Teo}[Regeneraci\'on Cruda]
Sup\'ongase que $X\left(t\right)$ es un proceso con valores positivo crudamente regenerativo en $T$, y def\'inase $M=\sup\left\{|X\left(t\right)|:t\leq T\right\}$. Si $T$ es no aritm\'etico y $M$ y $MT$ tienen media finita, entonces
\begin{eqnarray*}
lim_{t\rightarrow\infty}\esp\left[X\left(t\right)\right]=\frac{1}{\mu}\int_{\rea_{+}}h\left(s\right)ds,
\end{eqnarray*}
donde $h\left(t\right)=\esp\left[X\left(t\right)\indora\left(T>t\right)\right]$.
\end{Teo}

\begin{Def}
Para el proceso $\left\{\left(N\left(t\right),X\left(t\right)\right):t\geq0\right\}$, sus trayectoria muestrales en el intervalo de tiempo $\left[T_{n-1},T_{n}\right)$ est\'an descritas por
\begin{eqnarray*}
\zeta_{n}=\left(\xi_{n},\left\{X\left(T_{n-1}+t\right):0\leq t<\xi_{n}\right\}\right)
\end{eqnarray*}
Este $\zeta_{n}$ es el $n$-\'esimo segmento del proceso. El proceso es regenerativo sobre los tiempos $T_{n}$ si sus segmentos $\zeta_{n}$ son independientes e id\'enticamennte distribuidos.
\end{Def}


\begin{Note}
Si $\tilde{X}\left(t\right)$ con espacio de estados $\tilde{S}$ es regenerativo sobre $T_{n}$, entonces $X\left(t\right)=f\left(\tilde{X}\left(t\right)\right)$ tambi\'en es regenerativo sobre $T_{n}$, para cualquier funci\'on $f:\tilde{S}\rightarrow S$.
\end{Note}

\begin{Note}
Los procesos regenerativos son crudamente regenerativos, pero no al rev\'es.
\end{Note}


\begin{Note}
Un proceso estoc\'astico a tiempo continuo o discreto es regenerativo si existe un proceso de renovaci\'on  tal que los segmentos del proceso entre tiempos de renovaci\'on sucesivos son i.i.d., es decir, para $\left\{X\left(t\right):t\geq0\right\}$ proceso estoc\'astico a tiempo continuo con espacio de estados $S$, espacio m\'etrico.
\end{Note}

Para $\left\{X\left(t\right):t\geq0\right\}$ Proceso Estoc\'astico a tiempo continuo con estado de espacios $S$, que es un espacio m\'etrico, con trayectorias continuas por la derecha y con l\'imites por la izquierda c.s. Sea $N\left(t\right)$ un proceso de renovaci\'on en $\rea_{+}$ definido en el mismo espacio de probabilidad que $X\left(t\right)$, con tiempos de renovaci\'on $T$ y tiempos de inter-renovaci\'on $\xi_{n}=T_{n}-T_{n-1}$, con misma distribuci\'on $F$ de media finita $\mu$.



\begin{Def}
Para el proceso $\left\{\left(N\left(t\right),X\left(t\right)\right):t\geq0\right\}$, sus trayectoria muestrales en el intervalo de tiempo $\left[T_{n-1},T_{n}\right)$ est\'an descritas por
\begin{eqnarray*}
\zeta_{n}=\left(\xi_{n},\left\{X\left(T_{n-1}+t\right):0\leq t<\xi_{n}\right\}\right)
\end{eqnarray*}
Este $\zeta_{n}$ es el $n$-\'esimo segmento del proceso. El proceso es regenerativo sobre los tiempos $T_{n}$ si sus segmentos $\zeta_{n}$ son independientes e id\'enticamennte distribuidos.
\end{Def}

\begin{Note}
Un proceso regenerativo con media de la longitud de ciclo finita es llamado positivo recurrente.
\end{Note}

\begin{Teo}[Procesos Regenerativos]
Suponga que el proceso
\end{Teo}


\begin{Def}[Renewal Process Trinity]
Para un proceso de renovaci\'on $N\left(t\right)$, los siguientes procesos proveen de informaci\'on sobre los tiempos de renovaci\'on.
\begin{itemize}
\item $A\left(t\right)=t-T_{N\left(t\right)}$, el tiempo de recurrencia hacia atr\'as al tiempo $t$, que es el tiempo desde la \'ultima renovaci\'on para $t$.

\item $B\left(t\right)=T_{N\left(t\right)+1}-t$, el tiempo de recurrencia hacia adelante al tiempo $t$, residual del tiempo de renovaci\'on, que es el tiempo para la pr\'oxima renovaci\'on despu\'es de $t$.

\item $L\left(t\right)=\xi_{N\left(t\right)+1}=A\left(t\right)+B\left(t\right)$, la longitud del intervalo de renovaci\'on que contiene a $t$.
\end{itemize}
\end{Def}

\begin{Note}
El proceso tridimensional $\left(A\left(t\right),B\left(t\right),L\left(t\right)\right)$ es regenerativo sobre $T_{n}$, y por ende cada proceso lo es. Cada proceso $A\left(t\right)$ y $B\left(t\right)$ son procesos de MArkov a tiempo continuo con trayectorias continuas por partes en el espacio de estados $\rea_{+}$. Una expresi\'on conveniente para su distribuci\'on conjunta es, para $0\leq x<t,y\geq0$
\begin{equation}\label{NoRenovacion}
P\left\{A\left(t\right)>x,B\left(t\right)>y\right\}=
P\left\{N\left(t+y\right)-N\left((t-x)\right)=0\right\}
\end{equation}
\end{Note}

\begin{Ejem}[Tiempos de recurrencia Poisson]
Si $N\left(t\right)$ es un proceso Poisson con tasa $\lambda$, entonces de la expresi\'on (\ref{NoRenovacion}) se tiene que

\begin{eqnarray*}
\begin{array}{lc}
P\left\{A\left(t\right)>x,B\left(t\right)>y\right\}=e^{-\lambda\left(x+y\right)},&0\leq x<t,y\geq0,
\end{array}
\end{eqnarray*}
que es la probabilidad Poisson de no renovaciones en un intervalo de longitud $x+y$.

\end{Ejem}

\begin{Note}
Una cadena de Markov erg\'odica tiene la propiedad de ser estacionaria si la distribuci\'on de su estado al tiempo $0$ es su distribuci\'on estacionaria.
\end{Note}


\begin{Def}
Un proceso estoc\'astico a tiempo continuo $\left\{X\left(t\right):t\geq0\right\}$ en un espacio general es estacionario si sus distribuciones finito dimensionales son invariantes bajo cualquier  traslado: para cada $0\leq s_{1}<s_{2}<\cdots<s_{k}$ y $t\geq0$,
\begin{eqnarray*}
\left(X\left(s_{1}+t\right),\ldots,X\left(s_{k}+t\right)\right)=_{d}\left(X\left(s_{1}\right),\ldots,X\left(s_{k}\right)\right).
\end{eqnarray*}
\end{Def}

\begin{Note}
Un proceso de Markov es estacionario si $X\left(t\right)=_{d}X\left(0\right)$, $t\geq0$.
\end{Note}

Considerese el proceso $N\left(t\right)=\sum_{n}\indora\left(\tau_{n}\leq t\right)$ en $\rea_{+}$, con puntos $0<\tau_{1}<\tau_{2}<\cdots$.

\begin{Prop}
Si $N$ es un proceso puntual estacionario y $\esp\left[N\left(1\right)\right]<\infty$, entonces $\esp\left[N\left(t\right)\right]=t\esp\left[N\left(1\right)\right]$, $t\geq0$

\end{Prop}

\begin{Teo}
Los siguientes enunciados son equivalentes
\begin{itemize}
\item[i)] El proceso retardado de renovaci\'on $N$ es estacionario.

\item[ii)] EL proceso de tiempos de recurrencia hacia adelante $B\left(t\right)$ es estacionario.


\item[iii)] $\esp\left[N\left(t\right)\right]=t/\mu$,


\item[iv)] $G\left(t\right)=F_{e}\left(t\right)=\frac{1}{\mu}\int_{0}^{t}\left[1-F\left(s\right)\right]ds$
\end{itemize}
Cuando estos enunciados son ciertos, $P\left\{B\left(t\right)\leq x\right\}=F_{e}\left(x\right)$, para $t,x\geq0$.

\end{Teo}

\begin{Note}
Una consecuencia del teorema anterior es que el Proceso Poisson es el \'unico proceso sin retardo que es estacionario.
\end{Note}

\begin{Coro}
El proceso de renovaci\'on $N\left(t\right)$ sin retardo, y cuyos tiempos de inter renonaci\'on tienen media finita, es estacionario si y s\'olo si es un proceso Poisson.

\end{Coro}


%________________________________________________________________________
%\subsection{Procesos Regenerativos}
%________________________________________________________________________



\begin{Note}
Si $\tilde{X}\left(t\right)$ con espacio de estados $\tilde{S}$ es regenerativo sobre $T_{n}$, entonces $X\left(t\right)=f\left(\tilde{X}\left(t\right)\right)$ tambi\'en es regenerativo sobre $T_{n}$, para cualquier funci\'on $f:\tilde{S}\rightarrow S$.
\end{Note}

\begin{Note}
Los procesos regenerativos son crudamente regenerativos, pero no al rev\'es.
\end{Note}
%\subsection*{Procesos Regenerativos: Sigman\cite{Sigman1}}
\begin{Def}[Definici\'on Cl\'asica]
Un proceso estoc\'astico $X=\left\{X\left(t\right):t\geq0\right\}$ es llamado regenerativo is existe una variable aleatoria $R_{1}>0$ tal que
\begin{itemize}
\item[i)] $\left\{X\left(t+R_{1}\right):t\geq0\right\}$ es independiente de $\left\{\left\{X\left(t\right):t<R_{1}\right\},\right\}$
\item[ii)] $\left\{X\left(t+R_{1}\right):t\geq0\right\}$ es estoc\'asticamente equivalente a $\left\{X\left(t\right):t>0\right\}$
\end{itemize}

Llamamos a $R_{1}$ tiempo de regeneraci\'on, y decimos que $X$ se regenera en este punto.
\end{Def}

$\left\{X\left(t+R_{1}\right)\right\}$ es regenerativo con tiempo de regeneraci\'on $R_{2}$, independiente de $R_{1}$ pero con la misma distribuci\'on que $R_{1}$. Procediendo de esta manera se obtiene una secuencia de variables aleatorias independientes e id\'enticamente distribuidas $\left\{R_{n}\right\}$ llamados longitudes de ciclo. Si definimos a $Z_{k}\equiv R_{1}+R_{2}+\cdots+R_{k}$, se tiene un proceso de renovaci\'on llamado proceso de renovaci\'on encajado para $X$.




\begin{Def}
Para $x$ fijo y para cada $t\geq0$, sea $I_{x}\left(t\right)=1$ si $X\left(t\right)\leq x$,  $I_{x}\left(t\right)=0$ en caso contrario, y def\'inanse los tiempos promedio
\begin{eqnarray*}
\overline{X}&=&lim_{t\rightarrow\infty}\frac{1}{t}\int_{0}^{\infty}X\left(u\right)du\\
\prob\left(X_{\infty}\leq x\right)&=&lim_{t\rightarrow\infty}\frac{1}{t}\int_{0}^{\infty}I_{x}\left(u\right)du,
\end{eqnarray*}
cuando estos l\'imites existan.
\end{Def}

Como consecuencia del teorema de Renovaci\'on-Recompensa, se tiene que el primer l\'imite  existe y es igual a la constante
\begin{eqnarray*}
\overline{X}&=&\frac{\esp\left[\int_{0}^{R_{1}}X\left(t\right)dt\right]}{\esp\left[R_{1}\right]},
\end{eqnarray*}
suponiendo que ambas esperanzas son finitas.

\begin{Note}
\begin{itemize}
\item[a)] Si el proceso regenerativo $X$ es positivo recurrente y tiene trayectorias muestrales no negativas, entonces la ecuaci\'on anterior es v\'alida.
\item[b)] Si $X$ es positivo recurrente regenerativo, podemos construir una \'unica versi\'on estacionaria de este proceso, $X_{e}=\left\{X_{e}\left(t\right)\right\}$, donde $X_{e}$ es un proceso estoc\'astico regenerativo y estrictamente estacionario, con distribuci\'on marginal distribuida como $X_{\infty}$
\end{itemize}
\end{Note}

%________________________________________________________________________
%\subsection{Procesos Regenerativos}
%________________________________________________________________________

Para $\left\{X\left(t\right):t\geq0\right\}$ Proceso Estoc\'astico a tiempo continuo con estado de espacios $S$, que es un espacio m\'etrico, con trayectorias continuas por la derecha y con l\'imites por la izquierda c.s. Sea $N\left(t\right)$ un proceso de renovaci\'on en $\rea_{+}$ definido en el mismo espacio de probabilidad que $X\left(t\right)$, con tiempos de renovaci\'on $T$ y tiempos de inter-renovaci\'on $\xi_{n}=T_{n}-T_{n-1}$, con misma distribuci\'on $F$ de media finita $\mu$.



\begin{Def}
Para el proceso $\left\{\left(N\left(t\right),X\left(t\right)\right):t\geq0\right\}$, sus trayectoria muestrales en el intervalo de tiempo $\left[T_{n-1},T_{n}\right)$ est\'an descritas por
\begin{eqnarray*}
\zeta_{n}=\left(\xi_{n},\left\{X\left(T_{n-1}+t\right):0\leq t<\xi_{n}\right\}\right)
\end{eqnarray*}
Este $\zeta_{n}$ es el $n$-\'esimo segmento del proceso. El proceso es regenerativo sobre los tiempos $T_{n}$ si sus segmentos $\zeta_{n}$ son independientes e id\'enticamennte distribuidos.
\end{Def}


\begin{Note}
Si $\tilde{X}\left(t\right)$ con espacio de estados $\tilde{S}$ es regenerativo sobre $T_{n}$, entonces $X\left(t\right)=f\left(\tilde{X}\left(t\right)\right)$ tambi\'en es regenerativo sobre $T_{n}$, para cualquier funci\'on $f:\tilde{S}\rightarrow S$.
\end{Note}

\begin{Note}
Los procesos regenerativos son crudamente regenerativos, pero no al rev\'es.
\end{Note}

\begin{Def}[Definici\'on Cl\'asica]
Un proceso estoc\'astico $X=\left\{X\left(t\right):t\geq0\right\}$ es llamado regenerativo is existe una variable aleatoria $R_{1}>0$ tal que
\begin{itemize}
\item[i)] $\left\{X\left(t+R_{1}\right):t\geq0\right\}$ es independiente de $\left\{\left\{X\left(t\right):t<R_{1}\right\},\right\}$
\item[ii)] $\left\{X\left(t+R_{1}\right):t\geq0\right\}$ es estoc\'asticamente equivalente a $\left\{X\left(t\right):t>0\right\}$
\end{itemize}

Llamamos a $R_{1}$ tiempo de regeneraci\'on, y decimos que $X$ se regenera en este punto.
\end{Def}

$\left\{X\left(t+R_{1}\right)\right\}$ es regenerativo con tiempo de regeneraci\'on $R_{2}$, independiente de $R_{1}$ pero con la misma distribuci\'on que $R_{1}$. Procediendo de esta manera se obtiene una secuencia de variables aleatorias independientes e id\'enticamente distribuidas $\left\{R_{n}\right\}$ llamados longitudes de ciclo. Si definimos a $Z_{k}\equiv R_{1}+R_{2}+\cdots+R_{k}$, se tiene un proceso de renovaci\'on llamado proceso de renovaci\'on encajado para $X$.

\begin{Note}
Un proceso regenerativo con media de la longitud de ciclo finita es llamado positivo recurrente.
\end{Note}


\begin{Def}
Para $x$ fijo y para cada $t\geq0$, sea $I_{x}\left(t\right)=1$ si $X\left(t\right)\leq x$,  $I_{x}\left(t\right)=0$ en caso contrario, y def\'inanse los tiempos promedio
\begin{eqnarray*}
\overline{X}&=&lim_{t\rightarrow\infty}\frac{1}{t}\int_{0}^{\infty}X\left(u\right)du\\
\prob\left(X_{\infty}\leq x\right)&=&lim_{t\rightarrow\infty}\frac{1}{t}\int_{0}^{\infty}I_{x}\left(u\right)du,
\end{eqnarray*}
cuando estos l\'imites existan.
\end{Def}

Como consecuencia del teorema de Renovaci\'on-Recompensa, se tiene que el primer l\'imite  existe y es igual a la constante
\begin{eqnarray*}
\overline{X}&=&\frac{\esp\left[\int_{0}^{R_{1}}X\left(t\right)dt\right]}{\esp\left[R_{1}\right]},
\end{eqnarray*}
suponiendo que ambas esperanzas son finitas.

\begin{Note}
\begin{itemize}
\item[a)] Si el proceso regenerativo $X$ es positivo recurrente y tiene trayectorias muestrales no negativas, entonces la ecuaci\'on anterior es v\'alida.
\item[b)] Si $X$ es positivo recurrente regenerativo, podemos construir una \'unica versi\'on estacionaria de este proceso, $X_{e}=\left\{X_{e}\left(t\right)\right\}$, donde $X_{e}$ es un proceso estoc\'astico regenerativo y estrictamente estacionario, con distribuci\'on marginal distribuida como $X_{\infty}$
\end{itemize}
\end{Note}

%__________________________________________________________________________________________
%\subsection{Procesos Regenerativos Estacionarios - Stidham \cite{Stidham}}
%__________________________________________________________________________________________


Un proceso estoc\'astico a tiempo continuo $\left\{V\left(t\right),t\geq0\right\}$ es un proceso regenerativo si existe una sucesi\'on de variables aleatorias independientes e id\'enticamente distribuidas $\left\{X_{1},X_{2},\ldots\right\}$, sucesi\'on de renovaci\'on, tal que para cualquier conjunto de Borel $A$, 

\begin{eqnarray*}
\prob\left\{V\left(t\right)\in A|X_{1}+X_{2}+\cdots+X_{R\left(t\right)}=s,\left\{V\left(\tau\right),\tau<s\right\}\right\}=\prob\left\{V\left(t-s\right)\in A|X_{1}>t-s\right\},
\end{eqnarray*}
para todo $0\leq s\leq t$, donde $R\left(t\right)=\max\left\{X_{1}+X_{2}+\cdots+X_{j}\leq t\right\}=$n\'umero de renovaciones ({\emph{puntos de regeneraci\'on}}) que ocurren en $\left[0,t\right]$. El intervalo $\left[0,X_{1}\right)$ es llamado {\emph{primer ciclo de regeneraci\'on}} de $\left\{V\left(t \right),t\geq0\right\}$, $\left[X_{1},X_{1}+X_{2}\right)$ el {\emph{segundo ciclo de regeneraci\'on}}, y as\'i sucesivamente.

Sea $X=X_{1}$ y sea $F$ la funci\'on de distrbuci\'on de $X$


\begin{Def}
Se define el proceso estacionario, $\left\{V^{*}\left(t\right),t\geq0\right\}$, para $\left\{V\left(t\right),t\geq0\right\}$ por

\begin{eqnarray*}
\prob\left\{V\left(t\right)\in A\right\}=\frac{1}{\esp\left[X\right]}\int_{0}^{\infty}\prob\left\{V\left(t+x\right)\in A|X>x\right\}\left(1-F\left(x\right)\right)dx,
\end{eqnarray*} 
para todo $t\geq0$ y todo conjunto de Borel $A$.
\end{Def}

\begin{Def}
Una distribuci\'on se dice que es {\emph{aritm\'etica}} si todos sus puntos de incremento son m\'ultiplos de la forma $0,\lambda, 2\lambda,\ldots$ para alguna $\lambda>0$ entera.
\end{Def}


\begin{Def}
Una modificaci\'on medible de un proceso $\left\{V\left(t\right),t\geq0\right\}$, es una versi\'on de este, $\left\{V\left(t,w\right)\right\}$ conjuntamente medible para $t\geq0$ y para $w\in S$, $S$ espacio de estados para $\left\{V\left(t\right),t\geq0\right\}$.
\end{Def}

\begin{Teo}
Sea $\left\{V\left(t\right),t\geq\right\}$ un proceso regenerativo no negativo con modificaci\'on medible. Sea $\esp\left[X\right]<\infty$. Entonces el proceso estacionario dado por la ecuaci\'on anterior est\'a bien definido y tiene funci\'on de distribuci\'on independiente de $t$, adem\'as
\begin{itemize}
\item[i)] \begin{eqnarray*}
\esp\left[V^{*}\left(0\right)\right]&=&\frac{\esp\left[\int_{0}^{X}V\left(s\right)ds\right]}{\esp\left[X\right]}\end{eqnarray*}
\item[ii)] Si $\esp\left[V^{*}\left(0\right)\right]<\infty$, equivalentemente, si $\esp\left[\int_{0}^{X}V\left(s\right)ds\right]<\infty$,entonces
\begin{eqnarray*}
\frac{\int_{0}^{t}V\left(s\right)ds}{t}\rightarrow\frac{\esp\left[\int_{0}^{X}V\left(s\right)ds\right]}{\esp\left[X\right]}
\end{eqnarray*}
con probabilidad 1 y en media, cuando $t\rightarrow\infty$.
\end{itemize}
\end{Teo}
%
%___________________________________________________________________________________________
%\vspace{5.5cm}
%\chapter{Cadenas de Markov estacionarias}
%\vspace{-1.0cm}


%__________________________________________________________________________________________
%\subsection{Procesos Regenerativos Estacionarios - Stidham \cite{Stidham}}
%__________________________________________________________________________________________


Un proceso estoc\'astico a tiempo continuo $\left\{V\left(t\right),t\geq0\right\}$ es un proceso regenerativo si existe una sucesi\'on de variables aleatorias independientes e id\'enticamente distribuidas $\left\{X_{1},X_{2},\ldots\right\}$, sucesi\'on de renovaci\'on, tal que para cualquier conjunto de Borel $A$, 

\begin{eqnarray*}
\prob\left\{V\left(t\right)\in A|X_{1}+X_{2}+\cdots+X_{R\left(t\right)}=s,\left\{V\left(\tau\right),\tau<s\right\}\right\}=\prob\left\{V\left(t-s\right)\in A|X_{1}>t-s\right\},
\end{eqnarray*}
para todo $0\leq s\leq t$, donde $R\left(t\right)=\max\left\{X_{1}+X_{2}+\cdots+X_{j}\leq t\right\}=$n\'umero de renovaciones ({\emph{puntos de regeneraci\'on}}) que ocurren en $\left[0,t\right]$. El intervalo $\left[0,X_{1}\right)$ es llamado {\emph{primer ciclo de regeneraci\'on}} de $\left\{V\left(t \right),t\geq0\right\}$, $\left[X_{1},X_{1}+X_{2}\right)$ el {\emph{segundo ciclo de regeneraci\'on}}, y as\'i sucesivamente.

Sea $X=X_{1}$ y sea $F$ la funci\'on de distrbuci\'on de $X$


\begin{Def}
Se define el proceso estacionario, $\left\{V^{*}\left(t\right),t\geq0\right\}$, para $\left\{V\left(t\right),t\geq0\right\}$ por

\begin{eqnarray*}
\prob\left\{V\left(t\right)\in A\right\}=\frac{1}{\esp\left[X\right]}\int_{0}^{\infty}\prob\left\{V\left(t+x\right)\in A|X>x\right\}\left(1-F\left(x\right)\right)dx,
\end{eqnarray*} 
para todo $t\geq0$ y todo conjunto de Borel $A$.
\end{Def}

\begin{Def}
Una distribuci\'on se dice que es {\emph{aritm\'etica}} si todos sus puntos de incremento son m\'ultiplos de la forma $0,\lambda, 2\lambda,\ldots$ para alguna $\lambda>0$ entera.
\end{Def}


\begin{Def}
Una modificaci\'on medible de un proceso $\left\{V\left(t\right),t\geq0\right\}$, es una versi\'on de este, $\left\{V\left(t,w\right)\right\}$ conjuntamente medible para $t\geq0$ y para $w\in S$, $S$ espacio de estados para $\left\{V\left(t\right),t\geq0\right\}$.
\end{Def}

\begin{Teo}
Sea $\left\{V\left(t\right),t\geq\right\}$ un proceso regenerativo no negativo con modificaci\'on medible. Sea $\esp\left[X\right]<\infty$. Entonces el proceso estacionario dado por la ecuaci\'on anterior est\'a bien definido y tiene funci\'on de distribuci\'on independiente de $t$, adem\'as
\begin{itemize}
\item[i)] \begin{eqnarray*}
\esp\left[V^{*}\left(0\right)\right]&=&\frac{\esp\left[\int_{0}^{X}V\left(s\right)ds\right]}{\esp\left[X\right]}\end{eqnarray*}
\item[ii)] Si $\esp\left[V^{*}\left(0\right)\right]<\infty$, equivalentemente, si $\esp\left[\int_{0}^{X}V\left(s\right)ds\right]<\infty$,entonces
\begin{eqnarray*}
\frac{\int_{0}^{t}V\left(s\right)ds}{t}\rightarrow\frac{\esp\left[\int_{0}^{X}V\left(s\right)ds\right]}{\esp\left[X\right]}
\end{eqnarray*}
con probabilidad 1 y en media, cuando $t\rightarrow\infty$.
\end{itemize}
\end{Teo}

Para $\left\{X\left(t\right):t\geq0\right\}$ Proceso Estoc\'astico a tiempo continuo con estado de espacios $S$, que es un espacio m\'etrico, con trayectorias continuas por la derecha y con l\'imites por la izquierda c.s. Sea $N\left(t\right)$ un proceso de renovaci\'on en $\rea_{+}$ definido en el mismo espacio de probabilidad que $X\left(t\right)$, con tiempos de renovaci\'on $T$ y tiempos de inter-renovaci\'on $\xi_{n}=T_{n}-T_{n-1}$, con misma distribuci\'on $F$ de media finita $\mu$.


%______________________________________________________________________
%\subsection{Ejemplos, Notas importantes}


Sean $T_{1},T_{2},\ldots$ los puntos donde las longitudes de las colas de la red de sistemas de visitas c\'iclicas son cero simult\'aneamente, cuando la cola $Q_{j}$ es visitada por el servidor para dar servicio, es decir, $L_{1}\left(T_{i}\right)=0,L_{2}\left(T_{i}\right)=0,\hat{L}_{1}\left(T_{i}\right)=0$ y $\hat{L}_{2}\left(T_{i}\right)=0$, a estos puntos se les denominar\'a puntos regenerativos. Sea la funci\'on generadora de momentos para $L_{i}$, el n\'umero de usuarios en la cola $Q_{i}\left(z\right)$ en cualquier momento, est\'a dada por el tiempo promedio de $z^{L_{i}\left(t\right)}$ sobre el ciclo regenerativo definido anteriormente:

\begin{eqnarray*}
Q_{i}\left(z\right)&=&\esp\left[z^{L_{i}\left(t\right)}\right]=\frac{\esp\left[\sum_{m=1}^{M_{i}}\sum_{t=\tau_{i}\left(m\right)}^{\tau_{i}\left(m+1\right)-1}z^{L_{i}\left(t\right)}\right]}{\esp\left[\sum_{m=1}^{M_{i}}\tau_{i}\left(m+1\right)-\tau_{i}\left(m\right)\right]}
\end{eqnarray*}

$M_{i}$ es un tiempo de paro en el proceso regenerativo con $\esp\left[M_{i}\right]<\infty$\footnote{En Stidham\cite{Stidham} y Heyman  se muestra que una condici\'on suficiente para que el proceso regenerativo 
estacionario sea un procesoo estacionario es que el valor esperado del tiempo del ciclo regenerativo sea finito, es decir: $\esp\left[\sum_{m=1}^{M_{i}}C_{i}^{(m)}\right]<\infty$, como cada $C_{i}^{(m)}$ contiene intervalos de r\'eplica positivos, se tiene que $\esp\left[M_{i}\right]<\infty$, adem\'as, como $M_{i}>0$, se tiene que la condici\'on anterior es equivalente a tener que $\esp\left[C_{i}\right]<\infty$,
por lo tanto una condici\'on suficiente para la existencia del proceso regenerativo est\'a dada por $\sum_{k=1}^{N}\mu_{k}<1.$}, se sigue del lema de Wald que:


\begin{eqnarray*}
\esp\left[\sum_{m=1}^{M_{i}}\sum_{t=\tau_{i}\left(m\right)}^{\tau_{i}\left(m+1\right)-1}z^{L_{i}\left(t\right)}\right]&=&\esp\left[M_{i}\right]\esp\left[\sum_{t=\tau_{i}\left(m\right)}^{\tau_{i}\left(m+1\right)-1}z^{L_{i}\left(t\right)}\right]\\
\esp\left[\sum_{m=1}^{M_{i}}\tau_{i}\left(m+1\right)-\tau_{i}\left(m\right)\right]&=&\esp\left[M_{i}\right]\esp\left[\tau_{i}\left(m+1\right)-\tau_{i}\left(m\right)\right]
\end{eqnarray*}

por tanto se tiene que


\begin{eqnarray*}
Q_{i}\left(z\right)&=&\frac{\esp\left[\sum_{t=\tau_{i}\left(m\right)}^{\tau_{i}\left(m+1\right)-1}z^{L_{i}\left(t\right)}\right]}{\esp\left[\tau_{i}\left(m+1\right)-\tau_{i}\left(m\right)\right]}
\end{eqnarray*}

observar que el denominador es simplemente la duraci\'on promedio del tiempo del ciclo.


Haciendo las siguientes sustituciones en la ecuaci\'on (\ref{Corolario2}): $n\rightarrow t-\tau_{i}\left(m\right)$, $T \rightarrow \overline{\tau}_{i}\left(m\right)-\tau_{i}\left(m\right)$, $L_{n}\rightarrow L_{i}\left(t\right)$ y $F\left(z\right)=\esp\left[z^{L_{0}}\right]\rightarrow F_{i}\left(z\right)=\esp\left[z^{L_{i}\tau_{i}\left(m\right)}\right]$, se puede ver que

\begin{eqnarray}\label{Eq.Arribos.Primera}
\esp\left[\sum_{n=0}^{T-1}z^{L_{n}}\right]=
\esp\left[\sum_{t=\tau_{i}\left(m\right)}^{\overline{\tau}_{i}\left(m\right)-1}z^{L_{i}\left(t\right)}\right]
=z\frac{F_{i}\left(z\right)-1}{z-P_{i}\left(z\right)}
\end{eqnarray}

Por otra parte durante el tiempo de intervisita para la cola $i$, $L_{i}\left(t\right)$ solamente se incrementa de manera que el incremento por intervalo de tiempo est\'a dado por la funci\'on generadora de probabilidades de $P_{i}\left(z\right)$, por tanto la suma sobre el tiempo de intervisita puede evaluarse como:

\begin{eqnarray*}
\esp\left[\sum_{t=\tau_{i}\left(m\right)}^{\tau_{i}\left(m+1\right)-1}z^{L_{i}\left(t\right)}\right]&=&\esp\left[\sum_{t=\tau_{i}\left(m\right)}^{\tau_{i}\left(m+1\right)-1}\left\{P_{i}\left(z\right)\right\}^{t-\overline{\tau}_{i}\left(m\right)}\right]=\frac{1-\esp\left[\left\{P_{i}\left(z\right)\right\}^{\tau_{i}\left(m+1\right)-\overline{\tau}_{i}\left(m\right)}\right]}{1-P_{i}\left(z\right)}\\
&=&\frac{1-I_{i}\left[P_{i}\left(z\right)\right]}{1-P_{i}\left(z\right)}
\end{eqnarray*}
por tanto

\begin{eqnarray*}
\esp\left[\sum_{t=\tau_{i}\left(m\right)}^{\tau_{i}\left(m+1\right)-1}z^{L_{i}\left(t\right)}\right]&=&
\frac{1-F_{i}\left(z\right)}{1-P_{i}\left(z\right)}
\end{eqnarray*}

Por lo tanto

\begin{eqnarray*}
Q_{i}\left(z\right)&=&\frac{\esp\left[\sum_{t=\tau_{i}\left(m\right)}^{\tau_{i}
\left(m+1\right)-1}z^{L_{i}\left(t\right)}\right]}{\esp\left[\tau_{i}\left(m+1\right)-\tau_{i}\left(m\right)\right]}\\
&=&\frac{1}{\esp\left[\tau_{i}\left(m+1\right)-\tau_{i}\left(m\right)\right]}
\left\{
\esp\left[\sum_{t=\tau_{i}\left(m\right)}^{\overline{\tau}_{i}\left(m\right)-1}
z^{L_{i}\left(t\right)}\right]
+\esp\left[\sum_{t=\overline{\tau}_{i}\left(m\right)}^{\tau_{i}\left(m+1\right)-1}
z^{L_{i}\left(t\right)}\right]\right\}\\
&=&\frac{1}{\esp\left[\tau_{i}\left(m+1\right)-\tau_{i}\left(m\right)\right]}
\left\{
z\frac{F_{i}\left(z\right)-1}{z-P_{i}\left(z\right)}+\frac{1-F_{i}\left(z\right)}
{1-P_{i}\left(z\right)}
\right\}
\end{eqnarray*}

es decir

\begin{equation}
Q_{i}\left(z\right)=\frac{1}{\esp\left[C_{i}\right]}\cdot\frac{1-F_{i}\left(z\right)}{P_{i}\left(z\right)-z}\cdot\frac{\left(1-z\right)P_{i}\left(z\right)}{1-P_{i}\left(z\right)}
\end{equation}

\begin{Teo}
Dada una Red de Sistemas de Visitas C\'iclicas (RSVC), conformada por dos Sistemas de Visitas C\'iclicas (SVC), donde cada uno de ellos consta de dos colas tipo $M/M/1$. Los dos sistemas est\'an comunicados entre s\'i por medio de la transferencia de usuarios entre las colas $Q_{1}\leftrightarrow Q_{3}$ y $Q_{2}\leftrightarrow Q_{4}$. Se definen los eventos para los procesos de arribos al tiempo $t$, $A_{j}\left(t\right)=\left\{0 \textrm{ arribos en }Q_{j}\textrm{ al tiempo }t\right\}$ para alg\'un tiempo $t\geq0$ y $Q_{j}$ la cola $j$-\'esima en la RSVC, para $j=1,2,3,4$.  Existe un intervalo $I\neq\emptyset$ tal que para $T^{*}\in I$, tal que $\prob\left\{A_{1}\left(T^{*}\right),A_{2}\left(Tt^{*}\right),
A_{3}\left(T^{*}\right),A_{4}\left(T^{*}\right)|T^{*}\in I\right\}>0$.
\end{Teo}

\begin{proof}
Sin p\'erdida de generalidad podemos considerar como base del an\'alisis a la cola $Q_{1}$ del primer sistema que conforma la RSVC.

Sea $n>0$, ciclo en el primer sistema en el que se sabe que $L_{j}\left(\overline{\tau}_{1}\left(n\right)\right)=0$, pues la pol\'itica de servicio con que atienden los servidores es la exhaustiva. Como es sabido, para trasladarse a la siguiente cola, el servidor incurre en un tiempo de traslado $r_{1}\left(n\right)>0$, entonces tenemos que $\tau_{2}\left(n\right)=\overline{\tau}_{1}\left(n\right)+r_{1}\left(n\right)$.


Definamos el intervalo $I_{1}\equiv\left[\overline{\tau}_{1}\left(n\right),\tau_{2}\left(n\right)\right]$ de longitud $\xi_{1}=r_{1}\left(n\right)$. Dado que los tiempos entre arribo son exponenciales con tasa $\tilde{\mu}_{1}=\mu_{1}+\hat{\mu}_{1}$ ($\mu_{1}$ son los arribos a $Q_{1}$ por primera vez al sistema, mientras que $\hat{\mu}_{1}$ son los arribos de traslado procedentes de $Q_{3}$) se tiene que la probabilidad del evento $A_{1}\left(t\right)$ est\'a dada por 

\begin{equation}
\prob\left\{A_{1}\left(t\right)|t\in I_{1}\left(n\right)\right\}=e^{-\tilde{\mu}_{1}\xi_{1}\left(n\right)}.
\end{equation} 

Por otra parte, para la cola $Q_{2}$, el tiempo $\overline{\tau}_{2}\left(n-1\right)$ es tal que $L_{2}\left(\overline{\tau}_{2}\left(n-1\right)\right)=0$, es decir, es el tiempo en que la cola queda totalmente vac\'ia en el ciclo anterior a $n$. Entonces tenemos un sgundo intervalo $I_{2}\equiv\left[\overline{\tau}_{2}\left(n-1\right),\tau_{2}\left(n\right)\right]$. Por lo tanto la probabilidad del evento $A_{2}\left(t\right)$ tiene probabilidad dada por

\begin{equation}
\prob\left\{A_{2}\left(t\right)|t\in I_{2}\left(n\right)\right\}=e^{-\tilde{\mu}_{2}\xi_{2}\left(n\right)},
\end{equation} 

donde $\xi_{2}\left(n\right)=\tau_{2}\left(n\right)-\overline{\tau}_{2}\left(n-1\right)$.



Entonces, se tiene que

\begin{eqnarray*}
\prob\left\{A_{1}\left(t\right),A_{2}\left(t\right)|t\in I_{1}\left(n\right)\right\}&=&
\prob\left\{A_{1}\left(t\right)|t\in I_{1}\left(n\right)\right\}
\prob\left\{A_{2}\left(t\right)|t\in I_{1}\left(n\right)\right\}\\
&\geq&
\prob\left\{A_{1}\left(t\right)|t\in I_{1}\left(n\right)\right\}
\prob\left\{A_{2}\left(t\right)|t\in I_{2}\left(n\right)\right\}\\
&=&e^{-\tilde{\mu}_{1}\xi_{1}\left(n\right)}e^{-\tilde{\mu}_{2}\xi_{2}\left(n\right)}
=e^{-\left[\tilde{\mu}_{1}\xi_{1}\left(n\right)+\tilde{\mu}_{2}\xi_{2}\left(n\right)\right]}.
\end{eqnarray*}


es decir, 

\begin{equation}
\prob\left\{A_{1}\left(t\right),A_{2}\left(t\right)|t\in I_{1}\left(n\right)\right\}
=e^{-\left[\tilde{\mu}_{1}\xi_{1}\left(n\right)+\tilde{\mu}_{2}\xi_{2}
\left(n\right)\right]}>0.
\end{equation}

En lo que respecta a la relaci\'on entre los dos SVC que conforman la RSVC, se afirma que existe $m>0$ tal que $\overline{\tau}_{3}\left(m\right)<\tau_{2}\left(n\right)<\tau_{4}\left(m\right)$.

Para $Q_{3}$ sea $I_{3}=\left[\overline{\tau}_{3}\left(m\right),\tau_{4}\left(m\right)\right]$ con longitud  $\xi_{3}\left(m\right)=r_{3}\left(m\right)$, entonces 

\begin{equation}
\prob\left\{A_{3}\left(t\right)|t\in I_{3}\left(n\right)\right\}=e^{-\tilde{\mu}_{3}\xi_{3}\left(n\right)}.
\end{equation} 

An\'alogamente que como se hizo para $Q_{2}$, tenemos que para $Q_{4}$ se tiene el intervalo $I_{4}=\left[\overline{\tau}_{4}\left(m-1\right),\tau_{4}\left(m\right)\right]$ con longitud $\xi_{4}\left(m\right)=\tau_{4}\left(m\right)-\overline{\tau}_{4}\left(m-1\right)$, entonces


\begin{equation}
\prob\left\{A_{4}\left(t\right)|t\in I_{4}\left(m\right)\right\}=e^{-\tilde{\mu}_{4}\xi_{4}\left(n\right)}.
\end{equation} 

Al igual que para el primer sistema, dado que $I_{3}\left(m\right)\subset I_{4}\left(m\right)$, se tiene que

\begin{eqnarray*}
\xi_{3}\left(m\right)\leq\xi_{4}\left(m\right)&\Leftrightarrow& -\xi_{3}\left(m\right)\geq-\xi_{4}\left(m\right)
\\
-\tilde{\mu}_{4}\xi_{3}\left(m\right)\geq-\tilde{\mu}_{4}\xi_{4}\left(m\right)&\Leftrightarrow&
e^{-\tilde{\mu}_{4}\xi_{3}\left(m\right)}\geq e^{-\tilde{\mu}_{4}\xi_{4}\left(m\right)}\\
\prob\left\{A_{4}\left(t\right)|t\in I_{3}\left(m\right)\right\}&\geq&
\prob\left\{A_{4}\left(t\right)|t\in I_{4}\left(m\right)\right\}
\end{eqnarray*}

Entonces, dado que los eventos $A_{3}$ y $A_{4}$ son independientes, se tiene que

\begin{eqnarray*}
\prob\left\{A_{3}\left(t\right),A_{4}\left(t\right)|t\in I_{3}\left(m\right)\right\}&=&
\prob\left\{A_{3}\left(t\right)|t\in I_{3}\left(m\right)\right\}
\prob\left\{A_{4}\left(t\right)|t\in I_{3}\left(m\right)\right\}\\
&\geq&
\prob\left\{A_{3}\left(t\right)|t\in I_{3}\left(n\right)\right\}
\prob\left\{A_{4}\left(t\right)|t\in I_{4}\left(n\right)\right\}\\
&=&e^{-\tilde{\mu}_{3}\xi_{3}\left(m\right)}e^{-\tilde{\mu}_{4}\xi_{4}
\left(m\right)}
=e^{-\left[\tilde{\mu}_{3}\xi_{3}\left(m\right)+\tilde{\mu}_{4}\xi_{4}
\left(m\right)\right]}.
\end{eqnarray*}


es decir, 

\begin{equation}
\prob\left\{A_{3}\left(t\right),A_{4}\left(t\right)|t\in I_{3}\left(m\right)\right\}
=e^{-\left[\tilde{\mu}_{3}\xi_{3}\left(m\right)+\tilde{\mu}_{4}\xi_{4}
\left(m\right)\right]}>0.
\end{equation}

Por construcci\'on se tiene que $I\left(n,m\right)\equiv I_{1}\left(n\right)\cap I_{3}\left(m\right)\neq\emptyset$,entonces en particular se tienen las contenciones $I\left(n,m\right)\subseteq I_{1}\left(n\right)$ y $I\left(n,m\right)\subseteq I_{3}\left(m\right)$, por lo tanto si definimos $\xi_{n,m}\equiv\ell\left(I\left(n,m\right)\right)$ tenemos que

\begin{eqnarray*}
\xi_{n,m}\leq\xi_{1}\left(n\right)\textrm{ y }\xi_{n,m}\leq\xi_{3}\left(m\right)\textrm{ entonces }
-\xi_{n,m}\geq-\xi_{1}\left(n\right)\textrm{ y }-\xi_{n,m}\leq-\xi_{3}\left(m\right)\\
\end{eqnarray*}
por lo tanto tenemos las desigualdades 



\begin{eqnarray*}
\begin{array}{ll}
-\tilde{\mu}_{1}\xi_{n,m}\geq-\tilde{\mu}_{1}\xi_{1}\left(n\right),&
-\tilde{\mu}_{2}\xi_{n,m}\geq-\tilde{\mu}_{2}\xi_{1}\left(n\right)
\geq-\tilde{\mu}_{2}\xi_{2}\left(n\right),\\
-\tilde{\mu}_{3}\xi_{n,m}\geq-\tilde{\mu}_{3}\xi_{3}\left(m\right),&
-\tilde{\mu}_{4}\xi_{n,m}\geq-\tilde{\mu}_{4}\xi_{3}\left(m\right)
\geq-\tilde{\mu}_{4}\xi_{4}\left(m\right).
\end{array}
\end{eqnarray*}

Sea $T^{*}\in I_{n,m}$, entonces dado que en particular $T^{*}\in I_{1}\left(n\right)$ se cumple con probabilidad positiva que no hay arribos a las colas $Q_{1}$ y $Q_{2}$, en consecuencia, tampoco hay usuarios de transferencia para $Q_{3}$ y $Q_{4}$, es decir, $\tilde{\mu}_{1}=\mu_{1}$, $\tilde{\mu}_{2}=\mu_{2}$, $\tilde{\mu}_{3}=\mu_{3}$, $\tilde{\mu}_{4}=\mu_{4}$, es decir, los eventos $Q_{1}$ y $Q_{3}$ son condicionalmente independientes en el intervalo $I_{n,m}$; lo mismo ocurre para las colas $Q_{2}$ y $Q_{4}$, por lo tanto tenemos que


\begin{eqnarray}
\begin{array}{l}
\prob\left\{A_{1}\left(T^{*}\right),A_{2}\left(T^{*}\right),
A_{3}\left(T^{*}\right),A_{4}\left(T^{*}\right)|T^{*}\in I_{n,m}\right\}
=\prod_{j=1}^{4}\prob\left\{A_{j}\left(T^{*}\right)|T^{*}\in I_{n,m}\right\}\\
\geq\prob\left\{A_{1}\left(T^{*}\right)|T^{*}\in I_{1}\left(n\right)\right\}
\prob\left\{A_{2}\left(T^{*}\right)|T^{*}\in I_{2}\left(n\right)\right\}
\prob\left\{A_{3}\left(T^{*}\right)|T^{*}\in I_{3}\left(m\right)\right\}
\prob\left\{A_{4}\left(T^{*}\right)|T^{*}\in I_{4}\left(m\right)\right\}\\
=e^{-\mu_{1}\xi_{1}\left(n\right)}
e^{-\mu_{2}\xi_{2}\left(n\right)}
e^{-\mu_{3}\xi_{3}\left(m\right)}
e^{-\mu_{4}\xi_{4}\left(m\right)}
=e^{-\left[\tilde{\mu}_{1}\xi_{1}\left(n\right)
+\tilde{\mu}_{2}\xi_{2}\left(n\right)
+\tilde{\mu}_{3}\xi_{3}\left(m\right)
+\tilde{\mu}_{4}\xi_{4}
\left(m\right)\right]}>0.
\end{array}
\end{eqnarray}
\end{proof}


Estos resultados aparecen en Daley (1968) \cite{Daley68} para $\left\{T_{n}\right\}$ intervalos de inter-arribo, $\left\{D_{n}\right\}$ intervalos de inter-salida y $\left\{S_{n}\right\}$ tiempos de servicio.

\begin{itemize}
\item Si el proceso $\left\{T_{n}\right\}$ es Poisson, el proceso $\left\{D_{n}\right\}$ es no correlacionado si y s\'olo si es un proceso Poisso, lo cual ocurre si y s\'olo si $\left\{S_{n}\right\}$ son exponenciales negativas.

\item Si $\left\{S_{n}\right\}$ son exponenciales negativas, $\left\{D_{n}\right\}$ es un proceso de renovaci\'on  si y s\'olo si es un proceso Poisson, lo cual ocurre si y s\'olo si $\left\{T_{n}\right\}$ es un proceso Poisson.

\item $\esp\left(D_{n}\right)=\esp\left(T_{n}\right)$.

\item Para un sistema de visitas $GI/M/1$ se tiene el siguiente teorema:

\begin{Teo}
En un sistema estacionario $GI/M/1$ los intervalos de interpartida tienen
\begin{eqnarray*}
\esp\left(e^{-\theta D_{n}}\right)&=&\mu\left(\mu+\theta\right)^{-1}\left[\delta\theta
-\mu\left(1-\delta\right)\alpha\left(\theta\right)\right]
\left[\theta-\mu\left(1-\delta\right)^{-1}\right]\\
\alpha\left(\theta\right)&=&\esp\left[e^{-\theta T_{0}}\right]\\
var\left(D_{n}\right)&=&var\left(T_{0}\right)-\left(\tau^{-1}-\delta^{-1}\right)
2\delta\left(\esp\left(S_{0}\right)\right)^{2}\left(1-\delta\right)^{-1}.
\end{eqnarray*}
\end{Teo}



\begin{Teo}
El proceso de salida de un sistema de colas estacionario $GI/M/1$ es un proceso de renovaci\'on si y s\'olo si el proceso de entrada es un proceso Poisson, en cuyo caso el proceso de salida es un proceso Poisson.
\end{Teo}


\begin{Teo}
Los intervalos de interpartida $\left\{D_{n}\right\}$ de un sistema $M/G/1$ estacionario son no correlacionados si y s\'olo si la distribuci\'on de los tiempos de servicio es exponencial negativa, es decir, el sistema es de tipo  $M/M/1$.

\end{Teo}



\end{itemize}


%\section{Resultados para Procesos de Salida}

En Sigman, Thorison y Wolff \cite{Sigman2} prueban que para la existencia de un una sucesi\'on infinita no decreciente de tiempos de regeneraci\'on $\tau_{1}\leq\tau_{2}\leq\cdots$ en los cuales el proceso se regenera, basta un tiempo de regeneraci\'on $R_{1}$, donde $R_{j}=\tau_{j}-\tau_{j-1}$. Para tal efecto se requiere la existencia de un espacio de probabilidad $\left(\Omega,\mathcal{F},\prob\right)$, y proceso estoc\'astico $\textit{X}=\left\{X\left(t\right):t\geq0\right\}$ con espacio de estados $\left(S,\mathcal{R}\right)$, con $\mathcal{R}$ $\sigma$-\'algebra.

\begin{Prop}
Si existe una variable aleatoria no negativa $R_{1}$ tal que $\theta_{R\footnotesize{1}}X=_{D}X$, entonces $\left(\Omega,\mathcal{F},\prob\right)$ puede extenderse para soportar una sucesi\'on estacionaria de variables aleatorias $R=\left\{R_{k}:k\geq1\right\}$, tal que para $k\geq1$,
\begin{eqnarray*}
\theta_{k}\left(X,R\right)=_{D}\left(X,R\right).
\end{eqnarray*}

Adem\'as, para $k\geq1$, $\theta_{k}R$ es condicionalmente independiente de $\left(X,R_{1},\ldots,R_{k}\right)$, dado $\theta_{\tau k}X$.

\end{Prop}


\begin{itemize}
\item Doob en 1953 demostr\'o que el estado estacionario de un proceso de partida en un sistema de espera $M/G/\infty$, es Poisson con la misma tasa que el proceso de arribos.

\item Burke en 1968, fue el primero en demostrar que el estado estacionario de un proceso de salida de una cola $M/M/s$ es un proceso Poisson.

\item Disney en 1973 obtuvo el siguiente resultado:

\begin{Teo}
Para el sistema de espera $M/G/1/L$ con disciplina FIFO, el proceso $\textbf{I}$ es un proceso de renovaci\'on si y s\'olo si el proceso denominado longitud de la cola es estacionario y se cumple cualquiera de los siguientes casos:

\begin{itemize}
\item[a)] Los tiempos de servicio son identicamente cero;
\item[b)] $L=0$, para cualquier proceso de servicio $S$;
\item[c)] $L=1$ y $G=D$;
\item[d)] $L=\infty$ y $G=M$.
\end{itemize}
En estos casos, respectivamente, las distribuciones de interpartida $P\left\{T_{n+1}-T_{n}\leq t\right\}$ son


\begin{itemize}
\item[a)] $1-e^{-\lambda t}$, $t\geq0$;
\item[b)] $1-e^{-\lambda t}*F\left(t\right)$, $t\geq0$;
\item[c)] $1-e^{-\lambda t}*\indora_{d}\left(t\right)$, $t\geq0$;
\item[d)] $1-e^{-\lambda t}*F\left(t\right)$, $t\geq0$.
\end{itemize}
\end{Teo}


\item Finch (1959) mostr\'o que para los sistemas $M/G/1/L$, con $1\leq L\leq \infty$ con distribuciones de servicio dos veces diferenciable, solamente el sistema $M/M/1/\infty$ tiene proceso de salida de renovaci\'on estacionario.

\item King (1971) demostro que un sistema de colas estacionario $M/G/1/1$ tiene sus tiempos de interpartida sucesivas $D_{n}$ y $D_{n+1}$ son independientes, si y s\'olo si, $G=D$, en cuyo caso le proceso de salida es de renovaci\'on.

\item Disney (1973) demostr\'o que el \'unico sistema estacionario $M/G/1/L$, que tiene proceso de salida de renovaci\'on  son los sistemas $M/M/1$ y $M/D/1/1$.



\item El siguiente resultado es de Disney y Koning (1985)
\begin{Teo}
En un sistema de espera $M/G/s$, el estado estacionario del proceso de salida es un proceso Poisson para cualquier distribuci\'on de los tiempos de servicio si el sistema tiene cualquiera de las siguientes cuatro propiedades.

\begin{itemize}
\item[a)] $s=\infty$
\item[b)] La disciplina de servicio es de procesador compartido.
\item[c)] La disciplina de servicio es LCFS y preemptive resume, esto se cumple para $L<\infty$
\item[d)] $G=M$.
\end{itemize}

\end{Teo}

\item El siguiente resultado es de Alamatsaz (1983)

\begin{Teo}
En cualquier sistema de colas $GI/G/1/L$ con $1\leq L<\infty$ y distribuci\'on de interarribos $A$ y distribuci\'on de los tiempos de servicio $B$, tal que $A\left(0\right)=0$, $A\left(t\right)\left(1-B\left(t\right)\right)>0$ para alguna $t>0$ y $B\left(t\right)$ para toda $t>0$, es imposible que el proceso de salida estacionario sea de renovaci\'on.
\end{Teo}

\end{itemize}

Estos resultados aparecen en Daley (1968) \cite{Daley68} para $\left\{T_{n}\right\}$ intervalos de inter-arribo, $\left\{D_{n}\right\}$ intervalos de inter-salida y $\left\{S_{n}\right\}$ tiempos de servicio.

\begin{itemize}
\item Si el proceso $\left\{T_{n}\right\}$ es Poisson, el proceso $\left\{D_{n}\right\}$ es no correlacionado si y s\'olo si es un proceso Poisso, lo cual ocurre si y s\'olo si $\left\{S_{n}\right\}$ son exponenciales negativas.

\item Si $\left\{S_{n}\right\}$ son exponenciales negativas, $\left\{D_{n}\right\}$ es un proceso de renovaci\'on  si y s\'olo si es un proceso Poisson, lo cual ocurre si y s\'olo si $\left\{T_{n}\right\}$ es un proceso Poisson.

\item $\esp\left(D_{n}\right)=\esp\left(T_{n}\right)$.

\item Para un sistema de visitas $GI/M/1$ se tiene el siguiente teorema:

\begin{Teo}
En un sistema estacionario $GI/M/1$ los intervalos de interpartida tienen
\begin{eqnarray*}
\esp\left(e^{-\theta D_{n}}\right)&=&\mu\left(\mu+\theta\right)^{-1}\left[\delta\theta
-\mu\left(1-\delta\right)\alpha\left(\theta\right)\right]
\left[\theta-\mu\left(1-\delta\right)^{-1}\right]\\
\alpha\left(\theta\right)&=&\esp\left[e^{-\theta T_{0}}\right]\\
var\left(D_{n}\right)&=&var\left(T_{0}\right)-\left(\tau^{-1}-\delta^{-1}\right)
2\delta\left(\esp\left(S_{0}\right)\right)^{2}\left(1-\delta\right)^{-1}.
\end{eqnarray*}
\end{Teo}



\begin{Teo}
El proceso de salida de un sistema de colas estacionario $GI/M/1$ es un proceso de renovaci\'on si y s\'olo si el proceso de entrada es un proceso Poisson, en cuyo caso el proceso de salida es un proceso Poisson.
\end{Teo}


\begin{Teo}
Los intervalos de interpartida $\left\{D_{n}\right\}$ de un sistema $M/G/1$ estacionario son no correlacionados si y s\'olo si la distribuci\'on de los tiempos de servicio es exponencial negativa, es decir, el sistema es de tipo  $M/M/1$.

\end{Teo}



\end{itemize}
%\newpage
%________________________________________________________________________
%\section{Redes de Sistemas de Visitas C\'iclicas}
%________________________________________________________________________

Sean $Q_{1},Q_{2},Q_{3}$ y $Q_{4}$ en una Red de Sistemas de Visitas C\'iclicas (RSVC). Supongamos que cada una de las colas es del tipo $M/M/1$ con tasa de arribo $\mu_{i}$ y que la transferencia de usuarios entre los dos sistemas ocurre entre $Q_{1}\leftrightarrow Q_{3}$ y $Q_{2}\leftrightarrow Q_{4}$ con respectiva tasa de arribo igual a la tasa de salida $\hat{\mu}_{i}=\mu_{i}$, esto se sabe por lo desarrollado en la secci\'on anterior.  

Consideremos, sin p\'erdida de generalidad como base del an\'alisis, la cola $Q_{1}$ adem\'as supongamos al servidor lo comenzamos a observar una vez que termina de atender a la misma para desplazarse y llegar a $Q_{2}$, es decir al tiempo $\tau_{2}$.

Sea $n\in\nat$, $n>0$, ciclo del servidor en que regresa a $Q_{1}$ para dar servicio y atender conforme a la pol\'itica exhaustiva, entonces se tiene que $\overline{\tau}_{1}\left(n\right)$ es el tiempo del servidor en el sistema 1 en que termina de dar servicio a todos los usuarios presentes en la cola, por lo tanto se cumple que $L_{1}\left(\overline{\tau}_{1}\left(n\right)\right)=0$, entonces el servidor para llegar a $Q_{2}$ incurre en un tiempo de traslado $r_{1}$ y por tanto se cumple que $\tau_{2}\left(n\right)=\overline{\tau}_{1}\left(n\right)+r_{1}$. Dado que los tiempos entre arribos son exponenciales se cumple que 

\begin{eqnarray*}
\prob\left\{0 \textrm{ arribos en }Q_{1}\textrm{ en el intervalo }\left[\overline{\tau}_{1}\left(n\right),\overline{\tau}_{1}\left(n\right)+r_{1}\right]\right\}=e^{-\tilde{\mu}_{1}r_{1}},\\
\prob\left\{0 \textrm{ arribos en }Q_{2}\textrm{ en el intervalo }\left[\overline{\tau}_{1}\left(n\right),\overline{\tau}_{1}\left(n\right)+r_{1}\right]\right\}=e^{-\tilde{\mu}_{2}r_{1}}.
\end{eqnarray*}

El evento que nos interesa consiste en que no haya arribos desde que el servidor abandon\'o $Q_{2}$ y regresa nuevamente para dar servicio, es decir en el intervalo de tiempo $\left[\overline{\tau}_{2}\left(n-1\right),\tau_{2}\left(n\right)\right]$. Entonces, si hacemos


\begin{eqnarray*}
\varphi_{1}\left(n\right)&\equiv&\overline{\tau}_{1}\left(n\right)+r_{1}=\overline{\tau}_{2}\left(n-1\right)+r_{1}+r_{2}+\overline{\tau}_{1}\left(n\right)-\tau_{1}\left(n\right)\\
&=&\overline{\tau}_{2}\left(n-1\right)+\overline{\tau}_{1}\left(n\right)-\tau_{1}\left(n\right)+r,,
\end{eqnarray*}

y longitud del intervalo

\begin{eqnarray*}
\xi&\equiv&\overline{\tau}_{1}\left(n\right)+r_{1}-\overline{\tau}_{2}\left(n-1\right)
=\overline{\tau}_{2}\left(n-1\right)+\overline{\tau}_{1}\left(n\right)-\tau_{1}\left(n\right)+r-\overline{\tau}_{2}\left(n-1\right)\\
&=&\overline{\tau}_{1}\left(n\right)-\tau_{1}\left(n\right)+r.
\end{eqnarray*}


Entonces, determinemos la probabilidad del evento no arribos a $Q_{2}$ en $\left[\overline{\tau}_{2}\left(n-1\right),\varphi_{1}\left(n\right)\right]$:

\begin{eqnarray}
\prob\left\{0 \textrm{ arribos en }Q_{2}\textrm{ en el intervalo }\left[\overline{\tau}_{2}\left(n-1\right),\varphi_{1}\left(n\right)\right]\right\}
=e^{-\tilde{\mu}_{2}\xi}.
\end{eqnarray}

De manera an\'aloga, tenemos que la probabilidad de no arribos a $Q_{1}$ en $\left[\overline{\tau}_{2}\left(n-1\right),\varphi_{1}\left(n\right)\right]$ esta dada por

\begin{eqnarray}
\prob\left\{0 \textrm{ arribos en }Q_{1}\textrm{ en el intervalo }\left[\overline{\tau}_{2}\left(n-1\right),\varphi_{1}\left(n\right)\right]\right\}
=e^{-\tilde{\mu}_{1}\xi},
\end{eqnarray}

\begin{eqnarray}
\prob\left\{0 \textrm{ arribos en }Q_{2}\textrm{ en el intervalo }\left[\overline{\tau}_{2}\left(n-1\right),\varphi_{1}\left(n\right)\right]\right\}
=e^{-\tilde{\mu}_{2}\xi}.
\end{eqnarray}

Por tanto 

\begin{eqnarray}
\begin{array}{l}
\prob\left\{0 \textrm{ arribos en }Q_{1}\textrm{ y }Q_{2}\textrm{ en el intervalo }\left[\overline{\tau}_{2}\left(n-1\right),\varphi_{1}\left(n\right)\right]\right\}\\
=\prob\left\{0 \textrm{ arribos en }Q_{1}\textrm{ en el intervalo }\left[\overline{\tau}_{2}\left(n-1\right),\varphi_{1}\left(n\right)\right]\right\}\\
\times
\prob\left\{0 \textrm{ arribos en }Q_{2}\textrm{ en el intervalo }\left[\overline{\tau}_{2}\left(n-1\right),\varphi_{1}\left(n\right)\right]\right\}=e^{-\tilde{\mu}_{1}\xi}e^{-\tilde{\mu}_{2}\xi}
=e^{-\tilde{\mu}\xi}.
\end{array}
\end{eqnarray}

Para el segundo sistema, consideremos nuevamente $\overline{\tau}_{1}\left(n\right)+r_{1}$, sin p\'erdida de generalidad podemos suponer que existe $m>0$ tal que $\overline{\tau}_{3}\left(m\right)<\overline{\tau}_{1}+r_{1}<\tau_{4}\left(m\right)$, entonces

\begin{eqnarray}
\prob\left\{0 \textrm{ arribos en }Q_{3}\textrm{ en el intervalo }\left[\overline{\tau}_{3}\left(m\right),\overline{\tau}_{1}\left(n\right)+r_{1}\right]\right\}
=e^{-\tilde{\mu}_{3}\xi_{3}},
\end{eqnarray}
donde 
\begin{eqnarray}
\xi_{3}=\overline{\tau}_{1}\left(n\right)+r_{1}-\overline{\tau}_{3}\left(m\right)=
\overline{\tau}_{1}\left(n\right)-\overline{\tau}_{3}\left(m\right)+r_{1},
\end{eqnarray}

mientras que para $Q_{4}$ al igual que con $Q_{2}$ escribiremos $\tau_{4}\left(m\right)$ en t\'erminos de $\overline{\tau}_{4}\left(m-1\right)$:

$\varphi_{2}\equiv\tau_{4}\left(m\right)=\overline{\tau}_{4}\left(m-1\right)+r_{4}+\overline{\tau}_{3}\left(m\right)
-\tau_{3}\left(m\right)+r_{3}=\overline{\tau}_{4}\left(m-1\right)+\overline{\tau}_{3}\left(m\right)
-\tau_{3}\left(m\right)+\hat{r}$, adem\'as,

$\xi_{2}\equiv\varphi_{2}\left(m\right)-\overline{\tau}_{1}-r_{1}=\overline{\tau}_{4}\left(m-1\right)+\overline{\tau}_{3}\left(m\right)
-\tau_{3}\left(m\right)-\overline{\tau}_{1}\left(n\right)+\hat{r}-r_{1}$. 

Entonces


\begin{eqnarray}
\prob\left\{0 \textrm{ arribos en }Q_{4}\textrm{ en el intervalo }\left[\overline{\tau}_{1}\left(n\right)+r_{1},\varphi_{2}\left(m\right)\right]\right\}
=e^{-\tilde{\mu}_{4}\xi_{2}},
\end{eqnarray}

mientras que para $Q_{3}$ se tiene que 

\begin{eqnarray}
\prob\left\{0 \textrm{ arribos en }Q_{3}\textrm{ en el intervalo }\left[\overline{\tau}_{1}\left(n\right)+r_{1},\varphi_{2}\left(m\right)\right]\right\}
=e^{-\tilde{\mu}_{3}\xi_{2}}
\end{eqnarray}

Por tanto

\begin{eqnarray}
\prob\left\{0 \textrm{ arribos en }Q_{3}\wedge Q_{4}\textrm{ en el intervalo }\left[\overline{\tau}_{1}\left(n\right)+r_{1},\varphi_{2}\left(m\right)\right]\right\}
=e^{-\hat{\mu}\xi_{2}}
\end{eqnarray}
donde $\hat{\mu}=\tilde{\mu}_{3}+\tilde{\mu}_{4}$.

Ahora, definamos los intervalos $\mathcal{I}_{1}=\left[\overline{\tau}_{1}\left(n\right)+r_{1},\varphi_{1}\left(n\right)\right]$  y $\mathcal{I}_{2}=\left[\overline{\tau}_{1}\left(n\right)+r_{1},\varphi_{2}\left(m\right)\right]$, entonces, sea $\mathcal{I}=\mathcal{I}_{1}\cap\mathcal{I}_{2}$ el intervalo donde cada una de las colas se encuentran vac\'ias, es decir, si tomamos $T^{*}\in\mathcal{I}$, entonces  $L_{1}\left(T^{*}\right)=L_{2}\left(T^{*}\right)=L_{3}\left(T^{*}\right)=L_{4}\left(T^{*}\right)=0$.

Ahora, dado que por construcci\'on $\mathcal{I}\neq\emptyset$ y que para $T^{*}\in\mathcal{I}$ en ninguna de las colas han llegado usuarios, se tiene que no hay transferencia entre las colas, por lo tanto, el sistema 1 y el sistema 2 son condicionalmente independientes en $\mathcal{I}$, es decir

\begin{eqnarray}
\prob\left\{L_{1}\left(T^{*}\right),L_{2}\left(T^{*}\right),L_{3}\left(T^{*}\right),L_{4}\left(T^{*}\right)|T^{*}\in\mathcal{I}\right\}=\prod_{j=1}^{4}\prob\left\{L_{j}\left(T^{*}\right)\right\},
\end{eqnarray}

para $T^{*}\in\mathcal{I}$. 

%\newpage























%________________________________________________________________________
%\section{Procesos Regenerativos}
%________________________________________________________________________

%________________________________________________________________________
%\subsection*{Procesos Regenerativos Sigman, Thorisson y Wolff \cite{Sigman1}}
%________________________________________________________________________


\begin{Def}[Definici\'on Cl\'asica]
Un proceso estoc\'astico $X=\left\{X\left(t\right):t\geq0\right\}$ es llamado regenerativo is existe una variable aleatoria $R_{1}>0$ tal que
\begin{itemize}
\item[i)] $\left\{X\left(t+R_{1}\right):t\geq0\right\}$ es independiente de $\left\{\left\{X\left(t\right):t<R_{1}\right\},\right\}$
\item[ii)] $\left\{X\left(t+R_{1}\right):t\geq0\right\}$ es estoc\'asticamente equivalente a $\left\{X\left(t\right):t>0\right\}$
\end{itemize}

Llamamos a $R_{1}$ tiempo de regeneraci\'on, y decimos que $X$ se regenera en este punto.
\end{Def}

$\left\{X\left(t+R_{1}\right)\right\}$ es regenerativo con tiempo de regeneraci\'on $R_{2}$, independiente de $R_{1}$ pero con la misma distribuci\'on que $R_{1}$. Procediendo de esta manera se obtiene una secuencia de variables aleatorias independientes e id\'enticamente distribuidas $\left\{R_{n}\right\}$ llamados longitudes de ciclo. Si definimos a $Z_{k}\equiv R_{1}+R_{2}+\cdots+R_{k}$, se tiene un proceso de renovaci\'on llamado proceso de renovaci\'on encajado para $X$.


\begin{Note}
La existencia de un primer tiempo de regeneraci\'on, $R_{1}$, implica la existencia de una sucesi\'on completa de estos tiempos $R_{1},R_{2}\ldots,$ que satisfacen la propiedad deseada \cite{Sigman2}.
\end{Note}


\begin{Note} Para la cola $GI/GI/1$ los usuarios arriban con tiempos $t_{n}$ y son atendidos con tiempos de servicio $S_{n}$, los tiempos de arribo forman un proceso de renovaci\'on  con tiempos entre arribos independientes e identicamente distribuidos (\texttt{i.i.d.})$T_{n}=t_{n}-t_{n-1}$, adem\'as los tiempos de servicio son \texttt{i.i.d.} e independientes de los procesos de arribo. Por \textit{estable} se entiende que $\esp S_{n}<\esp T_{n}<\infty$.
\end{Note}
 


\begin{Def}
Para $x$ fijo y para cada $t\geq0$, sea $I_{x}\left(t\right)=1$ si $X\left(t\right)\leq x$,  $I_{x}\left(t\right)=0$ en caso contrario, y def\'inanse los tiempos promedio
\begin{eqnarray*}
\overline{X}&=&lim_{t\rightarrow\infty}\frac{1}{t}\int_{0}^{\infty}X\left(u\right)du\\
\prob\left(X_{\infty}\leq x\right)&=&lim_{t\rightarrow\infty}\frac{1}{t}\int_{0}^{\infty}I_{x}\left(u\right)du,
\end{eqnarray*}
cuando estos l\'imites existan.
\end{Def}

Como consecuencia del teorema de Renovaci\'on-Recompensa, se tiene que el primer l\'imite  existe y es igual a la constante
\begin{eqnarray*}
\overline{X}&=&\frac{\esp\left[\int_{0}^{R_{1}}X\left(t\right)dt\right]}{\esp\left[R_{1}\right]},
\end{eqnarray*}
suponiendo que ambas esperanzas son finitas.
 
\begin{Note}
Funciones de procesos regenerativos son regenerativas, es decir, si $X\left(t\right)$ es regenerativo y se define el proceso $Y\left(t\right)$ por $Y\left(t\right)=f\left(X\left(t\right)\right)$ para alguna funci\'on Borel medible $f\left(\cdot\right)$. Adem\'as $Y$ es regenerativo con los mismos tiempos de renovaci\'on que $X$. 

En general, los tiempos de renovaci\'on, $Z_{k}$ de un proceso regenerativo no requieren ser tiempos de paro con respecto a la evoluci\'on de $X\left(t\right)$.
\end{Note} 

\begin{Note}
Una funci\'on de un proceso de Markov, usualmente no ser\'a un proceso de Markov, sin embargo ser\'a regenerativo si el proceso de Markov lo es.
\end{Note}

 
\begin{Note}
Un proceso regenerativo con media de la longitud de ciclo finita es llamado positivo recurrente.
\end{Note}


\begin{Note}
\begin{itemize}
\item[a)] Si el proceso regenerativo $X$ es positivo recurrente y tiene trayectorias muestrales no negativas, entonces la ecuaci\'on anterior es v\'alida.
\item[b)] Si $X$ es positivo recurrente regenerativo, podemos construir una \'unica versi\'on estacionaria de este proceso, $X_{e}=\left\{X_{e}\left(t\right)\right\}$, donde $X_{e}$ es un proceso estoc\'astico regenerativo y estrictamente estacionario, con distribuci\'on marginal distribuida como $X_{\infty}$
\end{itemize}
\end{Note}


%__________________________________________________________________________________________
%\subsection*{Procesos Regenerativos Estacionarios - Stidham \cite{Stidham}}
%__________________________________________________________________________________________


Un proceso estoc\'astico a tiempo continuo $\left\{V\left(t\right),t\geq0\right\}$ es un proceso regenerativo si existe una sucesi\'on de variables aleatorias independientes e id\'enticamente distribuidas $\left\{X_{1},X_{2},\ldots\right\}$, sucesi\'on de renovaci\'on, tal que para cualquier conjunto de Borel $A$, 

\begin{eqnarray*}
\prob\left\{V\left(t\right)\in A|X_{1}+X_{2}+\cdots+X_{R\left(t\right)}=s,\left\{V\left(\tau\right),\tau<s\right\}\right\}=\prob\left\{V\left(t-s\right)\in A|X_{1}>t-s\right\},
\end{eqnarray*}
para todo $0\leq s\leq t$, donde $R\left(t\right)=\max\left\{X_{1}+X_{2}+\cdots+X_{j}\leq t\right\}=$n\'umero de renovaciones ({\emph{puntos de regeneraci\'on}}) que ocurren en $\left[0,t\right]$. El intervalo $\left[0,X_{1}\right)$ es llamado {\emph{primer ciclo de regeneraci\'on}} de $\left\{V\left(t \right),t\geq0\right\}$, $\left[X_{1},X_{1}+X_{2}\right)$ el {\emph{segundo ciclo de regeneraci\'on}}, y as\'i sucesivamente.

Sea $X=X_{1}$ y sea $F$ la funci\'on de distrbuci\'on de $X$


\begin{Def}
Se define el proceso estacionario, $\left\{V^{*}\left(t\right),t\geq0\right\}$, para $\left\{V\left(t\right),t\geq0\right\}$ por

\begin{eqnarray*}
\prob\left\{V\left(t\right)\in A\right\}=\frac{1}{\esp\left[X\right]}\int_{0}^{\infty}\prob\left\{V\left(t+x\right)\in A|X>x\right\}\left(1-F\left(x\right)\right)dx,
\end{eqnarray*} 
para todo $t\geq0$ y todo conjunto de Borel $A$.
\end{Def}

\begin{Def}
Una distribuci\'on se dice que es {\emph{aritm\'etica}} si todos sus puntos de incremento son m\'ultiplos de la forma $0,\lambda, 2\lambda,\ldots$ para alguna $\lambda>0$ entera.
\end{Def}


\begin{Def}
Una modificaci\'on medible de un proceso $\left\{V\left(t\right),t\geq0\right\}$, es una versi\'on de este, $\left\{V\left(t,w\right)\right\}$ conjuntamente medible para $t\geq0$ y para $w\in S$, $S$ espacio de estados para $\left\{V\left(t\right),t\geq0\right\}$.
\end{Def}

\begin{Teo}
Sea $\left\{V\left(t\right),t\geq\right\}$ un proceso regenerativo no negativo con modificaci\'on medible. Sea $\esp\left[X\right]<\infty$. Entonces el proceso estacionario dado por la ecuaci\'on anterior est\'a bien definido y tiene funci\'on de distribuci\'on independiente de $t$, adem\'as
\begin{itemize}
\item[i)] \begin{eqnarray*}
\esp\left[V^{*}\left(0\right)\right]&=&\frac{\esp\left[\int_{0}^{X}V\left(s\right)ds\right]}{\esp\left[X\right]}\end{eqnarray*}
\item[ii)] Si $\esp\left[V^{*}\left(0\right)\right]<\infty$, equivalentemente, si $\esp\left[\int_{0}^{X}V\left(s\right)ds\right]<\infty$,entonces
\begin{eqnarray*}
\frac{\int_{0}^{t}V\left(s\right)ds}{t}\rightarrow\frac{\esp\left[\int_{0}^{X}V\left(s\right)ds\right]}{\esp\left[X\right]}
\end{eqnarray*}
con probabilidad 1 y en media, cuando $t\rightarrow\infty$.
\end{itemize}
\end{Teo}

\begin{Coro}
Sea $\left\{V\left(t\right),t\geq0\right\}$ un proceso regenerativo no negativo, con modificaci\'on medible. Si $\esp <\infty$, $F$ es no-aritm\'etica, y para todo $x\geq0$, $P\left\{V\left(t\right)\leq x,C>x\right\}$ es de variaci\'on acotada como funci\'on de $t$ en cada intervalo finito $\left[0,\tau\right]$, entonces $V\left(t\right)$ converge en distribuci\'on  cuando $t\rightarrow\infty$ y $$\esp V=\frac{\esp \int_{0}^{X}V\left(s\right)ds}{\esp X}$$
Donde $V$ tiene la distribuci\'on l\'imite de $V\left(t\right)$ cuando $t\rightarrow\infty$.

\end{Coro}

Para el caso discreto se tienen resultados similares.



%______________________________________________________________________
%\section{Procesos de Renovaci\'on}
%______________________________________________________________________

\begin{Def}\label{Def.Tn}
Sean $0\leq T_{1}\leq T_{2}\leq \ldots$ son tiempos aleatorios infinitos en los cuales ocurren ciertos eventos. El n\'umero de tiempos $T_{n}$ en el intervalo $\left[0,t\right)$ es

\begin{eqnarray}
N\left(t\right)=\sum_{n=1}^{\infty}\indora\left(T_{n}\leq t\right),
\end{eqnarray}
para $t\geq0$.
\end{Def}

Si se consideran los puntos $T_{n}$ como elementos de $\rea_{+}$, y $N\left(t\right)$ es el n\'umero de puntos en $\rea$. El proceso denotado por $\left\{N\left(t\right):t\geq0\right\}$, denotado por $N\left(t\right)$, es un proceso puntual en $\rea_{+}$. Los $T_{n}$ son los tiempos de ocurrencia, el proceso puntual $N\left(t\right)$ es simple si su n\'umero de ocurrencias son distintas: $0<T_{1}<T_{2}<\ldots$ casi seguramente.

\begin{Def}
Un proceso puntual $N\left(t\right)$ es un proceso de renovaci\'on si los tiempos de interocurrencia $\xi_{n}=T_{n}-T_{n-1}$, para $n\geq1$, son independientes e identicamente distribuidos con distribuci\'on $F$, donde $F\left(0\right)=0$ y $T_{0}=0$. Los $T_{n}$ son llamados tiempos de renovaci\'on, referente a la independencia o renovaci\'on de la informaci\'on estoc\'astica en estos tiempos. Los $\xi_{n}$ son los tiempos de inter-renovaci\'on, y $N\left(t\right)$ es el n\'umero de renovaciones en el intervalo $\left[0,t\right)$
\end{Def}


\begin{Note}
Para definir un proceso de renovaci\'on para cualquier contexto, solamente hay que especificar una distribuci\'on $F$, con $F\left(0\right)=0$, para los tiempos de inter-renovaci\'on. La funci\'on $F$ en turno degune las otra variables aleatorias. De manera formal, existe un espacio de probabilidad y una sucesi\'on de variables aleatorias $\xi_{1},\xi_{2},\ldots$ definidas en este con distribuci\'on $F$. Entonces las otras cantidades son $T_{n}=\sum_{k=1}^{n}\xi_{k}$ y $N\left(t\right)=\sum_{n=1}^{\infty}\indora\left(T_{n}\leq t\right)$, donde $T_{n}\rightarrow\infty$ casi seguramente por la Ley Fuerte de los Grandes Números.
\end{Note}

%___________________________________________________________________________________________
%
%\subsection*{Teorema Principal de Renovaci\'on}
%___________________________________________________________________________________________
%

\begin{Note} Una funci\'on $h:\rea_{+}\rightarrow\rea$ es Directamente Riemann Integrable en los siguientes casos:
\begin{itemize}
\item[a)] $h\left(t\right)\geq0$ es decreciente y Riemann Integrable.
\item[b)] $h$ es continua excepto posiblemente en un conjunto de Lebesgue de medida 0, y $|h\left(t\right)|\leq b\left(t\right)$, donde $b$ es DRI.
\end{itemize}
\end{Note}

\begin{Teo}[Teorema Principal de Renovaci\'on]
Si $F$ es no aritm\'etica y $h\left(t\right)$ es Directamente Riemann Integrable (DRI), entonces

\begin{eqnarray*}
lim_{t\rightarrow\infty}U\star h=\frac{1}{\mu}\int_{\rea_{+}}h\left(s\right)ds.
\end{eqnarray*}
\end{Teo}

\begin{Prop}
Cualquier funci\'on $H\left(t\right)$ acotada en intervalos finitos y que es 0 para $t<0$ puede expresarse como
\begin{eqnarray*}
H\left(t\right)=U\star h\left(t\right)\textrm{,  donde }h\left(t\right)=H\left(t\right)-F\star H\left(t\right)
\end{eqnarray*}
\end{Prop}

\begin{Def}
Un proceso estoc\'astico $X\left(t\right)$ es crudamente regenerativo en un tiempo aleatorio positivo $T$ si
\begin{eqnarray*}
\esp\left[X\left(T+t\right)|T\right]=\esp\left[X\left(t\right)\right]\textrm{, para }t\geq0,\end{eqnarray*}
y con las esperanzas anteriores finitas.
\end{Def}

\begin{Prop}
Sup\'ongase que $X\left(t\right)$ es un proceso crudamente regenerativo en $T$, que tiene distribuci\'on $F$. Si $\esp\left[X\left(t\right)\right]$ es acotado en intervalos finitos, entonces
\begin{eqnarray*}
\esp\left[X\left(t\right)\right]=U\star h\left(t\right)\textrm{,  donde }h\left(t\right)=\esp\left[X\left(t\right)\indora\left(T>t\right)\right].
\end{eqnarray*}
\end{Prop}

\begin{Teo}[Regeneraci\'on Cruda]
Sup\'ongase que $X\left(t\right)$ es un proceso con valores positivo crudamente regenerativo en $T$, y def\'inase $M=\sup\left\{|X\left(t\right)|:t\leq T\right\}$. Si $T$ es no aritm\'etico y $M$ y $MT$ tienen media finita, entonces
\begin{eqnarray*}
lim_{t\rightarrow\infty}\esp\left[X\left(t\right)\right]=\frac{1}{\mu}\int_{\rea_{+}}h\left(s\right)ds,
\end{eqnarray*}
donde $h\left(t\right)=\esp\left[X\left(t\right)\indora\left(T>t\right)\right]$.
\end{Teo}

%___________________________________________________________________________________________
%
%\subsection*{Propiedades de los Procesos de Renovaci\'on}
%___________________________________________________________________________________________
%

Los tiempos $T_{n}$ est\'an relacionados con los conteos de $N\left(t\right)$ por

\begin{eqnarray*}
\left\{N\left(t\right)\geq n\right\}&=&\left\{T_{n}\leq t\right\}\\
T_{N\left(t\right)}\leq &t&<T_{N\left(t\right)+1},
\end{eqnarray*}

adem\'as $N\left(T_{n}\right)=n$, y 

\begin{eqnarray*}
N\left(t\right)=\max\left\{n:T_{n}\leq t\right\}=\min\left\{n:T_{n+1}>t\right\}
\end{eqnarray*}

Por propiedades de la convoluci\'on se sabe que

\begin{eqnarray*}
P\left\{T_{n}\leq t\right\}=F^{n\star}\left(t\right)
\end{eqnarray*}
que es la $n$-\'esima convoluci\'on de $F$. Entonces 

\begin{eqnarray*}
\left\{N\left(t\right)\geq n\right\}&=&\left\{T_{n}\leq t\right\}\\
P\left\{N\left(t\right)\leq n\right\}&=&1-F^{\left(n+1\right)\star}\left(t\right)
\end{eqnarray*}

Adem\'as usando el hecho de que $\esp\left[N\left(t\right)\right]=\sum_{n=1}^{\infty}P\left\{N\left(t\right)\geq n\right\}$
se tiene que

\begin{eqnarray*}
\esp\left[N\left(t\right)\right]=\sum_{n=1}^{\infty}F^{n\star}\left(t\right)
\end{eqnarray*}

\begin{Prop}
Para cada $t\geq0$, la funci\'on generadora de momentos $\esp\left[e^{\alpha N\left(t\right)}\right]$ existe para alguna $\alpha$ en una vecindad del 0, y de aqu\'i que $\esp\left[N\left(t\right)^{m}\right]<\infty$, para $m\geq1$.
\end{Prop}


\begin{Note}
Si el primer tiempo de renovaci\'on $\xi_{1}$ no tiene la misma distribuci\'on que el resto de las $\xi_{n}$, para $n\geq2$, a $N\left(t\right)$ se le llama Proceso de Renovaci\'on retardado, donde si $\xi$ tiene distribuci\'on $G$, entonces el tiempo $T_{n}$ de la $n$-\'esima renovaci\'on tiene distribuci\'on $G\star F^{\left(n-1\right)\star}\left(t\right)$
\end{Note}


\begin{Teo}
Para una constante $\mu\leq\infty$ ( o variable aleatoria), las siguientes expresiones son equivalentes:

\begin{eqnarray}
lim_{n\rightarrow\infty}n^{-1}T_{n}&=&\mu,\textrm{ c.s.}\\
lim_{t\rightarrow\infty}t^{-1}N\left(t\right)&=&1/\mu,\textrm{ c.s.}
\end{eqnarray}
\end{Teo}


Es decir, $T_{n}$ satisface la Ley Fuerte de los Grandes N\'umeros s\'i y s\'olo s\'i $N\left/t\right)$ la cumple.


\begin{Coro}[Ley Fuerte de los Grandes N\'umeros para Procesos de Renovaci\'on]
Si $N\left(t\right)$ es un proceso de renovaci\'on cuyos tiempos de inter-renovaci\'on tienen media $\mu\leq\infty$, entonces
\begin{eqnarray}
t^{-1}N\left(t\right)\rightarrow 1/\mu,\textrm{ c.s. cuando }t\rightarrow\infty.
\end{eqnarray}

\end{Coro}


Considerar el proceso estoc\'astico de valores reales $\left\{Z\left(t\right):t\geq0\right\}$ en el mismo espacio de probabilidad que $N\left(t\right)$

\begin{Def}
Para el proceso $\left\{Z\left(t\right):t\geq0\right\}$ se define la fluctuaci\'on m\'axima de $Z\left(t\right)$ en el intervalo $\left(T_{n-1},T_{n}\right]$:
\begin{eqnarray*}
M_{n}=\sup_{T_{n-1}<t\leq T_{n}}|Z\left(t\right)-Z\left(T_{n-1}\right)|
\end{eqnarray*}
\end{Def}

\begin{Teo}
Sup\'ongase que $n^{-1}T_{n}\rightarrow\mu$ c.s. cuando $n\rightarrow\infty$, donde $\mu\leq\infty$ es una constante o variable aleatoria. Sea $a$ una constante o variable aleatoria que puede ser infinita cuando $\mu$ es finita, y considere las expresiones l\'imite:
\begin{eqnarray}
lim_{n\rightarrow\infty}n^{-1}Z\left(T_{n}\right)&=&a,\textrm{ c.s.}\\
lim_{t\rightarrow\infty}t^{-1}Z\left(t\right)&=&a/\mu,\textrm{ c.s.}
\end{eqnarray}
La segunda expresi\'on implica la primera. Conversamente, la primera implica la segunda si el proceso $Z\left(t\right)$ es creciente, o si $lim_{n\rightarrow\infty}n^{-1}M_{n}=0$ c.s.
\end{Teo}

\begin{Coro}
Si $N\left(t\right)$ es un proceso de renovaci\'on, y $\left(Z\left(T_{n}\right)-Z\left(T_{n-1}\right),M_{n}\right)$, para $n\geq1$, son variables aleatorias independientes e id\'enticamente distribuidas con media finita, entonces,
\begin{eqnarray}
lim_{t\rightarrow\infty}t^{-1}Z\left(t\right)\rightarrow\frac{\esp\left[Z\left(T_{1}\right)-Z\left(T_{0}\right)\right]}{\esp\left[T_{1}\right]},\textrm{ c.s. cuando  }t\rightarrow\infty.
\end{eqnarray}
\end{Coro}



%___________________________________________________________________________________________
%
%\subsection{Propiedades de los Procesos de Renovaci\'on}
%___________________________________________________________________________________________
%

Los tiempos $T_{n}$ est\'an relacionados con los conteos de $N\left(t\right)$ por

\begin{eqnarray*}
\left\{N\left(t\right)\geq n\right\}&=&\left\{T_{n}\leq t\right\}\\
T_{N\left(t\right)}\leq &t&<T_{N\left(t\right)+1},
\end{eqnarray*}

adem\'as $N\left(T_{n}\right)=n$, y 

\begin{eqnarray*}
N\left(t\right)=\max\left\{n:T_{n}\leq t\right\}=\min\left\{n:T_{n+1}>t\right\}
\end{eqnarray*}

Por propiedades de la convoluci\'on se sabe que

\begin{eqnarray*}
P\left\{T_{n}\leq t\right\}=F^{n\star}\left(t\right)
\end{eqnarray*}
que es la $n$-\'esima convoluci\'on de $F$. Entonces 

\begin{eqnarray*}
\left\{N\left(t\right)\geq n\right\}&=&\left\{T_{n}\leq t\right\}\\
P\left\{N\left(t\right)\leq n\right\}&=&1-F^{\left(n+1\right)\star}\left(t\right)
\end{eqnarray*}

Adem\'as usando el hecho de que $\esp\left[N\left(t\right)\right]=\sum_{n=1}^{\infty}P\left\{N\left(t\right)\geq n\right\}$
se tiene que

\begin{eqnarray*}
\esp\left[N\left(t\right)\right]=\sum_{n=1}^{\infty}F^{n\star}\left(t\right)
\end{eqnarray*}

\begin{Prop}
Para cada $t\geq0$, la funci\'on generadora de momentos $\esp\left[e^{\alpha N\left(t\right)}\right]$ existe para alguna $\alpha$ en una vecindad del 0, y de aqu\'i que $\esp\left[N\left(t\right)^{m}\right]<\infty$, para $m\geq1$.
\end{Prop}


\begin{Note}
Si el primer tiempo de renovaci\'on $\xi_{1}$ no tiene la misma distribuci\'on que el resto de las $\xi_{n}$, para $n\geq2$, a $N\left(t\right)$ se le llama Proceso de Renovaci\'on retardado, donde si $\xi$ tiene distribuci\'on $G$, entonces el tiempo $T_{n}$ de la $n$-\'esima renovaci\'on tiene distribuci\'on $G\star F^{\left(n-1\right)\star}\left(t\right)$
\end{Note}


\begin{Teo}
Para una constante $\mu\leq\infty$ ( o variable aleatoria), las siguientes expresiones son equivalentes:

\begin{eqnarray}
lim_{n\rightarrow\infty}n^{-1}T_{n}&=&\mu,\textrm{ c.s.}\\
lim_{t\rightarrow\infty}t^{-1}N\left(t\right)&=&1/\mu,\textrm{ c.s.}
\end{eqnarray}
\end{Teo}


Es decir, $T_{n}$ satisface la Ley Fuerte de los Grandes N\'umeros s\'i y s\'olo s\'i $N\left/t\right)$ la cumple.


\begin{Coro}[Ley Fuerte de los Grandes N\'umeros para Procesos de Renovaci\'on]
Si $N\left(t\right)$ es un proceso de renovaci\'on cuyos tiempos de inter-renovaci\'on tienen media $\mu\leq\infty$, entonces
\begin{eqnarray}
t^{-1}N\left(t\right)\rightarrow 1/\mu,\textrm{ c.s. cuando }t\rightarrow\infty.
\end{eqnarray}

\end{Coro}


Considerar el proceso estoc\'astico de valores reales $\left\{Z\left(t\right):t\geq0\right\}$ en el mismo espacio de probabilidad que $N\left(t\right)$

\begin{Def}
Para el proceso $\left\{Z\left(t\right):t\geq0\right\}$ se define la fluctuaci\'on m\'axima de $Z\left(t\right)$ en el intervalo $\left(T_{n-1},T_{n}\right]$:
\begin{eqnarray*}
M_{n}=\sup_{T_{n-1}<t\leq T_{n}}|Z\left(t\right)-Z\left(T_{n-1}\right)|
\end{eqnarray*}
\end{Def}

\begin{Teo}
Sup\'ongase que $n^{-1}T_{n}\rightarrow\mu$ c.s. cuando $n\rightarrow\infty$, donde $\mu\leq\infty$ es una constante o variable aleatoria. Sea $a$ una constante o variable aleatoria que puede ser infinita cuando $\mu$ es finita, y considere las expresiones l\'imite:
\begin{eqnarray}
lim_{n\rightarrow\infty}n^{-1}Z\left(T_{n}\right)&=&a,\textrm{ c.s.}\\
lim_{t\rightarrow\infty}t^{-1}Z\left(t\right)&=&a/\mu,\textrm{ c.s.}
\end{eqnarray}
La segunda expresi\'on implica la primera. Conversamente, la primera implica la segunda si el proceso $Z\left(t\right)$ es creciente, o si $lim_{n\rightarrow\infty}n^{-1}M_{n}=0$ c.s.
\end{Teo}

\begin{Coro}
Si $N\left(t\right)$ es un proceso de renovaci\'on, y $\left(Z\left(T_{n}\right)-Z\left(T_{n-1}\right),M_{n}\right)$, para $n\geq1$, son variables aleatorias independientes e id\'enticamente distribuidas con media finita, entonces,
\begin{eqnarray}
lim_{t\rightarrow\infty}t^{-1}Z\left(t\right)\rightarrow\frac{\esp\left[Z\left(T_{1}\right)-Z\left(T_{0}\right)\right]}{\esp\left[T_{1}\right]},\textrm{ c.s. cuando  }t\rightarrow\infty.
\end{eqnarray}
\end{Coro}


%___________________________________________________________________________________________
%
%\subsection{Propiedades de los Procesos de Renovaci\'on}
%___________________________________________________________________________________________
%

Los tiempos $T_{n}$ est\'an relacionados con los conteos de $N\left(t\right)$ por

\begin{eqnarray*}
\left\{N\left(t\right)\geq n\right\}&=&\left\{T_{n}\leq t\right\}\\
T_{N\left(t\right)}\leq &t&<T_{N\left(t\right)+1},
\end{eqnarray*}

adem\'as $N\left(T_{n}\right)=n$, y 

\begin{eqnarray*}
N\left(t\right)=\max\left\{n:T_{n}\leq t\right\}=\min\left\{n:T_{n+1}>t\right\}
\end{eqnarray*}

Por propiedades de la convoluci\'on se sabe que

\begin{eqnarray*}
P\left\{T_{n}\leq t\right\}=F^{n\star}\left(t\right)
\end{eqnarray*}
que es la $n$-\'esima convoluci\'on de $F$. Entonces 

\begin{eqnarray*}
\left\{N\left(t\right)\geq n\right\}&=&\left\{T_{n}\leq t\right\}\\
P\left\{N\left(t\right)\leq n\right\}&=&1-F^{\left(n+1\right)\star}\left(t\right)
\end{eqnarray*}

Adem\'as usando el hecho de que $\esp\left[N\left(t\right)\right]=\sum_{n=1}^{\infty}P\left\{N\left(t\right)\geq n\right\}$
se tiene que

\begin{eqnarray*}
\esp\left[N\left(t\right)\right]=\sum_{n=1}^{\infty}F^{n\star}\left(t\right)
\end{eqnarray*}

\begin{Prop}
Para cada $t\geq0$, la funci\'on generadora de momentos $\esp\left[e^{\alpha N\left(t\right)}\right]$ existe para alguna $\alpha$ en una vecindad del 0, y de aqu\'i que $\esp\left[N\left(t\right)^{m}\right]<\infty$, para $m\geq1$.
\end{Prop}


\begin{Note}
Si el primer tiempo de renovaci\'on $\xi_{1}$ no tiene la misma distribuci\'on que el resto de las $\xi_{n}$, para $n\geq2$, a $N\left(t\right)$ se le llama Proceso de Renovaci\'on retardado, donde si $\xi$ tiene distribuci\'on $G$, entonces el tiempo $T_{n}$ de la $n$-\'esima renovaci\'on tiene distribuci\'on $G\star F^{\left(n-1\right)\star}\left(t\right)$
\end{Note}


\begin{Teo}
Para una constante $\mu\leq\infty$ ( o variable aleatoria), las siguientes expresiones son equivalentes:

\begin{eqnarray}
lim_{n\rightarrow\infty}n^{-1}T_{n}&=&\mu,\textrm{ c.s.}\\
lim_{t\rightarrow\infty}t^{-1}N\left(t\right)&=&1/\mu,\textrm{ c.s.}
\end{eqnarray}
\end{Teo}


Es decir, $T_{n}$ satisface la Ley Fuerte de los Grandes N\'umeros s\'i y s\'olo s\'i $N\left/t\right)$ la cumple.


\begin{Coro}[Ley Fuerte de los Grandes N\'umeros para Procesos de Renovaci\'on]
Si $N\left(t\right)$ es un proceso de renovaci\'on cuyos tiempos de inter-renovaci\'on tienen media $\mu\leq\infty$, entonces
\begin{eqnarray}
t^{-1}N\left(t\right)\rightarrow 1/\mu,\textrm{ c.s. cuando }t\rightarrow\infty.
\end{eqnarray}

\end{Coro}


Considerar el proceso estoc\'astico de valores reales $\left\{Z\left(t\right):t\geq0\right\}$ en el mismo espacio de probabilidad que $N\left(t\right)$

\begin{Def}
Para el proceso $\left\{Z\left(t\right):t\geq0\right\}$ se define la fluctuaci\'on m\'axima de $Z\left(t\right)$ en el intervalo $\left(T_{n-1},T_{n}\right]$:
\begin{eqnarray*}
M_{n}=\sup_{T_{n-1}<t\leq T_{n}}|Z\left(t\right)-Z\left(T_{n-1}\right)|
\end{eqnarray*}
\end{Def}

\begin{Teo}
Sup\'ongase que $n^{-1}T_{n}\rightarrow\mu$ c.s. cuando $n\rightarrow\infty$, donde $\mu\leq\infty$ es una constante o variable aleatoria. Sea $a$ una constante o variable aleatoria que puede ser infinita cuando $\mu$ es finita, y considere las expresiones l\'imite:
\begin{eqnarray}
lim_{n\rightarrow\infty}n^{-1}Z\left(T_{n}\right)&=&a,\textrm{ c.s.}\\
lim_{t\rightarrow\infty}t^{-1}Z\left(t\right)&=&a/\mu,\textrm{ c.s.}
\end{eqnarray}
La segunda expresi\'on implica la primera. Conversamente, la primera implica la segunda si el proceso $Z\left(t\right)$ es creciente, o si $lim_{n\rightarrow\infty}n^{-1}M_{n}=0$ c.s.
\end{Teo}

\begin{Coro}
Si $N\left(t\right)$ es un proceso de renovaci\'on, y $\left(Z\left(T_{n}\right)-Z\left(T_{n-1}\right),M_{n}\right)$, para $n\geq1$, son variables aleatorias independientes e id\'enticamente distribuidas con media finita, entonces,
\begin{eqnarray}
lim_{t\rightarrow\infty}t^{-1}Z\left(t\right)\rightarrow\frac{\esp\left[Z\left(T_{1}\right)-Z\left(T_{0}\right)\right]}{\esp\left[T_{1}\right]},\textrm{ c.s. cuando  }t\rightarrow\infty.
\end{eqnarray}
\end{Coro}

%___________________________________________________________________________________________
%
%\subsection{Propiedades de los Procesos de Renovaci\'on}
%___________________________________________________________________________________________
%

Los tiempos $T_{n}$ est\'an relacionados con los conteos de $N\left(t\right)$ por

\begin{eqnarray*}
\left\{N\left(t\right)\geq n\right\}&=&\left\{T_{n}\leq t\right\}\\
T_{N\left(t\right)}\leq &t&<T_{N\left(t\right)+1},
\end{eqnarray*}

adem\'as $N\left(T_{n}\right)=n$, y 

\begin{eqnarray*}
N\left(t\right)=\max\left\{n:T_{n}\leq t\right\}=\min\left\{n:T_{n+1}>t\right\}
\end{eqnarray*}

Por propiedades de la convoluci\'on se sabe que

\begin{eqnarray*}
P\left\{T_{n}\leq t\right\}=F^{n\star}\left(t\right)
\end{eqnarray*}
que es la $n$-\'esima convoluci\'on de $F$. Entonces 

\begin{eqnarray*}
\left\{N\left(t\right)\geq n\right\}&=&\left\{T_{n}\leq t\right\}\\
P\left\{N\left(t\right)\leq n\right\}&=&1-F^{\left(n+1\right)\star}\left(t\right)
\end{eqnarray*}

Adem\'as usando el hecho de que $\esp\left[N\left(t\right)\right]=\sum_{n=1}^{\infty}P\left\{N\left(t\right)\geq n\right\}$
se tiene que

\begin{eqnarray*}
\esp\left[N\left(t\right)\right]=\sum_{n=1}^{\infty}F^{n\star}\left(t\right)
\end{eqnarray*}

\begin{Prop}
Para cada $t\geq0$, la funci\'on generadora de momentos $\esp\left[e^{\alpha N\left(t\right)}\right]$ existe para alguna $\alpha$ en una vecindad del 0, y de aqu\'i que $\esp\left[N\left(t\right)^{m}\right]<\infty$, para $m\geq1$.
\end{Prop}


\begin{Note}
Si el primer tiempo de renovaci\'on $\xi_{1}$ no tiene la misma distribuci\'on que el resto de las $\xi_{n}$, para $n\geq2$, a $N\left(t\right)$ se le llama Proceso de Renovaci\'on retardado, donde si $\xi$ tiene distribuci\'on $G$, entonces el tiempo $T_{n}$ de la $n$-\'esima renovaci\'on tiene distribuci\'on $G\star F^{\left(n-1\right)\star}\left(t\right)$
\end{Note}


\begin{Teo}
Para una constante $\mu\leq\infty$ ( o variable aleatoria), las siguientes expresiones son equivalentes:

\begin{eqnarray}
lim_{n\rightarrow\infty}n^{-1}T_{n}&=&\mu,\textrm{ c.s.}\\
lim_{t\rightarrow\infty}t^{-1}N\left(t\right)&=&1/\mu,\textrm{ c.s.}
\end{eqnarray}
\end{Teo}


Es decir, $T_{n}$ satisface la Ley Fuerte de los Grandes N\'umeros s\'i y s\'olo s\'i $N\left/t\right)$ la cumple.


\begin{Coro}[Ley Fuerte de los Grandes N\'umeros para Procesos de Renovaci\'on]
Si $N\left(t\right)$ es un proceso de renovaci\'on cuyos tiempos de inter-renovaci\'on tienen media $\mu\leq\infty$, entonces
\begin{eqnarray}
t^{-1}N\left(t\right)\rightarrow 1/\mu,\textrm{ c.s. cuando }t\rightarrow\infty.
\end{eqnarray}

\end{Coro}


Considerar el proceso estoc\'astico de valores reales $\left\{Z\left(t\right):t\geq0\right\}$ en el mismo espacio de probabilidad que $N\left(t\right)$

\begin{Def}
Para el proceso $\left\{Z\left(t\right):t\geq0\right\}$ se define la fluctuaci\'on m\'axima de $Z\left(t\right)$ en el intervalo $\left(T_{n-1},T_{n}\right]$:
\begin{eqnarray*}
M_{n}=\sup_{T_{n-1}<t\leq T_{n}}|Z\left(t\right)-Z\left(T_{n-1}\right)|
\end{eqnarray*}
\end{Def}

\begin{Teo}
Sup\'ongase que $n^{-1}T_{n}\rightarrow\mu$ c.s. cuando $n\rightarrow\infty$, donde $\mu\leq\infty$ es una constante o variable aleatoria. Sea $a$ una constante o variable aleatoria que puede ser infinita cuando $\mu$ es finita, y considere las expresiones l\'imite:
\begin{eqnarray}
lim_{n\rightarrow\infty}n^{-1}Z\left(T_{n}\right)&=&a,\textrm{ c.s.}\\
lim_{t\rightarrow\infty}t^{-1}Z\left(t\right)&=&a/\mu,\textrm{ c.s.}
\end{eqnarray}
La segunda expresi\'on implica la primera. Conversamente, la primera implica la segunda si el proceso $Z\left(t\right)$ es creciente, o si $lim_{n\rightarrow\infty}n^{-1}M_{n}=0$ c.s.
\end{Teo}

\begin{Coro}
Si $N\left(t\right)$ es un proceso de renovaci\'on, y $\left(Z\left(T_{n}\right)-Z\left(T_{n-1}\right),M_{n}\right)$, para $n\geq1$, son variables aleatorias independientes e id\'enticamente distribuidas con media finita, entonces,
\begin{eqnarray}
lim_{t\rightarrow\infty}t^{-1}Z\left(t\right)\rightarrow\frac{\esp\left[Z\left(T_{1}\right)-Z\left(T_{0}\right)\right]}{\esp\left[T_{1}\right]},\textrm{ c.s. cuando  }t\rightarrow\infty.
\end{eqnarray}
\end{Coro}
%___________________________________________________________________________________________
%
%\subsection{Propiedades de los Procesos de Renovaci\'on}
%___________________________________________________________________________________________
%

Los tiempos $T_{n}$ est\'an relacionados con los conteos de $N\left(t\right)$ por

\begin{eqnarray*}
\left\{N\left(t\right)\geq n\right\}&=&\left\{T_{n}\leq t\right\}\\
T_{N\left(t\right)}\leq &t&<T_{N\left(t\right)+1},
\end{eqnarray*}

adem\'as $N\left(T_{n}\right)=n$, y 

\begin{eqnarray*}
N\left(t\right)=\max\left\{n:T_{n}\leq t\right\}=\min\left\{n:T_{n+1}>t\right\}
\end{eqnarray*}

Por propiedades de la convoluci\'on se sabe que

\begin{eqnarray*}
P\left\{T_{n}\leq t\right\}=F^{n\star}\left(t\right)
\end{eqnarray*}
que es la $n$-\'esima convoluci\'on de $F$. Entonces 

\begin{eqnarray*}
\left\{N\left(t\right)\geq n\right\}&=&\left\{T_{n}\leq t\right\}\\
P\left\{N\left(t\right)\leq n\right\}&=&1-F^{\left(n+1\right)\star}\left(t\right)
\end{eqnarray*}

Adem\'as usando el hecho de que $\esp\left[N\left(t\right)\right]=\sum_{n=1}^{\infty}P\left\{N\left(t\right)\geq n\right\}$
se tiene que

\begin{eqnarray*}
\esp\left[N\left(t\right)\right]=\sum_{n=1}^{\infty}F^{n\star}\left(t\right)
\end{eqnarray*}

\begin{Prop}
Para cada $t\geq0$, la funci\'on generadora de momentos $\esp\left[e^{\alpha N\left(t\right)}\right]$ existe para alguna $\alpha$ en una vecindad del 0, y de aqu\'i que $\esp\left[N\left(t\right)^{m}\right]<\infty$, para $m\geq1$.
\end{Prop}


\begin{Note}
Si el primer tiempo de renovaci\'on $\xi_{1}$ no tiene la misma distribuci\'on que el resto de las $\xi_{n}$, para $n\geq2$, a $N\left(t\right)$ se le llama Proceso de Renovaci\'on retardado, donde si $\xi$ tiene distribuci\'on $G$, entonces el tiempo $T_{n}$ de la $n$-\'esima renovaci\'on tiene distribuci\'on $G\star F^{\left(n-1\right)\star}\left(t\right)$
\end{Note}


\begin{Teo}
Para una constante $\mu\leq\infty$ ( o variable aleatoria), las siguientes expresiones son equivalentes:

\begin{eqnarray}
lim_{n\rightarrow\infty}n^{-1}T_{n}&=&\mu,\textrm{ c.s.}\\
lim_{t\rightarrow\infty}t^{-1}N\left(t\right)&=&1/\mu,\textrm{ c.s.}
\end{eqnarray}
\end{Teo}


Es decir, $T_{n}$ satisface la Ley Fuerte de los Grandes N\'umeros s\'i y s\'olo s\'i $N\left/t\right)$ la cumple.


\begin{Coro}[Ley Fuerte de los Grandes N\'umeros para Procesos de Renovaci\'on]
Si $N\left(t\right)$ es un proceso de renovaci\'on cuyos tiempos de inter-renovaci\'on tienen media $\mu\leq\infty$, entonces
\begin{eqnarray}
t^{-1}N\left(t\right)\rightarrow 1/\mu,\textrm{ c.s. cuando }t\rightarrow\infty.
\end{eqnarray}

\end{Coro}


Considerar el proceso estoc\'astico de valores reales $\left\{Z\left(t\right):t\geq0\right\}$ en el mismo espacio de probabilidad que $N\left(t\right)$

\begin{Def}
Para el proceso $\left\{Z\left(t\right):t\geq0\right\}$ se define la fluctuaci\'on m\'axima de $Z\left(t\right)$ en el intervalo $\left(T_{n-1},T_{n}\right]$:
\begin{eqnarray*}
M_{n}=\sup_{T_{n-1}<t\leq T_{n}}|Z\left(t\right)-Z\left(T_{n-1}\right)|
\end{eqnarray*}
\end{Def}

\begin{Teo}
Sup\'ongase que $n^{-1}T_{n}\rightarrow\mu$ c.s. cuando $n\rightarrow\infty$, donde $\mu\leq\infty$ es una constante o variable aleatoria. Sea $a$ una constante o variable aleatoria que puede ser infinita cuando $\mu$ es finita, y considere las expresiones l\'imite:
\begin{eqnarray}
lim_{n\rightarrow\infty}n^{-1}Z\left(T_{n}\right)&=&a,\textrm{ c.s.}\\
lim_{t\rightarrow\infty}t^{-1}Z\left(t\right)&=&a/\mu,\textrm{ c.s.}
\end{eqnarray}
La segunda expresi\'on implica la primera. Conversamente, la primera implica la segunda si el proceso $Z\left(t\right)$ es creciente, o si $lim_{n\rightarrow\infty}n^{-1}M_{n}=0$ c.s.
\end{Teo}

\begin{Coro}
Si $N\left(t\right)$ es un proceso de renovaci\'on, y $\left(Z\left(T_{n}\right)-Z\left(T_{n-1}\right),M_{n}\right)$, para $n\geq1$, son variables aleatorias independientes e id\'enticamente distribuidas con media finita, entonces,
\begin{eqnarray}
lim_{t\rightarrow\infty}t^{-1}Z\left(t\right)\rightarrow\frac{\esp\left[Z\left(T_{1}\right)-Z\left(T_{0}\right)\right]}{\esp\left[T_{1}\right]},\textrm{ c.s. cuando  }t\rightarrow\infty.
\end{eqnarray}
\end{Coro}


%___________________________________________________________________________________________
%
%\subsection*{Funci\'on de Renovaci\'on}
%___________________________________________________________________________________________
%


\begin{Def}
Sea $h\left(t\right)$ funci\'on de valores reales en $\rea$ acotada en intervalos finitos e igual a cero para $t<0$ La ecuaci\'on de renovaci\'on para $h\left(t\right)$ y la distribuci\'on $F$ es

\begin{eqnarray}\label{Ec.Renovacion}
H\left(t\right)=h\left(t\right)+\int_{\left[0,t\right]}H\left(t-s\right)dF\left(s\right)\textrm{,    }t\geq0,
\end{eqnarray}
donde $H\left(t\right)$ es una funci\'on de valores reales. Esto es $H=h+F\star H$. Decimos que $H\left(t\right)$ es soluci\'on de esta ecuaci\'on si satisface la ecuaci\'on, y es acotada en intervalos finitos e iguales a cero para $t<0$.
\end{Def}

\begin{Prop}
La funci\'on $U\star h\left(t\right)$ es la \'unica soluci\'on de la ecuaci\'on de renovaci\'on (\ref{Ec.Renovacion}).
\end{Prop}

\begin{Teo}[Teorema Renovaci\'on Elemental]
\begin{eqnarray*}
t^{-1}U\left(t\right)\rightarrow 1/\mu\textrm{,    cuando }t\rightarrow\infty.
\end{eqnarray*}
\end{Teo}

%___________________________________________________________________________________________
%
%\subsection{Funci\'on de Renovaci\'on}
%___________________________________________________________________________________________
%


Sup\'ongase que $N\left(t\right)$ es un proceso de renovaci\'on con distribuci\'on $F$ con media finita $\mu$.

\begin{Def}
La funci\'on de renovaci\'on asociada con la distribuci\'on $F$, del proceso $N\left(t\right)$, es
\begin{eqnarray*}
U\left(t\right)=\sum_{n=1}^{\infty}F^{n\star}\left(t\right),\textrm{   }t\geq0,
\end{eqnarray*}
donde $F^{0\star}\left(t\right)=\indora\left(t\geq0\right)$.
\end{Def}


\begin{Prop}
Sup\'ongase que la distribuci\'on de inter-renovaci\'on $F$ tiene densidad $f$. Entonces $U\left(t\right)$ tambi\'en tiene densidad, para $t>0$, y es $U^{'}\left(t\right)=\sum_{n=0}^{\infty}f^{n\star}\left(t\right)$. Adem\'as
\begin{eqnarray*}
\prob\left\{N\left(t\right)>N\left(t-\right)\right\}=0\textrm{,   }t\geq0.
\end{eqnarray*}
\end{Prop}

\begin{Def}
La Transformada de Laplace-Stieljes de $F$ est\'a dada por

\begin{eqnarray*}
\hat{F}\left(\alpha\right)=\int_{\rea_{+}}e^{-\alpha t}dF\left(t\right)\textrm{,  }\alpha\geq0.
\end{eqnarray*}
\end{Def}

Entonces

\begin{eqnarray*}
\hat{U}\left(\alpha\right)=\sum_{n=0}^{\infty}\hat{F^{n\star}}\left(\alpha\right)=\sum_{n=0}^{\infty}\hat{F}\left(\alpha\right)^{n}=\frac{1}{1-\hat{F}\left(\alpha\right)}.
\end{eqnarray*}


\begin{Prop}
La Transformada de Laplace $\hat{U}\left(\alpha\right)$ y $\hat{F}\left(\alpha\right)$ determina una a la otra de manera \'unica por la relaci\'on $\hat{U}\left(\alpha\right)=\frac{1}{1-\hat{F}\left(\alpha\right)}$.
\end{Prop}


\begin{Note}
Un proceso de renovaci\'on $N\left(t\right)$ cuyos tiempos de inter-renovaci\'on tienen media finita, es un proceso Poisson con tasa $\lambda$ si y s\'olo s\'i $\esp\left[U\left(t\right)\right]=\lambda t$, para $t\geq0$.
\end{Note}


\begin{Teo}
Sea $N\left(t\right)$ un proceso puntual simple con puntos de localizaci\'on $T_{n}$ tal que $\eta\left(t\right)=\esp\left[N\left(\right)\right]$ es finita para cada $t$. Entonces para cualquier funci\'on $f:\rea_{+}\rightarrow\rea$,
\begin{eqnarray*}
\esp\left[\sum_{n=1}^{N\left(\right)}f\left(T_{n}\right)\right]=\int_{\left(0,t\right]}f\left(s\right)d\eta\left(s\right)\textrm{,  }t\geq0,
\end{eqnarray*}
suponiendo que la integral exista. Adem\'as si $X_{1},X_{2},\ldots$ son variables aleatorias definidas en el mismo espacio de probabilidad que el proceso $N\left(t\right)$ tal que $\esp\left[X_{n}|T_{n}=s\right]=f\left(s\right)$, independiente de $n$. Entonces
\begin{eqnarray*}
\esp\left[\sum_{n=1}^{N\left(t\right)}X_{n}\right]=\int_{\left(0,t\right]}f\left(s\right)d\eta\left(s\right)\textrm{,  }t\geq0,
\end{eqnarray*} 
suponiendo que la integral exista. 
\end{Teo}

\begin{Coro}[Identidad de Wald para Renovaciones]
Para el proceso de renovaci\'on $N\left(t\right)$,
\begin{eqnarray*}
\esp\left[T_{N\left(t\right)+1}\right]=\mu\esp\left[N\left(t\right)+1\right]\textrm{,  }t\geq0,
\end{eqnarray*}  
\end{Coro}

%______________________________________________________________________
%\subsection{Procesos de Renovaci\'on}
%______________________________________________________________________

\begin{Def}\label{Def.Tn}
Sean $0\leq T_{1}\leq T_{2}\leq \ldots$ son tiempos aleatorios infinitos en los cuales ocurren ciertos eventos. El n\'umero de tiempos $T_{n}$ en el intervalo $\left[0,t\right)$ es

\begin{eqnarray}
N\left(t\right)=\sum_{n=1}^{\infty}\indora\left(T_{n}\leq t\right),
\end{eqnarray}
para $t\geq0$.
\end{Def}

Si se consideran los puntos $T_{n}$ como elementos de $\rea_{+}$, y $N\left(t\right)$ es el n\'umero de puntos en $\rea$. El proceso denotado por $\left\{N\left(t\right):t\geq0\right\}$, denotado por $N\left(t\right)$, es un proceso puntual en $\rea_{+}$. Los $T_{n}$ son los tiempos de ocurrencia, el proceso puntual $N\left(t\right)$ es simple si su n\'umero de ocurrencias son distintas: $0<T_{1}<T_{2}<\ldots$ casi seguramente.

\begin{Def}
Un proceso puntual $N\left(t\right)$ es un proceso de renovaci\'on si los tiempos de interocurrencia $\xi_{n}=T_{n}-T_{n-1}$, para $n\geq1$, son independientes e identicamente distribuidos con distribuci\'on $F$, donde $F\left(0\right)=0$ y $T_{0}=0$. Los $T_{n}$ son llamados tiempos de renovaci\'on, referente a la independencia o renovaci\'on de la informaci\'on estoc\'astica en estos tiempos. Los $\xi_{n}$ son los tiempos de inter-renovaci\'on, y $N\left(t\right)$ es el n\'umero de renovaciones en el intervalo $\left[0,t\right)$
\end{Def}


\begin{Note}
Para definir un proceso de renovaci\'on para cualquier contexto, solamente hay que especificar una distribuci\'on $F$, con $F\left(0\right)=0$, para los tiempos de inter-renovaci\'on. La funci\'on $F$ en turno degune las otra variables aleatorias. De manera formal, existe un espacio de probabilidad y una sucesi\'on de variables aleatorias $\xi_{1},\xi_{2},\ldots$ definidas en este con distribuci\'on $F$. Entonces las otras cantidades son $T_{n}=\sum_{k=1}^{n}\xi_{k}$ y $N\left(t\right)=\sum_{n=1}^{\infty}\indora\left(T_{n}\leq t\right)$, donde $T_{n}\rightarrow\infty$ casi seguramente por la Ley Fuerte de los Grandes Números.
\end{Note}

%___________________________________________________________________________________________
%
\subsection{Renewal and Regenerative Processes: Serfozo\cite{Serfozo}}
%___________________________________________________________________________________________
%
\begin{Def}\label{Def.Tn}
Sean $0\leq T_{1}\leq T_{2}\leq \ldots$ son tiempos aleatorios infinitos en los cuales ocurren ciertos eventos. El n\'umero de tiempos $T_{n}$ en el intervalo $\left[0,t\right)$ es

\begin{eqnarray}
N\left(t\right)=\sum_{n=1}^{\infty}\indora\left(T_{n}\leq t\right),
\end{eqnarray}
para $t\geq0$.
\end{Def}

Si se consideran los puntos $T_{n}$ como elementos de $\rea_{+}$, y $N\left(t\right)$ es el n\'umero de puntos en $\rea$. El proceso denotado por $\left\{N\left(t\right):t\geq0\right\}$, denotado por $N\left(t\right)$, es un proceso puntual en $\rea_{+}$. Los $T_{n}$ son los tiempos de ocurrencia, el proceso puntual $N\left(t\right)$ es simple si su n\'umero de ocurrencias son distintas: $0<T_{1}<T_{2}<\ldots$ casi seguramente.

\begin{Def}
Un proceso puntual $N\left(t\right)$ es un proceso de renovaci\'on si los tiempos de interocurrencia $\xi_{n}=T_{n}-T_{n-1}$, para $n\geq1$, son independientes e identicamente distribuidos con distribuci\'on $F$, donde $F\left(0\right)=0$ y $T_{0}=0$. Los $T_{n}$ son llamados tiempos de renovaci\'on, referente a la independencia o renovaci\'on de la informaci\'on estoc\'astica en estos tiempos. Los $\xi_{n}$ son los tiempos de inter-renovaci\'on, y $N\left(t\right)$ es el n\'umero de renovaciones en el intervalo $\left[0,t\right)$
\end{Def}


\begin{Note}
Para definir un proceso de renovaci\'on para cualquier contexto, solamente hay que especificar una distribuci\'on $F$, con $F\left(0\right)=0$, para los tiempos de inter-renovaci\'on. La funci\'on $F$ en turno degune las otra variables aleatorias. De manera formal, existe un espacio de probabilidad y una sucesi\'on de variables aleatorias $\xi_{1},\xi_{2},\ldots$ definidas en este con distribuci\'on $F$. Entonces las otras cantidades son $T_{n}=\sum_{k=1}^{n}\xi_{k}$ y $N\left(t\right)=\sum_{n=1}^{\infty}\indora\left(T_{n}\leq t\right)$, donde $T_{n}\rightarrow\infty$ casi seguramente por la Ley Fuerte de los Grandes N\'umeros.
\end{Note}







Los tiempos $T_{n}$ est\'an relacionados con los conteos de $N\left(t\right)$ por

\begin{eqnarray*}
\left\{N\left(t\right)\geq n\right\}&=&\left\{T_{n}\leq t\right\}\\
T_{N\left(t\right)}\leq &t&<T_{N\left(t\right)+1},
\end{eqnarray*}

adem\'as $N\left(T_{n}\right)=n$, y 

\begin{eqnarray*}
N\left(t\right)=\max\left\{n:T_{n}\leq t\right\}=\min\left\{n:T_{n+1}>t\right\}
\end{eqnarray*}

Por propiedades de la convoluci\'on se sabe que

\begin{eqnarray*}
P\left\{T_{n}\leq t\right\}=F^{n\star}\left(t\right)
\end{eqnarray*}
que es la $n$-\'esima convoluci\'on de $F$. Entonces 

\begin{eqnarray*}
\left\{N\left(t\right)\geq n\right\}&=&\left\{T_{n}\leq t\right\}\\
P\left\{N\left(t\right)\leq n\right\}&=&1-F^{\left(n+1\right)\star}\left(t\right)
\end{eqnarray*}

Adem\'as usando el hecho de que $\esp\left[N\left(t\right)\right]=\sum_{n=1}^{\infty}P\left\{N\left(t\right)\geq n\right\}$
se tiene que

\begin{eqnarray*}
\esp\left[N\left(t\right)\right]=\sum_{n=1}^{\infty}F^{n\star}\left(t\right)
\end{eqnarray*}

\begin{Prop}
Para cada $t\geq0$, la funci\'on generadora de momentos $\esp\left[e^{\alpha N\left(t\right)}\right]$ existe para alguna $\alpha$ en una vecindad del 0, y de aqu\'i que $\esp\left[N\left(t\right)^{m}\right]<\infty$, para $m\geq1$.
\end{Prop}

\begin{Ejem}[\textbf{Proceso Poisson}]

Suponga que se tienen tiempos de inter-renovaci\'on \textit{i.i.d.} del proceso de renovaci\'on $N\left(t\right)$ tienen distribuci\'on exponencial $F\left(t\right)=q-e^{-\lambda t}$ con tasa $\lambda$. Entonces $N\left(t\right)$ es un proceso Poisson con tasa $\lambda$.

\end{Ejem}


\begin{Note}
Si el primer tiempo de renovaci\'on $\xi_{1}$ no tiene la misma distribuci\'on que el resto de las $\xi_{n}$, para $n\geq2$, a $N\left(t\right)$ se le llama Proceso de Renovaci\'on retardado, donde si $\xi$ tiene distribuci\'on $G$, entonces el tiempo $T_{n}$ de la $n$-\'esima renovaci\'on tiene distribuci\'on $G\star F^{\left(n-1\right)\star}\left(t\right)$
\end{Note}


\begin{Teo}
Para una constante $\mu\leq\infty$ ( o variable aleatoria), las siguientes expresiones son equivalentes:

\begin{eqnarray}
lim_{n\rightarrow\infty}n^{-1}T_{n}&=&\mu,\textrm{ c.s.}\\
lim_{t\rightarrow\infty}t^{-1}N\left(t\right)&=&1/\mu,\textrm{ c.s.}
\end{eqnarray}
\end{Teo}


Es decir, $T_{n}$ satisface la Ley Fuerte de los Grandes N\'umeros s\'i y s\'olo s\'i $N\left/t\right)$ la cumple.


\begin{Coro}[Ley Fuerte de los Grandes N\'umeros para Procesos de Renovaci\'on]
Si $N\left(t\right)$ es un proceso de renovaci\'on cuyos tiempos de inter-renovaci\'on tienen media $\mu\leq\infty$, entonces
\begin{eqnarray}
t^{-1}N\left(t\right)\rightarrow 1/\mu,\textrm{ c.s. cuando }t\rightarrow\infty.
\end{eqnarray}

\end{Coro}


Considerar el proceso estoc\'astico de valores reales $\left\{Z\left(t\right):t\geq0\right\}$ en el mismo espacio de probabilidad que $N\left(t\right)$

\begin{Def}
Para el proceso $\left\{Z\left(t\right):t\geq0\right\}$ se define la fluctuaci\'on m\'axima de $Z\left(t\right)$ en el intervalo $\left(T_{n-1},T_{n}\right]$:
\begin{eqnarray*}
M_{n}=\sup_{T_{n-1}<t\leq T_{n}}|Z\left(t\right)-Z\left(T_{n-1}\right)|
\end{eqnarray*}
\end{Def}

\begin{Teo}
Sup\'ongase que $n^{-1}T_{n}\rightarrow\mu$ c.s. cuando $n\rightarrow\infty$, donde $\mu\leq\infty$ es una constante o variable aleatoria. Sea $a$ una constante o variable aleatoria que puede ser infinita cuando $\mu$ es finita, y considere las expresiones l\'imite:
\begin{eqnarray}
lim_{n\rightarrow\infty}n^{-1}Z\left(T_{n}\right)&=&a,\textrm{ c.s.}\\
lim_{t\rightarrow\infty}t^{-1}Z\left(t\right)&=&a/\mu,\textrm{ c.s.}
\end{eqnarray}
La segunda expresi\'on implica la primera. Conversamente, la primera implica la segunda si el proceso $Z\left(t\right)$ es creciente, o si $lim_{n\rightarrow\infty}n^{-1}M_{n}=0$ c.s.
\end{Teo}

\begin{Coro}
Si $N\left(t\right)$ es un proceso de renovaci\'on, y $\left(Z\left(T_{n}\right)-Z\left(T_{n-1}\right),M_{n}\right)$, para $n\geq1$, son variables aleatorias independientes e id\'enticamente distribuidas con media finita, entonces,
\begin{eqnarray}
lim_{t\rightarrow\infty}t^{-1}Z\left(t\right)\rightarrow\frac{\esp\left[Z\left(T_{1}\right)-Z\left(T_{0}\right)\right]}{\esp\left[T_{1}\right]},\textrm{ c.s. cuando  }t\rightarrow\infty.
\end{eqnarray}
\end{Coro}


Sup\'ongase que $N\left(t\right)$ es un proceso de renovaci\'on con distribuci\'on $F$ con media finita $\mu$.

\begin{Def}
La funci\'on de renovaci\'on asociada con la distribuci\'on $F$, del proceso $N\left(t\right)$, es
\begin{eqnarray*}
U\left(t\right)=\sum_{n=1}^{\infty}F^{n\star}\left(t\right),\textrm{   }t\geq0,
\end{eqnarray*}
donde $F^{0\star}\left(t\right)=\indora\left(t\geq0\right)$.
\end{Def}


\begin{Prop}
Sup\'ongase que la distribuci\'on de inter-renovaci\'on $F$ tiene densidad $f$. Entonces $U\left(t\right)$ tambi\'en tiene densidad, para $t>0$, y es $U^{'}\left(t\right)=\sum_{n=0}^{\infty}f^{n\star}\left(t\right)$. Adem\'as
\begin{eqnarray*}
\prob\left\{N\left(t\right)>N\left(t-\right)\right\}=0\textrm{,   }t\geq0.
\end{eqnarray*}
\end{Prop}

\begin{Def}
La Transformada de Laplace-Stieljes de $F$ est\'a dada por

\begin{eqnarray*}
\hat{F}\left(\alpha\right)=\int_{\rea_{+}}e^{-\alpha t}dF\left(t\right)\textrm{,  }\alpha\geq0.
\end{eqnarray*}
\end{Def}

Entonces

\begin{eqnarray*}
\hat{U}\left(\alpha\right)=\sum_{n=0}^{\infty}\hat{F^{n\star}}\left(\alpha\right)=\sum_{n=0}^{\infty}\hat{F}\left(\alpha\right)^{n}=\frac{1}{1-\hat{F}\left(\alpha\right)}.
\end{eqnarray*}


\begin{Prop}
La Transformada de Laplace $\hat{U}\left(\alpha\right)$ y $\hat{F}\left(\alpha\right)$ determina una a la otra de manera \'unica por la relaci\'on $\hat{U}\left(\alpha\right)=\frac{1}{1-\hat{F}\left(\alpha\right)}$.
\end{Prop}


\begin{Note}
Un proceso de renovaci\'on $N\left(t\right)$ cuyos tiempos de inter-renovaci\'on tienen media finita, es un proceso Poisson con tasa $\lambda$ si y s\'olo s\'i $\esp\left[U\left(t\right)\right]=\lambda t$, para $t\geq0$.
\end{Note}


\begin{Teo}
Sea $N\left(t\right)$ un proceso puntual simple con puntos de localizaci\'on $T_{n}$ tal que $\eta\left(t\right)=\esp\left[N\left(\right)\right]$ es finita para cada $t$. Entonces para cualquier funci\'on $f:\rea_{+}\rightarrow\rea$,
\begin{eqnarray*}
\esp\left[\sum_{n=1}^{N\left(\right)}f\left(T_{n}\right)\right]=\int_{\left(0,t\right]}f\left(s\right)d\eta\left(s\right)\textrm{,  }t\geq0,
\end{eqnarray*}
suponiendo que la integral exista. Adem\'as si $X_{1},X_{2},\ldots$ son variables aleatorias definidas en el mismo espacio de probabilidad que el proceso $N\left(t\right)$ tal que $\esp\left[X_{n}|T_{n}=s\right]=f\left(s\right)$, independiente de $n$. Entonces
\begin{eqnarray*}
\esp\left[\sum_{n=1}^{N\left(t\right)}X_{n}\right]=\int_{\left(0,t\right]}f\left(s\right)d\eta\left(s\right)\textrm{,  }t\geq0,
\end{eqnarray*} 
suponiendo que la integral exista. 
\end{Teo}

\begin{Coro}[Identidad de Wald para Renovaciones]
Para el proceso de renovaci\'on $N\left(t\right)$,
\begin{eqnarray*}
\esp\left[T_{N\left(t\right)+1}\right]=\mu\esp\left[N\left(t\right)+1\right]\textrm{,  }t\geq0,
\end{eqnarray*}  
\end{Coro}


\begin{Def}
Sea $h\left(t\right)$ funci\'on de valores reales en $\rea$ acotada en intervalos finitos e igual a cero para $t<0$ La ecuaci\'on de renovaci\'on para $h\left(t\right)$ y la distribuci\'on $F$ es

\begin{eqnarray}\label{Ec.Renovacion}
H\left(t\right)=h\left(t\right)+\int_{\left[0,t\right]}H\left(t-s\right)dF\left(s\right)\textrm{,    }t\geq0,
\end{eqnarray}
donde $H\left(t\right)$ es una funci\'on de valores reales. Esto es $H=h+F\star H$. Decimos que $H\left(t\right)$ es soluci\'on de esta ecuaci\'on si satisface la ecuaci\'on, y es acotada en intervalos finitos e iguales a cero para $t<0$.
\end{Def}

\begin{Prop}
La funci\'on $U\star h\left(t\right)$ es la \'unica soluci\'on de la ecuaci\'on de renovaci\'on (\ref{Ec.Renovacion}).
\end{Prop}

\begin{Teo}[Teorema Renovaci\'on Elemental]
\begin{eqnarray*}
t^{-1}U\left(t\right)\rightarrow 1/\mu\textrm{,    cuando }t\rightarrow\infty.
\end{eqnarray*}
\end{Teo}



Sup\'ongase que $N\left(t\right)$ es un proceso de renovaci\'on con distribuci\'on $F$ con media finita $\mu$.

\begin{Def}
La funci\'on de renovaci\'on asociada con la distribuci\'on $F$, del proceso $N\left(t\right)$, es
\begin{eqnarray*}
U\left(t\right)=\sum_{n=1}^{\infty}F^{n\star}\left(t\right),\textrm{   }t\geq0,
\end{eqnarray*}
donde $F^{0\star}\left(t\right)=\indora\left(t\geq0\right)$.
\end{Def}


\begin{Prop}
Sup\'ongase que la distribuci\'on de inter-renovaci\'on $F$ tiene densidad $f$. Entonces $U\left(t\right)$ tambi\'en tiene densidad, para $t>0$, y es $U^{'}\left(t\right)=\sum_{n=0}^{\infty}f^{n\star}\left(t\right)$. Adem\'as
\begin{eqnarray*}
\prob\left\{N\left(t\right)>N\left(t-\right)\right\}=0\textrm{,   }t\geq0.
\end{eqnarray*}
\end{Prop}

\begin{Def}
La Transformada de Laplace-Stieljes de $F$ est\'a dada por

\begin{eqnarray*}
\hat{F}\left(\alpha\right)=\int_{\rea_{+}}e^{-\alpha t}dF\left(t\right)\textrm{,  }\alpha\geq0.
\end{eqnarray*}
\end{Def}

Entonces

\begin{eqnarray*}
\hat{U}\left(\alpha\right)=\sum_{n=0}^{\infty}\hat{F^{n\star}}\left(\alpha\right)=\sum_{n=0}^{\infty}\hat{F}\left(\alpha\right)^{n}=\frac{1}{1-\hat{F}\left(\alpha\right)}.
\end{eqnarray*}


\begin{Prop}
La Transformada de Laplace $\hat{U}\left(\alpha\right)$ y $\hat{F}\left(\alpha\right)$ determina una a la otra de manera \'unica por la relaci\'on $\hat{U}\left(\alpha\right)=\frac{1}{1-\hat{F}\left(\alpha\right)}$.
\end{Prop}


\begin{Note}
Un proceso de renovaci\'on $N\left(t\right)$ cuyos tiempos de inter-renovaci\'on tienen media finita, es un proceso Poisson con tasa $\lambda$ si y s\'olo s\'i $\esp\left[U\left(t\right)\right]=\lambda t$, para $t\geq0$.
\end{Note}


\begin{Teo}
Sea $N\left(t\right)$ un proceso puntual simple con puntos de localizaci\'on $T_{n}$ tal que $\eta\left(t\right)=\esp\left[N\left(\right)\right]$ es finita para cada $t$. Entonces para cualquier funci\'on $f:\rea_{+}\rightarrow\rea$,
\begin{eqnarray*}
\esp\left[\sum_{n=1}^{N\left(\right)}f\left(T_{n}\right)\right]=\int_{\left(0,t\right]}f\left(s\right)d\eta\left(s\right)\textrm{,  }t\geq0,
\end{eqnarray*}
suponiendo que la integral exista. Adem\'as si $X_{1},X_{2},\ldots$ son variables aleatorias definidas en el mismo espacio de probabilidad que el proceso $N\left(t\right)$ tal que $\esp\left[X_{n}|T_{n}=s\right]=f\left(s\right)$, independiente de $n$. Entonces
\begin{eqnarray*}
\esp\left[\sum_{n=1}^{N\left(t\right)}X_{n}\right]=\int_{\left(0,t\right]}f\left(s\right)d\eta\left(s\right)\textrm{,  }t\geq0,
\end{eqnarray*} 
suponiendo que la integral exista. 
\end{Teo}

\begin{Coro}[Identidad de Wald para Renovaciones]
Para el proceso de renovaci\'on $N\left(t\right)$,
\begin{eqnarray*}
\esp\left[T_{N\left(t\right)+1}\right]=\mu\esp\left[N\left(t\right)+1\right]\textrm{,  }t\geq0,
\end{eqnarray*}  
\end{Coro}


\begin{Def}
Sea $h\left(t\right)$ funci\'on de valores reales en $\rea$ acotada en intervalos finitos e igual a cero para $t<0$ La ecuaci\'on de renovaci\'on para $h\left(t\right)$ y la distribuci\'on $F$ es

\begin{eqnarray}\label{Ec.Renovacion}
H\left(t\right)=h\left(t\right)+\int_{\left[0,t\right]}H\left(t-s\right)dF\left(s\right)\textrm{,    }t\geq0,
\end{eqnarray}
donde $H\left(t\right)$ es una funci\'on de valores reales. Esto es $H=h+F\star H$. Decimos que $H\left(t\right)$ es soluci\'on de esta ecuaci\'on si satisface la ecuaci\'on, y es acotada en intervalos finitos e iguales a cero para $t<0$.
\end{Def}

\begin{Prop}
La funci\'on $U\star h\left(t\right)$ es la \'unica soluci\'on de la ecuaci\'on de renovaci\'on (\ref{Ec.Renovacion}).
\end{Prop}

\begin{Teo}[Teorema Renovaci\'on Elemental]
\begin{eqnarray*}
t^{-1}U\left(t\right)\rightarrow 1/\mu\textrm{,    cuando }t\rightarrow\infty.
\end{eqnarray*}
\end{Teo}


\begin{Note} Una funci\'on $h:\rea_{+}\rightarrow\rea$ es Directamente Riemann Integrable en los siguientes casos:
\begin{itemize}
\item[a)] $h\left(t\right)\geq0$ es decreciente y Riemann Integrable.
\item[b)] $h$ es continua excepto posiblemente en un conjunto de Lebesgue de medida 0, y $|h\left(t\right)|\leq b\left(t\right)$, donde $b$ es DRI.
\end{itemize}
\end{Note}

\begin{Teo}[Teorema Principal de Renovaci\'on]
Si $F$ es no aritm\'etica y $h\left(t\right)$ es Directamente Riemann Integrable (DRI), entonces

\begin{eqnarray*}
lim_{t\rightarrow\infty}U\star h=\frac{1}{\mu}\int_{\rea_{+}}h\left(s\right)ds.
\end{eqnarray*}
\end{Teo}

\begin{Prop}
Cualquier funci\'on $H\left(t\right)$ acotada en intervalos finitos y que es 0 para $t<0$ puede expresarse como
\begin{eqnarray*}
H\left(t\right)=U\star h\left(t\right)\textrm{,  donde }h\left(t\right)=H\left(t\right)-F\star H\left(t\right)
\end{eqnarray*}
\end{Prop}

\begin{Def}
Un proceso estoc\'astico $X\left(t\right)$ es crudamente regenerativo en un tiempo aleatorio positivo $T$ si
\begin{eqnarray*}
\esp\left[X\left(T+t\right)|T\right]=\esp\left[X\left(t\right)\right]\textrm{, para }t\geq0,\end{eqnarray*}
y con las esperanzas anteriores finitas.
\end{Def}

\begin{Prop}
Sup\'ongase que $X\left(t\right)$ es un proceso crudamente regenerativo en $T$, que tiene distribuci\'on $F$. Si $\esp\left[X\left(t\right)\right]$ es acotado en intervalos finitos, entonces
\begin{eqnarray*}
\esp\left[X\left(t\right)\right]=U\star h\left(t\right)\textrm{,  donde }h\left(t\right)=\esp\left[X\left(t\right)\indora\left(T>t\right)\right].
\end{eqnarray*}
\end{Prop}

\begin{Teo}[Regeneraci\'on Cruda]
Sup\'ongase que $X\left(t\right)$ es un proceso con valores positivo crudamente regenerativo en $T$, y def\'inase $M=\sup\left\{|X\left(t\right)|:t\leq T\right\}$. Si $T$ es no aritm\'etico y $M$ y $MT$ tienen media finita, entonces
\begin{eqnarray*}
lim_{t\rightarrow\infty}\esp\left[X\left(t\right)\right]=\frac{1}{\mu}\int_{\rea_{+}}h\left(s\right)ds,
\end{eqnarray*}
donde $h\left(t\right)=\esp\left[X\left(t\right)\indora\left(T>t\right)\right]$.
\end{Teo}


\begin{Note} Una funci\'on $h:\rea_{+}\rightarrow\rea$ es Directamente Riemann Integrable en los siguientes casos:
\begin{itemize}
\item[a)] $h\left(t\right)\geq0$ es decreciente y Riemann Integrable.
\item[b)] $h$ es continua excepto posiblemente en un conjunto de Lebesgue de medida 0, y $|h\left(t\right)|\leq b\left(t\right)$, donde $b$ es DRI.
\end{itemize}
\end{Note}

\begin{Teo}[Teorema Principal de Renovaci\'on]
Si $F$ es no aritm\'etica y $h\left(t\right)$ es Directamente Riemann Integrable (DRI), entonces

\begin{eqnarray*}
lim_{t\rightarrow\infty}U\star h=\frac{1}{\mu}\int_{\rea_{+}}h\left(s\right)ds.
\end{eqnarray*}
\end{Teo}

\begin{Prop}
Cualquier funci\'on $H\left(t\right)$ acotada en intervalos finitos y que es 0 para $t<0$ puede expresarse como
\begin{eqnarray*}
H\left(t\right)=U\star h\left(t\right)\textrm{,  donde }h\left(t\right)=H\left(t\right)-F\star H\left(t\right)
\end{eqnarray*}
\end{Prop}

\begin{Def}
Un proceso estoc\'astico $X\left(t\right)$ es crudamente regenerativo en un tiempo aleatorio positivo $T$ si
\begin{eqnarray*}
\esp\left[X\left(T+t\right)|T\right]=\esp\left[X\left(t\right)\right]\textrm{, para }t\geq0,\end{eqnarray*}
y con las esperanzas anteriores finitas.
\end{Def}

\begin{Prop}
Sup\'ongase que $X\left(t\right)$ es un proceso crudamente regenerativo en $T$, que tiene distribuci\'on $F$. Si $\esp\left[X\left(t\right)\right]$ es acotado en intervalos finitos, entonces
\begin{eqnarray*}
\esp\left[X\left(t\right)\right]=U\star h\left(t\right)\textrm{,  donde }h\left(t\right)=\esp\left[X\left(t\right)\indora\left(T>t\right)\right].
\end{eqnarray*}
\end{Prop}

\begin{Teo}[Regeneraci\'on Cruda]
Sup\'ongase que $X\left(t\right)$ es un proceso con valores positivo crudamente regenerativo en $T$, y def\'inase $M=\sup\left\{|X\left(t\right)|:t\leq T\right\}$. Si $T$ es no aritm\'etico y $M$ y $MT$ tienen media finita, entonces
\begin{eqnarray*}
lim_{t\rightarrow\infty}\esp\left[X\left(t\right)\right]=\frac{1}{\mu}\int_{\rea_{+}}h\left(s\right)ds,
\end{eqnarray*}
donde $h\left(t\right)=\esp\left[X\left(t\right)\indora\left(T>t\right)\right]$.
\end{Teo}

\begin{Def}
Para el proceso $\left\{\left(N\left(t\right),X\left(t\right)\right):t\geq0\right\}$, sus trayectoria muestrales en el intervalo de tiempo $\left[T_{n-1},T_{n}\right)$ est\'an descritas por
\begin{eqnarray*}
\zeta_{n}=\left(\xi_{n},\left\{X\left(T_{n-1}+t\right):0\leq t<\xi_{n}\right\}\right)
\end{eqnarray*}
Este $\zeta_{n}$ es el $n$-\'esimo segmento del proceso. El proceso es regenerativo sobre los tiempos $T_{n}$ si sus segmentos $\zeta_{n}$ son independientes e id\'enticamennte distribuidos.
\end{Def}


\begin{Note}
Si $\tilde{X}\left(t\right)$ con espacio de estados $\tilde{S}$ es regenerativo sobre $T_{n}$, entonces $X\left(t\right)=f\left(\tilde{X}\left(t\right)\right)$ tambi\'en es regenerativo sobre $T_{n}$, para cualquier funci\'on $f:\tilde{S}\rightarrow S$.
\end{Note}

\begin{Note}
Los procesos regenerativos son crudamente regenerativos, pero no al rev\'es.
\end{Note}


\begin{Note}
Un proceso estoc\'astico a tiempo continuo o discreto es regenerativo si existe un proceso de renovaci\'on  tal que los segmentos del proceso entre tiempos de renovaci\'on sucesivos son i.i.d., es decir, para $\left\{X\left(t\right):t\geq0\right\}$ proceso estoc\'astico a tiempo continuo con espacio de estados $S$, espacio m\'etrico.
\end{Note}

Para $\left\{X\left(t\right):t\geq0\right\}$ Proceso Estoc\'astico a tiempo continuo con estado de espacios $S$, que es un espacio m\'etrico, con trayectorias continuas por la derecha y con l\'imites por la izquierda c.s. Sea $N\left(t\right)$ un proceso de renovaci\'on en $\rea_{+}$ definido en el mismo espacio de probabilidad que $X\left(t\right)$, con tiempos de renovaci\'on $T$ y tiempos de inter-renovaci\'on $\xi_{n}=T_{n}-T_{n-1}$, con misma distribuci\'on $F$ de media finita $\mu$.



\begin{Def}
Para el proceso $\left\{\left(N\left(t\right),X\left(t\right)\right):t\geq0\right\}$, sus trayectoria muestrales en el intervalo de tiempo $\left[T_{n-1},T_{n}\right)$ est\'an descritas por
\begin{eqnarray*}
\zeta_{n}=\left(\xi_{n},\left\{X\left(T_{n-1}+t\right):0\leq t<\xi_{n}\right\}\right)
\end{eqnarray*}
Este $\zeta_{n}$ es el $n$-\'esimo segmento del proceso. El proceso es regenerativo sobre los tiempos $T_{n}$ si sus segmentos $\zeta_{n}$ son independientes e id\'enticamennte distribuidos.
\end{Def}

\begin{Note}
Un proceso regenerativo con media de la longitud de ciclo finita es llamado positivo recurrente.
\end{Note}

\begin{Teo}[Procesos Regenerativos]
Suponga que el proceso
\end{Teo}


\begin{Def}[Renewal Process Trinity]
Para un proceso de renovaci\'on $N\left(t\right)$, los siguientes procesos proveen de informaci\'on sobre los tiempos de renovaci\'on.
\begin{itemize}
\item $A\left(t\right)=t-T_{N\left(t\right)}$, el tiempo de recurrencia hacia atr\'as al tiempo $t$, que es el tiempo desde la \'ultima renovaci\'on para $t$.

\item $B\left(t\right)=T_{N\left(t\right)+1}-t$, el tiempo de recurrencia hacia adelante al tiempo $t$, residual del tiempo de renovaci\'on, que es el tiempo para la pr\'oxima renovaci\'on despu\'es de $t$.

\item $L\left(t\right)=\xi_{N\left(t\right)+1}=A\left(t\right)+B\left(t\right)$, la longitud del intervalo de renovaci\'on que contiene a $t$.
\end{itemize}
\end{Def}

\begin{Note}
El proceso tridimensional $\left(A\left(t\right),B\left(t\right),L\left(t\right)\right)$ es regenerativo sobre $T_{n}$, y por ende cada proceso lo es. Cada proceso $A\left(t\right)$ y $B\left(t\right)$ son procesos de MArkov a tiempo continuo con trayectorias continuas por partes en el espacio de estados $\rea_{+}$. Una expresi\'on conveniente para su distribuci\'on conjunta es, para $0\leq x<t,y\geq0$
\begin{equation}\label{NoRenovacion}
P\left\{A\left(t\right)>x,B\left(t\right)>y\right\}=
P\left\{N\left(t+y\right)-N\left((t-x)\right)=0\right\}
\end{equation}
\end{Note}

\begin{Ejem}[Tiempos de recurrencia Poisson]
Si $N\left(t\right)$ es un proceso Poisson con tasa $\lambda$, entonces de la expresi\'on (\ref{NoRenovacion}) se tiene que

\begin{eqnarray*}
\begin{array}{lc}
P\left\{A\left(t\right)>x,B\left(t\right)>y\right\}=e^{-\lambda\left(x+y\right)},&0\leq x<t,y\geq0,
\end{array}
\end{eqnarray*}
que es la probabilidad Poisson de no renovaciones en un intervalo de longitud $x+y$.

\end{Ejem}

\begin{Note}
Una cadena de Markov erg\'odica tiene la propiedad de ser estacionaria si la distribuci\'on de su estado al tiempo $0$ es su distribuci\'on estacionaria.
\end{Note}


\begin{Def}
Un proceso estoc\'astico a tiempo continuo $\left\{X\left(t\right):t\geq0\right\}$ en un espacio general es estacionario si sus distribuciones finito dimensionales son invariantes bajo cualquier  traslado: para cada $0\leq s_{1}<s_{2}<\cdots<s_{k}$ y $t\geq0$,
\begin{eqnarray*}
\left(X\left(s_{1}+t\right),\ldots,X\left(s_{k}+t\right)\right)=_{d}\left(X\left(s_{1}\right),\ldots,X\left(s_{k}\right)\right).
\end{eqnarray*}
\end{Def}

\begin{Note}
Un proceso de Markov es estacionario si $X\left(t\right)=_{d}X\left(0\right)$, $t\geq0$.
\end{Note}

Considerese el proceso $N\left(t\right)=\sum_{n}\indora\left(\tau_{n}\leq t\right)$ en $\rea_{+}$, con puntos $0<\tau_{1}<\tau_{2}<\cdots$.

\begin{Prop}
Si $N$ es un proceso puntual estacionario y $\esp\left[N\left(1\right)\right]<\infty$, entonces $\esp\left[N\left(t\right)\right]=t\esp\left[N\left(1\right)\right]$, $t\geq0$

\end{Prop}

\begin{Teo}
Los siguientes enunciados son equivalentes
\begin{itemize}
\item[i)] El proceso retardado de renovaci\'on $N$ es estacionario.

\item[ii)] EL proceso de tiempos de recurrencia hacia adelante $B\left(t\right)$ es estacionario.


\item[iii)] $\esp\left[N\left(t\right)\right]=t/\mu$,


\item[iv)] $G\left(t\right)=F_{e}\left(t\right)=\frac{1}{\mu}\int_{0}^{t}\left[1-F\left(s\right)\right]ds$
\end{itemize}
Cuando estos enunciados son ciertos, $P\left\{B\left(t\right)\leq x\right\}=F_{e}\left(x\right)$, para $t,x\geq0$.

\end{Teo}

\begin{Note}
Una consecuencia del teorema anterior es que el Proceso Poisson es el \'unico proceso sin retardo que es estacionario.
\end{Note}

\begin{Coro}
El proceso de renovaci\'on $N\left(t\right)$ sin retardo, y cuyos tiempos de inter renonaci\'on tienen media finita, es estacionario si y s\'olo si es un proceso Poisson.

\end{Coro}

%______________________________________________________________________

%\section{Ejemplos, Notas importantes}
%______________________________________________________________________
%\section*{Ap\'endice A}
%__________________________________________________________________

%________________________________________________________________________
%\subsection*{Procesos Regenerativos}
%________________________________________________________________________



\begin{Note}
Si $\tilde{X}\left(t\right)$ con espacio de estados $\tilde{S}$ es regenerativo sobre $T_{n}$, entonces $X\left(t\right)=f\left(\tilde{X}\left(t\right)\right)$ tambi\'en es regenerativo sobre $T_{n}$, para cualquier funci\'on $f:\tilde{S}\rightarrow S$.
\end{Note}

\begin{Note}
Los procesos regenerativos son crudamente regenerativos, pero no al rev\'es.
\end{Note}
%\subsection*{Procesos Regenerativos: Sigman\cite{Sigman1}}
\begin{Def}[Definici\'on Cl\'asica]
Un proceso estoc\'astico $X=\left\{X\left(t\right):t\geq0\right\}$ es llamado regenerativo is existe una variable aleatoria $R_{1}>0$ tal que
\begin{itemize}
\item[i)] $\left\{X\left(t+R_{1}\right):t\geq0\right\}$ es independiente de $\left\{\left\{X\left(t\right):t<R_{1}\right\},\right\}$
\item[ii)] $\left\{X\left(t+R_{1}\right):t\geq0\right\}$ es estoc\'asticamente equivalente a $\left\{X\left(t\right):t>0\right\}$
\end{itemize}

Llamamos a $R_{1}$ tiempo de regeneraci\'on, y decimos que $X$ se regenera en este punto.
\end{Def}

$\left\{X\left(t+R_{1}\right)\right\}$ es regenerativo con tiempo de regeneraci\'on $R_{2}$, independiente de $R_{1}$ pero con la misma distribuci\'on que $R_{1}$. Procediendo de esta manera se obtiene una secuencia de variables aleatorias independientes e id\'enticamente distribuidas $\left\{R_{n}\right\}$ llamados longitudes de ciclo. Si definimos a $Z_{k}\equiv R_{1}+R_{2}+\cdots+R_{k}$, se tiene un proceso de renovaci\'on llamado proceso de renovaci\'on encajado para $X$.




\begin{Def}
Para $x$ fijo y para cada $t\geq0$, sea $I_{x}\left(t\right)=1$ si $X\left(t\right)\leq x$,  $I_{x}\left(t\right)=0$ en caso contrario, y def\'inanse los tiempos promedio
\begin{eqnarray*}
\overline{X}&=&lim_{t\rightarrow\infty}\frac{1}{t}\int_{0}^{\infty}X\left(u\right)du\\
\prob\left(X_{\infty}\leq x\right)&=&lim_{t\rightarrow\infty}\frac{1}{t}\int_{0}^{\infty}I_{x}\left(u\right)du,
\end{eqnarray*}
cuando estos l\'imites existan.
\end{Def}

Como consecuencia del teorema de Renovaci\'on-Recompensa, se tiene que el primer l\'imite  existe y es igual a la constante
\begin{eqnarray*}
\overline{X}&=&\frac{\esp\left[\int_{0}^{R_{1}}X\left(t\right)dt\right]}{\esp\left[R_{1}\right]},
\end{eqnarray*}
suponiendo que ambas esperanzas son finitas.

\begin{Note}
\begin{itemize}
\item[a)] Si el proceso regenerativo $X$ es positivo recurrente y tiene trayectorias muestrales no negativas, entonces la ecuaci\'on anterior es v\'alida.
\item[b)] Si $X$ es positivo recurrente regenerativo, podemos construir una \'unica versi\'on estacionaria de este proceso, $X_{e}=\left\{X_{e}\left(t\right)\right\}$, donde $X_{e}$ es un proceso estoc\'astico regenerativo y estrictamente estacionario, con distribuci\'on marginal distribuida como $X_{\infty}$
\end{itemize}
\end{Note}

Para $\left\{X\left(t\right):t\geq0\right\}$ Proceso Estoc\'astico a tiempo continuo con estado de espacios $S$, que es un espacio m\'etrico, con trayectorias continuas por la derecha y con l\'imites por la izquierda c.s. Sea $N\left(t\right)$ un proceso de renovaci\'on en $\rea_{+}$ definido en el mismo espacio de probabilidad que $X\left(t\right)$, con tiempos de renovaci\'on $T$ y tiempos de inter-renovaci\'on $\xi_{n}=T_{n}-T_{n-1}$, con misma distribuci\'on $F$ de media finita $\mu$.


\begin{Def}
Para el proceso $\left\{\left(N\left(t\right),X\left(t\right)\right):t\geq0\right\}$, sus trayectoria muestrales en el intervalo de tiempo $\left[T_{n-1},T_{n}\right)$ est\'an descritas por
\begin{eqnarray*}
\zeta_{n}=\left(\xi_{n},\left\{X\left(T_{n-1}+t\right):0\leq t<\xi_{n}\right\}\right)
\end{eqnarray*}
Este $\zeta_{n}$ es el $n$-\'esimo segmento del proceso. El proceso es regenerativo sobre los tiempos $T_{n}$ si sus segmentos $\zeta_{n}$ son independientes e id\'enticamennte distribuidos.
\end{Def}


\begin{Note}
Si $\tilde{X}\left(t\right)$ con espacio de estados $\tilde{S}$ es regenerativo sobre $T_{n}$, entonces $X\left(t\right)=f\left(\tilde{X}\left(t\right)\right)$ tambi\'en es regenerativo sobre $T_{n}$, para cualquier funci\'on $f:\tilde{S}\rightarrow S$.
\end{Note}

\begin{Note}
Los procesos regenerativos son crudamente regenerativos, pero no al rev\'es.
\end{Note}

\begin{Def}[Definici\'on Cl\'asica]
Un proceso estoc\'astico $X=\left\{X\left(t\right):t\geq0\right\}$ es llamado regenerativo is existe una variable aleatoria $R_{1}>0$ tal que
\begin{itemize}
\item[i)] $\left\{X\left(t+R_{1}\right):t\geq0\right\}$ es independiente de $\left\{\left\{X\left(t\right):t<R_{1}\right\},\right\}$
\item[ii)] $\left\{X\left(t+R_{1}\right):t\geq0\right\}$ es estoc\'asticamente equivalente a $\left\{X\left(t\right):t>0\right\}$
\end{itemize}

Llamamos a $R_{1}$ tiempo de regeneraci\'on, y decimos que $X$ se regenera en este punto.
\end{Def}

$\left\{X\left(t+R_{1}\right)\right\}$ es regenerativo con tiempo de regeneraci\'on $R_{2}$, independiente de $R_{1}$ pero con la misma distribuci\'on que $R_{1}$. Procediendo de esta manera se obtiene una secuencia de variables aleatorias independientes e id\'enticamente distribuidas $\left\{R_{n}\right\}$ llamados longitudes de ciclo. Si definimos a $Z_{k}\equiv R_{1}+R_{2}+\cdots+R_{k}$, se tiene un proceso de renovaci\'on llamado proceso de renovaci\'on encajado para $X$.

\begin{Note}
Un proceso regenerativo con media de la longitud de ciclo finita es llamado positivo recurrente.
\end{Note}


\begin{Def}
Para $x$ fijo y para cada $t\geq0$, sea $I_{x}\left(t\right)=1$ si $X\left(t\right)\leq x$,  $I_{x}\left(t\right)=0$ en caso contrario, y def\'inanse los tiempos promedio
\begin{eqnarray*}
\overline{X}&=&lim_{t\rightarrow\infty}\frac{1}{t}\int_{0}^{\infty}X\left(u\right)du\\
\prob\left(X_{\infty}\leq x\right)&=&lim_{t\rightarrow\infty}\frac{1}{t}\int_{0}^{\infty}I_{x}\left(u\right)du,
\end{eqnarray*}
cuando estos l\'imites existan.
\end{Def}

Como consecuencia del teorema de Renovaci\'on-Recompensa, se tiene que el primer l\'imite  existe y es igual a la constante
\begin{eqnarray*}
\overline{X}&=&\frac{\esp\left[\int_{0}^{R_{1}}X\left(t\right)dt\right]}{\esp\left[R_{1}\right]},
\end{eqnarray*}
suponiendo que ambas esperanzas son finitas.

\begin{Note}
\begin{itemize}
\item[a)] Si el proceso regenerativo $X$ es positivo recurrente y tiene trayectorias muestrales no negativas, entonces la ecuaci\'on anterior es v\'alida.
\item[b)] Si $X$ es positivo recurrente regenerativo, podemos construir una \'unica versi\'on estacionaria de este proceso, $X_{e}=\left\{X_{e}\left(t\right)\right\}$, donde $X_{e}$ es un proceso estoc\'astico regenerativo y estrictamente estacionario, con distribuci\'on marginal distribuida como $X_{\infty}$
\end{itemize}
\end{Note}

%__________________________________________________________________________________________
%\subsection{Procesos Regenerativos Estacionarios - Stidham \cite{Stidham}}
%__________________________________________________________________________________________


Un proceso estoc\'astico a tiempo continuo $\left\{V\left(t\right),t\geq0\right\}$ es un proceso regenerativo si existe una sucesi\'on de variables aleatorias independientes e id\'enticamente distribuidas $\left\{X_{1},X_{2},\ldots\right\}$, sucesi\'on de renovaci\'on, tal que para cualquier conjunto de Borel $A$, 

\begin{eqnarray*}
\prob\left\{V\left(t\right)\in A|X_{1}+X_{2}+\cdots+X_{R\left(t\right)}=s,\left\{V\left(\tau\right),\tau<s\right\}\right\}=\prob\left\{V\left(t-s\right)\in A|X_{1}>t-s\right\},
\end{eqnarray*}
para todo $0\leq s\leq t$, donde $R\left(t\right)=\max\left\{X_{1}+X_{2}+\cdots+X_{j}\leq t\right\}=$n\'umero de renovaciones ({\emph{puntos de regeneraci\'on}}) que ocurren en $\left[0,t\right]$. El intervalo $\left[0,X_{1}\right)$ es llamado {\emph{primer ciclo de regeneraci\'on}} de $\left\{V\left(t \right),t\geq0\right\}$, $\left[X_{1},X_{1}+X_{2}\right)$ el {\emph{segundo ciclo de regeneraci\'on}}, y as\'i sucesivamente.

Sea $X=X_{1}$ y sea $F$ la funci\'on de distrbuci\'on de $X$


\begin{Def}
Se define el proceso estacionario, $\left\{V^{*}\left(t\right),t\geq0\right\}$, para $\left\{V\left(t\right),t\geq0\right\}$ por

\begin{eqnarray*}
\prob\left\{V\left(t\right)\in A\right\}=\frac{1}{\esp\left[X\right]}\int_{0}^{\infty}\prob\left\{V\left(t+x\right)\in A|X>x\right\}\left(1-F\left(x\right)\right)dx,
\end{eqnarray*} 
para todo $t\geq0$ y todo conjunto de Borel $A$.
\end{Def}

\begin{Def}
Una distribuci\'on se dice que es {\emph{aritm\'etica}} si todos sus puntos de incremento son m\'ultiplos de la forma $0,\lambda, 2\lambda,\ldots$ para alguna $\lambda>0$ entera.
\end{Def}


\begin{Def}
Una modificaci\'on medible de un proceso $\left\{V\left(t\right),t\geq0\right\}$, es una versi\'on de este, $\left\{V\left(t,w\right)\right\}$ conjuntamente medible para $t\geq0$ y para $w\in S$, $S$ espacio de estados para $\left\{V\left(t\right),t\geq0\right\}$.
\end{Def}

\begin{Teo}
Sea $\left\{V\left(t\right),t\geq\right\}$ un proceso regenerativo no negativo con modificaci\'on medible. Sea $\esp\left[X\right]<\infty$. Entonces el proceso estacionario dado por la ecuaci\'on anterior est\'a bien definido y tiene funci\'on de distribuci\'on independiente de $t$, adem\'as
\begin{itemize}
\item[i)] \begin{eqnarray*}
\esp\left[V^{*}\left(0\right)\right]&=&\frac{\esp\left[\int_{0}^{X}V\left(s\right)ds\right]}{\esp\left[X\right]}\end{eqnarray*}
\item[ii)] Si $\esp\left[V^{*}\left(0\right)\right]<\infty$, equivalentemente, si $\esp\left[\int_{0}^{X}V\left(s\right)ds\right]<\infty$,entonces
\begin{eqnarray*}
\frac{\int_{0}^{t}V\left(s\right)ds}{t}\rightarrow\frac{\esp\left[\int_{0}^{X}V\left(s\right)ds\right]}{\esp\left[X\right]}
\end{eqnarray*}
con probabilidad 1 y en media, cuando $t\rightarrow\infty$.
\end{itemize}
\end{Teo}

