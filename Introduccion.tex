\section{Descripci\'on del curso}

\subsection{Presentaci\'on}

\textbf{El curso}

\begin{itemize}
    \item Indispensable
    \item Modalidad: Presencial y Semipresencial
    \item Horas de clase: 5 teor\'ia y 5 de pr\'acticas
\end{itemize}

\subsection{Parte I. Introducci\'on a la Bioestad\'istica}

\textbf{Unidad 1. Conceptos b\'asicos de Bioestad\'istica y Metodolog\'ia de la Investigaci\'on}

Prop\'ositos: Que el/la estudiante:
\begin{enumerate}
    \item Comprenda y utilice correctamente los conceptos b\'asicos de la bioestad\'istica.
    \item Elabore bases de datos para el an\'alisis estad\'istico de la informaci\'on obtenida para sus trabajos de investigaci\'on.
\end{enumerate}

\subsection{Parte II. Estad\'istica descriptiva}

\textbf{Unidad 2. An\'alisis de la informaci\'on: tabulaci\'on y visualizaci\'on}

Prop\'ositos: Que el/la estudiante:
\begin{enumerate}
    \item Elabore e interprete correctamente representaciones tabulares y visuales.
    \item Seleccione las mejores formas de representar visualmente la informaci\'on.
\end{enumerate}

\textbf{Unidad 3. An\'alisis de la informaci\'on: medidas de tendencia central, variabilidad y localizaci\'on}

Prop\'ositos: Que el/la estudiante:
\begin{enumerate}
    \item Seleccione las medidas de tendencia central y variabilidad m\'as adecuadas de acuerdo a los objetivos del an\'alisis estad\'istico.
    \item Aplique e interprete correctamente las medidas de tendencia central, variabilidad y localizaci\'on.
\end{enumerate}

\subsection{Parte III. Estad\'istica inferencial}

\textbf{Unidad 4. La distribuci\'on normal}

Prop\'ositos: Que el/la estudiante:
\begin{enumerate}
    \item Entienda los conceptos impl\'icitos, expl\'icitos y la \textit{universalidad} de la distribuci\'on normal.
    \item Seleccione y ejecute correctamente los procedimientos num\'ericos para determinar las probabilidades correctas a partir de valores $z$ y la determinaci\'on de valores $z$ para probabilidades conocidas.
    \item Utilice correctamente las probabilidades $z$ en la resoluci\'on de problemas.
\end{enumerate}

\textbf{Unidad 5. Prueba de hip\'otesis}

Prop\'ositos: Que el/la estudiante:
\begin{enumerate}
    \item Comprenda los fundamentos te\'oricos de las pruebas de hip\'otesis.
    \item Comprenda la utilidad de las pruebas de hip\'otesis y su utilidad dentro del proceso de investigaci\'on.
    \item Comprenda y sea capaz de llevar a cabo pruebas de hip\'otesis para comparar proporciones de dos muestras.
    \item Comprenda y sea capaz de realizar pruebas de hip\'otesis para comparar las medias de dos muestras independientes.
    \item Comprenda y realice pruebas de hip\'otesis de datos apareados.
    \item Ejecute correctamente los procesos num\'ericos para obtener los estad\'isticos de prueba para la toma de decisiones.
    \item Comprenda la relaci\'on entre la decisi\'on estad\'istica de resultado del contraste de hip\'otesis y las decisiones con respecto a la pregunta de investigaci\'on y las hip\'otesis de trabajo.
\end{enumerate}

\textbf{Unidad 6. Correlaci\'on y regresi\'on lineal simple}

Prop\'ositos: Que el/la estudiante:
\begin{enumerate}
    \item Aplique correctamente el an\'alisis de correlaci\'on.
    \item Aplique correctamente el ajuste del modelo de regresi\'on lineal simple.
    \item Interprete correctamente los resultados del an\'alisis de correlaci\'on lineal simple y del modelo ajustado de regresi\'on lineal simple.
\end{enumerate}

\textbf{EVALUACI\'ON}
\begin{itemize}
    \item \textbf{Evaluaci\'on diagn\'ostica.} Se evaluar\'an conocimientos indispensables de aritm\'etica, \'algebra, geometr\'ia anal\'itica, as\'i como habilidades sobre c\'alculos aritm\'eticos, relaciones y operaciones algebraicas, construcci\'on e interpretaci\'on de gr\'aficas e interpretaci\'on de ecuaciones y aspectos indispensables de biolog\'ia humana.
    \item \textbf{Evaluaciones formativas.} Al menos tres evaluaciones formativas, que en conjunto deber\'an abarcar la totalidad del programa del curso. La forma, contenido y momento en que se realicen las evaluaciones formativas depender\'an del criterio del profesor. El resultado de cada evaluaci\'on formativa consistir\'a de la calificaci\'on de cada evaluaci\'on y las tareas (ejercicios, lecturas, etc.) comprendidos en los temas de cada evaluaci\'on.
    \item \textbf{Evaluaci\'on final.} La carpeta valdr\'a el 40\% de la calificaci\'on final y los elementos a considerar en la carpeta son: ex\'amenes, tareas y participaci\'on de cada estudiante. El porcentaje de cada elemento lo determinar\'a el profesor del curso, pero en total debe ser igual al 40\%. El examen de certificaci\'on valdr\'a el 60\% y no habr\'a estudiantes exentos. Todos los estudiantes que quieran aprobar la materia deber\'an realizar el examen de certificaci\'on (60\% de la calificaci\'on final) y contar con la carpeta (40\% de la calificaci\'on final). La calificaci\'on del examen de certificaci\'on valdr\'a el 60\%, esto es que si los estudiantes obtienen 10 de calificaci\'on en el examen de certificaci\'on, porcentualmente es el 60\% o sea 6, y si obtienen 10 de calificaci\'on en la carpeta tendr\'an el 40\% o sea 4 puntos de la calificaci\'on final, en este ejemplo la calificaci\'on final es 10.
\end{itemize}

\section{Introducci\'on}

\subsection{Definici\'on de Estad\'istica}

\begin{itemize}
    \item La Estad\'istica es una ciencia formal que estudia la recolecci\'on, an\'alisis e interpretaci\'on de datos de una muestra representativa, ya sea para ayudar en la toma de decisiones o para explicar condiciones regulares o irregulares de alg\'un fen\'omeno o estudio aplicado, de ocurrencia en forma aleatoria o condicional. 
    \item Sin embargo, la estad\'istica es m\'as que eso, es decir, es el veh\'iculo que permite llevar a cabo el proceso relacionado con la investigaci\'on cient\'ifica. 
    \item Es transversal a una amplia variedad de disciplinas, desde la f\'isica hasta las ciencias sociales, desde las ciencias de la salud hasta el control de calidad. Se usa para la toma de decisiones en \'areas de negocios o instituciones gubernamentales.
\end{itemize}

\begin{Def}
    La Estad\'istica es la ciencia cuyo objetivo es reunir una informaci\'on cuantitativa concerniente a individuos, grupos, series de hechos, etc. y deducir de ello gracias al an\'alisis de estos datos unos significados precisos o unas previsiones para el futuro.
\end{Def}

\begin{itemize}
    \item La estad\'istica, en general, es la ciencia que trata de la recopilaci\'on, organizaci\'on presentaci\'on, an\'alisis e interpretaci\'on de datos num\'ericos con el fin de realizar una toma de decisi\'on m\'as efectiva.
\end{itemize}

\subsection{Utilidad e Importancia}

\begin{itemize}
    \item Los m\'etodos estad\'isticos tradicionalmente se utilizan para prop\'ositos descriptivos, para organizar y resumir datos num\'ericos. La estad\'istica descriptiva, por ejemplo trata de la tabulaci\'on de datos, su presentaci\'on en forma gr\'afica o ilustrativa y el c\'alculo de medidas descriptivas.
    \item Ahora bien, las t\'ecnicas estad\'isticas se aplican de manera amplia en mercadotecnia, contabilidad, control de calidad y en otras actividades; estudios de consumidores; an\'alisis de resultados en deportes; administradores de instituciones; en la educaci\'on; organismos pol\'iticos; m\'edicos; y por otras personas que intervienen en la toma de decisiones.
\end{itemize}

\subsection{Historia de la Estad\'istica}

\begin{itemize}
    \item Es dif\'icil conocer los or\'igenes de la Estad\'istica. Desde los comienzos de la civilizaci\'on han existido formas sencillas de estad\'istica, pues ya se utilizaban representaciones gr\'aficas y otros s\'imbolos en pieles, rocas, palos de madera y paredes de cuevas para contar el n\'umero de personas, animales o ciertas cosas. 
    \item Su origen empieza posiblemente en la isla de Cerde\~na, donde existen monumentos prehist\'oricos pertenecientes a los Nuragas, las primeros habitantes de la isla; estos monumentos constan de bloques de basalto superpuestos sin mortero y en cuyas paredes de encontraban grabados toscos signos que han sido interpretados con mucha verosimilidad como muescas que serv\'ian para llevar la cuenta del ganado y la caza.
    \item Los babilonios usaban ya peque\~nas tablillas de arcilla para recopilar datos en tablas sobre la producci\'on agr\'icola y los g\'eneros vendidos o cambiados mediante trueque. 
    \item Otros vestigios pueden ser hallados en el antiguo Egipto, cuyos faraones lograron recopilar, hacia el a\~no 3050 antes de Cristo, prolijos datos relativos a la poblaci\'on y la riqueza del pa\'is. De acuerdo al historiador griego Her\'odoto, dicho registro de riqueza y poblaci\'on se hizo con el objetivo de preparar la construcci\'on de las pir\'amides.
    \item En el mismo Egipto, Rams\'es II hizo un censo de las tierras con el objeto de verificar un nuevo reparto. En el antiguo Israel la Biblia da referencias, en el libro de los N\'umeros, de los datos estad\'isticos obtenidos en dos recuentos de la poblaci\'on hebrea. El rey David por otra parte, orden\'o a Joab, general del ej\'ercito hacer un censo de Israel con la finalidad de conocer el n\'umero de la poblaci\'on.
    \item Tambi\'en los chinos efectuaron censos hace m\'as de cuarenta siglos. Los griegos efectuaron censos peri\'odicamente con fines tributarios, sociales (divisi\'on de tierras) y militares (c\'alculo de recursos y hombres disponibles). 
    \item La investigaci\'on hist\'orica revela que se realizaron 69 censos para calcular los impuestos, determinar los derechos de voto y ponderar la potencia guerrera.
    \item Fueron los romanos, maestros de la organizaci\'on pol\'itica, quienes mejor supieron emplear los recursos de la estad\'istica. Cada cinco a\~nos realizaban un censo de la poblaci\'on y sus funcionarios p\'ublicos ten\'ian la obligaci\'on de anotar nacimientos, defunciones y matrimonios, sin olvidar los recuentos peri\'odicos del ganado y de las riquezas contenidas en las tierras conquistadas. Para el nacimiento de Cristo suced\'ia uno de estos empadronamientos de la poblaci\'on bajo la autoridad del imperio. 
    \item Durante los mil a\~nos siguientes a la ca\'ida del imperio Romano se realizaron muy pocas operaciones Estad\'isticas, con la notable excepci\'on de las relaciones de tierras pertenecientes a la Iglesia, compiladas por Pipino el Breve en el 758 y por Carlomagno en el 762 DC. Durante el siglo IX se realizaron en Francia algunos censos parciales de siervos. En Inglaterra, Guillermo el Conquistador recopil\'o el Domesday Book o libro del Gran Catastro para el a\~no 1086, un documento de la propiedad, extensi\'on y valor de las tierras de Inglaterra. Esa obra fue el primer compendio estad\'istico de Inglaterra. 
    \item Aunque Carlomagno, en Francia; y Guillermo el Conquistador, en Inglaterra, trataron de revivir la t\'ecnica romana, los m\'etodos estad\'isticos permanecieron casi olvidados durante la Edad Media.
    \item Durante los siglos XV, XVI, y XVII, hombres como Leonardo de Vinci, Nicol\'as Cop\'ernico, Galileo, Neper, William Harvey, Sir Francis Bacon y Ren\'e Descartes, hicieron grandes operaciones al m\'etodo cient\'ifico, de tal forma que cuando se crearon los Estados Nacionales y surgi\'o como fuerza el comercio internacional exist\'ia ya un m\'etodo capaz de aplicarse a los datos econ\'omicos. 
    \item Para el a\~no 1532 empezaron a registrarse en Inglaterra las defunciones debido al temor que Enrique VII ten\'ia por la peste.  M\'as o menos por la misma \'epoca, en Francia la ley exigi\'o a los cl\'erigos registrar los bautismos, fallecimientos y matrimonios. Durante un brote de peste que apareci\'o a fines de la d\'ecada de 1500, el gobierno ingl\'es comenz\'o a publicar estad\'istica semanales de los decesos. Esa costumbre continu\'o muchos a\~nos, y en 1632 estos Bills of Mortality (Cuentas de Mortalidad) conten\'ian los nacimientos y fallecimientos por sexo.
    \item En 1662, el capit\'an John Graunt us\'o documentos que abarcaban treinta a\~nos y efectu\'o predicciones sobre el n\'umero de personas que morir\'ian de varias enfermedades y sobre las proporciones de nacimientos de varones y mujeres que cabr\'ia esperar. El trabajo de Graunt, condensado en su obra \textit{Natural and Political Observations...Made upon the Bills of Mortality}, fue un esfuerzo innovador en el an\'alisis estad\'istico. Por el a\~no 1540 el alem\'an Sebasti\'an Muster realiz\'o una compilaci\'on estad\'istica de los recursos nacionales, comprensiva de datos sobre organizaci\'on pol\'itica, instrucciones sociales, comercio y poder\'io militar. 
    \item Durante el siglo XVII aport\'o indicaciones m\'as concretas de m\'etodos de observaci\'on y an\'alisis cuantitativo y ampli\'o los campos de la inferencia y la teor\'ia Estad\'istica.
    \item Los eruditos del siglo XVII demostraron especial inter\'es por la Estad\'istica Demogr\'afica como resultado de la especulaci\'on sobre si la poblaci\'on aumentaba, decrec\'ia o permanec\'ia est\'atica. En los tiempos modernos tales m\'etodos fueron resucitados por algunos reyes que necesitaban conocer las riquezas monetarias y el potencial humano de sus respectivos pa\'ises. 
    \item El primer empleo de los datos estad\'isticos para fines ajenos a la pol\'itica tuvo lugar en 1691 y estuvo a cargo de Gaspar Neumann, un profesor alem\'an que viv\'ia en Breslau. Este investigador se propuso destruir la antigua creencia popular de que en los a\~nos terminados en siete mor\'ia m\'as gente que en los restantes, y para lograrlo hurg\'o pacientemente en los archivos parroquiales de la ciudad. Despu\'es de revisar miles de partidas de defunci\'on pudo demostrar que en tales a\~nos no fallec\'ian m\'as personas que en los dem\'as. Los procedimientos de Neumann fueron conocidos por el astr\'onomo ingl\'es Halley, descubridor del cometa que lleva su nombre, quien los aplic\'o al estudio de la vida humana. Sus c\'alculos sirvieron de base para las tablas de mortalidad que hoy utilizan todas las compa\~n\'ias de seguros. Durante el siglo XVII y principios del XVIII, matem\'aticos como Bernoulli, Francis Maseres, Lagrange y Laplace desarrollaron la teor\'ia de probabilidades. No obstante durante cierto tiempo, la teor\'ia de las probabilidades limit\'o su aplicaci\'on a los juegos de azar y hasta el siglo XVIII no comenz\'o a aplicarse a los grandes problemas cient\'ificos.
    \item Godofredo Achenwall, profesor de la Universidad de Gotinga, acu\~n\'o en 1760 la palabra estad\'istica, que extrajo del t\'ermino italiano statista (estadista). Cre\'ia, y con sobrada raz\'on, que los datos de la nueva ciencia ser\'ian el aliado m\'as eficaz del gobernante consciente. La ra\'iz remota de la palabra se halla, por otra parte, en el t\'ermino latino status, que significa estado o situaci\'on; Esta etimolog\'ia aumenta el valor intr\'inseco de la palabra, por cuanto la estad\'istica revela el sentido cuantitativo de las m\'as variadas situaciones. Jacques Qu\'etelect es quien aplica las Estad\'isticas a las ciencias sociales. Este interpret\'o la teor\'ia de la probabilidad para su uso en las ciencias sociales y resolver la aplicaci\'on del principio de promedios y de la variabilidad a los fen\'omenos sociales.
    \item Qu\'etelect fue el primero en realizar la aplicaci\'on pr\'actica de todo el m\'etodo Estad\'istico, entonces conocido, a las diversas ramas de la ciencia.
    \item Entretanto, en el per\'iodo del 1800 al 1820 se desarrollaron dos conceptos matem\'aticos fundamentales para la teor\'ia Estad\'istica; la teor\'ia de los errores de observaci\'on, aportada por Laplace y Gauss; y la teor\'ia de los m\'inimos cuadrados desarrollada por Laplace, Gauss y Legendre. A finales del siglo XIX, Sir Francis Gaston ide\'o el m\'etodo conocido por Correlaci\'on, que ten\'ia por objeto medir la influencia relativa de los factores sobre las variables.
    \item De aqu\'i parti\'o el desarrollo del coeficiente de correlaci\'on creado por Karl Pearson y otros cultivadores de la ciencia biom\'etrica como J. Pease Norton, R. H. Hooker y G. Udny Yule, que efectuaron amplios estudios sobre la medida de las relaciones.
    \item Los progresos m\'as recientes en el campo de la Estad\'istica se refieren al ulterior desarrollo del c\'alculo de probabilidades, particularmente en la rama denominada indeterminismo o relatividad, se ha demostrado que el determinismo fue reconocido en la F\'isica como resultado de las investigaciones at\'omicas y que este principio se juzga aplicable tanto a las ciencias sociales como a las f\'isicas.
\end{itemize}

\subsection{Etapas de Desarrollo de la Estad\'istica}

La historia de la estad\'istica est\'a resumida en tres grandes etapas o fases.

\begin{itemize}
    \item \textbf{Fase 1: Los Censos:} Desde el momento en que se constituye una autoridad pol\'itica, la idea de inventariar de una forma m\'as o menos regular la poblaci\'on y las riquezas existentes en el territorio est\'a ligada a la conciencia de soberan\'ia y a los primeros esfuerzos administrativos.
    \item \textbf{Fase 2: De la Descripci\'on de los Conjuntos a la Aritm\'etica Pol\'itica:} Las ideas mercantilistas extra\~nan una intensificaci\'on de este tipo de investigaci\'on. Colbert multiplica las encuestas sobre art\'iculos manufacturados, el comercio y la poblaci\'on: los intendentes del Reino env\'ian a Par\'is sus memorias. Vauban, m\'as conocido por sus fortificaciones o su Dime Royale, que es la primera propuesta de un impuesto sobre los ingresos, se se\~nala como el verdadero precursor de los sondeos. M\'as tarde, Buf\'on se preocupa de esos problemas antes de dedicarse a la historia natural. La escuela inglesa proporciona un nuevo progreso al superar la fase puramente descriptiva.
\end{itemize}

Sus tres principales representantes son Graunt, Petty y Halley. El pen\'ultimo es autor de la famosa Aritm\'etica Pol\'itica. Chaptal, ministro del interior franc\'es, publica en 1801 el primer censo general de poblaci\'on, desarrolla los estudios industriales, de las producciones y los cambios, haci\'endose sistem\'aticos durantes las dos terceras partes del siglo XIX.

\begin{itemize}
    \item \textbf{Fase 3: Estad\'istica y C\'alculo de Probabilidades:} El c\'alculo de probabilidades se incorpora r\'apidamente como un instrumento de an\'alisis extremadamente poderoso para el estudio de los fen\'omenos econ\'omicos y sociales y en general para el estudio de fen\'omenos cuyas causas son demasiados complejas para conocerlos totalmente y hacer posible su an\'alisis.
\end{itemize}

\subsection{Divisi\'on de la Estad\'istica}

La Estad\'istica para su mejor estudio se ha dividido en dos grandes ramas: \textbf{la Estad\'istica Descriptiva y la Estad\'istica Inferencial}.

\begin{itemize}
    \item \textbf{Descriptiva:} consiste sobre todo en la presentaci\'on de datos en forma de tablas y gr\'aficas. Esta comprende cualquier actividad relacionada con los datos y est\'a dise\~nada para resumir o describir los mismos sin factores pertinentes adicionales; esto es, sin intentar inferir nada que vaya m\'as all\'a de los datos, como tales.
    \item \textbf{Inferencial:} se deriva de muestras, de observaciones hechas s\'olo acerca de una parte de un conjunto numeroso de elementos y esto implica que su an\'alisis requiere de generalizaciones que van m\'as all\'a de los datos. Como consecuencia, la caracter\'istica m\'as importante del reciente crecimiento de la estad\'istica ha sido un cambio en el \'enfasis de los m\'etodos que describen a m\'etodos que sirven para hacer generalizaciones. La Estad\'istica Inferencial investiga o analiza una poblaci\'on partiendo de una muestra tomada.
\end{itemize}

\subsection{Estad\'istica Inferencial}

Los m\'etodos b\'asicos de la estad\'istica inferencial son la estimaci\'on y el contraste de hip\'otesis, que juegan un papel fundamental en la investigaci\'on. Por tanto, algunos de los objetivos que se persiguen son:

\begin{itemize}
    \item Calcular los par\'ametros de la distribuci\'on de medias o proporciones muestrales de tama\~no $n$, extra\'idas de una poblaci\'on de media y varianza conocidas.
    \item Estimar la media o la proporci\'on de una poblaci\'on a partir de la media o proporci\'on muestral.
    \item Utilizar distintos tama\~nos muestrales para controlar la confianza y el error admitido.
    \item Contrastar los resultados obtenidos a partir de muestras.
    \item Visualizar gr\'aficamente, mediante las respectivas curvas normales, las estimaciones realizadas.
\end{itemize}

En definitiva, la idea es, a partir de una poblaci\'on se extrae una muestra por algunos de los m\'etodos existentes, con la que se generan datos num\'ericos que se van a utilizar para generar estad\'isticos con los que realizar estimaciones o contrastes poblacionales. Existen dos formas de estimar par\'ametros: la \textit{estimaci\'on puntual} y la \textit{estimaci\'on por intervalo de confianza}. En la primera se busca, con base en los datos muestrales, un \'unico valor estimado para el par\'ametro. Para la segunda, se determina un intervalo dentro del cual se encuentra el valor del par\'ametro, con una probabilidad determinada.

\begin{itemize}
    \item Si el objetivo del tratamiento estad\'istico inferencial, es efectuar generalizaciones acerca de la estructura, composici\'on o comportamiento de las poblaciones no observadas, a partir de una parte de la poblaci\'on, ser\'a necesario que la parcela de poblaci\'on examinada sea representativa del total. 
    \item Por ello, la selecci\'on de la muestra requiere unos requisitos que lo garanticen, debe ser representativa y aleatoria. 
    \item Adem\'as, la cantidad de elementos que integran la muestra (el tama\~no de la muestra) depende de m\'ultiples factores, como el dinero y el tiempo disponibles para el estudio, la importancia del tema analizado, la confiabilidad que se espera de los resultados, las caracter\'isticas propias del fen\'omeno analizado, etc\'etera. 
\end{itemize}

As\'i, a partir de la muestra seleccionada se realizan algunos c\'alculos y se estima el valor de los par\'ametros de la poblaci\'on tales como la media, la varianza, la desviaci\'on est\'andar, o la forma de la distribuci\'on, etc.

\subsection{M\'etodo Estad\'istico}

El conjunto de los m\'etodos que se utilizan para medir las caracter\'isticas de la informaci\'on, para resumir los valores individuales, y para analizar los datos a fin de extraerles el m\'aximo de informaci\'on, es lo que se llama \textit{m\'etodos estad\'isticos}. Los m\'etodos de an\'alisis para la informaci\'on cuantitativa se pueden dividir en los siguientes seis pasos:

\begin{enumerate}
    \item Definici\'on del problema.
    \item Recopilaci\'on de la informaci\'on existente.
    \item Obtenci\'on de informaci\'on original.
    \item Clasificaci\'on.
    \item Presentaci\'on.
    \item An\'alisis.
\end{enumerate}

El centro de gravedad de la metodolog\'ia estad\'istica se empieza a desplazar t\'ecnicas de computaci\'on intensiva aplicadas a grandes masas de datos, y se empieza a considerar el m\'etodo estad\'istico como un proceso iterativo de b\'usqueda del modelo ideal.

Las aplicaciones en este periodo de la Estad\'istica a la Econom\'ia conducen a una disciplina con contenido propio: la Econometr\'ia. La investigaci\'on estad\'istica en problemas militares durante la segunda guerra mundial y los nuevos m\'etodos de programaci\'on matem\'atica, dan lugar a la Investigaci\'on Operativa.

\subsection{Errores Estad\'isticos Comunes}

Al momento de recopilar los datos que ser\'an procesados se es susceptible de cometer errores as\'i como durante los c\'omputos de los mismos. No obstante, hay otros errores que no tienen nada que ver con la digitaci\'on y que no son tan f\'acilmente identificables. Algunos de \'estos errores son:

\begin{itemize}
    \item \textbf{Sesgo:} Es imposible ser completamente objetivo o no tener ideas preconcebidas antes de comenzar a estudiar un problema, y existen muchas maneras en que una perspectiva o estado mental pueda influir en la recopilaci\'on y en el an\'alisis de la informaci\'on. En estos casos se dice que hay un sesgo cuando el individuo da mayor peso a los datos que apoyan su opini\'on que a aquellos que la contradicen. Un caso extremo de sesgo ser\'ia la situaci\'on donde primero se toma una decisi\'on y despu\'es se utiliza el an\'alisis estad\'istico para justificar la decisi\'on ya tomada.
    \item \textbf{Datos No Comparables:} el establecer comparaciones es una de las partes m\'as importantes del an\'alisis estad\'istico, pero es extremadamente importante que tales comparaciones se hagan entre datos que sean comparables.
    \item \textbf{Proyecci\'on descuidada de tendencias:} la proyecci\'on simplista de tendencias pasadas hacia el futuro es uno de los errores que m\'as ha desacreditado el uso del an\'alisis estad\'istico.
    \item \textbf{Muestreo Incorrecto:} en la mayor\'ia de los estudios sucede que el volumen de informaci\'on disponible es tan inmenso que se hace necesario estudiar muestras, para derivar conclusiones acerca de la poblaci\'on a que pertenece la muestra. Si la muestra se selecciona correctamente, tendr\'a b\'asicamente las mismas propiedades que la poblaci\'on de la cual fue extra\'ida; pero si el muestreo se realiza incorrectamente, entonces puede suceder que los resultados no signifiquen nada.
\end{itemize}

En resumen se puede decir que la Estad\'istica es un conjunto de procedimientos para reunir, clasificar, codificar, procesar, analizar y resumir informaci\'on num\'erica adquirida sistem\'aticamente (Ritchey, 2002). Permite hacer inferencias a partir de una muestra para extrapolarlas a una poblaci\'on. Aunque normalmente se asocia a muchos c\'alculos y operaciones aritm\'eticas, y aunque las matem\'aticas est\'an involucradas, en su mayor parte sus fundamentos y uso apropiado pueden dominarse sin hacer referencia a habilidades matem\'aticas avanzadas. 

De hecho se trata de una forma de ver la realidad basada en el an\'alisis cuidadoso de los hechos (Ritchey, 2002). Es necesaria sin embargo la sistematizaci\'on para reducir el efecto que las emociones y las experiencias individuales puedan tener al interpretar esa realidad.

De esta manera la estad\'istica se relaciona con el m\'etodo cient\'ifico complement\'andolo como herramienta de an\'alisis y, aunque la investigaci\'on cient\'ifica no requiere necesariamente de la estad\'istica, \'esta valida muchos de los resultados cuantitativos derivados de la investigaci\'on. 

La obtenci\'on del conocimiento debe hacerse de manera sistem\'atica por lo que deben planearse todos los pasos que llevan desde el planteamiento de un problema, pasando por la elaboraci\'on de hip\'otesis y la manera en que van a ser probadas; la selecci\'on de sujetos (muestreo), los escenarios, los instrumentos que se utilizar\'an para obtener los datos, definir el procedimiento que se seguir\'a para esto \'ultimo, los controles que se deben hacer para asegurar que las intervenciones son las causas m\'as probables de los cambios esperados (dise\~no); 

El tratamiento de los datos de la investigaci\'on cient\'ifica tiene varias etapas:

\begin{itemize}
    \item En la etapa de recolecci\'on de datos del m\'etodo cient\'ifico, se define a la poblaci\'on de inter\'es y se selecciona una muestra o conjunto de personas representativas de la misma, se realizan experimentos o se emplean instrumentos ya existentes o de nueva creaci\'on, para medir los atributos de inter\'es necesarios para responder a las preguntas de investigaci\'on.Durante lo que es llamado trabajo de campo se obtienen los datos en crudo, es decir las respuestas directas de los sujetos uno por uno, se codifican (se les asignan valores a las respuestas), se capturan y se verifican para ser utilizados en las siguientes etapas.
    \item En la etapa de recuento, se organizan y ordenan los datos obtenidos de la muestra. Esta ser\'a descrita en la siguiente etapa utilizando la estad\'istica descriptiva, todas las investigaciones utilizan estad\'istica descriptiva, para conocer de manera organizada y resumida las caracter\'isticas de la muestra.
    \item En la etapa de an\'alisis se utilizan las pruebas estad\'isticas (estad\'istica inferencial) y en la interpretaci\'on se acepta o rechaza la hip\'otesis nula.
\end{itemize}

En investigaci\'on, el fen\'omeno en estudio puede ser cualitativo que implicar\'ia comprenderlo y explicarlo, o cuantitativo para compararlo y hacer inferencias. Se puede decir que si se hace an\'alisis se usan m\'etodos cuantitativos y si se hace descripci\'on se usan m\'etodos cualitativos. 

Medici\'on Para poder emplear el m\'etodo estad\'istico en un estudio es necesario medir las variables. 

\begin{itemize}
    \item Medir: es asignar valores a las propiedades de los objetos bajo ciertas reglas, esas reglas son los niveles de medici\'on.
    \item Cuantificar: es asignar valores a algo tomando un patr\'on de referencia. Por ejemplo, cuantificar es ver cu\'antos hombres y cu\'antas mujeres hay.
\end{itemize}

\textbf{Variable:} es una caracter\'istica o propiedad que asume diferentes valores dentro de una poblaci\'on de inter\'es y cuya variaci\'on es susceptible de medirse.

Las variables pueden clasificarse de acuerdo al tipo de valores que puede tomar como:

\begin{itemize}
    \item Discretas o categ\'oricas.- en las que los valores se relacionan a nombres, etiquetas o categor\'ias, no existe un significado num\'erico directo.
    \item Continuas.- los valores tienen un correlato num\'erico directo, son continuos y susceptibles de fraccionarse y de poder utilizarse en operaciones aritm\'eticas.
    \item Dicot\'omica.- s\'olo tienen dos valores posibles, la caracter\'istica est\'a ausente o presente.
    \item Policot\'omica.- pueden tomar tres valores o m\'as, pueden tomarse matices diferentes, en grados, jerarqu\'ias o magnitudes continuas.
\end{itemize}

En cuanto a una clasificaci\'on estad\'istica:

\begin{itemize}
    \item Aleatoria.- Aquella en la cual desconocemos el valor porque fluct\'ua de acuerdo a un evento debido al azar.
    \item Determin\'istica.- Aquella variable de la que se conoce el valor.
    \item Independiente.- aquellas variables que son manipuladas por el investigador. Define los grupos.
    \item Dependiente.- son mediciones que ocurren durante el experimento o tratamiento (resultado de la independiente), es la que se mide y compara entre los grupos.
\end{itemize}

\textbf{Niveles de Medici\'on}

\begin{itemize}
    \item Nominal: Las propiedades de la medici\'on nominal son:
    \begin{itemize}
        \item Exhaustiva: implica a todas las opciones.
        \item A los sujetos se les asignan categor\'ias, por lo que son mutuamente excluyentes. Es decir, la variable est\'a presente o no; tiene o no una caracter\'istica.
    \end{itemize}
    \item Ordinal: Las propiedades de la medici\'on ordinal son:
    \begin{itemize}
        \item El nivel ordinal posee transitividad, por lo que se tiene la capacidad de identificar que es mejor o mayor que otra, en ese sentido se pueden establecer jerarqu\'ias.
        \item Las distancias entre un valor y otro no son iguales.
    \end{itemize}
    \item Intervalo: 
    \begin{itemize}
        \item El nivel de medici\'on intervalar requiere distancias iguales entre cada valor. Por lo general utiliza datos cuantitativos. Por ejemplo: temperatura, atributos psicol\'ogicos (CI, nivel de autoestima, pruebas de conocimientos, etc.)
        \item Las unidades de calificaci\'on son equivalentes en todos los puntos de la escala. Una escala de intervalos implica: clasificaci\'on, magnitud y unidades de tama\~nos iguales (Brown, 2000).
        \item Se pueden hacer operaciones aritm\'eticas.
        \item Cuando se le pide al sujeto que califique una situaci\'on del 0 al 10 puede tomarse como un nivel de medici\'on de intervalo, siempre y cuando se incluya el 0.
    \end{itemize}
    \item Raz\'on: 
    \begin{itemize}
        \item La escala empieza a partir del 0 absoluto, por lo tanto incluye s\'olo los n\'umeros por su valor en s\'i, por lo que no pueden existir los n\'umeros con signo negativo. Por ejemplo: Peso corporal en kg., edad en a\~nos, estatura en cm.
        \item Convencionalmente los datos que son de nivel absoluto o de raz\'on son manejados como los datos intervalares.
    \end{itemize}
\end{itemize}

\subsection{T\'erminos comunes utilizados en Estad\'istica}

\begin{itemize}
    \item \textbf{Variable:} Consideraciones que una variable son una caracter\'istica o fen\'omeno que puede tomar distintos valores.
    \item \textbf{Dato:} Mediciones o cualidades que han sido recopiladas como resultado de observaciones.
    \item \textbf{Poblaci\'on:} Se considera el \'area de la cual son extra\'idos los datos. Es decir, es el conjunto de elementos o individuos que poseen una caracter\'istica com\'un y medible acerca de lo cual se desea informaci\'on. Es tambi\'en llamado Universo.
    \item \textbf{Muestra:} Es un subconjunto de la poblaci\'on, seleccionado de acuerdo a una regla o alg\'un plan de muestreo.
    \item \textbf{Censo:} Recopilaci\'on de todos los datos (de inter\'es para la investigaci\'on) de la poblaci\'on.
    \item \textbf{Estad\'istica:} Es una funci\'on o f\'ormula que depende de los datos de la muestra (es variable).
    \item \textbf{Par\'ametro:} Caracter\'istica medible de la poblaci\'on. Es un resumen num\'erico de alguna variable observada de la poblaci\'on. Los par\'ametros normales que se estudian son: \textit{La media poblacional, Proporci\'on.}
    \item \textbf{Estimador:} Un estimador de un par\'ametro es un estad\'istico que se emplea para conocer el par\'ametro desconocido.
    \item \textbf{Estad\'istico:} Es una funci\'on de los valores de la muestra. Es una variable aleatoria, cuyos valores dependen de la muestra seleccionada. Su distribuci\'on de probabilidad, se conoce como \textit{Distribuci\'on muestral del estad\'istico}.
    \item \textbf{Estimaci\'on:} Este t\'ermino indica que a partir de lo observado en una muestra (un resumen estad\'istico con las medidas que conocemos de Descriptiva) se extrapola o generaliza dicho resultado muestral a la poblaci\'on total, de modo que lo estimado es el valor generalizado a la poblaci\'on. Consiste en la b\'usqueda del valor de los par\'ametros poblacionales objeto de estudio. Puede ser puntual o por intervalo de confianza:
    \begin{itemize}
        \item \textit{Puntual:} cuando buscamos un valor concreto. Un estimador de un par\'ametro poblacional es una funci\'on de los datos muestrales. En pocas palabras, es una f\'ormula que depende de los valores obtenidos de una muestra, para realizar estimaciones. Lo que se pretende obtener es el valor exacto de un par\'ametro.
    \end{itemize}
    \item \textit{Intervalo de confianza:} cuando determinamos un intervalo, dentro del cual se supone que va a estar el valor del par\'ametro que se busca con una cierta probabilidad. El intervalo de confianza est\'a determinado por dos valores dentro de los cuales afirmamos que est\'a el verdadero par\'ametro con cierta probabilidad. Son unos l\'imites o margen de variabilidad que damos al valor estimado, para poder afirmar, bajo un criterio de probabilidad, que el verdadero valor no los rebasar\'a.
\end{itemize}

Este intervalo contiene al par\'ametro estimado con una determinada certeza o nivel de confianza. 

En la estimaci\'on por intervalos se usan los siguientes conceptos:

\begin{itemize}
    \item Variabilidad del par\'ametro: Si no se conoce, puede obtenerse una aproximaci\'on en los datos o en un estudio piloto. Tambi\'en hay m\'etodos para calcular el tama\~no de la muestra que prescinden de este aspecto. Habitualmente se usa como medida de esta variabilidad la desviaci\'on t\'ipica poblacional.
    \item Error de la estimaci\'on: Es una medida de su precisi\'on que se corresponde con la amplitud del intervalo de confianza. Cuanta m\'as precisi\'on se desee en la estimaci\'on de un par\'ametro, m\'as estrecho deber\'a ser el intervalo de confianza y, por tanto, menor el error, y m\'as sujetos deber\'an incluirse en la muestra estudiada. 
    \item Nivel de confianza: Es la probabilidad de que el verdadero valor del par\'ametro estimado en la poblaci\'on se sit\'ue en el intervalo de confianza obtenido. El nivel de confianza se denota por $1-\alpha$
    \item $p$-value : Tambi\'en llamado nivel de significaci\'on. Es la probabilidad (en tanto por uno) de fallar en nuestra estimaci\'on, esto es, la diferencia entre la certeza (1) y el nivel de confianza $1-\alpha$. 
    \item Valor cr\'itico: Se representa por $Z_{\alpha/2}$. Es el valor de la abscisa en una determinada distribuci\'on que deja a su derecha un \'area igual a 1/2, siendo $1-\alpha$ el nivel de confianza. Normalmente los valores cr\'iticos est\'an tabulados o pueden calcularse en funci\'on de la distribuci\'on de la poblaci\'on.
\end{itemize}

Para un tama\~no fijo de la muestra, los conceptos de error y nivel de confianza van relacionados. Si admitimos un error mayor, esto es, aumentamos el tama\~no del intervalo de confianza, tenemos tambi\'en una mayor probabilidad de \'exito en nuestra estimaci\'on, es decir, un mayor nivel de confianza. Por tanto, un aspecto que debe de tenerse en cuenta es el tama\~no muestral, ya que para disminuir el error que se comente habr\'a que aumentar el tama\~no muestral. Esto se resolver\'a, para un intervalo de confianza cualquiera, despejando el tama\~no de la muestra en cualquiera de las formulas de los intervalos de confianza que veremos a continuaci\'on, a partir del error m\'aximo permitido. Los intervalos de confianza pueden ser unilaterales o bilaterales:

\begin{itemize}
    \item \textbf{Contraste de Hip\'otesis:} Consiste en determinar si es aceptable, partiendo de datos muestrales, que la caracter\'istica o el par\'ametro poblacional estudiado tome un determinado valor o est\'e dentro de unos determinados valores.
    \item \textbf{Nivel de Confianza:} Indica la proporci\'on de veces que acertar\'iamos al afirmar que el par\'ametro est\'a dentro del intervalo al seleccionar muchas muestras.
\end{itemize}

\subsection{Muestreo:} 

Una muestra es representativa en la medida que es imagen de la poblaci\'on. 

En general, podemos decir que el tama\~no de una muestra depender\'a principalmente de: \textit{Nivel de precisi\'on deseado, Recursos disponibles, Tiempo involucrado en la investigaci\'on.} Adem\'as el plan de muestreo debe considerar \textit{La poblaci\'on, Par\'ametros a medir}.

Existe una gran cantidad de tipos de muestreo. En la pr\'actica los m\'as utilizados son los siguientes:

\begin{itemize}
    \item \textbf{MUESTREO ALEATORIO SIMPLE:} Es un m\'etodo de selecci\'on de $n$ unidades extra\'idas de $N$, de tal manera que cada una de las posibles muestras tiene la misma probabilidad de ser escogida. (En la pr\'actica, se enumeran las unidades de 1 a $N$, y a continuaci\'on se seleccionan $n$ n\'umeros aleatorios entre 1 y $N$, ya sea de tablas o de alguna urna con fichas numeradas).
    \item \textbf{MUESTREO ESTRATIFICADO ALEATORIO:} Se usa cuando la poblaci\'on est\'a agrupada en pocos estratos, cada uno de ellos son muchas entidades. Este muestreo consiste en sacar una muestra aleatoria simple de cada uno de los estratos. (Generalmente, de tama\~no proporcional al estrato).
    \item \textbf{MUESTREO SISTEM\'ATICO:} Se utiliza cuando las unidades de la poblaci\'on est\'an de alguna manera totalmente ordenadas. Para seleccionar una muestra de $n$ unidades, se divide la poblaci\'on en $n$ subpoblaciones de tama\~no $K = N/n$ y se toma al azar una unidad de la $K$ primeras y de ah\'i en adelante cada $K$-\'esima unidad.
    \item \textbf{MUESTREO POR CONGLOMERADO:} Se emplea cuando la poblaci\'on est\'a dividida en grupos o conglomerados peque\~nos. Consiste en obtener una muestra aleatoria simple de conglomerados y luego CENSAR cada uno de \'estos.
    \item \textbf{MUESTREO EN DOS ETAPAS (Biet\'apico):} En este caso la muestra se toma en dos pasos:
    \begin{itemize}
        \item Seleccionar una muestra de unidades primarias, y 
        \item Seleccionar una muestra de elementos a partir de cada unidad primaria escogida.
        \item \textit{Observaci\'on:} En la realidad es posible encontrarse con situaciones en las cuales no es posible aplicar libremente un tipo de muestreo, incluso estaremos obligados a mezclarlas en ocasiones.
    \end{itemize}
\end{itemize}

Las variables se pueden clasificar en dos grandes grupos:

\begin{itemize}
    \item \textbf{Variables categ\'oricas:} Son aquellas que pueden ser representadas a trav\'es de s\'imbolos, letras, palabras, etc. Los valores que toman se denominan categor\'ias, y los elementos que pertenecen a estas categor\'ias, se consideran id\'enticos respecto a la caracter\'istica que se est\'a midiendo. Las variables categ\'oricas de dividen en dos tipos: Ordinal y Nominal.
    \begin{itemize}
        \item \textbf{Las Ordinales:} son aquellas en que las categor\'ias tienen un orden impl\'icito. Admiten grados de calidad, es decir, existe una relaci\'on total entre las categor\'ias.
        \item \textbf{Las Nominales:} son aquellas donde no existe una relaci\'on de orden.
    \end{itemize}
    \item \textbf{Variables Num\'ericas:} Son aquellas que pueden tomar valores num\'ericos exclusivamente (mediciones). Se dividen en dos tipos: Discretas y continuas.
    \begin{itemize}
        \item \textbf{Discretas:} son aquellas que toman sus valores en un conjunto finito o infinito numerable.
        \item \textbf{Continuas:} Son aquellas que toman sus valores en un subconjunto de los n\'umeros reales, es decir en un intervalo. En general para las variables continuas el hombre ha debido inventar una medida para poder establecer una medici\'on de ellas.
    \end{itemize}
\end{itemize}

El prop\'osito de esta secci\'on es solamente indicar los malos usos comunes de datos estad\'isticos, sin incluir el uso de m\'etodos estad\'isticos complicados. Un estudiante deber\'ia estar alerta en relaci\'on con estos malos usos y deber\'ia hacer un gran esfuerzo para evitarlos a fin de ser un verdadero estad\'istico.

\textbf{Datos estad\'isticos inadecuados}

Los datos estad\'isticos son usados como la materia prima para un estudio estad\'istico. Cuando los datos son inadecuados, la conclusi\'on extra\'ida del estudio de los datos se vuelve obviamente inv\'alida. Por ejemplo, supongamos que deseamos encontrar el ingreso familiar t\'ipico del a\~no pasado en la ciudad Y de 50,000 familias y tenemos una muestra consistente del ingreso de solamente tres familias: 1 mill\'on, 2 millones y no ingreso. Si sumamos el ingreso de las tres familias y dividimos el total por 3, obtenemos un promedio de 1 mill\'on.

Entonces, extraemos una conclusi\'on basada en la muestra de que el ingreso familiar promedio durante el a\~no pasado en la ciudad fue de 1 mill\'on. Es obvio que la conclusi\'on es falsa, puesto que las cifras son extremas y el tama\~no de la muestra es demasiado peque\~no; por lo tanto la muestra no es representativa. 

Hay muchas otras clases de datos inadecuados. Por ejemplo, algunos datos son respuestas inexactas de una encuesta, porque las preguntas usadas en la misma son vagas o enga\~nosas, algunos datos son toscas estimaciones porque no hay disponibles datos exactos o es demasiado costosa su obtenci\'on, y algunos datos son irrelevantes en un problema dado, porque el estudio estad\'istico no est\'a bien planeado.

\subsection{Un sesgo del usuario}

Sesgo significa que un usuario d\'e los datos perjudicialmente de m\'as \'enfasis a los hechos, los cuales son empleados para mantener su predeterminada posici\'on u opini\'on. Los estad\'isticos son frecuentemente degradados por lemas tales como: \textit{Hay tres clases de mentiras: mentiras, mentiras reprobables y estad\'istica, y Las cifras no mienten, pero los mentirosos piensan.}

Hay dos clases de sesgos: conscientes e inconscientes. Ambos son comunes en el an\'alisis estad\'istico. Hay numerosos ejemplos de sesgos conscientes. Un anunciante frecuentemente usa la estad\'istica para probar que su producto es muy superior al producto de su competidor. Un pol\'itico prefiere usar la estad\'istica para sostener su punto de vista. Gerentes y l\'ideres de trabajadores pueden simult\'aneamente situar sus respectivas cifras estad\'isticas sobre la misma tabla de trato para mostrar que sus rechazos o peticiones son justificadas. 

Es casi imposible que un sesgo inconsciente est\'e completamente ausente en un trabajo estad\'istico. En lo que respecta al ser humano, es dif\'icil obtener una actitud completamente objetiva al abordar un problema, aun cuando un cient\'ifico deber\'ia tener una mente abierta. Un estad\'istico deber\'ia estar enterado del hecho de que su interpretaci\'on de los resultados del an\'alisis estad\'istico est\'a influenciado por su propia experiencia, conocimiento y antecedentes con relaci\'on al problema dado.

\subsection{Supuestos falsos}

Es muy frecuente que un an\'alisis estad\'istico contemple supuestos. Un investigador debe ser muy cuidadoso en este hecho, para evitar que \'estos sean falsos. Los supuestos falsos pueden ser originados por:

\begin{itemize}
    \item Quien usa los datos.
    \item Quien est\'a tratando de confundir (con intencionalidad).
    \item Ignorancia.
    \item Descuido.
\end{itemize}

\section{Muestreo}

\textbf{Muestreo:} Una muestra es representativa en la medida que es imagen de la poblaci\'on.

En general, podemos decir que el tama\~no de una muestra depender\'a principalmente de:

\begin{itemize}
    \item Nivel de precisi\'on deseado.
    \item Recursos disponibles.
    \item Tiempo involucrado en la investigaci\'on.
\end{itemize}

Adem\'as el plan de muestreo debe considerar

\begin{itemize}
    \item La poblaci\'on.
    \item Par\'ametros a medir.
\end{itemize}

Existe una gran cantidad de tipos de muestreo. En la pr\'actica los m\'as utilizados son los siguientes:

\begin{itemize}
    \item \textbf{MUESTREO ALEATORIO SIMPLE:} Es un m\'etodo de selecci\'on de $n$ unidades extra\'idas de $N$, de tal manera que cada una de las posibles muestras tiene la misma probabilidad de ser escogida. (En la pr\'actica, se enumeran las unidades de $1$ a $N$, y a continuaci\'on se seleccionan $n$ n\'umeros aleatorios entre $1$ y $N$, ya sea de tablas o de alguna urna con fichas numeradas).
    \item \textbf{MUESTREO ESTRATIFICADO ALEATORIO:} Se usa cuando la poblaci\'on est\'a agrupada en pocos estratos, cada uno de ellos son muchas entidades. Este muestreo consiste en sacar una muestra aleatoria simple de cada uno de los estratos. (Generalmente, de tama\~no proporcional al estrato).
    \item \textbf{MUESTREO SISTEM\'ATICO:} Se utiliza cuando las unidades de la poblaci\'on est\'an de alguna manera totalmente ordenadas. Para seleccionar una muestra de $n$ unidades, se divide la poblaci\'on en $n$ subpoblaciones de tama\~no $K = N/n$ y se toma al azar una unidad de la $K$ primeras y de ah\'i en adelante cada $K$-\'esima unidad.
    \item \textbf{MUESTREO POR CONGLOMERADO:} Se emplea cuando la poblaci\'on est\'a dividida en grupos o conglomerados peque\~nos. Consiste en obtener una muestra aleatoria simple de conglomerados y luego CENSAR cada uno de \'estos.
    \item \textbf{MUESTREO EN DOS ETAPAS (Biet\'apico):} En este caso la muestra se toma en dos pasos:
    \begin{itemize}
        \item Seleccionar una muestra de unidades primarias, y 
        \item Seleccionar una muestra de elementos a partir de cada unidad primaria escogida.
        \item \textit{Observaci\'on:} En la realidad es posible encontrarse con situaciones en las cuales no es posible aplicar libremente un tipo de muestreo, incluso estaremos obligados a mezclarlas en ocasiones.
    \end{itemize}
\end{itemize}

Las variables se pueden clasificar en dos grandes grupos:

\begin{itemize}
    \item \textbf{Variables categ\'oricas:} Son aquellas que pueden ser representadas a trav\'es de s\'imbolos, letras, palabras, etc. Los valores que toman se denominan categor\'ias, y los elementos que pertenecen a estas categor\'ias, se consideran id\'enticos respecto a la caracter\'istica que se est\'a midiendo. Las variables categ\'oricas de dividen en dos tipos: Ordinal y Nominal.
    \begin{itemize}
        \item \textbf{Las Ordinales:} son aquellas en que las categor\'ias tienen un orden impl\'icito. Admiten grados de calidad, es decir, existe una relaci\'on total entre las categor\'ias.
        \item \textbf{Las Nominales:} son aquellas donde no existe una relaci\'on de orden.
    \end{itemize}
    \item \textbf{Variables Num\'ericas:} Son aquellas que pueden tomar valores num\'ericos exclusivamente (mediciones). Se dividen en dos tipos: Discretas y continuas.
    \begin{itemize}
        \item \textbf{Discretas:} son aquellas que toman sus valores en un conjunto finito o infinito numerable.
        \item \textbf{Continuas:} Son aquellas que toman sus valores en un subconjunto de los n\'umeros reales, es decir en un intervalo. En general para las variables continuas el hombre ha debido inventar una medida para poder establecer una medici\'on de ellas.
    \end{itemize}
\end{itemize}

El prop\'osito de esta secci\'on es solamente indicar los malos usos comunes de datos estad\'isticos, sin incluir el uso de m\'etodos estad\'isticos complicados. Un estudiante deber\'ia estar alerta en relaci\'on con estos malos usos y deber\'ia hacer un gran esfuerzo para evitarlos a fin de ser un verdadero estad\'istico.

\textbf{Datos estad\'isticos inadecuados}

Los datos estad\'isticos son usados como la materia prima para un estudio estad\'istico. Cuando los datos son inadecuados, la conclusi\'on extra\'ida del estudio de los datos se vuelve obviamente inv\'alida. Por ejemplo, supongamos que deseamos encontrar el ingreso familiar t\'ipico del a\~no pasado en la ciudad Y de 50,000 familias y tenemos una muestra consistente del ingreso de solamente tres familias: 1 mill\'on, 2 millones y no ingreso. Si sumamos el ingreso de las tres familias y dividimos el total por 3, obtenemos un promedio de 1 mill\'on.

Entonces, extraemos una conclusi\'on basada en la muestra de que el ingreso familiar promedio durante el a\~no pasado en la ciudad fue de 1 mill\'on. Es obvio que la conclusi\'on es falsa, puesto que las cifras son extremas y el tama\~no de la muestra es demasiado peque\~no; por lo tanto la muestra no es representativa. 

Hay muchas otras clases de datos inadecuados. Por ejemplo, algunos datos son respuestas inexactas de una encuesta, porque las preguntas usadas en la misma son vagas o enga\~nosas, algunos datos son toscas estimaciones porque no hay disponibles datos exactos o es demasiado costosa su obtenci\'on, y algunos datos son irrelevantes en un problema dado, porque el estudio estad\'istico no est\'a bien planeado.

\subsection{Un sesgo del usuario}

Sesgo significa que un usuario d\'e los datos perjudicialmente de m\'as \'enfasis a los hechos, los cuales son empleados para mantener su predeterminada posici\'on u opini\'on. Los estad\'isticos son frecuentemente degradados por lemas tales como: \textit{Hay tres clases de mentiras: mentiras, mentiras reprobables y estad\'istica, y Las cifras no mienten, pero los mentirosos piensan.}

Hay dos clases de sesgos: conscientes e inconscientes. Ambos son comunes en el an\'alisis estad\'istico. Hay numerosos ejemplos de sesgos conscientes. Un anunciante frecuentemente usa la estad\'istica para probar que su producto es muy superior al producto de su competidor. Un pol\'itico prefiere usar la estad\'istica para sostener su punto de vista. Gerentes y l\'ideres de trabajadores pueden simult\'aneamente situar sus respectivas cifras estad\'isticas sobre la misma tabla de trato para mostrar que sus rechazos o peticiones son justificadas. 

Es casi imposible que un sesgo inconsciente est\'e completamente ausente en un trabajo estad\'istico. En lo que respecta al ser humano, es dif\'icil obtener una actitud completamente objetiva al abordar un problema, aun cuando un cient\'ifico deber\'ia tener una mente abierta. Un estad\'istico deber\'ia estar enterado del hecho de que su interpretaci\'on de los resultados del an\'alisis estad\'istico est\'a influenciado por su propia experiencia, conocimiento y antecedentes con relaci\'on al problema dado.

\subsection{Supuestos falsos}

Es muy frecuente que un an\'alisis estad\'istico contemple supuestos. Un investigador debe ser muy cuidadoso en este hecho, para evitar que \'estos sean falsos. Los supuestos falsos pueden ser originados por:

\begin{itemize}
    \item Quien usa los datos.
    \item Quien est\'a tratando de confundir (con intencionalidad).
    \item Ignorancia.
    \item Descuido.
\end{itemize}
