\documentclass{article}
\usepackage[utf8]{inputenc}
\usepackage{amsmath}
\usepackage{amssymb}

\title{Día 1: Introducción a la Regresión Logística}
\author{Carlos}
\date{\today}

\begin{document}

\maketitle

\section{Conceptos Básicos de la Regresión Logística}

La regresión logística es una técnica de modelado estadístico utilizada para predecir la probabilidad de un evento binario (es decir, un evento que tiene dos posibles resultados) en función de una o más variables independientes. A diferencia de la regresión lineal, que se utiliza para predecir valores continuos, la regresión logística se usa cuando la variable dependiente es categórica.

\section{Diferencias entre Regresión Lineal y Logística}

\subsection{Regresión Lineal}
La regresión lineal busca modelar la relación entre una variable dependiente continua $Y$ y una o más variables independientes $X_1, X_2, \ldots, X_n$ mediante una ecuación de la forma:
\[
Y = \beta_0 + \beta_1 X_1 + \beta_2 X_2 + \ldots + \beta_n X_n + \epsilon
\]
donde $\beta_0, \beta_1, \ldots, \beta_n$ son los coeficientes del modelo y $\epsilon$ es el término de error.

\subsection{Regresión Logística}
La regresión logística, en cambio, modela la probabilidad de que un evento ocurra (por ejemplo, éxito vs. fracaso) utilizando la función logística. La variable dependiente $Y$ es binaria, tomando valores de 0 o 1. La ecuación de la regresión logística es:
\[
\text{logit}(p) = \log\left(\frac{p}{1-p}\right) = \beta_0 + \beta_1 X_1 + \beta_2 X_2 + \ldots + \beta_n X_n
\]
donde $p$ es la probabilidad de que $Y=1$. La función logística es:
\[
p = \frac{1}{1 + e^{-(\beta_0 + \beta_1 X_1 + \beta_2 X_2 + \ldots + \beta_n X_n)}}
\]

\section{Casos de Uso de la Regresión Logística}

La regresión logística se utiliza en una variedad de campos para problemas de clasificación binaria, tales como:
\begin{itemize}
    \item \textbf{Medicina}: Predicción de la presencia o ausencia de una enfermedad.
    \item \textbf{Marketing}: Determinación de la probabilidad de que un cliente compre un producto.
    \item \textbf{Finanzas}: Evaluación del riesgo de crédito, es decir, si un cliente va a incumplir o no con un préstamo.
    \item \textbf{Seguridad}: Detección de fraudes o intrusiones.
\end{itemize}

\section{Implementación Básica en R}

Para implementar una regresión logística en R, primero es necesario instalar y cargar los paquetes necesarios. Aquí se muestra un ejemplo básico de implementación:

\subsection{Instalación y Configuración de R y RStudio}
\begin{itemize}
    \item Descargue e instale R desde \texttt{https://cran.r-project.org/}.
    \item Descargue e instale RStudio desde \texttt{https://rstudio.com/products/rstudio/download/}.
\end{itemize}

\subsection{Introducción Básica a R}
\begin{itemize}
    \item Sintaxis básica de R.
    \item Operaciones básicas: asignación, operaciones aritméticas, funciones básicas.
\end{itemize}

\subsection{Ejemplo de Regresión Logística en R}

\begin{verbatim}
# Instalación del paquete necesario
install.packages("stats")

# Carga del paquete
library(stats)

# Ejemplo de conjunto de datos
data <- data.frame(
  outcome = c(1, 0, 1, 0, 1, 1, 0, 1, 0, 0),
  predictor = c(2.3, 1.9, 3.1, 2.8, 3.6, 2.4, 2.1, 3.3, 2.2, 1.7)
)

# Ajuste del modelo de regresión logística
model <- glm(outcome ~ predictor, data = data, family = binomial)

# Resumen del modelo
summary(model)
\end{verbatim}

En este ejemplo, se utiliza el conjunto de datos `data` que contiene una variable de resultado binaria `outcome` y una variable predictora continua `predictor`. El modelo de regresión logística se ajusta utilizando la función \texttt{glm} con la familia binomial.

\end{document}
