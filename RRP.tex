\documentclass{article}
\usepackage[utf8]{inputenc}
\usepackage[spanish,english]{babel}
\usepackage{amsmath,amssymb,amsthm,amsfonts}
\usepackage{geometry}
\usepackage{hyperref}
\usepackage{fancyhdr}
\usepackage{titlesec}
\usepackage{listings}
\usepackage{graphicx,graphics}
\usepackage{multicol}
\usepackage{multirow}
\usepackage{color}
\usepackage{float} 
\usepackage{subfig}
\usepackage[figuresright]{rotating}
\usepackage{enumerate}
\usepackage{anysize} 
\usepackage{url}

\title{Procesos Regenerativos: Revisi\'on}
\author{Carlos E. Martínez-Rodríguez}
\date{Julio 2024}

\geometry{
  a4paper,
  left=25mm,
  right=25mm,
  top=30mm,
  bottom=30mm,
}

% Configuración de encabezados y pies de página
\pagestyle{fancy}
\fancyhf{}
\fancyhead[L]{\leftmark}
\fancyfoot[C]{\thepage}
\fancyfoot[R]{\rightmark}
\fancyfoot[L]{Carlos E. Martínez-Rodríguez}

% Definiciones de nuevos entornos
\newtheorem{Algthm}{Algoritmo}
\newtheorem{Def}{Definición}
\newtheorem{Ejem}{Ejemplo}
\newtheorem{Teo}{Teorema}
\newtheorem{Dem}{Demostración}
\newtheorem{Note}{Nota}
\newtheorem{Sol}{Solución}
\newtheorem{Prop}{Proposición}
\newtheorem{Cor}{Corolario}
\newtheorem{Col}{Corolario}
\newtheorem{Coro}{Corolario}
\newtheorem{Lemma}{Lema}
\newtheorem{Lem}{Lema}
\newtheorem{Lema}{Lema}
\newtheorem{Sup}{Supuestos}
\newtheorem{Assumption}{Supuestos}
\newtheorem{Remark}{Observación}
\newtheorem{Condition}{Condición}
\newtheorem{Theorem}{Teorema}
\newtheorem{Corollary}{Corolario}
\newtheorem{Ejemplo}{Ejemplo}
\newtheorem{Example}{Ejemplo}
\newtheorem{Obs}{Observación}

% Nuevos comandos
\def\RR{\mathbb{R}}
\def\ZZ{\mathbb{Z}}
\newcommand{\nat}{\mathbb{N}}
\newcommand{\ent}{\mathbb{Z}}
\newcommand{\rea}{\mathbb{R}}
\newcommand{\Eb}{\mathbf{E}}
\newcommand{\esp}{\mathbb{E}}
\newcommand{\prob}{\mathbb{P}}
\newcommand{\indora}{\mbox{$1$\hspace{-0.8ex}$1$}}
\newcommand{\ER}{\left(E,\mathcal{E}\right)}
\newcommand{\KM}{\left(P_{s,t}\right)}
\newcommand{\Xt}{\left(X_{t}\right)_{t\in I}}
\newcommand{\PE}{\left(X_{t}\right)_{t\in I}}
\newcommand{\SG}{\left(P_{t}\right)}
\newcommand{\CM}{\mathbf{P}^{x}}
\newcommand\mypar{\par\vspace{\baselineskip}}

\begin{document}

\maketitle

\tableofcontents
%<>===<>==<>===<>==<>===<>==<>===<>==<>===<>==<>===<>==<>===<>==<>===<>==<>===
%________________________________________________________________________
\section{Procesos Regenerativos: Thorisson}
%________________________________________________________________________
%________________________________________________________________________
\subsection{Tiempos de Regeneraci\'on para Redes de Sistemas de Visitas C\'iclicas}
%________________________________________________________________________
\begin{Teo}
Dada una Red de Sistemas de Visitas C\'iclicas (RSVC), conformada por dos Sistemas de Visitas C\'iclicas (SVC), donde cada uno de ellos consta de dos colas tipo $M/M/1$. Los dos sistemas est\'an comunicados entre s\'i por medio de la transferencia de usuarios entre las colas $Q_{1}\leftrightarrow Q_{3}$ y $Q_{2}\leftrightarrow Q_{4}$.

\end{Teo}

\begin{proof}

Para cada cola $Q_{j}$, $j=1,\ldots,4$, se tienen los siguientes procesos $L_{j}\left(t\right)$ el n\'umero de usuarios presentes en la cola al tiempo $t$, $A_{j}\left(t\right)$ el residual del tiempo de arribo del siguiente usuario. $B_{j}\left(t\right)$ el residual del tiempo de servicio del usuario que est\'a siendo atendido. $C_{j}\left(t\right)$ el residual del tiempo de traslado del servidor entre una cola y otra, en caso de que se encuentre dando servicio se considera $C_{j}\left(t\right)=0$, para $j=1,\ldots,4$. Con base en lo anterior se tienen los procesos
\begin{eqnarray}\label{Procesos.RSVC}
L\left(t\right)=\left(L_{j}\left(t\right)\right)_{j=1}^{4},
A\left(t\right)=\left(A_{j}\left(t\right)\right)_{j=1}^{4}, B\left(t\right)=\left(B_{j}\left(t\right)\right)_{j=1}^{4}
\textrm{ y } C\left(t\right)=\left(C_{j}\left(t\right)\right)_{j=1}^{4}.
\end{eqnarray}
Por lo tanto se tiene el proceso estoc\'astico
\begin{eqnarray}\label{Proceso.Estocastico.Z}
\mathbb{Z}=\left(L\left(t\right),A\left(t\right),
B\left(t\right),C\left(t\right)\right)
\end{eqnarray}
Para los procesos residuales de los tiempos de traslado, servicio y de arribos, su espacio de estados es un subconjunto de $\rea_{+}=\left[0,\infty\right)$, es decir, $E\subset\left[0,\infty\right)$ y por tanto $\mathcal{E}\subset\mathcal{B}\left[0,\infty\right)$, luego el espacio $\left(E,\mathcal{E}\right)$ es un espacio polaco.
Para cada proceso de residuales se tienen los siguientes espacios producto: Para $A\left(t\right)=\left(A_{j}\left(t\right)\right)_{j=1}^{4}$ se tiene el espacio producto $\left(E_{2},\mathcal{E}_{2}\right)=\otimes_{j=1}^{4}\left(E_{j},\mathcal{E}_{j}\right)$, para $B\left(t\right)=\left(B_{j}\left(t\right)\right)_{j=1}^{4}$ se tiene el espacio producto $\left(E_{3},\mathcal{E}_{3}\right)=\otimes_{j=1}^{4}\left(E_{j},\mathcal{E}_{j}\right)$,
para $C\left(t\right)=\left(C_{j}\left(t\right)\right)_{j=1}^{4}$ se tiene el espacio producto $\left(E_{4},\mathcal{E}_{4}\right)=\otimes_{j=1}^{4}\left(E_{j},\mathcal{E}_{j}\right)$.

En lo que respecta al proceso $L\left(t\right)=\left(L_{j}\left(t\right)\right)_{j=1}^{4}$
 el proceso de estados $E_{j}\subset\mathbb{N}$ y $\mathcal{E}_{j}\subset\sigma\left(E\right)$, por lo tanto el espacio producto $\left(E_{1},\mathcal{E}_{1}\right)=\otimes_{j=1}^{4}\left(E_{j},\mathcal{E}_{j}\right)$ que adem\'as tambi\'en resulta ser polaco. Entonces con los espacios productos $\left(E_{i},\mathcal{E}_{i}\right)_{i=1}^{4}$, se define el espacio producto $\left(E,\mathcal{E}\right)=\otimes_{i=1}^{4} \left(E_{i},\mathcal{E}_{i}\right)$ que nuevamente resulta ser un espacio polaco. De acuerdo con Thorisson existe un espacio de probabilidad $\left(\Omega,\mathcal{F},\prob\right)$ en el que est\'a definido el proceso estoc\'astico definido en  (\ref{Proceso.Estocastico.Z}) que toma valores en $\left(E,\mathcal{E}\right)$.
  
Con la finalidad de analizar las propiedades del proceso $\mathbb{Z}$ consideremos el conjunto de \'indices $\mathbb{I}=\left[0,\infty\right)$, entonces tenemos el elemento aleatorio $\mathbb{Z}=\left(Z\right)_{s\in\mathbb{I}}$ que est\'a definido en el espacio de probabilidad $\left(\Omega,\mathcal{F},\prob\right)$ y con valores en $\left(E,\mathcal{E}\right)$. El proceso $Z$ as\'i definido es un PEOSCT conforme a la definici\'on dada en (\ref{PEOSCT}). Ahora consideremos al espacio de trayectorias de $Z$ conforme a la definici\'on (\ref{Conjunto.Trayectorias}); por construcci\'on el espacio de trayectorias $H:=D_{E}\left[0,\infty\right)$ que por la nota (\ref{Conjunto.Trayectorias}) resulta ser que el Proceso es Canonicamente Conjuntamente Medible (CCM) y por la nota (\ref{Nota.ISI.sii.CCM}) adem\'as es Internamente Shift Invariante (ISI), es decir, resulta ser un proceso estoc\'astico one-side a tiempo continuo shift medible, y por lo tanto satisface la primera parte de las hip\'otesis del Teorema (\ref{Tma.Existencia.Tiempos.Regeneracion}). 

Conforme a la construcci\'on dada en la secci\'on 1, se tiene que los dos tiempos $S_{0}=0$ y $S_{1}=T^{*}$ satisfacen la segunda parte de las hip\'otesis del Teorema (\ref{Tma.Existencia.Tiempos.Regeneracion}) y por tanto se puede asegurar que existe un espacio de probabilidad $\left(\Omega,\mathcal{F},\prob\right)$ en el cu\'al existe una sucesi\'on de tiempos aleatorios en los cuales el proceso se regenera, es decir, se garantiza que existe una sucesi\'on de tiempos de regeneraci\'on $T_{0}, T_{1},\ldots$ en los cuales el proceso $L\left(T_{k}\right)=\left(0,0,0,0\right)$.

Adem\'as por el Corolario (\ref{Tma.Estacionariedad}) se garantiza que existe una versi\'on estacionaria del proceso $\left(Z,S\right)$.
\end{proof}

\newpage

%_________________________________________________________________________
\subsection{Introduction to Stochastic Processes}
%_________________________________________________________________________

\begin{Def}
Un elemento aleatorio con valores en un espacio medible $\left(E,\mathcal{E}\right)$, es un mapeo definido en un espacio de probabilidad $\left(\Omega,\mathcal{F},\prob\right)$ a $\left(E,\mathcal{E}\right)$, es decir,
para $A\in \mathcal{E}$,  se tiene que $\left\{Y\in A\right\}\in\mathcal{F}$, donde $\left\{Y\in A\right\}:=\left\{w\in\Omega:Y\left(w\right)\in A\right\}=:Y^{-1}A$.
\end{Def}

\begin{Note}
Tambi\'en se dice que $Y$ est\'a soportado por el espacio de probabilidad $\left(\Omega,\mathcal{F},\prob\right)$ y que $Y$ es un mapeo medible de $\Omega$ en $E$, es decir, es \textbf{$\mathcal{F}/\mathcal{E}$ medible}.
\end{Note}

\begin{Def}
Para cada $i\in \mathbb{I}$, sea $P_{i}$ una medida de probabilidad en un espacio medible $\left(E_{i},\mathcal{E}_{i}\right)$. Se define el espacio producto
$\otimes_{i\in\mathbb{I}}\left(E_{i},\mathcal{E}_{i}\right):=\left(\prod_{i\in\mathbb{I}}E_{i},\otimes_{i\in\mathbb{I}}\mathcal{E}_{i}\right)$, donde $\prod_{i\in\mathbb{I}}E_{i}$ es el producto cartesiano de los $E_{i}$'s, y $\otimes_{i\in\mathbb{I}}\mathcal{E}_{i}$ es la \textbf{$\sigma$-\'algebra producto}, es decir, es la $\sigma$-\'algebra m\'as peque\~na en $\prod_{i\in\mathbb{I}}E_{i}$ que hace al $i$-\'esimo mapeo proyecci\'on en $E_{i}$ medible para toda $i\in\mathbb{I}$, es la $\sigma$-\'algebra inducida por los mapeos proyecci\'on, es decir
$$\otimes_{i\in\mathbb{I}}\mathcal{E}_{i}:=\sigma\left\{\left\{y:y_{i}\in A\right\}:i\in\mathbb{I}\textrm{ y }A\in\mathcal{E}_{i}\right\}.$$
\end{Def}

\begin{Def}
Un espacio de probabilidad $\left(\tilde{\Omega},\tilde{\mathcal{F}},\tilde{\prob}\right)$ es una \textbf{extensi\'on de otro espacio de probabilidad $\left(\Omega,\mathcal{F},\prob\right)$} si $\left(\tilde{\Omega},\tilde{\mathcal{F}},\tilde{\prob}\right)$ soporta un elemento aleatorio $\xi\in\left(\Omega,\mathcal{F}\right)$ que tienen a $\prob$ como distribuci\'on.
\end{Def}

\begin{Teo}
Sea $\mathbb{I}$ un conjunto de \'indices arbitrario. Para cada $i\in\mathbb{I}$ sea $P_{i}$ una medida de probabilidad en un espacio medible $\left(E_{i},\mathcal{E}_{i}\right)$. Entonces existe una \'unica medida de probabilidad $\otimes_{i\in\mathbb{I}}P_{i}$ en $\otimes_{i\in\mathbb{I}}\left(E_{i},\mathcal{E}_{i}\right)$ tal que 

\begin{eqnarray*}
\otimes_{i\in\mathbb{I}}P_{i}\left(y\in\prod_{i\in\mathbb{I}}E_{i}:y_{i}\in A_{i_{1}},\ldots,y_{n}\in A_{i_{n}}\right)=P_{i_{1}}\left(A_{i_{n}}\right)\cdots P_{i_{n}}\left(A_{i_{n}}\right)
\end{eqnarray*}
para todos los enteros $n>0$, toda $i_{1},\ldots,i_{n}\in\mathbb{I}$ y todo $A_{i_{1}}\in\mathcal{E}_{i_{1}},\ldots,A_{i_{n}}\in\mathcal{E}_{i_{n}}$
\end{Teo}

La medida $\otimes_{i\in\mathbb{I}}P_{i}$ es llamada la \textbf{medida producto} y $\otimes_{i\in\mathbb{I}}\left(E_{i},\mathcal{E}_{i},P_{i}\right):=\left(\prod_{i\in\mathbb{I}},E_{i},\otimes_{i\in\mathbb{I}}\mathcal{E}_{i},\otimes_{i\in\mathbb{I}}P_{i}\right)$, es llamado \textbf{espacio de probabilidad producto}.


\begin{Def}
Un espacio medible $\left(E,\mathcal{E}\right)$ es \textbf{\textit{Polaco}} si existe una m\'etrica en $E$ tal que $E$ es completo, es decir cada sucesi\'on de Cauchy converge a un l\'imite en $E$, y \textit{separable}, $E$ tienen un subconjunto denso numerable, y tal que $\mathcal{E}$ es generado por conjuntos abiertos.
\end{Def}


\begin{Def}
Dos espacios medibles $\left(E,\mathcal{E}\right)$ y $\left(G,\mathcal{G}\right)$ son Borel equivalentes (\textit{isomorfos}) si existe una biyecci\'on $f:E\rightarrow G$ tal que $f$ es $\mathcal{E}/\mathcal{G}$ medible y su inversa $f^{-1}$ es $\mathcal{G}/\mathcal{E}$ medible. La biyecci\'on es una equivalencia de Borel.
\end{Def}

\begin{Def}
Un espacio medible  $\left(E,\mathcal{E}\right)$ es un \textbf{espacio est\'andar} si es Borel equivalente a $\left(G,\mathcal{G}\right)$, donde $G$ es un subconjunto de Borel de $\left[0,1\right]$ y $\mathcal{G}$ son los subconjuntos de Borel de $G$.
\end{Def}

\begin{Note}
Cualquier espacio polaco es un espacio est\'andar.
\end{Note}


\begin{Def}
Un proceso estoc\'astico con conjunto de \'indices $\mathbb{I}$ y espacio de estados $\left(E,\mathcal{E}\right)$ es una familia $Z=\left(\mathbb{Z}_{s}\right)_{s\in\mathbb{I}}$ donde $\mathbb{Z}_{s}$ son elementos aleatorios definidos en un espacio de probabilidad com\'un $\left(\Omega,\mathcal{F},\prob\right)$ y todos toman valores en $\left(E,\mathcal{E}\right)$.
\end{Def}

\begin{Def}\label{PEOSCT}
Un proceso estoc\'astico \textit{one-sided contiuous time} (\textbf{PEOSCT}) es un proceso estoc\'astico con conjunto de \'indices $\mathbb{I}=\left[0,\infty\right)$.
\end{Def}


El espacio $\left(E^{\mathbb{I}},\mathcal{E}^{\mathbb{I}}\right)$ denota el espacio producto $\left(E^{\mathbb{I}},\mathcal{E}^{\mathbb{I}}\right):=\otimes_{s\in\mathbb{I}}\left(E,\mathcal{E}\right)$. Vamos a considerar $\mathbb{Z}$ como un mapeo aleatorio, es decir, como un elemento aleatorio en $\left(E^{\mathbb{I}},\mathcal{E}^{\mathbb{I}}\right)$ definido por $Z\left(w\right)=\left(Z_{s}\left(w\right)\right)_{s\in\mathbb{I}}$ y $w\in\Omega$.

\begin{Note}
La distribuci\'on de un proceso estoc\'astico $Z$ es la distribuci\'on de $Z$ como un elemento aleatorio en $\left(E^{\mathbb{I}},\mathcal{E}^{\mathbb{I}}\right)$. La distribuci\'on de $Z$ esta determinada de manera \'unica por las distribuciones finito dimensionales.
\end{Note}

\begin{Note}
En particular cuando $Z$ toma valores reales, es decir, $\left(E,\mathcal{E}\right)=\left(\mathbb{R},\mathcal{B}\right)$ las distribuciones finito dimensionales est\'an determinadas por las funciones de distribuci\'on finito dimensionales

\begin{eqnarray}
\prob\left(Z_{t_{1}}\leq x_{1},\ldots,Z_{t_{n}}\leq x_{n}\right),x_{1},\ldots,x_{n}\in\mathbb{R},t_{1},\ldots,t_{n}\in\mathbb{I},n\geq1.
\end{eqnarray}
\end{Note}

\begin{Note}
Para espacios polacos $\left(E,\mathcal{E}\right)$ el \textbf{Teorema de Consistencia de Kolmogorov} asegura que dada una colecci\'on de distribuciones finito dimensionales consistentes, siempre existe un proceso estoc\'astico que posee tales distribuciones finito dimensionales.
\end{Note}


\begin{Def}\label{Conjunto.Trayectorias}
Las trayectorias de $Z$ son las realizaciones $Z\left(w\right)$ para $w\in\Omega$ del mapeo aleatorio $Z$.
\end{Def}

\begin{Note}
Algunas restricciones se imponen sobre las trayectorias, por ejemplo que sean continuas por la derecha, o continuas por la derecha con l\'imites por la izquierda, o de manera m\'as general, se pedir\'a que caigan en alg\'un subconjunto $H$ de $E^{\mathbb{I}}$. En este caso es natural considerar a $Z$ como un elemento aleatorio que no est\'a en $\left(E^{\mathbb{I}},\mathcal{E}^{\mathbb{I}}\right)$ sino en $\left(H,\mathcal{H}\right)$, donde $\mathcal{H}$ es la $\sigma$-\'algebra generada por los mapeos proyecci\'on que toman a $z\in H$ en $z_{t}\in E$ para $t\in\mathbb{I}$. A $\mathcal{H}$ se le conoce como la traza de $H$ en $E^{\mathbb{I}}$, es decir,
\begin{eqnarray}
\mathcal{H}:=E^{\mathbb{I}}\cap H&:=&\left\{A\cap H:A\in E^{\mathbb{I}}\right\}.\\
Z_{t}:\left(\Omega.\mathcal{F}\right)&\rightarrow&\left(H,\mathcal{H}\right)
\end{eqnarray}
\end{Note}


\begin{Note}
$Z$ tiene \textbf{trayectorias con valores en $H$} y cada $Z_{t}$ es un mapeo medible de $\left(\Omega,\mathcal{F}\right)$ a $\left(H,\mathcal{H}\right)$. Cuando se considera un espacio de trayectorias en particular $H$, al espacio $\left(H,\mathcal{H}\right)$ se le llama \textbf{el espacio de trayectorias de $Z$}.
\end{Note}

\begin{Note}
La distribuci\'on del proceso estoc\'astico $Z$ con espacio de trayectorias $\left(H,\mathcal{H}\right)$ es la distribuci\'on de $Z$ como  un elemento aleatorio en $\left(H,\mathcal{H}\right)$. La distribuci\'on, nuevemente, est\'a determinada de manera \'unica por las distribuciones finito dimensionales.
\end{Note}


\begin{Def}
Sea $Z$ un PEOSCT (ver definici\'on \ref{PEOSCT}) con espacio de estados $\left(E,\mathcal{E}\right)$ y sea $T$ un tiempo aleatorio en $\left[0,\infty\right)$. Por $Z_{T}$ se entiende el mapeo con valores en $E$ definido en $\Omega$ por:
\begin{eqnarray*}
Z_{T}\left(w\right)&:=&Z_{T\left(w\right)}\left(w\right), w\in\Omega.\\
Z_{t}:\left(\Omega,\mathcal{F}\right)&\rightarrow&\left(E,\mathcal{E}\right).
\end{eqnarray*}
\end{Def}

\begin{Def}
Un PEOSCT $Z$ es conjuntamente medible (\textbf{CM}), es decir un \textbf{PEOSCTCM}, si el mapeo que toma $\left(w,t\right)\in\Omega\times\left[0,\infty\right)$ a $Z_{t}\left(w\right)\in E$ es $\mathcal{F}\otimes\mathcal{B}\left[0,\infty\right)/\mathcal{E}$ medible.
\begin{eqnarray*}
\left(\Omega,\left[0,\infty\right)\right)&\rightarrow&\left(E,\mathcal{E}\right)\\
\left(w,t\right)&\rightarrow& Z_{t}\left(w\right).
\end{eqnarray*}
\end{Def}

\begin{Note}
Un PEOSCT-CM implica que el proceso es medible, dado que $Z_{T}$ es una composici\'on  de dos mapeos continuos: el primero que toma $w$ en $\left(w,T\left(w\right)\right)$ es $\mathcal{F}/\mathcal{F}\otimes\mathcal{B}\left[0,\infty\right)$ medible, mientras que el segundo toma $\left(w,T\left(w\right)\right)$ en $Z_{T\left(w\right)}\left(w\right)$ es $\mathcal{F}\otimes\mathcal{B}\left[0,\infty\right)/\mathcal{E}$ medible.
\end{Note}


\begin{Def}
Un PEOSCT con espacio de estados $\left(H,\mathcal{H}\right)$ es can\'onicamente conjuntamente medible (\textbf{CCM}) si el mapeo $\left(z,t\right)\in H\times\left[0,\infty\right)$ en $Z_{t}\in E$ es $\mathcal{H}\otimes\mathcal{B}\left[0,\infty\right)/\mathcal{E}$ medible.
\begin{eqnarray*}
\left(H\times\left[0,\infty\right),\mathcal{H}\times\mathcal{B}\left[0,\infty\right)\right)&\rightarrow& \left(E,\mathcal{E}\right)\\
\left(z,t\right)&\rightarrow& Z_{t}
\end{eqnarray*}
\end{Def}

\begin{Note}
Un PEOSCTCCM implica que el proceso es CM, dado que un PEOSCTCCM $Z$ es un mapeo de $\Omega\times\left[0,\infty\right)$ a $E$, es la composici\'on de dos mapeos medibles: el primero, toma $\left(w,t\right)$ en $\left(Z\left(w\right),t\right)$ es $\mathcal{F}\otimes\mathcal{B}\left[0,\infty\right)/\mathcal{H}\otimes\mathcal{B}\left[0,\infty\right)$ medible, y el segundo que toma $\left(Z\left(w\right),t\right)$  en $Z_{t}\left(w\right)$ es $\mathcal{H}\otimes\mathcal{B}\left[0,\infty\right)/\mathcal{E}$ medible. Por tanto CCM es una condici\'on m\'as fuerte que CM.
\begin{eqnarray*}
\left(\Omega\times\left[0,\infty\right),\mathcal{F}\times\mathcal{B}\left[0,\infty\right)\right)
&\rightarrow& 
\left(H\times\left[0,\infty\right),\mathcal{H}\times\mathcal{B}\left[0,\infty\right)\right)
\rightarrow\left(E,\mathcal{E}\right)\\
\left(w,t\right)&\rightarrow& 
\left(Z\left(w\right),t\right])\rightarrow Z_{t}\left(w\right)
\end{eqnarray*}

\end{Note}

\begin{Def}
Un conjunto de trayectorias $H$ de un PEOSCT $Z$, es internamente shift-invariante (\textbf{ISI}) si 
\begin{eqnarray*}
\left\{\left(z_{t+s}\right)_{s\in\left[0,\infty\right)}:z\in H\right\}=H\textrm{, }t\in\left[0,\infty\right).
\end{eqnarray*}
\end{Def}


\begin{Def}
Dado un PEOSCTISI, se define el mapeo-shift $\theta_{t}$, $t\in\left[0,\infty\right)$, de $H$ a $H$ por 
\begin{eqnarray*}
\theta_{t}z=\left(z_{t+s}\right)_{s\in\left[0,\infty\right)}\textrm{, }z\in H.
\end{eqnarray*}
\end{Def}

\begin{Def}
Se dice que un proceso $Z$ es shift-medible (\textbf{SM}) si $Z$ tiene un conjunto de trayectorias $H$ que es ISI y adem\'as el mapeo que toma $\left(z,t\right)\in H\times\left[0,\infty\right)$ en $\theta_{t}z\in H$ es $\mathcal{H}\otimes\mathcal{B}\left[0,\infty\right)/\mathcal{H}$ medible.
\begin{eqnarray*}
\left(H\times\left[0,\infty\right),\mathcal{H}\times\mathcal{B}\left[0,\infty\right)\right)
&\rightarrow& 
\left(H,\mathcal{H}\right)\\
\left(z,t\right)&\rightarrow& 
\theta_{t}\left(z\right)
\end{eqnarray*}

\end{Def}

\begin{Note}\label{Nota.ISI.sii.CCM}
Un proceso estoc\'astico (PEOSCT) con conjunto de trayectorias $H$ ISI es shift-medible si y s\'olo si es PEOSCTCCM.
\end{Note}

\begin{Note}\label{Nota.ISI.CCM}
\begin{itemize}
\item Por la nota (\ref{Nota.ISI.sii.CCM}) dado el espacio polaco $\left(E,\mathcal{E}\right)$ si se tiene el  conjunto de trayectorias $D_{E}\left[0,\infty\right)$, que es ISI, entonces cumple con ser CCM.

\item Si $G$ es abierto, podemos cubrirlo por bolas abiertas cuya cerradura este contenida en $G$, y como $G$ es segundo numerable como subespacio de $E$, lo podemos cubrir por una cantidad numerable de bolas abiertas.

\end{itemize}
\end{Note}


\begin{Note}
Los procesos estoc\'asticos $Z$ a tiempo discreto con espacio de estados polaco, tambi\'en tiene un espacio de trayectorias polaco y por tanto tiene distribuciones condicionales regulares.
\end{Note}

\begin{Teo}
El producto numerable de espacios polacos es polaco.
\end{Teo}

%__________________________________________________________
\subsection{One Sided Process}
%___________________________________________________________

%\begin{Def}
Sea $\left(\Omega,\mathcal{F},\prob\right)$ espacio de probabilidad que soporta al proceso $Z=\left(Z_{s}\right)_{s\in\left[0,\infty\right)}$ y $S=\left(S_{k}\right)_{0}^{\infty}$ donde $Z$ es un PEOSCTM con espacio de estados $\left(E,\mathcal{E}\right)$  y espacio de trayectorias $\left(H,\mathcal{H}\right)$  y adem\'as $S$ es una sucesi\'on de tiempos aleatorios one-sided que satisfacen la condici\'on $0\leq S_{0}<S_{1}<\cdots\rightarrow\infty$. Considerando $S$ como un mapeo medible de $\left(\Omega,\mathcal{F}\right)$ al espacio sucesi\'on $\left(L,\mathcal{L}\right)$, $S:\left(\Omega,\mathcal{F}\right)\rightarrow\left(L,\mathcal{L}\right)$, donde 
\begin{eqnarray*}
L=\left\{\left(s_{k}\right)_{0}^{\infty}\in\left[0,\infty\right)^{\left\{0,1,\ldots\right\}}:s_{0}<s_{1}<\cdots\rightarrow\infty\right\},
\end{eqnarray*}
donde $\mathcal{L}$ son los subconjuntos de Borel de $L$, es decir, $\mathcal{L}=L\cap\mathcal{B}^{\left\{0,1,\ldots\right\}}$.

As\'i el par $\left(Z,S\right)$ es un mapeo medible de  $\left(\Omega,\mathcal{F}\right)$ en $\left(H\times L,\mathcal{H}\otimes\mathcal{L}\right)$. El par $\mathcal{H}\otimes\mathcal{L}^{+}$ denotar\'a la clase de todas las funciones medibles de $\left(H\times L,\mathcal{H}\otimes\mathcal{L}\right)$ en $\left(\left[0,\infty\right),\mathcal{B}\left[0,\infty\right)\right)$.
%\end{Def}

\begin{eqnarray*}
\left(Z,S\right):\left(\Omega,\mathcal{F}\right)&\rightarrow& \left(H\times L,\mathcal{H}\times\mathcal{L}\right)\\
\mathcal{H}\times\mathcal{L}^{*}:\left(H\times L,\mathcal{H}\times\mathcal{L}\right)
&\rightarrow& 
\left(\left[0,\infty\right),\mathcal{B}\left[0,\infty\right)\right).
\end{eqnarray*}



%_________________________________________________________
\subsection{Regeneration: Shift-Measurability}
%__________________________________________________________

\begin{Def}
Sea $\theta_{t}$ el mapeo-shift conjunto de $H\times L$ en $H\times L$ dado por
\begin{eqnarray*}
\theta_{t}\left(z,\left(s_{k}\right)_{0}^{\infty}\right)=\theta_{t}\left(z,\left(s_{n_{t-}+k}-t\right)_{0}^{\infty}\right)
\end{eqnarray*}
donde 
$n_{t-}=inf\left\{n\geq1:s_{n}\geq t\right\}$.
\end{Def}


\begin{Note}
Con la finalidad de poder realizar los shift's sin complicaciones de medibilidad, se supondr\'a que $Z$ es shit-medible, es decir, el conjunto de trayectorias $H$ es invariante bajo shifts del tiempo y el mapeo que toma $\left(z,t\right)\in H\times\left[0,\infty\right)$ en $z_{t}\in E$ es $\mathcal{H}\otimes\mathcal{B}\left[0,\infty\right)/\mathcal{E}$ medible.
\end{Note}




%_________________________________________________________
\subsection{Cycle-Stationarity}
%_________________________________________________________
%\textit{\textbf{Faltan definiciones}}
\begin{Def}
Los tiempos aleatorios $S_{n}$ dividen $Z$ en 

\begin{itemize}
\item[a)] un retraso $D=\left(Z_{s}\right)_{s\in\left[0,\infty\right)}$,
\item[b)] una sucesi\'on de ciclos $C_{n}=\left(Z_{S_{n-1}+s}\right)_{
s\in\left[0,X_{n}\right)}$, $n\geq1$,
\item[c)] las longitudes de los ciclos $X_{n}=S_{n}-S_{n-1}$, $n\neq1$.
\end{itemize}
\end{Def}

\begin{Note}
\begin{itemize}
\item[a)] El retraso $D$ y los ciclos $C_{n}$ son procesos estoc\'asticos que se desvanecen en los tiempos aleatorios $S_{0}$ y $X_{n}$ respectivamente.
\item[b)] Las longitudes de los ciclos $X_{1},X_{2},\ldots$ y el retraso de la longitud (\textit{delay-length}) $S_{0}$ son obtenidos por el mismo mapeo medible de sus respectivos ciclos $C_{1},C_{2},\ldots$ y el retraso $D$. 
\item[c)] El par $\left(Z,S\right)$ es un mapeo medible del retraso y de los ciclos y viceversa.
\end{itemize}
\end{Note}

\begin{Def}
$\left(Z,S\right)$ es \textit{zero-delayed} si $S_{0}\equiv0$. Se define el par \textit{zero-delayed} por
\begin{eqnarray*}
\left(Z^{0},S^{0}\right):=\theta_{S_{0}}\left(Z,S\right)
\end{eqnarray*}
Entonces $S_{0}^{0}\equiv0$ y $S_{0}^{0}\equiv X_{1}^{0}$, mientras que para $n\geq1$ se tiene que $X_{n}^{0}\equiv X_{n}$ y $C_{n}^{0}\equiv C_{n}$.
\end{Def}

\begin{Def}
Se le llama al par $\left(Z,S\right)$ \textbf{ciclo-stacionario} si los ciclos forman una sucesi\'on estacionaria, es decir, con $=^{D}$ denota iguales en distribuci\'on:
\begin{eqnarray*}
\left(C_{n+1},C_{n+2},\ldots\right)=^{D}\left(C_{1},C_{2},\ldots\right),\geq0
\end{eqnarray*}
Ciclo-estacionareidad es equivalente a 
\begin{eqnarray*}
\theta_{S_{n}}\left(Z,S\right)=^{D}
\left(Z^{0},S^{0}\right),\geq0,
\end{eqnarray*}
donde $\left(C_{n+1},C_{n+2},\ldots\right)$ y $\theta_{S_{n}}\left(Z,S\right)$ son mapeos medibles de cada uno y que no dependen de $n$.
\end{Def}


\begin{Def}
Un par $\left(Z^{*},S^{*}\right)$ es \textbf{estacionario} si $\theta\left(Z^{*},S^{*}\right)=^{D}
\left(Z^{*},S^{*}\right)$, para $t\geq0$.
\end{Def}


\begin{Teo}\label{Teorema.2.1}
Supongase que $\left(Z,S\right)$ es cycle-stationary con $\esp\left[X_{1}\right]<\infty$. Sea $U$ distribuida uniformemente en $\left[0,1\right)$ e independiente de $\left(Z^{0},S^{0}\right)$ y sea $\prob^{*}$ la medida de probabilidad en $\left(\Omega,\prob\right)$ definida por $$d\prob^{*}=\frac{X_{1}}{\esp\left[X_{1}\right]}d\prob$$. Sea $\left(Z^{*},S^{*}\right)$ con distribuci\'on $\prob^{*}\left(\theta_{UX_{1}}\left(Z^{0},S^{0}\right)\in\cdot\right)$. Entonces $\left(Z^{*},S^{*}\right)$ es estacionario,
\begin{eqnarray*}
\esp\left[f\left(Z^{*},S^{*}\right)\right]=\esp\left[\int_{0}^{X_{1}}f\left(\theta_{s}\left(Z^{0},S^{0}\right)\right)ds\right]/\esp\left[X_{1}\right]
\end{eqnarray*}
$f\in\mathcal{H}\otimes\mathcal{L}^{+}$, and $S_{0}^{*}$ es continuo con funci\'on distribuci\'on $G_{\infty}$ definida por $$G_{\infty}\left(x\right):=\frac{\esp\left[X_{1}\right]\wedge x}{\esp\left[X_{1}\right]}$$ para $x\geq0$ y densidad $\prob\left[X_{1}>x\right]/\esp\left[X_{1}\right]$, con $x\geq0$.

\end{Teo}

%___________________________________________________________
\subsection{Classical Regeneration}
%___________________________________________________________

\begin{Def}
Dado un proceso \textbf{PEOSSM} (Proceso Estoc\'astico One Side Shift Medible) $Z$, se dice \textbf{regenerativo cl\'asico} con tiempos de regeneraci\'on $S$ si 

\begin{eqnarray*}
\theta_{S_{n}}\left(Z,S\right)=\left(Z^{0},S^{0}\right),n\geq0
\end{eqnarray*}
y adem\'as $\theta_{S_{n}}\left(Z,S\right)$ es independiente de $\left(\left(Z_{s}\right)s\in\left[0,S_{n}\right),S_{0},\ldots,S_{n}\right)$
Si lo anterior se cumple, al par $\left(Z,S\right)$ se le llama regenerativo cl\'asico.
\end{Def}

\begin{Note}
Si el par $\left(Z,S\right)$ es regenerativo cl\'asico, entonces las longitudes de los ciclos $X_{1},X_{2},\ldots,$ son i.i.d. e independientes de la longitud del retraso $S_{0}$, es decir, $S$ es un \textbf{proceso de renovaci\'on}. Las longitudes de los ciclos tambi\'en son llamados tiempos de inter-regeneraci\'on y tiempos de ocurrencia.

\end{Note}

%___________________________________________________________
\subsection{Stationary Version}
%___________________________________________________________



\begin{Teo}\label{Teo.3.1}
Sup\'ongase que el par $\left(Z,S\right)$ es regenerativo cl\'asico con $\esp\left[X_{1}\right]<\infty$. Entonces $\left(Z^{*},S^{*}\right)$ en el teorema \ref{Teorema.2.1} es una versi\'on estacionaria de $\left(Z,S\right)$.
\end{Teo}

%___________________________________________________________
\subsection{Spread Out}
%___________________________________________________________


\begin{Def}
Una variable aleatoria $X_{1}$ es \textbf{spread out} si existe una $n\geq1$ y una  funci\'on $f\in\mathcal{B}^{+}$ tal que $\int_{\rea}f\left(x\right)dx>0$ con $X_{2},X_{3},\ldots,X_{n}$ copias i.i.d  de $X_{1}$, $$\prob\left(X_{1}+\cdots+X_{n}\in B\right)\geq\int_{B}f\left(x\right)dx$$ para $B\in\mathcal{B}$.
\end{Def}

%___________________________________________________________
\subsection{Wide Sense Regeneration}
%___________________________________________________________


\begin{Def}
Dado un proceso estoc\'astico $Z$ se le llama \textit{wide-sense regenerative} (\textbf{WSR}) con tiempos de regeneraci\'on $S$ si $\theta_{S_{n}}\left(Z,S\right)=\left(Z^{0},S^{0}\right)$ para $n\geq0$ en distribuci\'on y $\theta_{S_{n}}\left(Z,S\right)$ es independiente de $\left(S_{0},S_{1},\ldots,S_{n}\right)$ para $n\geq0$.
Se dice que el par $\left(Z,S\right)$ es WSR si lo anterior se cumple.
\end{Def}


\begin{Note}
\begin{itemize}
\item El proceso de trayectorias $\left(\theta_{s}Z\right)_{s\in\left[0,\infty\right)}$ es WSR con tiempos de regeneraci\'on $S$ pero no es regenerativo cl\'asico.

\item Si $Z$ es cualquier proceso estacionario y $S$ es un proceso de renovaci\'on que es independiente de $Z$, entonces $\left(Z,S\right)$ es WSR pero en general no es regenerativo cl\'asico

\end{itemize}

\end{Note}


\begin{Note}
Para cualquier proceso estoc\'astico $Z$, el proceso de trayectorias $\left(\theta_{s}Z\right)_{s\in\left[0,\infty\right)}$ es siempre un proceso de Markov.
\end{Note}


\begin{Teo}\label{Teo.4.1}
Supongase que el par $\left(Z,S\right)$ es WSR con $\esp\left[X_{1}\right]<\infty$. Entonces $\left(Z^{*},S^{*}\right)$ en el teorema (\ref{Teorema.2.1}) es una versi\'on estacionaria de 
$\left(Z,S\right)$.
\end{Teo}


%___________________________________________________________
\subsection{Existence of Regeneration Times}
%___________________________________________________________


\begin{Teo}\label{Tma.Existencia.Tiempos.Regeneracion}
Sea $Z$ un Proceso Estoc\'astico un lado shift-medible \textit{one-sided shift-measurable stochastic process}, (PEOSCTSM),
y $S_{0}$ y $S_{1}$ tiempos aleatorios tales que $0\leq S_{0}<S_{1}$ y
\begin{equation}
\theta_{S_{1}}Z=\theta_{S_{0}}Z\textrm{ en distribuci\'on}.
\end{equation}

Entonces el espacio de probabilidad subyacente $\left(\Omega,\mathcal{F},\prob\right)$ puede extenderse para soportar una sucesi\'on de tiempos aleatorios $S$ tales que

\begin{eqnarray}
\theta_{S_{n}}\left(Z,S\right)=\left(Z^{0},S^{0}\right),n\geq0,\textrm{ en distribuci\'on},\\
\left(Z,S_{0},S_{1}\right)\textrm{ depende de }\left(X_{2},X_{3},\ldots\right)\textrm{ solamente a traves de }\theta_{S_{1}}Z.
\end{eqnarray}
\end{Teo}

\begin{Coro}\label{Tma.Estacionariedad}
Bajo las condiciones del Teorema anterior (\ref{Tma.Existencia.Tiempos.Regeneracion}), el par $\left(Z,S\right)$ es regenerativo cl\'asico. Si adem\'as se tiene que $\esp\left[X_{1}\right]<\infty$ por el Teorema (\ref{Teo.3.1}) existe un par $\left(Z^{*},S^{*}\right)$ que es una vesi\'on estacionaria de $\left(Z,S\right)$.
\end{Coro}

\newpage

%________________________________________________________________________
\section{Procesos Regenerativos}
%________________________________________________________________________
%______________________________________________________________________
%\subsection*{Procesos Regenerativos}
%________________________________________________________________________



\begin{Note}
Si $\tilde{X}\left(t\right)$ con espacio de estados $\tilde{S}$ es regenerativo sobre $T_{n}$, entonces $X\left(t\right)=f\left(\tilde{X}\left(t\right)\right)$ tambi\'en es regenerativo sobre $T_{n}$, para cualquier funci\'on $f:\tilde{S}\rightarrow S$.
\end{Note}

\begin{Note}
Los procesos regenerativos son crudamente regenerativos, pero no al rev\'es.
\end{Note}
%\subsection*{Procesos Regenerativos: Sigman\cite{Sigman1}}
\begin{Def}[Definici\'on Cl\'asica]
Un proceso estoc\'astico $X=\left\{X\left(t\right):t\geq0\right\}$ es llamado regenerativo is existe una variable aleatoria $R_{1}>0$ tal que
\begin{itemize}
\item[i)] $\left\{X\left(t+R_{1}\right):t\geq0\right\}$ es independiente de $\left\{\left\{X\left(t\right):t<R_{1}\right\},\right\}$
\item[ii)] $\left\{X\left(t+R_{1}\right):t\geq0\right\}$ es estoc\'asticamente equivalente a $\left\{X\left(t\right):t>0\right\}$
\end{itemize}

Llamamos a $R_{1}$ tiempo de regeneraci\'on, y decimos que $X$ se regenera en este punto.
\end{Def}

$\left\{X\left(t+R_{1}\right)\right\}$ es regenerativo con tiempo de regeneraci\'on $R_{2}$, independiente de $R_{1}$ pero con la misma distribuci\'on que $R_{1}$. Procediendo de esta manera se obtiene una secuencia de variables aleatorias independientes e id\'enticamente distribuidas $\left\{R_{n}\right\}$ llamados longitudes de ciclo. Si definimos a $Z_{k}\equiv R_{1}+R_{2}+\cdots+R_{k}$, se tiene un proceso de renovaci\'on llamado proceso de renovaci\'on encajado para $X$.




\begin{Def}
Para $x$ fijo y para cada $t\geq0$, sea $I_{x}\left(t\right)=1$ si $X\left(t\right)\leq x$,  $I_{x}\left(t\right)=0$ en caso contrario, y def\'inanse los tiempos promedio
\begin{eqnarray*}
\overline{X}&=&lim_{t\rightarrow\infty}\frac{1}{t}\int_{0}^{\infty}X\left(u\right)du\\
\prob\left(X_{\infty}\leq x\right)&=&lim_{t\rightarrow\infty}\frac{1}{t}\int_{0}^{\infty}I_{x}\left(u\right)du,
\end{eqnarray*}
cuando estos l\'imites existan.
\end{Def}

Como consecuencia del teorema de Renovaci\'on-Recompensa, se tiene que el primer l\'imite  existe y es igual a la constante
\begin{eqnarray*}
\overline{X}&=&\frac{\esp\left[\int_{0}^{R_{1}}X\left(t\right)dt\right]}{\esp\left[R_{1}\right]},
\end{eqnarray*}
suponiendo que ambas esperanzas son finitas.

\begin{Note}
\begin{itemize}
\item[a)] Si el proceso regenerativo $X$ es positivo recurrente y tiene trayectorias muestrales no negativas, entonces la ecuaci\'on anterior es v\'alida.
\item[b)] Si $X$ es positivo recurrente regenerativo, podemos construir una \'unica versi\'on estacionaria de este proceso, $X_{e}=\left\{X_{e}\left(t\right)\right\}$, donde $X_{e}$ es un proceso estoc\'astico regenerativo y estrictamente estacionario, con distribuci\'on marginal distribuida como $X_{\infty}$
\end{itemize}
\end{Note}

Para $\left\{X\left(t\right):t\geq0\right\}$ Proceso Estoc\'astico a tiempo continuo con estado de espacios $S$, que es un espacio m\'etrico, con trayectorias continuas por la derecha y con l\'imites por la izquierda c.s. Sea $N\left(t\right)$ un proceso de renovaci\'on en $\rea_{+}$ definido en el mismo espacio de probabilidad que $X\left(t\right)$, con tiempos de renovaci\'on $T$ y tiempos de inter-renovaci\'on $\xi_{n}=T_{n}-T_{n-1}$, con misma distribuci\'on $F$ de media finita $\mu$.


\begin{Def}
Para el proceso $\left\{\left(N\left(t\right),X\left(t\right)\right):t\geq0\right\}$, sus trayectoria muestrales en el intervalo de tiempo $\left[T_{n-1},T_{n}\right)$ est\'an descritas por
\begin{eqnarray*}
\zeta_{n}=\left(\xi_{n},\left\{X\left(T_{n-1}+t\right):0\leq t<\xi_{n}\right\}\right)
\end{eqnarray*}
Este $\zeta_{n}$ es el $n$-\'esimo segmento del proceso. El proceso es regenerativo sobre los tiempos $T_{n}$ si sus segmentos $\zeta_{n}$ son independientes e id\'enticamennte distribuidos.
\end{Def}


\begin{Note}
Si $\tilde{X}\left(t\right)$ con espacio de estados $\tilde{S}$ es regenerativo sobre $T_{n}$, entonces $X\left(t\right)=f\left(\tilde{X}\left(t\right)\right)$ tambi\'en es regenerativo sobre $T_{n}$, para cualquier funci\'on $f:\tilde{S}\rightarrow S$.
\end{Note}

\begin{Note}
Los procesos regenerativos son crudamente regenerativos, pero no al rev\'es.
\end{Note}

\begin{Def}[Definici\'on Cl\'asica]
Un proceso estoc\'astico $X=\left\{X\left(t\right):t\geq0\right\}$ es llamado regenerativo is existe una variable aleatoria $R_{1}>0$ tal que
\begin{itemize}
\item[i)] $\left\{X\left(t+R_{1}\right):t\geq0\right\}$ es independiente de $\left\{\left\{X\left(t\right):t<R_{1}\right\},\right\}$
\item[ii)] $\left\{X\left(t+R_{1}\right):t\geq0\right\}$ es estoc\'asticamente equivalente a $\left\{X\left(t\right):t>0\right\}$
\end{itemize}

Llamamos a $R_{1}$ tiempo de regeneraci\'on, y decimos que $X$ se regenera en este punto.
\end{Def}

$\left\{X\left(t+R_{1}\right)\right\}$ es regenerativo con tiempo de regeneraci\'on $R_{2}$, independiente de $R_{1}$ pero con la misma distribuci\'on que $R_{1}$. Procediendo de esta manera se obtiene una secuencia de variables aleatorias independientes e id\'enticamente distribuidas $\left\{R_{n}\right\}$ llamados longitudes de ciclo. Si definimos a $Z_{k}\equiv R_{1}+R_{2}+\cdots+R_{k}$, se tiene un proceso de renovaci\'on llamado proceso de renovaci\'on encajado para $X$.

\begin{Note}
Un proceso regenerativo con media de la longitud de ciclo finita es llamado positivo recurrente.
\end{Note}


%_________________________________________________________________________
%
%\section{Appendix F: Output Process and Regenerative Processes}
%_________________________________________________________________________
%
En Sigman, Thorison y Wolff \cite{Sigman2} prueban que para la existencia de un una sucesi\'on infinita no decreciente de tiempos de regeneraci\'on $\tau_{1}\leq\tau_{2}\leq\cdots$ en los cuales el proceso se regenera, basta un tiempo de regeneraci\'on $R_{1}$, donde $R_{j}=\tau_{j}-\tau_{j-1}$. Para tal efecto se requiere la existencia de un espacio de probabilidad $\left(\Omega,\mathcal{F},\prob\right)$, y proceso estoc\'astico $\textit{X}=\left\{X\left(t\right):t\geq0\right\}$ con espacio de estados $\left(S,\mathcal{R}\right)$, con $\mathcal{R}$ $\sigma$-\'algebra.

\begin{Prop}
Si existe una variable aleatoria no negativa $R_{1}$ tal que $\theta_{R1}X=_{D}X$, entonces $\left(\Omega,\mathcal{F},\prob\right)$ puede extenderse para soportar una sucesi\'on estacionaria de variables aleatorias $R=\left\{R_{k}:k\geq1\right\}$, tal que para $k\geq1$,
\begin{eqnarray*}
\theta_{k}\left(X,R\right)=_{D}\left(X,R\right).
\end{eqnarray*}

Adem\'as, para $k\geq1$, $\theta_{k}R$ es condicionalmente independiente de $\left(X,R_{1},\ldots,R_{k}\right)$, dado $\theta_{\tau k}X$.

\end{Prop}


\begin{itemize}
\item Doob en 1953 demostr\'o que el estado estacionario de un proceso de partida en un sistema de espera $M/G/\infty$, es Poisson con la misma tasa que el proceso de arribos.

\item Burke en 1968, fue el primero en demostrar que el estado estacionario de un proceso de salida de una cola $M/M/s$ es un proceso Poisson.

\item Disney en 1973 obtuvo el siguiente resultado:

\begin{Teo}
Para el sistema de espera $M/G/1/L$ con disciplina FIFO, el proceso $\textbf{I}$ es un proceso de renovaci\'on si y s\'olo si el proceso denominado longitud de la cola es estacionario y se cumple cualquiera de los siguientes casos:

\begin{itemize}
\item[a)] Los tiempos de servicio son identicamente cero;
\item[b)] $L=0$, para cualquier proceso de servicio $S$;
\item[c)] $L=1$ y $G=D$;
\item[d)] $L=\infty$ y $G=M$.
\end{itemize}
En estos casos, respectivamente, las distribuciones de interpartida $P\left\{T_{n+1}-T_{n}\leq t\right\}$ son


\begin{itemize}
\item[a)] $1-e^{-\lambda t}$, $t\geq0$;
\item[b)] $1-e^{-\lambda t}*F\left(t\right)$, $t\geq0$;
\item[c)] $1-e^{-\lambda t}*\indora_{d}\left(t\right)$, $t\geq0$;
\item[d)] $1-e^{-\lambda t}*F\left(t\right)$, $t\geq0$.
\end{itemize}
\end{Teo}


\item Finch (1959) mostr\'o que para los sistemas $M/G/1/L$, con $1\leq L\leq \infty$ con distribuciones de servicio dos veces diferenciable, solamente el sistema $M/M/1/\infty$ tiene proceso de salida de renovaci\'on estacionario.

\item King (1971) demostro que un sistema de colas estacionario $M/G/1/1$ tiene sus tiempos de interpartida sucesivas $D_{n}$ y $D_{n+1}$ son independientes, si y s\'olo si, $G=D$, en cuyo caso le proceso de salida es de renovaci\'on.

\item Disney (1973) demostr\'o que el \'unico sistema estacionario $M/G/1/L$, que tiene proceso de salida de renovaci\'on  son los sistemas $M/M/1$ y $M/D/1/1$.



\item El siguiente resultado es de Disney y Koning (1985)
\begin{Teo}
En un sistema de espera $M/G/s$, el estado estacionario del proceso de salida es un proceso Poisson para cualquier distribuci\'on de los tiempos de servicio si el sistema tiene cualquiera de las siguientes cuatro propiedades.

\begin{itemize}
\item[a)] $s=\infty$
\item[b)] La disciplina de servicio es de procesador compartido.
\item[c)] La disciplina de servicio es LCFS y preemptive resume, esto se cumple para $L<\infty$
\item[d)] $G=M$.
\end{itemize}

\end{Teo}

\item El siguiente resultado es de Alamatsaz (1983)

\begin{Teo}
En cualquier sistema de colas $GI/G/1/L$ con $1\leq L<\infty$ y distribuci\'on de interarribos $A$ y distribuci\'on de los tiempos de servicio $B$, tal que $A\left(0\right)=0$, $A\left(t\right)\left(1-B\left(t\right)\right)>0$ para alguna $t>0$ y $B\left(t\right)$ para toda $t>0$, es imposible que el proceso de salida estacionario sea de renovaci\'on.
\end{Teo}

\end{itemize}

%________________________________________________________________________
%\subsection*{Procesos Regenerativos}
%________________________________________________________________________



\begin{Note}
Si $\tilde{X}\left(t\right)$ con espacio de estados $\tilde{S}$ es regenerativo sobre $T_{n}$, entonces $X\left(t\right)=f\left(\tilde{X}\left(t\right)\right)$ tambi\'en es regenerativo sobre $T_{n}$, para cualquier funci\'on $f:\tilde{S}\rightarrow S$.
\end{Note}

\begin{Note}
Los procesos regenerativos son crudamente regenerativos, pero no al rev\'es.
\end{Note}
%\subsection*{Procesos Regenerativos: Sigman\cite{Sigman1}}
\begin{Def}[Definici\'on Cl\'asica]
Un proceso estoc\'astico $X=\left\{X\left(t\right):t\geq0\right\}$ es llamado regenerativo is existe una variable aleatoria $R_{1}>0$ tal que
\begin{itemize}
\item[i)] $\left\{X\left(t+R_{1}\right):t\geq0\right\}$ es independiente de $\left\{\left\{X\left(t\right):t<R_{1}\right\},\right\}$
\item[ii)] $\left\{X\left(t+R_{1}\right):t\geq0\right\}$ es estoc\'asticamente equivalente a $\left\{X\left(t\right):t>0\right\}$
\end{itemize}

Llamamos a $R_{1}$ tiempo de regeneraci\'on, y decimos que $X$ se regenera en este punto.
\end{Def}

$\left\{X\left(t+R_{1}\right)\right\}$ es regenerativo con tiempo de regeneraci\'on $R_{2}$, independiente de $R_{1}$ pero con la misma distribuci\'on que $R_{1}$. Procediendo de esta manera se obtiene una secuencia de variables aleatorias independientes e id\'enticamente distribuidas $\left\{R_{n}\right\}$ llamados longitudes de ciclo. Si definimos a $Z_{k}\equiv R_{1}+R_{2}+\cdots+R_{k}$, se tiene un proceso de renovaci\'on llamado proceso de renovaci\'on encajado para $X$.




\begin{Def}
Para $x$ fijo y para cada $t\geq0$, sea $I_{x}\left(t\right)=1$ si $X\left(t\right)\leq x$,  $I_{x}\left(t\right)=0$ en caso contrario, y def\'inanse los tiempos promedio
\begin{eqnarray*}
\overline{X}&=&lim_{t\rightarrow\infty}\frac{1}{t}\int_{0}^{\infty}X\left(u\right)du\\
\prob\left(X_{\infty}\leq x\right)&=&lim_{t\rightarrow\infty}\frac{1}{t}\int_{0}^{\infty}I_{x}\left(u\right)du,
\end{eqnarray*}
cuando estos l\'imites existan.
\end{Def}

Como consecuencia del teorema de Renovaci\'on-Recompensa, se tiene que el primer l\'imite  existe y es igual a la constante
\begin{eqnarray*}
\overline{X}&=&\frac{\esp\left[\int_{0}^{R_{1}}X\left(t\right)dt\right]}{\esp\left[R_{1}\right]},
\end{eqnarray*}
suponiendo que ambas esperanzas son finitas.

\begin{Note}
\begin{itemize}
\item[a)] Si el proceso regenerativo $X$ es positivo recurrente y tiene trayectorias muestrales no negativas, entonces la ecuaci\'on anterior es v\'alida.
\item[b)] Si $X$ es positivo recurrente regenerativo, podemos construir una \'unica versi\'on estacionaria de este proceso, $X_{e}=\left\{X_{e}\left(t\right)\right\}$, donde $X_{e}$ es un proceso estoc\'astico regenerativo y estrictamente estacionario, con distribuci\'on marginal distribuida como $X_{\infty}$
\end{itemize}
\end{Note}

Para $\left\{X\left(t\right):t\geq0\right\}$ Proceso Estoc\'astico a tiempo continuo con estado de espacios $S$, que es un espacio m\'etrico, con trayectorias continuas por la derecha y con l\'imites por la izquierda c.s. Sea $N\left(t\right)$ un proceso de renovaci\'on en $\rea_{+}$ definido en el mismo espacio de probabilidad que $X\left(t\right)$, con tiempos de renovaci\'on $T$ y tiempos de inter-renovaci\'on $\xi_{n}=T_{n}-T_{n-1}$, con misma distribuci\'on $F$ de media finita $\mu$.


\begin{Def}
Para el proceso $\left\{\left(N\left(t\right),X\left(t\right)\right):t\geq0\right\}$, sus trayectoria muestrales en el intervalo de tiempo $\left[T_{n-1},T_{n}\right)$ est\'an descritas por
\begin{eqnarray*}
\zeta_{n}=\left(\xi_{n},\left\{X\left(T_{n-1}+t\right):0\leq t<\xi_{n}\right\}\right)
\end{eqnarray*}
Este $\zeta_{n}$ es el $n$-\'esimo segmento del proceso. El proceso es regenerativo sobre los tiempos $T_{n}$ si sus segmentos $\zeta_{n}$ son independientes e id\'enticamennte distribuidos.
\end{Def}


\begin{Note}
Si $\tilde{X}\left(t\right)$ con espacio de estados $\tilde{S}$ es regenerativo sobre $T_{n}$, entonces $X\left(t\right)=f\left(\tilde{X}\left(t\right)\right)$ tambi\'en es regenerativo sobre $T_{n}$, para cualquier funci\'on $f:\tilde{S}\rightarrow S$.
\end{Note}

\begin{Note}
Los procesos regenerativos son crudamente regenerativos, pero no al rev\'es.
\end{Note}

\begin{Def}[Definici\'on Cl\'asica]
Un proceso estoc\'astico $X=\left\{X\left(t\right):t\geq0\right\}$ es llamado regenerativo is existe una variable aleatoria $R_{1}>0$ tal que
\begin{itemize}
\item[i)] $\left\{X\left(t+R_{1}\right):t\geq0\right\}$ es independiente de $\left\{\left\{X\left(t\right):t<R_{1}\right\},\right\}$
\item[ii)] $\left\{X\left(t+R_{1}\right):t\geq0\right\}$ es estoc\'asticamente equivalente a $\left\{X\left(t\right):t>0\right\}$
\end{itemize}

Llamamos a $R_{1}$ tiempo de regeneraci\'on, y decimos que $X$ se regenera en este punto.
\end{Def}

$\left\{X\left(t+R_{1}\right)\right\}$ es regenerativo con tiempo de regeneraci\'on $R_{2}$, independiente de $R_{1}$ pero con la misma distribuci\'on que $R_{1}$. Procediendo de esta manera se obtiene una secuencia de variables aleatorias independientes e id\'enticamente distribuidas $\left\{R_{n}\right\}$ llamados longitudes de ciclo. Si definimos a $Z_{k}\equiv R_{1}+R_{2}+\cdots+R_{k}$, se tiene un proceso de renovaci\'on llamado proceso de renovaci\'on encajado para $X$.

\begin{Note}
Un proceso regenerativo con media de la longitud de ciclo finita es llamado positivo recurrente.
\end{Note}


\begin{Def}
Para $x$ fijo y para cada $t\geq0$, sea $I_{x}\left(t\right)=1$ si $X\left(t\right)\leq x$,  $I_{x}\left(t\right)=0$ en caso contrario, y def\'inanse los tiempos promedio
\begin{eqnarray*}
\overline{X}&=&lim_{t\rightarrow\infty}\frac{1}{t}\int_{0}^{\infty}X\left(u\right)du\\
\prob\left(X_{\infty}\leq x\right)&=&lim_{t\rightarrow\infty}\frac{1}{t}\int_{0}^{\infty}I_{x}\left(u\right)du,
\end{eqnarray*}
cuando estos l\'imites existan.
\end{Def}

Como consecuencia del teorema de Renovaci\'on-Recompensa, se tiene que el primer l\'imite  existe y es igual a la constante
\begin{eqnarray*}
\overline{X}&=&\frac{\esp\left[\int_{0}^{R_{1}}X\left(t\right)dt\right]}{\esp\left[R_{1}\right]},
\end{eqnarray*}
suponiendo que ambas esperanzas son finitas.

\begin{Note}
\begin{itemize}
\item[a)] Si el proceso regenerativo $X$ es positivo recurrente y tiene trayectorias muestrales no negativas, entonces la ecuaci\'on anterior es v\'alida.
\item[b)] Si $X$ es positivo recurrente regenerativo, podemos construir una \'unica versi\'on estacionaria de este proceso, $X_{e}=\left\{X_{e}\left(t\right)\right\}$, donde $X_{e}$ es un proceso estoc\'astico regenerativo y estrictamente estacionario, con distribuci\'on marginal distribuida como $X_{\infty}$
\end{itemize}
\end{Note}

%________________________________________________________________________
%\subsection*{Procesos Regenerativos}
%________________________________________________________________________



\begin{Note}
Si $\tilde{X}\left(t\right)$ con espacio de estados $\tilde{S}$ es regenerativo sobre $T_{n}$, entonces $X\left(t\right)=f\left(\tilde{X}\left(t\right)\right)$ tambi\'en es regenerativo sobre $T_{n}$, para cualquier funci\'on $f:\tilde{S}\rightarrow S$.
\end{Note}

\begin{Note}
Los procesos regenerativos son crudamente regenerativos, pero no al rev\'es.
\end{Note}
%\subsection*{Procesos Regenerativos: Sigman\cite{Sigman1}}
\begin{Def}[Definici\'on Cl\'asica]
Un proceso estoc\'astico $X=\left\{X\left(t\right):t\geq0\right\}$ es llamado regenerativo is existe una variable aleatoria $R_{1}>0$ tal que
\begin{itemize}
\item[i)] $\left\{X\left(t+R_{1}\right):t\geq0\right\}$ es independiente de $\left\{\left\{X\left(t\right):t<R_{1}\right\},\right\}$
\item[ii)] $\left\{X\left(t+R_{1}\right):t\geq0\right\}$ es estoc\'asticamente equivalente a $\left\{X\left(t\right):t>0\right\}$
\end{itemize}

Llamamos a $R_{1}$ tiempo de regeneraci\'on, y decimos que $X$ se regenera en este punto.
\end{Def}

$\left\{X\left(t+R_{1}\right)\right\}$ es regenerativo con tiempo de regeneraci\'on $R_{2}$, independiente de $R_{1}$ pero con la misma distribuci\'on que $R_{1}$. Procediendo de esta manera se obtiene una secuencia de variables aleatorias independientes e id\'enticamente distribuidas $\left\{R_{n}\right\}$ llamados longitudes de ciclo. Si definimos a $Z_{k}\equiv R_{1}+R_{2}+\cdots+R_{k}$, se tiene un proceso de renovaci\'on llamado proceso de renovaci\'on encajado para $X$.




\begin{Def}
Para $x$ fijo y para cada $t\geq0$, sea $I_{x}\left(t\right)=1$ si $X\left(t\right)\leq x$,  $I_{x}\left(t\right)=0$ en caso contrario, y def\'inanse los tiempos promedio
\begin{eqnarray*}
\overline{X}&=&lim_{t\rightarrow\infty}\frac{1}{t}\int_{0}^{\infty}X\left(u\right)du\\
\prob\left(X_{\infty}\leq x\right)&=&lim_{t\rightarrow\infty}\frac{1}{t}\int_{0}^{\infty}I_{x}\left(u\right)du,
\end{eqnarray*}
cuando estos l\'imites existan.
\end{Def}

Como consecuencia del teorema de Renovaci\'on-Recompensa, se tiene que el primer l\'imite  existe y es igual a la constante
\begin{eqnarray*}
\overline{X}&=&\frac{\esp\left[\int_{0}^{R_{1}}X\left(t\right)dt\right]}{\esp\left[R_{1}\right]},
\end{eqnarray*}
suponiendo que ambas esperanzas son finitas.

\begin{Note}
\begin{itemize}
\item[a)] Si el proceso regenerativo $X$ es positivo recurrente y tiene trayectorias muestrales no negativas, entonces la ecuaci\'on anterior es v\'alida.
\item[b)] Si $X$ es positivo recurrente regenerativo, podemos construir una \'unica versi\'on estacionaria de este proceso, $X_{e}=\left\{X_{e}\left(t\right)\right\}$, donde $X_{e}$ es un proceso estoc\'astico regenerativo y estrictamente estacionario, con distribuci\'on marginal distribuida como $X_{\infty}$
\end{itemize}
\end{Note}

%________________________________________________________________________
%\subsection*{Procesos Regenerativos}
%________________________________________________________________________

Para $\left\{X\left(t\right):t\geq0\right\}$ Proceso Estoc\'astico a tiempo continuo con estado de espacios $S$, que es un espacio m\'etrico, con trayectorias continuas por la derecha y con l\'imites por la izquierda c.s. Sea $N\left(t\right)$ un proceso de renovaci\'on en $\rea_{+}$ definido en el mismo espacio de probabilidad que $X\left(t\right)$, con tiempos de renovaci\'on $T$ y tiempos de inter-renovaci\'on $\xi_{n}=T_{n}-T_{n-1}$, con misma distribuci\'on $F$ de media finita $\mu$.



\begin{Def}
Para el proceso $\left\{\left(N\left(t\right),X\left(t\right)\right):t\geq0\right\}$, sus trayectoria muestrales en el intervalo de tiempo $\left[T_{n-1},T_{n}\right)$ est\'an descritas por
\begin{eqnarray*}
\zeta_{n}=\left(\xi_{n},\left\{X\left(T_{n-1}+t\right):0\leq t<\xi_{n}\right\}\right)
\end{eqnarray*}
Este $\zeta_{n}$ es el $n$-\'esimo segmento del proceso. El proceso es regenerativo sobre los tiempos $T_{n}$ si sus segmentos $\zeta_{n}$ son independientes e id\'enticamennte distribuidos.
\end{Def}


\begin{Note}
Si $\tilde{X}\left(t\right)$ con espacio de estados $\tilde{S}$ es regenerativo sobre $T_{n}$, entonces $X\left(t\right)=f\left(\tilde{X}\left(t\right)\right)$ tambi\'en es regenerativo sobre $T_{n}$, para cualquier funci\'on $f:\tilde{S}\rightarrow S$.
\end{Note}

\begin{Note}
Los procesos regenerativos son crudamente regenerativos, pero no al rev\'es.
\end{Note}

\begin{Def}[Definici\'on Cl\'asica]
Un proceso estoc\'astico $X=\left\{X\left(t\right):t\geq0\right\}$ es llamado regenerativo is existe una variable aleatoria $R_{1}>0$ tal que
\begin{itemize}
\item[i)] $\left\{X\left(t+R_{1}\right):t\geq0\right\}$ es independiente de $\left\{\left\{X\left(t\right):t<R_{1}\right\},\right\}$
\item[ii)] $\left\{X\left(t+R_{1}\right):t\geq0\right\}$ es estoc\'asticamente equivalente a $\left\{X\left(t\right):t>0\right\}$
\end{itemize}

Llamamos a $R_{1}$ tiempo de regeneraci\'on, y decimos que $X$ se regenera en este punto.
\end{Def}

$\left\{X\left(t+R_{1}\right)\right\}$ es regenerativo con tiempo de regeneraci\'on $R_{2}$, independiente de $R_{1}$ pero con la misma distribuci\'on que $R_{1}$. Procediendo de esta manera se obtiene una secuencia de variables aleatorias independientes e id\'enticamente distribuidas $\left\{R_{n}\right\}$ llamados longitudes de ciclo. Si definimos a $Z_{k}\equiv R_{1}+R_{2}+\cdots+R_{k}$, se tiene un proceso de renovaci\'on llamado proceso de renovaci\'on encajado para $X$.

\begin{Note}
Un proceso regenerativo con media de la longitud de ciclo finita es llamado positivo recurrente.
\end{Note}


\begin{Def}
Para $x$ fijo y para cada $t\geq0$, sea $I_{x}\left(t\right)=1$ si $X\left(t\right)\leq x$,  $I_{x}\left(t\right)=0$ en caso contrario, y def\'inanse los tiempos promedio
\begin{eqnarray*}
\overline{X}&=&lim_{t\rightarrow\infty}\frac{1}{t}\int_{0}^{\infty}X\left(u\right)du\\
\prob\left(X_{\infty}\leq x\right)&=&lim_{t\rightarrow\infty}\frac{1}{t}\int_{0}^{\infty}I_{x}\left(u\right)du,
\end{eqnarray*}
cuando estos l\'imites existan.
\end{Def}

Como consecuencia del teorema de Renovaci\'on-Recompensa, se tiene que el primer l\'imite  existe y es igual a la constante
\begin{eqnarray*}
\overline{X}&=&\frac{\esp\left[\int_{0}^{R_{1}}X\left(t\right)dt\right]}{\esp\left[R_{1}\right]},
\end{eqnarray*}
suponiendo que ambas esperanzas son finitas.

\begin{Note}
\begin{itemize}
\item[a)] Si el proceso regenerativo $X$ es positivo recurrente y tiene trayectorias muestrales no negativas, entonces la ecuaci\'on anterior es v\'alida.
\item[b)] Si $X$ es positivo recurrente regenerativo, podemos construir una \'unica versi\'on estacionaria de este proceso, $X_{e}=\left\{X_{e}\left(t\right)\right\}$, donde $X_{e}$ es un proceso estoc\'astico regenerativo y estrictamente estacionario, con distribuci\'on marginal distribuida como $X_{\infty}$
\end{itemize}
\end{Note}

%________________________________________________________________________
\section{Procesos Regenerativos Sigman, Thorisson y Wolff \cite{Sigman1}}
%________________________________________________________________________


\begin{Def}[Definici\'on Cl\'asica]
Un proceso estoc\'astico $X=\left\{X\left(t\right):t\geq0\right\}$ es llamado regenerativo is existe una variable aleatoria $R_{1}>0$ tal que
\begin{itemize}
\item[i)] $\left\{X\left(t+R_{1}\right):t\geq0\right\}$ es independiente de $\left\{\left\{X\left(t\right):t<R_{1}\right\},\right\}$
\item[ii)] $\left\{X\left(t+R_{1}\right):t\geq0\right\}$ es estoc\'asticamente equivalente a $\left\{X\left(t\right):t>0\right\}$
\end{itemize}

Llamamos a $R_{1}$ tiempo de regeneraci\'on, y decimos que $X$ se regenera en este punto.
\end{Def}

$\left\{X\left(t+R_{1}\right)\right\}$ es regenerativo con tiempo de regeneraci\'on $R_{2}$, independiente de $R_{1}$ pero con la misma distribuci\'on que $R_{1}$. Procediendo de esta manera se obtiene una secuencia de variables aleatorias independientes e id\'enticamente distribuidas $\left\{R_{n}\right\}$ llamados longitudes de ciclo. Si definimos a $Z_{k}\equiv R_{1}+R_{2}+\cdots+R_{k}$, se tiene un proceso de renovaci\'on llamado proceso de renovaci\'on encajado para $X$.


\begin{Note}
La existencia de un primer tiempo de regeneraci\'on, $R_{1}$, implica la existencia de una sucesi\'on completa de estos tiempos $R_{1},R_{2}\ldots,$ que satisfacen la propiedad deseada \cite{Sigman2}.
\end{Note}


\begin{Note} Para la cola $GI/GI/1$ los usuarios arriban con tiempos $t_{n}$ y son atendidos con tiempos de servicio $S_{n}$, los tiempos de arribo forman un proceso de renovaci\'on  con tiempos entre arribos independientes e identicamente distribuidos (\texttt{i.i.d.})$T_{n}=t_{n}-t_{n-1}$, adem\'as los tiempos de servicio son \texttt{i.i.d.} e independientes de los procesos de arribo. Por \textit{estable} se entiende que $\esp S_{n}<\esp T_{n}<\infty$.
\end{Note}
 


\begin{Def}
Para $x$ fijo y para cada $t\geq0$, sea $I_{x}\left(t\right)=1$ si $X\left(t\right)\leq x$,  $I_{x}\left(t\right)=0$ en caso contrario, y def\'inanse los tiempos promedio
\begin{eqnarray*}
\overline{X}&=&lim_{t\rightarrow\infty}\frac{1}{t}\int_{0}^{\infty}X\left(u\right)du\\
\prob\left(X_{\infty}\leq x\right)&=&lim_{t\rightarrow\infty}\frac{1}{t}\int_{0}^{\infty}I_{x}\left(u\right)du,
\end{eqnarray*}
cuando estos l\'imites existan.
\end{Def}

Como consecuencia del teorema de Renovaci\'on-Recompensa, se tiene que el primer l\'imite  existe y es igual a la constante
\begin{eqnarray*}
\overline{X}&=&\frac{\esp\left[\int_{0}^{R_{1}}X\left(t\right)dt\right]}{\esp\left[R_{1}\right]},
\end{eqnarray*}
suponiendo que ambas esperanzas son finitas.
 
\begin{Note}
Funciones de procesos regenerativos son regenerativas, es decir, si $X\left(t\right)$ es regenerativo y se define el proceso $Y\left(t\right)$ por $Y\left(t\right)=f\left(X\left(t\right)\right)$ para alguna funci\'on Borel medible $f\left(\cdot\right)$. Adem\'as $Y$ es regenerativo con los mismos tiempos de renovaci\'on que $X$. 

En general, los tiempos de renovaci\'on, $Z_{k}$ de un proceso regenerativo no requieren ser tiempos de paro con respecto a la evoluci\'on de $X\left(t\right)$.
\end{Note} 

\begin{Note}
Una funci\'on de un proceso de Markov, usualmente no ser\'a un proceso de Markov, sin embargo ser\'a regenerativo si el proceso de Markov lo es.
\end{Note}

 
\begin{Note}
Un proceso regenerativo con media de la longitud de ciclo finita es llamado positivo recurrente.
\end{Note}


\begin{Note}
\begin{itemize}
\item[a)] Si el proceso regenerativo $X$ es positivo recurrente y tiene trayectorias muestrales no negativas, entonces la ecuaci\'on anterior es v\'alida.
\item[b)] Si $X$ es positivo recurrente regenerativo, podemos construir una \'unica versi\'on estacionaria de este proceso, $X_{e}=\left\{X_{e}\left(t\right)\right\}$, donde $X_{e}$ es un proceso estoc\'astico regenerativo y estrictamente estacionario, con distribuci\'on marginal distribuida como $X_{\infty}$
\end{itemize}
\end{Note}


%________________________________________________________________________
%\subsection*{Procesos Regenerativos Sigman, Thorisson y Wolff \cite{Sigman1}}
%________________________________________________________________________


\begin{Def}[Definici\'on Cl\'asica]
Un proceso estoc\'astico $X=\left\{X\left(t\right):t\geq0\right\}$ es llamado regenerativo is existe una variable aleatoria $R_{1}>0$ tal que
\begin{itemize}
\item[i)] $\left\{X\left(t+R_{1}\right):t\geq0\right\}$ es independiente de $\left\{\left\{X\left(t\right):t<R_{1}\right\},\right\}$
\item[ii)] $\left\{X\left(t+R_{1}\right):t\geq0\right\}$ es estoc\'asticamente equivalente a $\left\{X\left(t\right):t>0\right\}$
\end{itemize}

Llamamos a $R_{1}$ tiempo de regeneraci\'on, y decimos que $X$ se regenera en este punto.
\end{Def}

$\left\{X\left(t+R_{1}\right)\right\}$ es regenerativo con tiempo de regeneraci\'on $R_{2}$, independiente de $R_{1}$ pero con la misma distribuci\'on que $R_{1}$. Procediendo de esta manera se obtiene una secuencia de variables aleatorias independientes e id\'enticamente distribuidas $\left\{R_{n}\right\}$ llamados longitudes de ciclo. Si definimos a $Z_{k}\equiv R_{1}+R_{2}+\cdots+R_{k}$, se tiene un proceso de renovaci\'on llamado proceso de renovaci\'on encajado para $X$.


\begin{Note}
La existencia de un primer tiempo de regeneraci\'on, $R_{1}$, implica la existencia de una sucesi\'on completa de estos tiempos $R_{1},R_{2}\ldots,$ que satisfacen la propiedad deseada \cite{Sigman2}.
\end{Note}


\begin{Note} Para la cola $GI/GI/1$ los usuarios arriban con tiempos $t_{n}$ y son atendidos con tiempos de servicio $S_{n}$, los tiempos de arribo forman un proceso de renovaci\'on  con tiempos entre arribos independientes e identicamente distribuidos (\texttt{i.i.d.})$T_{n}=t_{n}-t_{n-1}$, adem\'as los tiempos de servicio son \texttt{i.i.d.} e independientes de los procesos de arribo. Por \textit{estable} se entiende que $\esp S_{n}<\esp T_{n}<\infty$.
\end{Note}
 


\begin{Def}
Para $x$ fijo y para cada $t\geq0$, sea $I_{x}\left(t\right)=1$ si $X\left(t\right)\leq x$,  $I_{x}\left(t\right)=0$ en caso contrario, y def\'inanse los tiempos promedio
\begin{eqnarray*}
\overline{X}&=&lim_{t\rightarrow\infty}\frac{1}{t}\int_{0}^{\infty}X\left(u\right)du\\
\prob\left(X_{\infty}\leq x\right)&=&lim_{t\rightarrow\infty}\frac{1}{t}\int_{0}^{\infty}I_{x}\left(u\right)du,
\end{eqnarray*}
cuando estos l\'imites existan.
\end{Def}

Como consecuencia del teorema de Renovaci\'on-Recompensa, se tiene que el primer l\'imite  existe y es igual a la constante
\begin{eqnarray*}
\overline{X}&=&\frac{\esp\left[\int_{0}^{R_{1}}X\left(t\right)dt\right]}{\esp\left[R_{1}\right]},
\end{eqnarray*}
suponiendo que ambas esperanzas son finitas.
 
\begin{Note}
Funciones de procesos regenerativos son regenerativas, es decir, si $X\left(t\right)$ es regenerativo y se define el proceso $Y\left(t\right)$ por $Y\left(t\right)=f\left(X\left(t\right)\right)$ para alguna funci\'on Borel medible $f\left(\cdot\right)$. Adem\'as $Y$ es regenerativo con los mismos tiempos de renovaci\'on que $X$. 

En general, los tiempos de renovaci\'on, $Z_{k}$ de un proceso regenerativo no requieren ser tiempos de paro con respecto a la evoluci\'on de $X\left(t\right)$.
\end{Note} 

\begin{Note}
Una funci\'on de un proceso de Markov, usualmente no ser\'a un proceso de Markov, sin embargo ser\'a regenerativo si el proceso de Markov lo es.
\end{Note}

 
\begin{Note}
Un proceso regenerativo con media de la longitud de ciclo finita es llamado positivo recurrente.
\end{Note}


\begin{Note}
\begin{itemize}
\item[a)] Si el proceso regenerativo $X$ es positivo recurrente y tiene trayectorias muestrales no negativas, entonces la ecuaci\'on anterior es v\'alida.
\item[b)] Si $X$ es positivo recurrente regenerativo, podemos construir una \'unica versi\'on estacionaria de este proceso, $X_{e}=\left\{X_{e}\left(t\right)\right\}$, donde $X_{e}$ es un proceso estoc\'astico regenerativo y estrictamente estacionario, con distribuci\'on marginal distribuida como $X_{\infty}$
\end{itemize}
\end{Note}


%________________________________________________________________________
%\subsection*{Procesos Regenerativos Sigman, Thorisson y Wolff \cite{Sigman1}}
%________________________________________________________________________


\begin{Def}[Definici\'on Cl\'asica]
Un proceso estoc\'astico $X=\left\{X\left(t\right):t\geq0\right\}$ es llamado regenerativo is existe una variable aleatoria $R_{1}>0$ tal que
\begin{itemize}
\item[i)] $\left\{X\left(t+R_{1}\right):t\geq0\right\}$ es independiente de $\left\{\left\{X\left(t\right):t<R_{1}\right\},\right\}$
\item[ii)] $\left\{X\left(t+R_{1}\right):t\geq0\right\}$ es estoc\'asticamente equivalente a $\left\{X\left(t\right):t>0\right\}$
\end{itemize}

Llamamos a $R_{1}$ tiempo de regeneraci\'on, y decimos que $X$ se regenera en este punto.
\end{Def}

$\left\{X\left(t+R_{1}\right)\right\}$ es regenerativo con tiempo de regeneraci\'on $R_{2}$, independiente de $R_{1}$ pero con la misma distribuci\'on que $R_{1}$. Procediendo de esta manera se obtiene una secuencia de variables aleatorias independientes e id\'enticamente distribuidas $\left\{R_{n}\right\}$ llamados longitudes de ciclo. Si definimos a $Z_{k}\equiv R_{1}+R_{2}+\cdots+R_{k}$, se tiene un proceso de renovaci\'on llamado proceso de renovaci\'on encajado para $X$.


\begin{Note}
La existencia de un primer tiempo de regeneraci\'on, $R_{1}$, implica la existencia de una sucesi\'on completa de estos tiempos $R_{1},R_{2}\ldots,$ que satisfacen la propiedad deseada \cite{Sigman2}.
\end{Note}


\begin{Note} Para la cola $GI/GI/1$ los usuarios arriban con tiempos $t_{n}$ y son atendidos con tiempos de servicio $S_{n}$, los tiempos de arribo forman un proceso de renovaci\'on  con tiempos entre arribos independientes e identicamente distribuidos (\texttt{i.i.d.})$T_{n}=t_{n}-t_{n-1}$, adem\'as los tiempos de servicio son \texttt{i.i.d.} e independientes de los procesos de arribo. Por \textit{estable} se entiende que $\esp S_{n}<\esp T_{n}<\infty$.
\end{Note}
 


\begin{Def}
Para $x$ fijo y para cada $t\geq0$, sea $I_{x}\left(t\right)=1$ si $X\left(t\right)\leq x$,  $I_{x}\left(t\right)=0$ en caso contrario, y def\'inanse los tiempos promedio
\begin{eqnarray*}
\overline{X}&=&lim_{t\rightarrow\infty}\frac{1}{t}\int_{0}^{\infty}X\left(u\right)du\\
\prob\left(X_{\infty}\leq x\right)&=&lim_{t\rightarrow\infty}\frac{1}{t}\int_{0}^{\infty}I_{x}\left(u\right)du,
\end{eqnarray*}
cuando estos l\'imites existan.
\end{Def}

Como consecuencia del teorema de Renovaci\'on-Recompensa, se tiene que el primer l\'imite  existe y es igual a la constante
\begin{eqnarray*}
\overline{X}&=&\frac{\esp\left[\int_{0}^{R_{1}}X\left(t\right)dt\right]}{\esp\left[R_{1}\right]},
\end{eqnarray*}
suponiendo que ambas esperanzas son finitas.
 
\begin{Note}
Funciones de procesos regenerativos son regenerativas, es decir, si $X\left(t\right)$ es regenerativo y se define el proceso $Y\left(t\right)$ por $Y\left(t\right)=f\left(X\left(t\right)\right)$ para alguna funci\'on Borel medible $f\left(\cdot\right)$. Adem\'as $Y$ es regenerativo con los mismos tiempos de renovaci\'on que $X$. 

En general, los tiempos de renovaci\'on, $Z_{k}$ de un proceso regenerativo no requieren ser tiempos de paro con respecto a la evoluci\'on de $X\left(t\right)$.
\end{Note} 

\begin{Note}
Una funci\'on de un proceso de Markov, usualmente no ser\'a un proceso de Markov, sin embargo ser\'a regenerativo si el proceso de Markov lo es.
\end{Note}

 
\begin{Note}
Un proceso regenerativo con media de la longitud de ciclo finita es llamado positivo recurrente.
\end{Note}


\begin{Note}
\begin{itemize}
\item[a)] Si el proceso regenerativo $X$ es positivo recurrente y tiene trayectorias muestrales no negativas, entonces la ecuaci\'on anterior es v\'alida.
\item[b)] Si $X$ es positivo recurrente regenerativo, podemos construir una \'unica versi\'on estacionaria de este proceso, $X_{e}=\left\{X_{e}\left(t\right)\right\}$, donde $X_{e}$ es un proceso estoc\'astico regenerativo y estrictamente estacionario, con distribuci\'on marginal distribuida como $X_{\infty}$
\end{itemize}
\end{Note}


%__________________________________________________________________________________________
\section{Procesos Regenerativos Estacionarios - Stidham \cite{Stidham}}
%__________________________________________________________________________________________


Un proceso estoc\'astico a tiempo continuo $\left\{V\left(t\right),t\geq0\right\}$ es un proceso regenerativo si existe una sucesi\'on de variables aleatorias independientes e id\'enticamente distribuidas $\left\{X_{1},X_{2},\ldots\right\}$, sucesi\'on de renovaci\'on, tal que para cualquier conjunto de Borel $A$, 

\begin{eqnarray*}
\prob\left\{V\left(t\right)\in A|X_{1}+X_{2}+\cdots+X_{R\left(t\right)}=s,\left\{V\left(\tau\right),\tau<s\right\}\right\}=\prob\left\{V\left(t-s\right)\in A|X_{1}>t-s\right\},
\end{eqnarray*}
para todo $0\leq s\leq t$, donde $R\left(t\right)=\max\left\{X_{1}+X_{2}+\cdots+X_{j}\leq t\right\}=$n\'umero de renovaciones ({\emph{puntos de regeneraci\'on}}) que ocurren en $\left[0,t\right]$. El intervalo $\left[0,X_{1}\right)$ es llamado {\emph{primer ciclo de regeneraci\'on}} de $\left\{V\left(t \right),t\geq0\right\}$, $\left[X_{1},X_{1}+X_{2}\right)$ el {\emph{segundo ciclo de regeneraci\'on}}, y as\'i sucesivamente.

Sea $X=X_{1}$ y sea $F$ la funci\'on de distrbuci\'on de $X$


\begin{Def}
Se define el proceso estacionario, $\left\{V^{*}\left(t\right),t\geq0\right\}$, para $\left\{V\left(t\right),t\geq0\right\}$ por

\begin{eqnarray*}
\prob\left\{V\left(t\right)\in A\right\}=\frac{1}{\esp\left[X\right]}\int_{0}^{\infty}\prob\left\{V\left(t+x\right)\in A|X>x\right\}\left(1-F\left(x\right)\right)dx,
\end{eqnarray*} 
para todo $t\geq0$ y todo conjunto de Borel $A$.
\end{Def}

\begin{Def}
Una distribuci\'on se dice que es {\emph{aritm\'etica}} si todos sus puntos de incremento son m\'ultiplos de la forma $0,\lambda, 2\lambda,\ldots$ para alguna $\lambda>0$ entera.
\end{Def}


\begin{Def}
Una modificaci\'on medible de un proceso $\left\{V\left(t\right),t\geq0\right\}$, es una versi\'on de este, $\left\{V\left(t,w\right)\right\}$ conjuntamente medible para $t\geq0$ y para $w\in S$, $S$ espacio de estados para $\left\{V\left(t\right),t\geq0\right\}$.
\end{Def}

\begin{Teo}
Sea $\left\{V\left(t\right),t\geq\right\}$ un proceso regenerativo no negativo con modificaci\'on medible. Sea $\esp\left[X\right]<\infty$. Entonces el proceso estacionario dado por la ecuaci\'on anterior est\'a bien definido y tiene funci\'on de distribuci\'on independiente de $t$, adem\'as
\begin{itemize}
\item[i)] \begin{eqnarray*}
\esp\left[V^{*}\left(0\right)\right]&=&\frac{\esp\left[\int_{0}^{X}V\left(s\right)ds\right]}{\esp\left[X\right]}\end{eqnarray*}
\item[ii)] Si $\esp\left[V^{*}\left(0\right)\right]<\infty$, equivalentemente, si $\esp\left[\int_{0}^{X}V\left(s\right)ds\right]<\infty$,entonces
\begin{eqnarray*}
\frac{\int_{0}^{t}V\left(s\right)ds}{t}\rightarrow\frac{\esp\left[\int_{0}^{X}V\left(s\right)ds\right]}{\esp\left[X\right]}
\end{eqnarray*}
con probabilidad 1 y en media, cuando $t\rightarrow\infty$.
\end{itemize}
\end{Teo}

\begin{Coro}
Sea $\left\{V\left(t\right),t\geq0\right\}$ un proceso regenerativo no negativo, con modificaci\'on medible. Si $\esp <\infty$, $F$ es no-aritm\'etica, y para todo $x\geq0$, $P\left\{V\left(t\right)\leq x,C>x\right\}$ es de variaci\'on acotada como funci\'on de $t$ en cada intervalo finito $\left[0,\tau\right]$, entonces $V\left(t\right)$ converge en distribuci\'on  cuando $t\rightarrow\infty$ y $$\esp V=\frac{\esp \int_{0}^{X}V\left(s\right)ds}{\esp X}$$
Donde $V$ tiene la distribuci\'on l\'imite de $V\left(t\right)$ cuando $t\rightarrow\infty$.

\end{Coro}

Para el caso discreto se tienen resultados similares.


%__________________________________________________________________________________________
%\subsection*{Procesos Regenerativos Estacionarios - Stidham \cite{Stidham}}
%__________________________________________________________________________________________


Un proceso estoc\'astico a tiempo continuo $\left\{V\left(t\right),t\geq0\right\}$ es un proceso regenerativo si existe una sucesi\'on de variables aleatorias independientes e id\'enticamente distribuidas $\left\{X_{1},X_{2},\ldots\right\}$, sucesi\'on de renovaci\'on, tal que para cualquier conjunto de Borel $A$, 

\begin{eqnarray*}
\prob\left\{V\left(t\right)\in A|X_{1}+X_{2}+\cdots+X_{R\left(t\right)}=s,\left\{V\left(\tau\right),\tau<s\right\}\right\}=\prob\left\{V\left(t-s\right)\in A|X_{1}>t-s\right\},
\end{eqnarray*}
para todo $0\leq s\leq t$, donde $R\left(t\right)=\max\left\{X_{1}+X_{2}+\cdots+X_{j}\leq t\right\}=$n\'umero de renovaciones ({\emph{puntos de regeneraci\'on}}) que ocurren en $\left[0,t\right]$. El intervalo $\left[0,X_{1}\right)$ es llamado {\emph{primer ciclo de regeneraci\'on}} de $\left\{V\left(t \right),t\geq0\right\}$, $\left[X_{1},X_{1}+X_{2}\right)$ el {\emph{segundo ciclo de regeneraci\'on}}, y as\'i sucesivamente.

Sea $X=X_{1}$ y sea $F$ la funci\'on de distrbuci\'on de $X$


\begin{Def}
Se define el proceso estacionario, $\left\{V^{*}\left(t\right),t\geq0\right\}$, para $\left\{V\left(t\right),t\geq0\right\}$ por

\begin{eqnarray*}
\prob\left\{V\left(t\right)\in A\right\}=\frac{1}{\esp\left[X\right]}\int_{0}^{\infty}\prob\left\{V\left(t+x\right)\in A|X>x\right\}\left(1-F\left(x\right)\right)dx,
\end{eqnarray*} 
para todo $t\geq0$ y todo conjunto de Borel $A$.
\end{Def}

\begin{Def}
Una distribuci\'on se dice que es {\emph{aritm\'etica}} si todos sus puntos de incremento son m\'ultiplos de la forma $0,\lambda, 2\lambda,\ldots$ para alguna $\lambda>0$ entera.
\end{Def}


\begin{Def}
Una modificaci\'on medible de un proceso $\left\{V\left(t\right),t\geq0\right\}$, es una versi\'on de este, $\left\{V\left(t,w\right)\right\}$ conjuntamente medible para $t\geq0$ y para $w\in S$, $S$ espacio de estados para $\left\{V\left(t\right),t\geq0\right\}$.
\end{Def}

\begin{Teo}
Sea $\left\{V\left(t\right),t\geq\right\}$ un proceso regenerativo no negativo con modificaci\'on medible. Sea $\esp\left[X\right]<\infty$. Entonces el proceso estacionario dado por la ecuaci\'on anterior est\'a bien definido y tiene funci\'on de distribuci\'on independiente de $t$, adem\'as
\begin{itemize}
\item[i)] \begin{eqnarray*}
\esp\left[V^{*}\left(0\right)\right]&=&\frac{\esp\left[\int_{0}^{X}V\left(s\right)ds\right]}{\esp\left[X\right]}\end{eqnarray*}
\item[ii)] Si $\esp\left[V^{*}\left(0\right)\right]<\infty$, equivalentemente, si $\esp\left[\int_{0}^{X}V\left(s\right)ds\right]<\infty$,entonces
\begin{eqnarray*}
\frac{\int_{0}^{t}V\left(s\right)ds}{t}\rightarrow\frac{\esp\left[\int_{0}^{X}V\left(s\right)ds\right]}{\esp\left[X\right]}
\end{eqnarray*}
con probabilidad 1 y en media, cuando $t\rightarrow\infty$.
\end{itemize}
\end{Teo}


%__________________________________________________________________________________________
%\subsection*{Procesos Regenerativos Estacionarios - Stidham \cite{Stidham}}
%__________________________________________________________________________________________


Un proceso estoc\'astico a tiempo continuo $\left\{V\left(t\right),t\geq0\right\}$ es un proceso regenerativo si existe una sucesi\'on de variables aleatorias independientes e id\'enticamente distribuidas $\left\{X_{1},X_{2},\ldots\right\}$, sucesi\'on de renovaci\'on, tal que para cualquier conjunto de Borel $A$, 

\begin{eqnarray*}
\prob\left\{V\left(t\right)\in A|X_{1}+X_{2}+\cdots+X_{R\left(t\right)}=s,\left\{V\left(\tau\right),\tau<s\right\}\right\}=\prob\left\{V\left(t-s\right)\in A|X_{1}>t-s\right\},
\end{eqnarray*}
para todo $0\leq s\leq t$, donde $R\left(t\right)=\max\left\{X_{1}+X_{2}+\cdots+X_{j}\leq t\right\}=$n\'umero de renovaciones ({\emph{puntos de regeneraci\'on}}) que ocurren en $\left[0,t\right]$. El intervalo $\left[0,X_{1}\right)$ es llamado {\emph{primer ciclo de regeneraci\'on}} de $\left\{V\left(t \right),t\geq0\right\}$, $\left[X_{1},X_{1}+X_{2}\right)$ el {\emph{segundo ciclo de regeneraci\'on}}, y as\'i sucesivamente.

Sea $X=X_{1}$ y sea $F$ la funci\'on de distrbuci\'on de $X$


\begin{Def}
Se define el proceso estacionario, $\left\{V^{*}\left(t\right),t\geq0\right\}$, para $\left\{V\left(t\right),t\geq0\right\}$ por

\begin{eqnarray*}
\prob\left\{V\left(t\right)\in A\right\}=\frac{1}{\esp\left[X\right]}\int_{0}^{\infty}\prob\left\{V\left(t+x\right)\in A|X>x\right\}\left(1-F\left(x\right)\right)dx,
\end{eqnarray*} 
para todo $t\geq0$ y todo conjunto de Borel $A$.
\end{Def}

\begin{Def}
Una distribuci\'on se dice que es {\emph{aritm\'etica}} si todos sus puntos de incremento son m\'ultiplos de la forma $0,\lambda, 2\lambda,\ldots$ para alguna $\lambda>0$ entera.
\end{Def}


\begin{Def}
Una modificaci\'on medible de un proceso $\left\{V\left(t\right),t\geq0\right\}$, es una versi\'on de este, $\left\{V\left(t,w\right)\right\}$ conjuntamente medible para $t\geq0$ y para $w\in S$, $S$ espacio de estados para $\left\{V\left(t\right),t\geq0\right\}$.
\end{Def}

\begin{Teo}
Sea $\left\{V\left(t\right),t\geq\right\}$ un proceso regenerativo no negativo con modificaci\'on medible. Sea $\esp\left[X\right]<\infty$. Entonces el proceso estacionario dado por la ecuaci\'on anterior est\'a bien definido y tiene funci\'on de distribuci\'on independiente de $t$, adem\'as
\begin{itemize}
\item[i)] \begin{eqnarray*}
\esp\left[V^{*}\left(0\right)\right]&=&\frac{\esp\left[\int_{0}^{X}V\left(s\right)ds\right]}{\esp\left[X\right]}\end{eqnarray*}
\item[ii)] Si $\esp\left[V^{*}\left(0\right)\right]<\infty$, equivalentemente, si $\esp\left[\int_{0}^{X}V\left(s\right)ds\right]<\infty$,entonces
\begin{eqnarray*}
\frac{\int_{0}^{t}V\left(s\right)ds}{t}\rightarrow\frac{\esp\left[\int_{0}^{X}V\left(s\right)ds\right]}{\esp\left[X\right]}
\end{eqnarray*}
con probabilidad 1 y en media, cuando $t\rightarrow\infty$.
\end{itemize}
\end{Teo}

Sea la funci\'on generadora de momentos para $L_{i}$, el n\'umero de usuarios en la cola $Q_{i}\left(z\right)$ en cualquier momento, est\'a dada por el tiempo promedio de $z^{L_{i}\left(t\right)}$ sobre el ciclo regenerativo definido anteriormente. Entonces 



Es decir, es posible determinar las longitudes de las colas a cualquier tiempo $t$. Entonces, determinando el primer momento es posible ver que


\begin{Def}
El tiempo de Ciclo $C_{i}$ es el periodo de tiempo que comienza cuando la cola $i$ es visitada por primera vez en un ciclo, y termina cuando es visitado nuevamente en el pr\'oximo ciclo. La duraci\'on del mismo est\'a dada por $\tau_{i}\left(m+1\right)-\tau_{i}\left(m\right)$, o equivalentemente $\overline{\tau}_{i}\left(m+1\right)-\overline{\tau}_{i}\left(m\right)$ bajo condiciones de estabilidad.
\end{Def}


\begin{Def}
El tiempo de intervisita $I_{i}$ es el periodo de tiempo que comienza cuando se ha completado el servicio en un ciclo y termina cuando es visitada nuevamente en el pr\'oximo ciclo. Su  duraci\'on del mismo est\'a dada por $\tau_{i}\left(m+1\right)-\overline{\tau}_{i}\left(m\right)$.
\end{Def}

La duraci\'on del tiempo de intervisita es $\tau_{i}\left(m+1\right)-\overline{\tau}\left(m\right)$. Dado que el n\'umero de usuarios presentes en $Q_{i}$ al tiempo $t=\tau_{i}\left(m+1\right)$ es igual al n\'umero de arribos durante el intervalo de tiempo $\left[\overline{\tau}\left(m\right),\tau_{i}\left(m+1\right)\right]$ se tiene que


\begin{eqnarray*}
\esp\left[z_{i}^{L_{i}\left(\tau_{i}\left(m+1\right)\right)}\right]=\esp\left[\left\{P_{i}\left(z_{i}\right)\right\}^{\tau_{i}\left(m+1\right)-\overline{\tau}\left(m\right)}\right]
\end{eqnarray*}

entonces, si $I_{i}\left(z\right)=\esp\left[z^{\tau_{i}\left(m+1\right)-\overline{\tau}\left(m\right)}\right]$
se tiene que $F_{i}\left(z\right)=I_{i}\left[P_{i}\left(z\right)\right]$
para $i=1,2$.

Conforme a la definici\'on dada al principio del cap\'itulo, definici\'on (\ref{Def.Tn}), sean $T_{1},T_{2},\ldots$ los puntos donde las longitudes de las colas de la red de sistemas de visitas c\'iclicas son cero simult\'aneamente, cuando la cola $Q_{j}$ es visitada por el servidor para dar servicio, es decir, $L_{1}\left(T_{i}\right)=0,L_{2}\left(T_{i}\right)=0,\hat{L}_{1}\left(T_{i}\right)=0$ y $\hat{L}_{2}\left(T_{i}\right)=0$, a estos puntos se les denominar\'a puntos regenerativos. Entonces, 

\begin{Def}
Al intervalo de tiempo entre dos puntos regenerativos se le llamar\'a ciclo regenerativo.
\end{Def}

\begin{Def}
Para $T_{i}$ se define, $M_{i}$, el n\'umero de ciclos de visita a la cola $Q_{l}$, durante el ciclo regenerativo, es decir, $M_{i}$ es un proceso de renovaci\'on.
\end{Def}

\begin{Def}
Para cada uno de los $M_{i}$'s, se definen a su vez la duraci\'on de cada uno de estos ciclos de visita en el ciclo regenerativo, $C_{i}^{(m)}$, para $m=1,2,\ldots,M_{i}$, que a su vez, tambi\'en es n proceso de renovaci\'on.
\end{Def}

\footnote{In Stidham and  Heyman \cite{Stidham} shows that is sufficient for the regenerative process to be stationary that the mean regenerative cycle time is finite: $\esp\left[\sum_{m=1}^{M_{i}}C_{i}^{(m)}\right]<\infty$, 


 como cada $C_{i}^{(m)}$ contiene intervalos de r\'eplica positivos, se tiene que $\esp\left[M_{i}\right]<\infty$, adem\'as, como $M_{i}>0$, se tiene que la condici\'on anterior es equivalente a tener que $\esp\left[C_{i}\right]<\infty$,
por lo tanto una condici\'on suficiente para la existencia del proceso regenerativo est\'a dada por $\sum_{k=1}^{N}\mu_{k}<1.$}

%__________________________________________________________________________________________
%\subsection*{Procesos Regenerativos Estacionarios - Stidham \cite{Stidham}}
%__________________________________________________________________________________________


Un proceso estoc\'astico a tiempo continuo $\left\{V\left(t\right),t\geq0\right\}$ es un proceso regenerativo si existe una sucesi\'on de variables aleatorias independientes e id\'enticamente distribuidas $\left\{X_{1},X_{2},\ldots\right\}$, sucesi\'on de renovaci\'on, tal que para cualquier conjunto de Borel $A$, 

\begin{eqnarray*}
\prob\left\{V\left(t\right)\in A|X_{1}+X_{2}+\cdots+X_{R\left(t\right)}=s,\left\{V\left(\tau\right),\tau<s\right\}\right\}=\prob\left\{V\left(t-s\right)\in A|X_{1}>t-s\right\},
\end{eqnarray*}
para todo $0\leq s\leq t$, donde $R\left(t\right)=\max\left\{X_{1}+X_{2}+\cdots+X_{j}\leq t\right\}=$n\'umero de renovaciones ({\emph{puntos de regeneraci\'on}}) que ocurren en $\left[0,t\right]$. El intervalo $\left[0,X_{1}\right)$ es llamado {\emph{primer ciclo de regeneraci\'on}} de $\left\{V\left(t \right),t\geq0\right\}$, $\left[X_{1},X_{1}+X_{2}\right)$ el {\emph{segundo ciclo de regeneraci\'on}}, y as\'i sucesivamente.

Sea $X=X_{1}$ y sea $F$ la funci\'on de distrbuci\'on de $X$


\begin{Def}
Se define el proceso estacionario, $\left\{V^{*}\left(t\right),t\geq0\right\}$, para $\left\{V\left(t\right),t\geq0\right\}$ por

\begin{eqnarray*}
\prob\left\{V\left(t\right)\in A\right\}=\frac{1}{\esp\left[X\right]}\int_{0}^{\infty}\prob\left\{V\left(t+x\right)\in A|X>x\right\}\left(1-F\left(x\right)\right)dx,
\end{eqnarray*} 
para todo $t\geq0$ y todo conjunto de Borel $A$.
\end{Def}

\begin{Def}
Una distribuci\'on se dice que es {\emph{aritm\'etica}} si todos sus puntos de incremento son m\'ultiplos de la forma $0,\lambda, 2\lambda,\ldots$ para alguna $\lambda>0$ entera.
\end{Def}


\begin{Def}
Una modificaci\'on medible de un proceso $\left\{V\left(t\right),t\geq0\right\}$, es una versi\'on de este, $\left\{V\left(t,w\right)\right\}$ conjuntamente medible para $t\geq0$ y para $w\in S$, $S$ espacio de estados para $\left\{V\left(t\right),t\geq0\right\}$.
\end{Def}

\begin{Teo}
Sea $\left\{V\left(t\right),t\geq\right\}$ un proceso regenerativo no negativo con modificaci\'on medible. Sea $\esp\left[X\right]<\infty$. Entonces el proceso estacionario dado por la ecuaci\'on anterior est\'a bien definido y tiene funci\'on de distribuci\'on independiente de $t$, adem\'as
\begin{itemize}
\item[i)] \begin{eqnarray*}
\esp\left[V^{*}\left(0\right)\right]&=&\frac{\esp\left[\int_{0}^{X}V\left(s\right)ds\right]}{\esp\left[X\right]}\end{eqnarray*}
\item[ii)] Si $\esp\left[V^{*}\left(0\right)\right]<\infty$, equivalentemente, si $\esp\left[\int_{0}^{X}V\left(s\right)ds\right]<\infty$,entonces
\begin{eqnarray*}
\frac{\int_{0}^{t}V\left(s\right)ds}{t}\rightarrow\frac{\esp\left[\int_{0}^{X}V\left(s\right)ds\right]}{\esp\left[X\right]}
\end{eqnarray*}
con probabilidad 1 y en media, cuando $t\rightarrow\infty$.
\end{itemize}
\end{Teo}

\begin{Coro}
Sea $\left\{V\left(t\right),t\geq0\right\}$ un proceso regenerativo no negativo, con modificaci\'on medible. Si $\esp <\infty$, $F$ es no-aritm\'etica, y para todo $x\geq0$, $P\left\{V\left(t\right)\leq x,C>x\right\}$ es de variaci\'on acotada como funci\'on de $t$ en cada intervalo finito $\left[0,\tau\right]$, entonces $V\left(t\right)$ converge en distribuci\'on  cuando $t\rightarrow\infty$ y $$\esp V=\frac{\esp \int_{0}^{X}V\left(s\right)ds}{\esp X}$$
Donde $V$ tiene la distribuci\'on l\'imite de $V\left(t\right)$ cuando $t\rightarrow\infty$.

\end{Coro}

Para el caso discreto se tienen resultados similares.



%__________________________________________________________________________________________
%\subsection*{Procesos Regenerativos Estacionarios - Stidham \cite{Stidham}}
%__________________________________________________________________________________________


Un proceso estoc\'astico a tiempo continuo $\left\{V\left(t\right),t\geq0\right\}$ es un proceso regenerativo si existe una sucesi\'on de variables aleatorias independientes e id\'enticamente distribuidas $\left\{X_{1},X_{2},\ldots\right\}$, sucesi\'on de renovaci\'on, tal que para cualquier conjunto de Borel $A$, 

\begin{eqnarray*}
\prob\left\{V\left(t\right)\in A|X_{1}+X_{2}+\cdots+X_{R\left(t\right)}=s,\left\{V\left(\tau\right),\tau<s\right\}\right\}=\prob\left\{V\left(t-s\right)\in A|X_{1}>t-s\right\},
\end{eqnarray*}
para todo $0\leq s\leq t$, donde $R\left(t\right)=\max\left\{X_{1}+X_{2}+\cdots+X_{j}\leq t\right\}=$n\'umero de renovaciones ({\emph{puntos de regeneraci\'on}}) que ocurren en $\left[0,t\right]$. El intervalo $\left[0,X_{1}\right)$ es llamado {\emph{primer ciclo de regeneraci\'on}} de $\left\{V\left(t \right),t\geq0\right\}$, $\left[X_{1},X_{1}+X_{2}\right)$ el {\emph{segundo ciclo de regeneraci\'on}}, y as\'i sucesivamente.

Sea $X=X_{1}$ y sea $F$ la funci\'on de distrbuci\'on de $X$


\begin{Def}
Se define el proceso estacionario, $\left\{V^{*}\left(t\right),t\geq0\right\}$, para $\left\{V\left(t\right),t\geq0\right\}$ por

\begin{eqnarray*}
\prob\left\{V\left(t\right)\in A\right\}=\frac{1}{\esp\left[X\right]}\int_{0}^{\infty}\prob\left\{V\left(t+x\right)\in A|X>x\right\}\left(1-F\left(x\right)\right)dx,
\end{eqnarray*} 
para todo $t\geq0$ y todo conjunto de Borel $A$.
\end{Def}

\begin{Def}
Una distribuci\'on se dice que es {\emph{aritm\'etica}} si todos sus puntos de incremento son m\'ultiplos de la forma $0,\lambda, 2\lambda,\ldots$ para alguna $\lambda>0$ entera.
\end{Def}


\begin{Def}
Una modificaci\'on medible de un proceso $\left\{V\left(t\right),t\geq0\right\}$, es una versi\'on de este, $\left\{V\left(t,w\right)\right\}$ conjuntamente medible para $t\geq0$ y para $w\in S$, $S$ espacio de estados para $\left\{V\left(t\right),t\geq0\right\}$.
\end{Def}

\begin{Teo}
Sea $\left\{V\left(t\right),t\geq\right\}$ un proceso regenerativo no negativo con modificaci\'on medible. Sea $\esp\left[X\right]<\infty$. Entonces el proceso estacionario dado por la ecuaci\'on anterior est\'a bien definido y tiene funci\'on de distribuci\'on independiente de $t$, adem\'as
\begin{itemize}
\item[i)] \begin{eqnarray*}
\esp\left[V^{*}\left(0\right)\right]&=&\frac{\esp\left[\int_{0}^{X}V\left(s\right)ds\right]}{\esp\left[X\right]}\end{eqnarray*}
\item[ii)] Si $\esp\left[V^{*}\left(0\right)\right]<\infty$, equivalentemente, si $\esp\left[\int_{0}^{X}V\left(s\right)ds\right]<\infty$,entonces
\begin{eqnarray*}
\frac{\int_{0}^{t}V\left(s\right)ds}{t}\rightarrow\frac{\esp\left[\int_{0}^{X}V\left(s\right)ds\right]}{\esp\left[X\right]}
\end{eqnarray*}
con probabilidad 1 y en media, cuando $t\rightarrow\infty$.
\end{itemize}
\end{Teo}

%__________________________________________________________________________________________
%\subsection*{Procesos Regenerativos Estacionarios - Stidham \cite{Stidham}}
%__________________________________________________________________________________________


Un proceso estoc\'astico a tiempo continuo $\left\{V\left(t\right),t\geq0\right\}$ es un proceso regenerativo si existe una sucesi\'on de variables aleatorias independientes e id\'enticamente distribuidas $\left\{X_{1},X_{2},\ldots\right\}$, sucesi\'on de renovaci\'on, tal que para cualquier conjunto de Borel $A$, 

\begin{eqnarray*}
\prob\left\{V\left(t\right)\in A|X_{1}+X_{2}+\cdots+X_{R\left(t\right)}=s,\left\{V\left(\tau\right),\tau<s\right\}\right\}=\prob\left\{V\left(t-s\right)\in A|X_{1}>t-s\right\},
\end{eqnarray*}
para todo $0\leq s\leq t$, donde $R\left(t\right)=\max\left\{X_{1}+X_{2}+\cdots+X_{j}\leq t\right\}=$n\'umero de renovaciones ({\emph{puntos de regeneraci\'on}}) que ocurren en $\left[0,t\right]$. El intervalo $\left[0,X_{1}\right)$ es llamado {\emph{primer ciclo de regeneraci\'on}} de $\left\{V\left(t \right),t\geq0\right\}$, $\left[X_{1},X_{1}+X_{2}\right)$ el {\emph{segundo ciclo de regeneraci\'on}}, y as\'i sucesivamente.

Sea $X=X_{1}$ y sea $F$ la funci\'on de distrbuci\'on de $X$


\begin{Def}
Se define el proceso estacionario, $\left\{V^{*}\left(t\right),t\geq0\right\}$, para $\left\{V\left(t\right),t\geq0\right\}$ por

\begin{eqnarray*}
\prob\left\{V\left(t\right)\in A\right\}=\frac{1}{\esp\left[X\right]}\int_{0}^{\infty}\prob\left\{V\left(t+x\right)\in A|X>x\right\}\left(1-F\left(x\right)\right)dx,
\end{eqnarray*} 
para todo $t\geq0$ y todo conjunto de Borel $A$.
\end{Def}

\begin{Def}
Una distribuci\'on se dice que es {\emph{aritm\'etica}} si todos sus puntos de incremento son m\'ultiplos de la forma $0,\lambda, 2\lambda,\ldots$ para alguna $\lambda>0$ entera.
\end{Def}


\begin{Def}
Una modificaci\'on medible de un proceso $\left\{V\left(t\right),t\geq0\right\}$, es una versi\'on de este, $\left\{V\left(t,w\right)\right\}$ conjuntamente medible para $t\geq0$ y para $w\in S$, $S$ espacio de estados para $\left\{V\left(t\right),t\geq0\right\}$.
\end{Def}

\begin{Teo}
Sea $\left\{V\left(t\right),t\geq\right\}$ un proceso regenerativo no negativo con modificaci\'on medible. Sea $\esp\left[X\right]<\infty$. Entonces el proceso estacionario dado por la ecuaci\'on anterior est\'a bien definido y tiene funci\'on de distribuci\'on independiente de $t$, adem\'as
\begin{itemize}
\item[i)] \begin{eqnarray*}
\esp\left[V^{*}\left(0\right)\right]&=&\frac{\esp\left[\int_{0}^{X}V\left(s\right)ds\right]}{\esp\left[X\right]}\end{eqnarray*}
\item[ii)] Si $\esp\left[V^{*}\left(0\right)\right]<\infty$, equivalentemente, si $\esp\left[\int_{0}^{X}V\left(s\right)ds\right]<\infty$,entonces
\begin{eqnarray*}
\frac{\int_{0}^{t}V\left(s\right)ds}{t}\rightarrow\frac{\esp\left[\int_{0}^{X}V\left(s\right)ds\right]}{\esp\left[X\right]}
\end{eqnarray*}
con probabilidad 1 y en media, cuando $t\rightarrow\infty$.
\end{itemize}
\end{Teo}

Para $\left\{X\left(t\right):t\geq0\right\}$ Proceso Estoc\'astico a tiempo continuo con estado de espacios $S$, que es un espacio m\'etrico, con trayectorias continuas por la derecha y con l\'imites por la izquierda c.s. Sea $N\left(t\right)$ un proceso de renovaci\'on en $\rea_{+}$ definido en el mismo espacio de probabilidad que $X\left(t\right)$, con tiempos de renovaci\'on $T$ y tiempos de inter-renovaci\'on $\xi_{n}=T_{n}-T_{n-1}$, con misma distribuci\'on $F$ de media finita $\mu$.

%__________________________________________________________________________________________
%\subsection*{Procesos Regenerativos Estacionarios - Stidham \cite{Stidham}}
%__________________________________________________________________________________________


Un proceso estoc\'astico a tiempo continuo $\left\{V\left(t\right),t\geq0\right\}$ es un proceso regenerativo si existe una sucesi\'on de variables aleatorias independientes e id\'enticamente distribuidas $\left\{X_{1},X_{2},\ldots\right\}$, sucesi\'on de renovaci\'on, tal que para cualquier conjunto de Borel $A$, 

\begin{eqnarray*}
\prob\left\{V\left(t\right)\in A|X_{1}+X_{2}+\cdots+X_{R\left(t\right)}=s,\left\{V\left(\tau\right),\tau<s\right\}\right\}=\prob\left\{V\left(t-s\right)\in A|X_{1}>t-s\right\},
\end{eqnarray*}
para todo $0\leq s\leq t$, donde $R\left(t\right)=\max\left\{X_{1}+X_{2}+\cdots+X_{j}\leq t\right\}=$n\'umero de renovaciones ({\emph{puntos de regeneraci\'on}}) que ocurren en $\left[0,t\right]$. El intervalo $\left[0,X_{1}\right)$ es llamado {\emph{primer ciclo de regeneraci\'on}} de $\left\{V\left(t \right),t\geq0\right\}$, $\left[X_{1},X_{1}+X_{2}\right)$ el {\emph{segundo ciclo de regeneraci\'on}}, y as\'i sucesivamente.

Sea $X=X_{1}$ y sea $F$ la funci\'on de distrbuci\'on de $X$


\begin{Def}
Se define el proceso estacionario, $\left\{V^{*}\left(t\right),t\geq0\right\}$, para $\left\{V\left(t\right),t\geq0\right\}$ por

\begin{eqnarray*}
\prob\left\{V\left(t\right)\in A\right\}=\frac{1}{\esp\left[X\right]}\int_{0}^{\infty}\prob\left\{V\left(t+x\right)\in A|X>x\right\}\left(1-F\left(x\right)\right)dx,
\end{eqnarray*} 
para todo $t\geq0$ y todo conjunto de Borel $A$.
\end{Def}

\begin{Def}
Una distribuci\'on se dice que es {\emph{aritm\'etica}} si todos sus puntos de incremento son m\'ultiplos de la forma $0,\lambda, 2\lambda,\ldots$ para alguna $\lambda>0$ entera.
\end{Def}


\begin{Def}
Una modificaci\'on medible de un proceso $\left\{V\left(t\right),t\geq0\right\}$, es una versi\'on de este, $\left\{V\left(t,w\right)\right\}$ conjuntamente medible para $t\geq0$ y para $w\in S$, $S$ espacio de estados para $\left\{V\left(t\right),t\geq0\right\}$.
\end{Def}

\begin{Teo}
Sea $\left\{V\left(t\right),t\geq\right\}$ un proceso regenerativo no negativo con modificaci\'on medible. Sea $\esp\left[X\right]<\infty$. Entonces el proceso estacionario dado por la ecuaci\'on anterior est\'a bien definido y tiene funci\'on de distribuci\'on independiente de $t$, adem\'as
\begin{itemize}
\item[i)] \begin{eqnarray*}
\esp\left[V^{*}\left(0\right)\right]&=&\frac{\esp\left[\int_{0}^{X}V\left(s\right)ds\right]}{\esp\left[X\right]}\end{eqnarray*}
\item[ii)] Si $\esp\left[V^{*}\left(0\right)\right]<\infty$, equivalentemente, si $\esp\left[\int_{0}^{X}V\left(s\right)ds\right]<\infty$,entonces
\begin{eqnarray*}
\frac{\int_{0}^{t}V\left(s\right)ds}{t}\rightarrow\frac{\esp\left[\int_{0}^{X}V\left(s\right)ds\right]}{\esp\left[X\right]}
\end{eqnarray*}
con probabilidad 1 y en media, cuando $t\rightarrow\infty$.
\end{itemize}
\end{Teo}

\begin{Coro}
Sea $\left\{V\left(t\right),t\geq0\right\}$ un proceso regenerativo no negativo, con modificaci\'on medible. Si $\esp <\infty$, $F$ es no-aritm\'etica, y para todo $x\geq0$, $P\left\{V\left(t\right)\leq x,C>x\right\}$ es de variaci\'on acotada como funci\'on de $t$ en cada intervalo finito $\left[0,\tau\right]$, entonces $V\left(t\right)$ converge en distribuci\'on  cuando $t\rightarrow\infty$ y $$\esp V=\frac{\esp \int_{0}^{X}V\left(s\right)ds}{\esp X}$$
Donde $V$ tiene la distribuci\'on l\'imite de $V\left(t\right)$ cuando $t\rightarrow\infty$.

\end{Coro}

Para el caso discreto se tienen resultados similares.


%___________________________________________________________________________________________
%
\section{Teorema Principal de Renovaci\'on}
%___________________________________________________________________________________________
%

\begin{Note} Una funci\'on $h:\rea_{+}\rightarrow\rea$ es Directamente Riemann Integrable en los siguientes casos:
\begin{itemize}
\item[a)] $h\left(t\right)\geq0$ es decreciente y Riemann Integrable.
\item[b)] $h$ es continua excepto posiblemente en un conjunto de Lebesgue de medida 0, y $|h\left(t\right)|\leq b\left(t\right)$, donde $b$ es DRI.
\end{itemize}
\end{Note}

\begin{Teo}[Teorema Principal de Renovaci\'on]
Si $F$ es no aritm\'etica y $h\left(t\right)$ es Directamente Riemann Integrable (DRI), entonces

\begin{eqnarray*}
lim_{t\rightarrow\infty}U\star h=\frac{1}{\mu}\int_{\rea_{+}}h\left(s\right)ds.
\end{eqnarray*}
\end{Teo}

\begin{Prop}
Cualquier funci\'on $H\left(t\right)$ acotada en intervalos finitos y que es 0 para $t<0$ puede expresarse como
\begin{eqnarray*}
H\left(t\right)=U\star h\left(t\right)\textrm{,  donde }h\left(t\right)=H\left(t\right)-F\star H\left(t\right)
\end{eqnarray*}
\end{Prop}

\begin{Def}
Un proceso estoc\'astico $X\left(t\right)$ es crudamente regenerativo en un tiempo aleatorio positivo $T$ si
\begin{eqnarray*}
\esp\left[X\left(T+t\right)|T\right]=\esp\left[X\left(t\right)\right]\textrm{, para }t\geq0,\end{eqnarray*}
y con las esperanzas anteriores finitas.
\end{Def}

\begin{Prop}
Sup\'ongase que $X\left(t\right)$ es un proceso crudamente regenerativo en $T$, que tiene distribuci\'on $F$. Si $\esp\left[X\left(t\right)\right]$ es acotado en intervalos finitos, entonces
\begin{eqnarray*}
\esp\left[X\left(t\right)\right]=U\star h\left(t\right)\textrm{,  donde }h\left(t\right)=\esp\left[X\left(t\right)\indora\left(T>t\right)\right].
\end{eqnarray*}
\end{Prop}

\begin{Teo}[Regeneraci\'on Cruda]
Sup\'ongase que $X\left(t\right)$ es un proceso con valores positivo crudamente regenerativo en $T$, y def\'inase $M=\sup\left\{|X\left(t\right)|:t\leq T\right\}$. Si $T$ es no aritm\'etico y $M$ y $MT$ tienen media finita, entonces
\begin{eqnarray*}
lim_{t\rightarrow\infty}\esp\left[X\left(t\right)\right]=\frac{1}{\mu}\int_{\rea_{+}}h\left(s\right)ds,
\end{eqnarray*}
donde $h\left(t\right)=\esp\left[X\left(t\right)\indora\left(T>t\right)\right]$.
\end{Teo}

%___________________________________________________________________________________________
%
%\subsection*{Teorema Principal de Renovaci\'on}
%___________________________________________________________________________________________
%

\begin{Note} Una funci\'on $h:\rea_{+}\rightarrow\rea$ es Directamente Riemann Integrable en los siguientes casos:
\begin{itemize}
\item[a)] $h\left(t\right)\geq0$ es decreciente y Riemann Integrable.
\item[b)] $h$ es continua excepto posiblemente en un conjunto de Lebesgue de medida 0, y $|h\left(t\right)|\leq b\left(t\right)$, donde $b$ es DRI.
\end{itemize}
\end{Note}

\begin{Teo}[Teorema Principal de Renovaci\'on]
Si $F$ es no aritm\'etica y $h\left(t\right)$ es Directamente Riemann Integrable (DRI), entonces

\begin{eqnarray*}
lim_{t\rightarrow\infty}U\star h=\frac{1}{\mu}\int_{\rea_{+}}h\left(s\right)ds.
\end{eqnarray*}
\end{Teo}

\begin{Prop}
Cualquier funci\'on $H\left(t\right)$ acotada en intervalos finitos y que es 0 para $t<0$ puede expresarse como
\begin{eqnarray*}
H\left(t\right)=U\star h\left(t\right)\textrm{,  donde }h\left(t\right)=H\left(t\right)-F\star H\left(t\right)
\end{eqnarray*}
\end{Prop}

\begin{Def}
Un proceso estoc\'astico $X\left(t\right)$ es crudamente regenerativo en un tiempo aleatorio positivo $T$ si
\begin{eqnarray*}
\esp\left[X\left(T+t\right)|T\right]=\esp\left[X\left(t\right)\right]\textrm{, para }t\geq0,\end{eqnarray*}
y con las esperanzas anteriores finitas.
\end{Def}

\begin{Prop}
Sup\'ongase que $X\left(t\right)$ es un proceso crudamente regenerativo en $T$, que tiene distribuci\'on $F$. Si $\esp\left[X\left(t\right)\right]$ es acotado en intervalos finitos, entonces
\begin{eqnarray*}
\esp\left[X\left(t\right)\right]=U\star h\left(t\right)\textrm{,  donde }h\left(t\right)=\esp\left[X\left(t\right)\indora\left(T>t\right)\right].
\end{eqnarray*}
\end{Prop}

\begin{Teo}[Regeneraci\'on Cruda]
Sup\'ongase que $X\left(t\right)$ es un proceso con valores positivo crudamente regenerativo en $T$, y def\'inase $M=\sup\left\{|X\left(t\right)|:t\leq T\right\}$. Si $T$ es no aritm\'etico y $M$ y $MT$ tienen media finita, entonces
\begin{eqnarray*}
lim_{t\rightarrow\infty}\esp\left[X\left(t\right)\right]=\frac{1}{\mu}\int_{\rea_{+}}h\left(s\right)ds,
\end{eqnarray*}
donde $h\left(t\right)=\esp\left[X\left(t\right)\indora\left(T>t\right)\right]$.
\end{Teo}


%___________________________________________________________________________________________
%
%\subsection*{Teorema Principal de Renovaci\'on}
%___________________________________________________________________________________________
%

\begin{Note} Una funci\'on $h:\rea_{+}\rightarrow\rea$ es Directamente Riemann Integrable en los siguientes casos:
\begin{itemize}
\item[a)] $h\left(t\right)\geq0$ es decreciente y Riemann Integrable.
\item[b)] $h$ es continua excepto posiblemente en un conjunto de Lebesgue de medida 0, y $|h\left(t\right)|\leq b\left(t\right)$, donde $b$ es DRI.
\end{itemize}
\end{Note}

\begin{Teo}[Teorema Principal de Renovaci\'on]
Si $F$ es no aritm\'etica y $h\left(t\right)$ es Directamente Riemann Integrable (DRI), entonces

\begin{eqnarray*}
lim_{t\rightarrow\infty}U\star h=\frac{1}{\mu}\int_{\rea_{+}}h\left(s\right)ds.
\end{eqnarray*}
\end{Teo}

\begin{Prop}
Cualquier funci\'on $H\left(t\right)$ acotada en intervalos finitos y que es 0 para $t<0$ puede expresarse como
\begin{eqnarray*}
H\left(t\right)=U\star h\left(t\right)\textrm{,  donde }h\left(t\right)=H\left(t\right)-F\star H\left(t\right)
\end{eqnarray*}
\end{Prop}

\begin{Def}
Un proceso estoc\'astico $X\left(t\right)$ es crudamente regenerativo en un tiempo aleatorio positivo $T$ si
\begin{eqnarray*}
\esp\left[X\left(T+t\right)|T\right]=\esp\left[X\left(t\right)\right]\textrm{, para }t\geq0,\end{eqnarray*}
y con las esperanzas anteriores finitas.
\end{Def}

\begin{Prop}
Sup\'ongase que $X\left(t\right)$ es un proceso crudamente regenerativo en $T$, que tiene distribuci\'on $F$. Si $\esp\left[X\left(t\right)\right]$ es acotado en intervalos finitos, entonces
\begin{eqnarray*}
\esp\left[X\left(t\right)\right]=U\star h\left(t\right)\textrm{,  donde }h\left(t\right)=\esp\left[X\left(t\right)\indora\left(T>t\right)\right].
\end{eqnarray*}
\end{Prop}

\begin{Teo}[Regeneraci\'on Cruda]
Sup\'ongase que $X\left(t\right)$ es un proceso con valores positivo crudamente regenerativo en $T$, y def\'inase $M=\sup\left\{|X\left(t\right)|:t\leq T\right\}$. Si $T$ es no aritm\'etico y $M$ y $MT$ tienen media finita, entonces
\begin{eqnarray*}
lim_{t\rightarrow\infty}\esp\left[X\left(t\right)\right]=\frac{1}{\mu}\int_{\rea_{+}}h\left(s\right)ds,
\end{eqnarray*}
donde $h\left(t\right)=\esp\left[X\left(t\right)\indora\left(T>t\right)\right]$.
\end{Teo}

%___________________________________________________________________________________________
%
\section{Propiedades de los Procesos de Renovaci\'on}
%___________________________________________________________________________________________
%

Los tiempos $T_{n}$ est\'an relacionados con los conteos de $N\left(t\right)$ por

\begin{eqnarray*}
\left\{N\left(t\right)\geq n\right\}&=&\left\{T_{n}\leq t\right\}\\
T_{N\left(t\right)}\leq &t&<T_{N\left(t\right)+1},
\end{eqnarray*}

adem\'as $N\left(T_{n}\right)=n$, y 

\begin{eqnarray*}
N\left(t\right)=\max\left\{n:T_{n}\leq t\right\}=\min\left\{n:T_{n+1}>t\right\}
\end{eqnarray*}

Por propiedades de la convoluci\'on se sabe que

\begin{eqnarray*}
P\left\{T_{n}\leq t\right\}=F^{n\star}\left(t\right)
\end{eqnarray*}
que es la $n$-\'esima convoluci\'on de $F$. Entonces 

\begin{eqnarray*}
\left\{N\left(t\right)\geq n\right\}&=&\left\{T_{n}\leq t\right\}\\
P\left\{N\left(t\right)\leq n\right\}&=&1-F^{\left(n+1\right)\star}\left(t\right)
\end{eqnarray*}

Adem\'as usando el hecho de que $\esp\left[N\left(t\right)\right]=\sum_{n=1}^{\infty}P\left\{N\left(t\right)\geq n\right\}$
se tiene que

\begin{eqnarray*}
\esp\left[N\left(t\right)\right]=\sum_{n=1}^{\infty}F^{n\star}\left(t\right)
\end{eqnarray*}

\begin{Prop}
Para cada $t\geq0$, la funci\'on generadora de momentos $\esp\left[e^{\alpha N\left(t\right)}\right]$ existe para alguna $\alpha$ en una vecindad del 0, y de aqu\'i que $\esp\left[N\left(t\right)^{m}\right]<\infty$, para $m\geq1$.
\end{Prop}


\begin{Note}
Si el primer tiempo de renovaci\'on $\xi_{1}$ no tiene la misma distribuci\'on que el resto de las $\xi_{n}$, para $n\geq2$, a $N\left(t\right)$ se le llama Proceso de Renovaci\'on retardado, donde si $\xi$ tiene distribuci\'on $G$, entonces el tiempo $T_{n}$ de la $n$-\'esima renovaci\'on tiene distribuci\'on $G\star F^{\left(n-1\right)\star}\left(t\right)$
\end{Note}


\begin{Teo}
Para una constante $\mu\leq\infty$ ( o variable aleatoria), las siguientes expresiones son equivalentes:

\begin{eqnarray}
lim_{n\rightarrow\infty}n^{-1}T_{n}&=&\mu,\textrm{ c.s.}\\
lim_{t\rightarrow\infty}t^{-1}N\left(t\right)&=&1/\mu,\textrm{ c.s.}
\end{eqnarray}
\end{Teo}


Es decir, $T_{n}$ satisface la Ley Fuerte de los Grandes N\'umeros s\'i y s\'olo s\'i $N\left/t\right)$ la cumple.


\begin{Coro}[Ley Fuerte de los Grandes N\'umeros para Procesos de Renovaci\'on]
Si $N\left(t\right)$ es un proceso de renovaci\'on cuyos tiempos de inter-renovaci\'on tienen media $\mu\leq\infty$, entonces
\begin{eqnarray}
t^{-1}N\left(t\right)\rightarrow 1/\mu,\textrm{ c.s. cuando }t\rightarrow\infty.
\end{eqnarray}

\end{Coro}


Considerar el proceso estoc\'astico de valores reales $\left\{Z\left(t\right):t\geq0\right\}$ en el mismo espacio de probabilidad que $N\left(t\right)$

\begin{Def}
Para el proceso $\left\{Z\left(t\right):t\geq0\right\}$ se define la fluctuaci\'on m\'axima de $Z\left(t\right)$ en el intervalo $\left(T_{n-1},T_{n}\right]$:
\begin{eqnarray*}
M_{n}=\sup_{T_{n-1}<t\leq T_{n}}|Z\left(t\right)-Z\left(T_{n-1}\right)|
\end{eqnarray*}
\end{Def}

\begin{Teo}
Sup\'ongase que $n^{-1}T_{n}\rightarrow\mu$ c.s. cuando $n\rightarrow\infty$, donde $\mu\leq\infty$ es una constante o variable aleatoria. Sea $a$ una constante o variable aleatoria que puede ser infinita cuando $\mu$ es finita, y considere las expresiones l\'imite:
\begin{eqnarray}
lim_{n\rightarrow\infty}n^{-1}Z\left(T_{n}\right)&=&a,\textrm{ c.s.}\\
lim_{t\rightarrow\infty}t^{-1}Z\left(t\right)&=&a/\mu,\textrm{ c.s.}
\end{eqnarray}
La segunda expresi\'on implica la primera. Conversamente, la primera implica la segunda si el proceso $Z\left(t\right)$ es creciente, o si $lim_{n\rightarrow\infty}n^{-1}M_{n}=0$ c.s.
\end{Teo}

\begin{Coro}
Si $N\left(t\right)$ es un proceso de renovaci\'on, y $\left(Z\left(T_{n}\right)-Z\left(T_{n-1}\right),M_{n}\right)$, para $n\geq1$, son variables aleatorias independientes e id\'enticamente distribuidas con media finita, entonces,
\begin{eqnarray}
lim_{t\rightarrow\infty}t^{-1}Z\left(t\right)\rightarrow\frac{\esp\left[Z\left(T_{1}\right)-Z\left(T_{0}\right)\right]}{\esp\left[T_{1}\right]},\textrm{ c.s. cuando  }t\rightarrow\infty.
\end{eqnarray}
\end{Coro}


%___________________________________________________________________________________________
%
%\subsection*{Propiedades de los Procesos de Renovaci\'on}
%___________________________________________________________________________________________
%

Los tiempos $T_{n}$ est\'an relacionados con los conteos de $N\left(t\right)$ por

\begin{eqnarray*}
\left\{N\left(t\right)\geq n\right\}&=&\left\{T_{n}\leq t\right\}\\
T_{N\left(t\right)}\leq &t&<T_{N\left(t\right)+1},
\end{eqnarray*}

adem\'as $N\left(T_{n}\right)=n$, y 

\begin{eqnarray*}
N\left(t\right)=\max\left\{n:T_{n}\leq t\right\}=\min\left\{n:T_{n+1}>t\right\}
\end{eqnarray*}

Por propiedades de la convoluci\'on se sabe que

\begin{eqnarray*}
P\left\{T_{n}\leq t\right\}=F^{n\star}\left(t\right)
\end{eqnarray*}
que es la $n$-\'esima convoluci\'on de $F$. Entonces 

\begin{eqnarray*}
\left\{N\left(t\right)\geq n\right\}&=&\left\{T_{n}\leq t\right\}\\
P\left\{N\left(t\right)\leq n\right\}&=&1-F^{\left(n+1\right)\star}\left(t\right)
\end{eqnarray*}

Adem\'as usando el hecho de que $\esp\left[N\left(t\right)\right]=\sum_{n=1}^{\infty}P\left\{N\left(t\right)\geq n\right\}$
se tiene que

\begin{eqnarray*}
\esp\left[N\left(t\right)\right]=\sum_{n=1}^{\infty}F^{n\star}\left(t\right)
\end{eqnarray*}

\begin{Prop}
Para cada $t\geq0$, la funci\'on generadora de momentos $\esp\left[e^{\alpha N\left(t\right)}\right]$ existe para alguna $\alpha$ en una vecindad del 0, y de aqu\'i que $\esp\left[N\left(t\right)^{m}\right]<\infty$, para $m\geq1$.
\end{Prop}


\begin{Note}
Si el primer tiempo de renovaci\'on $\xi_{1}$ no tiene la misma distribuci\'on que el resto de las $\xi_{n}$, para $n\geq2$, a $N\left(t\right)$ se le llama Proceso de Renovaci\'on retardado, donde si $\xi$ tiene distribuci\'on $G$, entonces el tiempo $T_{n}$ de la $n$-\'esima renovaci\'on tiene distribuci\'on $G\star F^{\left(n-1\right)\star}\left(t\right)$
\end{Note}


\begin{Teo}
Para una constante $\mu\leq\infty$ ( o variable aleatoria), las siguientes expresiones son equivalentes:

\begin{eqnarray}
lim_{n\rightarrow\infty}n^{-1}T_{n}&=&\mu,\textrm{ c.s.}\\
lim_{t\rightarrow\infty}t^{-1}N\left(t\right)&=&1/\mu,\textrm{ c.s.}
\end{eqnarray}
\end{Teo}


Es decir, $T_{n}$ satisface la Ley Fuerte de los Grandes N\'umeros s\'i y s\'olo s\'i $N\left/t\right)$ la cumple.


\begin{Coro}[Ley Fuerte de los Grandes N\'umeros para Procesos de Renovaci\'on]
Si $N\left(t\right)$ es un proceso de renovaci\'on cuyos tiempos de inter-renovaci\'on tienen media $\mu\leq\infty$, entonces
\begin{eqnarray}
t^{-1}N\left(t\right)\rightarrow 1/\mu,\textrm{ c.s. cuando }t\rightarrow\infty.
\end{eqnarray}

\end{Coro}


Considerar el proceso estoc\'astico de valores reales $\left\{Z\left(t\right):t\geq0\right\}$ en el mismo espacio de probabilidad que $N\left(t\right)$

\begin{Def}
Para el proceso $\left\{Z\left(t\right):t\geq0\right\}$ se define la fluctuaci\'on m\'axima de $Z\left(t\right)$ en el intervalo $\left(T_{n-1},T_{n}\right]$:
\begin{eqnarray*}
M_{n}=\sup_{T_{n-1}<t\leq T_{n}}|Z\left(t\right)-Z\left(T_{n-1}\right)|
\end{eqnarray*}
\end{Def}

\begin{Teo}
Sup\'ongase que $n^{-1}T_{n}\rightarrow\mu$ c.s. cuando $n\rightarrow\infty$, donde $\mu\leq\infty$ es una constante o variable aleatoria. Sea $a$ una constante o variable aleatoria que puede ser infinita cuando $\mu$ es finita, y considere las expresiones l\'imite:
\begin{eqnarray}
lim_{n\rightarrow\infty}n^{-1}Z\left(T_{n}\right)&=&a,\textrm{ c.s.}\\
lim_{t\rightarrow\infty}t^{-1}Z\left(t\right)&=&a/\mu,\textrm{ c.s.}
\end{eqnarray}
La segunda expresi\'on implica la primera. Conversamente, la primera implica la segunda si el proceso $Z\left(t\right)$ es creciente, o si $lim_{n\rightarrow\infty}n^{-1}M_{n}=0$ c.s.
\end{Teo}

\begin{Coro}
Si $N\left(t\right)$ es un proceso de renovaci\'on, y $\left(Z\left(T_{n}\right)-Z\left(T_{n-1}\right),M_{n}\right)$, para $n\geq1$, son variables aleatorias independientes e id\'enticamente distribuidas con media finita, entonces,
\begin{eqnarray}
lim_{t\rightarrow\infty}t^{-1}Z\left(t\right)\rightarrow\frac{\esp\left[Z\left(T_{1}\right)-Z\left(T_{0}\right)\right]}{\esp\left[T_{1}\right]},\textrm{ c.s. cuando  }t\rightarrow\infty.
\end{eqnarray}
\end{Coro}



%___________________________________________________________________________________________
%
%\subsection*{Propiedades de los Procesos de Renovaci\'on}
%___________________________________________________________________________________________
%

Los tiempos $T_{n}$ est\'an relacionados con los conteos de $N\left(t\right)$ por

\begin{eqnarray*}
\left\{N\left(t\right)\geq n\right\}&=&\left\{T_{n}\leq t\right\}\\
T_{N\left(t\right)}\leq &t&<T_{N\left(t\right)+1},
\end{eqnarray*}

adem\'as $N\left(T_{n}\right)=n$, y 

\begin{eqnarray*}
N\left(t\right)=\max\left\{n:T_{n}\leq t\right\}=\min\left\{n:T_{n+1}>t\right\}
\end{eqnarray*}

Por propiedades de la convoluci\'on se sabe que

\begin{eqnarray*}
P\left\{T_{n}\leq t\right\}=F^{n\star}\left(t\right)
\end{eqnarray*}
que es la $n$-\'esima convoluci\'on de $F$. Entonces 

\begin{eqnarray*}
\left\{N\left(t\right)\geq n\right\}&=&\left\{T_{n}\leq t\right\}\\
P\left\{N\left(t\right)\leq n\right\}&=&1-F^{\left(n+1\right)\star}\left(t\right)
\end{eqnarray*}

Adem\'as usando el hecho de que $\esp\left[N\left(t\right)\right]=\sum_{n=1}^{\infty}P\left\{N\left(t\right)\geq n\right\}$
se tiene que

\begin{eqnarray*}
\esp\left[N\left(t\right)\right]=\sum_{n=1}^{\infty}F^{n\star}\left(t\right)
\end{eqnarray*}

\begin{Prop}
Para cada $t\geq0$, la funci\'on generadora de momentos $\esp\left[e^{\alpha N\left(t\right)}\right]$ existe para alguna $\alpha$ en una vecindad del 0, y de aqu\'i que $\esp\left[N\left(t\right)^{m}\right]<\infty$, para $m\geq1$.
\end{Prop}


\begin{Note}
Si el primer tiempo de renovaci\'on $\xi_{1}$ no tiene la misma distribuci\'on que el resto de las $\xi_{n}$, para $n\geq2$, a $N\left(t\right)$ se le llama Proceso de Renovaci\'on retardado, donde si $\xi$ tiene distribuci\'on $G$, entonces el tiempo $T_{n}$ de la $n$-\'esima renovaci\'on tiene distribuci\'on $G\star F^{\left(n-1\right)\star}\left(t\right)$
\end{Note}


\begin{Teo}
Para una constante $\mu\leq\infty$ ( o variable aleatoria), las siguientes expresiones son equivalentes:

\begin{eqnarray}
lim_{n\rightarrow\infty}n^{-1}T_{n}&=&\mu,\textrm{ c.s.}\\
lim_{t\rightarrow\infty}t^{-1}N\left(t\right)&=&1/\mu,\textrm{ c.s.}
\end{eqnarray}
\end{Teo}


Es decir, $T_{n}$ satisface la Ley Fuerte de los Grandes N\'umeros s\'i y s\'olo s\'i $N\left/t\right)$ la cumple.


\begin{Coro}[Ley Fuerte de los Grandes N\'umeros para Procesos de Renovaci\'on]
Si $N\left(t\right)$ es un proceso de renovaci\'on cuyos tiempos de inter-renovaci\'on tienen media $\mu\leq\infty$, entonces
\begin{eqnarray}
t^{-1}N\left(t\right)\rightarrow 1/\mu,\textrm{ c.s. cuando }t\rightarrow\infty.
\end{eqnarray}

\end{Coro}


Considerar el proceso estoc\'astico de valores reales $\left\{Z\left(t\right):t\geq0\right\}$ en el mismo espacio de probabilidad que $N\left(t\right)$

\begin{Def}
Para el proceso $\left\{Z\left(t\right):t\geq0\right\}$ se define la fluctuaci\'on m\'axima de $Z\left(t\right)$ en el intervalo $\left(T_{n-1},T_{n}\right]$:
\begin{eqnarray*}
M_{n}=\sup_{T_{n-1}<t\leq T_{n}}|Z\left(t\right)-Z\left(T_{n-1}\right)|
\end{eqnarray*}
\end{Def}

\begin{Teo}
Sup\'ongase que $n^{-1}T_{n}\rightarrow\mu$ c.s. cuando $n\rightarrow\infty$, donde $\mu\leq\infty$ es una constante o variable aleatoria. Sea $a$ una constante o variable aleatoria que puede ser infinita cuando $\mu$ es finita, y considere las expresiones l\'imite:
\begin{eqnarray}
lim_{n\rightarrow\infty}n^{-1}Z\left(T_{n}\right)&=&a,\textrm{ c.s.}\\
lim_{t\rightarrow\infty}t^{-1}Z\left(t\right)&=&a/\mu,\textrm{ c.s.}
\end{eqnarray}
La segunda expresi\'on implica la primera. Conversamente, la primera implica la segunda si el proceso $Z\left(t\right)$ es creciente, o si $lim_{n\rightarrow\infty}n^{-1}M_{n}=0$ c.s.
\end{Teo}

\begin{Coro}
Si $N\left(t\right)$ es un proceso de renovaci\'on, y $\left(Z\left(T_{n}\right)-Z\left(T_{n-1}\right),M_{n}\right)$, para $n\geq1$, son variables aleatorias independientes e id\'enticamente distribuidas con media finita, entonces,
\begin{eqnarray}
lim_{t\rightarrow\infty}t^{-1}Z\left(t\right)\rightarrow\frac{\esp\left[Z\left(T_{1}\right)-Z\left(T_{0}\right)\right]}{\esp\left[T_{1}\right]},\textrm{ c.s. cuando  }t\rightarrow\infty.
\end{eqnarray}
\end{Coro}


%___________________________________________________________________________________________
%
%\subsection*{Propiedades de los Procesos de Renovaci\'on}
%___________________________________________________________________________________________
%

Los tiempos $T_{n}$ est\'an relacionados con los conteos de $N\left(t\right)$ por

\begin{eqnarray*}
\left\{N\left(t\right)\geq n\right\}&=&\left\{T_{n}\leq t\right\}\\
T_{N\left(t\right)}\leq &t&<T_{N\left(t\right)+1},
\end{eqnarray*}

adem\'as $N\left(T_{n}\right)=n$, y 

\begin{eqnarray*}
N\left(t\right)=\max\left\{n:T_{n}\leq t\right\}=\min\left\{n:T_{n+1}>t\right\}
\end{eqnarray*}

Por propiedades de la convoluci\'on se sabe que

\begin{eqnarray*}
P\left\{T_{n}\leq t\right\}=F^{n\star}\left(t\right)
\end{eqnarray*}
que es la $n$-\'esima convoluci\'on de $F$. Entonces 

\begin{eqnarray*}
\left\{N\left(t\right)\geq n\right\}&=&\left\{T_{n}\leq t\right\}\\
P\left\{N\left(t\right)\leq n\right\}&=&1-F^{\left(n+1\right)\star}\left(t\right)
\end{eqnarray*}

Adem\'as usando el hecho de que $\esp\left[N\left(t\right)\right]=\sum_{n=1}^{\infty}P\left\{N\left(t\right)\geq n\right\}$
se tiene que

\begin{eqnarray*}
\esp\left[N\left(t\right)\right]=\sum_{n=1}^{\infty}F^{n\star}\left(t\right)
\end{eqnarray*}

\begin{Prop}
Para cada $t\geq0$, la funci\'on generadora de momentos $\esp\left[e^{\alpha N\left(t\right)}\right]$ existe para alguna $\alpha$ en una vecindad del 0, y de aqu\'i que $\esp\left[N\left(t\right)^{m}\right]<\infty$, para $m\geq1$.
\end{Prop}


\begin{Note}
Si el primer tiempo de renovaci\'on $\xi_{1}$ no tiene la misma distribuci\'on que el resto de las $\xi_{n}$, para $n\geq2$, a $N\left(t\right)$ se le llama Proceso de Renovaci\'on retardado, donde si $\xi$ tiene distribuci\'on $G$, entonces el tiempo $T_{n}$ de la $n$-\'esima renovaci\'on tiene distribuci\'on $G\star F^{\left(n-1\right)\star}\left(t\right)$
\end{Note}


\begin{Teo}
Para una constante $\mu\leq\infty$ ( o variable aleatoria), las siguientes expresiones son equivalentes:

\begin{eqnarray}
lim_{n\rightarrow\infty}n^{-1}T_{n}&=&\mu,\textrm{ c.s.}\\
lim_{t\rightarrow\infty}t^{-1}N\left(t\right)&=&1/\mu,\textrm{ c.s.}
\end{eqnarray}
\end{Teo}


Es decir, $T_{n}$ satisface la Ley Fuerte de los Grandes N\'umeros s\'i y s\'olo s\'i $N\left/t\right)$ la cumple.


\begin{Coro}[Ley Fuerte de los Grandes N\'umeros para Procesos de Renovaci\'on]
Si $N\left(t\right)$ es un proceso de renovaci\'on cuyos tiempos de inter-renovaci\'on tienen media $\mu\leq\infty$, entonces
\begin{eqnarray}
t^{-1}N\left(t\right)\rightarrow 1/\mu,\textrm{ c.s. cuando }t\rightarrow\infty.
\end{eqnarray}

\end{Coro}


Considerar el proceso estoc\'astico de valores reales $\left\{Z\left(t\right):t\geq0\right\}$ en el mismo espacio de probabilidad que $N\left(t\right)$

\begin{Def}
Para el proceso $\left\{Z\left(t\right):t\geq0\right\}$ se define la fluctuaci\'on m\'axima de $Z\left(t\right)$ en el intervalo $\left(T_{n-1},T_{n}\right]$:
\begin{eqnarray*}
M_{n}=\sup_{T_{n-1}<t\leq T_{n}}|Z\left(t\right)-Z\left(T_{n-1}\right)|
\end{eqnarray*}
\end{Def}

\begin{Teo}
Sup\'ongase que $n^{-1}T_{n}\rightarrow\mu$ c.s. cuando $n\rightarrow\infty$, donde $\mu\leq\infty$ es una constante o variable aleatoria. Sea $a$ una constante o variable aleatoria que puede ser infinita cuando $\mu$ es finita, y considere las expresiones l\'imite:
\begin{eqnarray}
lim_{n\rightarrow\infty}n^{-1}Z\left(T_{n}\right)&=&a,\textrm{ c.s.}\\
lim_{t\rightarrow\infty}t^{-1}Z\left(t\right)&=&a/\mu,\textrm{ c.s.}
\end{eqnarray}
La segunda expresi\'on implica la primera. Conversamente, la primera implica la segunda si el proceso $Z\left(t\right)$ es creciente, o si $lim_{n\rightarrow\infty}n^{-1}M_{n}=0$ c.s.
\end{Teo}

\begin{Coro}
Si $N\left(t\right)$ es un proceso de renovaci\'on, y $\left(Z\left(T_{n}\right)-Z\left(T_{n-1}\right),M_{n}\right)$, para $n\geq1$, son variables aleatorias independientes e id\'enticamente distribuidas con media finita, entonces,
\begin{eqnarray}
lim_{t\rightarrow\infty}t^{-1}Z\left(t\right)\rightarrow\frac{\esp\left[Z\left(T_{1}\right)-Z\left(T_{0}\right)\right]}{\esp\left[T_{1}\right]},\textrm{ c.s. cuando  }t\rightarrow\infty.
\end{eqnarray}
\end{Coro}

%___________________________________________________________________________________________
%
%\subsection*{Propiedades de los Procesos de Renovaci\'on}
%___________________________________________________________________________________________
%

Los tiempos $T_{n}$ est\'an relacionados con los conteos de $N\left(t\right)$ por

\begin{eqnarray*}
\left\{N\left(t\right)\geq n\right\}&=&\left\{T_{n}\leq t\right\}\\
T_{N\left(t\right)}\leq &t&<T_{N\left(t\right)+1},
\end{eqnarray*}

adem\'as $N\left(T_{n}\right)=n$, y 

\begin{eqnarray*}
N\left(t\right)=\max\left\{n:T_{n}\leq t\right\}=\min\left\{n:T_{n+1}>t\right\}
\end{eqnarray*}

Por propiedades de la convoluci\'on se sabe que

\begin{eqnarray*}
P\left\{T_{n}\leq t\right\}=F^{n\star}\left(t\right)
\end{eqnarray*}
que es la $n$-\'esima convoluci\'on de $F$. Entonces 

\begin{eqnarray*}
\left\{N\left(t\right)\geq n\right\}&=&\left\{T_{n}\leq t\right\}\\
P\left\{N\left(t\right)\leq n\right\}&=&1-F^{\left(n+1\right)\star}\left(t\right)
\end{eqnarray*}

Adem\'as usando el hecho de que $\esp\left[N\left(t\right)\right]=\sum_{n=1}^{\infty}P\left\{N\left(t\right)\geq n\right\}$
se tiene que

\begin{eqnarray*}
\esp\left[N\left(t\right)\right]=\sum_{n=1}^{\infty}F^{n\star}\left(t\right)
\end{eqnarray*}

\begin{Prop}
Para cada $t\geq0$, la funci\'on generadora de momentos $\esp\left[e^{\alpha N\left(t\right)}\right]$ existe para alguna $\alpha$ en una vecindad del 0, y de aqu\'i que $\esp\left[N\left(t\right)^{m}\right]<\infty$, para $m\geq1$.
\end{Prop}


\begin{Note}
Si el primer tiempo de renovaci\'on $\xi_{1}$ no tiene la misma distribuci\'on que el resto de las $\xi_{n}$, para $n\geq2$, a $N\left(t\right)$ se le llama Proceso de Renovaci\'on retardado, donde si $\xi$ tiene distribuci\'on $G$, entonces el tiempo $T_{n}$ de la $n$-\'esima renovaci\'on tiene distribuci\'on $G\star F^{\left(n-1\right)\star}\left(t\right)$
\end{Note}


\begin{Teo}
Para una constante $\mu\leq\infty$ ( o variable aleatoria), las siguientes expresiones son equivalentes:

\begin{eqnarray}
lim_{n\rightarrow\infty}n^{-1}T_{n}&=&\mu,\textrm{ c.s.}\\
lim_{t\rightarrow\infty}t^{-1}N\left(t\right)&=&1/\mu,\textrm{ c.s.}
\end{eqnarray}
\end{Teo}


Es decir, $T_{n}$ satisface la Ley Fuerte de los Grandes N\'umeros s\'i y s\'olo s\'i $N\left/t\right)$ la cumple.


\begin{Coro}[Ley Fuerte de los Grandes N\'umeros para Procesos de Renovaci\'on]
Si $N\left(t\right)$ es un proceso de renovaci\'on cuyos tiempos de inter-renovaci\'on tienen media $\mu\leq\infty$, entonces
\begin{eqnarray}
t^{-1}N\left(t\right)\rightarrow 1/\mu,\textrm{ c.s. cuando }t\rightarrow\infty.
\end{eqnarray}

\end{Coro}


Considerar el proceso estoc\'astico de valores reales $\left\{Z\left(t\right):t\geq0\right\}$ en el mismo espacio de probabilidad que $N\left(t\right)$

\begin{Def}
Para el proceso $\left\{Z\left(t\right):t\geq0\right\}$ se define la fluctuaci\'on m\'axima de $Z\left(t\right)$ en el intervalo $\left(T_{n-1},T_{n}\right]$:
\begin{eqnarray*}
M_{n}=\sup_{T_{n-1}<t\leq T_{n}}|Z\left(t\right)-Z\left(T_{n-1}\right)|
\end{eqnarray*}
\end{Def}

\begin{Teo}
Sup\'ongase que $n^{-1}T_{n}\rightarrow\mu$ c.s. cuando $n\rightarrow\infty$, donde $\mu\leq\infty$ es una constante o variable aleatoria. Sea $a$ una constante o variable aleatoria que puede ser infinita cuando $\mu$ es finita, y considere las expresiones l\'imite:
\begin{eqnarray}
lim_{n\rightarrow\infty}n^{-1}Z\left(T_{n}\right)&=&a,\textrm{ c.s.}\\
lim_{t\rightarrow\infty}t^{-1}Z\left(t\right)&=&a/\mu,\textrm{ c.s.}
\end{eqnarray}
La segunda expresi\'on implica la primera. Conversamente, la primera implica la segunda si el proceso $Z\left(t\right)$ es creciente, o si $lim_{n\rightarrow\infty}n^{-1}M_{n}=0$ c.s.
\end{Teo}

\begin{Coro}
Si $N\left(t\right)$ es un proceso de renovaci\'on, y $\left(Z\left(T_{n}\right)-Z\left(T_{n-1}\right),M_{n}\right)$, para $n\geq1$, son variables aleatorias independientes e id\'enticamente distribuidas con media finita, entonces,
\begin{eqnarray}
lim_{t\rightarrow\infty}t^{-1}Z\left(t\right)\rightarrow\frac{\esp\left[Z\left(T_{1}\right)-Z\left(T_{0}\right)\right]}{\esp\left[T_{1}\right]},\textrm{ c.s. cuando  }t\rightarrow\infty.
\end{eqnarray}
\end{Coro}
%___________________________________________________________________________________________
%
%\subsection*{Propiedades de los Procesos de Renovaci\'on}
%___________________________________________________________________________________________
%

Los tiempos $T_{n}$ est\'an relacionados con los conteos de $N\left(t\right)$ por

\begin{eqnarray*}
\left\{N\left(t\right)\geq n\right\}&=&\left\{T_{n}\leq t\right\}\\
T_{N\left(t\right)}\leq &t&<T_{N\left(t\right)+1},
\end{eqnarray*}

adem\'as $N\left(T_{n}\right)=n$, y 

\begin{eqnarray*}
N\left(t\right)=\max\left\{n:T_{n}\leq t\right\}=\min\left\{n:T_{n+1}>t\right\}
\end{eqnarray*}

Por propiedades de la convoluci\'on se sabe que

\begin{eqnarray*}
P\left\{T_{n}\leq t\right\}=F^{n\star}\left(t\right)
\end{eqnarray*}
que es la $n$-\'esima convoluci\'on de $F$. Entonces 

\begin{eqnarray*}
\left\{N\left(t\right)\geq n\right\}&=&\left\{T_{n}\leq t\right\}\\
P\left\{N\left(t\right)\leq n\right\}&=&1-F^{\left(n+1\right)\star}\left(t\right)
\end{eqnarray*}

Adem\'as usando el hecho de que $\esp\left[N\left(t\right)\right]=\sum_{n=1}^{\infty}P\left\{N\left(t\right)\geq n\right\}$
se tiene que

\begin{eqnarray*}
\esp\left[N\left(t\right)\right]=\sum_{n=1}^{\infty}F^{n\star}\left(t\right)
\end{eqnarray*}

\begin{Prop}
Para cada $t\geq0$, la funci\'on generadora de momentos $\esp\left[e^{\alpha N\left(t\right)}\right]$ existe para alguna $\alpha$ en una vecindad del 0, y de aqu\'i que $\esp\left[N\left(t\right)^{m}\right]<\infty$, para $m\geq1$.
\end{Prop}


\begin{Note}
Si el primer tiempo de renovaci\'on $\xi_{1}$ no tiene la misma distribuci\'on que el resto de las $\xi_{n}$, para $n\geq2$, a $N\left(t\right)$ se le llama Proceso de Renovaci\'on retardado, donde si $\xi$ tiene distribuci\'on $G$, entonces el tiempo $T_{n}$ de la $n$-\'esima renovaci\'on tiene distribuci\'on $G\star F^{\left(n-1\right)\star}\left(t\right)$
\end{Note}


\begin{Teo}
Para una constante $\mu\leq\infty$ ( o variable aleatoria), las siguientes expresiones son equivalentes:

\begin{eqnarray}
lim_{n\rightarrow\infty}n^{-1}T_{n}&=&\mu,\textrm{ c.s.}\\
lim_{t\rightarrow\infty}t^{-1}N\left(t\right)&=&1/\mu,\textrm{ c.s.}
\end{eqnarray}
\end{Teo}


Es decir, $T_{n}$ satisface la Ley Fuerte de los Grandes N\'umeros s\'i y s\'olo s\'i $N\left/t\right)$ la cumple.


\begin{Coro}[Ley Fuerte de los Grandes N\'umeros para Procesos de Renovaci\'on]
Si $N\left(t\right)$ es un proceso de renovaci\'on cuyos tiempos de inter-renovaci\'on tienen media $\mu\leq\infty$, entonces
\begin{eqnarray}
t^{-1}N\left(t\right)\rightarrow 1/\mu,\textrm{ c.s. cuando }t\rightarrow\infty.
\end{eqnarray}

\end{Coro}


Considerar el proceso estoc\'astico de valores reales $\left\{Z\left(t\right):t\geq0\right\}$ en el mismo espacio de probabilidad que $N\left(t\right)$

\begin{Def}
Para el proceso $\left\{Z\left(t\right):t\geq0\right\}$ se define la fluctuaci\'on m\'axima de $Z\left(t\right)$ en el intervalo $\left(T_{n-1},T_{n}\right]$:
\begin{eqnarray*}
M_{n}=\sup_{T_{n-1}<t\leq T_{n}}|Z\left(t\right)-Z\left(T_{n-1}\right)|
\end{eqnarray*}
\end{Def}

\begin{Teo}
Sup\'ongase que $n^{-1}T_{n}\rightarrow\mu$ c.s. cuando $n\rightarrow\infty$, donde $\mu\leq\infty$ es una constante o variable aleatoria. Sea $a$ una constante o variable aleatoria que puede ser infinita cuando $\mu$ es finita, y considere las expresiones l\'imite:
\begin{eqnarray}
lim_{n\rightarrow\infty}n^{-1}Z\left(T_{n}\right)&=&a,\textrm{ c.s.}\\
lim_{t\rightarrow\infty}t^{-1}Z\left(t\right)&=&a/\mu,\textrm{ c.s.}
\end{eqnarray}
La segunda expresi\'on implica la primera. Conversamente, la primera implica la segunda si el proceso $Z\left(t\right)$ es creciente, o si $lim_{n\rightarrow\infty}n^{-1}M_{n}=0$ c.s.
\end{Teo}

\begin{Coro}
Si $N\left(t\right)$ es un proceso de renovaci\'on, y $\left(Z\left(T_{n}\right)-Z\left(T_{n-1}\right),M_{n}\right)$, para $n\geq1$, son variables aleatorias independientes e id\'enticamente distribuidas con media finita, entonces,
\begin{eqnarray}
lim_{t\rightarrow\infty}t^{-1}Z\left(t\right)\rightarrow\frac{\esp\left[Z\left(T_{1}\right)-Z\left(T_{0}\right)\right]}{\esp\left[T_{1}\right]},\textrm{ c.s. cuando  }t\rightarrow\infty.
\end{eqnarray}
\end{Coro}


%___________________________________________________________________________________________
%
%\subsection*{Propiedades de los Procesos de Renovaci\'on}
%___________________________________________________________________________________________
%

Los tiempos $T_{n}$ est\'an relacionados con los conteos de $N\left(t\right)$ por

\begin{eqnarray*}
\left\{N\left(t\right)\geq n\right\}&=&\left\{T_{n}\leq t\right\}\\
T_{N\left(t\right)}\leq &t&<T_{N\left(t\right)+1},
\end{eqnarray*}

adem\'as $N\left(T_{n}\right)=n$, y 

\begin{eqnarray*}
N\left(t\right)=\max\left\{n:T_{n}\leq t\right\}=\min\left\{n:T_{n+1}>t\right\}
\end{eqnarray*}

Por propiedades de la convoluci\'on se sabe que

\begin{eqnarray*}
P\left\{T_{n}\leq t\right\}=F^{n\star}\left(t\right)
\end{eqnarray*}
que es la $n$-\'esima convoluci\'on de $F$. Entonces 

\begin{eqnarray*}
\left\{N\left(t\right)\geq n\right\}&=&\left\{T_{n}\leq t\right\}\\
P\left\{N\left(t\right)\leq n\right\}&=&1-F^{\left(n+1\right)\star}\left(t\right)
\end{eqnarray*}

Adem\'as usando el hecho de que $\esp\left[N\left(t\right)\right]=\sum_{n=1}^{\infty}P\left\{N\left(t\right)\geq n\right\}$
se tiene que

\begin{eqnarray*}
\esp\left[N\left(t\right)\right]=\sum_{n=1}^{\infty}F^{n\star}\left(t\right)
\end{eqnarray*}

\begin{Prop}
Para cada $t\geq0$, la funci\'on generadora de momentos $\esp\left[e^{\alpha N\left(t\right)}\right]$ existe para alguna $\alpha$ en una vecindad del 0, y de aqu\'i que $\esp\left[N\left(t\right)^{m}\right]<\infty$, para $m\geq1$.
\end{Prop}


\begin{Note}
Si el primer tiempo de renovaci\'on $\xi_{1}$ no tiene la misma distribuci\'on que el resto de las $\xi_{n}$, para $n\geq2$, a $N\left(t\right)$ se le llama Proceso de Renovaci\'on retardado, donde si $\xi$ tiene distribuci\'on $G$, entonces el tiempo $T_{n}$ de la $n$-\'esima renovaci\'on tiene distribuci\'on $G\star F^{\left(n-1\right)\star}\left(t\right)$
\end{Note}


\begin{Teo}
Para una constante $\mu\leq\infty$ ( o variable aleatoria), las siguientes expresiones son equivalentes:

\begin{eqnarray}
lim_{n\rightarrow\infty}n^{-1}T_{n}&=&\mu,\textrm{ c.s.}\\
lim_{t\rightarrow\infty}t^{-1}N\left(t\right)&=&1/\mu,\textrm{ c.s.}
\end{eqnarray}
\end{Teo}


Es decir, $T_{n}$ satisface la Ley Fuerte de los Grandes N\'umeros s\'i y s\'olo s\'i $N\left/t\right)$ la cumple.


\begin{Coro}[Ley Fuerte de los Grandes N\'umeros para Procesos de Renovaci\'on]
Si $N\left(t\right)$ es un proceso de renovaci\'on cuyos tiempos de inter-renovaci\'on tienen media $\mu\leq\infty$, entonces
\begin{eqnarray}
t^{-1}N\left(t\right)\rightarrow 1/\mu,\textrm{ c.s. cuando }t\rightarrow\infty.
\end{eqnarray}

\end{Coro}


Considerar el proceso estoc\'astico de valores reales $\left\{Z\left(t\right):t\geq0\right\}$ en el mismo espacio de probabilidad que $N\left(t\right)$

\begin{Def}
Para el proceso $\left\{Z\left(t\right):t\geq0\right\}$ se define la fluctuaci\'on m\'axima de $Z\left(t\right)$ en el intervalo $\left(T_{n-1},T_{n}\right]$:
\begin{eqnarray*}
M_{n}=\sup_{T_{n-1}<t\leq T_{n}}|Z\left(t\right)-Z\left(T_{n-1}\right)|
\end{eqnarray*}
\end{Def}

\begin{Teo}
Sup\'ongase que $n^{-1}T_{n}\rightarrow\mu$ c.s. cuando $n\rightarrow\infty$, donde $\mu\leq\infty$ es una constante o variable aleatoria. Sea $a$ una constante o variable aleatoria que puede ser infinita cuando $\mu$ es finita, y considere las expresiones l\'imite:
\begin{eqnarray}
lim_{n\rightarrow\infty}n^{-1}Z\left(T_{n}\right)&=&a,\textrm{ c.s.}\\
lim_{t\rightarrow\infty}t^{-1}Z\left(t\right)&=&a/\mu,\textrm{ c.s.}
\end{eqnarray}
La segunda expresi\'on implica la primera. Conversamente, la primera implica la segunda si el proceso $Z\left(t\right)$ es creciente, o si $lim_{n\rightarrow\infty}n^{-1}M_{n}=0$ c.s.
\end{Teo}

\begin{Coro}
Si $N\left(t\right)$ es un proceso de renovaci\'on, y $\left(Z\left(T_{n}\right)-Z\left(T_{n-1}\right),M_{n}\right)$, para $n\geq1$, son variables aleatorias independientes e id\'enticamente distribuidas con media finita, entonces,
\begin{eqnarray}
lim_{t\rightarrow\infty}t^{-1}Z\left(t\right)\rightarrow\frac{\esp\left[Z\left(T_{1}\right)-Z\left(T_{0}\right)\right]}{\esp\left[T_{1}\right]},\textrm{ c.s. cuando  }t\rightarrow\infty.
\end{eqnarray}
\end{Coro}



%___________________________________________________________________________________________
%
%\subsection*{Propiedades de los Procesos de Renovaci\'on}
%___________________________________________________________________________________________
%

Los tiempos $T_{n}$ est\'an relacionados con los conteos de $N\left(t\right)$ por

\begin{eqnarray*}
\left\{N\left(t\right)\geq n\right\}&=&\left\{T_{n}\leq t\right\}\\
T_{N\left(t\right)}\leq &t&<T_{N\left(t\right)+1},
\end{eqnarray*}

adem\'as $N\left(T_{n}\right)=n$, y 

\begin{eqnarray*}
N\left(t\right)=\max\left\{n:T_{n}\leq t\right\}=\min\left\{n:T_{n+1}>t\right\}
\end{eqnarray*}

Por propiedades de la convoluci\'on se sabe que

\begin{eqnarray*}
P\left\{T_{n}\leq t\right\}=F^{n\star}\left(t\right)
\end{eqnarray*}
que es la $n$-\'esima convoluci\'on de $F$. Entonces 

\begin{eqnarray*}
\left\{N\left(t\right)\geq n\right\}&=&\left\{T_{n}\leq t\right\}\\
P\left\{N\left(t\right)\leq n\right\}&=&1-F^{\left(n+1\right)\star}\left(t\right)
\end{eqnarray*}

Adem\'as usando el hecho de que $\esp\left[N\left(t\right)\right]=\sum_{n=1}^{\infty}P\left\{N\left(t\right)\geq n\right\}$
se tiene que

\begin{eqnarray*}
\esp\left[N\left(t\right)\right]=\sum_{n=1}^{\infty}F^{n\star}\left(t\right)
\end{eqnarray*}

\begin{Prop}
Para cada $t\geq0$, la funci\'on generadora de momentos $\esp\left[e^{\alpha N\left(t\right)}\right]$ existe para alguna $\alpha$ en una vecindad del 0, y de aqu\'i que $\esp\left[N\left(t\right)^{m}\right]<\infty$, para $m\geq1$.
\end{Prop}


\begin{Note}
Si el primer tiempo de renovaci\'on $\xi_{1}$ no tiene la misma distribuci\'on que el resto de las $\xi_{n}$, para $n\geq2$, a $N\left(t\right)$ se le llama Proceso de Renovaci\'on retardado, donde si $\xi$ tiene distribuci\'on $G$, entonces el tiempo $T_{n}$ de la $n$-\'esima renovaci\'on tiene distribuci\'on $G\star F^{\left(n-1\right)\star}\left(t\right)$
\end{Note}


\begin{Teo}
Para una constante $\mu\leq\infty$ ( o variable aleatoria), las siguientes expresiones son equivalentes:

\begin{eqnarray}
lim_{n\rightarrow\infty}n^{-1}T_{n}&=&\mu,\textrm{ c.s.}\\
lim_{t\rightarrow\infty}t^{-1}N\left(t\right)&=&1/\mu,\textrm{ c.s.}
\end{eqnarray}
\end{Teo}


Es decir, $T_{n}$ satisface la Ley Fuerte de los Grandes N\'umeros s\'i y s\'olo s\'i $N\left/t\right)$ la cumple.


\begin{Coro}[Ley Fuerte de los Grandes N\'umeros para Procesos de Renovaci\'on]
Si $N\left(t\right)$ es un proceso de renovaci\'on cuyos tiempos de inter-renovaci\'on tienen media $\mu\leq\infty$, entonces
\begin{eqnarray}
t^{-1}N\left(t\right)\rightarrow 1/\mu,\textrm{ c.s. cuando }t\rightarrow\infty.
\end{eqnarray}

\end{Coro}


Considerar el proceso estoc\'astico de valores reales $\left\{Z\left(t\right):t\geq0\right\}$ en el mismo espacio de probabilidad que $N\left(t\right)$

\begin{Def}
Para el proceso $\left\{Z\left(t\right):t\geq0\right\}$ se define la fluctuaci\'on m\'axima de $Z\left(t\right)$ en el intervalo $\left(T_{n-1},T_{n}\right]$:
\begin{eqnarray*}
M_{n}=\sup_{T_{n-1}<t\leq T_{n}}|Z\left(t\right)-Z\left(T_{n-1}\right)|
\end{eqnarray*}
\end{Def}

\begin{Teo}
Sup\'ongase que $n^{-1}T_{n}\rightarrow\mu$ c.s. cuando $n\rightarrow\infty$, donde $\mu\leq\infty$ es una constante o variable aleatoria. Sea $a$ una constante o variable aleatoria que puede ser infinita cuando $\mu$ es finita, y considere las expresiones l\'imite:
\begin{eqnarray}
lim_{n\rightarrow\infty}n^{-1}Z\left(T_{n}\right)&=&a,\textrm{ c.s.}\\
lim_{t\rightarrow\infty}t^{-1}Z\left(t\right)&=&a/\mu,\textrm{ c.s.}
\end{eqnarray}
La segunda expresi\'on implica la primera. Conversamente, la primera implica la segunda si el proceso $Z\left(t\right)$ es creciente, o si $lim_{n\rightarrow\infty}n^{-1}M_{n}=0$ c.s.
\end{Teo}

\begin{Coro}
Si $N\left(t\right)$ es un proceso de renovaci\'on, y $\left(Z\left(T_{n}\right)-Z\left(T_{n-1}\right),M_{n}\right)$, para $n\geq1$, son variables aleatorias independientes e id\'enticamente distribuidas con media finita, entonces,
\begin{eqnarray}
lim_{t\rightarrow\infty}t^{-1}Z\left(t\right)\rightarrow\frac{\esp\left[Z\left(T_{1}\right)-Z\left(T_{0}\right)\right]}{\esp\left[T_{1}\right]},\textrm{ c.s. cuando  }t\rightarrow\infty.
\end{eqnarray}
\end{Coro}


%___________________________________________________________________________________________
%
%\subsection*{Propiedades de los Procesos de Renovaci\'on}
%___________________________________________________________________________________________
%

Los tiempos $T_{n}$ est\'an relacionados con los conteos de $N\left(t\right)$ por

\begin{eqnarray*}
\left\{N\left(t\right)\geq n\right\}&=&\left\{T_{n}\leq t\right\}\\
T_{N\left(t\right)}\leq &t&<T_{N\left(t\right)+1},
\end{eqnarray*}

adem\'as $N\left(T_{n}\right)=n$, y 

\begin{eqnarray*}
N\left(t\right)=\max\left\{n:T_{n}\leq t\right\}=\min\left\{n:T_{n+1}>t\right\}
\end{eqnarray*}

Por propiedades de la convoluci\'on se sabe que

\begin{eqnarray*}
P\left\{T_{n}\leq t\right\}=F^{n\star}\left(t\right)
\end{eqnarray*}
que es la $n$-\'esima convoluci\'on de $F$. Entonces 

\begin{eqnarray*}
\left\{N\left(t\right)\geq n\right\}&=&\left\{T_{n}\leq t\right\}\\
P\left\{N\left(t\right)\leq n\right\}&=&1-F^{\left(n+1\right)\star}\left(t\right)
\end{eqnarray*}

Adem\'as usando el hecho de que $\esp\left[N\left(t\right)\right]=\sum_{n=1}^{\infty}P\left\{N\left(t\right)\geq n\right\}$
se tiene que

\begin{eqnarray*}
\esp\left[N\left(t\right)\right]=\sum_{n=1}^{\infty}F^{n\star}\left(t\right)
\end{eqnarray*}

\begin{Prop}
Para cada $t\geq0$, la funci\'on generadora de momentos $\esp\left[e^{\alpha N\left(t\right)}\right]$ existe para alguna $\alpha$ en una vecindad del 0, y de aqu\'i que $\esp\left[N\left(t\right)^{m}\right]<\infty$, para $m\geq1$.
\end{Prop}


\begin{Note}
Si el primer tiempo de renovaci\'on $\xi_{1}$ no tiene la misma distribuci\'on que el resto de las $\xi_{n}$, para $n\geq2$, a $N\left(t\right)$ se le llama Proceso de Renovaci\'on retardado, donde si $\xi$ tiene distribuci\'on $G$, entonces el tiempo $T_{n}$ de la $n$-\'esima renovaci\'on tiene distribuci\'on $G\star F^{\left(n-1\right)\star}\left(t\right)$
\end{Note}


\begin{Teo}
Para una constante $\mu\leq\infty$ ( o variable aleatoria), las siguientes expresiones son equivalentes:

\begin{eqnarray}
lim_{n\rightarrow\infty}n^{-1}T_{n}&=&\mu,\textrm{ c.s.}\\
lim_{t\rightarrow\infty}t^{-1}N\left(t\right)&=&1/\mu,\textrm{ c.s.}
\end{eqnarray}
\end{Teo}


Es decir, $T_{n}$ satisface la Ley Fuerte de los Grandes N\'umeros s\'i y s\'olo s\'i $N\left/t\right)$ la cumple.


\begin{Coro}[Ley Fuerte de los Grandes N\'umeros para Procesos de Renovaci\'on]
Si $N\left(t\right)$ es un proceso de renovaci\'on cuyos tiempos de inter-renovaci\'on tienen media $\mu\leq\infty$, entonces
\begin{eqnarray}
t^{-1}N\left(t\right)\rightarrow 1/\mu,\textrm{ c.s. cuando }t\rightarrow\infty.
\end{eqnarray}

\end{Coro}


Considerar el proceso estoc\'astico de valores reales $\left\{Z\left(t\right):t\geq0\right\}$ en el mismo espacio de probabilidad que $N\left(t\right)$

\begin{Def}
Para el proceso $\left\{Z\left(t\right):t\geq0\right\}$ se define la fluctuaci\'on m\'axima de $Z\left(t\right)$ en el intervalo $\left(T_{n-1},T_{n}\right]$:
\begin{eqnarray*}
M_{n}=\sup_{T_{n-1}<t\leq T_{n}}|Z\left(t\right)-Z\left(T_{n-1}\right)|
\end{eqnarray*}
\end{Def}

\begin{Teo}
Sup\'ongase que $n^{-1}T_{n}\rightarrow\mu$ c.s. cuando $n\rightarrow\infty$, donde $\mu\leq\infty$ es una constante o variable aleatoria. Sea $a$ una constante o variable aleatoria que puede ser infinita cuando $\mu$ es finita, y considere las expresiones l\'imite:
\begin{eqnarray}
lim_{n\rightarrow\infty}n^{-1}Z\left(T_{n}\right)&=&a,\textrm{ c.s.}\\
lim_{t\rightarrow\infty}t^{-1}Z\left(t\right)&=&a/\mu,\textrm{ c.s.}
\end{eqnarray}
La segunda expresi\'on implica la primera. Conversamente, la primera implica la segunda si el proceso $Z\left(t\right)$ es creciente, o si $lim_{n\rightarrow\infty}n^{-1}M_{n}=0$ c.s.
\end{Teo}

\begin{Coro}
Si $N\left(t\right)$ es un proceso de renovaci\'on, y $\left(Z\left(T_{n}\right)-Z\left(T_{n-1}\right),M_{n}\right)$, para $n\geq1$, son variables aleatorias independientes e id\'enticamente distribuidas con media finita, entonces,
\begin{eqnarray}
lim_{t\rightarrow\infty}t^{-1}Z\left(t\right)\rightarrow\frac{\esp\left[Z\left(T_{1}\right)-Z\left(T_{0}\right)\right]}{\esp\left[T_{1}\right]},\textrm{ c.s. cuando  }t\rightarrow\infty.
\end{eqnarray}
\end{Coro}

%___________________________________________________________________________________________
%
%\subsection*{Propiedades de los Procesos de Renovaci\'on}
%___________________________________________________________________________________________
%

Los tiempos $T_{n}$ est\'an relacionados con los conteos de $N\left(t\right)$ por

\begin{eqnarray*}
\left\{N\left(t\right)\geq n\right\}&=&\left\{T_{n}\leq t\right\}\\
T_{N\left(t\right)}\leq &t&<T_{N\left(t\right)+1},
\end{eqnarray*}

adem\'as $N\left(T_{n}\right)=n$, y 

\begin{eqnarray*}
N\left(t\right)=\max\left\{n:T_{n}\leq t\right\}=\min\left\{n:T_{n+1}>t\right\}
\end{eqnarray*}

Por propiedades de la convoluci\'on se sabe que

\begin{eqnarray*}
P\left\{T_{n}\leq t\right\}=F^{n\star}\left(t\right)
\end{eqnarray*}
que es la $n$-\'esima convoluci\'on de $F$. Entonces 

\begin{eqnarray*}
\left\{N\left(t\right)\geq n\right\}&=&\left\{T_{n}\leq t\right\}\\
P\left\{N\left(t\right)\leq n\right\}&=&1-F^{\left(n+1\right)\star}\left(t\right)
\end{eqnarray*}

Adem\'as usando el hecho de que $\esp\left[N\left(t\right)\right]=\sum_{n=1}^{\infty}P\left\{N\left(t\right)\geq n\right\}$
se tiene que

\begin{eqnarray*}
\esp\left[N\left(t\right)\right]=\sum_{n=1}^{\infty}F^{n\star}\left(t\right)
\end{eqnarray*}

\begin{Prop}
Para cada $t\geq0$, la funci\'on generadora de momentos $\esp\left[e^{\alpha N\left(t\right)}\right]$ existe para alguna $\alpha$ en una vecindad del 0, y de aqu\'i que $\esp\left[N\left(t\right)^{m}\right]<\infty$, para $m\geq1$.
\end{Prop}


\begin{Note}
Si el primer tiempo de renovaci\'on $\xi_{1}$ no tiene la misma distribuci\'on que el resto de las $\xi_{n}$, para $n\geq2$, a $N\left(t\right)$ se le llama Proceso de Renovaci\'on retardado, donde si $\xi$ tiene distribuci\'on $G$, entonces el tiempo $T_{n}$ de la $n$-\'esima renovaci\'on tiene distribuci\'on $G\star F^{\left(n-1\right)\star}\left(t\right)$
\end{Note}


\begin{Teo}
Para una constante $\mu\leq\infty$ ( o variable aleatoria), las siguientes expresiones son equivalentes:

\begin{eqnarray}
lim_{n\rightarrow\infty}n^{-1}T_{n}&=&\mu,\textrm{ c.s.}\\
lim_{t\rightarrow\infty}t^{-1}N\left(t\right)&=&1/\mu,\textrm{ c.s.}
\end{eqnarray}
\end{Teo}


Es decir, $T_{n}$ satisface la Ley Fuerte de los Grandes N\'umeros s\'i y s\'olo s\'i $N\left/t\right)$ la cumple.


\begin{Coro}[Ley Fuerte de los Grandes N\'umeros para Procesos de Renovaci\'on]
Si $N\left(t\right)$ es un proceso de renovaci\'on cuyos tiempos de inter-renovaci\'on tienen media $\mu\leq\infty$, entonces
\begin{eqnarray}
t^{-1}N\left(t\right)\rightarrow 1/\mu,\textrm{ c.s. cuando }t\rightarrow\infty.
\end{eqnarray}

\end{Coro}


Considerar el proceso estoc\'astico de valores reales $\left\{Z\left(t\right):t\geq0\right\}$ en el mismo espacio de probabilidad que $N\left(t\right)$

\begin{Def}
Para el proceso $\left\{Z\left(t\right):t\geq0\right\}$ se define la fluctuaci\'on m\'axima de $Z\left(t\right)$ en el intervalo $\left(T_{n-1},T_{n}\right]$:
\begin{eqnarray*}
M_{n}=\sup_{T_{n-1}<t\leq T_{n}}|Z\left(t\right)-Z\left(T_{n-1}\right)|
\end{eqnarray*}
\end{Def}

\begin{Teo}
Sup\'ongase que $n^{-1}T_{n}\rightarrow\mu$ c.s. cuando $n\rightarrow\infty$, donde $\mu\leq\infty$ es una constante o variable aleatoria. Sea $a$ una constante o variable aleatoria que puede ser infinita cuando $\mu$ es finita, y considere las expresiones l\'imite:
\begin{eqnarray}
lim_{n\rightarrow\infty}n^{-1}Z\left(T_{n}\right)&=&a,\textrm{ c.s.}\\
lim_{t\rightarrow\infty}t^{-1}Z\left(t\right)&=&a/\mu,\textrm{ c.s.}
\end{eqnarray}
La segunda expresi\'on implica la primera. Conversamente, la primera implica la segunda si el proceso $Z\left(t\right)$ es creciente, o si $lim_{n\rightarrow\infty}n^{-1}M_{n}=0$ c.s.
\end{Teo}

\begin{Coro}
Si $N\left(t\right)$ es un proceso de renovaci\'on, y $\left(Z\left(T_{n}\right)-Z\left(T_{n-1}\right),M_{n}\right)$, para $n\geq1$, son variables aleatorias independientes e id\'enticamente distribuidas con media finita, entonces,
\begin{eqnarray}
lim_{t\rightarrow\infty}t^{-1}Z\left(t\right)\rightarrow\frac{\esp\left[Z\left(T_{1}\right)-Z\left(T_{0}\right)\right]}{\esp\left[T_{1}\right]},\textrm{ c.s. cuando  }t\rightarrow\infty.
\end{eqnarray}
\end{Coro}
%___________________________________________________________________________________________
%
%\subsection*{Propiedades de los Procesos de Renovaci\'on}
%___________________________________________________________________________________________
%

Los tiempos $T_{n}$ est\'an relacionados con los conteos de $N\left(t\right)$ por

\begin{eqnarray*}
\left\{N\left(t\right)\geq n\right\}&=&\left\{T_{n}\leq t\right\}\\
T_{N\left(t\right)}\leq &t&<T_{N\left(t\right)+1},
\end{eqnarray*}

adem\'as $N\left(T_{n}\right)=n$, y 

\begin{eqnarray*}
N\left(t\right)=\max\left\{n:T_{n}\leq t\right\}=\min\left\{n:T_{n+1}>t\right\}
\end{eqnarray*}

Por propiedades de la convoluci\'on se sabe que

\begin{eqnarray*}
P\left\{T_{n}\leq t\right\}=F^{n\star}\left(t\right)
\end{eqnarray*}
que es la $n$-\'esima convoluci\'on de $F$. Entonces 

\begin{eqnarray*}
\left\{N\left(t\right)\geq n\right\}&=&\left\{T_{n}\leq t\right\}\\
P\left\{N\left(t\right)\leq n\right\}&=&1-F^{\left(n+1\right)\star}\left(t\right)
\end{eqnarray*}

Adem\'as usando el hecho de que $\esp\left[N\left(t\right)\right]=\sum_{n=1}^{\infty}P\left\{N\left(t\right)\geq n\right\}$
se tiene que

\begin{eqnarray*}
\esp\left[N\left(t\right)\right]=\sum_{n=1}^{\infty}F^{n\star}\left(t\right)
\end{eqnarray*}

\begin{Prop}
Para cada $t\geq0$, la funci\'on generadora de momentos $\esp\left[e^{\alpha N\left(t\right)}\right]$ existe para alguna $\alpha$ en una vecindad del 0, y de aqu\'i que $\esp\left[N\left(t\right)^{m}\right]<\infty$, para $m\geq1$.
\end{Prop}


\begin{Note}
Si el primer tiempo de renovaci\'on $\xi_{1}$ no tiene la misma distribuci\'on que el resto de las $\xi_{n}$, para $n\geq2$, a $N\left(t\right)$ se le llama Proceso de Renovaci\'on retardado, donde si $\xi$ tiene distribuci\'on $G$, entonces el tiempo $T_{n}$ de la $n$-\'esima renovaci\'on tiene distribuci\'on $G\star F^{\left(n-1\right)\star}\left(t\right)$
\end{Note}


\begin{Teo}
Para una constante $\mu\leq\infty$ ( o variable aleatoria), las siguientes expresiones son equivalentes:

\begin{eqnarray}
lim_{n\rightarrow\infty}n^{-1}T_{n}&=&\mu,\textrm{ c.s.}\\
lim_{t\rightarrow\infty}t^{-1}N\left(t\right)&=&1/\mu,\textrm{ c.s.}
\end{eqnarray}
\end{Teo}


Es decir, $T_{n}$ satisface la Ley Fuerte de los Grandes N\'umeros s\'i y s\'olo s\'i $N\left/t\right)$ la cumple.


\begin{Coro}[Ley Fuerte de los Grandes N\'umeros para Procesos de Renovaci\'on]
Si $N\left(t\right)$ es un proceso de renovaci\'on cuyos tiempos de inter-renovaci\'on tienen media $\mu\leq\infty$, entonces
\begin{eqnarray}
t^{-1}N\left(t\right)\rightarrow 1/\mu,\textrm{ c.s. cuando }t\rightarrow\infty.
\end{eqnarray}

\end{Coro}


Considerar el proceso estoc\'astico de valores reales $\left\{Z\left(t\right):t\geq0\right\}$ en el mismo espacio de probabilidad que $N\left(t\right)$

\begin{Def}
Para el proceso $\left\{Z\left(t\right):t\geq0\right\}$ se define la fluctuaci\'on m\'axima de $Z\left(t\right)$ en el intervalo $\left(T_{n-1},T_{n}\right]$:
\begin{eqnarray*}
M_{n}=\sup_{T_{n-1}<t\leq T_{n}}|Z\left(t\right)-Z\left(T_{n-1}\right)|
\end{eqnarray*}
\end{Def}

\begin{Teo}
Sup\'ongase que $n^{-1}T_{n}\rightarrow\mu$ c.s. cuando $n\rightarrow\infty$, donde $\mu\leq\infty$ es una constante o variable aleatoria. Sea $a$ una constante o variable aleatoria que puede ser infinita cuando $\mu$ es finita, y considere las expresiones l\'imite:
\begin{eqnarray}
lim_{n\rightarrow\infty}n^{-1}Z\left(T_{n}\right)&=&a,\textrm{ c.s.}\\
lim_{t\rightarrow\infty}t^{-1}Z\left(t\right)&=&a/\mu,\textrm{ c.s.}
\end{eqnarray}
La segunda expresi\'on implica la primera. Conversamente, la primera implica la segunda si el proceso $Z\left(t\right)$ es creciente, o si $lim_{n\rightarrow\infty}n^{-1}M_{n}=0$ c.s.
\end{Teo}

\begin{Coro}
Si $N\left(t\right)$ es un proceso de renovaci\'on, y $\left(Z\left(T_{n}\right)-Z\left(T_{n-1}\right),M_{n}\right)$, para $n\geq1$, son variables aleatorias independientes e id\'enticamente distribuidas con media finita, entonces,
\begin{eqnarray}
lim_{t\rightarrow\infty}t^{-1}Z\left(t\right)\rightarrow\frac{\esp\left[Z\left(T_{1}\right)-Z\left(T_{0}\right)\right]}{\esp\left[T_{1}\right]},\textrm{ c.s. cuando  }t\rightarrow\infty.
\end{eqnarray}
\end{Coro}


%___________________________________________________________________________________________
%
\section{Funci\'on de Renovaci\'on}
%___________________________________________________________________________________________
%


\begin{Def}
Sea $h\left(t\right)$ funci\'on de valores reales en $\rea$ acotada en intervalos finitos e igual a cero para $t<0$ La ecuaci\'on de renovaci\'on para $h\left(t\right)$ y la distribuci\'on $F$ es

\begin{eqnarray}\label{Ec.Renovacion}
H\left(t\right)=h\left(t\right)+\int_{\left[0,t\right]}H\left(t-s\right)dF\left(s\right)\textrm{,    }t\geq0,
\end{eqnarray}
donde $H\left(t\right)$ es una funci\'on de valores reales. Esto es $H=h+F\star H$. Decimos que $H\left(t\right)$ es soluci\'on de esta ecuaci\'on si satisface la ecuaci\'on, y es acotada en intervalos finitos e iguales a cero para $t<0$.
\end{Def}

\begin{Prop}
La funci\'on $U\star h\left(t\right)$ es la \'unica soluci\'on de la ecuaci\'on de renovaci\'on (\ref{Ec.Renovacion}).
\end{Prop}

\begin{Teo}[Teorema Renovaci\'on Elemental]
\begin{eqnarray*}
t^{-1}U\left(t\right)\rightarrow 1/\mu\textrm{,    cuando }t\rightarrow\infty.
\end{eqnarray*}
\end{Teo}

%___________________________________________________________________________________________
%
%\subsection*{Funci\'on de Renovaci\'on}
%___________________________________________________________________________________________
%


Sup\'ongase que $N\left(t\right)$ es un proceso de renovaci\'on con distribuci\'on $F$ con media finita $\mu$.

\begin{Def}
La funci\'on de renovaci\'on asociada con la distribuci\'on $F$, del proceso $N\left(t\right)$, es
\begin{eqnarray*}
U\left(t\right)=\sum_{n=1}^{\infty}F^{n\star}\left(t\right),\textrm{   }t\geq0,
\end{eqnarray*}
donde $F^{0\star}\left(t\right)=\indora\left(t\geq0\right)$.
\end{Def}


\begin{Prop}
Sup\'ongase que la distribuci\'on de inter-renovaci\'on $F$ tiene densidad $f$. Entonces $U\left(t\right)$ tambi\'en tiene densidad, para $t>0$, y es $U^{'}\left(t\right)=\sum_{n=0}^{\infty}f^{n\star}\left(t\right)$. Adem\'as
\begin{eqnarray*}
\prob\left\{N\left(t\right)>N\left(t-\right)\right\}=0\textrm{,   }t\geq0.
\end{eqnarray*}
\end{Prop}

\begin{Def}
La Transformada de Laplace-Stieljes de $F$ est\'a dada por

\begin{eqnarray*}
\hat{F}\left(\alpha\right)=\int_{\rea_{+}}e^{-\alpha t}dF\left(t\right)\textrm{,  }\alpha\geq0.
\end{eqnarray*}
\end{Def}

Entonces

\begin{eqnarray*}
\hat{U}\left(\alpha\right)=\sum_{n=0}^{\infty}\hat{F^{n\star}}\left(\alpha\right)=\sum_{n=0}^{\infty}\hat{F}\left(\alpha\right)^{n}=\frac{1}{1-\hat{F}\left(\alpha\right)}.
\end{eqnarray*}


\begin{Prop}
La Transformada de Laplace $\hat{U}\left(\alpha\right)$ y $\hat{F}\left(\alpha\right)$ determina una a la otra de manera \'unica por la relaci\'on $\hat{U}\left(\alpha\right)=\frac{1}{1-\hat{F}\left(\alpha\right)}$.
\end{Prop}


\begin{Note}
Un proceso de renovaci\'on $N\left(t\right)$ cuyos tiempos de inter-renovaci\'on tienen media finita, es un proceso Poisson con tasa $\lambda$ si y s\'olo s\'i $\esp\left[U\left(t\right)\right]=\lambda t$, para $t\geq0$.
\end{Note}


\begin{Teo}
Sea $N\left(t\right)$ un proceso puntual simple con puntos de localizaci\'on $T_{n}$ tal que $\eta\left(t\right)=\esp\left[N\left(\right)\right]$ es finita para cada $t$. Entonces para cualquier funci\'on $f:\rea_{+}\rightarrow\rea$,
\begin{eqnarray*}
\esp\left[\sum_{n=1}^{N\left(\right)}f\left(T_{n}\right)\right]=\int_{\left(0,t\right]}f\left(s\right)d\eta\left(s\right)\textrm{,  }t\geq0,
\end{eqnarray*}
suponiendo que la integral exista. Adem\'as si $X_{1},X_{2},\ldots$ son variables aleatorias definidas en el mismo espacio de probabilidad que el proceso $N\left(t\right)$ tal que $\esp\left[X_{n}|T_{n}=s\right]=f\left(s\right)$, independiente de $n$. Entonces
\begin{eqnarray*}
\esp\left[\sum_{n=1}^{N\left(t\right)}X_{n}\right]=\int_{\left(0,t\right]}f\left(s\right)d\eta\left(s\right)\textrm{,  }t\geq0,
\end{eqnarray*} 
suponiendo que la integral exista. 
\end{Teo}

\begin{Coro}[Identidad de Wald para Renovaciones]
Para el proceso de renovaci\'on $N\left(t\right)$,
\begin{eqnarray*}
\esp\left[T_{N\left(t\right)+1}\right]=\mu\esp\left[N\left(t\right)+1\right]\textrm{,  }t\geq0,
\end{eqnarray*}  
\end{Coro}



%___________________________________________________________________________________________
%
%\subsection*{Funci\'on de Renovaci\'on}
%___________________________________________________________________________________________
%


\begin{Def}
Sea $h\left(t\right)$ funci\'on de valores reales en $\rea$ acotada en intervalos finitos e igual a cero para $t<0$ La ecuaci\'on de renovaci\'on para $h\left(t\right)$ y la distribuci\'on $F$ es

\begin{eqnarray}%\label{Ec.Renovacion}
H\left(t\right)=h\left(t\right)+\int_{\left[0,t\right]}H\left(t-s\right)dF\left(s\right)\textrm{,    }t\geq0,
\end{eqnarray}
donde $H\left(t\right)$ es una funci\'on de valores reales. Esto es $H=h+F\star H$. Decimos que $H\left(t\right)$ es soluci\'on de esta ecuaci\'on si satisface la ecuaci\'on, y es acotada en intervalos finitos e iguales a cero para $t<0$.
\end{Def}

\begin{Prop}
La funci\'on $U\star h\left(t\right)$ es la \'unica soluci\'on de la ecuaci\'on de renovaci\'on (\ref{Ec.Renovacion}).
\end{Prop}

\begin{Teo}[Teorema Renovaci\'on Elemental]
\begin{eqnarray*}
t^{-1}U\left(t\right)\rightarrow 1/\mu\textrm{,    cuando }t\rightarrow\infty.
\end{eqnarray*}
\end{Teo}

%___________________________________________________________________________________________
%
%\subsection*{Funci\'on de Renovaci\'on}
%___________________________________________________________________________________________
%


Sup\'ongase que $N\left(t\right)$ es un proceso de renovaci\'on con distribuci\'on $F$ con media finita $\mu$.

\begin{Def}
La funci\'on de renovaci\'on asociada con la distribuci\'on $F$, del proceso $N\left(t\right)$, es
\begin{eqnarray*}
U\left(t\right)=\sum_{n=1}^{\infty}F^{n\star}\left(t\right),\textrm{   }t\geq0,
\end{eqnarray*}
donde $F^{0\star}\left(t\right)=\indora\left(t\geq0\right)$.
\end{Def}


\begin{Prop}
Sup\'ongase que la distribuci\'on de inter-renovaci\'on $F$ tiene densidad $f$. Entonces $U\left(t\right)$ tambi\'en tiene densidad, para $t>0$, y es $U^{'}\left(t\right)=\sum_{n=0}^{\infty}f^{n\star}\left(t\right)$. Adem\'as
\begin{eqnarray*}
\prob\left\{N\left(t\right)>N\left(t-\right)\right\}=0\textrm{,   }t\geq0.
\end{eqnarray*}
\end{Prop}

\begin{Def}
La Transformada de Laplace-Stieljes de $F$ est\'a dada por

\begin{eqnarray*}
\hat{F}\left(\alpha\right)=\int_{\rea_{+}}e^{-\alpha t}dF\left(t\right)\textrm{,  }\alpha\geq0.
\end{eqnarray*}
\end{Def}

Entonces

\begin{eqnarray*}
\hat{U}\left(\alpha\right)=\sum_{n=0}^{\infty}\hat{F^{n\star}}\left(\alpha\right)=\sum_{n=0}^{\infty}\hat{F}\left(\alpha\right)^{n}=\frac{1}{1-\hat{F}\left(\alpha\right)}.
\end{eqnarray*}


\begin{Prop}
La Transformada de Laplace $\hat{U}\left(\alpha\right)$ y $\hat{F}\left(\alpha\right)$ determina una a la otra de manera \'unica por la relaci\'on $\hat{U}\left(\alpha\right)=\frac{1}{1-\hat{F}\left(\alpha\right)}$.
\end{Prop}


\begin{Note}
Un proceso de renovaci\'on $N\left(t\right)$ cuyos tiempos de inter-renovaci\'on tienen media finita, es un proceso Poisson con tasa $\lambda$ si y s\'olo s\'i $\esp\left[U\left(t\right)\right]=\lambda t$, para $t\geq0$.
\end{Note}


\begin{Teo}
Sea $N\left(t\right)$ un proceso puntual simple con puntos de localizaci\'on $T_{n}$ tal que $\eta\left(t\right)=\esp\left[N\left(\right)\right]$ es finita para cada $t$. Entonces para cualquier funci\'on $f:\rea_{+}\rightarrow\rea$,
\begin{eqnarray*}
\esp\left[\sum_{n=1}^{N\left(\right)}f\left(T_{n}\right)\right]=\int_{\left(0,t\right]}f\left(s\right)d\eta\left(s\right)\textrm{,  }t\geq0,
\end{eqnarray*}
suponiendo que la integral exista. Adem\'as si $X_{1},X_{2},\ldots$ son variables aleatorias definidas en el mismo espacio de probabilidad que el proceso $N\left(t\right)$ tal que $\esp\left[X_{n}|T_{n}=s\right]=f\left(s\right)$, independiente de $n$. Entonces
\begin{eqnarray*}
\esp\left[\sum_{n=1}^{N\left(t\right)}X_{n}\right]=\int_{\left(0,t\right]}f\left(s\right)d\eta\left(s\right)\textrm{,  }t\geq0,
\end{eqnarray*} 
suponiendo que la integral exista. 
\end{Teo}

\begin{Coro}[Identidad de Wald para Renovaciones]
Para el proceso de renovaci\'on $N\left(t\right)$,
\begin{eqnarray*}
\esp\left[T_{N\left(t\right)+1}\right]=\mu\esp\left[N\left(t\right)+1\right]\textrm{,  }t\geq0,
\end{eqnarray*}  
\end{Coro}


%___________________________________________________________________________________________
%
%\subsection*{Funci\'on de Renovaci\'on}
%___________________________________________________________________________________________
%


\begin{Def}
Sea $h\left(t\right)$ funci\'on de valores reales en $\rea$ acotada en intervalos finitos e igual a cero para $t<0$ La ecuaci\'on de renovaci\'on para $h\left(t\right)$ y la distribuci\'on $F$ es

\begin{eqnarray}\label{Ec.Renovacion}
H\left(t\right)=h\left(t\right)+\int_{\left[0,t\right]}H\left(t-s\right)dF\left(s\right)\textrm{,    }t\geq0,
\end{eqnarray}
donde $H\left(t\right)$ es una funci\'on de valores reales. Esto es $H=h+F\star H$. Decimos que $H\left(t\right)$ es soluci\'on de esta ecuaci\'on si satisface la ecuaci\'on, y es acotada en intervalos finitos e iguales a cero para $t<0$.
\end{Def}

\begin{Prop}
La funci\'on $U\star h\left(t\right)$ es la \'unica soluci\'on de la ecuaci\'on de renovaci\'on (\ref{Ec.Renovacion}).
\end{Prop}

\begin{Teo}[Teorema Renovaci\'on Elemental]
\begin{eqnarray*}
t^{-1}U\left(t\right)\rightarrow 1/\mu\textrm{,    cuando }t\rightarrow\infty.
\end{eqnarray*}
\end{Teo}

%___________________________________________________________________________________________
%
%\subsection*{Funci\'on de Renovaci\'on}
%___________________________________________________________________________________________
%


Sup\'ongase que $N\left(t\right)$ es un proceso de renovaci\'on con distribuci\'on $F$ con media finita $\mu$.

\begin{Def}
La funci\'on de renovaci\'on asociada con la distribuci\'on $F$, del proceso $N\left(t\right)$, es
\begin{eqnarray*}
U\left(t\right)=\sum_{n=1}^{\infty}F^{n\star}\left(t\right),\textrm{   }t\geq0,
\end{eqnarray*}
donde $F^{0\star}\left(t\right)=\indora\left(t\geq0\right)$.
\end{Def}


\begin{Prop}
Sup\'ongase que la distribuci\'on de inter-renovaci\'on $F$ tiene densidad $f$. Entonces $U\left(t\right)$ tambi\'en tiene densidad, para $t>0$, y es $U^{'}\left(t\right)=\sum_{n=0}^{\infty}f^{n\star}\left(t\right)$. Adem\'as
\begin{eqnarray*}
\prob\left\{N\left(t\right)>N\left(t-\right)\right\}=0\textrm{,   }t\geq0.
\end{eqnarray*}
\end{Prop}

\begin{Def}
La Transformada de Laplace-Stieljes de $F$ est\'a dada por

\begin{eqnarray*}
\hat{F}\left(\alpha\right)=\int_{\rea_{+}}e^{-\alpha t}dF\left(t\right)\textrm{,  }\alpha\geq0.
\end{eqnarray*}
\end{Def}

Entonces

\begin{eqnarray*}
\hat{U}\left(\alpha\right)=\sum_{n=0}^{\infty}\hat{F^{n\star}}\left(\alpha\right)=\sum_{n=0}^{\infty}\hat{F}\left(\alpha\right)^{n}=\frac{1}{1-\hat{F}\left(\alpha\right)}.
\end{eqnarray*}


\begin{Prop}
La Transformada de Laplace $\hat{U}\left(\alpha\right)$ y $\hat{F}\left(\alpha\right)$ determina una a la otra de manera \'unica por la relaci\'on $\hat{U}\left(\alpha\right)=\frac{1}{1-\hat{F}\left(\alpha\right)}$.
\end{Prop}


\begin{Note}
Un proceso de renovaci\'on $N\left(t\right)$ cuyos tiempos de inter-renovaci\'on tienen media finita, es un proceso Poisson con tasa $\lambda$ si y s\'olo s\'i $\esp\left[U\left(t\right)\right]=\lambda t$, para $t\geq0$.
\end{Note}


\begin{Teo}
Sea $N\left(t\right)$ un proceso puntual simple con puntos de localizaci\'on $T_{n}$ tal que $\eta\left(t\right)=\esp\left[N\left(\right)\right]$ es finita para cada $t$. Entonces para cualquier funci\'on $f:\rea_{+}\rightarrow\rea$,
\begin{eqnarray*}
\esp\left[\sum_{n=1}^{N\left(\right)}f\left(T_{n}\right)\right]=\int_{\left(0,t\right]}f\left(s\right)d\eta\left(s\right)\textrm{,  }t\geq0,
\end{eqnarray*}
suponiendo que la integral exista. Adem\'as si $X_{1},X_{2},\ldots$ son variables aleatorias definidas en el mismo espacio de probabilidad que el proceso $N\left(t\right)$ tal que $\esp\left[X_{n}|T_{n}=s\right]=f\left(s\right)$, independiente de $n$. Entonces
\begin{eqnarray*}
\esp\left[\sum_{n=1}^{N\left(t\right)}X_{n}\right]=\int_{\left(0,t\right]}f\left(s\right)d\eta\left(s\right)\textrm{,  }t\geq0,
\end{eqnarray*} 
suponiendo que la integral exista. 
\end{Teo}

\begin{Coro}[Identidad de Wald para Renovaciones]
Para el proceso de renovaci\'on $N\left(t\right)$,
\begin{eqnarray*}
\esp\left[T_{N\left(t\right)+1}\right]=\mu\esp\left[N\left(t\right)+1\right]\textrm{,  }t\geq0,
\end{eqnarray*}  
\end{Coro}

%______________________________________________________________________
\section{Procesos de Renovaci\'on}
%______________________________________________________________________

\begin{Def}\label{Def.Tn}
Sean $0\leq T_{1}\leq T_{2}\leq \ldots$ son tiempos aleatorios infinitos en los cuales ocurren ciertos eventos. El n\'umero de tiempos $T_{n}$ en el intervalo $\left[0,t\right)$ es

\begin{eqnarray}
N\left(t\right)=\sum_{n=1}^{\infty}\indora\left(T_{n}\leq t\right),
\end{eqnarray}
para $t\geq0$.
\end{Def}

Si se consideran los puntos $T_{n}$ como elementos de $\rea_{+}$, y $N\left(t\right)$ es el n\'umero de puntos en $\rea$. El proceso denotado por $\left\{N\left(t\right):t\geq0\right\}$, denotado por $N\left(t\right)$, es un proceso puntual en $\rea_{+}$. Los $T_{n}$ son los tiempos de ocurrencia, el proceso puntual $N\left(t\right)$ es simple si su n\'umero de ocurrencias son distintas: $0<T_{1}<T_{2}<\ldots$ casi seguramente.

\begin{Def}
Un proceso puntual $N\left(t\right)$ es un proceso de renovaci\'on si los tiempos de interocurrencia $\xi_{n}=T_{n}-T_{n-1}$, para $n\geq1$, son independientes e identicamente distribuidos con distribuci\'on $F$, donde $F\left(0\right)=0$ y $T_{0}=0$. Los $T_{n}$ son llamados tiempos de renovaci\'on, referente a la independencia o renovaci\'on de la informaci\'on estoc\'astica en estos tiempos. Los $\xi_{n}$ son los tiempos de inter-renovaci\'on, y $N\left(t\right)$ es el n\'umero de renovaciones en el intervalo $\left[0,t\right)$
\end{Def}


\begin{Note}
Para definir un proceso de renovaci\'on para cualquier contexto, solamente hay que especificar una distribuci\'on $F$, con $F\left(0\right)=0$, para los tiempos de inter-renovaci\'on. La funci\'on $F$ en turno degune las otra variables aleatorias. De manera formal, existe un espacio de probabilidad y una sucesi\'on de variables aleatorias $\xi_{1},\xi_{2},\ldots$ definidas en este con distribuci\'on $F$. Entonces las otras cantidades son $T_{n}=\sum_{k=1}^{n}\xi_{k}$ y $N\left(t\right)=\sum_{n=1}^{\infty}\indora\left(T_{n}\leq t\right)$, donde $T_{n}\rightarrow\infty$ casi seguramente por la Ley Fuerte de los Grandes Números.
\end{Note}


\begin{Ejem}[\textbf{Proceso Poisson}]

Suponga que se tienen tiempos de inter-renovaci\'on \textit{i.i.d.} del proceso de renovaci\'on $N\left(t\right)$ tienen distribuci\'on exponencial $F\left(t\right)=q-e^{-\lambda t}$ con tasa $\lambda$. Entonces $N\left(t\right)$ es un proceso Poisson con tasa $\lambda$.

\end{Ejem}


\begin{Note}
Si el primer tiempo de renovaci\'on $\xi_{1}$ no tiene la misma distribuci\'on que el resto de las $\xi_{n}$, para $n\geq2$, a $N\left(t\right)$ se le llama Proceso de Renovaci\'on retardado, donde si $\xi$ tiene distribuci\'on $G$, entonces el tiempo $T_{n}$ de la $n$-\'esima renovaci\'on tiene distribuci\'on $G\star F^{\left(n-1\right)\star}\left(t\right)$
\end{Note}

\begin{Note} Una funci\'on $h:\rea_{+}\rightarrow\rea$ es Directamente Riemann Integrable en los siguientes casos:
\begin{itemize}
\item[a)] $h\left(t\right)\geq0$ es decreciente y Riemann Integrable.
\item[b)] $h$ es continua excepto posiblemente en un conjunto de Lebesgue de medida 0, y $|h\left(t\right)|\leq b\left(t\right)$, donde $b$ es DRI.
\end{itemize}
\end{Note}

\begin{Teo}[Teorema Principal de Renovaci\'on]
Si $F$ es no aritm\'etica y $h\left(t\right)$ es Directamente Riemann Integrable (DRI), entonces

\begin{eqnarray*}
lim_{t\rightarrow\infty}U\star h=\frac{1}{\mu}\int_{\rea_{+}}h\left(s\right)ds.
\end{eqnarray*}
\end{Teo}

\begin{Prop}
Cualquier funci\'on $H\left(t\right)$ acotada en intervalos finitos y que es 0 para $t<0$ puede expresarse como
\begin{eqnarray*}
H\left(t\right)=U\star h\left(t\right)\textrm{,  donde }h\left(t\right)=H\left(t\right)-F\star H\left(t\right)
\end{eqnarray*}
\end{Prop}

\begin{Def}
Un proceso estoc\'astico $X\left(t\right)$ es crudamente regenerativo en un tiempo aleatorio positivo $T$ si
\begin{eqnarray*}
\esp\left[X\left(T+t\right)|T\right]=\esp\left[X\left(t\right)\right]\textrm{, para }t\geq0,\end{eqnarray*}
y con las esperanzas anteriores finitas.
\end{Def}

\begin{Prop}
Sup\'ongase que $X\left(t\right)$ es un proceso crudamente regenerativo en $T$, que tiene distribuci\'on $F$. Si $\esp\left[X\left(t\right)\right]$ es acotado en intervalos finitos, entonces
\begin{eqnarray*}
\esp\left[X\left(t\right)\right]=U\star h\left(t\right)\textrm{,  donde }h\left(t\right)=\esp\left[X\left(t\right)\indora\left(T>t\right)\right].
\end{eqnarray*}
\end{Prop}

\begin{Teo}[Regeneraci\'on Cruda]
Sup\'ongase que $X\left(t\right)$ es un proceso con valores positivo crudamente regenerativo en $T$, y def\'inase $M=\sup\left\{|X\left(t\right)|:t\leq T\right\}$. Si $T$ es no aritm\'etico y $M$ y $MT$ tienen media finita, entonces
\begin{eqnarray*}
lim_{t\rightarrow\infty}\esp\left[X\left(t\right)\right]=\frac{1}{\mu}\int_{\rea_{+}}h\left(s\right)ds,
\end{eqnarray*}
donde $h\left(t\right)=\esp\left[X\left(t\right)\indora\left(T>t\right)\right]$.
\end{Teo}


\begin{Note} Una funci\'on $h:\rea_{+}\rightarrow\rea$ es Directamente Riemann Integrable en los siguientes casos:
\begin{itemize}
\item[a)] $h\left(t\right)\geq0$ es decreciente y Riemann Integrable.
\item[b)] $h$ es continua excepto posiblemente en un conjunto de Lebesgue de medida 0, y $|h\left(t\right)|\leq b\left(t\right)$, donde $b$ es DRI.
\end{itemize}
\end{Note}

\begin{Teo}[Teorema Principal de Renovaci\'on]
Si $F$ es no aritm\'etica y $h\left(t\right)$ es Directamente Riemann Integrable (DRI), entonces

\begin{eqnarray*}
lim_{t\rightarrow\infty}U\star h=\frac{1}{\mu}\int_{\rea_{+}}h\left(s\right)ds.
\end{eqnarray*}
\end{Teo}

\begin{Prop}
Cualquier funci\'on $H\left(t\right)$ acotada en intervalos finitos y que es 0 para $t<0$ puede expresarse como
\begin{eqnarray*}
H\left(t\right)=U\star h\left(t\right)\textrm{,  donde }h\left(t\right)=H\left(t\right)-F\star H\left(t\right)
\end{eqnarray*}
\end{Prop}

\begin{Def}
Un proceso estoc\'astico $X\left(t\right)$ es crudamente regenerativo en un tiempo aleatorio positivo $T$ si
\begin{eqnarray*}
\esp\left[X\left(T+t\right)|T\right]=\esp\left[X\left(t\right)\right]\textrm{, para }t\geq0,\end{eqnarray*}
y con las esperanzas anteriores finitas.
\end{Def}

\begin{Prop}
Sup\'ongase que $X\left(t\right)$ es un proceso crudamente regenerativo en $T$, que tiene distribuci\'on $F$. Si $\esp\left[X\left(t\right)\right]$ es acotado en intervalos finitos, entonces
\begin{eqnarray*}
\esp\left[X\left(t\right)\right]=U\star h\left(t\right)\textrm{,  donde }h\left(t\right)=\esp\left[X\left(t\right)\indora\left(T>t\right)\right].
\end{eqnarray*}
\end{Prop}

\begin{Teo}[Regeneraci\'on Cruda]
Sup\'ongase que $X\left(t\right)$ es un proceso con valores positivo crudamente regenerativo en $T$, y def\'inase $M=\sup\left\{|X\left(t\right)|:t\leq T\right\}$. Si $T$ es no aritm\'etico y $M$ y $MT$ tienen media finita, entonces
\begin{eqnarray*}
lim_{t\rightarrow\infty}\esp\left[X\left(t\right)\right]=\frac{1}{\mu}\int_{\rea_{+}}h\left(s\right)ds,
\end{eqnarray*}
donde $h\left(t\right)=\esp\left[X\left(t\right)\indora\left(T>t\right)\right]$.
\end{Teo}

\begin{Def}
Para el proceso $\left\{\left(N\left(t\right),X\left(t\right)\right):t\geq0\right\}$, sus trayectoria muestrales en el intervalo de tiempo $\left[T_{n-1},T_{n}\right)$ est\'an descritas por
\begin{eqnarray*}
\zeta_{n}=\left(\xi_{n},\left\{X\left(T_{n-1}+t\right):0\leq t<\xi_{n}\right\}\right)
\end{eqnarray*}
Este $\zeta_{n}$ es el $n$-\'esimo segmento del proceso. El proceso es regenerativo sobre los tiempos $T_{n}$ si sus segmentos $\zeta_{n}$ son independientes e id\'enticamennte distribuidos.
\end{Def}


\begin{Note}
Si $\tilde{X}\left(t\right)$ con espacio de estados $\tilde{S}$ es regenerativo sobre $T_{n}$, entonces $X\left(t\right)=f\left(\tilde{X}\left(t\right)\right)$ tambi\'en es regenerativo sobre $T_{n}$, para cualquier funci\'on $f:\tilde{S}\rightarrow S$.
\end{Note}

\begin{Note}
Los procesos regenerativos son crudamente regenerativos, pero no al rev\'es.
\end{Note}


\begin{Note}
Un proceso estoc\'astico a tiempo continuo o discreto es regenerativo si existe un proceso de renovaci\'on  tal que los segmentos del proceso entre tiempos de renovaci\'on sucesivos son i.i.d., es decir, para $\left\{X\left(t\right):t\geq0\right\}$ proceso estoc\'astico a tiempo continuo con espacio de estados $S$, espacio m\'etrico.
\end{Note}

Para $\left\{X\left(t\right):t\geq0\right\}$ Proceso Estoc\'astico a tiempo continuo con estado de espacios $S$, que es un espacio m\'etrico, con trayectorias continuas por la derecha y con l\'imites por la izquierda c.s. Sea $N\left(t\right)$ un proceso de renovaci\'on en $\rea_{+}$ definido en el mismo espacio de probabilidad que $X\left(t\right)$, con tiempos de renovaci\'on $T$ y tiempos de inter-renovaci\'on $\xi_{n}=T_{n}-T_{n-1}$, con misma distribuci\'on $F$ de media finita $\mu$.



\begin{Def}
Para el proceso $\left\{\left(N\left(t\right),X\left(t\right)\right):t\geq0\right\}$, sus trayectoria muestrales en el intervalo de tiempo $\left[T_{n-1},T_{n}\right)$ est\'an descritas por
\begin{eqnarray*}
\zeta_{n}=\left(\xi_{n},\left\{X\left(T_{n-1}+t\right):0\leq t<\xi_{n}\right\}\right)
\end{eqnarray*}
Este $\zeta_{n}$ es el $n$-\'esimo segmento del proceso. El proceso es regenerativo sobre los tiempos $T_{n}$ si sus segmentos $\zeta_{n}$ son independientes e id\'enticamennte distribuidos.
\end{Def}

\begin{Note}
Un proceso regenerativo con media de la longitud de ciclo finita es llamado positivo recurrente.
\end{Note}

\begin{Teo}[Procesos Regenerativos]
Suponga que el proceso
\end{Teo}


\begin{Def}[Renewal Process Trinity]
Para un proceso de renovaci\'on $N\left(t\right)$, los siguientes procesos proveen de informaci\'on sobre los tiempos de renovaci\'on.
\begin{itemize}
\item $A\left(t\right)=t-T_{N\left(t\right)}$, el tiempo de recurrencia hacia atr\'as al tiempo $t$, que es el tiempo desde la \'ultima renovaci\'on para $t$.

\item $B\left(t\right)=T_{N\left(t\right)+1}-t$, el tiempo de recurrencia hacia adelante al tiempo $t$, residual del tiempo de renovaci\'on, que es el tiempo para la pr\'oxima renovaci\'on despu\'es de $t$.

\item $L\left(t\right)=\xi_{N\left(t\right)+1}=A\left(t\right)+B\left(t\right)$, la longitud del intervalo de renovaci\'on que contiene a $t$.
\end{itemize}
\end{Def}

\begin{Note}
El proceso tridimensional $\left(A\left(t\right),B\left(t\right),L\left(t\right)\right)$ es regenerativo sobre $T_{n}$, y por ende cada proceso lo es. Cada proceso $A\left(t\right)$ y $B\left(t\right)$ son procesos de MArkov a tiempo continuo con trayectorias continuas por partes en el espacio de estados $\rea_{+}$. Una expresi\'on conveniente para su distribuci\'on conjunta es, para $0\leq x<t,y\geq0$
\begin{equation}\label{NoRenovacion}
P\left\{A\left(t\right)>x,B\left(t\right)>y\right\}=
P\left\{N\left(t+y\right)-N\left((t-x)\right)=0\right\}
\end{equation}
\end{Note}

\begin{Ejem}[Tiempos de recurrencia Poisson]
Si $N\left(t\right)$ es un proceso Poisson con tasa $\lambda$, entonces de la expresi\'on (\ref{NoRenovacion}) se tiene que

\begin{eqnarray*}
\begin{array}{lc}
P\left\{A\left(t\right)>x,B\left(t\right)>y\right\}=e^{-\lambda\left(x+y\right)},&0\leq x<t,y\geq0,
\end{array}
\end{eqnarray*}
que es la probabilidad Poisson de no renovaciones en un intervalo de longitud $x+y$.

\end{Ejem}

\begin{Note}
Una cadena de Markov erg\'odica tiene la propiedad de ser estacionaria si la distribución de su estado al tiempo $0$ es su distribuci\'on estacionaria.
\end{Note}


\begin{Def}
Un proceso estoc\'astico a tiempo continuo $\left\{X\left(t\right):t\geq0\right\}$ en un espacio general es estacionario si sus distribuciones finito dimensionales son invariantes bajo cualquier  traslado: para cada $0\leq s_{1}<s_{2}<\cdots<s_{k}$ y $t\geq0$,
\begin{eqnarray*}
\left(X\left(s_{1}+t\right),\ldots,X\left(s_{k}+t\right)\right)=_{d}\left(X\left(s_{1}\right),\ldots,X\left(s_{k}\right)\right).
\end{eqnarray*}
\end{Def}

\begin{Note}
Un proceso de Markov es estacionario si $X\left(t\right)=_{d}X\left(0\right)$, $t\geq0$.
\end{Note}

Considerese el proceso $N\left(t\right)=\sum_{n}\indora\left(\tau_{n}\leq t\right)$ en $\rea_{+}$, con puntos $0<\tau_{1}<\tau_{2}<\cdots$.

\begin{Prop}
Si $N$ es un proceso puntual estacionario y $\esp\left[N\left(1\right)\right]<\infty$, entonces $\esp\left[N\left(t\right)\right]=t\esp\left[N\left(1\right)\right]$, $t\geq0$

\end{Prop}

\begin{Teo}
Los siguientes enunciados son equivalentes
\begin{itemize}
\item[i)] El proceso retardado de renovaci\'on $N$ es estacionario.

\item[ii)] EL proceso de tiempos de recurrencia hacia adelante $B\left(t\right)$ es estacionario.


\item[iii)] $\esp\left[N\left(t\right)\right]=t/\mu$,


\item[iv)] $G\left(t\right)=F_{e}\left(t\right)=\frac{1}{\mu}\int_{0}^{t}\left[1-F\left(s\right)\right]ds$
\end{itemize}
Cuando estos enunciados son ciertos, $P\left\{B\left(t\right)\leq x\right\}=F_{e}\left(x\right)$, para $t,x\geq0$.

\end{Teo}

\begin{Note}
Una consecuencia del teorema anterior es que el Proceso Poisson es el \'unico proceso sin retardo que es estacionario.
\end{Note}

\begin{Coro}
El proceso de renovaci\'on $N\left(t\right)$ sin retardo, y cuyos tiempos de inter renonaci\'on tienen media finita, es estacionario si y s\'olo si es un proceso Poisson.

\end{Coro}

%______________________________________________________________________
%\subsection*{Procesos de Renovaci\'on}
%______________________________________________________________________

\begin{Def}\label{Def.Tn}
Sean $0\leq T_{1}\leq T_{2}\leq \ldots$ son tiempos aleatorios infinitos en los cuales ocurren ciertos eventos. El n\'umero de tiempos $T_{n}$ en el intervalo $\left[0,t\right)$ es

\begin{eqnarray}
N\left(t\right)=\sum_{n=1}^{\infty}\indora\left(T_{n}\leq t\right),
\end{eqnarray}
para $t\geq0$.
\end{Def}

Si se consideran los puntos $T_{n}$ como elementos de $\rea_{+}$, y $N\left(t\right)$ es el n\'umero de puntos en $\rea$. El proceso denotado por $\left\{N\left(t\right):t\geq0\right\}$, denotado por $N\left(t\right)$, es un proceso puntual en $\rea_{+}$. Los $T_{n}$ son los tiempos de ocurrencia, el proceso puntual $N\left(t\right)$ es simple si su n\'umero de ocurrencias son distintas: $0<T_{1}<T_{2}<\ldots$ casi seguramente.

\begin{Def}
Un proceso puntual $N\left(t\right)$ es un proceso de renovaci\'on si los tiempos de interocurrencia $\xi_{n}=T_{n}-T_{n-1}$, para $n\geq1$, son independientes e identicamente distribuidos con distribuci\'on $F$, donde $F\left(0\right)=0$ y $T_{0}=0$. Los $T_{n}$ son llamados tiempos de renovaci\'on, referente a la independencia o renovaci\'on de la informaci\'on estoc\'astica en estos tiempos. Los $\xi_{n}$ son los tiempos de inter-renovaci\'on, y $N\left(t\right)$ es el n\'umero de renovaciones en el intervalo $\left[0,t\right)$
\end{Def}


\begin{Note}
Para definir un proceso de renovaci\'on para cualquier contexto, solamente hay que especificar una distribuci\'on $F$, con $F\left(0\right)=0$, para los tiempos de inter-renovaci\'on. La funci\'on $F$ en turno degune las otra variables aleatorias. De manera formal, existe un espacio de probabilidad y una sucesi\'on de variables aleatorias $\xi_{1},\xi_{2},\ldots$ definidas en este con distribuci\'on $F$. Entonces las otras cantidades son $T_{n}=\sum_{k=1}^{n}\xi_{k}$ y $N\left(t\right)=\sum_{n=1}^{\infty}\indora\left(T_{n}\leq t\right)$, donde $T_{n}\rightarrow\infty$ casi seguramente por la Ley Fuerte de los Grandes Números.
\end{Note}





%______________________________________________________________________
%\subsection*{Procesos de Renovaci\'on}
%______________________________________________________________________

\begin{Def}%\label{Def.Tn}
Sean $0\leq T_{1}\leq T_{2}\leq \ldots$ son tiempos aleatorios infinitos en los cuales ocurren ciertos eventos. El n\'umero de tiempos $T_{n}$ en el intervalo $\left[0,t\right)$ es

\begin{eqnarray}
N\left(t\right)=\sum_{n=1}^{\infty}\indora\left(T_{n}\leq t\right),
\end{eqnarray}
para $t\geq0$.
\end{Def}

Si se consideran los puntos $T_{n}$ como elementos de $\rea_{+}$, y $N\left(t\right)$ es el n\'umero de puntos en $\rea$. El proceso denotado por $\left\{N\left(t\right):t\geq0\right\}$, denotado por $N\left(t\right)$, es un proceso puntual en $\rea_{+}$. Los $T_{n}$ son los tiempos de ocurrencia, el proceso puntual $N\left(t\right)$ es simple si su n\'umero de ocurrencias son distintas: $0<T_{1}<T_{2}<\ldots$ casi seguramente.

\begin{Def}
Un proceso puntual $N\left(t\right)$ es un proceso de renovaci\'on si los tiempos de interocurrencia $\xi_{n}=T_{n}-T_{n-1}$, para $n\geq1$, son independientes e identicamente distribuidos con distribuci\'on $F$, donde $F\left(0\right)=0$ y $T_{0}=0$. Los $T_{n}$ son llamados tiempos de renovaci\'on, referente a la independencia o renovaci\'on de la informaci\'on estoc\'astica en estos tiempos. Los $\xi_{n}$ son los tiempos de inter-renovaci\'on, y $N\left(t\right)$ es el n\'umero de renovaciones en el intervalo $\left[0,t\right)$
\end{Def}


\begin{Note}
Para definir un proceso de renovaci\'on para cualquier contexto, solamente hay que especificar una distribuci\'on $F$, con $F\left(0\right)=0$, para los tiempos de inter-renovaci\'on. La funci\'on $F$ en turno degune las otra variables aleatorias. De manera formal, existe un espacio de probabilidad y una sucesi\'on de variables aleatorias $\xi_{1},\xi_{2},\ldots$ definidas en este con distribuci\'on $F$. Entonces las otras cantidades son $T_{n}=\sum_{k=1}^{n}\xi_{k}$ y $N\left(t\right)=\sum_{n=1}^{\infty}\indora\left(T_{n}\leq t\right)$, donde $T_{n}\rightarrow\infty$ casi seguramente por la Ley Fuerte de los Grandes Números.
\end{Note}


%______________________________________________________________________
%\subsection*{Procesos de Renovaci\'on}
%______________________________________________________________________

\begin{Def}%\label{Def.Tn}
Sean $0\leq T_{1}\leq T_{2}\leq \ldots$ son tiempos aleatorios infinitos en los cuales ocurren ciertos eventos. El n\'umero de tiempos $T_{n}$ en el intervalo $\left[0,t\right)$ es

\begin{eqnarray}
N\left(t\right)=\sum_{n=1}^{\infty}\indora\left(T_{n}\leq t\right),
\end{eqnarray}
para $t\geq0$.
\end{Def}

Si se consideran los puntos $T_{n}$ como elementos de $\rea_{+}$, y $N\left(t\right)$ es el n\'umero de puntos en $\rea$. El proceso denotado por $\left\{N\left(t\right):t\geq0\right\}$, denotado por $N\left(t\right)$, es un proceso puntual en $\rea_{+}$. Los $T_{n}$ son los tiempos de ocurrencia, el proceso puntual $N\left(t\right)$ es simple si su n\'umero de ocurrencias son distintas: $0<T_{1}<T_{2}<\ldots$ casi seguramente.

\begin{Def}
Un proceso puntual $N\left(t\right)$ es un proceso de renovaci\'on si los tiempos de interocurrencia $\xi_{n}=T_{n}-T_{n-1}$, para $n\geq1$, son independientes e identicamente distribuidos con distribuci\'on $F$, donde $F\left(0\right)=0$ y $T_{0}=0$. Los $T_{n}$ son llamados tiempos de renovaci\'on, referente a la independencia o renovaci\'on de la informaci\'on estoc\'astica en estos tiempos. Los $\xi_{n}$ son los tiempos de inter-renovaci\'on, y $N\left(t\right)$ es el n\'umero de renovaciones en el intervalo $\left[0,t\right)$
\end{Def}


\begin{Note}
Para definir un proceso de renovaci\'on para cualquier contexto, solamente hay que especificar una distribuci\'on $F$, con $F\left(0\right)=0$, para los tiempos de inter-renovaci\'on. La funci\'on $F$ en turno degune las otra variables aleatorias. De manera formal, existe un espacio de probabilidad y una sucesi\'on de variables aleatorias $\xi_{1},\xi_{2},\ldots$ definidas en este con distribuci\'on $F$. Entonces las otras cantidades son $T_{n}=\sum_{k=1}^{n}\xi_{k}$ y $N\left(t\right)=\sum_{n=1}^{\infty}\indora\left(T_{n}\leq t\right)$, donde $T_{n}\rightarrow\infty$ casi seguramente por la Ley Fuerte de los Grandes Números.
\end{Note}


%______________________________________________________________________
%\subsection*{Procesos de Renovaci\'on}
%______________________________________________________________________

\begin{Def}\label{Def.Tn}
Sean $0\leq T_{1}\leq T_{2}\leq \ldots$ son tiempos aleatorios infinitos en los cuales ocurren ciertos eventos. El n\'umero de tiempos $T_{n}$ en el intervalo $\left[0,t\right)$ es

\begin{eqnarray}
N\left(t\right)=\sum_{n=1}^{\infty}\indora\left(T_{n}\leq t\right),
\end{eqnarray}
para $t\geq0$.
\end{Def}

Si se consideran los puntos $T_{n}$ como elementos de $\rea_{+}$, y $N\left(t\right)$ es el n\'umero de puntos en $\rea$. El proceso denotado por $\left\{N\left(t\right):t\geq0\right\}$, denotado por $N\left(t\right)$, es un proceso puntual en $\rea_{+}$. Los $T_{n}$ son los tiempos de ocurrencia, el proceso puntual $N\left(t\right)$ es simple si su n\'umero de ocurrencias son distintas: $0<T_{1}<T_{2}<\ldots$ casi seguramente.

\begin{Def}
Un proceso puntual $N\left(t\right)$ es un proceso de renovaci\'on si los tiempos de interocurrencia $\xi_{n}=T_{n}-T_{n-1}$, para $n\geq1$, son independientes e identicamente distribuidos con distribuci\'on $F$, donde $F\left(0\right)=0$ y $T_{0}=0$. Los $T_{n}$ son llamados tiempos de renovaci\'on, referente a la independencia o renovaci\'on de la informaci\'on estoc\'astica en estos tiempos. Los $\xi_{n}$ son los tiempos de inter-renovaci\'on, y $N\left(t\right)$ es el n\'umero de renovaciones en el intervalo $\left[0,t\right)$
\end{Def}


\begin{Note}
Para definir un proceso de renovaci\'on para cualquier contexto, solamente hay que especificar una distribuci\'on $F$, con $F\left(0\right)=0$, para los tiempos de inter-renovaci\'on. La funci\'on $F$ en turno degune las otra variables aleatorias. De manera formal, existe un espacio de probabilidad y una sucesi\'on de variables aleatorias $\xi_{1},\xi_{2},\ldots$ definidas en este con distribuci\'on $F$. Entonces las otras cantidades son $T_{n}=\sum_{k=1}^{n}\xi_{k}$ y $N\left(t\right)=\sum_{n=1}^{\infty}\indora\left(T_{n}\leq t\right)$, donde $T_{n}\rightarrow\infty$ casi seguramente por la Ley Fuerte de los Grandes Números.
\end{Note}


%______________________________________________________________________
%\subsection*{Procesos de Renovaci\'on}
%______________________________________________________________________

\begin{Def}\label{Def.Tn}
Sean $0\leq T_{1}\leq T_{2}\leq \ldots$ son tiempos aleatorios infinitos en los cuales ocurren ciertos eventos. El n\'umero de tiempos $T_{n}$ en el intervalo $\left[0,t\right)$ es

\begin{eqnarray}
N\left(t\right)=\sum_{n=1}^{\infty}\indora\left(T_{n}\leq t\right),
\end{eqnarray}
para $t\geq0$.
\end{Def}

Si se consideran los puntos $T_{n}$ como elementos de $\rea_{+}$, y $N\left(t\right)$ es el n\'umero de puntos en $\rea$. El proceso denotado por $\left\{N\left(t\right):t\geq0\right\}$, denotado por $N\left(t\right)$, es un proceso puntual en $\rea_{+}$. Los $T_{n}$ son los tiempos de ocurrencia, el proceso puntual $N\left(t\right)$ es simple si su n\'umero de ocurrencias son distintas: $0<T_{1}<T_{2}<\ldots$ casi seguramente.

\begin{Def}
Un proceso puntual $N\left(t\right)$ es un proceso de renovaci\'on si los tiempos de interocurrencia $\xi_{n}=T_{n}-T_{n-1}$, para $n\geq1$, son independientes e identicamente distribuidos con distribuci\'on $F$, donde $F\left(0\right)=0$ y $T_{0}=0$. Los $T_{n}$ son llamados tiempos de renovaci\'on, referente a la independencia o renovaci\'on de la informaci\'on estoc\'astica en estos tiempos. Los $\xi_{n}$ son los tiempos de inter-renovaci\'on, y $N\left(t\right)$ es el n\'umero de renovaciones en el intervalo $\left[0,t\right)$
\end{Def}


\begin{Note}
Para definir un proceso de renovaci\'on para cualquier contexto, solamente hay que especificar una distribuci\'on $F$, con $F\left(0\right)=0$, para los tiempos de inter-renovaci\'on. La funci\'on $F$ en turno degune las otra variables aleatorias. De manera formal, existe un espacio de probabilidad y una sucesi\'on de variables aleatorias $\xi_{1},\xi_{2},\ldots$ definidas en este con distribuci\'on $F$. Entonces las otras cantidades son $T_{n}=\sum_{k=1}^{n}\xi_{k}$ y $N\left(t\right)=\sum_{n=1}^{\infty}\indora\left(T_{n}\leq t\right)$, donde $T_{n}\rightarrow\infty$ casi seguramente por la Ley Fuerte de los Grandes Números.
\end{Note}
%___________________________________________________________________________________________
%
\section{Renewal and Regenerative Processes: Serfozo\cite{Serfozo}}\label{Appendix.E}
%___________________________________________________________________________________________
%
\begin{Def}\label{Def.Tn}
Sean $0\leq T_{1}\leq T_{2}\leq \ldots$ son tiempos aleatorios infinitos en los cuales ocurren ciertos eventos. El n\'umero de tiempos $T_{n}$ en el intervalo $\left[0,t\right)$ es

\begin{eqnarray}
N\left(t\right)=\sum_{n=1}^{\infty}\indora\left(T_{n}\leq t\right),
\end{eqnarray}
para $t\geq0$.
\end{Def}

Si se consideran los puntos $T_{n}$ como elementos de $\rea_{+}$, y $N\left(t\right)$ es el n\'umero de puntos en $\rea$. El proceso denotado por $\left\{N\left(t\right):t\geq0\right\}$, denotado por $N\left(t\right)$, es un proceso puntual en $\rea_{+}$. Los $T_{n}$ son los tiempos de ocurrencia, el proceso puntual $N\left(t\right)$ es simple si su n\'umero de ocurrencias son distintas: $0<T_{1}<T_{2}<\ldots$ casi seguramente.

\begin{Def}
Un proceso puntual $N\left(t\right)$ es un proceso de renovaci\'on si los tiempos de interocurrencia $\xi_{n}=T_{n}-T_{n-1}$, para $n\geq1$, son independientes e identicamente distribuidos con distribuci\'on $F$, donde $F\left(0\right)=0$ y $T_{0}=0$. Los $T_{n}$ son llamados tiempos de renovaci\'on, referente a la independencia o renovaci\'on de la informaci\'on estoc\'astica en estos tiempos. Los $\xi_{n}$ son los tiempos de inter-renovaci\'on, y $N\left(t\right)$ es el n\'umero de renovaciones en el intervalo $\left[0,t\right)$
\end{Def}


\begin{Note}
Para definir un proceso de renovaci\'on para cualquier contexto, solamente hay que especificar una distribuci\'on $F$, con $F\left(0\right)=0$, para los tiempos de inter-renovaci\'on. La funci\'on $F$ en turno degune las otra variables aleatorias. De manera formal, existe un espacio de probabilidad y una sucesi\'on de variables aleatorias $\xi_{1},\xi_{2},\ldots$ definidas en este con distribuci\'on $F$. Entonces las otras cantidades son $T_{n}=\sum_{k=1}^{n}\xi_{k}$ y $N\left(t\right)=\sum_{n=1}^{\infty}\indora\left(T_{n}\leq t\right)$, donde $T_{n}\rightarrow\infty$ casi seguramente por la Ley Fuerte de los Grandes N\'umeros.
\end{Note}







Los tiempos $T_{n}$ est\'an relacionados con los conteos de $N\left(t\right)$ por

\begin{eqnarray*}
\left\{N\left(t\right)\geq n\right\}&=&\left\{T_{n}\leq t\right\}\\
T_{N\left(t\right)}\leq &t&<T_{N\left(t\right)+1},
\end{eqnarray*}

adem\'as $N\left(T_{n}\right)=n$, y 

\begin{eqnarray*}
N\left(t\right)=\max\left\{n:T_{n}\leq t\right\}=\min\left\{n:T_{n+1}>t\right\}
\end{eqnarray*}

Por propiedades de la convoluci\'on se sabe que

\begin{eqnarray*}
P\left\{T_{n}\leq t\right\}=F^{n\star}\left(t\right)
\end{eqnarray*}
que es la $n$-\'esima convoluci\'on de $F$. Entonces 

\begin{eqnarray*}
\left\{N\left(t\right)\geq n\right\}&=&\left\{T_{n}\leq t\right\}\\
P\left\{N\left(t\right)\leq n\right\}&=&1-F^{\left(n+1\right)\star}\left(t\right)
\end{eqnarray*}

Adem\'as usando el hecho de que $\esp\left[N\left(t\right)\right]=\sum_{n=1}^{\infty}P\left\{N\left(t\right)\geq n\right\}$
se tiene que

\begin{eqnarray*}
\esp\left[N\left(t\right)\right]=\sum_{n=1}^{\infty}F^{n\star}\left(t\right)
\end{eqnarray*}

\begin{Prop}
Para cada $t\geq0$, la funci\'on generadora de momentos $\esp\left[e^{\alpha N\left(t\right)}\right]$ existe para alguna $\alpha$ en una vecindad del 0, y de aqu\'i que $\esp\left[N\left(t\right)^{m}\right]<\infty$, para $m\geq1$.
\end{Prop}

\begin{Ejem}[\textbf{Proceso Poisson}]

Suponga que se tienen tiempos de inter-renovaci\'on \textit{i.i.d.} del proceso de renovaci\'on $N\left(t\right)$ tienen distribuci\'on exponencial $F\left(t\right)=q-e^{-\lambda t}$ con tasa $\lambda$. Entonces $N\left(t\right)$ es un proceso Poisson con tasa $\lambda$.

\end{Ejem}


\begin{Note}
Si el primer tiempo de renovaci\'on $\xi_{1}$ no tiene la misma distribuci\'on que el resto de las $\xi_{n}$, para $n\geq2$, a $N\left(t\right)$ se le llama Proceso de Renovaci\'on retardado, donde si $\xi$ tiene distribuci\'on $G$, entonces el tiempo $T_{n}$ de la $n$-\'esima renovaci\'on tiene distribuci\'on $G\star F^{\left(n-1\right)\star}\left(t\right)$
\end{Note}


\begin{Teo}
Para una constante $\mu\leq\infty$ ( o variable aleatoria), las siguientes expresiones son equivalentes:

\begin{eqnarray}
lim_{n\rightarrow\infty}n^{-1}T_{n}&=&\mu,\textrm{ c.s.}\\
lim_{t\rightarrow\infty}t^{-1}N\left(t\right)&=&1/\mu,\textrm{ c.s.}
\end{eqnarray}
\end{Teo}


Es decir, $T_{n}$ satisface la Ley Fuerte de los Grandes N\'umeros s\'i y s\'olo s\'i $N\left/t\right)$ la cumple.


\begin{Coro}[Ley Fuerte de los Grandes N\'umeros para Procesos de Renovaci\'on]
Si $N\left(t\right)$ es un proceso de renovaci\'on cuyos tiempos de inter-renovaci\'on tienen media $\mu\leq\infty$, entonces
\begin{eqnarray}
t^{-1}N\left(t\right)\rightarrow 1/\mu,\textrm{ c.s. cuando }t\rightarrow\infty.
\end{eqnarray}

\end{Coro}


Considerar el proceso estoc\'astico de valores reales $\left\{Z\left(t\right):t\geq0\right\}$ en el mismo espacio de probabilidad que $N\left(t\right)$

\begin{Def}
Para el proceso $\left\{Z\left(t\right):t\geq0\right\}$ se define la fluctuaci\'on m\'axima de $Z\left(t\right)$ en el intervalo $\left(T_{n-1},T_{n}\right]$:
\begin{eqnarray*}
M_{n}=\sup_{T_{n-1}<t\leq T_{n}}|Z\left(t\right)-Z\left(T_{n-1}\right)|
\end{eqnarray*}
\end{Def}

\begin{Teo}
Sup\'ongase que $n^{-1}T_{n}\rightarrow\mu$ c.s. cuando $n\rightarrow\infty$, donde $\mu\leq\infty$ es una constante o variable aleatoria. Sea $a$ una constante o variable aleatoria que puede ser infinita cuando $\mu$ es finita, y considere las expresiones l\'imite:
\begin{eqnarray}
lim_{n\rightarrow\infty}n^{-1}Z\left(T_{n}\right)&=&a,\textrm{ c.s.}\\
lim_{t\rightarrow\infty}t^{-1}Z\left(t\right)&=&a/\mu,\textrm{ c.s.}
\end{eqnarray}
La segunda expresi\'on implica la primera. Conversamente, la primera implica la segunda si el proceso $Z\left(t\right)$ es creciente, o si $lim_{n\rightarrow\infty}n^{-1}M_{n}=0$ c.s.
\end{Teo}

\begin{Coro}
Si $N\left(t\right)$ es un proceso de renovaci\'on, y $\left(Z\left(T_{n}\right)-Z\left(T_{n-1}\right),M_{n}\right)$, para $n\geq1$, son variables aleatorias independientes e id\'enticamente distribuidas con media finita, entonces,
\begin{eqnarray}
lim_{t\rightarrow\infty}t^{-1}Z\left(t\right)\rightarrow\frac{\esp\left[Z\left(T_{1}\right)-Z\left(T_{0}\right)\right]}{\esp\left[T_{1}\right]},\textrm{ c.s. cuando  }t\rightarrow\infty.
\end{eqnarray}
\end{Coro}


Sup\'ongase que $N\left(t\right)$ es un proceso de renovaci\'on con distribuci\'on $F$ con media finita $\mu$.

\begin{Def}
La funci\'on de renovaci\'on asociada con la distribuci\'on $F$, del proceso $N\left(t\right)$, es
\begin{eqnarray*}
U\left(t\right)=\sum_{n=1}^{\infty}F^{n\star}\left(t\right),\textrm{   }t\geq0,
\end{eqnarray*}
donde $F^{0\star}\left(t\right)=\indora\left(t\geq0\right)$.
\end{Def}


\begin{Prop}
Sup\'ongase que la distribuci\'on de inter-renovaci\'on $F$ tiene densidad $f$. Entonces $U\left(t\right)$ tambi\'en tiene densidad, para $t>0$, y es $U^{'}\left(t\right)=\sum_{n=0}^{\infty}f^{n\star}\left(t\right)$. Adem\'as
\begin{eqnarray*}
\prob\left\{N\left(t\right)>N\left(t-\right)\right\}=0\textrm{,   }t\geq0.
\end{eqnarray*}
\end{Prop}

\begin{Def}
La Transformada de Laplace-Stieljes de $F$ est\'a dada por

\begin{eqnarray*}
\hat{F}\left(\alpha\right)=\int_{\rea_{+}}e^{-\alpha t}dF\left(t\right)\textrm{,  }\alpha\geq0.
\end{eqnarray*}
\end{Def}

Entonces

\begin{eqnarray*}
\hat{U}\left(\alpha\right)=\sum_{n=0}^{\infty}\hat{F^{n\star}}\left(\alpha\right)=\sum_{n=0}^{\infty}\hat{F}\left(\alpha\right)^{n}=\frac{1}{1-\hat{F}\left(\alpha\right)}.
\end{eqnarray*}


\begin{Prop}
La Transformada de Laplace $\hat{U}\left(\alpha\right)$ y $\hat{F}\left(\alpha\right)$ determina una a la otra de manera \'unica por la relaci\'on $\hat{U}\left(\alpha\right)=\frac{1}{1-\hat{F}\left(\alpha\right)}$.
\end{Prop}


\begin{Note}
Un proceso de renovaci\'on $N\left(t\right)$ cuyos tiempos de inter-renovaci\'on tienen media finita, es un proceso Poisson con tasa $\lambda$ si y s\'olo s\'i $\esp\left[U\left(t\right)\right]=\lambda t$, para $t\geq0$.
\end{Note}


\begin{Teo}
Sea $N\left(t\right)$ un proceso puntual simple con puntos de localizaci\'on $T_{n}$ tal que $\eta\left(t\right)=\esp\left[N\left(\right)\right]$ es finita para cada $t$. Entonces para cualquier funci\'on $f:\rea_{+}\rightarrow\rea$,
\begin{eqnarray*}
\esp\left[\sum_{n=1}^{N\left(\right)}f\left(T_{n}\right)\right]=\int_{\left(0,t\right]}f\left(s\right)d\eta\left(s\right)\textrm{,  }t\geq0,
\end{eqnarray*}
suponiendo que la integral exista. Adem\'as si $X_{1},X_{2},\ldots$ son variables aleatorias definidas en el mismo espacio de probabilidad que el proceso $N\left(t\right)$ tal que $\esp\left[X_{n}|T_{n}=s\right]=f\left(s\right)$, independiente de $n$. Entonces
\begin{eqnarray*}
\esp\left[\sum_{n=1}^{N\left(t\right)}X_{n}\right]=\int_{\left(0,t\right]}f\left(s\right)d\eta\left(s\right)\textrm{,  }t\geq0,
\end{eqnarray*} 
suponiendo que la integral exista. 
\end{Teo}

\begin{Coro}[Identidad de Wald para Renovaciones]
Para el proceso de renovaci\'on $N\left(t\right)$,
\begin{eqnarray*}
\esp\left[T_{N\left(t\right)+1}\right]=\mu\esp\left[N\left(t\right)+1\right]\textrm{,  }t\geq0,
\end{eqnarray*}  
\end{Coro}


\begin{Def}
Sea $h\left(t\right)$ funci\'on de valores reales en $\rea$ acotada en intervalos finitos e igual a cero para $t<0$ La ecuaci\'on de renovaci\'on para $h\left(t\right)$ y la distribuci\'on $F$ es

\begin{eqnarray}\label{Ec.Renovacion}
H\left(t\right)=h\left(t\right)+\int_{\left[0,t\right]}H\left(t-s\right)dF\left(s\right)\textrm{,    }t\geq0,
\end{eqnarray}
donde $H\left(t\right)$ es una funci\'on de valores reales. Esto es $H=h+F\star H$. Decimos que $H\left(t\right)$ es soluci\'on de esta ecuaci\'on si satisface la ecuaci\'on, y es acotada en intervalos finitos e iguales a cero para $t<0$.
\end{Def}

\begin{Prop}
La funci\'on $U\star h\left(t\right)$ es la \'unica soluci\'on de la ecuaci\'on de renovaci\'on (\ref{Ec.Renovacion}).
\end{Prop}

\begin{Teo}[Teorema Renovaci\'on Elemental]
\begin{eqnarray*}
t^{-1}U\left(t\right)\rightarrow 1/\mu\textrm{,    cuando }t\rightarrow\infty.
\end{eqnarray*}
\end{Teo}



Sup\'ongase que $N\left(t\right)$ es un proceso de renovaci\'on con distribuci\'on $F$ con media finita $\mu$.

\begin{Def}
La funci\'on de renovaci\'on asociada con la distribuci\'on $F$, del proceso $N\left(t\right)$, es
\begin{eqnarray*}
U\left(t\right)=\sum_{n=1}^{\infty}F^{n\star}\left(t\right),\textrm{   }t\geq0,
\end{eqnarray*}
donde $F^{0\star}\left(t\right)=\indora\left(t\geq0\right)$.
\end{Def}


\begin{Prop}
Sup\'ongase que la distribuci\'on de inter-renovaci\'on $F$ tiene densidad $f$. Entonces $U\left(t\right)$ tambi\'en tiene densidad, para $t>0$, y es $U^{'}\left(t\right)=\sum_{n=0}^{\infty}f^{n\star}\left(t\right)$. Adem\'as
\begin{eqnarray*}
\prob\left\{N\left(t\right)>N\left(t-\right)\right\}=0\textrm{,   }t\geq0.
\end{eqnarray*}
\end{Prop}

\begin{Def}
La Transformada de Laplace-Stieljes de $F$ est\'a dada por

\begin{eqnarray*}
\hat{F}\left(\alpha\right)=\int_{\rea_{+}}e^{-\alpha t}dF\left(t\right)\textrm{,  }\alpha\geq0.
\end{eqnarray*}
\end{Def}

Entonces

\begin{eqnarray*}
\hat{U}\left(\alpha\right)=\sum_{n=0}^{\infty}\hat{F^{n\star}}\left(\alpha\right)=\sum_{n=0}^{\infty}\hat{F}\left(\alpha\right)^{n}=\frac{1}{1-\hat{F}\left(\alpha\right)}.
\end{eqnarray*}


\begin{Prop}
La Transformada de Laplace $\hat{U}\left(\alpha\right)$ y $\hat{F}\left(\alpha\right)$ determina una a la otra de manera \'unica por la relaci\'on $\hat{U}\left(\alpha\right)=\frac{1}{1-\hat{F}\left(\alpha\right)}$.
\end{Prop}


\begin{Note}
Un proceso de renovaci\'on $N\left(t\right)$ cuyos tiempos de inter-renovaci\'on tienen media finita, es un proceso Poisson con tasa $\lambda$ si y s\'olo s\'i $\esp\left[U\left(t\right)\right]=\lambda t$, para $t\geq0$.
\end{Note}


\begin{Teo}
Sea $N\left(t\right)$ un proceso puntual simple con puntos de localizaci\'on $T_{n}$ tal que $\eta\left(t\right)=\esp\left[N\left(\right)\right]$ es finita para cada $t$. Entonces para cualquier funci\'on $f:\rea_{+}\rightarrow\rea$,
\begin{eqnarray*}
\esp\left[\sum_{n=1}^{N\left(\right)}f\left(T_{n}\right)\right]=\int_{\left(0,t\right]}f\left(s\right)d\eta\left(s\right)\textrm{,  }t\geq0,
\end{eqnarray*}
suponiendo que la integral exista. Adem\'as si $X_{1},X_{2},\ldots$ son variables aleatorias definidas en el mismo espacio de probabilidad que el proceso $N\left(t\right)$ tal que $\esp\left[X_{n}|T_{n}=s\right]=f\left(s\right)$, independiente de $n$. Entonces
\begin{eqnarray*}
\esp\left[\sum_{n=1}^{N\left(t\right)}X_{n}\right]=\int_{\left(0,t\right]}f\left(s\right)d\eta\left(s\right)\textrm{,  }t\geq0,
\end{eqnarray*} 
suponiendo que la integral exista. 
\end{Teo}

\begin{Coro}[Identidad de Wald para Renovaciones]
Para el proceso de renovaci\'on $N\left(t\right)$,
\begin{eqnarray*}
\esp\left[T_{N\left(t\right)+1}\right]=\mu\esp\left[N\left(t\right)+1\right]\textrm{,  }t\geq0,
\end{eqnarray*}  
\end{Coro}


\begin{Def}
Sea $h\left(t\right)$ funci\'on de valores reales en $\rea$ acotada en intervalos finitos e igual a cero para $t<0$ La ecuaci\'on de renovaci\'on para $h\left(t\right)$ y la distribuci\'on $F$ es

\begin{eqnarray}\label{Ec.Renovacion}
H\left(t\right)=h\left(t\right)+\int_{\left[0,t\right]}H\left(t-s\right)dF\left(s\right)\textrm{,    }t\geq0,
\end{eqnarray}
donde $H\left(t\right)$ es una funci\'on de valores reales. Esto es $H=h+F\star H$. Decimos que $H\left(t\right)$ es soluci\'on de esta ecuaci\'on si satisface la ecuaci\'on, y es acotada en intervalos finitos e iguales a cero para $t<0$.
\end{Def}

\begin{Prop}
La funci\'on $U\star h\left(t\right)$ es la \'unica soluci\'on de la ecuaci\'on de renovaci\'on (\ref{Ec.Renovacion}).
\end{Prop}

\begin{Teo}[Teorema Renovaci\'on Elemental]
\begin{eqnarray*}
t^{-1}U\left(t\right)\rightarrow 1/\mu\textrm{,    cuando }t\rightarrow\infty.
\end{eqnarray*}
\end{Teo}


\begin{Note} Una funci\'on $h:\rea_{+}\rightarrow\rea$ es Directamente Riemann Integrable en los siguientes casos:
\begin{itemize}
\item[a)] $h\left(t\right)\geq0$ es decreciente y Riemann Integrable.
\item[b)] $h$ es continua excepto posiblemente en un conjunto de Lebesgue de medida 0, y $|h\left(t\right)|\leq b\left(t\right)$, donde $b$ es DRI.
\end{itemize}
\end{Note}

\begin{Teo}[Teorema Principal de Renovaci\'on]
Si $F$ es no aritm\'etica y $h\left(t\right)$ es Directamente Riemann Integrable (DRI), entonces

\begin{eqnarray*}
lim_{t\rightarrow\infty}U\star h=\frac{1}{\mu}\int_{\rea_{+}}h\left(s\right)ds.
\end{eqnarray*}
\end{Teo}

\begin{Prop}
Cualquier funci\'on $H\left(t\right)$ acotada en intervalos finitos y que es 0 para $t<0$ puede expresarse como
\begin{eqnarray*}
H\left(t\right)=U\star h\left(t\right)\textrm{,  donde }h\left(t\right)=H\left(t\right)-F\star H\left(t\right)
\end{eqnarray*}
\end{Prop}

\begin{Def}
Un proceso estoc\'astico $X\left(t\right)$ es crudamente regenerativo en un tiempo aleatorio positivo $T$ si
\begin{eqnarray*}
\esp\left[X\left(T+t\right)|T\right]=\esp\left[X\left(t\right)\right]\textrm{, para }t\geq0,\end{eqnarray*}
y con las esperanzas anteriores finitas.
\end{Def}

\begin{Prop}
Sup\'ongase que $X\left(t\right)$ es un proceso crudamente regenerativo en $T$, que tiene distribuci\'on $F$. Si $\esp\left[X\left(t\right)\right]$ es acotado en intervalos finitos, entonces
\begin{eqnarray*}
\esp\left[X\left(t\right)\right]=U\star h\left(t\right)\textrm{,  donde }h\left(t\right)=\esp\left[X\left(t\right)\indora\left(T>t\right)\right].
\end{eqnarray*}
\end{Prop}

\begin{Teo}[Regeneraci\'on Cruda]
Sup\'ongase que $X\left(t\right)$ es un proceso con valores positivo crudamente regenerativo en $T$, y def\'inase $M=\sup\left\{|X\left(t\right)|:t\leq T\right\}$. Si $T$ es no aritm\'etico y $M$ y $MT$ tienen media finita, entonces
\begin{eqnarray*}
lim_{t\rightarrow\infty}\esp\left[X\left(t\right)\right]=\frac{1}{\mu}\int_{\rea_{+}}h\left(s\right)ds,
\end{eqnarray*}
donde $h\left(t\right)=\esp\left[X\left(t\right)\indora\left(T>t\right)\right]$.
\end{Teo}


\begin{Note} Una funci\'on $h:\rea_{+}\rightarrow\rea$ es Directamente Riemann Integrable en los siguientes casos:
\begin{itemize}
\item[a)] $h\left(t\right)\geq0$ es decreciente y Riemann Integrable.
\item[b)] $h$ es continua excepto posiblemente en un conjunto de Lebesgue de medida 0, y $|h\left(t\right)|\leq b\left(t\right)$, donde $b$ es DRI.
\end{itemize}
\end{Note}

\begin{Teo}[Teorema Principal de Renovaci\'on]
Si $F$ es no aritm\'etica y $h\left(t\right)$ es Directamente Riemann Integrable (DRI), entonces

\begin{eqnarray*}
lim_{t\rightarrow\infty}U\star h=\frac{1}{\mu}\int_{\rea_{+}}h\left(s\right)ds.
\end{eqnarray*}
\end{Teo}

\begin{Prop}
Cualquier funci\'on $H\left(t\right)$ acotada en intervalos finitos y que es 0 para $t<0$ puede expresarse como
\begin{eqnarray*}
H\left(t\right)=U\star h\left(t\right)\textrm{,  donde }h\left(t\right)=H\left(t\right)-F\star H\left(t\right)
\end{eqnarray*}
\end{Prop}

\begin{Def}
Un proceso estoc\'astico $X\left(t\right)$ es crudamente regenerativo en un tiempo aleatorio positivo $T$ si
\begin{eqnarray*}
\esp\left[X\left(T+t\right)|T\right]=\esp\left[X\left(t\right)\right]\textrm{, para }t\geq0,\end{eqnarray*}
y con las esperanzas anteriores finitas.
\end{Def}

\begin{Prop}
Sup\'ongase que $X\left(t\right)$ es un proceso crudamente regenerativo en $T$, que tiene distribuci\'on $F$. Si $\esp\left[X\left(t\right)\right]$ es acotado en intervalos finitos, entonces
\begin{eqnarray*}
\esp\left[X\left(t\right)\right]=U\star h\left(t\right)\textrm{,  donde }h\left(t\right)=\esp\left[X\left(t\right)\indora\left(T>t\right)\right].
\end{eqnarray*}
\end{Prop}

\begin{Teo}[Regeneraci\'on Cruda]
Sup\'ongase que $X\left(t\right)$ es un proceso con valores positivo crudamente regenerativo en $T$, y def\'inase $M=\sup\left\{|X\left(t\right)|:t\leq T\right\}$. Si $T$ es no aritm\'etico y $M$ y $MT$ tienen media finita, entonces
\begin{eqnarray*}
lim_{t\rightarrow\infty}\esp\left[X\left(t\right)\right]=\frac{1}{\mu}\int_{\rea_{+}}h\left(s\right)ds,
\end{eqnarray*}
donde $h\left(t\right)=\esp\left[X\left(t\right)\indora\left(T>t\right)\right]$.
\end{Teo}

\begin{Def}
Para el proceso $\left\{\left(N\left(t\right),X\left(t\right)\right):t\geq0\right\}$, sus trayectoria muestrales en el intervalo de tiempo $\left[T_{n-1},T_{n}\right)$ est\'an descritas por
\begin{eqnarray*}
\zeta_{n}=\left(\xi_{n},\left\{X\left(T_{n-1}+t\right):0\leq t<\xi_{n}\right\}\right)
\end{eqnarray*}
Este $\zeta_{n}$ es el $n$-\'esimo segmento del proceso. El proceso es regenerativo sobre los tiempos $T_{n}$ si sus segmentos $\zeta_{n}$ son independientes e id\'enticamennte distribuidos.
\end{Def}


\begin{Note}
Si $\tilde{X}\left(t\right)$ con espacio de estados $\tilde{S}$ es regenerativo sobre $T_{n}$, entonces $X\left(t\right)=f\left(\tilde{X}\left(t\right)\right)$ tambi\'en es regenerativo sobre $T_{n}$, para cualquier funci\'on $f:\tilde{S}\rightarrow S$.
\end{Note}

\begin{Note}
Los procesos regenerativos son crudamente regenerativos, pero no al rev\'es.
\end{Note}


\begin{Note}
Un proceso estoc\'astico a tiempo continuo o discreto es regenerativo si existe un proceso de renovaci\'on  tal que los segmentos del proceso entre tiempos de renovaci\'on sucesivos son i.i.d., es decir, para $\left\{X\left(t\right):t\geq0\right\}$ proceso estoc\'astico a tiempo continuo con espacio de estados $S$, espacio m\'etrico.
\end{Note}

Para $\left\{X\left(t\right):t\geq0\right\}$ Proceso Estoc\'astico a tiempo continuo con estado de espacios $S$, que es un espacio m\'etrico, con trayectorias continuas por la derecha y con l\'imites por la izquierda c.s. Sea $N\left(t\right)$ un proceso de renovaci\'on en $\rea_{+}$ definido en el mismo espacio de probabilidad que $X\left(t\right)$, con tiempos de renovaci\'on $T$ y tiempos de inter-renovaci\'on $\xi_{n}=T_{n}-T_{n-1}$, con misma distribuci\'on $F$ de media finita $\mu$.



\begin{Def}
Para el proceso $\left\{\left(N\left(t\right),X\left(t\right)\right):t\geq0\right\}$, sus trayectoria muestrales en el intervalo de tiempo $\left[T_{n-1},T_{n}\right)$ est\'an descritas por
\begin{eqnarray*}
\zeta_{n}=\left(\xi_{n},\left\{X\left(T_{n-1}+t\right):0\leq t<\xi_{n}\right\}\right)
\end{eqnarray*}
Este $\zeta_{n}$ es el $n$-\'esimo segmento del proceso. El proceso es regenerativo sobre los tiempos $T_{n}$ si sus segmentos $\zeta_{n}$ son independientes e id\'enticamennte distribuidos.
\end{Def}

\begin{Note}
Un proceso regenerativo con media de la longitud de ciclo finita es llamado positivo recurrente.
\end{Note}

\begin{Teo}[Procesos Regenerativos]
Suponga que el proceso
\end{Teo}


\begin{Def}[Renewal Process Trinity]
Para un proceso de renovaci\'on $N\left(t\right)$, los siguientes procesos proveen de informaci\'on sobre los tiempos de renovaci\'on.
\begin{itemize}
\item $A\left(t\right)=t-T_{N\left(t\right)}$, el tiempo de recurrencia hacia atr\'as al tiempo $t$, que es el tiempo desde la \'ultima renovaci\'on para $t$.

\item $B\left(t\right)=T_{N\left(t\right)+1}-t$, el tiempo de recurrencia hacia adelante al tiempo $t$, residual del tiempo de renovaci\'on, que es el tiempo para la pr\'oxima renovaci\'on despu\'es de $t$.

\item $L\left(t\right)=\xi_{N\left(t\right)+1}=A\left(t\right)+B\left(t\right)$, la longitud del intervalo de renovaci\'on que contiene a $t$.
\end{itemize}
\end{Def}

\begin{Note}
El proceso tridimensional $\left(A\left(t\right),B\left(t\right),L\left(t\right)\right)$ es regenerativo sobre $T_{n}$, y por ende cada proceso lo es. Cada proceso $A\left(t\right)$ y $B\left(t\right)$ son procesos de MArkov a tiempo continuo con trayectorias continuas por partes en el espacio de estados $\rea_{+}$. Una expresi\'on conveniente para su distribuci\'on conjunta es, para $0\leq x<t,y\geq0$
\begin{equation}\label{NoRenovacion}
P\left\{A\left(t\right)>x,B\left(t\right)>y\right\}=
P\left\{N\left(t+y\right)-N\left((t-x)\right)=0\right\}
\end{equation}
\end{Note}

\begin{Ejem}[Tiempos de recurrencia Poisson]
Si $N\left(t\right)$ es un proceso Poisson con tasa $\lambda$, entonces de la expresi\'on (\ref{NoRenovacion}) se tiene que

\begin{eqnarray*}
\begin{array}{lc}
P\left\{A\left(t\right)>x,B\left(t\right)>y\right\}=e^{-\lambda\left(x+y\right)},&0\leq x<t,y\geq0,
\end{array}
\end{eqnarray*}
que es la probabilidad Poisson de no renovaciones en un intervalo de longitud $x+y$.

\end{Ejem}

\begin{Note}
Una cadena de Markov erg\'odica tiene la propiedad de ser estacionaria si la distribución de su estado al tiempo $0$ es su distribuci\'on estacionaria.
\end{Note}


\begin{Def}
Un proceso estoc\'astico a tiempo continuo $\left\{X\left(t\right):t\geq0\right\}$ en un espacio general es estacionario si sus distribuciones finito dimensionales son invariantes bajo cualquier  traslado: para cada $0\leq s_{1}<s_{2}<\cdots<s_{k}$ y $t\geq0$,
\begin{eqnarray*}
\left(X\left(s_{1}+t\right),\ldots,X\left(s_{k}+t\right)\right)=_{d}\left(X\left(s_{1}\right),\ldots,X\left(s_{k}\right)\right).
\end{eqnarray*}
\end{Def}

\begin{Note}
Un proceso de Markov es estacionario si $X\left(t\right)=_{d}X\left(0\right)$, $t\geq0$.
\end{Note}

Considerese el proceso $N\left(t\right)=\sum_{n}\indora\left(\tau_{n}\leq t\right)$ en $\rea_{+}$, con puntos $0<\tau_{1}<\tau_{2}<\cdots$.

\begin{Prop}
Si $N$ es un proceso puntual estacionario y $\esp\left[N\left(1\right)\right]<\infty$, entonces $\esp\left[N\left(t\right)\right]=t\esp\left[N\left(1\right)\right]$, $t\geq0$

\end{Prop}

\begin{Teo}
Los siguientes enunciados son equivalentes
\begin{itemize}
\item[i)] El proceso retardado de renovaci\'on $N$ es estacionario.

\item[ii)] EL proceso de tiempos de recurrencia hacia adelante $B\left(t\right)$ es estacionario.


\item[iii)] $\esp\left[N\left(t\right)\right]=t/\mu$,


\item[iv)] $G\left(t\right)=F_{e}\left(t\right)=\frac{1}{\mu}\int_{0}^{t}\left[1-F\left(s\right)\right]ds$
\end{itemize}
Cuando estos enunciados son ciertos, $P\left\{B\left(t\right)\leq x\right\}=F_{e}\left(x\right)$, para $t,x\geq0$.

\end{Teo}

\begin{Note}
Una consecuencia del teorema anterior es que el Proceso Poisson es el \'unico proceso sin retardo que es estacionario.
\end{Note}

\begin{Coro}
El proceso de renovaci\'on $N\left(t\right)$ sin retardo, y cuyos tiempos de inter renonaci\'on tienen media finita, es estacionario si y s\'olo si es un proceso Poisson.

\end{Coro}





%___________________________________________________________________________________________
%
%\subsection*{Renewal and Regenerative Processes: Serfozo\cite{Serfozo}}
%___________________________________________________________________________________________
%
\begin{Def}%\label{Def.Tn}
Sean $0\leq T_{1}\leq T_{2}\leq \ldots$ son tiempos aleatorios infinitos en los cuales ocurren ciertos eventos. El n\'umero de tiempos $T_{n}$ en el intervalo $\left[0,t\right)$ es

\begin{eqnarray}
N\left(t\right)=\sum_{n=1}^{\infty}\indora\left(T_{n}\leq t\right),
\end{eqnarray}
para $t\geq0$.
\end{Def}

Si se consideran los puntos $T_{n}$ como elementos de $\rea_{+}$, y $N\left(t\right)$ es el n\'umero de puntos en $\rea$. El proceso denotado por $\left\{N\left(t\right):t\geq0\right\}$, denotado por $N\left(t\right)$, es un proceso puntual en $\rea_{+}$. Los $T_{n}$ son los tiempos de ocurrencia, el proceso puntual $N\left(t\right)$ es simple si su n\'umero de ocurrencias son distintas: $0<T_{1}<T_{2}<\ldots$ casi seguramente.

\begin{Def}
Un proceso puntual $N\left(t\right)$ es un proceso de renovaci\'on si los tiempos de interocurrencia $\xi_{n}=T_{n}-T_{n-1}$, para $n\geq1$, son independientes e identicamente distribuidos con distribuci\'on $F$, donde $F\left(0\right)=0$ y $T_{0}=0$. Los $T_{n}$ son llamados tiempos de renovaci\'on, referente a la independencia o renovaci\'on de la informaci\'on estoc\'astica en estos tiempos. Los $\xi_{n}$ son los tiempos de inter-renovaci\'on, y $N\left(t\right)$ es el n\'umero de renovaciones en el intervalo $\left[0,t\right)$
\end{Def}


\begin{Note}
Para definir un proceso de renovaci\'on para cualquier contexto, solamente hay que especificar una distribuci\'on $F$, con $F\left(0\right)=0$, para los tiempos de inter-renovaci\'on. La funci\'on $F$ en turno degune las otra variables aleatorias. De manera formal, existe un espacio de probabilidad y una sucesi\'on de variables aleatorias $\xi_{1},\xi_{2},\ldots$ definidas en este con distribuci\'on $F$. Entonces las otras cantidades son $T_{n}=\sum_{k=1}^{n}\xi_{k}$ y $N\left(t\right)=\sum_{n=1}^{\infty}\indora\left(T_{n}\leq t\right)$, donde $T_{n}\rightarrow\infty$ casi seguramente por la Ley Fuerte de los Grandes N\'umeros.
\end{Note}







Los tiempos $T_{n}$ est\'an relacionados con los conteos de $N\left(t\right)$ por

\begin{eqnarray*}
\left\{N\left(t\right)\geq n\right\}&=&\left\{T_{n}\leq t\right\}\\
T_{N\left(t\right)}\leq &t&<T_{N\left(t\right)+1},
\end{eqnarray*}

adem\'as $N\left(T_{n}\right)=n$, y 

\begin{eqnarray*}
N\left(t\right)=\max\left\{n:T_{n}\leq t\right\}=\min\left\{n:T_{n+1}>t\right\}
\end{eqnarray*}

Por propiedades de la convoluci\'on se sabe que

\begin{eqnarray*}
P\left\{T_{n}\leq t\right\}=F^{n\star}\left(t\right)
\end{eqnarray*}
que es la $n$-\'esima convoluci\'on de $F$. Entonces 

\begin{eqnarray*}
\left\{N\left(t\right)\geq n\right\}&=&\left\{T_{n}\leq t\right\}\\
P\left\{N\left(t\right)\leq n\right\}&=&1-F^{\left(n+1\right)\star}\left(t\right)
\end{eqnarray*}

Adem\'as usando el hecho de que $\esp\left[N\left(t\right)\right]=\sum_{n=1}^{\infty}P\left\{N\left(t\right)\geq n\right\}$
se tiene que

\begin{eqnarray*}
\esp\left[N\left(t\right)\right]=\sum_{n=1}^{\infty}F^{n\star}\left(t\right)
\end{eqnarray*}

\begin{Prop}
Para cada $t\geq0$, la funci\'on generadora de momentos $\esp\left[e^{\alpha N\left(t\right)}\right]$ existe para alguna $\alpha$ en una vecindad del 0, y de aqu\'i que $\esp\left[N\left(t\right)^{m}\right]<\infty$, para $m\geq1$.
\end{Prop}

\begin{Ejem}[\textbf{Proceso Poisson}]

Suponga que se tienen tiempos de inter-renovaci\'on \textit{i.i.d.} del proceso de renovaci\'on $N\left(t\right)$ tienen distribuci\'on exponencial $F\left(t\right)=q-e^{-\lambda t}$ con tasa $\lambda$. Entonces $N\left(t\right)$ es un proceso Poisson con tasa $\lambda$.

\end{Ejem}


\begin{Note}
Si el primer tiempo de renovaci\'on $\xi_{1}$ no tiene la misma distribuci\'on que el resto de las $\xi_{n}$, para $n\geq2$, a $N\left(t\right)$ se le llama Proceso de Renovaci\'on retardado, donde si $\xi$ tiene distribuci\'on $G$, entonces el tiempo $T_{n}$ de la $n$-\'esima renovaci\'on tiene distribuci\'on $G\star F^{\left(n-1\right)\star}\left(t\right)$
\end{Note}


\begin{Teo}
Para una constante $\mu\leq\infty$ ( o variable aleatoria), las siguientes expresiones son equivalentes:

\begin{eqnarray}
lim_{n\rightarrow\infty}n^{-1}T_{n}&=&\mu,\textrm{ c.s.}\\
lim_{t\rightarrow\infty}t^{-1}N\left(t\right)&=&1/\mu,\textrm{ c.s.}
\end{eqnarray}
\end{Teo}


Es decir, $T_{n}$ satisface la Ley Fuerte de los Grandes N\'umeros s\'i y s\'olo s\'i $N\left/t\right)$ la cumple.


\begin{Coro}[Ley Fuerte de los Grandes N\'umeros para Procesos de Renovaci\'on]
Si $N\left(t\right)$ es un proceso de renovaci\'on cuyos tiempos de inter-renovaci\'on tienen media $\mu\leq\infty$, entonces
\begin{eqnarray}
t^{-1}N\left(t\right)\rightarrow 1/\mu,\textrm{ c.s. cuando }t\rightarrow\infty.
\end{eqnarray}

\end{Coro}


Considerar el proceso estoc\'astico de valores reales $\left\{Z\left(t\right):t\geq0\right\}$ en el mismo espacio de probabilidad que $N\left(t\right)$

\begin{Def}
Para el proceso $\left\{Z\left(t\right):t\geq0\right\}$ se define la fluctuaci\'on m\'axima de $Z\left(t\right)$ en el intervalo $\left(T_{n-1},T_{n}\right]$:
\begin{eqnarray*}
M_{n}=\sup_{T_{n-1}<t\leq T_{n}}|Z\left(t\right)-Z\left(T_{n-1}\right)|
\end{eqnarray*}
\end{Def}

\begin{Teo}
Sup\'ongase que $n^{-1}T_{n}\rightarrow\mu$ c.s. cuando $n\rightarrow\infty$, donde $\mu\leq\infty$ es una constante o variable aleatoria. Sea $a$ una constante o variable aleatoria que puede ser infinita cuando $\mu$ es finita, y considere las expresiones l\'imite:
\begin{eqnarray}
lim_{n\rightarrow\infty}n^{-1}Z\left(T_{n}\right)&=&a,\textrm{ c.s.}\\
lim_{t\rightarrow\infty}t^{-1}Z\left(t\right)&=&a/\mu,\textrm{ c.s.}
\end{eqnarray}
La segunda expresi\'on implica la primera. Conversamente, la primera implica la segunda si el proceso $Z\left(t\right)$ es creciente, o si $lim_{n\rightarrow\infty}n^{-1}M_{n}=0$ c.s.
\end{Teo}

\begin{Coro}
Si $N\left(t\right)$ es un proceso de renovaci\'on, y $\left(Z\left(T_{n}\right)-Z\left(T_{n-1}\right),M_{n}\right)$, para $n\geq1$, son variables aleatorias independientes e id\'enticamente distribuidas con media finita, entonces,
\begin{eqnarray}
lim_{t\rightarrow\infty}t^{-1}Z\left(t\right)\rightarrow\frac{\esp\left[Z\left(T_{1}\right)-Z\left(T_{0}\right)\right]}{\esp\left[T_{1}\right]},\textrm{ c.s. cuando  }t\rightarrow\infty.
\end{eqnarray}
\end{Coro}


Sup\'ongase que $N\left(t\right)$ es un proceso de renovaci\'on con distribuci\'on $F$ con media finita $\mu$.

\begin{Def}
La funci\'on de renovaci\'on asociada con la distribuci\'on $F$, del proceso $N\left(t\right)$, es
\begin{eqnarray*}
U\left(t\right)=\sum_{n=1}^{\infty}F^{n\star}\left(t\right),\textrm{   }t\geq0,
\end{eqnarray*}
donde $F^{0\star}\left(t\right)=\indora\left(t\geq0\right)$.
\end{Def}


\begin{Prop}
Sup\'ongase que la distribuci\'on de inter-renovaci\'on $F$ tiene densidad $f$. Entonces $U\left(t\right)$ tambi\'en tiene densidad, para $t>0$, y es $U^{'}\left(t\right)=\sum_{n=0}^{\infty}f^{n\star}\left(t\right)$. Adem\'as
\begin{eqnarray*}
\prob\left\{N\left(t\right)>N\left(t-\right)\right\}=0\textrm{,   }t\geq0.
\end{eqnarray*}
\end{Prop}

\begin{Def}
La Transformada de Laplace-Stieljes de $F$ est\'a dada por

\begin{eqnarray*}
\hat{F}\left(\alpha\right)=\int_{\rea_{+}}e^{-\alpha t}dF\left(t\right)\textrm{,  }\alpha\geq0.
\end{eqnarray*}
\end{Def}

Entonces

\begin{eqnarray*}
\hat{U}\left(\alpha\right)=\sum_{n=0}^{\infty}\hat{F^{n\star}}\left(\alpha\right)=\sum_{n=0}^{\infty}\hat{F}\left(\alpha\right)^{n}=\frac{1}{1-\hat{F}\left(\alpha\right)}.
\end{eqnarray*}


\begin{Prop}
La Transformada de Laplace $\hat{U}\left(\alpha\right)$ y $\hat{F}\left(\alpha\right)$ determina una a la otra de manera \'unica por la relaci\'on $\hat{U}\left(\alpha\right)=\frac{1}{1-\hat{F}\left(\alpha\right)}$.
\end{Prop}


\begin{Note}
Un proceso de renovaci\'on $N\left(t\right)$ cuyos tiempos de inter-renovaci\'on tienen media finita, es un proceso Poisson con tasa $\lambda$ si y s\'olo s\'i $\esp\left[U\left(t\right)\right]=\lambda t$, para $t\geq0$.
\end{Note}


\begin{Teo}
Sea $N\left(t\right)$ un proceso puntual simple con puntos de localizaci\'on $T_{n}$ tal que $\eta\left(t\right)=\esp\left[N\left(\right)\right]$ es finita para cada $t$. Entonces para cualquier funci\'on $f:\rea_{+}\rightarrow\rea$,
\begin{eqnarray*}
\esp\left[\sum_{n=1}^{N\left(\right)}f\left(T_{n}\right)\right]=\int_{\left(0,t\right]}f\left(s\right)d\eta\left(s\right)\textrm{,  }t\geq0,
\end{eqnarray*}
suponiendo que la integral exista. Adem\'as si $X_{1},X_{2},\ldots$ son variables aleatorias definidas en el mismo espacio de probabilidad que el proceso $N\left(t\right)$ tal que $\esp\left[X_{n}|T_{n}=s\right]=f\left(s\right)$, independiente de $n$. Entonces
\begin{eqnarray*}
\esp\left[\sum_{n=1}^{N\left(t\right)}X_{n}\right]=\int_{\left(0,t\right]}f\left(s\right)d\eta\left(s\right)\textrm{,  }t\geq0,
\end{eqnarray*} 
suponiendo que la integral exista. 
\end{Teo}

\begin{Coro}[Identidad de Wald para Renovaciones]
Para el proceso de renovaci\'on $N\left(t\right)$,
\begin{eqnarray*}
\esp\left[T_{N\left(t\right)+1}\right]=\mu\esp\left[N\left(t\right)+1\right]\textrm{,  }t\geq0,
\end{eqnarray*}  
\end{Coro}


\begin{Def}
Sea $h\left(t\right)$ funci\'on de valores reales en $\rea$ acotada en intervalos finitos e igual a cero para $t<0$ La ecuaci\'on de renovaci\'on para $h\left(t\right)$ y la distribuci\'on $F$ es

\begin{eqnarray}%\label{Ec.Renovacion}
H\left(t\right)=h\left(t\right)+\int_{\left[0,t\right]}H\left(t-s\right)dF\left(s\right)\textrm{,    }t\geq0,
\end{eqnarray}
donde $H\left(t\right)$ es una funci\'on de valores reales. Esto es $H=h+F\star H$. Decimos que $H\left(t\right)$ es soluci\'on de esta ecuaci\'on si satisface la ecuaci\'on, y es acotada en intervalos finitos e iguales a cero para $t<0$.
\end{Def}

\begin{Prop}
La funci\'on $U\star h\left(t\right)$ es la \'unica soluci\'on de la ecuaci\'on de renovaci\'on (\ref{Ec.Renovacion}).
\end{Prop}

\begin{Teo}[Teorema Renovaci\'on Elemental]
\begin{eqnarray*}
t^{-1}U\left(t\right)\rightarrow 1/\mu\textrm{,    cuando }t\rightarrow\infty.
\end{eqnarray*}
\end{Teo}



Sup\'ongase que $N\left(t\right)$ es un proceso de renovaci\'on con distribuci\'on $F$ con media finita $\mu$.

\begin{Def}
La funci\'on de renovaci\'on asociada con la distribuci\'on $F$, del proceso $N\left(t\right)$, es
\begin{eqnarray*}
U\left(t\right)=\sum_{n=1}^{\infty}F^{n\star}\left(t\right),\textrm{   }t\geq0,
\end{eqnarray*}
donde $F^{0\star}\left(t\right)=\indora\left(t\geq0\right)$.
\end{Def}


\begin{Prop}
Sup\'ongase que la distribuci\'on de inter-renovaci\'on $F$ tiene densidad $f$. Entonces $U\left(t\right)$ tambi\'en tiene densidad, para $t>0$, y es $U^{'}\left(t\right)=\sum_{n=0}^{\infty}f^{n\star}\left(t\right)$. Adem\'as
\begin{eqnarray*}
\prob\left\{N\left(t\right)>N\left(t-\right)\right\}=0\textrm{,   }t\geq0.
\end{eqnarray*}
\end{Prop}

\begin{Def}
La Transformada de Laplace-Stieljes de $F$ est\'a dada por

\begin{eqnarray*}
\hat{F}\left(\alpha\right)=\int_{\rea_{+}}e^{-\alpha t}dF\left(t\right)\textrm{,  }\alpha\geq0.
\end{eqnarray*}
\end{Def}

Entonces

\begin{eqnarray*}
\hat{U}\left(\alpha\right)=\sum_{n=0}^{\infty}\hat{F^{n\star}}\left(\alpha\right)=\sum_{n=0}^{\infty}\hat{F}\left(\alpha\right)^{n}=\frac{1}{1-\hat{F}\left(\alpha\right)}.
\end{eqnarray*}


\begin{Prop}
La Transformada de Laplace $\hat{U}\left(\alpha\right)$ y $\hat{F}\left(\alpha\right)$ determina una a la otra de manera \'unica por la relaci\'on $\hat{U}\left(\alpha\right)=\frac{1}{1-\hat{F}\left(\alpha\right)}$.
\end{Prop}


\begin{Note}
Un proceso de renovaci\'on $N\left(t\right)$ cuyos tiempos de inter-renovaci\'on tienen media finita, es un proceso Poisson con tasa $\lambda$ si y s\'olo s\'i $\esp\left[U\left(t\right)\right]=\lambda t$, para $t\geq0$.
\end{Note}


\begin{Teo}
Sea $N\left(t\right)$ un proceso puntual simple con puntos de localizaci\'on $T_{n}$ tal que $\eta\left(t\right)=\esp\left[N\left(\right)\right]$ es finita para cada $t$. Entonces para cualquier funci\'on $f:\rea_{+}\rightarrow\rea$,
\begin{eqnarray*}
\esp\left[\sum_{n=1}^{N\left(\right)}f\left(T_{n}\right)\right]=\int_{\left(0,t\right]}f\left(s\right)d\eta\left(s\right)\textrm{,  }t\geq0,
\end{eqnarray*}
suponiendo que la integral exista. Adem\'as si $X_{1},X_{2},\ldots$ son variables aleatorias definidas en el mismo espacio de probabilidad que el proceso $N\left(t\right)$ tal que $\esp\left[X_{n}|T_{n}=s\right]=f\left(s\right)$, independiente de $n$. Entonces
\begin{eqnarray*}
\esp\left[\sum_{n=1}^{N\left(t\right)}X_{n}\right]=\int_{\left(0,t\right]}f\left(s\right)d\eta\left(s\right)\textrm{,  }t\geq0,
\end{eqnarray*} 
suponiendo que la integral exista. 
\end{Teo}

\begin{Coro}[Identidad de Wald para Renovaciones]
Para el proceso de renovaci\'on $N\left(t\right)$,
\begin{eqnarray*}
\esp\left[T_{N\left(t\right)+1}\right]=\mu\esp\left[N\left(t\right)+1\right]\textrm{,  }t\geq0,
\end{eqnarray*}  
\end{Coro}


\begin{Def}
Sea $h\left(t\right)$ funci\'on de valores reales en $\rea$ acotada en intervalos finitos e igual a cero para $t<0$ La ecuaci\'on de renovaci\'on para $h\left(t\right)$ y la distribuci\'on $F$ es

\begin{eqnarray}%\label{Ec.Renovacion}
H\left(t\right)=h\left(t\right)+\int_{\left[0,t\right]}H\left(t-s\right)dF\left(s\right)\textrm{,    }t\geq0,
\end{eqnarray}
donde $H\left(t\right)$ es una funci\'on de valores reales. Esto es $H=h+F\star H$. Decimos que $H\left(t\right)$ es soluci\'on de esta ecuaci\'on si satisface la ecuaci\'on, y es acotada en intervalos finitos e iguales a cero para $t<0$.
\end{Def}

\begin{Prop}
La funci\'on $U\star h\left(t\right)$ es la \'unica soluci\'on de la ecuaci\'on de renovaci\'on (\ref{Ec.Renovacion}).
\end{Prop}

\begin{Teo}[Teorema Renovaci\'on Elemental]
\begin{eqnarray*}
t^{-1}U\left(t\right)\rightarrow 1/\mu\textrm{,    cuando }t\rightarrow\infty.
\end{eqnarray*}
\end{Teo}


\begin{Note} Una funci\'on $h:\rea_{+}\rightarrow\rea$ es Directamente Riemann Integrable en los siguientes casos:
\begin{itemize}
\item[a)] $h\left(t\right)\geq0$ es decreciente y Riemann Integrable.
\item[b)] $h$ es continua excepto posiblemente en un conjunto de Lebesgue de medida 0, y $|h\left(t\right)|\leq b\left(t\right)$, donde $b$ es DRI.
\end{itemize}
\end{Note}

\begin{Teo}[Teorema Principal de Renovaci\'on]
Si $F$ es no aritm\'etica y $h\left(t\right)$ es Directamente Riemann Integrable (DRI), entonces

\begin{eqnarray*}
lim_{t\rightarrow\infty}U\star h=\frac{1}{\mu}\int_{\rea_{+}}h\left(s\right)ds.
\end{eqnarray*}
\end{Teo}

\begin{Prop}
Cualquier funci\'on $H\left(t\right)$ acotada en intervalos finitos y que es 0 para $t<0$ puede expresarse como
\begin{eqnarray*}
H\left(t\right)=U\star h\left(t\right)\textrm{,  donde }h\left(t\right)=H\left(t\right)-F\star H\left(t\right)
\end{eqnarray*}
\end{Prop}

\begin{Def}
Un proceso estoc\'astico $X\left(t\right)$ es crudamente regenerativo en un tiempo aleatorio positivo $T$ si
\begin{eqnarray*}
\esp\left[X\left(T+t\right)|T\right]=\esp\left[X\left(t\right)\right]\textrm{, para }t\geq0,\end{eqnarray*}
y con las esperanzas anteriores finitas.
\end{Def}

\begin{Prop}
Sup\'ongase que $X\left(t\right)$ es un proceso crudamente regenerativo en $T$, que tiene distribuci\'on $F$. Si $\esp\left[X\left(t\right)\right]$ es acotado en intervalos finitos, entonces
\begin{eqnarray*}
\esp\left[X\left(t\right)\right]=U\star h\left(t\right)\textrm{,  donde }h\left(t\right)=\esp\left[X\left(t\right)\indora\left(T>t\right)\right].
\end{eqnarray*}
\end{Prop}

\begin{Teo}[Regeneraci\'on Cruda]
Sup\'ongase que $X\left(t\right)$ es un proceso con valores positivo crudamente regenerativo en $T$, y def\'inase $M=\sup\left\{|X\left(t\right)|:t\leq T\right\}$. Si $T$ es no aritm\'etico y $M$ y $MT$ tienen media finita, entonces
\begin{eqnarray*}
lim_{t\rightarrow\infty}\esp\left[X\left(t\right)\right]=\frac{1}{\mu}\int_{\rea_{+}}h\left(s\right)ds,
\end{eqnarray*}
donde $h\left(t\right)=\esp\left[X\left(t\right)\indora\left(T>t\right)\right]$.
\end{Teo}


\begin{Note} Una funci\'on $h:\rea_{+}\rightarrow\rea$ es Directamente Riemann Integrable en los siguientes casos:
\begin{itemize}
\item[a)] $h\left(t\right)\geq0$ es decreciente y Riemann Integrable.
\item[b)] $h$ es continua excepto posiblemente en un conjunto de Lebesgue de medida 0, y $|h\left(t\right)|\leq b\left(t\right)$, donde $b$ es DRI.
\end{itemize}
\end{Note}

\begin{Teo}[Teorema Principal de Renovaci\'on]
Si $F$ es no aritm\'etica y $h\left(t\right)$ es Directamente Riemann Integrable (DRI), entonces

\begin{eqnarray*}
lim_{t\rightarrow\infty}U\star h=\frac{1}{\mu}\int_{\rea_{+}}h\left(s\right)ds.
\end{eqnarray*}
\end{Teo}

\begin{Prop}
Cualquier funci\'on $H\left(t\right)$ acotada en intervalos finitos y que es 0 para $t<0$ puede expresarse como
\begin{eqnarray*}
H\left(t\right)=U\star h\left(t\right)\textrm{,  donde }h\left(t\right)=H\left(t\right)-F\star H\left(t\right)
\end{eqnarray*}
\end{Prop}

\begin{Def}
Un proceso estoc\'astico $X\left(t\right)$ es crudamente regenerativo en un tiempo aleatorio positivo $T$ si
\begin{eqnarray*}
\esp\left[X\left(T+t\right)|T\right]=\esp\left[X\left(t\right)\right]\textrm{, para }t\geq0,\end{eqnarray*}
y con las esperanzas anteriores finitas.
\end{Def}

\begin{Prop}
Sup\'ongase que $X\left(t\right)$ es un proceso crudamente regenerativo en $T$, que tiene distribuci\'on $F$. Si $\esp\left[X\left(t\right)\right]$ es acotado en intervalos finitos, entonces
\begin{eqnarray*}
\esp\left[X\left(t\right)\right]=U\star h\left(t\right)\textrm{,  donde }h\left(t\right)=\esp\left[X\left(t\right)\indora\left(T>t\right)\right].
\end{eqnarray*}
\end{Prop}

\begin{Teo}[Regeneraci\'on Cruda]
Sup\'ongase que $X\left(t\right)$ es un proceso con valores positivo crudamente regenerativo en $T$, y def\'inase $M=\sup\left\{|X\left(t\right)|:t\leq T\right\}$. Si $T$ es no aritm\'etico y $M$ y $MT$ tienen media finita, entonces
\begin{eqnarray*}
lim_{t\rightarrow\infty}\esp\left[X\left(t\right)\right]=\frac{1}{\mu}\int_{\rea_{+}}h\left(s\right)ds,
\end{eqnarray*}
donde $h\left(t\right)=\esp\left[X\left(t\right)\indora\left(T>t\right)\right]$.
\end{Teo}

\begin{Def}
Para el proceso $\left\{\left(N\left(t\right),X\left(t\right)\right):t\geq0\right\}$, sus trayectoria muestrales en el intervalo de tiempo $\left[T_{n-1},T_{n}\right)$ est\'an descritas por
\begin{eqnarray*}
\zeta_{n}=\left(\xi_{n},\left\{X\left(T_{n-1}+t\right):0\leq t<\xi_{n}\right\}\right)
\end{eqnarray*}
Este $\zeta_{n}$ es el $n$-\'esimo segmento del proceso. El proceso es regenerativo sobre los tiempos $T_{n}$ si sus segmentos $\zeta_{n}$ son independientes e id\'enticamennte distribuidos.
\end{Def}


\begin{Note}
Si $\tilde{X}\left(t\right)$ con espacio de estados $\tilde{S}$ es regenerativo sobre $T_{n}$, entonces $X\left(t\right)=f\left(\tilde{X}\left(t\right)\right)$ tambi\'en es regenerativo sobre $T_{n}$, para cualquier funci\'on $f:\tilde{S}\rightarrow S$.
\end{Note}

\begin{Note}
Los procesos regenerativos son crudamente regenerativos, pero no al rev\'es.
\end{Note}


\begin{Note}
Un proceso estoc\'astico a tiempo continuo o discreto es regenerativo si existe un proceso de renovaci\'on  tal que los segmentos del proceso entre tiempos de renovaci\'on sucesivos son i.i.d., es decir, para $\left\{X\left(t\right):t\geq0\right\}$ proceso estoc\'astico a tiempo continuo con espacio de estados $S$, espacio m\'etrico.
\end{Note}

Para $\left\{X\left(t\right):t\geq0\right\}$ Proceso Estoc\'astico a tiempo continuo con estado de espacios $S$, que es un espacio m\'etrico, con trayectorias continuas por la derecha y con l\'imites por la izquierda c.s. Sea $N\left(t\right)$ un proceso de renovaci\'on en $\rea_{+}$ definido en el mismo espacio de probabilidad que $X\left(t\right)$, con tiempos de renovaci\'on $T$ y tiempos de inter-renovaci\'on $\xi_{n}=T_{n}-T_{n-1}$, con misma distribuci\'on $F$ de media finita $\mu$.



\begin{Def}
Para el proceso $\left\{\left(N\left(t\right),X\left(t\right)\right):t\geq0\right\}$, sus trayectoria muestrales en el intervalo de tiempo $\left[T_{n-1},T_{n}\right)$ est\'an descritas por
\begin{eqnarray*}
\zeta_{n}=\left(\xi_{n},\left\{X\left(T_{n-1}+t\right):0\leq t<\xi_{n}\right\}\right)
\end{eqnarray*}
Este $\zeta_{n}$ es el $n$-\'esimo segmento del proceso. El proceso es regenerativo sobre los tiempos $T_{n}$ si sus segmentos $\zeta_{n}$ son independientes e id\'enticamennte distribuidos.
\end{Def}

\begin{Note}
Un proceso regenerativo con media de la longitud de ciclo finita es llamado positivo recurrente.
\end{Note}

\begin{Teo}[Procesos Regenerativos]
Suponga que el proceso
\end{Teo}


\begin{Def}[Renewal Process Trinity]
Para un proceso de renovaci\'on $N\left(t\right)$, los siguientes procesos proveen de informaci\'on sobre los tiempos de renovaci\'on.
\begin{itemize}
\item $A\left(t\right)=t-T_{N\left(t\right)}$, el tiempo de recurrencia hacia atr\'as al tiempo $t$, que es el tiempo desde la \'ultima renovaci\'on para $t$.

\item $B\left(t\right)=T_{N\left(t\right)+1}-t$, el tiempo de recurrencia hacia adelante al tiempo $t$, residual del tiempo de renovaci\'on, que es el tiempo para la pr\'oxima renovaci\'on despu\'es de $t$.

\item $L\left(t\right)=\xi_{N\left(t\right)+1}=A\left(t\right)+B\left(t\right)$, la longitud del intervalo de renovaci\'on que contiene a $t$.
\end{itemize}
\end{Def}

\begin{Note}
El proceso tridimensional $\left(A\left(t\right),B\left(t\right),L\left(t\right)\right)$ es regenerativo sobre $T_{n}$, y por ende cada proceso lo es. Cada proceso $A\left(t\right)$ y $B\left(t\right)$ son procesos de MArkov a tiempo continuo con trayectorias continuas por partes en el espacio de estados $\rea_{+}$. Una expresi\'on conveniente para su distribuci\'on conjunta es, para $0\leq x<t,y\geq0$
\begin{equation}\label{NoRenovacion}
P\left\{A\left(t\right)>x,B\left(t\right)>y\right\}=
P\left\{N\left(t+y\right)-N\left((t-x)\right)=0\right\}
\end{equation}
\end{Note}

\begin{Ejem}[Tiempos de recurrencia Poisson]
Si $N\left(t\right)$ es un proceso Poisson con tasa $\lambda$, entonces de la expresi\'on (\ref{NoRenovacion}) se tiene que

\begin{eqnarray*}
\begin{array}{lc}
P\left\{A\left(t\right)>x,B\left(t\right)>y\right\}=e^{-\lambda\left(x+y\right)},&0\leq x<t,y\geq0,
\end{array}
\end{eqnarray*}
que es la probabilidad Poisson de no renovaciones en un intervalo de longitud $x+y$.

\end{Ejem}

\begin{Note}
Una cadena de Markov erg\'odica tiene la propiedad de ser estacionaria si la distribución de su estado al tiempo $0$ es su distribuci\'on estacionaria.
\end{Note}


\begin{Def}
Un proceso estoc\'astico a tiempo continuo $\left\{X\left(t\right):t\geq0\right\}$ en un espacio general es estacionario si sus distribuciones finito dimensionales son invariantes bajo cualquier  traslado: para cada $0\leq s_{1}<s_{2}<\cdots<s_{k}$ y $t\geq0$,
\begin{eqnarray*}
\left(X\left(s_{1}+t\right),\ldots,X\left(s_{k}+t\right)\right)=_{d}\left(X\left(s_{1}\right),\ldots,X\left(s_{k}\right)\right).
\end{eqnarray*}
\end{Def}

\begin{Note}
Un proceso de Markov es estacionario si $X\left(t\right)=_{d}X\left(0\right)$, $t\geq0$.
\end{Note}

Considerese el proceso $N\left(t\right)=\sum_{n}\indora\left(\tau_{n}\leq t\right)$ en $\rea_{+}$, con puntos $0<\tau_{1}<\tau_{2}<\cdots$.

\begin{Prop}
Si $N$ es un proceso puntual estacionario y $\esp\left[N\left(1\right)\right]<\infty$, entonces $\esp\left[N\left(t\right)\right]=t\esp\left[N\left(1\right)\right]$, $t\geq0$

\end{Prop}

\begin{Teo}
Los siguientes enunciados son equivalentes
\begin{itemize}
\item[i)] El proceso retardado de renovaci\'on $N$ es estacionario.

\item[ii)] EL proceso de tiempos de recurrencia hacia adelante $B\left(t\right)$ es estacionario.


\item[iii)] $\esp\left[N\left(t\right)\right]=t/\mu$,


\item[iv)] $G\left(t\right)=F_{e}\left(t\right)=\frac{1}{\mu}\int_{0}^{t}\left[1-F\left(s\right)\right]ds$
\end{itemize}
Cuando estos enunciados son ciertos, $P\left\{B\left(t\right)\leq x\right\}=F_{e}\left(x\right)$, para $t,x\geq0$.

\end{Teo}

\begin{Note}
Una consecuencia del teorema anterior es que el Proceso Poisson es el \'unico proceso sin retardo que es estacionario.
\end{Note}

\begin{Coro}
El proceso de renovaci\'on $N\left(t\right)$ sin retardo, y cuyos tiempos de inter renonaci\'on tienen media finita, es estacionario si y s\'olo si es un proceso Poisson.

\end{Coro}




%______________________________________________________________________
\section{Resultados para Procesos de Salida}
%______________________________________________________________________
En Sigman, Thorison y Wolff \cite{Sigman2} prueban que para la existencia de un una sucesi\'on infinita no decreciente de tiempos de regeneraci\'on $\tau_{1}\leq\tau_{2}\leq\cdots$ en los cuales el proceso se regenera, basta un tiempo de regeneraci\'on $R_{1}$, donde $R_{j}=\tau_{j}-\tau_{j-1}$. Para tal efecto se requiere la existencia de un espacio de probabilidad $\left(\Omega,\mathcal{F},\prob\right)$, y proceso estoc\'astico $\textit{X}=\left\{X\left(t\right):t\geq0\right\}$ con espacio de estados $\left(S,\mathcal{R}\right)$, con $\mathcal{R}$ $\sigma$-\'algebra.

\begin{Prop}
Si existe una variable aleatoria no negativa $R_{1}$ tal que $\theta_{R\footnotesize{1}}X=_{D}X$, entonces $\left(\Omega,\mathcal{F},\prob\right)$ puede extenderse para soportar una sucesi\'on estacionaria de variables aleatorias $R=\left\{R_{k}:k\geq1\right\}$, tal que para $k\geq1$,
\begin{eqnarray*}
\theta_{k}\left(X,R\right)=_{D}\left(X,R\right).
\end{eqnarray*}

Adem\'as, para $k\geq1$, $\theta_{k}R$ es condicionalmente independiente de $\left(X,R_{1},\ldots,R_{k}\right)$, dado $\theta_{\tau k}X$.

\end{Prop}


\begin{itemize}
\item Doob en 1953 demostr\'o que el estado estacionario de un proceso de partida en un sistema de espera $M/G/\infty$, es Poisson con la misma tasa que el proceso de arribos.

\item Burke en 1968, fue el primero en demostrar que el estado estacionario de un proceso de salida de una cola $M/M/s$ es un proceso Poisson.

\item Disney en 1973 obtuvo el siguiente resultado:

\begin{Teo}
Para el sistema de espera $M/G/1/L$ con disciplina FIFO, el proceso $\textbf{I}$ es un proceso de renovaci\'on si y s\'olo si el proceso denominado longitud de la cola es estacionario y se cumple cualquiera de los siguientes casos:

\begin{itemize}
\item[a)] Los tiempos de servicio son identicamente cero;
\item[b)] $L=0$, para cualquier proceso de servicio $S$;
\item[c)] $L=1$ y $G=D$;
\item[d)] $L=\infty$ y $G=M$.
\end{itemize}
En estos casos, respectivamente, las distribuciones de interpartida $P\left\{T_{n+1}-T_{n}\leq t\right\}$ son


\begin{itemize}
\item[a)] $1-e^{-\lambda t}$, $t\geq0$;
\item[b)] $1-e^{-\lambda t}*F\left(t\right)$, $t\geq0$;
\item[c)] $1-e^{-\lambda t}*\indora_{d}\left(t\right)$, $t\geq0$;
\item[d)] $1-e^{-\lambda t}*F\left(t\right)$, $t\geq0$.
\end{itemize}
\end{Teo}


\item Finch (1959) mostr\'o que para los sistemas $M/G/1/L$, con $1\leq L\leq \infty$ con distribuciones de servicio dos veces diferenciable, solamente el sistema $M/M/1/\infty$ tiene proceso de salida de renovaci\'on estacionario.

\item King (1971) demostro que un sistema de colas estacionario $M/G/1/1$ tiene sus tiempos de interpartida sucesivas $D_{n}$ y $D_{n+1}$ son independientes, si y s\'olo si, $G=D$, en cuyo caso le proceso de salida es de renovaci\'on.

\item Disney (1973) demostr\'o que el \'unico sistema estacionario $M/G/1/L$, que tiene proceso de salida de renovaci\'on  son los sistemas $M/M/1$ y $M/D/1/1$.



\item El siguiente resultado es de Disney y Koning (1985)
\begin{Teo}
En un sistema de espera $M/G/s$, el estado estacionario del proceso de salida es un proceso Poisson para cualquier distribuci\'on de los tiempos de servicio si el sistema tiene cualquiera de las siguientes cuatro propiedades.

\begin{itemize}
\item[a)] $s=\infty$
\item[b)] La disciplina de servicio es de procesador compartido.
\item[c)] La disciplina de servicio es LCFS y preemptive resume, esto se cumple para $L<\infty$
\item[d)] $G=M$.
\end{itemize}

\end{Teo}

\item El siguiente resultado es de Alamatsaz (1983)

\begin{Teo}
En cualquier sistema de colas $GI/G/1/L$ con $1\leq L<\infty$ y distribuci\'on de interarribos $A$ y distribuci\'on de los tiempos de servicio $B$, tal que $A\left(0\right)=0$, $A\left(t\right)\left(1-B\left(t\right)\right)>0$ para alguna $t>0$ y $B\left(t\right)$ para toda $t>0$, es imposible que el proceso de salida estacionario sea de renovaci\'on.
\end{Teo}

\end{itemize}

Estos resultados aparecen en Daley (1968) \cite{Daley68} para $\left\{T_{n}\right\}$ intervalos de inter-arribo, $\left\{D_{n}\right\}$ intervalos de inter-salida y $\left\{S_{n}\right\}$ tiempos de servicio.

\begin{itemize}
\item Si el proceso $\left\{T_{n}\right\}$ es Poisson, el proceso $\left\{D_{n}\right\}$ es no correlacionado si y s\'olo si es un proceso Poisso, lo cual ocurre si y s\'olo si $\left\{S_{n}\right\}$ son exponenciales negativas.

\item Si $\left\{S_{n}\right\}$ son exponenciales negativas, $\left\{D_{n}\right\}$ es un proceso de renovaci\'on  si y s\'olo si es un proceso Poisson, lo cual ocurre si y s\'olo si $\left\{T_{n}\right\}$ es un proceso Poisson.

\item $\esp\left(D_{n}\right)=\esp\left(T_{n}\right)$.

\item Para un sistema de visitas $GI/M/1$ se tiene el siguiente teorema:

\begin{Teo}
En un sistema estacionario $GI/M/1$ los intervalos de interpartida tienen
\begin{eqnarray*}
\esp\left(e^{-\theta D_{n}}\right)&=&\mu\left(\mu+\theta\right)^{-1}\left[\delta\theta
-\mu\left(1-\delta\right)\alpha\left(\theta\right)\right]
\left[\theta-\mu\left(1-\delta\right)^{-1}\right]\\
\alpha\left(\theta\right)&=&\esp\left[e^{-\theta T_{0}}\right]\\
var\left(D_{n}\right)&=&var\left(T_{0}\right)-\left(\tau^{-1}-\delta^{-1}\right)
2\delta\left(\esp\left(S_{0}\right)\right)^{2}\left(1-\delta\right)^{-1}.
\end{eqnarray*}
\end{Teo}



\begin{Teo}
El proceso de salida de un sistema de colas estacionario $GI/M/1$ es un proceso de renovaci\'on si y s\'olo si el proceso de entrada es un proceso Poisson, en cuyo caso el proceso de salida es un proceso Poisson.
\end{Teo}


\begin{Teo}
Los intervalos de interpartida $\left\{D_{n}\right\}$ de un sistema $M/G/1$ estacionario son no correlacionados si y s\'olo si la distribuci\'on de los tiempos de servicio es exponencial negativa, es decir, el sistema es de tipo  $M/M/1$.

\end{Teo}



\end{itemize}





\begin{thebibliography}{99}

\bibitem{ISL}
James, G., Witten, D., Hastie, T., and Tibshirani, R. (2013). \textit{An Introduction to Statistical Learning: with Applications in R}. Springer.

\bibitem{Logistic}
Hosmer, D. W., Lemeshow, S., and Sturdivant, R. X. (2013). \textit{Applied Logistic Regression} (3rd ed.). Wiley.

\bibitem{PatternRecognition}
Bishop, C. M. (2006). \textit{Pattern Recognition and Machine Learning}. Springer.

\bibitem{Harrell}
Harrell, F. E. (2015). \textit{Regression Modeling Strategies: With Applications to Linear Models, Logistic and Ordinal Regression, and Survival Analysis}. Springer.

\bibitem{RDocumentation}
R Documentation and Tutorials: \url{https://cran.r-project.org/manuals.html}

\bibitem{RBlogger}
Tutorials on R-bloggers: \url{https://www.r-bloggers.com/}

\bibitem{CourseraML}
Coursera: \textit{Machine Learning} by Andrew Ng.

\bibitem{edXDS}
edX: \textit{Data Science and Machine Learning Essentials} by Microsoft.

% Libros adicionales
\bibitem{Ross}
Ross, S. M. (2014). \textit{Introduction to Probability and Statistics for Engineers and Scientists}. Academic Press.

\bibitem{DeGroot}
DeGroot, M. H., and Schervish, M. J. (2012). \textit{Probability and Statistics} (4th ed.). Pearson.

\bibitem{Hogg}
Hogg, R. V., McKean, J., and Craig, A. T. (2019). \textit{Introduction to Mathematical Statistics} (8th ed.). Pearson.

\bibitem{Kleinbaum}
Kleinbaum, D. G., and Klein, M. (2010). \textit{Logistic Regression: A Self-Learning Text} (3rd ed.). Springer.

% Artículos y tutoriales adicionales
\bibitem{Wasserman}
Wasserman, L. (2004). \textit{All of Statistics: A Concise Course in Statistical Inference}. Springer.

\bibitem{KhanAcademy}
Probability and Statistics Tutorials on Khan Academy: \url{https://www.khanacademy.org/math/statistics-probability}

\bibitem{OnlineStatBook}
Online Statistics Education: \url{http://onlinestatbook.com/}

\bibitem{Peng}
Peng, C. Y. J., Lee, K. L., and Ingersoll, G. M. (2002). \textit{An Introduction to Logistic Regression Analysis and Reporting}. The Journal of Educational Research.

\bibitem{Agresti}
Agresti, A. (2007). \textit{An Introduction to Categorical Data Analysis} (2nd ed.). Wiley.

\bibitem{Han}
Han, J., Pei, J., and Kamber, M. (2011). \textit{Data Mining: Concepts and Techniques}. Morgan Kaufmann.

\bibitem{TowardsDataScience}
Data Cleaning and Preprocessing on Towards Data Science: \url{https://towardsdatascience.com/data-cleaning-and-preprocessing}

\bibitem{Molinaro}
Molinaro, A. M., Simon, R., and Pfeiffer, R. M. (2005). \textit{Prediction error estimation: a comparison of resampling methods}. Bioinformatics.

\bibitem{EvaluatingModels}
Evaluating Machine Learning Models on Towards Data Science: \url{https://towardsdatascience.com/evaluating-machine-learning-models}

\bibitem{LogisticRegressionGuide}
Practical Guide to Logistic Regression in R on Towards Data Science: \url{https://towardsdatascience.com/practical-guide-to-logistic-regression-in-r}

% Cursos en línea adicionales
\bibitem{CourseraStatistics}
Coursera: \textit{Statistics with R} by Duke University.

\bibitem{edXProbability}
edX: \textit{Data Science: Probability} by Harvard University.

\bibitem{CourseraLogistic}
Coursera: \textit{Logistic Regression} by Stanford University.

\bibitem{edXInference}
edX: \textit{Data Science: Inference and Modeling} by Harvard University.

\bibitem{CourseraWrangling}
Coursera: \textit{Data Science: Wrangling and Cleaning} by Johns Hopkins University.

\bibitem{edXRBasics}
edX: \textit{Data Science: R Basics} by Harvard University.

\bibitem{CourseraRegression}
Coursera: \textit{Regression Models} by Johns Hopkins University.

\bibitem{edXStatInference}
edX: \textit{Data Science: Statistical Inference} by Harvard University.

\bibitem{SurvivalAnalysis}
An Introduction to Survival Analysis on Towards Data Science: \url{https://towardsdatascience.com/an-introduction-to-survival-analysis}

\bibitem{MultinomialLogistic}
Multinomial Logistic Regression on DataCamp: \url{https://www.datacamp.com/community/tutorials/multinomial-logistic-regression-R}

\bibitem{CourseraSurvival}
Coursera: \textit{Survival Analysis} by Johns Hopkins University.

\bibitem{edXHighthroughput}
edX: \textit{Data Science: Statistical Inference and Modeling for High-throughput Experiments} by Harvard University.

\end{thebibliography}


\end{document}
