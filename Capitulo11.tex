
\section{Conceptos Básicos}
El análisis de supervivencia es una rama de la estad\'istica que se ocupa del análisis del tiempo que transcurre hasta que ocurre un evento de inter\'es, com\'unmente referido como "tiempo de falla". Este campo es ampliamente utilizado en medicina, biolog\'ia, ingenier\'ia, ciencias sociales, y otros campos.

\section{Definici\'on de Eventos y Tiempos}
En el análisis de supervivencia, un "evento" se refiere a la ocurrencia de un evento espec\'ifico, como la muerte, la falla de un componente, la reca\'ida de una enfermedad, etc. El "tiempo de supervivencia" es el tiempo que transcurre desde un punto de inicio definido hasta la ocurrencia del evento.

\section{Censura}
La censura ocurre cuando la informaci\'on completa sobre el tiempo hasta el evento no está disponible para todos los individuos en el estudio. Hay tres tipos principales de censura:
\begin{itemize}
    \item \textbf{Censura a la derecha:} Ocurre cuando el evento de inter\'es no se ha observado para algunos sujetos antes del final del estudio.
    \item \textbf{Censura a la izquierda:} Ocurre cuando el evento de inter\'es ocurri\'o antes del inicio del periodo de observaci\'on.
    \item \textbf{Censura por intervalo:} Ocurre cuando el evento de inter\'es se sabe que ocurri\'o en un intervalo de tiempo, pero no se conoce el momento exacto.
\end{itemize}

\section{Funci\'on de Supervivencia}
La funci\'on de supervivencia, $S(t)$, se define como la probabilidad de que un individuo sobreviva más allá de un tiempo $t$. Matemáticamente, se expresa como:
\begin{eqnarray*}
S(t) = P(T > t)
\end{eqnarray*}
donde $T$ es una variable aleatoria que representa el tiempo hasta el evento. La funci\'on de supervivencia tiene las siguientes propiedades:
\begin{itemize}
    \item $S(0) = 1$: Esto indica que al inicio (tiempo $t=0$), la probabilidad de haber experimentado el evento es cero, por lo tanto, la supervivencia es del 100%.
    \item $\lim_{t \to \infty} S(t) = 0$: A medida que el tiempo tiende al infinito, la probabilidad de que cualquier individuo a\'un no haya experimentado el evento tiende a cero.
    \item $S(t)$ es una funci\'on no creciente: Esto significa que a medida que el tiempo avanza, la probabilidad de supervivencia no aumenta.
\end{itemize}

\section{Funci\'on de Densidad de Probabilidad}
La funci\'on de densidad de probabilidad $f(t)$ describe la probabilidad de que el evento ocurra en un instante de tiempo espec\'ifico. Se define como:
\begin{eqnarray*}
f(t) = \frac{dF(t)}{dt}
\end{eqnarray*}
donde $F(t)$ es la funci\'on de distribuci\'on acumulada, $F(t) = P(T \leq t)$. La relaci\'on entre $S(t)$ y $f(t)$ es:
\begin{eqnarray*}
f(t) = -\frac{dS(t)}{dt}
\end{eqnarray*}

\section{Funci\'on de Riesgo}
La funci\'on de riesgo, $\lambda(t)$, tambi\'en conocida como funci\'on de tasa de fallas o hazard rate, se define como la tasa instant\'anea de ocurrencia del evento en el tiempo $t$, dado que el individuo ha sobrevivido hasta el tiempo $t$. Matem\'aticamente, se expresa como:
\begin{eqnarray*}
\lambda(t) = \lim_{\Delta t \to 0} \frac{P(t \leq T < t + \Delta t \mid T \geq t)}{\Delta t}
\end{eqnarray*}
Esto se puede reescribir usando $f(t)$ y $S(t)$ como:
\begin{eqnarray*}
\lambda(t) = \frac{f(t)}{S(t)}
\end{eqnarray*}

\section{Relaci\'on entre Funci\'on de Supervivencia y Funci\'on de Riesgo}
La funci\'on de supervivencia y la funci\'on de riesgo est\'an relacionadas a trav\'es de la siguiente ecuaci\'on:
\begin{eqnarray*}
S(t) = \exp\left(-\int_0^t \lambda(u) \, du\right)
\end{eqnarray*}
Esta f\'ormula se deriva del hecho de que la funci\'on de supervivencia es la probabilidad acumulativa de no haber experimentado el evento hasta el tiempo $t$, y $\lambda(t)$ es la tasa instant\'anea de ocurrencia del evento.

La funci\'on de riesgo tambi\'en puede ser expresada como:
\begin{eqnarray*}
\lambda(t) = -\frac{d}{dt} \log S(t)
\end{eqnarray*}

\section{Deducci\'on de la Funci\'on de Supervivencia}
La relaci\'on entre la funci\'on de supervivencia y la funci\'on de riesgo se puede deducir integrando la funci\'on de riesgo:
\begin{eqnarray*}
S(t) &=& \exp\left(-\int_0^t \lambda(u) \, du\right) \\
\log S(t) &=& -\int_0^t \lambda(u) \, du \\
\frac{d}{dt} \log S(t) &=& -\lambda(t) \\
\lambda(t) &=& -\frac{d}{dt} \log S(t)
\end{eqnarray*}

\section{Ejemplo de C\'alculo}
Supongamos que tenemos una muestra de tiempos de supervivencia $T_1, T_2, \ldots, T_n$. Podemos estimar la funci\'on de supervivencia emp\'irica como:
\begin{eqnarray*}
\hat{S}(t) = \frac{\text{N\'umero de individuos que sobreviven m\'as all\'a de } t}{\text{N\'umero total de individuos en riesgo en } t}
\end{eqnarray*}
y la funci\'on de riesgo emp\'irica como:
\begin{eqnarray*}
\hat{\lambda}(t) = \frac{\text{N\'umero de eventos en } t}{\text{N\'umero de individuos en riesgo en } t}
\end{eqnarray*}

\section{Conclusi\'on}
El an\'alisis de supervivencia es una herramienta poderosa para analizar datos de tiempo hasta evento. Entender los conceptos b\'asicos como la funci\'on de supervivencia y la funci\'on de riesgo es fundamental para el an\'alisis m\'as avanzado.
