
\section{Introducci\'on}
El proyecto final proporciona una oportunidad para aplicar los conceptos y t\'ecnicas aprendidas en el curso de análisis de supervivencia. Este cap\'itulo incluye una gu\'ia para desarrollar un proyecto de análisis de supervivencia y una revisi\'on de los conceptos clave.

\section{Desarrollo del Proyecto}
El proyecto final debe incluir los siguientes componentes:
\begin{enumerate}
    \item Definici\'on del problema: Identificar la pregunta de investigaci\'on y los objetivos del análisis de supervivencia.
    \item Descripci\'on de los datos: Presentar los datos utilizados, incluyendo las covariables y la estructura de los datos.
    \item Análisis exploratorio: Realizar un análisis descriptivo de los datos, incluyendo la censura y la distribuci\'on de los tiempos de supervivencia.
    \item Ajuste del modelo: Ajustar modelos de supervivencia adecuados (Kaplan-Meier, Cox, AFT) y evaluar su bondad de ajuste.
    \item Diagn\'ostico del modelo: Realizar diagn\'osticos para evaluar los supuestos del modelo y la influencia de observaciones individuales.
    \item Interpretaci\'on de resultados: Interpretar los coeficientes del modelo y las curvas de supervivencia ajustadas.
    \item Conclusiones: Resumir los hallazgos del análisis y proporcionar recomendaciones basadas en los resultados.
\end{enumerate}

\section{Revisi\'on de Conceptos Clave}
Una revisi\'on de los conceptos clave del an\'alisis de supervivencia incluye:
\begin{itemize}
    \item \textbf{Funci\'on de Supervivencia:} Define la probabilidad de sobrevivir m\'as all\'a de un tiempo espec\'ifico.
    \item \textbf{Funci\'on de Riesgo:} Define la tasa instant\'anea de ocurrencia del evento.
    \item \textbf{Estimador de Kaplan-Meier:} Proporciona una estimaci\'on no param\'etrica de la funci\'on de supervivencia.
    \item \textbf{Test de Log-rank:} Compara curvas de supervivencia entre diferentes grupos.
    \item \textbf{Modelo de Cox:} Eval\'ua el efecto de m\'ultiples covariables sobre el tiempo hasta el evento, asumiendo proporcionalidad de riesgos.
    \item \textbf{Modelos AFT:} Modelan el efecto de las covariables multiplicando el tiempo de supervivencia por una constante.
    \item \textbf{An\'alisis Multivariado:} Considera interacciones y efectos no lineales entre m\'ultiples covariables.
    \item \textbf{Supervivencia en Datos Complicados:} Maneja censura por intervalo, datos truncados y competing risks.
\end{itemize}

\section{Ejemplo de Proyecto Final}
A continuaci\'on se presenta un ejemplo de estructura de un proyecto final de an\'alisis de supervivencia:

\subsection{Definici\'on del Problema}
Analizar el efecto del tratamiento y la edad sobre la supervivencia de pacientes con una enfermedad espec\'ifica.

\subsection{Descripci\'on de los Datos}
Datos de supervivencia de 100 pacientes, con covariables: edad, sexo y tipo de tratamiento. Los tiempos de supervivencia est\'an censurados a la derecha.

\subsection{An\'alisis Exploratorio}
Realizar histogramas y curvas de Kaplan-Meier para explorar la distribuci\'on de los tiempos de supervivencia y la censura.

\subsection{Ajuste del Modelo}
Ajustar un modelo de Cox y un modelo AFT con las covariables edad y tratamiento.

\subsection{Diagn\'ostico del Modelo}
Evaluar la proporcionalidad de riesgos y realizar an\'alisis de residuos para identificar observaciones influyentes.

\subsection{Interpretaci\'on de Resultados}
Interpretar los coeficientes del modelo y las curvas de supervivencia ajustadas para diferentes niveles de las covariables.

\subsection{Conclusiones}
Resumir los hallazgos y proporcionar recomendaciones sobre el efecto del tratamiento y la edad en la supervivencia de los pacientes.

\section{Conclusi\'on}
El proyecto final es una oportunidad para aplicar los conocimientos adquiridos en un contexto pr\'actico. La revisi\'on de los conceptos clave y la aplicaci\'on de t\'ecnicas adecuadas de an\'alisis de supervivencia aseguran un an\'alisis riguroso y significativo.

