
\section{Introducci\'on}
Comparar curvas de supervivencia es crucial para determinar si existen diferencias significativas en las tasas de supervivencia entre diferentes grupos. Las pruebas de hip\'otesis, como el test de log-rank, son herramientas comunes para esta comparaci\'on.

\section{Test de Log-rank}
El test de log-rank se utiliza para comparar las curvas de supervivencia de dos o más grupos. La hip\'otesis nula es que no hay diferencia en las funciones de riesgo entre los grupos.

\subsection{F\'ormula del Test de Log-rank}
El estad\'istico del test de log-rank se define como:
\begin{eqnarray*}
\chi^2 = \frac{\left(\sum_{i=1}^k (O_i - E_i)\right)^2}{\sum_{i=1}^k V_i}
\end{eqnarray*}
donde:
\begin{itemize}
    \item $O_i$ es el n\'umero observado de eventos en el grupo $i$.
    \item $E_i$ es el n\'umero esperado de eventos en el grupo $i$.
    \item $V_i$ es la varianza del n\'umero de eventos en el grupo $i$.
\end{itemize}

\subsection{Cálculo de $E_i$ y $V_i$}
El n\'umero esperado de eventos $E_i$ y la varianza $V_i$ se calculan como:
\begin{eqnarray*}
E_i &=& \frac{d_i \cdot n_i}{n} \\
V_i &=& \frac{d_i \cdot (n - d_i) \cdot n_i \cdot (n - n_i)}{n^2 \cdot (n - 1)}
\end{eqnarray*}
donde:
\begin{itemize}
    \item $d_i$ es el n\'umero total de eventos en el grupo $i$.
    \item $n_i$ es el n\'umero de individuos en riesgo en el grupo $i$.
    \item $n$ es el n\'umero total de individuos en todos los grupos.
\end{itemize}

\section{Ejemplo de C\'alculo del Test de Log-rank}
Supongamos que tenemos dos grupos con los siguientes datos de eventos:

\begin{table}[h]
\centering
\begin{tabular}{|c|c|c|c|}
\hline
Grupo & Tiempo ($t_i$) & Eventos ($O_i$) & En Riesgo ($n_i$) \\
\hline
1 & 2 & 1 & 5 \\
1 & 4 & 1 & 4 \\
2 & 3 & 1 & 4 \\
2 & 5 & 1 & 3 \\
\hline
\end{tabular}
\caption{Ejemplo de datos para el test de log-rank}
\end{table}

Calculemos $E_i$ y $V_i$ para cada grupo:

\begin{eqnarray*}
E_1 &=& \frac{2 \cdot 5}{9} + \frac{2 \cdot 4}{8} = \frac{10}{9} + \frac{8}{8} = 1.11 + 1 = 2.11 \\
V_1 &=& \frac{2 \cdot 7 \cdot 5 \cdot 4}{81 \cdot 8} = \frac{2 \cdot 7 \cdot 5 \cdot 4}{648} = \frac{280}{648} = 0.432 \\
E_2 &=& \frac{2 \cdot 4}{9} + \frac{2 \cdot 3}{8} = \frac{8}{9} + \frac{6}{8} = 0.89 + 0.75 = 1.64 \\
V_2 &=& \frac{2 \cdot 7 \cdot 4 \cdot 4}{81 \cdot 8} = \frac{2 \cdot 7 \cdot 4 \cdot 4}{648} = \frac{224}{648} = 0.346 \\
\end{eqnarray*}

El estad\'istico de log-rank se calcula como:
\begin{eqnarray*}
\chi^2 &=& \frac{\left((1 - 2.11) + (1 - 1.64)\right)^2}{0.432 + 0.346} \\
       &=& \frac{\left(-1.11 - 0.64\right)^2}{0.778} \\
       &=& \frac{3.04}{0.778} \\
       &=& 3.91
\end{eqnarray*}

El valor p se puede obtener comparando $\chi^2$ con una distribuci\'on $\chi^2$ con un grado de libertad (dado que estamos comparando dos grupos).

\section{Interpretaci\'on del Test de Log-rank}
Un valor p peque\~no (generalmente menos de 0.05) indica que hay una diferencia significativa en las curvas de supervivencia entre los grupos. Un valor p grande sugiere que no hay suficiente evidencia para rechazar la hip\'otesis nula de que las curvas de supervivencia son iguales.

\section{Pruebas Alternativas}
Adem\'as del test de log-rank, existen otras pruebas para comparar curvas de supervivencia, como el test de Wilcoxon (Breslow), que da m\'as peso a los eventos en tiempos tempranos.

\section{Conclusi\'on}
El test de log-rank es una herramienta esencial para comparar curvas de supervivencia entre diferentes grupos. Su c\'alculo se basa en la diferencia entre los eventos observados y esperados en cada grupo, y su interpretaci\'on puede ayudar a identificar diferencias significativas en la supervivencia.

