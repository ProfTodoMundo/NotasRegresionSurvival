

\section{Introducci\'on}

El diagn\'ostico del modelo y el ajuste de par\'ametros son pasos esenciales para mejorar la precisi\'on y la robustez de los modelos de regresi\'on log\'istica. Este cap\'itulo se enfoca en las t\'ecnicas para diagnosticar problemas en los modelos y en m\'etodos para ajustar los par\'ametros de manera \'optima.

\section{Diagn\'ostico del Modelo}

El diagn\'ostico del modelo implica evaluar el rendimiento del modelo y detectar posibles problemas, como el sobreajuste, la multicolinealidad y la influencia de puntos de datos individuales.

\subsection{Residuos}

Los residuos son las diferencias entre los valores observados y los valores predichos por el modelo. El an\'alisis de residuos puede revelar patrones que indican problemas con el modelo.

\begin{eqnarray*}
\text{Residuo}_i = y_i - \hat{y}_i
\end{eqnarray*}

\subsubsection{Residuos Estudiantizados}

Los residuos estudiantizados se ajustan por la variabilidad del residuo y se utilizan para detectar outliers.

\begin{eqnarray*}
r_i = \frac{\text{Residuo}_i}{\hat{\sigma} \sqrt{1 - h_i}}
\end{eqnarray*}
donde $h_i$ es el leverage del punto de datos.

\subsection{Influencia}

La influencia mide el impacto de un punto de datos en los coeficientes del modelo. Los puntos con alta influencia pueden distorsionar el modelo.

\subsubsection{Distancia de Cook}

La distancia de Cook es una medida de la influencia de un punto de datos en los coeficientes del modelo.

\begin{eqnarray*}
D_i = \frac{r_i^2}{p} \cdot \frac{h_i}{1 - h_i}
\end{eqnarray*}
donde $p$ es el n\'umero de par\'ametros en el modelo.

\subsection{Multicolinealidad}

La multicolinealidad ocurre cuando dos o m\'as variables independientes est\'an altamente correlacionadas. Esto puede inflar las varianzas de los coeficientes y hacer que el modelo sea inestable.

\subsubsection{Factor de Inflaci\'on de la Varianza (VIF)}

El VIF mide cu\'anto se inflan las varianzas de los coeficientes debido a la multicolinealidad.

\begin{eqnarray*}
\text{VIF}_j = \frac{1}{1 - R_j^2}
\end{eqnarray*}
donde $R_j^2$ es el coeficiente de determinaci\'on de la regresi\'on de la variable $j$ contra todas las dem\'as variables.

\section{Ajuste de Par\'ametros}

El ajuste de par\'ametros implica seleccionar los valores \'optimos para los hiperpar\'ametros del modelo. Esto puede mejorar el rendimiento y prevenir el sobreajuste.

\subsection{Grid Search}

El grid search es un m\'etodo exhaustivo para ajustar los par\'ametros. Se define una rejilla de posibles valores de par\'ametros y se eval\'ua el rendimiento del modelo para cada combinaci\'on.

\subsection{Random Search}

El random search selecciona aleatoriamente combinaciones de valores de par\'ametros dentro de un rango especificado. Es menos exhaustivo que el grid search, pero puede ser m\'as eficiente.

\subsection{Bayesian Optimization}

La optimizaci\'on bayesiana utiliza modelos probabil\'isticos para seleccionar iterativamente los valores de par\'ametros m\'as prometedores.

\section{Implementaci\'on en R}

\subsection{Diagn\'ostico del Modelo}

\begin{verbatim}
# Cargar el paquete necesario
library(car)

# Residuos estudentizados
dataTrain$resid <- rstudent(model)
hist(dataTrain$resid, breaks = 20, main = "Residuos Estudentizados")

# Distancia de Cook
dataTrain$cook <- cooks.distance(model)
plot(dataTrain$cook, type = "h", main = "Distancia de Cook")

# Factor de Inflaci\'on de la Varianza
vif_values <- vif(model)
print(vif_values)
\end{verbatim}

\subsection{Ajuste de Par\'ametros}

\begin{verbatim}
# Grid Search con caret
control <- trainControl(method = "cv", number = 10)
tune_grid <- expand.grid(.alpha = c(0, 0.5, 1), .lambda = seq(0.01, 0.1, by = 0.01))

model_tune <- train(var1 ~ ., data = dataTrain, method = "glmnet", 
                    trControl = control, tuneGrid = tune_grid)

print(model_tune)
\end{verbatim}

