\section{Introducci\'on}

Implementar un modelo de regresi\'on log\'istica en datos reales implica varias etapas, desde la limpieza de datos hasta la evaluaci\'on y validaci\'on del modelo. Este cap\'itulo presenta un ejemplo pr\'actico de la implementaci\'on de un modelo de regresi\'on log\'istica utilizando un conjunto de datos real.

\section{Conjunto de Datos}

Para este ejemplo, utilizaremos un conjunto de datos disponible p\'ublicamente que contiene informaci\'on sobre clientes bancarios. El objetivo es predecir si un cliente suscribir\'a un dep\'osito a plazo fijo.

\section{Preparaci\'on de Datos}

\subsection{Carga y Exploraci\'on de Datos}

Primero, cargamos y exploramos el conjunto de datos para entender su estructura y contenido.

\begin{verbatim}
# Cargar el paquete necesario
library(dplyr)

# Cargar el conjunto de datos
data <- read.csv("bank.csv")

# Explorar los datos
str(data)
summary(data)
\end{verbatim}

\subsection{Limpieza de Datos}

El siguiente paso es limpiar los datos, lo que incluye tratar los valores faltantes y eliminar las duplicidades.

\begin{verbatim}
# Eliminar duplicados
data <- data %>% distinct()

# Imputar valores faltantes (si existen)
data <- data %>% mutate_if(is.numeric, ~ifelse(is.na(.), mean(., na.rm = TRUE), .))
\end{verbatim}

\subsection{Codificaci\'on de Variables Categ\'oricas}

Convertimos las variables categ\'oricas en variables num\'ericas utilizando la codificaci\'on one-hot.

\begin{verbatim}
# Codificaci\'on one-hot de variables categ\'oricas
data <- data %>% mutate(across(where(is.factor), ~ as.numeric(as.factor(.))))
\end{verbatim}

\section{Divisi\'on de Datos}

Dividimos los datos en conjuntos de entrenamiento y prueba.

\begin{verbatim}
# Dividir los datos en conjuntos de entrenamiento y prueba
set.seed(123)
trainIndex <- createDataPartition(data$y, p = .8, list = FALSE, times = 1)
dataTrain <- data[trainIndex,]
dataTest <- data[-trainIndex,]
\end{verbatim}

\section{Entrenamiento del Modelo}

Entrenamos un modelo de regresi\'on log\'istica utilizando el conjunto de entrenamiento.

\begin{verbatim}
# Entrenar el modelo de regresi\'on log\'istica
model <- glm(y ~ ., data = dataTrain, family = "binomial")

# Resumen del modelo
summary(model)
\end{verbatim}

\section{Evaluaci\'on del Modelo}

Evaluamos el rendimiento del modelo utilizando el conjunto de prueba.

\begin{verbatim}
# Predicciones en el conjunto de prueba
predictions <- predict(model, dataTest, type = "response")

# Convertir probabilidades a etiquetas
predicted_labels <- ifelse(predictions > 0.5, 1, 0)

# Matriz de confusi\'on
confusionMatrix(predicted_labels, dataTest$y)
\end{verbatim}

\section{Interpretaci\'on de los Resultados}

Interpretamos los coeficientes del modelo y las odds ratios.

\begin{verbatim}
# Coeficientes del modelo
coef(model)

# Odds ratios
exp(coef(model))
\end{verbatim}

